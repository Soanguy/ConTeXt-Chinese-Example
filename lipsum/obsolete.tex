%%%%%%%%%%%%%%%%%%%%%%%%%%%%%%%%%%%%%%%%%%%%%%%%
% command collections
%%%%%%%%%%%%%%%%%%%%%%%%%%%%%%%%%%%%%%%%%%%%%%%%%%%%%%%%%%%%%%%%%%%%%%%%%%%%%%%%%



%\determinelistcharacteristics[chapter][criterium=text]
% number of chapters: \structurelistsize
%\doif{\currentstructureshownumber}{no}{0}{\structurenumber}
%\showfontparameters

%\currentheadlevel% chapter = 2 section = 3 etc.
%\bodyfontsizevariable{\fontclass}{\fontsize}            % 獲取當前字體該字號縮放倍率
%%%%%%%%%%%%%%%%%%%%%%%%%%%
%\page\setuplayout[reset] % reset layout to \setuplayout
%\input knuth             % default is reset to layout ConTeXt designed
%\page\resetlayout        % reset layout to \setuplayout[Customized_layout]
%\input knuth             % for example \setuplayout[exotic]
%\def\MyMumOrderedMeTo[#1]%
%  {\begingroup
%   \def\processitem##1{\page\setuplayout[##1]
%   \previouslayout\currentlayout\showsetups[mm]}%
%   \processcommalist[#1]\processitem
%   \endgroup}
%\MyMumOrderedMeTo[sidemirror,propiorroot,propiorgplden,gridneeds,moderate,wide,
%                  narrow,moderatecomments,moderatecommentsii,exotic,ipad,kindle]
%\processseparatedlist[alfa+beta+gamma][+]\message
%\processcommalist[#1]\processitem
%%%%%%%%%%%%%%%%%%%%%%%%%%%
% \startfrontmatter  \stopfrontmatter
% \startbodymatter   \stopbodymatter
% \startappendices   \stopappendices
% \startbackmatter   \stopbackmatter
% The environment allows values to be set that only apply within the named
% section block rather than globally.
% Titleheads for special sections of a document,
% like abbreviations and appendices .
% Examples of titleheads are Content, Tables, Figures, Abbreviations, Index etc.
%#############################
% usefulcommand: \completepagenumber :place number with defined format
%                \placepagenumber    :place number with empty   format
%%%%%%%%%%%%%%%%%%%%%%%%%%%%%%%%%%%%%%%%%%%%%%%%%%%%%%%%%%%%%%%%%%%%%%%%%%%%%%%%%
%\definefontfallback[fallback-serif]
%                   [name:imingregular*fallback-default]
%                   [0x0000-0x10FFFF]
%\definefontfallback[fallback-serifbold]
%                   [name:imingregular*fallback-default]
%                   [0x0000-0x10FFFF]
%\definefontfallback[fallback-serifitalic]
%                   [name:twkai981*fallback-default]
%                   [0x0000-0x10FFFF]
%\definefontfallback[fallback-serifbolditalic]
%                   [name:twkai981*fallback-default]
%                   [0x0000-0x10FFFF]
%\definefontfallback[fallback-sans]
%                   [name:sourcehansanscnregular*fallback-default]
%                   [0x0000-0x10FFFF]
%\definefontfallback[fallback-sansbold]
%                   [name:sourcehansanscnregular*fallback-default]
%                   [0x0000-0x10FFFF]
%\definefontfallback[fallback-sansitalic]
%                   [name:twkai981*fallback-default]
%                   [0x0000-0x10FFFF]
%\definefontfallback[fallback-sansbolditalic]
%                   [name:twkai981*fallback-default]
%                   [0x0000-0x10FFFF]
%\definefontfallback[fallback-mono]
%                   [name:adobefangsongstdregular*fallback-default]
%                   [0x0000-0x10FFFF]
%\definefontfallback[fallback-monobold]
%                   [name:adobefangsongstdregular*fallback-default]
%                   [0x0000-0x10FFFF]
%\definefontfallback[fallback-monoitalic]
%                   [name:twkai981*fallback-default]
%                   [0x0000-0x10FFFF]
%\definefontfallback[fallback-monobolditalic]
%                   [name:twkai981*fallback-default]
%                   [0x0000-0x10FFFF]
%\definefontfallback[fallback-handw]
%                   [name:twkai981*fallback-default]
%                   [0x0000-0x10FFFF]
%\definefontfallback[fallback-handwbold]
%                   [name:twkai981*fallback-default]
%                   [0x0000-0x10FFFF]
%\definefontfallback[fallback-handwitalic]
%                   [name:twkai981*fallback-default]
%                   [0x0000-0x10FFFF]
%\definefontfallback[fallback-handwbolditalic]
%                   [name:twkai981*fallback-default]
%                   [0x0000-0x10FFFF]
% \setuplanguage[cn][bidi=r2l] set text direction

% some command maybe useful
% \parfillskip       行結束時填充的伸縮量
% \penalty 0          視情況可以在此處斷行
% \penalty 10000      絕對不在此處斷行
% \penalty -10000     絕對在此處斷行
% \parfillskip=0pt 我心匪石,不可轉也。
% 我心匪席,不可卷也。\penalty 10000 我心匪席,不可卷也。
% put some thing vertically-centered:    \null\vfill\input Knuth\vfill\null
%%%%%%%%%%%%%%%
%% To use the counter mechanism with your commands you have to first use
%% \installcounterassociation to create the two new commands
%% \register...counter (this ensures the default counter values are used
%% when you don't set anything) and \synchronize...counters (which updates
%% the counter values when you use the setup of your own command).
%% Unlike the framed mechanism this no longer works with the root instance
%% of your own commands because we create a new counter only when you
%% create a instance (with \define...) of the new command.

%\setvariables[gridpage][
%  wordperline=35,
%  lineperpage=40,
%  lineheight=1.5em,
%  wordperlineatmargin=8
%]
\newcount \wordperline          \wordperline         = 33
\newcount \lineperpage          \lineperpage         = 30
\newcount \wordperlineatmargin  \wordperlineatmargin = 10
\newcount \numhtfheight
          \numhtfheight=\number\numexpr\lineperpage * \number\lineheight \relax
\newcount \linehtfheight
          \linehtfheight=\number\numexpr \numhtfheight / 65536 \relax
\newdimen \temptemphtfheight
          \temptemphtfheight=\the\dimexpr 1pt * \linehtfheight \relax
\newdimen \htfheight
          \htfheight=\the\dimexpr\temptemphtfheight + \the\dimexpr 6\lineheight\relax\relax
%%%%%%%%%%%%%%%%%%%%%%%%%%%%%%%%%%%%%%%%%%%%%%%%%%%%%
\definemeasure[wordperl][\the\bodyfontsize*\number\wordperline*1]
\definemeasure[lineperg][\the\baselineskip*\number\lineperpage*1]
\definemeasure[textmgw][\the\bodyfontsize*\number\wordperlineatmargin*1]
\definemeasure[distances][\the\bodyfontsize]
\definemeasure[htfheighted][\the\htfheight]
%%%%%%%%%%%%%%%%%%%%%%%%%%%%%%%%%%%%%%%%%%%%%%%%%%%%%
\definemeasure[temptextwidth][\measured{wordperl}]
\definemeasure[temprightmarginwidth][\measured{textmgw}]
\definemeasure[temprightmargindistance][\measured{distances}]
\definemeasure[temprightedge][3\measured{distances}]
\definemeasure[tempheight][\measured{htfheighted}]
\definemeasure[tempheader][2\lineheight]
\definemeasure[tempfooter][2\lineheight]
\definemeasure[tempheaderdistance][\lineheight]
\definemeasure[tempfooterdistance][\lineheight]
%%%%%%%%%%%%%%%%%%%%%%%%%%%%%%%%%%%%%%%%%%%%%%%%%%%%%
%\definemeasure[tempbackspace][\measured{hremained}]
%\definemeasure[hremained] [\paperwidth-\textwidth-\rightmargindistance-
%                          \rightmarginwidth-\rightedgedistance-\rightedgewidth]
%\definemeasure[tempheader][\dimexpr (\paperheight-\measure{lineperg}
%                           -\measure{tempheaderdistance}*2
%                           -\topspace-\bottomheight-\bottomdistance)/8*3\relax]
%\definemeasure[tempfooter][\dimexpr (\paperheight-\measure{lineperg}
%                           -\measure{tempheaderdistance}*2
%                           -\topspace-\bottomheight-\bottomdistance)/8*5\relax]
%%%%%%%%%%%%%%%%%%%%%%%%%%%%%%%%%%%%%%%%%%%%%%%%%%%%%
%  leftmargin=\measure{tempbackspace},
%  leftmargindistance=\measure{temprightmargindistance},
%  leftedge=0pt,
%  leftedgedistance=0pt,
%  height=\measure{tempheight},
%  backspace=\measure{tempbackspace},
%  bottom=12pt,
\definelayout[gridneeds][
  width=\measure{temptextwidth},
  rightmargin=\measure{temprightmarginwidth},
  rightedge=\measure{temprightedge},
  rightmargindistance=\measure{temprightmargindistance},
  rightedgedistance=\measure{temprightmargindistance},
%  header=\measure{tempheader},
%  footer=\measure{tempfooter},
%  headerdistance=\measure{tempheaderdistance},
%  footerdistance=\measure{tempfooterdistance},
  height=fit,%\measure{tempheight},
  location=doublesided,
  topspace=0pt,
  bottomspace=0pt,]


\startluacode
local  StringHelper = {}
function StringHelper.GetBytes(char)
   if not char then
      return 0
   end
   local code = string.byte(char)
   if code < 127 then
      return 1
   elseif code <= 223 then
      return 2
   elseif code <= 239 then
      return 3
   elseif code <= 247 then
      return 4
   else
      return 0
   end
end
function StringHelper.Sub(str, startIndex, endIndex)
   local tempStr = str
   local byteStart = 1 -- string.sub截取的开始位置
   local byteEnd = -1 -- string.sub截取的结束位置
   local index = 0  -- 字符记数
   local bytes = 0  -- 字符的字节记数
   startIndex = math.max(startIndex, 1)
   endIndex = endIndex or -1
   while string.len(tempStr) > 0 do
      if index == startIndex - 1 then
         byteStart = bytes+1;
      elseif index == endIndex then
         byteEnd = bytes;
         break;
      end
      bytes = bytes + StringHelper.GetBytes(tempStr)
      tempStr = string.sub(str, bytes+1)
      index = index + 1
   end
   return string.sub(str, byteStart, byteEnd)
end
function get_str_len ( s )
  str_len = utf8.len ( s )
  tex.print(str_len)
end
function get_nth_str ( n , s )
  nth_str = StringHelper.Sub ( s , n , n)
  tex.print(nth_str)
end
function strsplit ( s )
  t = ""
  n = 1
  ctx_hss = context.hss
  str_len = utf8.len ( s )
  while (str_len > n) do
    t = StringHelper.Sub ( s , n , n) .. "{,}"
    n = n + 1
    tex.sprint(t)
  end
  context(tex.hss)
  last_str = StringHelper.Sub ( s , n , n+1)
  tex.sprint(last_str)
end
\stopluacode
\newdimen\autojustify_width
\def\autojustify{\dosingleempty\doautojustify}
\def\doautojustify[#1]#2{%
 \doifsomethingelse{#1}{\autojustify_width = #1}{\autojustify_width = 3em}%
 \newcount\cnt_str_len \cnt_str_len = \directlua{get_str_len ("#2")}%\hskip .01pt%
 \hbox to \autojustify_width {%
 \dorecurse{\numexpr\cnt_str_len - 1}%
           {\directlua{get_nth_str(\recurselevel,"#2")}\hfil}%
            \directlua{get_nth_str(\number\cnt_str_len,"#2")}%
}}
%\autojustify[8em]{截取的开始}截取
%\autojustify{截取}截取


%%%%%%%%%%%%%%%
\def\@gobbletwo#1#2{}
\def\xColorNthChar#1#2{%
\ifnum\ifx\empty#21\else#1\fi=1%
\color[red]{#2}\expandafter\@gobbletwo%
\else#2\fi\xColorNthChar{\numexpr#1-1\relax}}
\def\ColorNthChar#1#2{\xColorNthChar{#1}#2\empty}
%\ColorNthChar{5}{12345}
%\ColorNthChar{23}{examination}
%\ColorNthChar{8}{examination}
\startluacode
  function isInteger(n)
        if n % 1 == 0 then
          tex.print("true")
        else
          tex.print("false")
        end
  end
\stopluacode
\def\float_calc#1{\directlua{tex.sprint(#1)}}
\def\ifInteger#1{\directluacode{isInteger(#1)}}
%\doif{\ifInteger{\float_calc{\the\answer_seq/6}}}{true}{\eTR}%
\def\autoright#1{%
  {\unskip\nobreak\hfil\penalty50
   \hskip2em\hbox{}\nobreak\hfil#1
      \parfillskip=0pt \finalhyphendemerits=0 \par}}
\def\vapour#1{\vrule width 0pt \nobreak%
    \newdimen\tempwidth%
    \setbox0=\hbox{#1}%
    \tempwidth=\wd0%
    \hbox to \tempwidth{}%
    \hskip 0pt plus 0pt minus 0pt}


\setupMPvariables[ShadowedText][shadowedtext={\bfb text} ]
\startuniqueMPgraphic{ShadowedText} 
   path tt ; tt := lmt_outline [
        kind = "outline",
        text = "\MPvar{shadowedtext} ",];
   draw tt withpattern image (
        fill fullcircle scaled 5mm withcolor "darkgreen" ;)
   withpatternscale (1/100,1/30);
\stopuniqueMPgraphic
\defineoverlay[Myframe][\uniqueMPgraphic{ShadowedText} ]
\inframed[background=Myframe]{\bfb text} 
\inframed[background=Myframe]{\bfb text} 
\inframed[background=Myframe]{\bfb text} 
%
%\setupexternalfigure[location=default]
%\defineoverlay
%  [BackGround]
%  [\ifzeronum\namedheadnumber{chapter} \relax%
%   \orelse\ifodd\namedheadnumber{chapter} \relax
%     \externalfigure[mill][factor=min]%
%   \else
%     \mirror{\externalfigure[mill][factor=min]} %
%   \fi]
%
%\setupbackgrounds[page][background=BackGround]

%\def\ifpara#1{
%\begingroup
%  \toks0={#1} %
%  \edef\param{\the\toks0} %
%\ifx\param\empty
%    it is empty
%\else
%    it is not empty
%\fi
%\endgroup} 
%
%\startitemize[n,packed]
%  \item \ifpara{} 
%  \item \ifpara{ } 
%  \item \ifpara{{} } 
%  \item \ifpara{$\not$} 
%  \item \ifpara{\rlap{Text} } 
%  \item \ifpara{#} 
%\stopitemize
%
%\def\cs#1{%
%  \if\relax\detokenize{#1} \relax
%    The argument is empty%
%  \else
%    The argument #1 is non empty%
%  \fi
%} 
%
%\cs{abc}  \cs{} 
%
%\chapter{Foo} 
%
%\input knuth
%
%\chapter{Bar} 


\installnamespace {warichu}
\installsimplecommandhandler \????warichu {warichu} \????warichu
\setupwarichu[style=tfxx,
              voffset=-1.5pt,
              distance=1pt plus 1pt,
              left={},right={},
              reference=,
              page=,% manually set odd or even page
              mode=]%checkmode]% will show info of related warichu width
\definepagestate[warichu]
\newif\ifcheckmode
\newcount\c_warichu_n
\newdimen\d_textwd_warichu   \newdimen\d_restwd_warichu%
\newdimen\d_oddpos_warichu   \newdimen\d_evenpos_warichu%
\newdimen\d_autopos_warichu  \newdimen\d_curline_warichu%
\newdimen\d_temp_leftskip    \newdimen\d_temp_rightskip%
\newdimen\d_temp_textwd      \newdimen\d_temp_avbhsize%
\newdimen\d_temp_lastline
\def\check_page%
  {\dontleavehmode%
   \autosetpagestate{warichu}%
   \ifodd0\autopagestaterealpage{warichu}%
          ODD Page \else EVEN Page\fi}
\def\ShowMode%
  {\ifcheckmode
    \ininnermargin[style=ttxx,color=red]{
    \the\c_warichu_n:\conditional_level::
    \ifhmode \ifinner inner \fi hmode
    \else\ifvmode \ifinner inner \fi vmode
    \else\ifmmode \ifinner inner \fi mmode
    \else \ifinner inner \fi unset
    \fi\fi\fi}
  \else\relax\fi}
\def\get_split_box#1{%
    \setbox\scratchboxone\hbox{\hsplit\scratchbox width #1}
    %shrinkcriterium  0%stretchcriterium 0%
    \parindent=0pt\relax%
    \unhbox\scratchboxone\unskip\unskip\unpenalty%
%    \par\allowbreak%
    }
\tolerant\def\rawwarichu[#1]#:#2{\begingroup%
  \let\par\endgraf\let\\\endgraf%
  \def\last_linewidth_warichu{}
  \dontleavehmode\unskip\unskip\unpenalty%
  \d_oddpos_warichu =\dimexpr+\backspace+\textwidth-\rightskip
                 \ifnum\hangafter<\zerocount
                 \ifdim\hangindent>\zeropoint-\else+\fi
                 \hangindent\fi\relax
  \d_evenpos_warichu=\dimexpr+\paperwidth-\backspace-\rightskip
                 \ifnum\hangafter<\zerocount
                 \ifdim\hangindent>\zeropoint-\else+\fi
                 \hangindent\fi\relax
  \global\advance\c_warichu_n by 1\relax
  \setupwarichu[#1]%
  \hskip\warichuparameter\c!distance\relax\warichuparameter\c!left
  %must put before xypos and will not influences box splitting.
  \doif{\warichuparameter{mode}}{checkmode}{\checkmodetrue}%
  \doif{\warichuparameter{mode}}{}         {\checkmodefalse}%
  \edef\xpos_warichu{warichu:\number\c_warichu_n}%
  \edef\xpos_prev_warichu{warichu:\dimexpr\c_warichu_n-1\relax}%
  \edef\xpos_next_warichu{warichu:\dimexpr\c_warichu_n+1\relax}%
  \xypos\xpos_warichu% 定義起始點
  \d_curline_warichu=\dimexpr\MPx\xpos_warichu\relax%
  \reference[\warichuparameter\c!reference]{\warichuparameter\c!reference}%
  \autosetpagestate{warichu}% 獲取當前正文終點寬度
  \if\relax\warichuparameter{page}\relax%
      \ifdoublesided%
          \ifodd0\autopagestaterealpage{warichu}%
               \d_autopos_warichu=\d_oddpos_warichu%
          \else\d_autopos_warichu=\d_evenpos_warichu\fi%
      \else    \d_autopos_warichu=\d_oddpos_warichu \fi%
  \else \doifelse{\warichuparameter{page}}{odd}%
              {\d_autopos_warichu=\d_oddpos_warichu}%
              {\d_autopos_warichu=\d_evenpos_warichu}%
  \fi\relax%
  \d_restwd_warichu=\doifelse{\warichuparameter{restwidth}}{}
      {\ifdim% 獲取當前行可用長度
       \dimexpr\d_autopos_warichu-\d_curline_warichu\relax<1em 0pt\else%
       \dimexpr\d_autopos_warichu-\d_curline_warichu\relax\fi}
      {\warichuparameter{restwidth}}%
  \setbox\scratchbox\hbox{\setnostrut%
         \usewarichustyleandcolor\c!style\c!color #2\relax}%
  \d_textwd_warichu=\wd\scratchbox%dimen belowed is for checkmode
  \d_temp_leftskip =\leftskip        \relax
  \d_temp_rightskip=\rightskip       \relax
  \d_temp_avbhsize =\availablehsize  \relax
  \d_temp_textwd   =\d_textwd_warichu\relax
  %如果剩餘為 0 換行並重置剩餘為行長
  \ifdim\d_restwd_warichu = 0pt\break\d_restwd_warichu=\availablehsize\fi%
  \ifdim\d_restwd_warichu > \dimexpr\d_textwd_warichu/2+2pt\relax% 第一層 if
    \ifdim\d_textwd_warichu  < 16pt\relax% 過小的盒子不處理
      \def\conditional_level{box  < 16pt}\ShowMode%
      \vbox{\hsize\wd\scratchbox\box\scratchbox}%
    \else%剩餘寬度可以容納盒子
      \def\conditional_level{rest > box/2}\ShowMode%
      \raise\warichuparameter\c!voffset%
      \vbox{\offinterlineskip%
            \hsize\dimexpr\wd\scratchbox/2+1pt\relax
            \get_split_box{\dimexpr\wd\scratchbox/2+1pt\relax}%
            \get_split_box{\dimexpr\wd\scratchbox+1pt\relax}}%
    \fi%
  \else% 第一層 else
      \def\conditional_level{rest < box/2}\ShowMode%
      \raise\warichuparameter\c!voffset%
      \vbox{\offinterlineskip%
            \hsize\dimexpr\d_restwd_warichu\relax%
            \get_split_box{\dimexpr\hsize\relax}%
            \get_split_box{\dimexpr\hsize\relax}}\break%
      \doloop{\ifdim\wd\scratchbox>\dimexpr\availablehsize*2-1pt\relax%
                  \def\conditional_level{loop:box > avhsize*2}\ShowMode%
                  \raise\warichuparameter\c!voffset%
                  \vbox{\offinterlineskip%
                        \hsize\availablehsize%
                        \get_split_box{\dimexpr\hsize\relax}%
                        \get_split_box{\dimexpr\hsize\relax}}\break%
              \else%\availablehsize will fit narrower environment
                  \def\conditional_level{loop:laststep}\ShowMode%
                  \d_temp_lastline=\dimexpr\wd\scratchbox/2+1pt\relax
                  \edef\last_linewidth_warichu{\the\d_temp_lastline}%
                  \raise\warichuparameter\c!voffset%
                  \vbox{\offinterlineskip%
                        \hsize\dimexpr\wd\scratchbox/2+1pt\relax%
                        \get_split_box{\dimexpr\hsize\relax}%
                        \get_split_box{\dimexpr\wd\scratchbox+1pt\relax}}%
                  \ifdim\dimexpr-\d_temp_lastline+\d_temp_avbhsize\relax<1.1em%
                        \break\fi%if rest width of last line <= 1em break line.
              \exitloop\fi}%
  \fi\allowbreak\unskip\unskip\unpenalty% 第一層 fi
  \warichuparameter\c!right\hskip\warichuparameter\c!distance\relax%
  \ifcheckmode%
  \inoutermargin[stack=yes,voffset=-\baselineskip]
  {\ttxx\setuplocalinterlinespace[line=1pt]
  \starttabulate[|l|l|]
      \NC wd_warichubox     \NC \the\d_textwd_warichu  \NC\NR
      \NC d_oddpos_warichu  \NC \the\d_oddpos_warichu  \NC\NR
      \NC d_evenpos_warichu \NC \the\d_evenpos_warichu \NC\NR
      \NC d_autopos_warichu \NC \the\d_autopos_warichu \NC\NR
      \NC d_curline_warichu \NC \the\d_curline_warichu \NC\NR
      \NC d_restwd_warichu  \NC \the\d_restwd_warichu  \NC\NR
      \NC textwidth         \NC \the\textwidth         \NC\NR
      \NC availablehsize    \NC \the\d_temp_avbhsize   \NC\NR
      \NC leftskip          \NC \the\d_temp_leftskip   \NC\NR
      \NC rightskip         \NC \the\d_temp_rightskip  \NC\NR
      \NC pageno            \NC \the\pageno            \NC\NR
      \NC page state        \NC \check_page            \NC\NR
      \NC warichu count     \NC \the\c_warichu_n       \NC\NR
      \NC conditional_level \NC \conditional_level     \NC\NR
      \NC last_linewidth    \NC \last_linewidth_warichu\NC\NR
  \stoptabulate}\fi%
  \endgroup}
\tolerant\def\warichu[#1]#:#2{\bgroup\setupwarichu[#1]\processlist├┤{==}\rawwarichu├#2┤\egroup}



%\definefontfamily[kozukax][rm][kozukaminchopr6n]
%                 [tf=name:kozminpr6nregular,
%                  it=name:utrilloprom,
%                  bf=name:kozminpr6nbold,
%                  bi=name:utrilloprodb,]
%\definefontfamily[kozukax][ss][kozukagothicpr6n]
%                 [tf=name:kozgopr6nregular,
%                  it=name:utrilloprom,
%                  bf=name:kozgopr6nbold,
%                  bi=name:utrilloprodb,]
%\definefontfamily[kozukax][tt][fottsukuardgothicstd]
%                 [tf=name:tsukuardgothicstdr,
%                  it=name:tsukuardgothicstdr,
%                  bf=name:tsukuardgothicstdb,
%                  bi=name:tsukuardgothicstdb,]
%\definefontfamily[kozukax][hw][fotutrillopro]
%                 [tf=name:utrilloprom,
%                  it=name:utrilloprom,
%                  bf=name:utrilloprodb,
%                  bi=name:utrilloprodb,]
