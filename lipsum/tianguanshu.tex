卷二十七・天官書第五

〔索隱〕案天文有五官。官者,星官也。星座有尊卑,若人之官曹列位,故曰天官。〔正義〕張衡云:「文曜麗乎天,其動者有七,日月五星是也。日者,陽精之宗;月者,陰精之宗;五星,五行之精。衆星列布,體生於地,精成於天,列居錯峙,各有所屬,在野象物,在朝象官,在人象事。其以神著有五列焉,是有三十五名:一居中央,謂之北斗;四布於方各七,爲二十八舍;日月運行,曆示吉凶也。」

中宮〔索隱〕姚氏案:春秋元命包云「官之爲言宣也,宣氣立精爲神垣」。又文耀鉤曰「中宮大帝,其精北極星。含元出氣,流精生一也」。天極星,〔索隱〕案:爾雅「北極謂之北辰」。又春秋合誠圖云「北辰,其星五,在紫微中」。楊泉物理論云「北極,天之中,陽氣之北極也。極南爲太陽,極北爲太陰。日、月、五星行太陰則無光,行太陽則能照,故爲昏明寒暑之限極也。」其一明者,太一常居也;〔索隱〕案:春秋合誠圖云「紫微,大帝室,太一之精也」。〔正義〕泰一,天帝之別名也。劉伯莊云:「泰一,天神之最尊貴者也。」旁三星三公,〔正義〕三公三星在北斗杓東,又三公三星在北斗魁西,並爲太尉、司徒、司空之象,主變出陰陽,主佐機務。占以徙爲不吉,居常則安,金、火守之並爲咎也。或曰子屬。後句四星,〔索隱〕句音鉤。句,曲也。末大星正妃,〔索隱〕案:援神契云「辰極橫,后妃四星從,端大妃光明」。又案:星經以後句四星名爲四輔,其句陳六星爲六宮,亦主六軍,與此不同也。餘三星後宮之屬也。環之匡衞十二星,藩臣。皆曰紫宮。〔索隱〕案:元命包曰「紫之言此也,宮之言中也,言天神運動,陰陽開閉,皆在此中也」。宋均又以爲十二軍,中外位各定,總謂之紫宮也。

前列直斗口〔索隱〕直,劉氏云如字,直,當也。又音值也。三星,隨北端兌,〔索隱〕隋斗端兌。隋音湯果反。劉氏云「斗,一作『北』」。案:漢書天文志作「北」。端作「耑」。兌作「銳」。銳謂星形尖銳也。若見若不,曰陰德,〔索隱〕案:文耀鉤曰「陰德爲天下綱」。宋均以爲陰行德者,道常也。〔正義〕星經云:「陰德二星在紫微宮內,尚書西,主施德惠者,故贊陰德遺惠,周急賑撫。占以不明爲宜;明,新君踐極也。」又云:「陰德星,中宮女主之象。星動搖,釁起宮掖,貴嬪內妾惡之。」或曰天一。〔正義〕天一一星,疆閶闔外,天帝之神,主戰鬬,知人吉凶。明而有光,則陰陽和,萬物成,人主吉;不然,反是太一一星次天一南,亦天帝之神,主使十六神,知風雨、水旱、兵革,饑饉、疾疫。占以不明及移爲災也。星經云:「天一、太一二星主王者即位,令諸立赤子而傳國位者。星不欲微;微則廢立不當其次,宗廟不享食矣。」紫宮左三星曰天槍,〔索隱〕楚庚反。右五星曰天棓,〔集解〕蘇林曰:「音『𣘙打』之『𣘙』。」〔索隱〕棓音皮,韋昭音剖。又詩緯曰:「槍三星,棓五星,在斗杓左右,主槍人棓人。」石氏星讚云「槍棓八星,備非常」也。〔正義〕棓,龐掌反。天棓五星在女床東北,天子先驅,所以禦兵也。占:星不具,國兵起也。後六星絕漢抵營室,曰閣道。〔索隱〕絕,度也。抵,屬也。又案:樂汁圖云「閣道,北斗輔」。石氏云「閣道六星,神所乘也」。〔正義〕漢,天河也。直度曰絕。抵,至也。營室七星,天子之宮,亦爲玄宮,亦爲清廟,主上公,亦天子離宮別館也。王者道被草木,營室歷九象而可觀。閣道六星在王良北,飛閣之道,天子欲遊別宮之道。占:一星不見則輦路不通,動搖則宮掖之內起兵也。

北斗七星,〔索隱〕案:春秋運斗樞云「斗,第一天樞,第二旋,第三璣,第四權,第五衡,第六開陽,第七搖光。第一至第四爲魁,第五至第七爲標,合而爲斗。」文耀鉤云「斗者,天之喉舌。玉衡屬杓,魁爲琁璣」。徐整長曆云「北斗七星,星閒相去九千里。其二陰星不見者,相去八千里也」。所謂「旋、璣、玉衡〔索隱〕案:尚書「旋」作「璿」。馬融云「璿,美玉也。機,渾天儀,可轉旋,故曰機。衡,其中橫筩。以璿爲機,以玉爲衡,蓋貴天象也」。鄭玄注大傳云「渾儀中筩爲旋機,外規爲玉衡」也。以齊七政」。〔索隱〕案:尚書大傳云「七政,謂春、秋、冬、夏、天文、地理、人道,所以爲政也。人道政而萬事順成」。又馬融注尚書云「七政者,北斗七星,各有所主:第一曰正日;第二曰主月法;第三曰命火,謂熒惑也;第四曰煞土,謂填星也;第五曰伐水,謂辰星也;第六曰危木,謂歲星也;第七曰剽金,謂太白也。日、月、五星各異,故曰七政也」。杓攜龍角,〔集解〕孟康曰:「杓,北斗杓也。龍角,東方宿也。攜,連也。」〔正義〕案:角星爲天關,其閒天門,其內天庭,黃道所經,七耀所行。左角爲理,主刑,其南爲太陽道;右角爲將,主兵,其北爲太陰道也。蓋天之三門,故其星明大則天下太平,賢人在位;不然,反是也。衡殷南斗,〔集解〕晉灼曰:「衡,斗之中央。殷,中也。」〔索隱〕案:晉灼云「殷,中也」。宋均云「殷,當也」。魁枕參首。〔正義〕枕,之禁反。衡,斗衡也。魁,斗第一星也。言北方斗,斗衡直當北之魁,枕於參星之首;北斗之杓連於龍角。南斗六星爲天廟,丞相、大宰之位,主薦賢良,授爵祿,又主兵,一曰天機。南二星,魁、天粱;中央一星,天相;北二星,天府庭也。占:斗星盛明,王道和平,爵祿行;不然,反是。參主斬刈,又爲天獄,主殺罰。其中三星橫列者,三將軍,東北曰左肩,主左將;西北曰右肩,主右將;東南曰左足,主後將;西南曰右足,主偏將:故軒轅氏占參應七將也。中央三小星曰伐,天之都尉也,主戎狄之國。不欲明;若明與參等,大臣謀亂,兵起,夷狄內戰。七將皆明,主天下兵振;芒角張,王道缺;參失色,軍散敗;參芒角動搖,邊候有急;參左足入玉井中,及金、火守,皆爲起兵。用昬建者杓;〔索隱〕用昏建中者杓。說文云「杓,斗柄」。音匹遙反,即招搖。杓,自華以西南。〔集解〕孟康曰:「傳曰『斗第七星法太白主,杓,斗之尾也』。尾爲陰,又其用昏,昏陰位,在西方,故主西南。」〔正義〕杓,東北第七星也。華,華山也。言北斗昏建用斗杓,星指寅也。杓,華山西南之地也。夜半建者衡;〔集解〕徐廣曰:「第五星。」孟康曰:「假令杓昏建寅,衡夜半亦建寅。」 〔索隱〕孟康曰:「假令杓昏建寅,衡夜半亦建寅也。」衡,殷中州河、濟之閒。〔正義〕衡,北斗衡也。言北斗夜半建用斗衡指寅。殷,當也。斗衡黃河、濟水之閒地也。平旦建者魁;魁,海岱以東北也。〔集解〕孟康曰:「傳曰『斗第一星法於日,主齊也』。魁,斗之首;首,陽也,又其用在明陽與明德,在東方,故主東北齊分。」〔正義〕言北斗旦建用斗魁指寅也。海岱,代郡也。言魁星主海岱之東北地也。隨三時所指,有前三建也。斗爲帝車,運于中央,〔索隱〕姚氏案:宋均曰「言是大帝乘車廵狩,故無所不紀也。」臨制四鄉。分陰陽,建四時,均五行,移節度,定諸紀,皆繫於斗。

斗魁戴匡六星〔集解〕晉灼曰:「似匡,故曰戴匡也。」曰文昌宮:〔索隱〕文耀鉤曰「文昌宮爲天府」。孝經援神契云「文者精所聚,昌者揚天紀」。輔拂並居,以成天象,故曰文昌。一曰上將,二曰次將,三曰貴相,四曰司命,五曰司中,六曰司祿。〔索隱〕春秋元命包曰:「上將建威武,次將正左右,貴相理文緒,司祿賞功進士,司命主老幼,司災主災咎也。」在斗魁中,貴人之牢。〔集解〕孟康曰:「傳曰『天理四星在斗魁中。貴人牢名曰天理』。」〔索隱〕在魁中,貴人牢。樂汁圖云「天理理貴人牢」。宋均曰「以理牢獄」也。 〔正義〕占:明,及其中有星,此貴人下獄也。魁下六星,兩兩相比者,名曰三能。〔集解〕蘇林曰:「能音台。」〔索隱〕魁下六星,兩兩相比,曰三台。案:漢書東方朔「願陳泰階六符」。孟康曰「泰階,三台也,台星凡六星。六符,六星之符驗也」。應劭引黃帝泰階六符經曰「泰階者,天子之三階:上階,上星爲男主,下星爲女主;中階,上星爲諸侯三公,下星爲卿大夫;下階,上星爲士,下星爲庶人。三階平,則陰陽和,風雨時;不平,則稼穡不成,冬雷夏霜,天行暴令,好興甲兵。修宮榭,廣苑囿,則上階爲之坼也」。三能色齊,君臣和;不齊,爲乖戾。輔星〔集解〕孟康曰:「在北斗第六星旁。」明近,〔正義〕大臣之象也。占:欲其小而明;若大而明,則臣奪君政;小而不明,則臣不任職;明大與斗合,國兵暴起;暗而逺斗,臣不死則奪;若近臣專賞,排賢用佞,則輔生角;近臣擅國符印,將謀社稷,則輔生翼;不然,則死也。輔臣親彊;斥小,疏弱。〔集解〕蘇林曰:「斥,逺也。」

杓端有兩星:一內爲矛,招搖;〔集解〕孟康曰:「近北斗者招搖,招搖爲天矛。」晉灼曰:「更河三星,天矛、鋒、招搖,一星耳。」〔索隱〕案:詩記曆樞云「更河中招搖爲胡兵」。宋均云「招搖星在更河內」。又樂汁圖云「更河天矛」,宋均以爲更河名天矛,則更河是星名也。一外爲盾,天鋒。〔集解〕晉灼曰:「外,逺北斗也。在招搖南,一名玄戈。」〔正義〕星經云:「梗河星爲戟劔之星,若星不見或進退不定,鋒鏑亂起,將爲邊境之患也。」有句圜十五星,〔索隱〕句音鉤。圜音員。其形如連環,即貫索星也。屬杓,〔正義〕屬音燭。曰賤人之牢。〔索隱〕案:詩記曆樞云「賤人牢,一曰天獄」。又樂汁圖云「連營,賤人牢」。宋均以爲連營,貫索也。〔正義〕貫索九星在七公前,一曰連索,主法律,禁暴彊,故爲賤人牢也。牢口一星爲門,欲其開也。占:星悉見,則獄事繁;不見,則刑務簡;動搖,則斧鉞用;中虛,則改元;口開,則有赦;人主憂,若閉口,及星入牢中,有自繫死者。常夜候之,一星不見,有小喜;二星不見,則賜祿;三星不見,則人主德令且赦。逺十七日,近十六日。若有客星出,視其小大:大,有大赦;小,亦如之也。其牢中星實則囚多,虛則開出。

天一、槍、棓、矛、盾動搖,角大,兵起。〔集解〕李奇曰:「角,芒角。」

東宮蒼龍,〔索隱〕案文耀鉤云「東宮蒼帝,其精爲龍」也。房、心。〔索隱〕案爾雅云「大辰,房、心、尾也」。李廵曰「大辰,蒼龍宿,體最明也」。心爲明堂,〔索隱〕春秋說題辭云:「房、心爲明堂,天王布政之宮。」尚書運期授曰:「房,四表之道。」宋均云:「四星閒有三道,日、月、五星所從出入也。」大星天王,前後星子屬。〔索隱〕鴻範五行傳曰:「心之大星,天王也。前星,太子;後星,庶子。」不欲直,直則天王失計。房爲府,曰天駟。〔索隱〕房爲天府,曰天駟。爾雅云:「天駟,房。」詩記曆樞云:「房爲天馬,主車駕。」宋均云:「房旣近心,爲明堂,又別爲天府及天駟也。」其陰,右驂。〔正義〕房星,君之位,亦主左驂,亦主良馬,故爲駟。王者恆祠之,是馬祖也。旁有兩星曰衿;〔索隱〕房有兩星曰衿。一音其炎反。元命包云:「鉤衿兩星,以閑防,神府闓舒,爲主鉤距,以備非常也。」〔正義〕占:明而近房,天下同心。鉤、鈐、房、心之閒有客星出及疏坼者,皆地動之祥也。北一星曰舝。〔集解〕徐廣曰:「音轄。」〔正義〕說文云:「舝,車軸耑鍵也。兩相穿背也。」星經云:「鍵閉一星,在房東北,掌管籥也。」占:不居其所,則津梁不通,宮門不禁;居,則反是也。東北曲十二星曰旗。〔正義〕兩旗者,左旗九星,在河鼓左也;右旗九星,在河鼓右也。皆天之鼓旗,所以爲旌表。占:欲其明大光潤,將軍吉;不然,爲兵憂;及不居其所,則津梁不通;動搖,則兵起也。旗中四星曰天市;〔正義〕天市二十三星,在房、心東北,主國市聚交易之所,一曰天旗。明則市吏急,商人無利;忽然不明,反是。市中星衆則歲實,稀則歲虛。熒惑犯,戮不忠之臣。彗星出,當徙市易都。客星入,兵大起;出之,有貴喪也。中六星曰市樓。市中星衆者實;其虛則秏。〔正義〕秏,貧無也。房南衆星曰騎官。

左角,李;右角,將。〔索隱〕李即理,理,法官也。故元命包云「左角理,物以起;右角將,帥而動」。又石氏云「左角爲天田,右角爲天門」也。大角者,天王帝廷。〔索隱〕大角,天王帝廷。案:援神契云「大角爲坐候」。宋均云「坐,帝坐也」。〔正義〕大角一星,在兩攝提閒,人君之象也。占:其明盛黃潤,則天下大同也。其兩旁各有三星,鼎足句之,曰攝提。〔集解〕晉灼曰:「如鼎之句曲。」〔索隱〕案:元命包云「攝提之爲言提攜也。言提斗攜角以接於下也」。〔正義〕攝提六星,夾大角,大臣之象,恆直斗杓所指,紀八節,察萬事者也。占:色溫溫不明而大者,人君恐;客星入之,聖人受制也。攝提者,直斗杓所指,以建時節,故曰「攝提格」。亢爲疏廟,〔索隱〕元命包曰「亢四星爲廟廷」。又文耀鉤「爲疏廟」,宋均以爲疏,外也;廟,或爲朝也。〔正義〕聽政之所也。其占:明大,則輔臣忠,天下寧;不然,則反是也。主疾。其南北兩大星,曰南門。〔正義〕南門二星,在庫樓南,天之外門。占:明則氐、羌貢;暗則諸夷叛;客星守之,外兵且至也。氐爲天根,〔索隱〕爾雅云「天根,氐也」。孫炎以爲角、亢下繫於氐,若木之有根也。〔正義〕星經云:「氐四星爲路寢,聽朝所居。其占:明大,則臣下奉度。」合誠圖云:「氐爲宿宮也。」主疫。〔索隱〕宋均云:「疫,病也。三月榆莢落;故主疾疫也。然此時物雖生,而日宿在奎,行毒氣,故有疫也。」〔正義〕氐、房、心三宿爲火,於辰在卯,宋之分野。

尾爲九子,〔索隱〕宋均云:「屬後宮場,故得兼子。子必九者,取尾有九星也。」元命包云:「尾九星,箕四星,爲後宮之場也。」〔正義〕尾,箕。尾爲析木之津,於辰在寅,燕之分野。尾九星爲後宮,亦爲九子。星近心第一星爲后,次三星妃,次三星嬪,末二星妾。占:均明,大小相承,則後宮敘而多子;不然,則不;金、火守之,後宮兵起;若明暗不常,妃嫡乖亂,妾媵失序。曰君臣;斥絕,不和。箕爲敖客,〔索隱〕宋均云:「敖,調弄也。箕以簸揚,調弄象也。箕又受物,有去去來來,客之象也。」〔正義〕敖音傲。箕主八風,亦后妃之府也。移徙入河,國人相食;金、火入守,天下亂;月宿其野,爲風起。曰口舌。〔索隱〕詩云「維南有箕,載翕其舌」。又詩緯云「箕爲天口,主出氣」。是箕有舌,象讒言。詩曰「哆兮侈兮,成是南箕」,謂有敖客行謁請之也。

火犯守角,〔索隱〕案:韋昭曰「火,熒惑也」。則有戰。房、心,王者惡之也。〔正義〕熒惑犯守箕、尾,氐星自生芒角,則有戰陣之事。若熒惑守房、心,及房、心自生芒角,則王者惡之也。

南宮朱鳥,〔正義〕柳八星爲朱鳥咮,天之厨宰,主尚食,和滋味。權、衡。〔集解〕孟康曰:「軒轅爲權,太微爲衡。」〔索隱〕案:文耀鉤云「南宮赤帝,其精爲朱鳥」。孟康曰:「軒轅爲權,太微爲衡」也。〔正義〕權四星在軒轅尾西,主烽火,備警急。占以明爲安靜;不明,則警急;動搖芒角亦如之。衡,太微之庭也。衡,太微,三光之廷。〔索隱〕宋均曰:「太微,天帝南宮也。三光,日、月、五星也。」匡衞十二星,藩臣:〔索隱〕十二星,蕃臣。春秋合誠圖曰:「太微主法式,陳星十二,以備武急也。」〔正義〕太微宮垣十星,在翼、軫地,天子之宮庭,五帝之坐,十二諸侯之府也。其外藩,九卿也。南藩中二星閒爲端門。次東第一星爲左執法,廷尉之象;第二星爲上相;第三星爲次相;第四星爲次將;第五星爲上將。端門西第一星爲右執法,御史大夫之象也;第二星爲上將;第三星爲次將;第四星爲次相;第五星爲上相。其東垣北左執法、上相兩星閒名曰左掖門;上相兩星閒名曰東華門;上相、次相、上將、次將閒名曰太陽門。其西垣右執法、上將閒名曰右掖門;上將閒名曰西華門;次將、次相閒名曰中華門;次相兩星閒名曰太陰門。各依其名,是其職也。占與紫宮垣同也。西,將;東,相;南四星,執法;中,端門;門左右,掖門。門內六星,諸侯。〔正義〕內五諸侯五星,列在帝庭。其星並欲光明潤澤;若枯燥,則各於其處受其災變,大至誅戮,小至流亡;若動搖,則擅命以干主者。審其分以占之,則無惑也。又云諸侯五星在東井北河,主刺舉,戒不虞。又曰理陰陽,察得失。一曰帝師,二曰帝友,三曰三公,四曰博士,五曰太史。此五者,爲天子定疑議也。占:明大潤澤,大小齊等,則國之福;不然,則上下相猜,忠臣不用。其內五星,五帝坐。〔索隱〕詩含神霧云五精星坐,其東蒼帝坐,神名靈威仰,精爲青龍之類是也。〔正義〕黃帝坐一星,在太微宮中,含樞紐之神。四星夾黃帝坐:蒼帝東方靈威仰之神;赤帝南方赤熛怒之神;白帝西方白昭矩之神;黑帝北方協光紀之神。五帝並設,神靈集謀者也。占:五座明而光,則天子得天地之心;不然,則失位;金、火來守,入太微,若順入,軌道,司其出之所守,則爲天子所誅也;其逆入若不軌道,以所犯名之,中坐成形。後聚一十五星,蔚然,〔集解〕徐廣曰:「一云『哀烏』。」曰郎位;〔索隱〕徐廣云:「一云『哀烏』。」案:漢書作「哀烏」,則「哀烏」「蔚然」皆星之貌狀。其星爲郎位。〔正義〕郎位十五星,在太微中帝坐東北。周之元士,漢之光祿、中散、諫議,此三署郎中,是今之尚書郎。占:欲其大小均耀,光潤有常,吉也。傍一大星,將位也。〔索隱〕案:宋均云爲羣郎之將帥是也。〔正義〕將,子象反。郎將一星,在郎位東北,所以爲武備,今之左右中郎將。占:大而明,角,將恣不可當也。月、五星順入,軌道,〔索隱〕韋昭云:「謂循軌道不邪逆也。順入,從西入之也。」〔正義〕謂月、五星順入軌道,入太微庭也。司其出,所守,天子所誅也。〔索隱〕宋均云:「司察日、月、五星所守列宿,若請官屬不去十日者,於是天子命使誅討之也。」其逆入,若不軌道,以所犯命之;中坐,成形,〔集解〕晉灼曰:「中坐,犯帝坐也。成形,禍福之形見也。」〔索隱〕其逆入,不軌道。宋均云:「逆入,從東入;不軌道,不由康衢而入者也。以其所犯命之者,亦謂隨所犯之位,天子命誅其人也。」 〔正義〕命,名也。謂月、五星逆入,不依軌道,司察其所犯太微中帝坐,帝坐必成其刑戮,皆是羣下相從而謀上也。皆羣下從謀也。金、火尤甚。〔索隱〕案:火主銷物而金爲兵,故尤急。然則木、水、土爲小變也。〔正義〕若金、火逆入,不軌道,犯帝坐,尤甚於月及水、土、木也。廷藩西有隋星五,〔集解〕隋音他果反。〔索隱〕宋均云「南北爲隋」。又他果反,隋爲垂下。曰少微,士大夫。〔索隱〕春秋合誠圖云「少微,處士位」。又天官占云「少微一名處士星」也。〔正義〕廷,太微廷;藩,衞也。少微四星,在太微西,南北列:第一星,處士也;第二星,議士也;第三星,博士也;第四星,大夫也。占以明大黃潤,則賢士舉;不明;反是;月、五星犯守,處士憂,宰相易也。權,軒轅。軒轅,黃龍體。〔集解〕孟康曰:「形如騰龍。」〔索隱〕援神契曰「軒轅十二星,后宮所居。」石氏星讚以軒轅龍體,主后妃也。〔正義〕軒轅十七星,在七星北,黃龍之體,主雷雨之神,後宮之象也。陰陽交感,激爲雷電,和爲雨,怒爲風,亂爲霧,凝爲霜,散爲露,聚爲雲氣,立爲虹蜺,離爲背璚,分爲抱珥。二十四變,皆軒轅主之。其大星,女主也;次北一星,夫人也;次北一星,妃也;其次諸星皆次妃之屬。女主南一小星,女御也;左一星,少民,后宗也;右一星,大民,太后宗也。占:欲其小黃而明,吉;大明,則爲後宮爭競;移徙,則國人流迸;東西角大張而振,后族敗;水、火、金守軒轅,女主惡也。前大星,女主象;旁小星,御者後宮屬。月、五星守犯者,如衡占。〔索隱〕宋均云:「責在后黨嬉,讒賊興,招此祥。」案:亦當天子命誅也。

東井爲水事。〔索隱〕元命包云:「東井八星,主水衡也。」其西曲星曰鉞。〔正義〕東井八星,鉞一星,輿鬼四星,一星爲質,爲鶉首,於辰在未,皆秦之分野。一大星,黃道之所經,爲天之亭候,主水衡事,法令所取平也。王者用法平,則井星明而端列。鉞一星附井之前,主伺奢淫而斬之。占:不欲其明;明與井齊,或搖動,則天子用鉞於大臣;月宿井,有風雨之變也。鉞北,北河;南,南河;〔正義〕南河三星,北河三星,分夾東井南北,置而爲戒。南河南戒,一曰陽門,亦曰越門;北河北戒,一曰陰門,亦爲胡門。兩戒閒,三光之常道也。占以南星不見則南道不通,北亦如之;動搖及火守,中國兵起也。又云動則胡、越爲變,或連近臣以結之。兩河、天闕閒爲關梁。〔索隱〕宋均云:「兩河六星,知逆邪。言關梁之限,知邪偽也。」〔正義〕闕丘二星在南河南,天子之雙闕,諸侯之兩觀,亦象魏縣書之府。金、火守之,主兵戰闕下也。輿鬼,鬼祠事;中白者爲質。〔集解〕晉灼曰:「輿鬼五星,其中白者爲質。」〔正義〕輿鬼四星,主祠事,天目也,主視明察姦謀。東北星主積馬,東南星主積兵,西南星主積布帛,西北星主積金玉,隨其變占之。中一星爲積屍,一名質,主喪死祠祀。占:鬼星明大,穀成;不明,百姓散。質欲其沒不明;明則兵起,大臣誅,下人死之。火守南北河,兵起,穀不登。故德成衡,觀成潢,〔集解〕晉灼曰:「日、月、五星不軌道也。衡,太微廷也。觀,占也。潢,五帝車舍。」傷成鉞,〔集解〕晉灼曰:「賊傷之占,先成形於鉞。」〔索隱〕案:德成衡,衡則能平物,故有德公平者,先成形於衡。觀成潢,爲帝車舍,言王者遊觀,亦先成形於潢也。傷成鉞者,傷,敗也,言王者敗德,亦先成形於鉞,以言有敗亂則有鉞誅之。然案文耀鉤則云「德成潢,敗成鉞」,其意異也。又此下文「禍成井,誅成質」,皆是東井下義。總列於此也。禍成井,〔集解〕晉灼曰:「東井主水事,火入一星居其旁,天子且以火敗,故曰禍也。」誅成質。〔集解〕晉灼曰:「熒惑入輿鬼、天質,占曰大臣有誅。」

栁爲鳥注,主木草。〔索隱〕案:漢書天文志「注」作「喙」。爾雅云「鳥喙謂之柳」。孫炎云「喙,朱鳥之口,柳其星聚也」。以注爲柳星,故主草木。 〔正義〕喙,丁救反,一作「注」。柳八星,星七星,張六星,爲鶉火,於辰在午,皆周之分野。柳爲朱鳥咮,天之厨宰,主尚食,和滋味。占以順明爲吉;金、火守之,國兵大起。七星,頸,爲員官。主急事。〔索隱〕七星,頸,爲員宮,主急事。案:宋均云「頸,朱鳥頸也。員宮,喉也。物在喉嚨,終不久留,故主急事也」。〔正義〕七星爲頸,一名天都,主衣裳文繡,主急事。以明爲吉,暗爲凶;金、火守之,國兵大起。張,素,爲厨,主觴客。〔索隱〕素,嗉也。爾雅云「鳥張嗉」。郭璞云「嗉,鳥受食之處也」。〔正義〕張六星,六爲嗉,主天厨食飲賞賚觴客。占以明爲吉,暗爲凶。金、火守之,國兵大起。翼爲羽翮,主逺客。〔正義〕翼二十二星,軫四星,長沙一星,轄二星,合軫七星皆爲鶉尾,於辰在巳,楚之分野。翼二十二星爲天樂府,又主夷狄,亦主逺客。占:明大,禮樂興,四夷服;徙,則天子舉兵以罰亂者。

軫爲車,主風。〔索隱〕宋均云:「軫四星居中,又有二星爲左右轄,車之象也。軫與巽同位,爲風,車動行疾似之也。」〔正義〕軫四星,主冢宰輔臣,又主車騎,亦主風。占:明大,則車騎用;太白守之,天下學校散,文儒失業,兵戈大興;熒惑守之,南方有不用命之國,當發兵伐之;辰星守之,徐、泗有戮之者。其旁有一小星,曰長沙,〔正義〕長沙一星在軫中,主壽命。占:明,主長壽,子孫昌也。星星不欲明;明與四星等,若五星入軫星中,兵大起。〔索隱〕宋均云:「五星主行使。使動,兵車亦動也。」軫南衆星曰天庫樓;〔正義〕天庫一星,主太白,秦也,在五車中。庫有五車。車星角若益衆,及不具,無處車馬。

西宮〔索隱〕文耀鉤云:「西宮白帝,其精白虎。」咸池,〔正義〕咸池三星,在五車中,天潢南,魚鳥之所託也。金犯守之,兵起;火守之,有災也。曰天五潢。五潢,五帝車舍。〔索隱〕案:元命包云「咸池主五穀,有星五者各有所職。咸池,言穀生於水,含秀含實,主秋垂,故一名『五帝車舍』,以車載穀而販也」。〔正義〕五車五星,三柱九星,在畢東北,天子五兵車舍也。西北大星曰天庫,主太白,秦也。次東北曰天獄,主辰,燕、趙也。次東曰天倉,主歲,衞、魯也。次東南曰司空,主鎮,楚也。次西南曰卿,主熒惑,魏也。占:五車均明,柱皆見,則倉庫實;不見,其國絕食,兵見起。五車、三柱有變,各以其國占之。三柱入出一月,米貴三倍,期二年;出三月,貴十倍,期三年;柱出不與天倉相近,軍出,米貴,轉粟千里;柱倒出,尤甚。火入,天下旱;金入,兵;水入,水也。火入,旱;金,兵;水,水。〔索隱〕謂火、金、水入五潢,則各致此災也。案:宋均云「不言木、土者,木、土德星,於此不爲害故也」。中有三柱;柱不具,兵起。

奎曰封豕,爲溝瀆。〔正義〕奎,苦圭反,十六星。婁三星爲降婁,於辰在戌,魯之分野。奎,天之府庫,一曰天豕,亦曰封豕,主溝瀆。西南大星,所謂天豕目。占以明爲吉。星不欲團圓,團圓則兵起。暗則臣干命之咎,亦不欲開闔無常,當有白衣稱命於山谷者。五星犯奎,人主爽德,權臣擅命,不可禁者。王者宗祀不潔,則奎動搖。若燄燄有光,則近臣謀上之應,亦庶人饑饉之厄。太白守奎,胡、貊之憂,可以伐之。熒惑星守之,則有水之憂,連以三年。填星、歲星守之,中國之利,外國不利,可以興師動衆,斬斷無道。婁爲聚衆。〔正義〕婁三星爲苑,牧養犧牲以共祭祀,亦曰聚衆。占:動搖,則衆兵聚;金、火守之,兵起也。胃爲天倉。〔正義〕胃三星,昴七星,畢八星,爲大梁,於辰在酉,趙之分野。胃主倉廩,五穀之府也。占:明則天下和平,五穀豐稔;不然,反是也。其南衆星曰廥積。〔集解〕如淳曰:「芻積爲廥也。」〔正義〕芻六星,在天苑西,主積草者。不見,則牛馬暴死;火守,災起也。

昴曰髦頭,〔正義〕昴七星爲髦頭,胡星,亦爲獄事。明,天下獄訟平;暗爲刑罰濫。六星明與大星等,大水且至,其兵大起;搖動若跳躍者,胡兵大起;一星不見,皆兵之憂也。胡星也,爲白衣會。畢曰罕車,〔索隱〕爾雅云「濁謂之畢」。孫炎以爲掩兔之畢或呼爲濁,因名星云。〔正義〕畢八星,曰罕車,爲邊兵,主弋獵。其大星曰天高,一曰邊將,主四夷之尉也。星明大,天下安,逺夷入貢;失色,邊亂。畢動,兵起;月宿則多雨。毛萇云「畢所以掩兔也」。爲邊兵,主弋獵。其大星旁小星爲附耳。〔正義〕附耳一星,屬畢大星之下,次天高東南隅,主爲人主聽得失,伺愆過。星明,則中國微,邊寇警;移動,則讒佞行;入畢,國起兵。附耳搖動,有讒亂臣在側。昴、畢間爲天街。〔索隱〕元命包云:「畢爲天階。」爾雅云:「大梁,昴。」孫炎云:「昴、畢之閒,日、月、五星出入要道,若津梁也。」〔正義〕天街二星,在畢、昴閒,主國界也。街南爲華夏之國,街北爲夷狄之國。土、金守,胡兵入也。其陰,陰國;陽,陽國。〔集解〕孟康曰:「陰,西南,象坤維,河山已北國;陽,河山已南國。」

參爲白虎。〔正義〕觜三星,參三星,外四星爲實沈,於辰在申,魏之分野,爲白虎形也。參,色林反,下同。三星直是也,爲衡石。〔集解〕孟康曰:「參三星者,白虎宿中,東西直,似稱衡。」下有三星,兌,曰罰,〔集解〕孟康曰:「在參閒。上小下大,故曰銳。」晉灼曰:「三星少斜列,無銳形。」〔正義〕罰,亦作「伐」。春秋運斗樞云「參伐事主斬艾」也。爲斬艾事。其外四星,左右肩股也。小三星隅置,曰觜觿,爲虎首,主葆旅事。〔集解〕如淳曰:「關中俗謂桑榆孽生爲葆。」晉灼曰:「葆,菜也。禾野生曰旅,今之飢民采旅也。」〔索隱〕姚氏案:「宋均云葆,守也。旅猶軍旅也。言佐參伐以斬艾除凶也。」〔正義〕觜,子思反。觿,胡規反。葆音保。觜觿爲虎首,主收斂葆旅事也。葆旅,野生之可食者。占:金、水來守,國易正,災起也。其南有四星,曰天厠。〔正義〕天廁四星,在屏東,主溷也。占:色黃,吉;青與白,皆凶;不見,則人寢疾。厠下一星,曰天矢。〔正義〕天矢一星,在厠南。占與天厠同也。矢黃則吉;青、白、黑,凶。其西有句曲〔正義〕包音鉤。九星,三處羅:一曰天旗,〔正義〕參旗九星,在參西,天旗也,指麾逺近以從命者。王者斬伐當理,則天旗曲直順理;不然,則兵動於外,可以憂之。若明而稀,則邊寇動;不然,則不。二曰天苑,〔正義〕天苑十六星,如環狀,在畢南,天子養禽獸所。稀暗,則多死也。三曰九游。〔集解〕徐廣曰:「音流。」〔正義〕九游九星,在玉井西南,天子之兵旗,所以導軍進退,亦領州列邦。並不欲搖動,搖動則九州分散,人民失業,信命一不通,於中國憂。以金、火守之,亂起也。其東有大星曰狼。〔正義〕狼一星,參東南。狼爲野將,主侵掠。占:非其處,則人相食;色黃白而明,吉;赤,角,兵起;金、木、火守,亦如之。狼角變色,多盜賊。下有四星曰弧,〔正義〕弧九星,在狼東南,天之弓也。以伐叛懷逺,又主備賊盜之知姦邪者。弧矢向狼動移,多盜;明大變色,亦如之。矢不直狼,又多盜;引滿,則天下盡兵也。直狼。狼比地有大星,〔集解〕晉灼曰:「比地,近地也。」曰南極老人。〔正義〕老人一星,在弧南,一曰南極,爲人主占壽命延長之應。常以秋分之曙見於景,春分之夕見於丁。見,國長命,故謂之壽昌,天下安寧;不見,人主憂也。老人見,治安;不見,兵起。常以秋分時候之于南郊。

附耳入畢中,兵起。

北宮玄武,〔索隱〕文耀鉤云:「北宮黑帝,其精玄武。」〔正義〕南斗六星,牽牛六星,並北宮玄武之宿。虛、危。〔索隱〕爾雅云「玄枵,虛也」。又云「北陸,虛也」。解者以陸爲道。孫炎曰「陸,中也;北方之宿中也」。〔正義〕虛二星,危三星,爲衣枵,於辰在子,齊之分野。虛主死喪哭泣事,又爲邑居廟堂祭祀禱祝之事;亦天之冢宰,主平理天下,覆藏萬物。占:動,則有死喪哭泣之應;火守,則天子將兵;水守,則人饑饉;金守,臣下兵起。危爲宗廟祀事,主天市架屋。占:動,則有土功;火守,天下兵;水守,下謀上也。危爲蓋屋;〔索隱〕宋均云:「危上一星高,旁兩星隋下,似乎蓋屋也。」〔正義〕蓋屋二星,在危南,主天子所居宮室之官也。占:金、火守入,國兵起;孛,彗尤甚。危爲架屋,蓋屋自有星,恐文誤也。虛爲哭泣之事。〔索隱〕虛爲哭泣事。姚氏案荊州占,以爲其宿二星,南星主哭泣。虛中六星,不欲明,明則有大喪也。

其南有衆星,曰羽林天軍。〔正義〕羽林四十五星,三三而聚,散在壘壁南,天軍也。亦天宿衞之兵革出。不見,則天下亂;金、火、水入,軍起也。軍西爲壘,〔正義〕壘壁陳十二星,橫列在營室南,天軍之垣壘。占:五星入,皆兵起,將軍死也。或曰鉞。旁有一大星爲北落。北落若微亡,軍星動角益希,及五星犯北落,〔正義〕北落師門一星,在羽林西南。天軍之門也。長安城北落門,以象此也。主非常,以候兵。占:明,則軍安;微弱,則兵起;金、火守,有兵,爲虜犯塞;土、木則吉。入軍,軍起。火、金、水尤甚:火,軍憂;水,患;木、土,軍吉。〔集解〕漢書音義曰:「木星、土星入北落,則吉也。」危東六星,兩兩相比,曰司空。〔正義〕比音鼻。比,近也。危東兩兩相比者,是司命等星也。司空唯一星耳,又不在危東,恐「命」字誤爲「空」也。司命二星,在虛北,主喪送;司祿二星,在司命北,主官司;危二星,在司祿北,主危亡;司非二星,在危北,主愆過:皆寘司之職。占:大,爲君憂;常則吉也。

營室〔索隱〕元命包云:「營室十星,埏陶精類,始立紀綱,包物爲室。」又爾雅云:「營室謂之定。」郭璞云:「定,正也。天下作宮室,皆以營室中爲正也。」爲清廟,曰離宮、閣道。〔索隱〕案:荊州占云「閣道,王良旗也,有六星」。漢中四星,曰天駟。〔索隱〕案:元命包云「漢中四星曰騎,一曰天駟也」。旁一星,曰王良。〔索隱〕春秋合誠圖云:「王良主天馬也。」〔正義〕王良五星,在奎北河中,天子奉御官也。其動策馬,則兵騎滿野;客星守之,津橋不通;金、火守入,皆兵之憂。王良策馬,〔正義〕策一星,在王良前,主天子僕也。占以動搖移在王良前,或居馬後,別爲策馬,策馬而兵動也。案:豫章周騰字叔達,南昌人,爲侍御史。桓帝當南郊,平明應出,騰仰觀,曰:「夫王者象星,今宮中星及策馬星悉不動,上明日必不出。」至四更,皇太子卒,遂止也。車騎滿野。旁有八星,絕漢,曰天潢。〔索隱〕元命包曰:「潢主河渠,所以度神,通四方。」宋均云:「天潢,天津也。津,湊也,故主計度也。」天潢旁,江星。〔正義〕天江四星,在尾北,主太陰也。不欲明;明而動,水暴出;其星明大,水不禁也。江星動,人涉水。

杵、臼四星,在危南。〔正義〕杵、臼三星,在丈人星旁,主軍糧。占:正下直臼,吉;與臼不相當,軍糧絕也。臼星在南,主舂。其占:覆則歲大饑,仰則大熟也。匏瓜,〔索隱〕案:荊州占云「匏瓜,一名天雞,在河鼓東。匏瓜明,歲則大熟也」。〔正義〕匏音白包反。匏瓜五星,在離珠北,天子果園。占:明大光潤,歲熟;不,則包果之實不登;客守,魚鹽貴也。有青黑星守之,魚鹽貴。

南斗〔正義〕南斗六星,在南也。爲廟,其北建星。〔正義〕建六星,在斗北,臨黃道,天之都關也。斗建之閒,七耀之道,亦主旗輅。占:動搖,則人勞;不然,則不;月暈,蛟龍見,牛馬疫;月、五星犯守,大臣相謀爲,關梁不通及大水也。建星者,旗也。牽牛爲犧牲。〔正義〕牽牛爲犧牲,亦爲關梁。其北二星,一曰即路,一曰聚火。又上一星,主道路;次二星,主關梁;次三星,主南越。占:明大,關梁通;不明,不通,天下牛疫死;移入漢中,天下乃亂。其北河鼓。〔索隱〕爾雅云:「河鼓謂之牽牛。」孫炎曰:「河鼓之旗十二星,在牽牛北。或名河鼓爲牽牛也。」河鼓大星,上將;左右,左右將。〔正義〕河鼓三星,在牽牛北,主軍鼓。蓋天子三將軍,中央大星大將軍,其南左星左將軍,其北右星右將軍,所以備關梁而拒難也。占:明大光潤,將軍吉;動搖差戾,亂兵起;直,將有功;曲,則將失計也。自昔傳牽牛織女七月七日相見,此星也。婺女,〔索隱〕爾雅云「須女謂之務女」是也。一作「婺」。〔正義〕須女四星,亦婺女,天少府也。南斗、牽牛、須女皆爲星紀,於辰在丑,越之分野,而斗牛爲吳之分野也。,須女,賤妾之稱,婦職之卑者,主布帛裁製嫁娶。占:水守之,萬物不成;火守,布帛貴,人多死;土守,有女喪;金守,兵起也。其北織女。〔正義〕織女三星,在河北天紀東,天女也,主果蓏絲帛珍寶。占:王者至孝於神明,則三星俱明;不然,則暗而微,天下女工廢;明,則理;大星怒而角,布帛涌貴;不見,則兵起。晉書天文志云:「晉太史令陳卓總甘、石、巫咸三家所著星圖,大凡二百八十三官,一千四百六十四星,以爲定紀。今略其昭昭者,以備天官云。」織女,天女孫也。〔集解〕徐廣曰:「孫,一作『名』。」〔索隱〕織女,天孫也。案:荊州占云「織女,一名天女,天子女也」。

察日、月之行〔正義〕晉灼云:「太歲在四仲,則歲行三宿;太歲在四孟四季,則歲行二宿。二八十六,三四十二,而行二十八宿,十二歲而周天。」以揆歲星順逆。〔索隱〕姚氏案:天官占云「歲星,一曰應星,一曰經星,一曰紀星」。物理論云「歲行一次,謂之歲星,則十二歲而星一周天也」。〔正義〕天官云:「歲星者,東方木之精,蒼帝之象也。其色明而內黃,天下安寧。夫歲星欲春不動,動則農廢。歲星盈縮,所在之國不可伐,可以罰人;失次,則民多病;見,則喜。其所居國,人主有福,不可以搖動。人主怒,無光,仁道失。歲星順行,仁德加也。歲星農官,主五穀。」天文志云:「春日,甲乙;四時,春也 。五常,仁;五事,貌也。人主仁虧,貌失,逆時令,傷木氣,則罰見歲星。」曰東方木,主春,日甲乙。義失者,罰出歲星。歲星贏縮,〔索隱〕案:天文志曰「凡五星早出爲贏,贏爲客;晚出爲縮,縮爲主人。五星贏縮,必有天應見杓也」。以其舍命國。〔正義〕舍,所止宿也。命,名也。所在國不可伐,可以罰人。其趨舍〔索隱〕趨音聚,謂促。而前曰贏,退舍曰縮。贏,其國有兵不復;縮,其國有憂,將亡,〔正義〕將音子匠反。國傾敗。其所在,五星皆從而聚〔索隱〕案:漢高帝元年,五星皆聚于東井是也。據天文志,其年歲星在東井,故四星從而聚之也。於一舍,其下之國可以義致天下。

以攝提格歲:〔索隱〕太歲在寅,歲星正月晨出東方。案:爾雅「歲在寅爲攝提格」。李廵云「言萬物承陽起,故曰攝提格。格,起也」。歲陰左行在寅,歲星右轉居丑。正月,與斗、牽牛晨出東方,名曰監德。〔索隱〕歲星正月晨見東方之名。已下出石氏星經文,乃云「星在斗牽牛,失次見杓」也。漢書天文志則載甘氏及太初星曆,所在之宿不同也。色蒼蒼有光。其失次,有應見栁。歲早,水;晚,旱。

歲星出,東行十二度,百日而止,反逆行;逆行八度,百日,復東行。歲行三十度十六分度之七,率日行十二分度之一,十二歲而周天。出常東方,以晨;入於西方,用昬。

單閼歲:〔索隱〕在卯也。歲星二月晨出東方。爾雅云「卯爲單閼」。李廵云:「陽氣推萬物而起,故曰單閼。單,盡也。閼,止也。」歲陰在卯,星居子。以二月與婺女、虛、危晨出,曰降入。〔索隱〕即歲星二月晨見東方之名。其餘並準此。大有光。其失次,有應見張,名曰降入,其歲大水。

執徐歲:〔索隱〕爾雅「辰爲執徐」。李廵云:「伏蟄之物皆敦舒而出,故曰執徐。執,蟄;徐,舒也。」歲陰在辰,星居亥,以三月居,與營室、東壁晨出,曰青章。青青甚章。其失次;有應見軫,曰青章。歲早,旱;晚,水。

大荒駱歲:〔索隱〕爾雅云「在巳爲大荒駱」。姚氏云:「言萬物皆熾盛而大出,霍然落落,故曰荒駱也。」歲陰在巳,星居戌。以四月與奎、婁、胃、昴晨出,曰跰踵。〔集解〕徐廣曰:「一曰『路璋』。」〔索隱〕天文志作「路𡺽」。字詁云𡺽,今作「踵」也。〔正義〕跰,白邊反。踵,之勇反。熊熊赤色,有光。其失次,有應見亢。

敦牂歲:〔索隱〕爾雅云「在午爲敦牂」。孫炎云「敦,盛;牂,壯也。言萬物盛壯」。韋昭云「敦音頓」也。歲陰在午,星居酉。以五月與胃、昴、畢晨出,曰開明。〔集解〕徐廣曰:「一曰『天津』。」〔索隱〕天文志作「啟明」。炎炎有光。〔正義〕炎,鹽驗反。偃兵;唯利公王,不利治兵。其失次,有應見房。歲早,旱;晚,水。

叶洽歲:〔索隱〕爾雅云「在未爲協洽」。李廵云:「陽氣欲化萬物,故曰協洽。協,和;洽,合也。」歲陰在未,星居申。以六月與觜觿、〔正義〕觜,子斯反。觿,胡規反。參晨出,曰長列。昭昭有光。利行兵。其失次,有應見箕。

涒灘歲:〔索隱〕涒曰爾雅云「在申爲涒灘」。李廵曰:「涒灘,物吐秀傾垂之貌也。」涒音他昆反,灘音他丹反。歲陰在申,星居未。以七月與東井、輿鬼晨出,曰大音。昭昭白。其失次,有應見牽牛。

作鄂歲:〔索隱〕爾雅「在酉爲作鄂」。李廵云「作咢,皆物芒枝起之貌」。咢音愕。今案:下文云「作鄂有芒」,則李廵解亦近得。天文志云「作詻」,音五格反,與史記及爾雅並異也。歲陰在酉,星居午。以八月與栁、七星、張晨出,曰爲長王。作作有芒。國其昌,熟穀。其失次,有應見危曰大章。有旱而昌,有女喪,民疾。

閹茂歲:〔索隱〕爾雅云「在戌曰閹茂」。孫炎云「萬物皆蔽冒,故曰閹茂。閹,蔽也;茂,冒也」。天文志作「掩茂」也。歲陰在戌,星居巳。以九月與翼、軫晨出,曰天睢。〔索隱〕劉氏音吁唯反也。白色大明。其失次,有應見東壁。歲水,女喪。

大淵獻歲:〔索隱〕爾雅云「在亥爲大淵獻」。孫炎云:「淵,深也。大獻萬物於深,謂蓋藏之於外耳。」歲陰在亥,星居辰。以十月與角、亢晨出,曰大章。〔集解〕徐廣曰:「一曰『天皇』。」〔索隱〕徐廣云一作「天皇」。案:天文志亦作「天皇」也。蒼蒼然,星若躍而陰出旦,是謂「正平」。起師旅,其率必武;其國有德,將有四海。其失次,有應見婁。

困敦歲:〔索隱〕爾雅「在子爲困敦」。孫炎云:「困敦,混沌也。言萬物初萌,混沌於黃泉之下也。」歲陰在子,星居卯。以十一月與氐、房、心晨出,曰天泉。玄色甚明。江池其昌,不利起兵。其失次,有應在昴。

赤奮若歲:〔索隱〕爾雅「在丑爲赤奮若」。李廵云:「言陽氣奮迅。若,順也。」歲陰在丑,星居寅,以十二月與尾、箕晨出,曰天皓。〔索隱〕音昊。漢志作「昊」。黫然〔索隱〕於閑反。黑色甚明。其失次,有應見參。

當居不居,居之又左右搖,未當去去之,與他星會,其國凶。所居久,國有德厚。其角動,乍小乍大,若色數變,人主有憂。

其失次舍以下,進而東北,三月生天棓,〔正義〕棓音蒲講反。歲星之精散而爲天槍、天棓、天衝、天猾、國皇、天欃,及登天、荊真,若天猿、天垣、蒼彗,皆以廣凶災也。天棓者,一名覺星,本類星而末銳,長四丈,出東北方、西方。其出,則天下兵爭也。長四丈,〔索隱〕案天文志,此皆甘氏星經文,而志又兼載石氏,此不取。石氏名申夫,甘氏名德。末兌,進而東南,三月生彗星,〔正義〕天彗者,一名埽星,本類星,末類彗,小者數寸長,長或竟天,而體無光,假日之光,故夕見則東指,晨見則西指,若日南北,皆隨日光而指。光芒所及爲災變,見則兵起;除舊布新,彗所指之處弱也。長二丈,類彗。退而西北,三月生天欃,〔集解〕韋昭曰:「欃音『參差』之『參』。」〔正義〕欃,楚咸反。天欃者,在西南,長四丈,銳。京房云「天欃爲兵,赤地千里。枯骨籍籍」。天文志云天槍主兵亂也。長四丈,末兌。退而西南,三月生天槍,〔正義〕槍,楚行反。天槍者,長數丈,兩頭銳,出西南方。其見,不過三月,必有破國亂君伏死其辜。天文志云「孝文時,天槍夕出西南,占曰爲兵喪亂,其六年十一月,匈奴入上郡、雲中,漢起兵以衞京師」也。長數丈,兩頭兌。謹視其所見之國,不可舉事用兵。其出如浮如沈,其國有土功;如沈如浮,其野亡。色赤而有角,其所居國昌。迎〔集解〕徐廣曰:「一作『御』。」角而戰者,不勝。星色赤黃而沈,所居野大穰。〔正義〕穰,人羊反,豐熟也。色青白而赤灰,所居野有憂。歲星入月,其野有逐相;與太白鬬,〔集解〕韋昭曰:「星相擊爲鬬。」其野有破軍。

歲星一曰攝提,曰重華,曰應星,曰紀星。營室爲清廟,歲星廟也。

察剛氣〔集解〕徐廣曰:「剛,一作『罰』。」 〔索隱〕徐廣云剛一作「罰」。案:姚氏引廣雅「熒惑謂之執法」。天官占云「熒惑方伯象,司察妖孽」。則此文「察罰氣」爲是。以處熒惑。〔索隱〕春秋緯文耀鉤云:「赤帝熛怒之神,爲熒惑焉,位在南方,禮失則罰出。」晉灼云:「常以十月入太微,受制而出行列宿,司無道,出入無常。」曰南方火,主夏,日丙、丁。禮失,罰出熒惑,熒惑失行是也。出則有兵,入則兵散。以其舍命國。熒惑,熒惑爲勃亂,殘賊、疾、喪、饑、兵。〔集解〕徐廣曰:「以下云『熒惑爲理,外則理兵,內則理政』。」〔正義〕天官占云:「熒惑爲執法之星,其行無常,以其舍命國:爲殘賊,爲疾,爲喪,爲饑,爲兵。環繞句己,芒角動搖,乍前乍後,其殃逾甚。熒惑主死喪,大鴻臚之象;主甲兵,大司馬之義;伺驕奢亂孽,執法官也。其精爲風伯,惑童兒歌謠嬉戲也。」反道二舍以上,居之,三月有殃,五月受兵,七月半亡地,九月太半亡地。因與俱出入,國絕祀。居之,殃還至,雖大當小;〔索隱〕案還音旋。旋,疾也。若熒惑反道居其舍,所致殃禍速至,則雖大反小。久而至,當小反大。〔索隱〕案:久謂行遲也。如此,禍小反大,言久腊毒也。其南爲丈夫,北爲女子喪。〔索隱〕案:宋均云「熒惑守輿鬼南,爲丈夫受其咎;北,則女子受其凶也」。若角動繞環之,及乍前乍後,左右,殃益大。與他星鬬,〔正義〕凡五星鬬,皆爲戰鬬。兵不在外,則爲內亂。鬬謂光芒相及。光相逮,爲害;不相逮,不害。五星皆從而聚于一舍,〔正義〕三星若合,是謂驚立絕行,其國外內有兵與喪,人民饑乏,改立侯王。四星若合,是爲大陽,其國兵喪暴起,君子憂,小人流。五星若合,是謂易行,有德者受慶,掩有四方;無德者受殃,乃以死亡也。其下國可以禮致天下。

法,出東行十六舍而止;逆行二舍;六旬,復東行,自所止數十舍,十月而入西方;伏〔集解〕晉灼曰:「伏不見。」行五月,出東方。其出西方曰「反明」,主命者惡之。東行急,一日行一度半。

其行東、西、南、北疾也。兵各聚其下;用戰,順之勝,逆之敗。熒惑從太白,軍憂;離之,軍却。出太白陰,有分軍;行其陽,有偏將戰。當其行,太白逮之,破軍殺將。〔索隱〕宋均云:「太白宿,主軍來衝拒也。」其入守犯太微、〔集解〕孟康曰:「犯,七寸已內光芒相及也。」韋昭曰:「自下觸之曰『犯』,居其宿曰『守』。」軒轅、營室,主命惡之。心爲明堂,熒惑廟也。謹候此。

曆斗之會以定填星之位。〔索隱〕曆斗之會以定鎮星之位。晉灼曰:「常以甲辰之元始建斗,歲鎮一宿,二十八歲而周天。」廣雅曰:「鎮星,一名地侯。」文耀鉤云:「鎮,黃帝含樞紐之精,其體旋璣,中宿之分也。」曰中央土,主季夏,日戊、己,黃帝,主德,女主象也。歲填一宿,其所居國吉。未當居而居,若已去而復還,還居之,其國得土,不乃得女。若當居而不居,旣已居之,又西東去,其國失土,不乃失女,不可舉事用兵。其居久,其國福厚;易,福薄。〔集解〕徐廣曰:「易猶輕速也。」

其一名曰地侯,主歲。歲行十二度百十二分度之五,日行二十八分度之一,二十八歲周天。其所居,五星皆從而聚于一舍,其下之國,可重致天下。〔正義〕重音逐隴反。言五星皆從填星,其下之國倚重而致天下,以填主土故也。禮、德、義、殺、刑盡失,而填星乃爲之動搖。

贏,爲王不寧;其縮,有軍不復。填星,其色黃,九芒,音曰黃鐘宮。其失次上二三宿曰贏,有主命不成,不乃大水。失次下二三宿曰縮,有后戚,其歲不復,不乃天裂若地動。

斗爲文太室,填星廟,天子之星也。

木星與土合,爲內亂。饑,〔正義〕星經云:「凡五星,木與土合爲內亂,饑;與水合爲變謀,更事;與火合爲旱;與金合爲白衣會也。」主勿用戰,敗;水則變謀而更事;火爲旱;金爲白衣會若水。金在南牝牡,〔索隱〕晉灼曰:「歲,陽也,太白,陰也,故曰牝牡也。」〔正義〕星經云:「金在南,木在北,名曰牝牡,年穀大熟;金在北,木在南,其年或有或無。」年穀熟,金在北,歲偏無。火與水合爲焠,〔集解〕晉灼曰:「火入水,故曰焠。」〔索隱〕火與水合曰焠。案:謂火與水俱從填星合也。〔正義〕焠,忽內反。星經云:「凡五星,火與水合爲焠,用兵舉事大敗;與金合爲鑠,爲喪,不可舉事,用兵從軍爲憂;離之,軍卻;與土合爲憂,主孽卿;與木合,饑,戰敗也。」與金合爲鑠,爲喪,皆不可舉事,用兵大敗。土爲憂,主孽卿;〔索隱〕案:文耀鉤云「水土合則成鑪冶,鑪冶成則火興,火興則土之子焠,金成消爍,消爍則土無子輔父,無子輔父則益妖孽,故子憂」。大饑,戰敗,爲北軍,〔正義〕爲北,軍北也。凡軍敗曰北。軍困,舉事大敗。土與水合,穰而擁閼,〔正義〕擁,於拱反。閼,烏葛反。有覆軍,〔集解〕徐廣曰:「或云木、火、土三星若合,是謂驚立絕行。」其國不可舉事。出,亡地;入,得地。金爲疾,爲內兵,亡地。三星若合,其宿地國外內有兵與喪,改立公王。四星合,兵喪並起,君子憂,小人流。五星合,是爲易行,有德,受慶,改立大人,掩有四方,子孫蕃昌;無德,受殃若亡。五星皆大,其事亦大;皆小,事亦小。

蚤出者爲贏,贏者爲客。晚出者爲縮,縮者爲主人。必有天應見於杓星。同舍爲合。相陵爲鬬,〔集解〕孟康曰:「陵,相冒占過也。」韋昭曰:「突掩爲陵。」七寸以內必之矣。〔索隱〕案:韋昭云必有禍也。

五星色白圜,爲喪旱;赤圜,則中不平,爲兵;青圜,爲憂水;黑圜,爲疾,多死;黃圜,則吉。赤角犯我城,黃角地之爭,白角哭泣之聲,青角有兵憂,黑角則水。意,〔集解〕徐廣曰:「一作『志』。」行窮兵之所終。五星同色,天下偃兵,百姓寧昌。春風秋雨,冬寒夏暑,動搖常以此。

填星出百二十日而逆西行,西行百二十日反東行。見三百三十日而入,入三十日復出東方。太歲在甲寅,鎮星在東壁,故在營室。

察日行以處位〔索隱〕案太白晨出東方曰啟明,故察日行以處太白之位也。太白。〔索隱〕韓詩云「太白晨出東方爲啟明,昏見西方爲長庚」。又孫炎注爾雅,以爲晨出東方高三丈,命曰啟明;昏見西方高三舍,命曰太白。〔正義〕晉灼云:「常以正月甲寅與熒惑晨出東方,二百四十日而入,入四十日又出西方,二百四十日而入,入三十五日而復出東方。出以寅、戌,入以丑、未。」天官占云:「太白者,西方金之精,白帝之子,上公、大將軍之象也。一名殷星,一名大正,一名熒星,一名官星,一名梁星,一名滅星,一名大囂,一名大衰,一名大爽。徑一百里。」天文志云:「其日庚辛;四時,秋也;五常,義也;五事,言也。人主義虧言失,逆時令,傷金氣,罰見太白:春見東方,以晨;秋見西方,以夕。」曰西方,秋,司兵月行及天矢〔正義〕太白五芒出,早爲月蝕,晚爲天矢及彗。其精散爲天杵、天柎、伏靈、大敗、司姦、天狗、賊星、天殘、卒起星,是古曆星;若竹彗、牆星、猿星、白雚,皆以示變也。日庚、辛,主殺。殺失者,罰出太白。太白失行,以其舍命國。其出行十八舍二百四十日而入。入東方,伏行十一舍百三十日;其入西方,伏行三舍十六日而出。當出不出,當入不入,是謂失舍,不有破軍,必有國君之篡。

其紀上元,〔索隱〕案上元是古曆之名,言用上元紀曆法,則攝提歲而太白與營室晨出東方,至角而入;與營室夕出西方,至角而入。凡出入東西各五,爲八歲二百三十日,復與營室晨出東方。大率歲一周天也。〔正義〕其紀上元,是星古曆初起上元之法也。以攝提格之歲,與營室晨出東方,至角而入;與營室夕出西方,至角而入;與角晨出,入畢;與角夕出,入畢;與畢晨出,入箕;與畢夕出,入箕;與箕晨出,入栁;與箕夕出,入栁;與栁晨出,入營室;與栁夕出,入營室。凡出入東西各五,爲八歲,二百二十日,〔集解〕徐廣曰:「一云『三十二日』。」復與營室晨出東方。其大率,歲一周天。其始出東方,行遲,率日半度,一百二十日,必逆行一二舍;上極而反,東行,行日一度半,一百二十日入。其庳,近日,曰明星,柔;高,逺日,曰大囂,〔正義〕徐廣曰:「一作『變』。」剛。其始,出西行疾,率日一度半,百二十日;上極而行遲,日半度,百二十日,旦入,必逆行一二舍而入。其庳,近日,曰大白,柔;高,逺日,曰大相,剛。出以辰、戌,入以丑、未。

當出不出,未當入而入,天下偃兵,兵在外,入。未當出而出,當入而不入,下起兵,有破國。其當期出也,其國昌。其出東爲東,入東爲北方;出西爲西,入西爲南方。所居久,其鄉利;疾,〔集解〕蘇林曰:「疾過也。」其鄉凶。

出西逆行至東,正西國吉。出東至西,正東國吉。其出不經天;經天,天下革政。〔索隱〕孟康曰:「謂出東入西,出西入東也。太白陰星,出東當伏東,出西當伏西,過午爲經天。」又晉灼曰:「日,陽也,日出則星沒。太白晝見午上爲經天。」

小以角動,兵起。始出大,後小,兵弱;出小,後大,兵強。出高,用兵深吉,淺凶;庳,淺吉,深凶。日方南金居其南,日方北金居其北,曰贏,〔正義〕鄭玄云:「方猶向也。謂晝漏半而置土圭表陰陽,審其南北也。影短於土圭謂之日南,是地於日爲近南也;長於土圭謂之日北,是地於日爲近北也。凡日影於地,千里而差一寸。」周禮云:「日南則影短多暑,日北則影長多寒。」孟康云:「金謂太白也。影,日中之影也。」侯王不寧,用兵進吉退凶。日方南金居其北,日方北金居其南,曰縮,侯王有憂,用兵退吉進凶。用兵象太白:太白行疾,疾行;遲,遲行。角,敢戰。動搖躁,躁。國以靜,靜。順角所指,吉;反之,皆凶。出則出兵,入則入兵。赤角,有戰;白角,有喪;黑圜角,憂,有水事;青圜小角,憂,有木事;黃圜和角,有土事,有年。〔正義〕太白星圓,天下和平;若芒角,有土事。有年謂豐熟也。其已出三日而復,有微入,入三日乃復盛出,是謂耎,〔集解〕晉灼曰:「耎,退之不進。」〔索隱〕耎音奴亂反。其下國有軍敗將北。其已入三日又復微出,出三日而復盛入,其下國有憂;師有糧食兵革,遺人用之;〔正義〕遺,唯季反。卒雖衆,將爲人虜。其出西失行,外國敗;其出東失行,中國敗。其色大圜黃󸽒,〔集解〕音澤。可爲好事;其圜大赤,兵盛不戰。

太白白,比狼;〔正義〕比,卑耳反,下同。比,類也。赤,比心;黃,比參左肩;蒼,比參右肩;黑,比奎大星。〔正義〕晉書天文志云:「凡五星有色,大小不同,各依其行而應時節。色變有類:凡青,比參左肩;赤,比心大星;黃,比參右肩;白,比狼星;黑,比奎大星。不失本色而應其四時者,吉;色害其行,凶也。」五星皆從太白而聚乎一舍,其下之國可以兵從天下。居實,有得也;居虛,無得也。〔索隱〕按:實謂星所合居之宿;虛謂贏縮也。行勝色,〔集解〕晉灼曰:「太白行得度者,勝色也。」〔正義〕勝音升剩反,下同。色勝位,有位勝無位,有色勝無色,行得盡勝之。〔集解〕晉灼曰:「行應天度,唯有色得位;行盡勝之,行重而色位輕。」星經「得」字作「德」。〔正義〕晉書天文志云:「凡五星所出所直之辰,其國爲得位者,歲星以德,熒惑爲禮,鎮星有福,太白兵強,辰陰陽和。所直之辰,順其色而角者勝,其色害者敗;居實有得,居虛無得也。色勝位,行勝色,行得盡勝之。」出而留桑榆閒,〔集解〕晉灼曰:「行遲而下也。正出,舉目平正,出桑榆上者餘二千里。」疾其下國。〔正義〕疾,漢書作「病」也。上而疾,未盡其日,過參天,〔集解〕晉灼曰:「三分天過其一,此在戌酉之閒。」疾其對國。〔集解〕孟康曰:「謂出東入西,出西入東。」上復下,下復上,有反將。其入月,將僇。金、木星合,光,其下戰不合,兵雖起而不鬬;合相毀,野有破軍。出西方,昬而出陰,陰兵彊;暮食出,小弱;夜半出,中弱;雞鳴出,大弱:是謂陰陷於陽。其在東方,乘明而出陽,陽兵之彊,雞鳴出,小弱;夜半出,中弱;昬出,大弱:是謂陽陷於陰。太白伏也,以出兵,兵有殃。其出卯南,南勝北方;出卯北,北勝南方;正在卯,東國利。出酉北,北勝南方;出酉南,南勝北方;正在酉,西國勝。

其與列星相犯,小戰;五星,大戰。其相犯,太白出其南,南國敗;出其北,北國敗。行疾,武;不行,文。色白五芒,出蚤爲月蝕,晚爲天夭及彗星,將發其國。出東爲德,舉事左之迎之,吉。出西爲刑,舉事右之背之,吉。反之皆凶。太白光見景,戰勝。晝見而經天,是謂爭明,彊國弱,小國彊,女主昌。

亢爲疏廟,太白廟也。太白,大臣也,其號上公。其他名殷星、太正、營星、觀星、宮星、明星、大衰、大澤、終星、大相、天浩、序星、月緯。大司馬位謹候此。

察日辰之會,〔索隱〕案:下文「正四時及星辰之會」是也。〔正義〕晉灼云:「常以二月春分見奎、婁,五月夏至見東井,八月秋分見角、亢,十一月冬至見牽牛。出以辰、戌,入以丑、未,二旬而入。晨候之東方,夕候之西方也。」以治辰星之位。〔索隱〕案:皇甫謐曰「辰星,一名毚星,或曰鉤星」。元命包曰「北方辰星水,生物布其紀,故辰星理四時」。宋均曰「辰星正四時之位,得與北辰同名也」。曰北方水,太陰之精,主冬,日壬、癸。刑失者,罰出辰星,〔正義〕天官占云:「辰星,北水之精,黑帝之子,宰相之祥也。一名細極,一名鈎星,一名爨星,一名伺祠。徑一百里。亦偏將、廷尉象也。」天文志云:「其日壬、癸。四時,冬也;五常,智也;五事,聽也。人主智虧聽失,逆時令,傷水氣,則罰見辰星也。」以其宿命國。

是正四時:仲春春分,夕出郊奎、婁、胃東五舍,爲齊;仲夏夏至,夕出郊東井、輿鬼、栁東七舍,爲楚;仲秋秋分,夕出郊角、亢、氐、房東四舍,爲漢;仲冬冬至,晨出郊東方,與尾、箕、斗、牽牛俱西,爲中國。其出入常以辰、戌、丑、未。

其蚤,爲月蝕;〔集解〕孟康曰:「辰星、月相淩不見者,則所蝕也。」〔索隱〕案:宋均云「辰星與月同精,月爲大臣,先期而出,是躁也。失則當誅,故月蝕見祥」。晚,爲彗星〔集解〕張晏曰:「彗,所以除舊布新。」〔索隱〕案:宋均云「辰星,陰也,彗亦陰,陰謀未成,故晚出也」。及天夭。其時宜效不效爲失,〔正義〕效,見也。言宜見不見,爲失罰之也。追兵在外不戰。一時不出,其時不和;四時不出,天下大飢。其當效而出也,色白爲旱,黃爲五穀熟,赤爲兵,黑爲水。出東方,大而白,有兵於外,解。常在東方,其赤,中國勝;其西而赤,外國利。無兵於外而赤,兵起。其與太白俱出東方,皆赤而角,外國大敗,中國勝;其與太白俱出西方,皆赤而角,外國利。五星分天之中,積于東方,中國利;積于西方,外國用者利。五星皆從辰星而聚于一舍,其所舍之國可以法致天下。辰星不出,太白爲客;其出,太白爲主。出而與太白不相從,野雖有軍,不戰。出東方,太白出西方;若出西方,太白出東方,爲格,〔索隱〕謂辰星出西方。辰,水也。太白出東方。太白,金也。水生金,母子不相從,故上有軍不戰。今母子各出一方,故爲格。格謂不和同,故野雖有兵不戰也。野雖有兵不戰。失其時而出,爲當寒反溫,當溫反寒。當出不出,是謂繫卒,兵大起。其入太白中而上出,破軍殺將,客軍勝;下出,客亡地。辰星來抵太白,太白不去,將死。正旗上出,〔索隱〕正旗出。案:旗蓋太白芒角,似旌旗。〔正義〕旗,星名,有九星。言辰星上則破軍殺將,客勝也。破軍殺將,客勝;下出,客亡地。視旗所指,以命破軍。其繞環太白,若與鬬,大戰,客勝。免過太白,〔索隱〕案廣雅云「辰星謂之免星」,則辰星之別名免,或作「毚」也。〔正義〕漢書云「辰星過太白,閒可椷劔」,明廣雅是也。閒可椷劔,〔集解〕蘇林曰:「椷音函。函,容也。其間可容一劔。」〔索隱〕案蘇林所說,則椷字本有函音,故字從咸也。小戰,客勝。免居太白前,軍罷;出太白左,小戰;摩太白,有數萬人戰,主人吏死;出太白右,去三尺,軍急約戰。青角,兵憂;黑角,水。赤行窮兵之所終。

免七命,曰小正、辰星、天欃、安周星、細爽、能星、鈎星。〔索隱〕謂星凡有七名。命者,名也。小正,一也;辰星,二也;天免,三也;安周星,四也;細爽,五也;能星,六也;鈎星,七也。其色黃而小,出而易處,天下之文變而不善矣。免五色,青圜憂,白圜喪,赤圜中不平,黑圜吉。赤角犯我城,黃角地之爭,白角號泣之聲。

其出東方,行四舍四十八日,其數二十日,而反入于東方;其出西方,行四舍四十八日,其數二十日,而反入于西方。其一候之營室、角、畢、箕、栁。出房、心閒,地動。

辰星之色:春,青黃;夏,赤白;秋,青白,而歲熟;冬,黃而不明。即變其色,其時不昌。春不見,大風,秋則不實。夏不見,有六十日之旱,月蝕。秋不見,有兵,春則不生。冬不見,陰雨六十日,有流邑,夏則不長。

角、亢、氐,兖州。房、心,豫州。尾、箕,幽州。斗,江、湖。牽牛、婺女,楊州。虛、危,青州。營室至東壁,并州。奎、婁、胃,徐州。昴、畢,冀州。觜觿、參,益州。〔正義〕括地志云:「漢武帝置十三州,改梁州爲益州廣漢。廣漢,今益州咎縣是也。分今河內、上黨、雲中。」然案星經,益州,魏地,畢、觜、參之分,今河內、上黨、雲中是。未詳也。東井、輿鬼,雍州。栁、七星、張,三河。翼、軫,荊州。

七星爲員官,辰星廟,蠻夷星也。

兩軍相當,日暈;〔集解〕如淳曰:「暈讀曰運。」暈等,力鈞;厚長大,有勝;薄短小,無勝。重抱大破無。抱爲和,背不和爲分離相去。直爲自立,立侯王;指暈,若曰殺將。負且戴,有喜。圍在中,中勝;在外,外勝。青外赤中,以和相去;赤外青中,以惡相去。氣暈先至而後去,居軍勝。先至先去,前利後病;後至後去,前病後利;後至先去,前後皆病,居暈不勝。見而去,其發疾,雖勝無功。見半日以上,功。太白虹屈短,〔集解〕李奇曰:「屈,或爲『尾』也。」韋昭曰:「短而直。」上下兊,有者下大流血。日暈制勝,近期三十日,逺期六十日。

其食,食所不利;復生,生所利;而食益盡,爲主位。以其直及日所宿,加以日時,用命其國也。

月行中道,〔索隱〕案:中道,房星之中閒也。房有四星,若人之房三閒有四表然,故曰房。南爲陽閒,北爲陰閒,則中道房星之中閒也。故房是日、月、五星之行道,然黃道亦經房、心。若月行得中道,故陰陽和平;若行陰閒,多陰事;陽閒,則人主驕恣;若歷陰星、陽星之南北太陰、太陽之道,即有大水若兵,及大旱若喪也。安寧和平。陰閒,多水,陰事。外北三尺,陰星。〔索隱〕案:謂陰閒外北三尺曰陰星,又北三尺曰太陰道,則下陽星及太陽亦在陽閒之南各三尺也。北三尺,太陰,大水,兵。陽閒,驕恣。陽星,多暴獄。太陽,大旱喪也。〔索隱〕太陰,太陽,皆道也。月行近之,故有水旱兵喪也。角、天門,十月爲四月,十一月爲五月,〔索隱〕角閒天門。謂月行入角與天門,若十月犯之,當爲來年四月成災;十一月,則主五月也。十二月爲六月,水發,近三尺,逺五尺。犯四輔,輔臣誅。〔索隱〕案:謂月犯房星也。四輔,房四星也。房以輔心,故曰四輔。行南北河,以陰陽言,旱水兵喪。〔正義〕南河三星,北河三星,若月行北河以陰,則水、兵;南河以陽,則旱、喪也。

月蝕歲星,〔正義〕孟康云:「凡星入月,見月中,爲星蝕月;月掩星,星滅,爲月蝕星也。」其宿地,饑若亡。熒惑也亂,填星也下犯上,太白也彊國以戰敗,辰星也女亂。食大角,〔集解〕徐廣曰:「一云『食于大角』。」〔正義〕大角一星,在兩攝提閒,人君之象也。主命者惡之;心,則爲內賊亂也;列星,其宿地憂。〔索隱〕謂月蝕列星二十八宿,當其分地有憂。憂謂兵及喪也。

月食始日,五月者六,六月者五,五月復六,六月者一,而五月者,凡五百一十三月而復始。〔索隱〕始日謂食始起之日也。依此文計,唯有一百二十一月,與元數甚爲懸校,旣無太初曆術,不可得而推定。今以漢志三統曆法計,則六月者七,五月者一,又六月者一,五月者一,凡一百三十五月而復始耳。或術家各異,或傳寫錯謬,故此不同,無以明知也。故月蝕,常也;日蝕,爲不臧也。甲、乙,四海之外,日月不占。〔集解〕晉灼曰:「海外逺,甲乙日時不以占候。」丙、丁,江、淮、海岱也。戊、己,中州、河、濟也。庚、辛,華山以西。壬、癸,恒山以北。日蝕,國君;月蝕,將相當之。

國皇星,〔正義〕國皇星者,大而赤,類南極老人,去地三丈,如炬火。見則內外有兵喪之難。大而赤,〔集解〕孟康曰:「歲星之精散所爲也。五星之精散爲六十四變,記不盡。」狀類南極。〔集解〕徐廣曰:「老人星也。」所出,其下起兵,兵彊;其衝不利。

昭明星,〔索隱〕案:春秋合誠圖云「赤帝之精,象如太白,七芒」。釋名爲筆星,氣有一枝,末銳似筆,亦曰筆星也。大而白,無角,乍上乍下。〔集解〕孟康曰:「形如三足机,机上有九彗上向,熒惑之精。」所出國,起兵,多變。

五殘星,〔索隱〕孟康云:「星表有青氣如暈,有毛,填星之精也。」〔正義〕五殘,一名五鋒,出正東東方之分野。狀類辰星,去地可六七丈。見則五分毀敗之徵,大臣誅亡之象。出正東東方之野。其星狀類辰星,去地可六丈。

大〔集解〕徐廣曰:「大,一作『六』。」賊星,〔集解〕孟康曰:「形如彗,九尺,太白之精。」〔正義〕大賊星者,一名六賊,出正南,南方之野。星去地可六丈,大而赤,數動有光,出則禍合天下。出正南南方之野。星去地可六丈,大而赤,數動,有光。

司危星,〔集解〕孟康曰:「星大而有尾,兩角,熒惑之精也。」〔正義〕司危者,出正西西方分野也。大如太白,去地可六丈,見則天子以不義失國而豪傑起。出正西西方之野。星去地可六丈,大而白,類太白。

獄漢星,〔集解〕孟康曰:「青中赤表,下有二彗縱橫,亦填星之精。」漢書天文志獄漢一名咸漢。出正北北方之野。星去地可六丈,大而赤,數動,察之中青。此四野星所出,出非其方,其下有兵,衝不利。

四填星,所出四隅,去地可四丈。

地維咸光,亦出四隅,去地可三丈,若月始出。所見,下有亂;亂者亡,有德者昌。

燭星,狀如太白,〔集解〕孟康曰:「星上有三彗上出,亦填星之精。」其出也不行。見則滅。所燭者,城邑亂。

如星非星,如雲非雲,命曰歸邪。〔集解〕李奇曰:「邪音蛇。」孟康曰:「星有兩赤彗上向,上有蓋狀如氣,下連星。」歸邪出,必有歸國者。

星者,金之散氣,本曰火。〔集解〕孟康曰星名。星衆,國吉;少則凶。

漢者,亦金之散氣,〔索隱〕案水生金,散氣即水氣。河圖括地象曰「河精爲天漢」也。其本曰水。漢,星多,多水,少則旱,〔集解〕孟康曰:「漢,河漢也。水生於金。多,少,謂漢中星。」其大經也。

天鼓,有音如雷非雷,音在地而下及地。其所往者,兵發其下。

天狗,狀如大奔星,〔集解〕孟康曰:「星有尾,旁有短彗,下有如狗形者,亦太白之精。」有聲,其下止地,類狗。所墮及炎火,〔索隱〕豔音也。望之如火光炎炎衝天。其下圜如數頃田處,上兊者則有黃色,千里破軍殺將。

格澤星〔索隱〕一音鶴鐸,又音格宅。格,胡客反。者,如炎火之狀。黃白,起地而上。下大,上兊。其見也,不種而穫;不有土功,必有大害。

蚩尤之旗,〔集解〕孟康曰:「熒惑之精也。」晉灼曰:「呂氏春秋曰其色黃上白下。」類彗而後曲,象旗。見則王者征伐四方。

旬始,出於北斗旁,〔集解〕徐廣曰:「蚩尤也。旬,一作『營』。」狀如雄雞。其怒,青黑,象伏鼈。〔集解〕李奇曰:「怒當音帑。」晉灼曰:「帑,雌也。或曰怒則色青。」

枉矢,類大流星,蛇行而倉黑,望之如有毛羽然。

長庚,如一匹布著天。〔正義〕著音直略反。此星見,兵起。

星墜至地,則石也。〔正義〕春秋云「星隕如雨」是也。今吳郡西鄉見有落星石,其石天下多有也。河、濟之閒,時有墜星。

天精而見景星。〔集解〕孟康曰:「精,明也。有赤方氣與青方氣相連,赤方中有兩黃星,青方中一黃星,凡三星合爲景星。」〔索隱〕韋昭云「精謂清朗」。漢書作「夝」,亦作「暒」。郭璞注三蒼云「暒,雨止無雲也」。〔正義〕景星狀如半月,生於晦朔,助月爲明。見則人君有德,明聖之慶也。景星者,德星也。其狀無常,出於有道之國。

凡望雲氣,〔正義〕春秋元命包云:「陰陽聚爲雲氣也。」釋名云:「雲猶云,衆盛也。氣猶餼然也。有聲即無形也。」仰而望之,三四百里;平望,在桑榆上,餘二千里;登高而望之,下屬地者三千里。雲氣有獸居上者,勝。〔正義〕勝音升剩反。雲雨氣相敵也。兵書云:「雲或如雄雞臨城,有城必降。」

自華以南,氣下黑上赤。嵩高、三河之郊,氣正赤。恒山之北,氣下黑上青。勃、碣、海、岱之閒,氣皆黑。江、淮之閒,氣皆白。

徒氣白。土功氣黃。車氣乍高乍下,往往而聚。騎氣卑而布。卒氣摶。〔集解〕如淳曰:「摶,專也。或曰摶,徒端反。」前卑而後高者,疾;前方而後高,兊;而卑者,郄。其氣平者其行徐。前高而後卑者,不止而反。氣相遇者,〔索隱〕遇音偶。漢書作「禺」卑勝高,兊勝方。氣來卑而循車通者,〔集解〕車通,車轍也。避漢武諱,故曰通。不過三四日,去之五六里見。氣來高七八尺者,不過五六日,去之十餘里見。氣來高丈餘二丈者,不過三四十日,去之五六十里見。

稍雲精白者,其將悍,其士怯。其大根而前絕逺者,當戰。青白,其前低者,戰勝;其前赤而仰者,戰不勝。陣雲如立垣。杼雲類杼。〔索隱〕姚氏案:兵書云「營上雲氣如織,勿與戰也。」軸雲摶兩端兊。杓雲〔索隱〕杓,劉氏音時酌反。說文音丁了反。許慎注淮南云「杓,引也」。如繩者,居前亘天,其半半天。其蛪〔索隱〕五結反。亦作「蜺」,音同。者類闕旗故。鉤雲句曲。〔正義〕句音古侯反。諸此雲見,以五色合占。而澤摶密,〔正義〕崔豹古今注云:「黃帝與蚩尤戰於涿鹿之野,常有五色雲氣,金枝玉葉,止於帝上,有花蘤之象,故因作華蓋也。」京房易兆候云:「視四方常有大雲,五色具,其下賢人隱也。青雲潤澤蔽日在西北,爲舉賢良也。」其見動人,及有占;兵必起,合鬬其直。

王朔所候,決於日旁。日旁雲氣,人主象。〔正義〕洛書云:「有雲象人,青衣無手,在日西,天子之氣。」皆如其形以占。

故北夷之氣如羣畜穹閭,〔索隱〕鄒云一作「弓閭」。天文志作「弓」字,音穹。蓋謂以氈爲閭,崇穹然。又宋均云「穹,獸名」,亦異說也。南夷之氣類舟舩幡旗。大水處,敗軍場,破國之虛,下有積錢,〔集解〕徐廣曰:「古作『泉』字。」金寶之上,皆有氣,不可不察。海旁蜄氣象樓臺;廣野氣成宮闕然。雲氣各象其山川人民所聚積。〔正義〕淮南子云:「土地各以類生人,是故山氣多勇,澤氣多瘖,風氣多聾,林氣多躄,木氣多傴,石氣多力,險阻氣多壽,谷氣多痺,丘氣多狂,廟氣多仁,陵氣多貪,輕土多利足,重土多遲,清水音小,濁水音大,湍水人重,中土多聖人。皆象其氣,皆應其類也。」

故候息秏者,入國邑,視封疆田疇之正治,〔集解〕如淳曰:「蔡邕云麻田曰疇。」城郭室屋門戶之潤澤,次至車服畜產精華。實息者,吉;虛秏者,凶。

若煙非煙,若雲非雲,郁郁紛紛,蕭索輪囷,是謂卿雲。〔正義〕卿音慶。卿雲見,喜氣也。若霧〔索隱〕音如字,一音蒙,一音亡遘反。爾雅云「天氣下地不應曰霧」,言蒙昧不明之意也。非霧,衣冠而不濡,見則其域被甲而趨。

天雷電、蝦虹、辟歷、夜明者,陽氣之動者也,春夏則發,秋冬則藏,故候者無不司之。

天開縣物,〔集解〕孟康曰:「謂天裂而見物象,天開示縣象。」地動坼絕。〔正義〕趙世家幽繆王遷五年,「代地動,自樂徐以西,北至平陰,臺屋牆垣太半壞,地坼東西百三十步」。山崩及徙,川塞谿垘;〔集解〕徐廣曰:「土雍曰垘,音服。」駰案:孟康曰「谿,谷也。垘,崩也」。蘇林曰「伏,流也」。水澹澤竭,地長見象。城郭門閭,閨臬枯稾;宮廟邸第,人民所次。謠俗車服,觀民飲食。五穀草木,觀其所屬。倉府廄庫,四通之路。六畜禽獸,所產去就;魚龞鳥鼠,觀其所處。鬼哭若呼,其人逢俉。化言〔集解〕俉,迎也。伯莊曰:「音五故反。」〔索隱〕俉音五故反。逢俉謂相逢而驚也。亦作「迕」,音同。「化」當爲「訛」,字之誤耳。誠然。

凡候歲美惡,謹候歲始。歲始或冬至日,產氣始萌。臘明日,人衆卒歲,一會飲食,發陽氣,故曰初歲。正月旦,王者歲首;立春日,四時之卒始也。〔索隱〕謂立春日是去年四時之終卒,今年之始也。四始者,候之日。〔正義〕謂正月旦歲之始,時之始,日之始,月之始,故云「四始」。言以四時之日候歲吉凶也。

而漢魏鮮〔集解〕孟康曰:「人姓名,作占候者。」集臈明正月旦決八風。風從南方來,大旱;西南,小旱;西方,有兵;西北,戎菽爲,〔集解〕孟康曰:「戎菽,胡豆也。爲,成也。」〔索隱〕戎叔爲。韋昭云「戎叔,大豆也。爲,成也」。又郭璞注爾雅亦云「戎叔,胡豆」。孟康同也。小雨,〔集解〕徐廣曰:「一無此上兩字。」趣兵;〔索隱〕趣音促。謂風從西北來,則戎叔成。而又有小雨,則國兵趣起也。北方,爲中歲;東北,爲上歲;〔集解〕韋昭曰:「歲大穰。」東方,大水;東南,民有疾疫,歲惡。故八風各與其衝對,課多者爲勝。多勝少,久勝亟,疾勝徐。旦至食,爲麥;食至日昳,爲稷;昳至餔,爲黍;餔至下餔,爲菽;下餔至日入,爲麻。欲終日,有雨,有雲,有風,有日。〔正義〕正月旦,欲其終一日有風有日,則一歲之中五穀豐熟,無災害也。日當其時者,深而多實;無雲有風,日當其時,淺而多實;有雲風無,日當其時,深而少實;有日,無雲,不風,當其時者稼有敗。如食頃,小敗;熟五斗米頃,大敗。則風復起,有雲,其稼復起。各以其時用雲色占種其所宜。其雨雪若寒,歲惡。

是日光明,聽都邑人民之聲。聲宮,則歲善,吉;商,則有兵;徵,旱;羽,水;角,歲惡。

或從正月旦比數雨。〔索隱〕比音鼻律反。數音疏矩反。謂以次數日以候一歲之雨,以知豐穰也。率日食一升,至七升而極;〔集解〕孟康曰:「正月一日雨,民有一升之食;二日雨,民有二升之食;如此至七日。」過之,不占。數至十二日,日直其月,占水旱。〔集解〕孟康曰:「月一日雨,正月水。」爲其環城千里,內占,則其爲天下候,竟正月。〔集解〕孟康曰:「月三十日周天,歷二十八宿,然後可占天下。」〔正義〕案:月列宿,日、風、雲有變,占其國,并太歲所在,則知其歲豐稔、水旱、饑饉也。月所離列宿,〔索隱〕月離于畢。案:韋昭云「離,歷也」。日、風、雲,占其國。然必察太歲所在。在金,穰;水,毀;木,饑;火,旱。此其大經也。

正月上甲,風從東方,宜蠶;風從西方,若旦黃雲,惡。

冬至短極,縣土炭,〔集解〕孟康曰:「先冬至三日,縣土炭於衡兩端,輕重適均,冬至日陽氣至則炭重,夏至日陰氣至則土重。」晉灼曰:「蔡邕律曆記『候鍾律權土炭,冬至陽氣應黃鍾通,土炭輕而衡仰,夏至陰氣應蕤賔通,土炭重而衡低。進退先後,五日之中』。」炭動,鹿解角,蘭根出,泉水躍,略以知日至,決要晷景。歲星所在,五穀逢昌。其對爲衝,歲乃有殃。〔正義〕言晷景歲星行不失次,則無災異,五穀逢其昌盛;若晷景歲星行而失舍有所衝,則歲乃有殃禍災變也。

太史公曰:自初生民以來,世主曷嘗不曆日月星辰?及至五家、〔索隱〕案:謂五紀,歲、月、日、星辰、曆數,各有一家顓學習之,故曰「五家」也。三代,紹而明之,〔正義〕五家,黃帝、高陽、高辛、唐虞、堯舜也。三代,夏、殷、周也。言生民以來,何曾不曆日、月、星辰,及至五帝、三王,亦於紹繼而明天數陰陽也。內冠帶,外夷狄,分中國爲十有二州,仰則觀象於天,俯則法類於地。天則有日月,地則有陰陽。天有五星,地有五行。天則有列宿,地則有州域。三光者,陰陽之精,氣本在地,而聖人統理之。

幽厲以往,尚矣。所見天變,皆國殊窟穴,家占物怪,以合時應,其文圖籍禨祥不法。〔正義〕禨音機。顧野王云「禨祥,吉凶之先見也」。案:自古以來所見天變,國皆異具,所說不同,及家占物怪,用合時應者書,其文并圖籍,凶吉並不可法則。故孔子論六經,記異事而說其所應,不書變見之蹤也。是以孔子論六經,紀異而說不書。至天道命,不傳;傳其人,不待告;〔正義〕待,須也。言天道性命,忽有志事,可傳授之則傳,其大指微妙,自在天性,不須深告語也。告非其人,雖言不著。〔正義〕著,作慮反。著,明也。言天道性命,告非其人,雖爲言說,不得著明微妙,曉其意也。

昔之傳天數者:高辛之前,重、黎;〔正義〕左傳云蔡墨曰「少昊氏之子曰黎,爲火正,號祝融」,即火行之官,知天數。於唐、虞,羲、和;〔正義〕羲氏,和氏,掌天地四時之官也。有夏,昆吾;〔正義〕昆吾,陸終之子。虞翻云「昆吾名樊,爲己姓,封昆吾」。世本云昆吾衞者也。殷商,巫咸;〔正義〕巫咸,殷賢臣也,本吳人,冢在蘇州常熟海隅山上。子賢,亦在此也。周室,史佚、萇弘;〔正義〕史佚,周武王時太史尹佚也。萇弘,周靈王時大夫也。於宋,子韋;鄭則裨竈;〔正義〕裨竈,鄭大夫也。在齊,甘公;〔集解〕徐廣曰:「或曰甘公名德也,本是魯人。」〔正義〕七錄云楚人,戰國時作天文星占八卷。楚,唐眛;〔正義〕莫葛反。趙,尹皐;魏,石申。〔正義〕七錄云石申,魏人,戰國時作天文八卷也。

夫天運,三十歲一小變,百年中變,五百載大變;三大變一紀,三紀而大備:此其大數也。爲國者必貴三五。〔索隱〕三五謂三十歲一小變,五百歲一大變。上下各千歲,然后天人之際續備。

太史公推古天變,未有可考于今者。蓋略以春秋二百四十二年之閒,〔正義〕謂從隱公元年至哀公十四年獲麟也。隱公十一年,桓公十八年,莊公三十二年,閔公二年,僖公三十三年,文公十八年,宣公十八年,成公十八年,襄公三十一年,昭公三十二年,定公十五年,哀公十四年:凡二百四十二年也。日蝕三十六,〔正義〕謂隱公三年二月乙巳;桓公三年七月壬辰朔,十七年十月朔;莊公十八年三月朔,二十五年六月辛未朔,二十六年十二月癸亥朔,三十年九月庚午朔;僖公五年九月戊申朔,十二年三月庚午朔,十五年五月朔;文公元年二月癸亥朔,十五年六月辛卯朔;宣公八年七月庚子朔,十年四月丙辰朔,十七年六月癸卯朔;成公十六年六月丙辰朔,十七年七月丁巳朔;襄公十四年二月乙未朔,十五年八月丁巳朔,二十年十月丙辰朔,二十一年九月庚戌朔,十月庚辰朔,二十三年二月癸酉朔,二十四年七月甲子朔,八月癸巳朔,二十七年十二月乙亥朔;昭公七年四月甲辰朔,十五年六月丁巳朔,十七年六月甲戌朔,二十一年七月壬午朔,二十二年十二月癸酉朔,二十四年五月乙未朔,三十年十二月辛亥朔;定公五年三月辛亥朔,十二年十一月丙寅朔,十五年八月庚辰朔:凡蝕三十六也。彗星三見,〔正義〕謂文公十四年七月有星入于北斗,昭公十七年冬有星孛于大辰,哀公十三年有星孛于東方。宋襄公時星隕如雨。〔正義〕謂僖公十六年正月戊申朔,隕石于宋五也。天子微,諸侯力政,〔集解〕徐廣曰:「一作『征』。」五伯代興,〔正義〕趙岐注孟子云齊桓、晉文、秦穆、宋襄、楚莊也。更爲主命,自是之後,衆暴寡,大并小。秦、楚、吳、越,夷狄也,爲彊伯。〔正義〕秦祖非子初邑於秦,地在西戎。楚子鬻熊始封丹陽,荊蠻。吳太伯居吳,周章因封吳,號句吳。越祖少康之子初封於越,以守禹祀,地稱東越。皆戎夷之地,故言夷狄也。後秦穆、楚莊、吳闔閭、越句踐皆得封爲伯也。田氏篡齊,〔正義〕周安王二十三年,齊康公卒,田和并齊而立爲齊侯。三家分晉,〔正義〕周安王二十六年,魏武侯、韓文侯、趙敬侯共滅晉靜而三分其地。並爲戰國。爭於攻取,兵革更起,城邑數屠,因以飢饉,疾疫焦苦,臣主共憂患,其察禨祥候星氣尤急。近世十二諸侯七國相王,〔正義〕王,于放反。謂漢孝景三年,吳王濞、楚王戊、趙王遂、濟南王辟光、淄川王賢、膠東王雄渠也。言從衡者繼踵,而皐、唐、甘、石因時務論其書傳,故其占驗淩雜米鹽。〔正義〕淩雜,交亂也。米鹽,細碎也。言臯、唐、甘、石等因時務論其書傳中災異所記錄者,故其占驗交亂細碎。其語在漢書五行志中也。

二十八舍主十二州,〔正義〕二十八舍,謂東方角、亢、氐、房、心、尾、箕;北方斗、牛、女、虛、危、室、壁;西方奎、婁、胃、昴、畢、觜、參;南方井、鬼、柳、星、張、翼、軫。星經云:「角、亢,鄭之分野,兗州;氐、房、心,宋之分野,豫州;尾、箕,燕之分野,幽州;南斗、牽牛,吳、越之分野,揚州;須女、虛,齊之分野,青州;危、室、壁,衞之分野,并州;奎、婁,魯之分野,徐州;胃、昴,趙之分野,冀州;畢、觜、參,魏之分野,益州;東井、輿鬼,秦之分野,雍州;柳、星、張,周之分野,三河;翼、軫,楚之分野,荊州也。」斗秉兼之,所從來久矣。〔正義〕言北斗所建秉十二辰,兼十二州,二十八宿,自古所用,從來久逺矣。秦之疆也,候在太白,占於狼、弧。〔正義〕太白、狼、弧,皆西方之星,故秦占候也。吳、楚之疆,候在熒惑,占於鳥衡。〔正義〕熒惑、鳥衡,皆南方之星,故吳、楚之占候也。鳥衡,栁星也。一本作「注張」也。燕、齊之疆,候在辰星,占於虛、危。〔正義〕辰星、虛、危,皆北方之星,故燕、齊占候也。宋、鄭之疆,候在歲星,占於房、心。〔正義〕歲星、房、心,皆東方之星,故宋、鄭占候也。晉之疆,亦候在辰星,占於參罰。〔正義〕辰星、參、罰,皆北方西方之星,故晉占候也。

及秦并吞三晉、燕、代,自河山以南者中國。〔正義〕河,黃河也。山,華山也。從華山及黃河以南爲中國也。中國於四海內則在東南,爲陽;〔正義〕爾雅云「九夷,八狄,七戎,六蠻,謂之四海之內」。中國,從河山東南爲陽也。陽則日、歲星、熒惑、填星;〔正義〕日,人質反。填音鎮。日,陽也。歲星屬東方,熒惑屬南方,填星屬中央,皆在南及東,爲陽也。占於街南,畢主之。〔正義〕天街二星,主畢、昴,主國界也。街南爲華夏之國,街北爲夷狄之國,則畢星主陽。其西北則狐、狢、月氏諸衣旃裘引弓之民,爲陰;〔正義〕貉音陌。氏音支。從河山西北及秦、晉爲陰也。陰則月、太白、辰星;〔正義〕月,陰也。太白屬西方,辰星屬北方,皆在北及西,爲陰也。占於街北,昴主之。〔正義〕天街星北爲夷狄之國,則昴星主之,陰也。故中國山川東北流,其維,首在隴、蜀,尾沒于勃、碣。〔正義〕言中國山及川東北流行,若南山首在崑崙蔥嶺,東北行,連隴山至南山、華山,渡河東北盡碣石山。黃河首起崑崙山;渭水、岷江發源出隴山:皆東北東入渤海也。是以秦、晉好用兵,〔集解〕韋昭曰:「秦晉西南維之北爲陰,猶與胡、貉引弓之民同,故好用兵。」復占太白,太白主中國;而胡、狢數侵掠,〔正義〕主猶領也,入也。星經云「太白在北,月在南,中國敗;太白在南,月在北,中國不敗也」。是胡貉數侵掠之也。獨占辰星,辰星出入躁疾,常主夷狄:其大經也。此更爲客主人。〔正義〕更,格行反,下同。星經云:「辰星不出,太白爲客;辰星出,太白爲主人。辰星、太白不相從,雖有軍不戰。辰星出東方,太白出西方,若辰星出西方,太白出東方,爲『格野』,雖有兵不戰;合宿乃戰。辰星入太白中五日,及入而上出,破軍殺將,客勝;不出,客亡地。視旗所指。」熒惑爲孛,外則理兵,內則理政。故曰「雖有明天子,必視熒惑所在」。〔索隱〕必視熒惑之所在。此據春秋緯文耀鉤,故言「故曰」。諸侯更彊,時菑異記,無可錄者。

秦始皇之時,十五年彗星四見,久者八十日,長或竟天。其後秦遂以兵滅六王,并中國,外攘四夷,死人如亂麻,因以張楚並起,三十年之閒〔正義〕謂從秦始皇十六年起兵滅韓,至漢高祖五年滅項羽,則三十六年矣。兵相駘藉,〔集解〕蘇林曰:「駘音臺,登躡也。」不可勝數。自蚩尤以來,未嘗若斯也。

項羽救鉅鹿,枉矢西流,山東遂合從諸侯,西坑秦人,誅屠咸陽。

漢之興,五星聚于東井。平城之圍,〔索隱〕漢高祖之七年。月暈參、畢七重。〔索隱〕案:天文志「其占者畢、昴閒天街也。街北,胡也。街南,中國也。昴爲匈奴;參爲趙;畢爲邊兵。是歲高祖自將兵擊匈奴,至平城,爲冒頓所圍,七日乃解」。則天象有若符契。七重,主七日也。諸呂作亂,日蝕,晝晦。吳楚七國叛逆,彗星數丈,天狗過梁野;及兵起,遂伏尸流血其下。元光、元狩,蚩尤之旗再見,長則半天。其後京師師四出,〔正義〕元光元年,太中大夫衞青等伐匈奴;元狩二年,冠軍侯霍去病等擊胡;元鼎五年,衞尉路博德等破南越;及韓說破東越,并破西南夷,開十餘郡;元年,樓船將軍楊僕擊朝鮮也。誅夷狄者數十年,而伐胡尤甚。越之亡,熒惑守斗;〔正義〕南斗爲吳、越之分野。朝鮮之拔,星茀〔索隱〕音佩,即孛星也。于河;〔索隱〕案:天文志「武帝元封之中,星孛于河戍,其占曰『南戍爲越門,北戍爲胡門』。其後漢兵擊拔朝鮮,以爲樂浪、玄菟郡。朝鮮在海中,越之象,居北方,胡之域也」。其河戍即南河、北河也。戒兵征大宛,星茀招搖:〔正義〕招搖一星,次北斗杓端,主胡兵也。占:角變,則兵革大行。此其犖犖〔索隱〕力角反。犖犖,大事分明也。大者。若至委曲小變,不可勝道。由是觀之,未有不先形見而應隨之者也。

夫自漢之爲天數者,星則唐都,氣則王朔,占歲則魏鮮。故甘、石曆五星法,唯獨熒惑有反逆行;逆行所守,及他星逆行,日月薄蝕,〔集解〕孟康曰:「日月無光曰薄。京房易傳曰『日赤黃爲薄』。或曰不交而蝕曰薄。」韋昭曰:「氣往迫之爲薄,虧毀爲蝕。」皆以爲占。

余觀史記,考行事,百年之中,五星無出而不反逆行,反逆行,嘗盛大而變色;日月薄蝕,行南北有時:此其大度也。故紫宮、〔正義〕中宮也。房心、〔正義〕東宮也。權衡、〔正義〕南宮也咸池、〔正義〕西宮也。虛危〔正義〕北宮也。列宿部星,〔正義〕五官列宿部內之星也此天之五官坐位也,爲經,不移徙,大小有差,闊狹有常。〔集解〕孟康曰:「闊狹,若三台星相去逺近。」水、火、金、木、填星,〔集解〕徐廣曰:「木、火、土三星若合,是謂驚位絕行。」此五星者,天之五佐,〔正義〕言水、火、金、木、土五星佐天行德也。爲經緯,見伏有時,〔正義〕五星行南北爲經,東西爲緯也。所過行贏縮有度。

日變脩德,月變省刑,星變結和。凡天變,過度乃占。國君彊大,有德者昌;羽小,飾詐者亡。太上脩德,其次脩政,其次脩救,次脩穰,正下無之。夫常星之變希見,而三光之占亟用。日月暈適,〔集解〕徐廣曰:「適者,災變咎徵也。」李斐曰:「適,見災于天。劉向以爲日、月蝕及星逆行,非太平之常。自周衰以來,人事多亂,故天文應之遂變耳。」駰案:孟康曰「暈,日旁氣也。適,日之將食,先有黑氣之變」。雲風,此天之客氣,其發見亦有大運。然其與政事俯仰,最近大人之符。此五者,天之感動。爲天數者,必通三五。〔索隱〕案三謂三辰,五謂五星。終始古今,深觀時變,察其精粗,則天官備矣。

蒼帝行德,天門爲之開。〔索隱〕案謂王者行春令,布德澤,被天下,應靈威仰之帝,而天門爲之開,以發德化也。天門,即左右角閒也。〔正義〕爲,于偽反,下同。蒼帝,東方靈威仰之帝也。春,萬物開發,東作起,則天發其德化,天門爲之開也。赤帝行德,天牢爲之空。〔索隱〕亦謂王者行德,以應火精之帝。謂舉大禮,封諸侯之地,則是赤帝行德。夏陽,主舒散,故天牢爲之空,則人主當赦宥也。 〔正義〕赤帝,南方赤熛怒之帝也。夏萬物茂盛,功作大興,則天施德惠,天牢爲之空虛也。天牢六星,在北斗魁下,不對中台,主秉禁暴,亦貴人之牢也。黃帝行德,天夭爲之起。〔正義〕黃帝,中央含樞紐之帝。季夏萬物盛大,則當大赦,含養羣品也。風從西北來,必以庚、辛。一秋中,五至,大赦;三至,小赦。白帝行德,以正月二十日、二十一日,月暈圍,常大赦載,謂有太陽也。一曰:〔索隱〕一曰,二曰,案謂星家之異說,太史公兼記之耳。白帝行德,畢、昴爲之圍。圍三暮,德乃成;〔正義〕白帝,西方白招矩之帝也。秋萬物咸成,則暈圍畢、昴三暮,帝德乃成也。不三暮,及圍不合,德不成。二曰:以辰圍,不出其旬。黑帝行德,天關爲之動。〔正義〕黑帝,北方協光紀之帝也。冬萬物閉藏,爲之動,爲之開閉也。天關一星,在五車南,畢西北,爲天門,日、月、五星所道,主邊事,亦爲限隔內外,障絕往來,禁道之作違者。占:芒,角,有兵起;五星守之,主貴人多死也。天行德,天子更立年;〔索隱〕案天,謂北極,紫微宮也。言王者當天心,則北辰有光耀,是行德也。北辰光耀,則天子更立年也。不德,風雨破石。三能、三衡者,天廷也。〔索隱〕上云「南宮朱鳥,權衡,衡,太微,三光之廷」,則三衡者即太微也。其謂之三者,爲日、月、五星也。然斗第六第五星亦名衡,又參三星亦名衡,然並不爲天廷也。 〔正義〕晉書天文志云:「三台,主開德宣符也,所以和陰陽而理萬物也。三衡者,北斗魁四星爲璇璣,杓三星爲玉衡,人君之象,號令主也。又太微,天子宮庭也。太微爲衡,衡主平也,爲天庭理,法平辭理也。」案:言三台、三衡者,皆天帝之庭,號令舒散平理也,故言三台、三衡。言若有客星出三台、三衡之廷,必有奇異教令也。客星出天廷,有奇令。

索隱述贊曰:在天成象,有同影響。觀文察變,其來自往。天官旣書,太史攸掌。雲物必記,星辰可仰。盈縮匪𠎱,應驗無爽。至哉玄監,云誰欲網!