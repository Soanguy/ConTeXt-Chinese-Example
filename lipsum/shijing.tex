《诗经》全集  

国风·周南


关雎

关关雎鸠,在河之洲。窈窕淑女,君子好逑。

参差荇菜,左右流之。窈窕淑女,寤寐求之。

求之不得,寤寐思服。悠哉悠哉,辗转反侧。

参差荇菜,左右采之。窈窕淑女,琴瑟友之。

参差荇菜,左右芼之。窈窕淑女,钟鼓乐之。

葛覃

葛之覃兮,施于中谷,维叶萋萋。

黄鸟于飞,集于灌木,其鸣喈喈。

葛之覃兮,施于中谷,维叶莫莫。

是刈是濩,为絺为绤,服之无斁。

言告师氏,言告言归。薄污我私,

薄浣我衣。害浣害否,归宁父母。

卷耳

采采卷耳,不盈顷筐。嗟我怀人,置彼周行。

陟彼崔嵬,我马虺隤。我姑酌彼金罍,维以不永怀。

陟彼高冈,我马玄黄。我姑酌彼兕觥,维以不永伤。

陟彼砠矣,我马瘏矣,我仆痡矣,云何吁矣。

樛木

南有樛木,葛藟累之。乐只君子,福履绥之。

南有樛木,葛藟荒之。乐只君子,福履将之。

南有樛木,葛藟萦之。乐只君子,福履成之。

螽斯

螽斯羽,诜诜兮。宜尔子孙,振振兮。

螽斯羽,薨薨兮。宜尔子孙。绳绳兮。

螽斯羽,揖揖兮。宜尔子孙,蛰蛰兮。

桃夭

桃之夭夭,灼灼其华。之子于归,宜其室家。

桃之夭夭,有蕡其实。之子于归,宜其家室。

桃之夭夭,其叶蓁蓁。之子于归,宜其家人。

兔罝

肃肃兔罝,椓之丁丁。赳赳武夫,公侯干城。

肃肃兔罝,施于中逵。赳赳武夫,公侯好仇。

肃肃免罝,施于中林。赳赳武夫,公侯腹心。

芣苡

采采芣苡,薄言采之。采采芣苡,薄言有之。

采采芣苡,薄言掇之。采采芣苡,薄言捋之。

采采芣苡,薄言袺之。采采芣苡,薄言擷之。

汉广

南有乔木,不可休息。汉有游女,不可求思。汉之广矣,不可泳思。江之永矣,不可方思。

翘翘错薪,言刈其楚。之子于归,言秣其马。汉之广矣,不可泳思。江之永矣,不可方思。

翘翘错薪,言刈其蒌。之子于归。言秣其驹。汉之广矣,不可泳思。江之永矣,不可方思。

汝坟

遵彼汝坟,伐其条枚。未见君子,惄如调饥。

遵彼汝坟,伐其条肄。既见君子,不我遐弃。

鲂鱼赪尾,王室如毁。虽则如毁,父母孔迩。

麟之趾

麟之趾,振振公子,于嗟麟兮。麟之定,振振公姓,于嗟麟兮。麟之角,振振公族,于嗟麟兮。




国风·召南


鹊巢

维鹊有巢,维鸠居之。之子于归,百两御之。

维鹊有巢,维鸠方之。之子于归,百两将之。

维鹊有巢,维鸠盈之。之子于归,百两成之。

采蘩

于以采蘩?于沼于沚。于以用之?公侯之事。

于以采蘩?于涧之中。于以用之?公侯之宫。

被之僮僮,夙夜在公。被之祁祁,薄言还归。

草虫

喓喓草虫,趯趯阜螽。未见君子,忧心忡忡。亦既见止,亦既觏止,我心则降。

陟彼南山,言采其蕨。未见君子,忧心惙惙。亦既见止,亦既觏止,我心则说。

陟彼南山,言采其薇。未见君子,我心伤悲。亦既见止,亦既觏止,我心则夷。

采蘋

于以采蘋?南涧之滨。于以采藻?于彼行潦。

于以盛之?维筐及筥。于以湘之?维锜及釜。

于以奠之?宗室牖下。谁其尸之?有齐季女。

甘棠

蔽芾甘棠,勿剪勿伐,召伯所茏。

蔽芾甘棠,勿剪勿败,召伯所憩。

蔽芾甘棠,勿剪勿拜,召伯所说。

行露

厌浥行露,岂不夙夜,谓行多露。

谁谓雀无角?何以穿我屋?谁谓女无家?何以速我狱?虽速我狱,室家不足!

谁谓鼠无牙?何以穿我墉?谁谓女无家?何以速我讼?虽速我讼,亦不女从!

羔羊

羔羊之皮,素丝五紽。退食自公,委蛇委蛇。

羔羊之革,素丝五緎。委蛇委蛇,自公退食。

羔羊之缝,素丝五总。委蛇委蛇,退食自公。

殷其雷

殷其雷,在南山之阳。何斯违斯,莫敢或遑?振振君子,归哉归哉!

殷其雷,在南山之侧。何斯违斯,莫敢遑息?振振君子,归哉归哉!

殷其雷,在南山之下。何斯违斯,莫或遑处?振振君子,归哉归哉!

摽有梅

摽有梅,其实七兮。求我庶士,迨其吉兮。

摽有梅,其实三兮。求我庶士,迨其今兮。

摽有梅,顷筐塈之。求我庶士,迨其谓之。

小星

嘒彼小星,三五在东。肃肃宵征,夙夜在公。实命不同!

嘒彼小星,维参与昴。肃肃宵征,抱衾与裯。实命不犹!

江有汜

江有汜,之子归,不我以。不我以,其后也悔。

江有渚,之子归,不我与。不我与,其后也处。

江有沱,之子归,不我过。不我过,其啸也歌。

野有死麕

野有死麕,白茅包之。有女怀春,吉士诱之。

林有朴。,野有死鹿。白茅纯束,有女如玉。

舒而脱脱兮,无感我帨兮,无使尨也吠。

何彼襛矣

何彼襛矣,唐棣之华?曷不肃雍?王姬之车。

何彼襛矣,华如桃李?平王之孙,齐侯之子。

其钓维何?维丝伊缗。齐侯之子,平王之孙。

驺虞

彼茁者葭,壹发五豝,于嗟乎驺虞!

彼茁者蓬,壹发五豵,于嗟乎驺虞!




国风·邶风


柏舟

泛彼柏舟,亦泛其流。耿耿不寐,如有隐忧。微我无酒,以敖以游。

我心匪鉴,不可以茹。亦有兄弟,不可以据。薄言往诉,逢彼之怒。

我心匪石,不可转也。我心匪席,不可卷也。威仪棣棣,不可选也。

忧心悄悄,愠于群小。觏闵既多,受侮不少。静言思之,寤辟有摽。

日居月诸,胡迭而微?心之忧矣,如匪浣衣。静言思之,不能奋飞。

绿衣

绿兮衣兮,绿衣黄裹。心之忧矣,曷维其已!

绿兮衣兮,绿衣黄裳。心之忧矣,曷维其亡!

绿兮丝兮,女所治兮。我思古人,俾无訧兮!

絺兮绤兮,凄其以风。我思古人,实获我心!

燕燕

燕燕于飞,差池其羽。之子于归,远送于野。瞻望弗及,泣涕如雨。

燕燕于飞,颉之颃之。之子于归,远于将之。瞻望弗及,伫立以泣。

燕燕于飞,下上其音。之子于归,远送于南。瞻望弗及,实劳我心。

仲氏任只,其心塞渊。终温且惠,淑慎其身。先君之思,以勖寡人。

日月

日居月诸,照临下土。乃如之人兮,逝不古处?胡能有定?宁不我顾。

日居月诸,下土是冒。乃如之人兮,逝不相好。胡能有定?宁不我报。

日居月诸,出自东方。乃如之人兮,德音无良。胡能有定?俾也可忘。

日居月诸,东方自出。父兮母兮,畜我不卒。胡能有定?报我不述。

终风

终风且暴,顾我则笑,谑浪笑敖,中心是悼。

终风且霾,惠然肯来,莫往莫来,悠悠我思。

终风且曀,不日有曀,寤言不寐,愿言则嚏。

曀曀其阴,虺虺其雷,寤言不寐,愿言则怀。

击鼓

击鼓其镗,踊跃用兵。土国城漕,我独南行。

从孙子仲,平陈与宋。不我以归,忧心有忡。

爰居爰处?爰丧其马?于以求之?于林之下。

死生契阔,与子成说。执子之手,与子偕老。

于嗟阔兮,不我活兮。于嗟洵兮,不我信兮。

凯风

凯风自南,吹彼棘心。棘心夭夭,母氏劬劳。

凯风自南,吹彼棘薪。母氏圣善,我无令人。

爰有寒痊在浚之下。有子七人,母氏劳苦。

睍黄鸟,载好其音。?有子七人,莫慰母心。

雄雉

雄雉于飞,泄泄其羽。我之怀矣,自诒伊阻。

雄雉于飞,下上其音。展矣君子,实劳我心。

瞻彼日月,悠悠我思。道之云远,曷云能来?

百尔君子,不知德行。不忮不求,何用不臧。

匏有苦叶

匏有苦叶,济有深涉。深则厉,浅则揭。

有瀰济盈,有鷕雉鸣。济盈不濡轨,雉鸣求其牡。

雍雍鸣雁,旭日始旦。士如归妻,迨冰未泮。

招招舟子,人涉卬否。不涉卬否,卬须我友。

谷风

习习谷风,以阴以雨。黾勉同心,不宜有怒。采葑采菲,无以下体?德音莫违,及尔同死。

行道迟迟,中心有违。不远伊迩,薄送我畿。谁谓荼苦,其甘如荠。宴尔新婚,如兄如弟。

泾以渭浊,湜湜其沚。宴尔新婚,不我屑以。毋逝我梁,毋发我笱。我躬不阅,遑恤我后。

就其深矣,方之舟之。就其浅矣,泳之游之。何有何亡,黾勉求之。凡民有丧,匍匐救之。

不我能畜,反以我为仇。既阻我德,贾用不售。昔育恐育鞫,及尔颠覆。既生既育,比予于毒。

我有旨蓄,亦以御冬。宴尔新婚,以我御穷。有洸有溃,既诒我肄。不念昔者,伊余来塈。

式微

式微,式微,胡不归?微君之故,胡为乎中露!

式微,式微,胡不归?微君之躬,胡为乎泥中!

旄丘

旄丘之葛兮,何诞之节兮。叔兮伯兮,何多日也?

何其处也?必有与也!何其久也?必有以也!

狐裘蒙戎,匪车不东。叔兮伯兮,靡所与同。

琐兮尾兮,流离之子。叔兮伯兮,褎如充耳。

简兮

简兮简兮,方将万舞。日之方中,在前上处。

硕人俣俣,公庭万舞。有力如虎,执辔如组。

左手执龠,右手秉翟。赫如渥赭,公言锡爵。

山有榛,隰有苓。云谁之思?西方美人。彼美人兮,西方之人兮。

泉水

毖彼泉水,亦流于淇。有怀于卫,靡日不思。娈彼诸姬,聊与之谋。

出宿于 ,饮饯于祢。女子有行,远父母兄弟,问我诸姑,遂及伯姊。

出宿于干,饮饯于言。载脂载辖,还车言迈。遄臻于卫,不瑕有害?

我思肥泉,兹之永叹。思须与漕,我心悠悠。驾言出游,以写我忧。

北门

出自北门,忧心殷殷。终窭且贫,莫知我艰。已焉哉!天实为之,谓之何哉!

王事适我,政事一埤益我。我入自外,室人交遍谪我。已焉哉!天实为之,谓之何哉!

王事敦我,政事一埤遗我。我入自外,室人交遍摧我。已焉哉!天实为之,谓之何哉!

北风

北风其凉,雨雪其雱。惠而好我,携手同行。其虚其邪?既亟只且!

北风其喈,雨雪其霏。惠而好我,携手同归。其虚其邪?既亟只且!

莫赤匪狐,莫黑匪乌。惠而好我,携手同车。其虚其邪?既亟只且!

静女

静女其姝,俟我于城隅。爱而不见,搔首踟蹰。

静女其娈,贻我彤管。彤管有炜,说怿女美。

自牧归荑,洵美且异。匪女之为美,美人之贻。

新台

新台有氵此,河水瀰瀰。燕婉之求,蘧篨不鲜。

新台有洒,河水浼浼。燕婉之求,蘧篨不殄。

鱼网之设,鸿则离之。燕婉之求,得此戚施。

二子乘舟

二子乘舟,泛泛其景。愿言思子,中心养养!

二子乘舟,泛泛其逝。愿言思子,不瑕有害?




国风·鄘风


柏舟

泛彼柏舟,在彼中河。髧彼两髦,实维我仪。之死矢靡它。母也天只,不谅人只!

泛彼柏舟,在彼河侧。髧彼两髦,实维我特。之死矢靡慝。母也天只,不谅人只!

墙有茨

墙有茨,不可扫也。中冓之言,不可道也。所可道也,言之丑也。

墙有茨,不可襄也。中冓之言,不可详也。所可详也,言之长也。

墙有茨,不可束也。中冓之言,不可读也。所可读也,言之辱也。

君子偕老

君子偕老,副笄六珈。委委佗佗,如山如河。象服是宜。子之不淑,云如之何?

玼兮玼兮,其之翟也。鬒发如云,不屑髢也。玉之瑱也,象之揥也。扬且之晳也。胡然而天也!胡然而帝也!

瑳兮瑳兮,其之展也,蒙彼绉絺,是绁袢也。子之清扬,扬且之颜也,展如之人兮,邦之媛也!

桑中

爰采唐矣?沬之乡矣。云谁之思?美孟姜矣。期我乎桑中,要我乎上宫,送我乎淇之上矣。

爰采麦矣?沬之北矣。云谁之思?美孟弋矣。期我乎桑中,要我乎上宫,送我乎淇之上矣。

爰采葑矣?沬之东矣。云谁之思?美孟庸矣。期我乎桑中,要我乎上宫,送我乎淇之上矣。

鹑之奔奔

鹑之奔奔,鹊之彊彊。人之无良,我以为兄!

鹊之彊彊,鹑之奔奔。人之无良,我以为君!

定之方中

定之方中,作于楚宫。揆之以日,作于?/td> 
 
 

升彼虚矣,以望楚矣。望楚与堂,景山与京。降观于桑,卜云其吉,终然允臧。

灵雨既零,命彼倌人,星言夙驾,说于桑田。匪直也人,秉心塞渊,騋牝三千。

蝃蝀

蝃蝀在东,莫之敢指。女子有行,远父母兄弟。

朝隮于西,崇朝其雨。女子有行,远兄弟父母。

乃如之人也,怀婚姻也。大无信也,不知命也!

相鼠

相鼠有皮,人而无仪!人而无仪,不死何为?

相鼠有齿,人而无止!人而无止,不死何俟?

相鼠有体,人而无礼,人而无礼!胡不遄死?

干旄

孑孑干旄,在浚之郊。素丝纰之,良马四之。彼姝者子,何以畀之?

孑孑干旟,在浚之都。素丝组之,良马五之。彼姝者子,何以予之?

孑孑干旌,在浚之城。素丝祝之,良马六之。彼姝者子,何以告之?

载驰

载驰载驱,归唁卫侯。驱马悠悠,言至于漕。大夫跋涉,我心则忧。

既不我嘉,不能旋反。视尔不臧,我思不远。既不我嘉,不能旋济?视尔不臧,我思不閟。

陟彼阿丘,言采其蝱。女子善怀,亦各有行。许人尤之,众稚且狂。

我行其野,芃芃其麦。控于大邦,谁因谁极?大夫君子,无我有尤。百尔所思,不如我所之。




国风·卫风


淇奥

瞻彼淇奥,绿竹猗猗。有匪君子,如切如磋,如琢如磨,瑟兮僴兮,赫兮咺兮。有匪君子,终不可谖兮。

瞻彼淇奥,绿竹青青。有匪君子,充耳 秀莹,会弁如星。瑟兮僴兮。赫兮咺兮,有匪君子,终不可谖兮。

瞻彼淇奥,绿竹如箦。有匪君子,如金如锡,如圭如璧。宽兮绰兮,猗重较兮。善戏谑兮,不为虐兮。

考槃

考槃在涧,硕人之宽。独寐寤言,永矢弗谖。

考槃在阿,硕人之薖。独寐寤歌,永矢弗过。

考槃在陆,硕人之轴。独寐寤宿,永矢弗告。

硕人

硕人其颀,衣锦褧衣。齐侯之子,卫侯之妻。东宫之妹,邢侯之姨,谭公维私。

手如柔荑,肤如凝脂,领如蝤蛴,齿如瓠犀,螓首蛾眉,巧笑倩兮,美目盼兮。

硕人敖敖,说于农郊。四牡有骄,朱幩镳镳。翟茀以朝。大夫夙退,无使君劳。

河水洋洋,北流活活。施罛濊濊,鳣鲔发发。葭菼揭揭,庶姜孽孽,庶士有朅。

氓

氓之蚩蚩,抱布贸丝。匪来贸丝,来即我谋。送子涉淇,至于顿丘。匪我愆期,子无良媒。将子无怒,秋以为期。

乘彼垝垣,以望复关。不见复关,泣涕涟涟。既见复关,载笑载言。尔卜尔筮,体无咎言。以尔车来,以我贿迁。

桑之未落,其叶沃若。于嗟鸠兮!无食桑葚。于嗟女兮!无与士耽。士之耽兮,犹可说也。女之耽兮,不可说也。

桑之落矣,其黄而陨。自我徂尔,三岁食贫。淇水汤汤,渐车帷裳。女也不爽,士贰其行。士也罔极,二三其德。

三岁为妇,靡室劳矣。夙兴夜寐,靡有朝矣 。言既遂矣,至于暴矣。兄弟不知,咥其笑矣。静言思之,躬自悼矣。

及尔偕老,老使我怨。淇则有岸,隰则有泮。总角之宴,言笑晏晏,信誓旦旦,不思其反。反是不思,亦已焉哉!

竹竿

籊籊竹竿,以钓于淇。岂不尔思?远莫致之。

泉源在左,淇水在右。女子有行,远兄弟父母。

淇水在右,泉源在左。巧笑之瑳,佩玉之傩。

淇水滺滺,桧楫松舟。驾言出游,以写我忧。

芄兰

芄兰之支 ,童子佩觿。虽则佩觿,能不我知。容兮遂兮,垂带悸兮。

芄兰之叶,童子佩韘。虽则佩韘,能不我甲。容兮遂兮,垂带悸兮。

河广

谁谓河广?一苇杭之。谁谓宋远?跂予望之。

谁谓河广?曾不容刀。谁谓宋远?曾不崇朝。

伯兮

伯兮朅兮,邦之桀兮。伯也执殳,为王前驱。

自伯之东,首如飞蓬。岂无膏沐?谁适为容!

其雨其雨,杲杲出日。愿言思伯,甘心首疾。

焉得谖草?言树之背。愿言思伯。使我心痗。

有狐

有狐绥绥,在彼淇梁。心之忧矣,之子无裳。

有狐绥绥,在彼淇厉。心之忧矣,之子无带。

有狐绥绥,在彼淇侧。心之忧矣,之子无服。

木瓜

投我以木瓜,报之以琼琚。匪报也,永以为好也!

投我以木桃,报之以琼瑶。匪报也,永以为好也!

投我以木李,报之以琼玖。匪报也,永以为好也!




国风·王风


黍离

彼黍离离,彼稷之苗。行迈靡靡,中心摇摇。知我者,谓我心忧;不知我者,谓我何求。悠悠苍天,此何人哉?

彼黍离离,彼稷之穗。行迈靡靡,中心如醉。知我者,谓我心忧;不知我者,谓我何求。悠悠苍天,此何人哉?

彼黍离离,彼稷之实。行迈靡靡,中心如噎。知我者,谓我心忧;不知我者,谓我何求。悠悠苍天,此何人哉?

君子于役

君子于役,不知其期。曷至哉?鸡栖于埘。日之夕矣,羊牛下来。君子于役,如之何勿思!

君子于役,不日不月。曷其有佸?鸡栖于桀。日之夕矣,羊牛下括。君子于役,苟无饥渴?

君子阳阳

君子阳阳,左执簧,右招我由房,其乐只且!

君子陶陶,左执翿,右招我由敖,其乐只且!

扬之水

扬之水,不流束薪。彼其之子,不与我戍申。怀哉怀哉,曷月予还归哉!

扬之水,不流束楚。彼其之子,不与我戍甫。怀哉怀哉,曷月予还归哉!

扬之水,不流束蒲。彼其之子,不与我戍许。怀哉怀哉,曷月予还归哉!

中谷有蓷

中谷有蓷,其干矣。有女仳离,慨其 矣。慨其 矣,遇人之艰难矣。

中谷有蓷,其修矣。有女仳离,条其歗矣。条其歗矣,遇人之不淑矣。

中谷有蓷,其湿矣。有女仳离,啜其泣矣。啜其泣矣,何嗟及矣。

兔爰

有兔爰爰,雉离于罗。我生之初,尚无位我生之后,逢此百罹。尚寐无吪。

有兔爰爰,雉离于罦。我生之初,尚无造;我生之后,逢此百忧。尚寐无觉。

有兔爰爰,雉离于罿。我生之初,尚无庸;我生之后,逢此百凶。尚寐无聪。

葛藟

绵绵葛藟,在河之浒。终远兄弟,谓他人父。谓他人父,亦莫我顾!

绵绵葛爰,在河之藟。终远兄弟,谓他人母。谓他人母,亦莫我有!

绵绵葛藟,在河之漘。终远兄弟,谓他人昆。谓他人昆,亦莫我闻!

采葛

彼采葛兮,一日不见,如三月兮!

彼采萧兮,一日不见,如三秋兮!

彼采艾兮!一日不见,如三岁兮!

大车

大车槛槛,毳衣如菼。岂不尔思?畏子不敢。

大车啍啍,毳衣如璊,岂不尔思?畏子不奔。

谷则异室,死则同穴。谓予不信,有如皎日。

丘中有麻

丘中有麻,彼留子嗟。彼留子嗟,将其来施施。

丘中有麦,彼留子国。彼留子国,将其来食。

丘中有李,彼留之子。彼留之子,贻我佩玖。




国风·郑风


缁衣

缁衣之宜兮,敝予又改为兮。适子之馆兮。还予授子之粲兮。

缁衣之好兮,敝予又改造兮。适子之馆兮,还予授子之粲兮。

缁衣之席兮,敝予又改作兮。适子之馆兮,还予授子之粲兮。

将仲子

将仲子兮,无逾我里,无折我树杞。岂敢爱之?畏我父母。仲可怀也,父母之言亦可畏也。

将仲子兮,无逾我墙,无折我树桑。岂敢爱之?畏我诸兄。仲可怀也,诸兄之言亦可畏也。

将仲子兮,无逾我园,无折我树檀。岂敢爱之?畏人之多言。仲可怀也,人之多言亦可畏也。

叔于田

叔于田,巷无居人。岂无居人?不如叔也。洵美且仁。

叔于狩,巷无饮酒。岂无饮酒?不如叔也。洵美且好。

叔适野,巷无服马。岂无服马?不如叔也。洵美且武。

大叔于田

大叔于田,乘乘马。执辔如组,两骖如舞。叔在薮,火烈具举。袒裼暴虎,献于公所。将叔勿狃,戒其伤女。

叔于田,乘乘黄。两服上襄,两骖雁行。叔在薮,火烈具扬。叔善射忌,又良御忌。抑罄控忌,抑纵送忌。

叔于田,乘乘鸨。两服齐首,两骖如手。叔在薮,火烈具阜。叔马慢忌,叔发罕忌,抑释掤忌,抑鬯弓忌。

清人

清人在彭,驷介旁旁。二矛重英,河上乎翱翔。

清人在消,驷介麃麃。二矛重乔,河上乎逍遥。

清人在轴,驷介陶陶。左旋右抽,中军作好。

羔裘

羔裘如濡,洵直且侯。彼其之子,舍命不渝。

羔裘豹饰,孔武有力。彼其之子,邦之司直。

羔裘晏兮,三英粲兮。彼其之子,邦之彦兮。

遵大路

遵大路兮,掺执子之衤去。无我恶兮,不蓵故也!

遵大路兮,掺执子之手兮。无我魗兮,不蓵好也!

女曰鸡鸣

女曰鸡鸣,士曰昧旦。子兴视夜,明星有烂。将翱将翔,弋凫与雁。

弋言加之,与子宜之。宜言饮酒,与子偕老。琴瑟在御,莫不静好。

知子之来之,杂佩以赠之。知子之顺之,杂佩以问之。知子之好之,杂佩以报之。

有女同车

有女同车,颜如舜华。将翱将翔,佩玉琼琚。彼美孟姜,洵美且都。

有女同行,颜如舜英。将翱将翔,佩玉将将。彼美孟姜,德音不忘。

山有扶苏

山有扶苏,隰有荷华。不见子都,乃见狂且。

山有乔松,隰有游龙,不见子充,乃见狡童。

萚兮

萚兮萚兮,风其吹女。叔兮伯兮,倡予和女。

萚兮萚兮,风其漂女。叔兮伯兮,倡予要女。

狡童

彼狡童兮,不与我言兮。维子之故,使我不能餐兮。

彼狡童兮,不与我食兮。维子之故,使我不能息兮。

褰裳

子惠思我,褰裳涉溱。子不我思,岂无他人?狂童之狂也且!

子惠思我,褰裳涉洧。子不我思,岂无他士?狂童之狂也且!

丰

子之丰兮,俟我乎巷兮,悔予不送兮。

子之昌兮,俟我乎堂兮,悔予不将兮。

衣锦褧衣,裳锦褧裳。叔兮伯兮,驾予与行。

裳锦褧裳,衣锦褧衣。叔兮伯兮,驾予与归。

东门之墠

东门之墠,茹藘在阪。其室则迩,其人甚远。

东门之栗,有践家室。岂不尔思?子不我即!

风雨

风雨凄凄,鸡鸣喈喈,既见君子。云胡不夷?

风雨潇潇,鸡鸣胶胶。既见君子,云胡不瘳?

风雨如晦,鸡鸣不已。既见君子,云胡不喜?

子衿

青青子衿,悠悠我心。纵我不往,子宁不嗣音?

青青子佩,悠悠我思。纵我不往,子宁不来?

挑兮达兮,在城阙兮。一日不见,如三月兮。

扬之水

扬之水,不流束楚。终鲜兄弟,维予与女。无信人之言,人实诳女。

扬之水,不流束薪。终鲜兄弟,维予二人。无信人之言,人实不信。

出其东门

出其东门,有女如云。虽则如云。匪我思存。缟衣綦巾,聊乐我员。

出其闉阇,有女如荼。虽则如荼,匪我思且。缟衣茹藘,聊可与娱。

野有蔓草

野有蔓草,零露漙兮。有美一人,清扬婉兮。邂逅相遇,适我愿兮。

野有蔓草,零露瀼瀼。有美一人,婉如清扬。邂逅相遇,与子偕臧。

溱洧

溱与洧,方涣涣兮。士与女,方秉蕳兮。女曰观乎?士曰既且。且往观乎?洧之外,洵訏且乐。维士与女,伊其相谑,赠之以勺药。

溱与洧,浏其清矣。士与女,殷其盈矣。女曰观乎?士曰既且。且往观乎?洧之外,洵訏且乐。维士与女,伊其将谑,赠之以勺药。




国风·齐风


鸡鸣

鸡既鸣矣,朝既盈矣。匪鸡则鸣,苍蝇之声。

东方明矣,朝既昌矣。匪东方则明,月出之光。

虫飞薨薨,甘与子同梦。会且归矣,无庶予子憎。

还

子之还兮,遭我乎狃之间兮。并驱从两肩兮,揖我谓我儇兮。

子之茂兮,遭我乎狃之道兮。并驱从两牡兮,揖我谓我好兮。

子之昌兮,遭我乎狃之阳兮。并驱从两狼兮,揖我谓我臧兮。

著

俟我于著乎而,充耳以素乎而,尚之以琼华乎而。

俟我于庭乎而,充耳以青乎而,尚之以琼莹乎而。

俟我于堂乎而,充耳以黄乎而,尚之以琼英乎而。

东方之日

东方之日兮,彼姝者子,在我室兮。在我室兮,履我即兮。

东方之月兮,彼姝者子,在我闼兮。在我闼兮,履我发兮。

东方未明

东方未明,颠倒衣裳。颠之倒之,自公召之。

东方未晞,颠倒裳衣。倒之颠之,自公令之。

折柳樊圃,狂夫瞿瞿。不能辰夜,不夙则莫。

南山

南山崔崔,雄狐绥绥。鲁道有荡,齐子由归。既曰归止,曷又怀止?

葛屦五两,冠緌双止。鲁道有荡,齐子庸止。既曰庸止,曷又从止?

蓺麻如之何?衡从其亩。取妻如之何?必告父母。既曰告止,曷又鞠止?

析薪如之何?匪斧不克。取妻如之何?匪媒不得。既曰得止,曷又极止?

甫田

无田甫田,维莠骄骄。无思远人,劳心忉忉。

无田甫田,维莠桀桀。无思远人,劳心怛怛。

婉兮娈兮。总角丱兮。未几见兮,突而弁兮!

卢令

卢令令,其人美且仁。

卢重环,其人美且鬈。

卢重鋂,其人美且偲。

敝笱

敝笱在梁,其鱼鲂鳏。齐子归止,其从如云。

敝笱在梁,其鱼鲂鱮。齐子归止,其从如雨。

敝笱在梁,其鱼唯唯。齐子归止,其从如水。

载驱

载驱薄薄,蕈茀朱郭。鲁道有荡,齐子发夕。

四骊济济,垂辔沵沵。鲁道有荡,齐子岂弟。

汶水汤汤,行人彭彭。鲁道有荡,齐子翱翔。

汶水滔滔,行人儦儦。鲁道有荡,齐了游敖。

猗嗟

猗嗟昌兮,颀而长兮。抑若扬兮,美目扬兮。巧趋跄兮,射则臧兮。

猗嗟名兮,美目清兮。仪既成兮,终日射侯,不出正兮,展我甥兮。

猗嗟娈兮,清扬婉兮。舞则选兮,射则贯兮,四矢反兮,以御乱兮。




国风·魏风


葛屦

纠纠葛屦,可以履霜?掺掺女手,可以缝裳?要之襋之,好人服之。

好人提提,宛然左辟,佩其象揥。维是褊心,是以为刺。

汾沮洳

彼汾沮洳,言采其莫。彼其之子,美无度。美无度,殊异乎公路。

彼汾一方,言采其桑。彼其之子,美如英。美如英,殊异乎公行。

彼汾一曲,言采其藚。彼其之子,美如玉。美如玉,殊异乎公族。

园有桃

园有桃,其实之肴。心之忧矣,我歌且谣。不知我者,谓我士也骄。

彼人是哉,子曰何其?心之忧矣,其谁知之?其谁知之,盖亦勿思!

园有棘,其实之食。心之忧矣,聊以行国。不知我者,谓我士也罔极。

彼人是哉,子曰何其?心之忧矣,其谁知之?其谁知之,盖亦勿思!

陟岵

陟彼岵兮,瞻望父兮。父曰:嗟!予子行役,夙夜无已。上慎旃哉,犹来!无止!

陟彼屺兮,瞻望母兮。母曰:嗟!予季行役,夙夜无寐。上慎旃哉,犹来!无弃!

陟彼冈兮,瞻望兄兮。兄曰:嗟!予弟行役,夙夜必偕。上慎旃哉,犹来!无死!

十亩之间

十亩之间兮,桑者闲闲兮,行与子还兮。

十亩之外兮,桑者泄泄兮,行与子逝兮。

伐檀

坎坎伐檀兮,置之河之干兮。河水清且涟猗。不稼不穑,胡取禾三百廛兮?

不狩不猎,胡瞻尔庭有县貆兮?彼君子兮,不素餐兮!

坎坎伐辐兮,置之河之侧兮。河水清且直猗。不稼不穑,胡取禾三百亿兮?

不狩不猎,胡瞻尔庭有县特兮?彼君子兮,不素食兮!

坎坎伐轮兮,置之河之漘兮。河水清且沦猗。不稼不穑,胡取禾三百囷兮?

不狩不猎,胡瞻尔庭有县鹑兮?彼君子兮,不素飧兮!

硕鼠

硕鼠硕鼠,无食我黍!三岁贯女,莫我肯顾。逝将去女,适彼乐土。乐土乐土,爰得我所。

硕鼠硕鼠,无食我麦!三岁贯女,莫我肯德。逝将去女,适彼乐国。乐国乐国,爰得我直。

硕鼠硕鼠,无食我苗!三岁贯女,莫我肯劳。逝将去女,适彼乐郊。乐郊乐郊,谁之永号?




国风·唐风


蟋蟀

蟋蟀在堂,岁聿其莫。今我不乐,日月其除。无已大康,职思其居。好乐无荒,良士瞿瞿。

蟋蟀在堂,岁聿其逝。今我不乐,日月其迈。无已大康,职思其外。好乐无荒,良士蹶蹶。

蟋蟀在堂,役车其休。今我不乐,日月其慆。无以大康。职思其忧。好乐无荒,良士休休。

山有枢

山有枢,隰有榆。子有衣裳,弗曳弗娄。子有车马,弗驰弗驱。宛其死矣,他人是愉。

山有栲,隰有杻。子有廷内,弗洒弗扫。子有钟鼓,弗鼓弗考。宛其死矣,他人是保。

山有漆,隰有栗。子有酒食,何不日鼓瑟?且以喜乐,且以永日。宛其死矣,他人入室。

扬之水

扬之水,白石凿凿。素衣朱襮,从子于沃。既见君子,云何不乐?

扬之水,白石皓皓。素衣朱绣,从子于鹄。既见君子,云何其忧?

扬之水,白石粼粼。我闻有命,不敢以告人。

椒聊

椒聊之实,蕃衍盈升。彼其之子,硕大无朋。椒聊且,远条且。

椒聊之实,蕃衍盈匊。彼其之子,硕大且笃。椒聊且,远条且。

绸缪

绸缪束薪,三星在天。今夕何夕,见此良人?子兮子兮,如此良人何?

绸缪束刍,三星在隅。今夕何夕,见此邂逅?子兮子兮,如此邂逅何?

绸缪束楚,三星在户。今夕何夕,见此粲者?子兮子兮,如此粲者何?

杕杜

有杕之杜,其叶湑湑。独行踽踽。岂无他人?不如我同父。

嗟行之人,胡不比焉?人无兄弟,胡不佽焉?

有杕之杜,其叶箐箐。独行 。岂无他人?不如我同姓。

嗟行之人,胡不比焉?人无兄弟,胡不佽焉?

羔裘

羔裘豹祛,自我人居居。岂无他人?维子之故。

羔裘豹褎,自我人究究。岂无他人?维子之好。

鸨羽

肃肃鸨羽,集于苞栩。王事靡盬,不能蓺稷黍。父母何怙?悠悠苍天,曷其有所?

肃肃鸨翼,集于苞棘。王事靡盬,不能蓺黍稷。父母何食?悠悠苍天,曷其有极?

肃肃鸨行,集于苞桑,王事靡盬,不能蓺稻梁。父母何尝?悠悠苍天,曷其有常?

无衣

岂曰无衣七兮?不如子之衣,安且吉兮!

岂曰无衣六兮?不如子之衣,安且燠兮!

有杕之杜

有杕之杜,生于道左。彼君子兮,噬肯适我?中心好之,曷饮食之?

有杕之杜,生于道周。彼君子兮,噬肯来游?中心好之,曷饮食之?

葛生

葛生蒙楚,蔹蔓于野。予美亡此,谁与?独处?

葛生蒙棘,蔹蔓于域。予美亡此,谁与?独息?

角枕粲兮,锦衾烂兮。予美亡此,谁与?独旦?

夏之日,冬之夜。百岁之后,归于其居。

冬之夜,夏之日。百岁之后,归于其室。

采苓

采苓采苓,首阳之巅。人之为言,苟亦无信。舍旃舍旃,苟亦无然。人之为言,胡得焉?

采苦采苦,首阳之下。人之为言,苟亦无与。舍旃舍旃,苟亦无然。人之为言,胡得焉?

采葑采葑,首阳之东。人之为言,苟亦无从。舍旃舍旃,苟亦无然。人之为言,胡得焉?




国风·秦风


车邻

有车邻邻,有马白颠。未见君子,寺人之令。

阪有漆,隰有栗。既见君子,并坐鼓瑟。今者不乐,逝者其耋。

阪有桑,隰有杨。既见君子,并坐鼓簧。今者不乐,逝者其亡。

驷驖

驷驖孔阜,六辔在手。公之媚子,从公于狩。

奉时辰牡,辰牡孔硕。公曰左之,舍拔则获。

游于北园,四马既闲。輶车鸾镳,载猃歇骄。

小戎

小戎俴收,五楘梁辀。游环胁驱,阴靷鋈续。文茵畅毂,驾我骐 。

言念君子,温其如玉。在其板屋,乱我心曲。

四牡孔阜,六辔在手。骐骝是中,騧骊是骖。龙盾之合,鋈以觼軜。

言念君子,温其在邑。方何为期?胡然我念之!

俴驷孔群,厹矛鋈錞。蒙伐有苑,虎韔镂膺。

交韔二弓,竹闭绲縢。言念君子,载寝载兴。厌厌良人,秩秩德音。

蒹葭

蒹葭苍苍,白露为霜。所谓伊人,在水一方,溯洄从之,道阻且长。溯游从之,宛在水中央。

蒹葭萋萋,白露未晞。所谓伊人,在水之湄。溯洄从之,道阻且跻。溯游从之,宛在水中坻。

蒹葭采采,白露未已。所谓伊人,在水之涘。溯洄从之,道阻且右。溯游从之,宛在水中沚。

终南

终南何有?有条有梅。君子至止,锦衣狐裘。颜如渥丹,其君也哉!

终南何有?有纪有堂。君子至止,黻衣绣裳。佩玉将将,寿考不忘!

黄鸟

交交黄鸟,止于棘。谁从穆公?子车奄息。维此奄息,百夫之特。临其穴,惴惴其栗。彼苍者天,歼我良人!如可赎兮,人百其身!

交交黄鸟,止于桑。谁从穆公?子车仲行。维此仲行,百夫之防。临其穴,惴惴其栗。彼苍者天,歼我良人!如可赎兮,人百其身!

交交黄鸟,止于楚。谁从穆公?子车针虎。维此针虎,百夫之御。临其穴,惴惴其栗。彼苍者天,歼我良人!如可赎兮,人百其身!

晨风

鴥彼晨风,郁彼北林。未见君子,忧心钦钦。如何如何,忘我实多!

山有苞栎,隰有六駮。未见君子,忧心靡乐。如何如何,忘我实多!

山有苞棣,隰有树檖。未见君子,忧心如醉。如何如何,忘我实多!

无衣

岂曰无衣?与子同袍。王于兴师,修我戈矛。与子同仇!

岂曰无衣?与子同泽。王于兴师,修我矛戟。与子偕作!

岂曰无衣?与子同裳。王于兴师,修我甲兵。与子偕行!

渭阳

我送舅氏,曰至渭阳。何以赠之?路车乘黄。

我送舅氏,悠悠我思。何以赠之?琼瑰玉佩。

权舆

於我乎,夏屋渠渠,今也每食无余。于嗟乎,不承权舆!

於我乎,每食四簋,今也每食不饱。于嗟乎,不承权舆!




国风·陈风


宛丘

子之汤兮,宛丘之上兮。洵有情兮,而无望兮。

坎其击鼓,宛丘之下。无冬无夏,值其鹭羽。

坎其击缶,宛丘之道。无冬无夏,值其鹭翿。

东门之枌

东门之枌,宛丘之栩。子仲之子,婆娑其下。

穀旦于差,南方之原。不绩其麻,市也婆娑。

穀旦于逝,越以鬷迈。视尔如荍,贻我握椒。

衡门

衡门之下,可以栖迟。泌之洋洋,可以乐饥。

岂其食鱼,必河之鲂?岂其取妻,必齐之姜?

岂其食鱼,必河之鲤?岂其取妻,必宋之子?

东门之池

东门之池,可以沤麻。彼美淑姬,可与晤歌。

东门之池,可以沤纻。彼美淑姬,可与晤语。

东门之池,可以沤菅。彼美淑姬,可与晤言。

东门之杨

东门之杨,其叶牂牂。昏以为期,明星煌煌。

东门之杨,其叶肺肺。昏以为期,明星晢晢。

墓门

墓门有棘,斧以斯之。夫也不良,国人知之。知而不已,谁昔然矣。

墓门有梅,有鸮萃止。夫也不良,歌以讯之。讯予不顾,颠倒思予。

防有鹊巢

防有鹊巢,邛有旨苕。谁侜予美?心焉忉忉。

中唐有甓,邛有旨鷊。谁侜予美?心焉惕惕。

月出

月出皎兮。佼人僚兮。舒窈纠兮。劳心悄兮。

月出皓兮。佼人懰兮。舒忧受兮。劳心慅兮。

月出照兮。佼人燎兮。舒夭绍兮。劳心惨兮。

株林

胡为乎株林?从夏南兮!匪适株林,从夏南兮!

驾我乘马,说于株野。乘我乘驹,朝食于株!

泽陂

彼泽之陂,有蒲与荷。有美一人,伤如之何?寤寐无为,涕泗滂沱。

彼泽之陂,有蒲与蕳。有美一人,硕大且卷。寤寐无为,中心悁悁。

彼泽之陂,有蒲菡萏。有美一人,硕大且俨。寤寐无为,辗转伏枕。




国风·桧风


羔裘

羔裘逍遥,狐裘以朝。岂不尔思?劳心忉忉。

羔裘翱翔,狐裘在堂。岂不尔思?我心忧伤。

羔裘如膏,日出有曜。岂不尔思?中心是悼。

素冠

庶见素冠兮,棘人栾栾兮。劳心慱慱兮。

庶见素衣兮,我心伤悲兮。聊与子同归兮。

庶见素韠兮,我心蕴结兮。聊与子如一兮。

隰有苌楚

隰有苌楚,猗傩其枝,夭之沃沃,乐子之无知。

隰有苌楚,猗傩其华,夭之沃沃。乐子之无家。

隰有苌楚,猗傩其实,夭之沃沃。乐子之无室。

匪风

匪风发兮,匪车偈兮。顾瞻周道,中心怛兮。

匪风飘兮,匪车嘌兮。顾瞻周道,中心吊兮。

谁能亨鱼?溉之釜鬵。谁将西归?怀之好音。




国风·曹风


蜉蝣

蜉蝣之羽,衣裳楚楚。心之忧矣,于我归处。

蜉蝣之翼,采采衣服。心之忧矣,于我归息。

蜉蝣掘阅,麻衣如雪。心之忧矣,于我归说。

候人

彼候人兮,何戈与祋。彼其之子,三百赤芾。

维鹈在梁,不?/td> 

南山

南山崔崔,雄狐绥绥。鲁道有荡,齐子由归。既曰归止,曷又怀止?

葛屦五两,冠緌双止。鲁道有荡,齐子庸止。既曰庸止,曷又从止?

蓺麻如之何?衡从其亩。取妻如之何?必告父母。既曰告止,曷又鞠止?

析薪如之何?匪斧不克。取妻如之何?匪媒不得。既曰得止,曷又极止?

甫田

无田甫田,维莠骄骄。无思远人,劳心忉忉。

无田甫田,维莠桀桀。无思远人,劳心怛怛。

婉兮娈兮。总角丱兮。未几见兮,突而弁兮!

卢令

卢令令,其人美且仁。

卢重环,其人美且鬈。

卢重鋂,其人美且偲。

敝笱

敝笱在梁,其鱼鲂鳏。齐子归止,其从如云。

敝笱在梁,其鱼鲂鱮。齐子归止,其从如雨。

敝笱在梁,其鱼唯唯。齐子归止,其从如水。

载驱

载驱薄薄,蕈茀朱郭。鲁道有荡,齐子发夕。

四骊济济,垂辔沵沵。鲁道有荡,齐子岂弟。

汶水汤汤,行人彭彭。鲁道有荡,齐子翱翔。

汶水滔滔,行人儦儦。鲁道有荡,齐了游敖。

猗嗟

猗嗟昌兮,颀而长兮。抑若扬兮,美目扬兮。巧趋跄兮,射则臧兮。

猗嗟名兮,美目清兮。仪既成兮,终日射侯,不出正兮,展我甥兮。

猗嗟娈兮,清扬婉兮。舞则选兮,射则贯兮,四矢反兮,以御乱兮。




国风·魏风


葛屦

纠纠葛屦,可以履霜?掺掺女手,可以缝裳?要之襋之,好人服之。

好人提提,宛然左辟,佩其象揥。维是褊心,是以为刺。

汾沮洳

彼汾沮洳,言采其莫。彼其之子,美无度。美无度,殊异乎公路。

彼汾一方,言采其桑。彼其之子,美如英。美如英,殊异乎公行。

彼汾一曲,言采其藚。彼其之子,美如玉。美如玉,殊异乎公族。

园有桃

园有桃,其实之肴。心之忧矣,我歌且谣。不知我者,谓我士也骄。

彼人是哉,子曰何其?心之忧矣,其谁知之?其谁知之,盖亦勿思!

园有棘,其实之食。心之忧矣,聊以行国。不知我者,谓我士也罔极。

彼人是哉,子曰何其?心之忧矣,其谁知之?其谁知之,盖亦勿思!

陟岵

陟彼岵兮,瞻望父兮。父曰:嗟!予子行役,夙夜无已。上慎旃哉,犹来!无止!

陟彼屺兮,瞻望母兮。母曰:嗟!予季行役,夙夜无寐。上慎旃哉,犹来!无弃!

陟彼冈兮,瞻望兄兮。兄曰:嗟!予弟行役,夙夜必偕。上慎旃哉,犹来!无死!

十亩之间

十亩之间兮,桑者闲闲兮,行与子还兮。

十亩之外兮,桑者泄泄兮,行与子逝兮。

伐檀

坎坎伐檀兮,置之河之干兮。河水清且涟猗。不稼不穑,胡取禾三百廛兮?

不狩不猎,胡瞻尔庭有县貆兮?彼君子兮,不素餐兮!

坎坎伐辐兮,置之河之侧兮。河水清且直猗。不稼不穑,胡取禾三百亿兮?

不狩不猎,胡瞻尔庭有县特兮?彼君子兮,不素食兮!

坎坎伐轮兮,置之河之漘兮。河水清且沦猗。不稼不穑,胡取禾三百囷兮?

不狩不猎,胡瞻尔庭有县鹑兮?彼君子兮,不素飧兮!

硕鼠

硕鼠硕鼠,无食我黍!三岁贯女,莫我肯顾。逝将去女,适彼乐土。乐土乐土,爰得我所。

硕鼠硕鼠,无食我麦!三岁贯女,莫我肯德。逝将去女,适彼乐国。乐国乐国,爰得我直。

硕鼠硕鼠,无食我苗!三岁贯女,莫我肯劳。逝将去女,适彼乐郊。乐郊乐郊,谁之永号?




国风·唐风


蟋蟀

蟋蟀在堂,岁聿其莫。今我不乐,日月其除。无已大康,职思其居。好乐无荒,良士瞿瞿。

蟋蟀在堂,岁聿其逝。今我不乐,日月其迈。无已大康,职思其外。好乐无荒,良士蹶蹶。

蟋蟀在堂,役车其休。今我不乐,日月其慆。无以大康。职思其忧。好乐无荒,良士休休。

山有枢

山有枢,隰有榆。子有衣裳,弗曳弗娄。子有车马,弗驰弗驱。宛其死矣,他人是愉。

山有栲,隰有杻。子有廷内,弗洒弗扫。子有钟鼓,弗鼓弗考。宛其死矣,他人是保。

山有漆,隰有栗。子有酒食,何不日鼓瑟?且以喜乐,且以永日。宛其死矣,他人入室。

扬之水

扬之水,白石凿凿。素衣朱襮,从子于沃。既见君子,云何不乐?

扬之水,白石皓皓。素衣朱绣,从子于鹄。既见君子,云何其忧?

扬之水,白石粼粼。我闻有命,不敢以告人。

椒聊

椒聊之实,蕃衍盈升。彼其之子,硕大无朋。椒聊且,远条且。

椒聊之实,蕃衍盈匊。彼其之子,硕大且笃。椒聊且,远条且。

绸缪

绸缪束薪,三星在天。今夕何夕,见此良人?子兮子兮,如此良人何?

绸缪束刍,三星在隅。今夕何夕,见此邂逅?子兮子兮,如此邂逅何?

绸缪束楚,三星在户。今夕何夕,见此粲者?子兮子兮,如此粲者何?

杕杜

有杕之杜,其叶湑湑。独行踽踽。岂无他人?不如我同父。

嗟行之人,胡不比焉?人无兄弟,胡不佽焉?

有杕之杜,其叶箐箐。独行 。岂无他人?不如我同姓。

嗟行之人,胡不比焉?人无兄弟,胡不佽焉?

羔裘

羔裘豹祛,自我人居居。岂无他人?维子之故。

羔裘豹褎,自我人究究。岂无他人?维子之好。

鸨羽

肃肃鸨羽,集于苞栩。王事靡盬,不能蓺稷黍。父母何怙?悠悠苍天,曷其有所?

肃肃鸨翼,集于苞棘。王事靡盬,不能蓺黍稷。父母何食?悠悠苍天,曷其有极?

肃肃鸨行,集于苞桑,王事靡盬,不能蓺稻梁。父母何尝?悠悠苍天,曷其有常?

无衣

岂曰无衣七兮?不如子之衣,安且吉兮!

岂曰无衣六兮?不如子之衣,安且燠兮!

有杕之杜

有杕之杜,生于道左。彼君子兮,噬肯适我?中心好之,曷饮食之?

有杕之杜,生于道周。彼君子兮,噬肯来游?中心好之,曷饮食之?

葛生

葛生蒙楚,蔹蔓于野。予美亡此,谁与?独处?

葛生蒙棘,蔹蔓于域。予美亡此,谁与?独息?

角枕粲兮,锦衾烂兮。予美亡此,谁与?独旦?

夏之日,冬之夜。百岁之后,归于其居。

冬之夜,夏之日。百岁之后,归于其室。

采苓

采苓采苓,首阳之巅。人之为言,苟亦无信。舍旃舍旃,苟亦无然。人之为言,胡得焉?

采苦采苦,首阳之下。人之为言,苟亦无与。舍旃舍旃,苟亦无然。人之为言,胡得焉?

采葑采葑,首阳之东。人之为言,苟亦无从。舍旃舍旃,苟亦无然。人之为言,胡得焉?




国风·秦风


车邻

有车邻邻,有马白颠。未见君子,寺人之令。

阪有漆,隰有栗。既见君子,并坐鼓瑟。今者不乐,逝者其耋。

阪有桑,隰有杨。既见君子,并坐鼓簧。今者不乐,逝者其亡。

驷驖

驷驖孔阜,六辔在手。公之媚子,从公于狩。

奉时辰牡,辰牡孔硕。公曰左之,舍拔则获。

游于北园,四马既闲。輶车鸾镳,载猃歇骄。

小戎

小戎俴收,五楘梁辀。游环胁驱,阴靷鋈续。文茵畅毂,驾我骐 。

言念君子,温其如玉。在其板屋,乱我心曲。

四牡孔阜,六辔在手。骐骝是中,騧骊是骖。龙盾之合,鋈以觼軜。

言念君子,温其在邑。方何为期?胡然我念之!

俴驷孔群,厹矛鋈錞。蒙伐有苑,虎韔镂膺。

交韔二弓,竹闭绲縢。言念君子,载寝载兴。厌厌良人,秩秩德音。

蒹葭

蒹葭苍苍,白露为霜。所谓伊人,在水一方,溯洄从之,道阻且长。溯游从之,宛在水中央。

蒹葭萋萋,白露未晞。所谓伊人,在水之湄。溯洄从之,道阻且跻。溯游从之,宛在水中坻。

蒹葭采采,白露未已。所谓伊人,在水之涘。溯洄从之,道阻且右。溯游从之,宛在水中沚。

终南

终南何有?有条有梅。君子至止,锦衣狐裘。颜如渥丹,其君也哉!

终南何有?有纪有堂。君子至止,黻衣绣裳。佩玉将将,寿考不忘!

黄鸟

交交黄鸟,止于棘。谁从穆公?子车奄息。维此奄息,百夫之特。临其穴,惴惴其栗。彼苍者天,歼我良人!如可赎兮,人百其身!

交交黄鸟,止于桑。谁从穆公?子车仲行。维此仲行,百夫之防。临其穴,惴惴其栗。彼苍者天,歼我良人!如可赎兮,人百其身!

交交黄鸟,止于楚。谁从穆公?子车针虎。维此针虎,百夫之御。临其穴,惴惴其栗。彼苍者天,歼我良人!如可赎兮,人百其身!

晨风

鴥彼晨风,郁彼北林。未见君子,忧心钦钦。如何如何,忘我实多!

山有苞栎,隰有六駮。未见君子,忧心靡乐。如何如何,忘我实多!

山有苞棣,隰有树檖。未见君子,忧心如醉。如何如何,忘我实多!

无衣

岂曰无衣?与子同袍。王于兴师,修我戈矛。与子同仇!

岂曰无衣?与子同泽。王于兴师,修我矛戟。与子偕作!

岂曰无衣?与子同裳。王于兴师,修我甲兵。与子偕行!

渭阳

我送舅氏,曰至渭阳。何以赠之?路车乘黄。

我送舅氏,悠悠我思。何以赠之?琼瑰玉佩。

权舆

於我乎,夏屋渠渠,今也每食无余。于嗟乎,不承权舆!

於我乎,每食四簋,今也每食不饱。于嗟乎,不承权舆!




国风·陈风


宛丘

子之汤兮,宛丘之上兮。洵有情兮,而无望兮。

坎其击鼓,宛丘之下。无冬无夏,值其鹭羽。

坎其击缶,宛丘之道。无冬无夏,值其鹭翿。

东门之枌

东门之枌,宛丘之栩。子仲之子,婆娑其下。

穀旦于差,南方之原。不绩其麻,市也婆娑。

穀旦于逝,越以鬷迈。视尔如荍,贻我握椒。

衡门

衡门之下,可以栖迟。泌之洋洋,可以乐饥。

岂其食鱼,必河之鲂?岂其取妻,必齐之姜?

岂其食鱼,必河之鲤?岂其取妻,必宋之子?

东门之池

东门之池,可以沤麻。彼美淑姬,可与晤歌。

东门之池,可以沤纻。彼美淑姬,可与晤语。

东门之池,可以沤菅。彼美淑姬,可与晤言。

东门之杨

东门之杨,其叶牂牂。昏以为期,明星煌煌。

东门之杨,其叶肺肺。昏以为期,明星晢晢。

墓门

墓门有棘,斧以斯之。夫也不良,国人知之。知而不已,谁昔然矣。

墓门有梅,有鸮萃止。夫也不良,歌以讯之。讯予不顾,颠倒思予。

防有鹊巢

防有鹊巢,邛有旨苕。谁侜予美?心焉忉忉。

中唐有甓,邛有旨鷊。谁侜予美?心焉惕惕。

月出

月出皎兮。佼人僚兮。舒窈纠兮。劳心悄兮。

月出皓兮。佼人懰兮。舒忧受兮。劳心慅兮。

月出照兮。佼人燎兮。舒夭绍兮。劳心惨兮。

株林

胡为乎株林?从夏南兮!匪适株林,从夏南兮!

驾我乘马,说于株野。乘我乘驹,朝食于株!

泽陂

彼泽之陂,有蒲与荷。有美一人,伤如之何?寤寐无为,涕泗滂沱。

彼泽之陂,有蒲与蕳。有美一人,硕大且卷。寤寐无为,中心悁悁。

彼泽之陂,有蒲菡萏。有美一人,硕大且俨。寤寐无为,辗转伏枕。




国风·桧风


羔裘

羔裘逍遥,狐裘以朝。岂不尔思?劳心忉忉。

羔裘翱翔,狐裘在堂。岂不尔思?我心忧伤。

羔裘如膏,日出有曜。岂不尔思?中心是悼。

素冠

庶见素冠兮,棘人栾栾兮。劳心慱慱兮。

庶见素衣兮,我心伤悲兮。聊与子同归兮。

庶见素韠兮,我心蕴结兮。聊与子如一兮。

隰有苌楚

隰有苌楚,猗傩其枝,夭之沃沃,乐子之无知。

隰有苌楚,猗傩其华,夭之沃沃。乐子之无家。

隰有苌楚,猗傩其实,夭之沃沃。乐子之无室。

匪风

匪风发兮,匪车偈兮。顾瞻周道,中心怛兮。

匪风飘兮,匪车嘌兮。顾瞻周道,中心吊兮。

谁能亨鱼?溉之釜鬵。谁将西归?怀之好音。




国风·曹风


蜉蝣

蜉蝣之羽,衣裳楚楚。心之忧矣,于我归处。

蜉蝣之翼,采采衣服。心之忧矣,于我归息。

蜉蝣掘阅,麻衣如雪。心之忧矣,于我归说。

候人

彼候人兮,何戈与祋。彼其之子,三百赤芾。

维鹈在梁,不濡其翼。彼其之子,不称其服。

维鹈在梁,不濡其咮。彼其之子,不遂其媾。

荟兮蔚兮,南山朝隮。婉兮娈兮,季女斯饥。

鳲鸠

鳲鸠在桑,其子七兮。淑人君子,其仪一兮。其仪一兮,心如结兮。

鳲鸠在桑,其子在梅。淑人君子,其带伊丝。其带伊丝,其弁伊骐。

鳲鸠在桑,其子在棘。淑人君子,其仪不忒。其仪不忒,正是四国。

鳲鸠在桑,其子在榛。淑人君子,正是国人,正是国人。胡不万年?

下泉

冽彼下泉,浸彼苞稂。忾我寤叹,念彼周京。

冽彼下泉,浸彼苞萧。忾我寤叹,念彼京周。

冽彼下泉,浸彼苞蓍。忾我寤叹,念彼京师。

芃芃黍苗,阴雨膏之。四国有王,郇伯劳之。

 


国风·豳风


七月

七月流火,九月授衣。一之日觱发,二之日栗烈。无衣无褐,何以卒岁?三之日于耜,四之日举趾。同我妇子,馌彼南亩。田畯至喜。

七月流火,九月授衣。春日载阳,有鸣仓庚。女执懿筐,遵彼微行,爰求柔桑。春日迟迟,采蘩祁祁。女心伤悲,殆及公子同归。

七月流火,八月萑苇。蚕月条桑,取彼斧斨。以伐远扬,猗彼女桑。七月鸣鵙,八月载绩。载玄载黄,我朱孔阳,为公子裳。

四月秀葽,五月鸣蜩。八月其获,十月陨萚。一之日于貉,取彼狐狸,为公子裘。二之日其同,载缵武功。言私其豵,献豜于公。

五月斯螽动股,六月莎鸡振羽。七月在野,八月在宇,九月在户,十月蟋蟀,入我床下。穹窒熏鼠,塞向墐户。嗟我妇子,曰为改岁,入此室处。

六月食郁及薁,七月亨葵及菽。八月剥枣,十月获稻。为此春酒,以介眉寿。七月食瓜,八月断壶,九月叔苴,采荼薪樗。食我农夫。

九月筑场圃,十月纳禾稼。黍稷重穋,禾麻菽麦。嗟我农夫,我稼既同,上入执宫功。昼尔于茅,宵尔索綯,亟其乘屋,其始播百谷。

二之日凿冰冲冲,三之日纳于凌阴。四之日其蚤,献羔祭韭。九月肃霜,十月涤场。朋酒斯飨,曰杀羔羊,跻彼公堂。称彼兕觥:万寿无疆!

鸱鸮

鸱鸮鸱鸮,既取我子,无毁我室。恩斯勤斯,鬻子之闵斯。

迨天之未阴雨,彻彼桑土,绸缪牖户。今女下民,或敢侮予?

予手拮据,予所捋荼。予所蓄租,予口卒瘏,曰予未有室家。

予羽谯谯,予尾翛翛,予室翘翘。风雨所漂摇,予维音哓哓!

东山

我徂东山,慆慆不归。我来自东,零雨其濛。我东曰归,我心西悲。制彼裳衣,勿士行枚。蜎蜎者蠋,烝在桑野。敦彼独宿,亦在车下。

我徂东山,慆慆不归。我来自东,零雨其濛。果臝之实,亦施于宇。伊威在室,蠨蛸在户。町畽鹿场,熠耀宵行。不可畏也,伊可怀也。

我徂东山,慆慆不归。我来自东,零雨其濛。鹳鸣于垤,妇叹于室。洒扫穹窒,我征聿至。有敦瓜苦,烝在栗薪。自我不见,于今三年。

我徂东山,慆慆不归。我来自东,零雨其濛。仓庚于飞,熠耀其羽。之子于归,皇驳其马。亲结其缡,九十其仪。其新孔嘉,其旧如之何?

破斧

既破我斧,又缺我斨。周公东征,四国是皇。哀我人斯,亦孔之将。

既破我斧,又缺我锜。周公东征,四国是遒。哀我人斯,亦孔之嘉。

既破我斧,又缺我銶。周公东征,四国是遒。哀我人斯,亦孔之休。

伐柯

伐柯如何?匪斧不克。取妻如何?匪媒不得。

伐柯伐柯,其则不远。我觏之子,笾豆有践。

九罭

九罭之鱼,鳟鲂。我觏之子,衮衣绣裳。

鸿飞遵渚,公归无所,於女信处。

鸿飞遵陆,公归不复,於女信宿。

是以有衮衣兮,无以我公归兮,无使我心悲兮。

狼跋

狼跋其胡,载疐其尾。公孙硕肤,赤舄几几。

狼疐其尾,载跋其胡。公孙硕肤,德音不瑕?




小雅·鹿鸣之什


鹿鸣

呦呦鹿鸣,食野之苹。我有嘉宾,鼓瑟吹笙。吹笙鼓簧,承筐是将。人之好我,示我周行。

呦呦鹿鸣,食野之蒿。我有嘉宾,德音孔昭。视民不恌,君子是则是效。我有旨酒,嘉宾式燕以敖。

呦呦鹿鸣,食野之芩。我有嘉宾,鼓瑟鼓琴。鼓瑟鼓琴,和乐且湛。我有旨酒,以燕乐嘉宾之心。

四牡

四牡騑騑,周道倭迟。岂不怀归?王事靡盬,我心伤悲。

四牡騑騑,啴啴骆马。岂不怀归?王事靡盬,不遑启处。

翩翩者鵻,载飞载下,集于苞栩。王事靡盬,不遑将父。

翩翩者鵻,载飞载止,集于苞杞。王事靡盬,不遑将母。

驾彼四骆,载骤骎骎。岂不怀归?是用作歌,将母来谂。

皇皇者华

皇皇者华,于彼原隰。駪駪征夫,每怀靡及。

我马维驹,六辔如濡。载驰载驱,周爰咨诹。

我马维骐,六辔如丝。载驰载驱,周爰咨谋。

我马维骆,六辔沃若。载驰载驱,周爰咨度。

我马维骃,六辔既均。载驰载驱,周爰咨询。

常棣

常棣之华,鄂不韡韡。凡今之人,莫如兄弟。

死丧之威,兄弟孔怀。原隰裒矣,兄弟求矣。

脊令在原,兄弟急难。每有良朋,况也永叹。

兄弟阋于墙,外御其务。每有良朋,烝也无戎。

丧乱既平,既安且宁。虽有兄弟,不如友生?

傧尔笾豆,饮酒之饫。兄弟既具,和乐且孺。

妻子好合,如鼓瑟琴。兄弟既翕,和乐且湛。

宜尔室家,乐尔妻帑。是究是图,亶其然乎?

伐木

伐木丁丁,鸟鸣嘤嘤。出自幽谷,迁于乔木。嘤其鸣矣,求其友声。相彼鸟矣,犹求友声。矧伊人矣,不求友生?神之听之,终和且平。

伐木许许,酾酒有藇!既有肥羜,以速诸父。宁适不来,微我弗顾。於粲洒扫,陈馈八簋。既有肥牡,以速诸舅。宁适不来,微我有咎。

伐木于阪,酾酒有衍。笾豆有践,兄弟无远。民之失德,乾餱以愆。有酒湑我,无酒酤我。坎坎鼓我,蹲蹲舞我。迨我暇矣,饮此湑矣。

天保

天保定尔,亦孔之固。俾尔单厚,何福不除?俾尔多益,以莫不庶。

天保定尔,俾尔戬穀。罄无不宜,受天百禄。降尔遐福,维日不足。

天保定尔,以莫不兴。如山如阜,如冈如陵,如川之方至,以莫不增。

吉蠲为饎,是用孝享。禴祠烝尝,于公先王。君曰:卜尔,万寿无疆。

神之吊矣,诒尔多福。民之质矣,日用饮食。群黎百姓,遍为尔德。

如月之恒,如日之升。如南山之寿,不骞不崩。如松柏之茂,无不尔或承。

采薇

采薇采薇,薇亦作止。曰归曰归,岁亦莫止。靡室靡家,玁狁之故。不遑启居,玁狁之故。

采薇采薇,薇亦柔止。曰归曰归,心亦忧止。忧心烈烈,载饥载渴。我戍未定,靡使归聘。

采薇采薇,薇亦刚止。曰归曰归,岁亦阳止。王事靡盬,不遑启处。忧心孔疚,我行不来!

彼尔维何?维常之华。彼路斯何?君子之车。戎车既驾,四牡业业。岂敢定居?一月三捷。

驾彼四牡,四牡骙骙。君子所依,小人所腓。四牡翼翼,象弭鱼服。岂不日戒?玁狁孔棘!

昔我往矣,杨柳依依。今我来思,雨雪霏霏。行道迟迟,载渴载饥。我心伤悲,莫知我哀!

出车

我出我车,于彼牧矣。自天子所,谓我来矣。召彼仆夫,谓之载矣。王事多难,维其棘矣。

我出我车,于彼郊矣。设此旐矣,建彼旄矣。彼旟旐斯,胡不旆旆?忧心悄悄,仆夫况瘁。

王命南仲,往城于方。出车彭彭,旂旐央央。天子命我,城彼朔方。赫赫南仲,玁狁于襄。

昔我往矣,黍稷方华。今我来思,雨雪载途。王事多难,不遑启居。岂不怀归?畏此简书。

喓喓草虫,趯趯阜螽。未见君子,忧心忡忡。既见君子,我心则降。赫赫南仲,薄伐西戎。

春日迟迟,卉木萋萋。仓庚喈喈,采蘩祁祁。执讯获丑,薄言还归。赫赫南仲,玁狁于夷。

杕杜

有杕之杜,有其实。王事靡盬,继嗣我日。日月阳止,女心伤止,征夫遑止。

有杕之杜,其叶萋萋。王事靡盬,我心伤悲。卉木萋止,女心悲止,征夫归止!

陟彼北山,言采其杞。王事靡盬,忧我父母。檀车幝幝,四牡痯痯,征夫不远!

匪载匪来,忧心孔疚。斯逝不至,而多为恤。卜筮偕止,会言近止,征夫迩止!

鱼丽

鱼丽于罶,鲿鲨。君子有酒,旨且多。

鱼丽于罶,鲂鳢。君子有酒,多且旨。

鱼丽于罶,鰋鲤。君子有酒,旨且有。物其多矣,

维其嘉矣!物其旨矣,维其偕矣!物其有矣,维其时矣!

南陔(今佚)

白华(今佚)

华黍(今佚)




小雅·南有嘉鱼之什


南有嘉鱼

南有嘉鱼,烝然罩罩。君子有酒,嘉宾式燕以乐。

南有嘉鱼,烝然汕汕。君子有酒,嘉宾式燕以衎。

南有樛木,甘瓠累之。君子有酒,嘉宾式燕绥之。

翩翩者鵻,烝然来思。君子有酒,嘉宾式燕又思。

南山有台

南山有台,北山有莱。乐只君子,邦家之基。乐只君子,万寿无期。

南山有桑,北山有杨。乐只君子,邦家之光。乐只君子,万寿无疆。

南山有杞,北山有李。乐只君子,民之父母。乐只君子,德音不已。

南山有栲,北山有杻。乐只君子,遐不眉寿。乐只君子,德音是茂。

南山有枸,北山有楰。乐只君子,遐不黄耇。乐只君子,保艾尔后。

由庚(今佚)

崇丘(今佚)

由仪(今佚)

蓼萧

蓼彼萧斯,零露湑兮。既见君子,我心写兮。燕笑语兮,是以有誉处兮。

蓼彼萧斯,零露瀼瀼。既见君子,为龙为光。其德不爽,寿考不忘。

蓼彼萧斯,零露泥泥。既见君子,孔燕岂弟。宜兄宜弟,令德寿岂。

蓼彼萧斯,零露浓浓。既见君子,鞗革忡忡。和鸾雍雍,万福攸同。

湛露

湛湛露斯,匪阳不晞。厌厌夜饮,不醉无归。

湛湛露斯,在彼丰草。厌厌夜饮,在宗载考。

湛湛露斯,在彼杞棘。显允君子,莫不令德。

其桐其椅,其实离离。岂弟君子,莫不令仪。

彤弓

彤弓召兮,受言藏之。我有嘉宾,中心贶之。钟鼓既设,一朝飨之。

彤弓召兮,受言载之。我有嘉宾,中心喜之。钟鼓既设,一朝右之。

彤弓召兮,受言櫜之。我有嘉宾,中心好之。钟鼓既设,一朝酬之。

菁菁者莪

菁菁者莪,在彼中阿。既见君子,乐且有仪。

菁菁者莪,在彼中沚。既见君子,我心则喜。

菁菁者莪,在彼中陵。既见君子,锡我百朋。

泛泛杨舟,载沉载浮。既见君子,我心则休。

六月

六月栖栖,戎车既饬。四牡骙骙,载是常服。玁狁孔炽,我是用急。王于出征,以匡王国。

比物四骊,闲之维则。维此六月,既成我服。我服既成,于三十里。王于出征,以佐天子。

四牡修广,其大有颙。薄伐玁狁,以奏肤公。有严有翼,共武之服。共武之服,以定王国。

玁狁匪茹,整居焦获。侵镐及方,至于泾阳。织文鸟章,白旆央央。元戎十乘,以先启行。

戎车既安,如轾如轩。四牡既佶,既佶且闲。薄伐玁狁,至于大原。文武吉甫,万邦为宪。

吉甫燕喜,既多受祉。来归自镐,我行永久。饮御诸友,炰鳖脍鲤。侯谁在矣?张仲孝友。

采芑

薄言采芑,于彼新田,呈此菑亩。方叔涖止,其车三千。师干之试,方叔率止。乘其四骐,四骐翼翼。路车有奭,簟茀鱼服,钩膺鞗革。

薄言采芑,于彼新田,于此中乡。方叔涖止,其车三千。旂旐央央,方叔率止。约軧错衡,八鸾玱玱。服其命服,朱芾斯皇,有玱葱珩。

鴥彼飞隼,其飞戾天,亦集爰止。方叔涖止,其车三千。师干之试,方叔率止。钲人伐鼓,陈师鞠旅。显允方叔,伐鼓渊渊,振旅阗阗。

蠢尔蛮荆,大邦为仇。方叔元老,克壮其犹。方叔率止,执讯获丑。戎车啴啴,啴啴焞焞,如霆如雷。显允方叔,征伐玁狁,蛮荆来威。

车攻

我车既攻,我马既同。四牡庞庞,驾言徂东。

田车既好,田牡孔阜。东有甫草,驾言行狩。

之子于苗,选徒嚣嚣。建旐设旄,搏兽于敖。

驾彼四牡,四牡奕奕。赤芾金舄,会同有绎。

决拾既佽,弓矢既调。射夫既同,助我举柴。

四黄既驾,两骖不猗。不失其驰,舍矢如破。

萧萧马鸣,悠悠旆旌。徒御不惊,大庖不盈。

之子于征,有闻无声。允矣君子,展也大成。

吉日

吉日维戊,既伯既祷。田车既好,四牡孔阜。升彼大阜,从其群丑。

吉日庚午,既差我马。兽之所同,麀鹿麌麌。漆沮之从,天子之所。

瞻彼中原,其祁孔有。儦儦俟俟,或群或友。悉率左右,以燕天子。

既张我弓,既挟我矢。发彼小豝,殪此大兕。以御宾客,且以酌醴。




小雅·鸿雁之什


鸿雁

鸿雁于飞,肃肃其羽。之子于征,劬劳于野。爰及矜人,哀此鳏寡。

鸿雁于飞,集于中泽。之子于垣,百堵皆作。虽则劬劳,其究安宅?

鸿雁于飞,哀鸣嗷嗷。维此哲人,谓我劬劳。维彼愚人,谓我宣骄。

庭燎

夜如何其?夜未央,庭燎之光。君子至止,鸾声将将。

夜如何其?夜未艾,庭燎晣晣。君子至止,鸾声哕哕。

夜如何其?夜乡晨,庭燎有辉。君子至止,言观其旂。

沔水

沔彼流水,朝宗于海。鴥彼飞隼,载飞载止。嗟我兄弟,邦人诸友。莫肯念乱,谁无父母?

沔彼流水,其流汤汤。鴥彼飞隼,载飞载扬。念彼不迹,载起载行。心之忧矣,不可弭忘。

鴥彼飞隼,率彼中陵。民之讹言,宁莫之惩?我友敬矣,谗言其兴。

鹤鸣

鹤鸣于九皋,声闻于野。鱼潜在渊,或在于渚。乐彼之园,爰有树檀,其下维萚。他山之石,可以为错。

鹤鸣于九皋,声闻于天。鱼在于渚,或潜在渊。乐彼之园,爰有树檀,其下维谷。他山之石,可以攻玉。

祈父

祈父,予王之爪牙。胡转予于恤,靡所止居?

祈父,予王之爪士。胡转予于恤,靡所厎止?

祈父,亶不聪。胡转予于恤?有母之尸饔。

白驹

皎皎白驹,食我场苗。絷之维之,以永今朝。所谓伊人,于焉逍遥?

皎皎白驹,食我场藿。絷之维之,以永今夕。所谓伊人,于焉嘉客?

皎皎白驹,贲然来思。尔公尔侯,逸豫无期?慎尔优游,勉尔遁思。

皎皎白驹,在彼空谷。生刍一束,其人如玉。毋金玉尔音,而有遐心。

黄鸟

黄鸟黄鸟,无集于穀,无啄我粟。此邦之人,不我肯穀。言旋言归,复我邦族。

黄鸟黄鸟,无集于桑,无啄我粱。此邦之人,不可与明。言旋言归,复我诸兄。

黄鸟黄鸟,无集于栩,无啄我黍。此邦之人,不可与处。言旋言归,复我诸父。

我行其野

我行其野,蔽芾其樗。婚姻之故,言就尔居。尔不我畜,复我邦家。

我行其野,言采其蓫。婚姻之故,言就尔宿。尔不我畜,言归斯复。

我行其野,言采其葍。不思旧姻,求尔新特。成不以富,亦祗以异。

斯干

秩秩斯干,幽幽南山。如竹苞矣,如松茂矣。兄及弟矣,式相好矣,无相犹矣。

似续妣祖,筑室百堵,西南其户。爰居爰处,爰笑爰语。

约之阁阁,椓之橐橐。风雨攸除,鸟鼠攸去,君子攸芋。

如跂斯翼,如矢斯棘,如鸟斯革,如翚斯飞,君子攸跻。

殖殖其庭,有觉其楹。哙哙其正,哕哕其冥。君子攸宁。

下莞上簟,乃安斯寝。乃寝乃兴,乃占我梦。吉梦维何?维熊维罴,维虺维蛇。

大人占之:维熊维罴,男子之祥;维虺维蛇,女子之祥。

乃生男子,载寝之床。载衣之裳,载弄之璋。其泣喤喤,朱芾斯皇,室家君王。

乃生女子,载寝之地。载衣之裼,载弄之瓦。无非无仪,唯酒食是议,无父母诒罹。

无羊

谁谓尔无羊?三百维群。谁谓尔无牛?九十其犉。尔羊来思,其角濈濈。尔牛来思,其耳湿湿。

或降于阿,或饮于池,或寝或讹。尔牧来思,何蓑何笠,或负其餱。三十维物,尔牲则具。

尔牧来思,以薪以蒸,以雌以雄。尔羊来思,矜矜兢兢,不骞不崩。麾之以肱,毕来既升。

牧人乃梦,众维鱼矣,旐维旟矣,大人占之;众维鱼矣,实维丰年;旐维旟矣,室家溱溱。




小雅·节南山之什


节南山

节彼南山,维石岩岩。赫赫师尹,民具尔瞻。忧心如惔,不敢戏谈。国既卒斩,何用不监!

节彼南山,有实其猗。赫赫师尹,不平谓何。天方荐瘥,丧乱弘多。民言无嘉,惨莫惩嗟。

尹氏大师,维周之氐;秉国之钧,四方是维。天子是毗,俾民不迷。不吊昊天,不宜空我师。

弗躬弗亲,庶民弗信。弗问弗仕,勿罔君子。式夷式已,无小人殆。琐琐姻亚,则无膴仕。

昊天不佣,降此鞠訩。昊天不惠,降此大戾。君子如届,俾民心阕。君子如夷,恶怒是违。

不吊昊天,乱靡有定。式月斯生,俾民不宁。忧心如酲,谁秉国成?不自为政,卒劳百姓。

驾彼四牡,四牡项领。我瞻四方,蹙蹙靡所骋。

方茂尔恶,相尔矛矣。既夷既怿,如相酬矣。

昊天不平,我王不宁。不惩其心,覆怨其正。

家父作诵,以究王訩。式讹尔心,以畜万邦。

正月

正月繁霜,我心忧伤。民之讹言,亦孔之将。念我独兮,忧心京京。哀我小心,癙忧以痒。

父母生我,胡俾我瘉?不自我先,不自我后。好言自口,莠言自口。忧心愈愈,是以有侮。

忧心惸惸,念我无禄。民之无辜,并其臣仆。哀我人斯,于何从禄?瞻乌爰止?于谁之屋?

瞻彼中林,侯薪侯蒸。民今方殆,视天梦梦。既克有定,靡人弗胜。有皇上帝,伊谁云憎?

谓山盖卑,为冈为陵。民之讹言,宁莫之惩。召彼故老,讯之占梦。具曰予圣,谁知乌之雌雄!

谓天盖高,不敢不局。谓地盖厚,不敢不蹐。维号斯言,有伦有脊。哀今之人,胡为虺蜴?

瞻彼阪田,有菀其特。天之杌我,如不我克。彼求我则,如不我得。执我仇仇,亦不我力。

心之忧矣,如或结之。今兹之正,胡然厉矣?燎之方扬,宁或灭之?赫赫宗周,褒姒灭之!

终其永怀,又窘阴雨。其车既载,乃弃尔辅。载输尔载,将伯助予!

无弃尔辅,员于尔辐。屡顾尔仆,不输尔载。终逾绝险,曾是不意。

鱼在于沼,亦匪克乐。潜虽伏矣,亦孔之炤。忧心惨惨,念国之为虐!

彼有旨酒,又有嘉肴。洽比其邻,婚姻孔云。念我独兮,忧心殷殷。

佌佌彼有屋,蔌蔌方有谷。民今之无禄,天夭是椓。哿矣富人,哀此惸独。

十月之交

十月之交,朔月辛卯。日有食之,亦孔之丑。彼月而微,此日而微;今此下民,亦孔之哀。

日月告凶,不用其行。四国无政,不用其良。彼月而食,则维其常;此日而食,于何不臧。

烨烨震电,不宁不令。百川沸腾,山冢崒崩。高岸为谷,深谷为陵。哀今之人,胡惨莫惩?

皇父卿士,番维司徒,家伯维宰,仲允膳夫,棸子内史,蹶维趣马,禹维师氏。醘妻煽方处。

抑此皇父,岂曰不时?胡为我作,不即我谋?彻我墙屋,田卒汙莱。曰予不戕,礼则然矣。

皇父孔圣,作都于向。择三有事,亶侯多藏。不慭遗一老,俾守我王。择有车马,以居徂向。

黾勉从事,不敢告劳。无罪无辜,谗口嚣嚣。下民之孽,匪降自天。噂沓背憎,职竞由人。

悠悠我里,亦孔之痗。四方有羡,我独居忧。民莫不逸,我独不敢休。天命不彻,我不敢效我友自逸。

雨无正

浩浩昊天,不骏其德。降丧饥馑,斩伐四国。旻天疾威,弗虑弗图。舍彼有罪,既伏其辜。若此无罪,沦胥以铺。

周宗既灭,靡所止戾。正大夫离居,莫知我勚。三事大夫,莫肯夙夜。邦君诸侯,莫肯朝夕。庶曰式臧,覆出为恶。

如何昊天,辟言不信。如彼行迈,则靡所臻。凡百君子,各敬尔身。胡不相畏,不畏于天?

戎成不退,饥成不遂。曾我暬御,惨惨日瘁。凡百君子,莫肯用讯。听言则答,谮言则退。

哀哉不能言,匪舌是出,维躬是瘁。哿矣能言,巧言如流,俾躬处休!

维曰予仕,孔棘且殆。云不何使,得罪于天子;亦云可使,怨及朋友。

谓尔迁于王都。曰予未有室家。鼠思泣血,无言不疾。昔尔出居,谁从作尔室?

小旻

旻天疾威,敷于下土。谋犹回遹,何日斯沮?谋臧不从,不臧覆用。我视谋犹,亦孔之邛。

潝潝訿訿,亦孔之哀。谋之其臧,则具是违。谋之不臧,则具是依。我视谋犹,伊于胡厎。

我龟既厌,不我告犹。谋夫孔多,是用不集。发言盈庭,谁敢执其咎?如匪行迈谋,是用不得于道。

哀哉为犹,匪先民是程,匪大犹是经。维迩言是听,维迩言是争。如彼筑室于道谋,是用不溃于成。

国虽靡止,或圣或否。民虽靡膴,或哲或谋,或肃或艾。如彼泉流,无沦胥以败。

不敢暴虎,不敢冯河。人知其一,莫知其他。战战兢兢,如临深渊,如履薄冰。

小宛

宛彼鸣鸠,翰飞戾天。我心忧伤,念昔先人。明发不寐,有怀二人。

人之齐圣,饮酒温克。彼昏不知,壹醉日富。各敬尔仪,天命不又。

中原有菽,庶民采之。螟蛉有子,蜾蠃负之。教诲尔子,式穀似之。

题彼脊令,载飞载鸣。我日斯迈,而月斯征。夙兴夜寐,毋忝尔所生。

交交桑扈,率场啄粟。哀我填寡,宜岸宜狱。握粟出卜,自何能穀?

温温恭人,如集于木。惴惴小心,如临于谷。战战兢兢,如履薄冰。

小弁

弁彼鸴斯,归飞提提。民莫不穀,我独于罹。何辜于天?我罪伊何?心之忧矣,云如之何?

踧踧周道,鞫为茂草。我心忧伤,惄焉如捣。假寐永叹,维忧用老。心之忧矣,疢如疾首。

维桑与梓,必恭敬止。靡瞻匪父,靡依匪母。不属于毛?不罹于里?天之生我,我辰安在?

菀彼柳斯,鸣蜩嘒嘒,有漼者渊,萑苇淠淠。譬彼舟流,不知所届,心之忧矣,不遑假寐。

鹿斯之奔,维足伎伎。雉之朝雊,尚求其雌。譬彼坏木,疾用无枝。心之忧矣,宁莫之知?

相彼投兔,尚或先之。行有死人,尚或墐之。君子秉心,维其忍之。心之忧矣,涕既陨之。

君子信谗,如或酬之。君子不惠,不舒究之。伐木掎矣,析薪扡矣。舍彼有罪,予之佗矣。

莫高匪山,莫浚匪泉。君子无易由言,耳属于垣。无逝我梁,无发我笱。我躬不阅,遑恤我后。

巧言

悠悠昊天,曰父母且。无罪无辜,乱如此幠。昊天已威,予慎无罪。昊天大幠,予慎无辜。

乱之初生,僭始既涵。乱之又生,君子信谗。君子如怒,乱庶遄沮。君子如祉,乱庶遄已。

君子屡盟,乱是用长。君子信盗,乱是用暴。盗言孔甘,乱是用餤。匪其止共,维王之邛。奕奕寝庙,君子作之。秩秩大猷,圣人莫之。

他人有心,予忖度之。跃跃毚兔,遇犬获之。荏染柔木,君子树之。往来行言,心焉数之。蛇蛇硕言,出自口矣。巧言如簧,颜之厚矣。

彼何人斯?居河之麋。无拳无勇,职为乱阶。既微且尰,尔勇伊何?为犹将多,尔居徒几何?

何人斯

彼何人斯?其心孔艰。胡逝我梁,不入我门?伊谁云从?维暴之云。

二人从行,谁为此祸?胡逝我梁,不入唁我?始者不如今,云不我可

彼何人斯?胡逝我陈?我闻其声,不见其身。不愧于人?不畏于天?

彼何人斯?其为飘风。胡不自北?胡不自南?胡逝我梁?絺搅我心。

尔之安行,亦不遑舍。尔之亟行,遑脂尔车。壹者之来,云何其盱。

尔还而入,我心易也。还而不入,否难知也。壹者之来,俾我絺也。

伯氏吹埙,仲氏吹篪。及尔如贯,谅不我郑出此三物,以诅尔斯。

为鬼为蜮,则不可得。有靦面目,视人罔极。作此好歌,以极反侧。

巷伯

萋兮斐兮,成是贝锦。彼谮人者,亦已大甚!

哆兮侈兮,成是南箕。彼谮人者,谁适与谋。

缉缉翩翩,谋欲谮人。慎尔言也,谓尔不信。

捷捷幡幡,谋欲谮言。岂不尔受?既其女迁。

骄人好好,劳人草草。苍天苍天,视彼骄人,矜此劳人。

彼谮人者,谁适与谋?取彼谮人,投畀豺虎。豺虎不食,投畀有北。有北不受,投畀有昊!

杨园之道,猗于亩丘。寺人孟子,作为此诗。凡百君子,敬而听之。




小雅·谷风之什


谷风

习习谷风,维风及雨。将恐将惧,维予与女。将安将乐,女转弃予。

习习谷风,维风及颓。将恐将惧,置予于怀。将安将乐,弃予如遗。

习习谷风,维山崔嵬。无草不死,无木不萎。忘我大德,思我小怨。

蓼莪

蓼蓼者莪,匪莪伊蒿。哀哀父母,生我劬劳。

蓼蓼者莪,匪莪伊蔚。哀哀父母,生我劳瘁。

瓶之罄矣,维罍之耻。鲜民之生,不如死之久矣。无父何怙?无母何恃?出则衔恤,入则靡至。

父兮生我,母兮鞠我。抚我畜我,长我育我,顾我复我,出入腹我。欲报之德。昊天罔极!

南山烈烈,飘风发发。民莫不穀,我独何害!南山律律,飘风弗弗。民莫不穀,我独不?/

维鹈在梁,不濡其咮。彼其之子,不遂其媾。

荟兮蔚兮,南山朝隮。婉兮娈兮,季女斯饥。

鳲鸠

鳲鸠在桑,其子七兮。淑人君子,其仪一兮。其仪一兮,心如结兮。

鳲鸠在桑,其子在梅。淑人君子,其带伊丝。其带伊丝,其弁伊骐。

鳲鸠在桑,其子在棘。淑人君子,其仪不忒。其仪不忒,正是四国。

鳲鸠在桑,其子在榛。淑人君子,正是国人,正是国人。胡不万年?

下泉

冽彼下泉,浸彼苞稂。忾我寤叹,念彼周京。

冽彼下泉,浸彼苞萧。忾我寤叹,念彼京周。

冽彼下泉,浸彼苞蓍。忾我寤叹,念彼京师。

芃芃黍苗,阴雨膏之。四国有王,郇伯劳之。

 


国风·豳风


七月

七月流火,九月授衣。一之日觱发,二之日栗烈。无衣无褐,何以卒岁?三之日于耜,四之日举趾。同我妇子,馌彼南亩。田畯至喜。

七月流火,九月授衣。春日载阳,有鸣仓庚。女执懿筐,遵彼微行,爰求柔桑。春日迟迟,采蘩祁祁。女心伤悲,殆及公子同归。

七月流火,八月萑苇。蚕月条桑,取彼斧斨。以伐远扬,猗彼女桑。七月鸣鵙,八月载绩。载玄载黄,我朱孔阳,为公子裳。

四月秀葽,五月鸣蜩。八月其获,十月陨萚。一之日于貉,取彼狐狸,为公子裘。二之日其同,载缵武功。言私其豵,献豜于公。

五月斯螽动股,六月莎鸡振羽。七月在野,八月在宇,九月在户,十月蟋蟀,入我床下。穹窒熏鼠,塞向墐户。嗟我妇子,曰为改岁,入此室处。

六月食郁及薁,七月亨葵及菽。八月剥枣,十月获稻。为此春酒,以介眉寿。七月食瓜,八月断壶,九月叔苴,采荼薪樗。食我农夫。

九月筑场圃,十月纳禾稼。黍稷重穋,禾麻菽麦。嗟我农夫,我稼既同,上入执宫功。昼尔于茅,宵尔索綯,亟其乘屋,其始播百谷。

二之日凿冰冲冲,三之日纳于凌阴。四之日其蚤,献羔祭韭。九月肃霜,十月涤场。朋酒斯飨,曰杀羔羊,跻彼公堂。称彼兕觥:万寿无疆!

鸱鸮

鸱鸮鸱鸮,既取我子,无毁我室。恩斯勤斯,鬻子之闵斯。

迨天之未阴雨,彻彼桑土,绸缪牖户。今女下民,或敢侮予?

予手拮据,予所捋荼。予所蓄租,予口卒瘏,曰予未有室家。

予羽谯谯,予尾翛翛,予室翘翘。风雨所漂摇,予维音哓哓!

东山

我徂东山,慆慆不归。我来自东,零雨其濛。我东曰归,我心西悲。制彼裳衣,勿士行枚。蜎蜎者蠋,烝在桑野。敦彼独宿,亦在车下。

我徂东山,慆慆不归。我来自东,零雨其濛。果臝之实,亦施于宇。伊威在室,蠨蛸在户。町畽鹿场,熠耀宵行。不可畏也,伊可怀也。

我徂东山,慆慆不归。我来自东,零雨其濛。鹳鸣于垤,妇叹于室。洒扫穹窒,我征聿至。有敦瓜苦,烝在栗薪。自我不见,于今三年。

我徂东山,慆慆不归。我来自东,零雨其濛。仓庚于飞,熠耀其羽。之子于归,皇驳其马。亲结其缡,九十其仪。其新孔嘉,其旧如之何?

破斧

既破我斧,又缺我斨。周公东征,四国是皇。哀我人斯,亦孔之将。

既破我斧,又缺我锜。周公东征,四国是遒。哀我人斯,亦孔之嘉。

既破我斧,又缺我銶。周公东征,四国是遒。哀我人斯,亦孔之休。

伐柯

伐柯如何?匪斧不克。取妻如何?匪媒不得。

伐柯伐柯,其则不远。我觏之子,笾豆有践。

九罭

九罭之鱼,鳟鲂。我觏之子,衮衣绣裳。

鸿飞遵渚,公归无所,於女信处。

鸿飞遵陆,公归不复,於女信宿。

是以有衮衣兮,无以我公归兮,无使我心悲兮。

狼跋

狼跋其胡,载疐其尾。公孙硕肤,赤舄几几。

狼疐其尾,载跋其胡。公孙硕肤,德音不瑕?




小雅·鹿鸣之什


鹿鸣

呦呦鹿鸣,食野之苹。我有嘉宾,鼓瑟吹笙。吹笙鼓簧,承筐是将。人之好我,示我周行。

呦呦鹿鸣,食野之蒿。我有嘉宾,德音孔昭。视民不恌,君子是则是效。我有旨酒,嘉宾式燕以敖。

呦呦鹿鸣,食野之芩。我有嘉宾,鼓瑟鼓琴。鼓瑟鼓琴,和乐且湛。我有旨酒,以燕乐嘉宾之心。

四牡

四牡騑騑,周道倭迟。岂不怀归?王事靡盬,我心伤悲。

四牡騑騑,啴啴骆马。岂不怀归?王事靡盬,不遑启处。

翩翩者鵻,载飞载下,集于苞栩。王事靡盬,不遑将父。

翩翩者鵻,载飞载止,集于苞杞。王事靡盬,不遑将母。

驾彼四骆,载骤骎骎。岂不怀归?是用作歌,将母来谂。

皇皇者华

皇皇者华,于彼原隰。駪駪征夫,每怀靡及。

我马维驹,六辔如濡。载驰载驱,周爰咨诹。

我马维骐,六辔如丝。载驰载驱,周爰咨谋。

我马维骆,六辔沃若。载驰载驱,周爰咨度。

我马维骃,六辔既均。载驰载驱,周爰咨询。

常棣

常棣之华,鄂不韡韡。凡今之人,莫如兄弟。

死丧之威,兄弟孔怀。原隰裒矣,兄弟求矣。

脊令在原,兄弟急难。每有良朋,况也永叹。

兄弟阋于墙,外御其务。每有良朋,烝也无戎。

丧乱既平,既安且宁。虽有兄弟,不如友生?

傧尔笾豆,饮酒之饫。兄弟既具,和乐且孺。

妻子好合,如鼓瑟琴。兄弟既翕,和乐且湛。

宜尔室家,乐尔妻帑。是究是图,亶其然乎?

伐木

伐木丁丁,鸟鸣嘤嘤。出自幽谷,迁于乔木。嘤其鸣矣,求其友声。相彼鸟矣,犹求友声。矧伊人矣,不求友生?神之听之,终和且平。

伐木许许,酾酒有藇!既有肥羜,以速诸父。宁适不来,微我弗顾。於粲洒扫,陈馈八簋。既有肥牡,以速诸舅。宁适不来,微我有咎。

伐木于阪,酾酒有衍。笾豆有践,兄弟无远。民之失德,乾餱以愆。有酒湑我,无酒酤我。坎坎鼓我,蹲蹲舞我。迨我暇矣,饮此湑矣。

天保

天保定尔,亦孔之固。俾尔单厚,何福不除?俾尔多益,以莫不庶。

天保定尔,俾尔戬穀。罄无不宜,受天百禄。降尔遐福,维日不足。

天保定尔,以莫不兴。如山如阜,如冈如陵,如川之方至,以莫不增。

吉蠲为饎,是用孝享。禴祠烝尝,于公先王。君曰:卜尔,万寿无疆。

神之吊矣,诒尔多福。民之质矣,日用饮食。群黎百姓,遍为尔德。

如月之恒,如日之升。如南山之寿,不骞不崩。如松柏之茂,无不尔或承。

采薇

采薇采薇,薇亦作止。曰归曰归,岁亦莫止。靡室靡家,玁狁之故。不遑启居,玁狁之故。

采薇采薇,薇亦柔止。曰归曰归,心亦忧止。忧心烈烈,载饥载渴。我戍未定,靡使归聘。

采薇采薇,薇亦刚止。曰归曰归,岁亦阳止。王事靡盬,不遑启处。忧心孔疚,我行不来!

彼尔维何?维常之华。彼路斯何?君子之车。戎车既驾,四牡业业。岂敢定居?一月三捷。

驾彼四牡,四牡骙骙。君子所依,小人所腓。四牡翼翼,象弭鱼服。岂不日戒?玁狁孔棘!

昔我往矣,杨柳依依。今我来思,雨雪霏霏。行道迟迟,载渴载饥。我心伤悲,莫知我哀!

出车

我出我车,于彼牧矣。自天子所,谓我来矣。召彼仆夫,谓之载矣。王事多难,维其棘矣。

我出我车,于彼郊矣。设此旐矣,建彼旄矣。彼旟旐斯,胡不旆旆?忧心悄悄,仆夫况瘁。

王命南仲,往城于方。出车彭彭,旂旐央央。天子命我,城彼朔方。赫赫南仲,玁狁于襄。

昔我往矣,黍稷方华。今我来思,雨雪载途。王事多难,不遑启居。岂不怀归?畏此简书。

喓喓草虫,趯趯阜螽。未见君子,忧心忡忡。既见君子,我心则降。赫赫南仲,薄伐西戎。

春日迟迟,卉木萋萋。仓庚喈喈,采蘩祁祁。执讯获丑,薄言还归。赫赫南仲,玁狁于夷。

杕杜

有杕之杜,有其实。王事靡盬,继嗣我日。日月阳止,女心伤止,征夫遑止。

有杕之杜,其叶萋萋。王事靡盬,我心伤悲。卉木萋止,女心悲止,征夫归止!

陟彼北山,言采其杞。王事靡盬,忧我父母。檀车幝幝,四牡痯痯,征夫不远!

匪载匪来,忧心孔疚。斯逝不至,而多为恤。卜筮偕止,会言近止,征夫迩止!

鱼丽

鱼丽于罶,鲿鲨。君子有酒,旨且多。

鱼丽于罶,鲂鳢。君子有酒,多且旨。

鱼丽于罶,鰋鲤。君子有酒,旨且有。物其多矣,

维其嘉矣!物其旨矣,维其偕矣!物其有矣,维其时矣!

南陔(今佚)

白华(今佚)

华黍(今佚)




小雅·南有嘉鱼之什


南有嘉鱼

南有嘉鱼,烝然罩罩。君子有酒,嘉宾式燕以乐。

南有嘉鱼,烝然汕汕。君子有酒,嘉宾式燕以衎。

南有樛木,甘瓠累之。君子有酒,嘉宾式燕绥之。

翩翩者鵻,烝然来思。君子有酒,嘉宾式燕又思。

南山有台

南山有台,北山有莱。乐只君子,邦家之基。乐只君子,万寿无期。

南山有桑,北山有杨。乐只君子,邦家之光。乐只君子,万寿无疆。

南山有杞,北山有李。乐只君子,民之父母。乐只君子,德音不已。

南山有栲,北山有杻。乐只君子,遐不眉寿。乐只君子,德音是茂。

南山有枸,北山有楰。乐只君子,遐不黄耇。乐只君子,保艾尔后。

由庚(今佚)

崇丘(今佚)

由仪(今佚)

蓼萧

蓼彼萧斯,零露湑兮。既见君子,我心写兮。燕笑语兮,是以有誉处兮。

蓼彼萧斯,零露瀼瀼。既见君子,为龙为光。其德不爽,寿考不忘。

蓼彼萧斯,零露泥泥。既见君子,孔燕岂弟。宜兄宜弟,令德寿岂。

蓼彼萧斯,零露浓浓。既见君子,鞗革忡忡。和鸾雍雍,万福攸同。

湛露

湛湛露斯,匪阳不晞。厌厌夜饮,不醉无归。

湛湛露斯,在彼丰草。厌厌夜饮,在宗载考。

湛湛露斯,在彼杞棘。显允君子,莫不令德。

其桐其椅,其实离离。岂弟君子,莫不令仪。

彤弓

彤弓召兮,受言藏之。我有嘉宾,中心贶之。钟鼓既设,一朝飨之。

彤弓召兮,受言载之。我有嘉宾,中心喜之。钟鼓既设,一朝右之。

彤弓召兮,受言櫜之。我有嘉宾,中心好之。钟鼓既设,一朝酬之。

菁菁者莪

菁菁者莪,在彼中阿。既见君子,乐且有仪。

菁菁者莪,在彼中沚。既见君子,我心则喜。

菁菁者莪,在彼中陵。既见君子,锡我百朋。

泛泛杨舟,载沉载浮。既见君子,我心则休。

六月

六月栖栖,戎车既饬。四牡骙骙,载是常服。玁狁孔炽,我是用急。王于出征,以匡王国。

比物四骊,闲之维则。维此六月,既成我服。我服既成,于三十里。王于出征,以佐天子。

四牡修广,其大有颙。薄伐玁狁,以奏肤公。有严有翼,共武之服。共武之服,以定王国。

玁狁匪茹,整居焦获。侵镐及方,至于泾阳。织文鸟章,白旆央央。元戎十乘,以先启行。

戎车既安,如轾如轩。四牡既佶,既佶且闲。薄伐玁狁,至于大原。文武吉甫,万邦为宪。

吉甫燕喜,既多受祉。来归自镐,我行永久。饮御诸友,炰鳖脍鲤。侯谁在矣?张仲孝友。

采芑

薄言采芑,于彼新田,呈此菑亩。方叔涖止,其车三千。师干之试,方叔率止。乘其四骐,四骐翼翼。路车有奭,簟茀鱼服,钩膺鞗革。

薄言采芑,于彼新田,于此中乡。方叔涖止,其车三千。旂旐央央,方叔率止。约軧错衡,八鸾玱玱。服其命服,朱芾斯皇,有玱葱珩。

鴥彼飞隼,其飞戾天,亦集爰止。方叔涖止,其车三千。师干之试,方叔率止。钲人伐鼓,陈师鞠旅。显允方叔,伐鼓渊渊,振旅阗阗。

蠢尔蛮荆,大邦为仇。方叔元老,克壮其犹。方叔率止,执讯获丑。戎车啴啴,啴啴焞焞,如霆如雷。显允方叔,征伐玁狁,蛮荆来威。

车攻

我车既攻,我马既同。四牡庞庞,驾言徂东。

田车既好,田牡孔阜。东有甫草,驾言行狩。

之子于苗,选徒嚣嚣。建旐设旄,搏兽于敖。

驾彼四牡,四牡奕奕。赤芾金舄,会同有绎。

决拾既佽,弓矢既调。射夫既同,助我举柴。

四黄既驾,两骖不猗。不失其驰,舍矢如破。

萧萧马鸣,悠悠旆旌。徒御不惊,大庖不盈。

之子于征,有闻无声。允矣君子,展也大成。

吉日

吉日维戊,既伯既祷。田车既好,四牡孔阜。升彼大阜,从其群丑。

吉日庚午,既差我马。兽之所同,麀鹿麌麌。漆沮之从,天子之所。

瞻彼中原,其祁孔有。儦儦俟俟,或群或友。悉率左右,以燕天子。

既张我弓,既挟我矢。发彼小豝,殪此大兕。以御宾客,且以酌醴。




小雅·鸿雁之什


鸿雁

鸿雁于飞,肃肃其羽。之子于征,劬劳于野。爰及矜人,哀此鳏寡。

鸿雁于飞,集于中泽。之子于垣,百堵皆作。虽则劬劳,其究安宅?

鸿雁于飞,哀鸣嗷嗷。维此哲人,谓我劬劳。维彼愚人,谓我宣骄。

庭燎

夜如何其?夜未央,庭燎之光。君子至止,鸾声将将。

夜如何其?夜未艾,庭燎晣晣。君子至止,鸾声哕哕。

夜如何其?夜乡晨,庭燎有辉。君子至止,言观其旂。

沔水

沔彼流水,朝宗于海。鴥彼飞隼,载飞载止。嗟我兄弟,邦人诸友。莫肯念乱,谁无父母?

沔彼流水,其流汤汤。鴥彼飞隼,载飞载扬。念彼不迹,载起载行。心之忧矣,不可弭忘。

鴥彼飞隼,率彼中陵。民之讹言,宁莫之惩?我友敬矣,谗言其兴。

鹤鸣

鹤鸣于九皋,声闻于野。鱼潜在渊,或在于渚。乐彼之园,爰有树檀,其下维萚。他山之石,可以为错。

鹤鸣于九皋,声闻于天。鱼在于渚,或潜在渊。乐彼之园,爰有树檀,其下维谷。他山之石,可以攻玉。

祈父

祈父,予王之爪牙。胡转予于恤,靡所止居?

祈父,予王之爪士。胡转予于恤,靡所厎止?

祈父,亶不聪。胡转予于恤?有母之尸饔。

白驹

皎皎白驹,食我场苗。絷之维之,以永今朝。所谓伊人,于焉逍遥?

皎皎白驹,食我场藿。絷之维之,以永今夕。所谓伊人,于焉嘉客?

皎皎白驹,贲然来思。尔公尔侯,逸豫无期?慎尔优游,勉尔遁思。

皎皎白驹,在彼空谷。生刍一束,其人如玉。毋金玉尔音,而有遐心。

黄鸟

黄鸟黄鸟,无集于穀,无啄我粟。此邦之人,不我肯穀。言旋言归,复我邦族。

黄鸟黄鸟,无集于桑,无啄我粱。此邦之人,不可与明。言旋言归,复我诸兄。

黄鸟黄鸟,无集于栩,无啄我黍。此邦之人,不可与处。言旋言归,复我诸父。

我行其野

我行其野,蔽芾其樗。婚姻之故,言就尔居。尔不我畜,复我邦家。

我行其野,言采其蓫。婚姻之故,言就尔宿。尔不我畜,言归斯复。

我行其野,言采其葍。不思旧姻,求尔新特。成不以富,亦祗以异。

斯干

秩秩斯干,幽幽南山。如竹苞矣,如松茂矣。兄及弟矣,式相好矣,无相犹矣。

似续妣祖,筑室百堵,西南其户。爰居爰处,爰笑爰语。

约之阁阁,椓之橐橐。风雨攸除,鸟鼠攸去,君子攸芋。

如跂斯翼,如矢斯棘,如鸟斯革,如翚斯飞,君子攸跻。

殖殖其庭,有觉其楹。哙哙其正,哕哕其冥。君子攸宁。

下莞上簟,乃安斯寝。乃寝乃兴,乃占我梦。吉梦维何?维熊维罴,维虺维蛇。

大人占之:维熊维罴,男子之祥;维虺维蛇,女子之祥。

乃生男子,载寝之床。载衣之裳,载弄之璋。其泣喤喤,朱芾斯皇,室家君王。

乃生女子,载寝之地。载衣之裼,载弄之瓦。无非无仪,唯酒食是议,无父母诒罹。

无羊

谁谓尔无羊?三百维群。谁谓尔无牛?九十其犉。尔羊来思,其角濈濈。尔牛来思,其耳湿湿。

或降于阿,或饮于池,或寝或讹。尔牧来思,何蓑何笠,或负其餱。三十维物,尔牲则具。

尔牧来思,以薪以蒸,以雌以雄。尔羊来思,矜矜兢兢,不骞不崩。麾之以肱,毕来既升。

牧人乃梦,众维鱼矣,旐维旟矣,大人占之;众维鱼矣,实维丰年;旐维旟矣,室家溱溱。




小雅·节南山之什


节南山

节彼南山,维石岩岩。赫赫师尹,民具尔瞻。忧心如惔,不敢戏谈。国既卒斩,何用不监!

节彼南山,有实其猗。赫赫师尹,不平谓何。天方荐瘥,丧乱弘多。民言无嘉,惨莫惩嗟。

尹氏大师,维周之氐;秉国之钧,四方是维。天子是毗,俾民不迷。不吊昊天,不宜空我师。

弗躬弗亲,庶民弗信。弗问弗仕,勿罔君子。式夷式已,无小人殆。琐琐姻亚,则无膴仕。

昊天不佣,降此鞠訩。昊天不惠,降此大戾。君子如届,俾民心阕。君子如夷,恶怒是违。

不吊昊天,乱靡有定。式月斯生,俾民不宁。忧心如酲,谁秉国成?不自为政,卒劳百姓。

驾彼四牡,四牡项领。我瞻四方,蹙蹙靡所骋。

方茂尔恶,相尔矛矣。既夷既怿,如相酬矣。

昊天不平,我王不宁。不惩其心,覆怨其正。

家父作诵,以究王訩。式讹尔心,以畜万邦。

正月

正月繁霜,我心忧伤。民之讹言,亦孔之将。念我独兮,忧心京京。哀我小心,癙忧以痒。

父母生我,胡俾我瘉?不自我先,不自我后。好言自口,莠言自口。忧心愈愈,是以有侮。

忧心惸惸,念我无禄。民之无辜,并其臣仆。哀我人斯,于何从禄?瞻乌爰止?于谁之屋?

瞻彼中林,侯薪侯蒸。民今方殆,视天梦梦。既克有定,靡人弗胜。有皇上帝,伊谁云憎?

谓山盖卑,为冈为陵。民之讹言,宁莫之惩。召彼故老,讯之占梦。具曰予圣,谁知乌之雌雄!

谓天盖高,不敢不局。谓地盖厚,不敢不蹐。维号斯言,有伦有脊。哀今之人,胡为虺蜴?

瞻彼阪田,有菀其特。天之杌我,如不我克。彼求我则,如不我得。执我仇仇,亦不我力。

心之忧矣,如或结之。今兹之正,胡然厉矣?燎之方扬,宁或灭之?赫赫宗周,褒姒灭之!

终其永怀,又窘阴雨。其车既载,乃弃尔辅。载输尔载,将伯助予!

无弃尔辅,员于尔辐。屡顾尔仆,不输尔载。终逾绝险,曾是不意。

鱼在于沼,亦匪克乐。潜虽伏矣,亦孔之炤。忧心惨惨,念国之为虐!

彼有旨酒,又有嘉肴。洽比其邻,婚姻孔云。念我独兮,忧心殷殷。

佌佌彼有屋,蔌蔌方有谷。民今之无禄,天夭是椓。哿矣富人,哀此惸独。

十月之交

十月之交,朔月辛卯。日有食之,亦孔之丑。彼月而微,此日而微;今此下民,亦孔之哀。

日月告凶,不用其行。四国无政,不用其良。彼月而食,则维其常;此日而食,于何不臧。

烨烨震电,不宁不令。百川沸腾,山冢崒崩。高岸为谷,深谷为陵。哀今之人,胡惨莫惩?

皇父卿士,番维司徒,家伯维宰,仲允膳夫,棸子内史,蹶维趣马,禹维师氏。醘妻煽方处。

抑此皇父,岂曰不时?胡为我作,不即我谋?彻我墙屋,田卒汙莱。曰予不戕,礼则然矣。

皇父孔圣,作都于向。择三有事,亶侯多藏。不慭遗一老,俾守我王。择有车马,以居徂向。

黾勉从事,不敢告劳。无罪无辜,谗口嚣嚣。下民之孽,匪降自天。噂沓背憎,职竞由人。

悠悠我里,亦孔之痗。四方有羡,我独居忧。民莫不逸,我独不敢休。天命不彻,我不敢效我友自逸。

雨无正

浩浩昊天,不骏其德。降丧饥馑,斩伐四国。旻天疾威,弗虑弗图。舍彼有罪,既伏其辜。若此无罪,沦胥以铺。

周宗既灭,靡所止戾。正大夫离居,莫知我勚。三事大夫,莫肯夙夜。邦君诸侯,莫肯朝夕。庶曰式臧,覆出为恶。

如何昊天,辟言不信。如彼行迈,则靡所臻。凡百君子,各敬尔身。胡不相畏,不畏于天?

戎成不退,饥成不遂。曾我暬御,惨惨日瘁。凡百君子,莫肯用讯。听言则答,谮言则退。

哀哉不能言,匪舌是出,维躬是瘁。哿矣能言,巧言如流,俾躬处休!

维曰予仕,孔棘且殆。云不何使,得罪于天子;亦云可使,怨及朋友。

谓尔迁于王都。曰予未有室家。鼠思泣血,无言不疾。昔尔出居,谁从作尔室?

小旻

旻天疾威,敷于下土。谋犹回遹,何日斯沮?谋臧不从,不臧覆用。我视谋犹,亦孔之邛。

潝潝訿訿,亦孔之哀。谋之其臧,则具是违。谋之不臧,则具是依。我视谋犹,伊于胡厎。

我龟既厌,不我告犹。谋夫孔多,是用不集。发言盈庭,谁敢执其咎?如匪行迈谋,是用不得于道。

哀哉为犹,匪先民是程,匪大犹是经。维迩言是听,维迩言是争。如彼筑室于道谋,是用不溃于成。

国虽靡止,或圣或否。民虽靡膴,或哲或谋,或肃或艾。如彼泉流,无沦胥以败。

不敢暴虎,不敢冯河。人知其一,莫知其他。战战兢兢,如临深渊,如履薄冰。

小宛

宛彼鸣鸠,翰飞戾天。我心忧伤,念昔先人。明发不寐,有怀二人。

人之齐圣,饮酒温克。彼昏不知,壹醉日富。各敬尔仪,天命不又。

中原有菽,庶民采之。螟蛉有子,蜾蠃负之。教诲尔子,式穀似之。

题彼脊令,载飞载鸣。我日斯迈,而月斯征。夙兴夜寐,毋忝尔所生。

交交桑扈,率场啄粟。哀我填寡,宜岸宜狱。握粟出卜,自何能穀?

温温恭人,如集于木。惴惴小心,如临于谷。战战兢兢,如履薄冰。

小弁

弁彼鸴斯,归飞提提。民莫不穀,我独于罹。何辜于天?我罪伊何?心之忧矣,云如之何?

踧踧周道,鞫为茂草。我心忧伤,惄焉如捣。假寐永叹,维忧用老。心之忧矣,疢如疾首。

维桑与梓,必恭敬止。靡瞻匪父,靡依匪母。不属于毛?不罹于里?天之生我,我辰安在?

菀彼柳斯,鸣蜩嘒嘒,有漼者渊,萑苇淠淠。譬彼舟流,不知所届,心之忧矣,不遑假寐。

鹿斯之奔,维足伎伎。雉之朝雊,尚求其雌。譬彼坏木,疾用无枝。心之忧矣,宁莫之知?

相彼投兔,尚或先之。行有死人,尚或墐之。君子秉心,维其忍之。心之忧矣,涕既陨之。

君子信谗,如或酬之。君子不惠,不舒究之。伐木掎矣,析薪扡矣。舍彼有罪,予之佗矣。

莫高匪山,莫浚匪泉。君子无易由言,耳属于垣。无逝我梁,无发我笱。我躬不阅,遑恤我后。

巧言

悠悠昊天,曰父母且。无罪无辜,乱如此幠。昊天已威,予慎无罪。昊天大幠,予慎无辜。

乱之初生,僭始既涵。乱之又生,君子信谗。君子如怒,乱庶遄沮。君子如祉,乱庶遄已。

君子屡盟,乱是用长。君子信盗,乱是用暴。盗言孔甘,乱是用餤。匪其止共,维王之邛。奕奕寝庙,君子作之。秩秩大猷,圣人莫之。

他人有心,予忖度之。跃跃毚兔,遇犬获之。荏染柔木,君子树之。往来行言,心焉数之。蛇蛇硕言,出自口矣。巧言如簧,颜之厚矣。

彼何人斯?居河之麋。无拳无勇,职为乱阶。既微且尰,尔勇伊何?为犹将多,尔居徒几何?

何人斯

彼何人斯?其心孔艰。胡逝我梁,不入我门?伊谁云从?维暴之云。

二人从行,谁为此祸?胡逝我梁,不入唁我?始者不如今,云不我可

彼何人斯?胡逝我陈?我闻其声,不见其身。不愧于人?不畏于天?

彼何人斯?其为飘风。胡不自北?胡不自南?胡逝我梁?絺搅我心。

尔之安行,亦不遑舍。尔之亟行,遑脂尔车。壹者之来,云何其盱。

尔还而入,我心易也。还而不入,否难知也。壹者之来,俾我絺也。

伯氏吹埙,仲氏吹篪。及尔如贯,谅不我郑出此三物,以诅尔斯。

为鬼为蜮,则不可得。有靦面目,视人罔极。作此好歌,以极反侧。

巷伯

萋兮斐兮,成是贝锦。彼谮人者,亦已大甚!

哆兮侈兮,成是南箕。彼谮人者,谁适与谋。

缉缉翩翩,谋欲谮人。慎尔言也,谓尔不信。

捷捷幡幡,谋欲谮言。岂不尔受?既其女迁。

骄人好好,劳人草草。苍天苍天,视彼骄人,矜此劳人。

彼谮人者,谁适与谋?取彼谮人,投畀豺虎。豺虎不食,投畀有北。有北不受,投畀有昊!

杨园之道,猗于亩丘。寺人孟子,作为此诗。凡百君子,敬而听之。




小雅·谷风之什


谷风

习习谷风,维风及雨。将恐将惧,维予与女。将安将乐,女转弃予。

习习谷风,维风及颓。将恐将惧,置予于怀。将安将乐,弃予如遗。

习习谷风,维山崔嵬。无草不死,无木不萎。忘我大德,思我小怨。

蓼莪

蓼蓼者莪,匪莪伊蒿。哀哀父母,生我劬劳。

蓼蓼者莪,匪莪伊蔚。哀哀父母,生我劳瘁。

瓶之罄矣,维罍之耻。鲜民之生,不如死之久矣。无父何怙?无母何恃?出则衔恤,入则靡至。

父兮生我,母兮鞠我。抚我畜我,长我育我,顾我复我,出入腹我。欲报之德。昊天罔极!

南山烈烈,飘风发发。民莫不穀,我独何害!南山律律,飘风弗弗。民莫不穀,我独不卒!

大东

有饛簋飧,有捄棘匕。周道如砥,其直如矢。君子所履,小人所视。眷言顾之,潸焉出涕。

小东大东,杼柚其空。纠纠葛屦,可以履霜。佻佻公子,行彼周行。既往既来,使我心疚。

有冽氿泉,无浸获薪。契契寤叹,哀我惮人。薪是获薪,尚可载也。哀我惮人,亦可息也。

东人之子,职劳不来。西人之子,粲粲衣服。舟人之子,熊罴是裘。私人之子,百僚是试。

或以其酒,不以其浆。鞙鞙佩璲,不以其长。维天有汉,监亦有光。跂彼织女,终日七襄。

虽则七襄,不成报章。彼牵牛,不以服箱。东有启明,西有长?庚。有捄天毕,载施之行。

维南有箕,不可以簸扬。维北有斗,不可以挹酒浆。维南有箕,载翕其舌。维北有斗,西柄之揭。

四月

四月维夏,六月徂署。先祖匪人,胡宁忍予?

秋日凄凄,百卉具腓。乱离瘼矣,爰其适归?

冬日烈烈,飘风发发。民莫不穀,我独何害?

山有嘉卉,侯栗侯梅。废为残贼,莫知其尤!

相彼泉水,载清载浊。我日构祸,曷云能穀?

滔滔江汉,南国之纪。尽瘁以仕,宁莫我有?

匪鹑匪鸢,翰飞戾天。匪鳣匪鲔,潜逃于渊。

山有蕨薇,隰有杞桋。君子作歌,维以告哀。

北山

陟彼北山,言采其杞。偕偕士子,朝夕从事。王事靡盬,忧我父母。

溥天之下,莫非王土;率土之滨,莫非王臣。大夫不均,我从事独贤。

四牡彭彭,王事傍傍。嘉我未老,鲜我方将。旅力方刚,经营四方。

或燕燕居息,或尽瘁事国;或息偃在床,或不已于行。

或不知叫号,或惨惨劬劳;或栖迟偃仰,或王事鞅掌。

或湛乐饮酒,或惨惨畏咎;或出入风议,或靡事不为。

无将大车

无将大车,祇自尘兮。无思百忧,祇自疧兮。

无将大车,维尘冥冥。无思百忧,不出于颎。

无将大车,维尘雍兮。无思百忧,祇自重兮。

小明

明明上天,照临下土。我征徂西,至于艽野。二月初吉,载离寒暑。心之忧矣,其毒

大苦。念彼共人,涕零如雨。岂不怀归?畏此罪罟!

昔我往矣,日月方除。曷云其还?岁聿云莫。念我独兮,我事孔庶。心之忧矣,惮我

不暇。念彼共人,眷眷怀顾!岂不怀归?畏此谴怒。

昔我往矣,日月方奥。曷云其还?政事愈蹙。岁聿云莫,采萧获菽。心之忧矣,自诒

伊戚。念彼共人,兴言出宿。岂不怀归?畏此反覆。

嗟尔君子,无恒安处。靖共尔位,正直是与。神之听之,式穀以女。

嗟尔君子,无恒安息。靖共尔位,好是正直。神之听之,介尔景福。

鼓钟

鼓钟将将,淮水汤汤,忧心且伤。淑人君子,怀允不忘。

鼓钟喈喈,淮水湝湝,忧心且悲。淑人君子,其德不回。

鼓钟伐鼛,淮有三洲,忧心且妯。淑人君子,其德不犹。

鼓钟钦钦,鼓瑟鼓琴,笙磬同音。以雅以南,以龠不僭。

楚茨

楚楚者茨,言抽其棘,自昔何为?我蓺黍稷。我黍与与,我稷翼翼。

我仓既盈,我庾维亿。以为酒食,以享以祀,以妥以侑,以介景福。

济济跄跄,絜尔牛羊,以往烝尝。或剥或亨,或肆或将。祝祭于祊,祀事孔明。先祖

是皇,神保是飨。孝孙有庆,报以介福,万寿无疆!

执爨 ,为俎孔硕,或燔或炙。君妇莫莫,为豆孔庶。为宾为客,献酬交错。礼仪卒

度,笑语卒获。神保是格,报以介福,万寿攸酢!

我孔 矣,式礼莫愆。工祝致告,徂赉孝孙。苾芬孝祀,神嗜饮食。卜尔百福,如几

如式。既齐既稷,既匡既敕。永锡尔极,时万时亿!

礼仪既备,钟鼓既戒,孝孙徂位,工祝致告,神具醉止,皇尸载起。鼓钟送尸,神保

聿归。诸宰君妇,废彻不迟。诸父兄弟,备言燕私。

乐具入奏,以绥后禄。尔肴既将,莫怨具庆。既醉既饱,小大稽首。神嗜饮食,使君

寿考。孔惠孔时,维其尽之。子子孙孙,勿替引之!

信南山

信彼南山,维禹甸之。畇畇原隰,曾孙田之。我疆我理,南东其亩。

上天同云。雨雪雰雰,益之以霡霂。既优既渥,既沾既足。生我百谷。

疆场翼翼,黍稷彧彧。曾孙之穑,以为酒食。畀我尸宾,寿考万年。

中田有庐,疆场有瓜。是剥是菹,献之皇祖。曾孙寿考,受天之祜。

祭以清酒,从以骍牡,享于祖考。执其鸾刀,以启其毛,取其血膋。

是烝是享,苾苾芬芬。祀事孔明,先祖是皇。报以介福。万寿无疆。

 

 
小雅·甫田之什


甫田 

倬彼甫田,岁取十千。我取其陈,食我农人。自古有年。今适南亩,或耘或耔。黍稷薿薿,攸介攸止,烝我髦士。

以我齐明,与我牺羊,以社以方。我田既臧,农夫之庆。琴瑟击鼓,以御田祖。以祈甘雨,以介我稷黍,以穀我士女。

曾孙来止,以其妇子。馌彼南亩,田畯至喜。攘其左右,尝其旨否。禾易长亩,终善且有。曾孙不怒,农夫克敏。

曾孙之稼,如茨如梁。曾孙之庾,如坻如京。乃求千斯仓,乃求万斯箱。黍稷稻粱,农夫之庆。报以介福,万寿无疆。

大田

大田多稼,既种既戒,既备乃事。以我覃耜,俶载南亩。播厥百谷,既庭且硕,曾孙是若。

既方既皂,既坚既好,不稂不莠。去其螟螣,及其蟊贼,无害我田稚。田祖有神,秉畀炎火。

有渰萋萋,兴雨祈祈。雨我公田,遂及我私。彼有不获稚,此有不敛穧,彼有遗秉,此有滞穗,伊寡妇之利。

曾孙来止,以其妇子。馌彼南亩,田畯至喜。来方禋祀,以其骍黑,与其黍稷。以享以祀,以介景福。

瞻彼洛矣

瞻彼洛矣,维水泱泱。君子至止,福禄如茨。韎韐有奭,以作六师。

瞻彼洛矣,维水泱泱。君子至止,鞸琫有珌。君子万年,保其家室。

瞻彼洛矣,维水泱泱。君子至止,福禄既同。君子万年,保其家邦。

裳裳者华

裳裳者华,其叶湑□兮。我觏之子,我心写兮。我心写兮,是以有誉处兮。

裳裳者华,芸其黄矣。我觏之子,维其有章矣。维其有章矣,是以有庆矣。

裳裳者华,或黄或白。我觏之子,乘其四骆。乘其四骆,六辔沃若。

左之左之,君子宜之。右之右之,君子有之。维其有之,是以似之。

桑扈

交交桑扈,有莺其羽。君子乐胥,受天之祜。

交交桑扈,有莺其领。君子乐胥,万邦之屏。

之屏之翰,百辟为宪。不戢不难,受福不那。

兕觥其觩,旨酒思柔。彼交匪敖,万福来求。

鸳鸯

鸳鸯于飞,毕之罗之。君子万年,福禄宜之。

鸳鸯在梁,戢其左翼。君子万年,

宜其遐福。乘马在厩,摧之秣之。君子万年,福禄艾之。

乘马在厩,秣之摧之。君子万年,福禄绥之。

頍弁

有頍者弁,实维伊何?尔酒既旨,尔肴既嘉。岂伊异人?兄弟匪他。茑与女萝,施于松柏。未见君子,忧心奕奕;既见君子,庶几说怿。

有頍者弁,实维何期?尔酒既旨,尔肴既时。岂伊异人?兄弟具来。茑与女萝,施于松上。未见君子,忧心怲怲;既见君子,庶几有臧。

有頍者弁,实维在首。尔酒既旨,尔肴既阜。岂伊异人?兄弟甥舅。如彼雨雪,先集维霰。死丧无日,无几相见。乐酒今夕,君子维宴。

车舝

间关车之舝兮,思娈季女逝兮。匪饥匪渴,德音来括。虽无好友?式燕且喜。

依彼平林,有集维鷮。辰彼硕女,令德来教。式燕且誉,好尔无射。

虽无旨酒?式饮庶几。虽无嘉肴?式食庶几。虽无德与女?式歌且舞?

陟彼高冈,析其柞薪。析其柞薪,其叶湑兮。鲜我觏尔,我心写兮。

高山仰止,景行行止。四牡騑騑,六辔如琴。觏尔新婚,以慰我心。

青蝇

营营青蝇,止于樊。岂弟君子,无信谗言。

营营青蝇,止于棘。谗人罔极,交乱四国。

营营青蝇,止于榛。谗人罔极,构我二人。

宾之初筵

宾之初筵,左右秩秩。笾豆有楚,殽核维旅。酒既和旨,饮酒孔偕。钟鼓既设,举酬逸逸。大侯既抗,弓矢斯张。射夫既同,献尔发功。

发彼有的,以祈尔爵。籥舞笙鼓,乐既和奏。烝衎烈祖,以洽百礼。百礼既至,有壬有林。锡尔纯嘏,子孙其湛。其湛曰乐,各奏尔能。宾载手仇,室人入又。酌彼康爵,以奏尔时。

宾之初筵,温温其恭。其未醉止,威仪反反。曰既醉止,威仪幡幡。舍其坐迁,屡舞仙仙。其未醉止,威仪抑抑。曰既醉止,威仪抑抑。是曰既醉,不知其秩。

宾既醉止,载号载呶。乱我笾豆,屡舞僛僛。是曰既醉,不知其邮。侧弁之俄,屡舞傞傞。既醉而出,并受其福。

醉而不出,是谓伐德。饮酒孔嘉,维其令仪。凡此饮酒,或醉或否。既立之监,或佐之史。彼醉不臧,不醉反耻。

式勿从谓,无俾大怠。匪言勿言,匪由勿语。由醉之言,俾出童羖。三爵不识,矧敢多又。




小雅·鱼藻之什


鱼藻

鱼在在藻,有颁其首。王在在镐,岂乐饮酒。

鱼在在藻,有莘其尾。王在在镐,饮酒乐岂。

鱼在在藻,依于其蒲。王在在镐,有那其居。

采菽

采菽采菽,筐之莒之。君子来朝,何锡予之?虽无予之?路车乘马。又何予之?玄衮及黼。

觱沸槛泉,言采其芹。君子来朝,言观其旂。其旂淠淠,鸾声嘒嘒。载骖载驷,君子所届。

赤芾在股,邪幅在下。彼交匪纾,天子所予。乐只君子,天子命之。乐只君子,福禄申之。

维柞之枝,其叶蓬蓬。乐只君子,殿天子之邦。乐只君子,万福攸同。平平左右,亦是率从。

汎汎杨舟,绋纚维之。乐只君子,天子葵之。乐只君子,福禄膍之。优哉游哉,亦是戾矣。

角弓

骍骍角弓,翩其反矣。兄弟婚姻,无胥远矣。

尔之远矣,民胥然矣。尔之教矣,民胥效矣。

此令兄弟,绰绰有裕。不令兄弟,交相为愈。

民之无良,相怨一方。受爵不让,至于已斯亡。

老马反为驹,不顾其后。如食宜饇,如酌孔取。

毋教猱升木,如涂涂附。君子有徽猷,小人与属。

雨雪瀌瀌,见晛曰消。莫肯下遗,式居娄骄。

雨雪浮浮,见晛曰流。如蛮如髦,我是用忧。

菀柳

有菀者柳,不尚息焉。上帝甚蹈,无自暱焉。俾予靖之,后予极焉。

有菀者柳,不尚愒焉。上帝甚蹈,无自瘵焉。俾予靖之,后予迈焉。

有鸟高飞,亦傅于天。彼人之心,于何其臻。曷予靖之,居以凶矜。

都人士

彼都人士,狐裘黄黄。其容不改,出言有章。行归于周,万民所望。

彼都人士,台笠缁撮。彼君子女,绸直如发。我不见兮,我心不说。

彼都人士,充耳琇实。彼君子女,谓之尹吉。我不见兮,我心苑结。

彼都人士,垂带而厉。彼君子女,卷发如虿。我不见兮,言从之迈。

匪伊垂之,带则有余。匪伊卷之,发则有旟。我不见兮,云何盱矣。

采绿

终朝采绿,不盈一匊。予发曲局,薄言归沐。

终朝采蓝,不盈一襜。五日为期,六日不詹。

之子于狩,言韔其弓。之子于钓,言纶之绳。

其钓维何?维鲂及鱮。维鲂及鱮,薄言观者。

黍苗

芃芃黍苗,阴雨膏之。悠悠南行,召伯劳之。

我任我辇,我车我牛。我行既集,盖云归哉。

我徒我御,我师我旅。我行既集,盖云归处。

肃肃谢功,召伯营之。烈烈征师,召伯成之。

原隰既平,泉流既清。召伯有成,王心则宁。

隰桑

隰桑有阿,其叶有难。既见君子,其乐如何。

隰桑有阿,其叶有沃。既见君子,云何不乐。

隰桑有阿,其叶有幽。既见君子,德音孔胶。

心乎爱矣,遐不谓矣?中心藏之,何日忘之!

白华

白华菅兮,白茅束兮。之子之远,俾我独兮。

英英白云,露彼菅茅。天步艰难,之子不犹。

滮池北流,浸彼稻田。啸歌伤怀,念彼硕人。

樵彼桑薪,卬烘于煁。维彼硕人,实劳我心。

鼓钟于宫,声闻于外。念子懆懆,视我迈迈。

有鹙在梁,有鹤在林。维彼硕人,实劳我心。

鸳鸯在梁,戢其左翼。之子无良,二三其德。

有扁斯石,履之卑兮。之子之远,俾我疧兮。

绵蛮

绵蛮黄鸟,止于丘阿。道之云远,我劳如何。饮之食之,教之诲之。命彼后车,谓之载之。

绵蛮黄鸟,止于丘隅。岂敢惮行,畏不能趋。饮之食之。教之诲之。命彼后车,谓之载之。

绵蛮黄鸟,止于丘侧。岂敢惮行,畏不能极。饮之食之,教之诲之。命彼后车,谓之载之。

瓠叶

幡幡瓠叶,采之亨之。君子有酒,酌言尝之。

有兔斯首,炮之燔之。君子有酒,酌言献之。

有兔斯首,燔之炙之。君子有酒,酌言酢之。

有兔斯首,燔之炮之。君子有酒,酌言酬之。

渐渐之石

渐渐之石,维其高矣。山川悠远,维其劳矣。武人东征,不遑朝矣。

渐渐之石,维其卒矣。山川悠远,曷其没矣?武人东征,不遑出矣。

有豕白蹢,烝涉波矣。月离于毕,俾滂沱矣。武人东征,不皇他矣。

苕之华

苕之华,芸其黄矣。心之忧矣,维其伤矣!

苕之华,其叶青青。知我如此,不如无生!

牂羊坟首,三星在罶。人可以食,鲜可以饱!

何草不黄

何草不黄?何日不行?何人不将?经营四方。

何草不玄?何人不矜?哀我征夫,独为匪民。

匪兕匪虎,率彼旷野。哀我征夫,朝夕不暇。

有芃者狐,率彼幽草。有栈之车,行彼周道。



 
大雅·文王之什


文王

文王在上,於昭于天,周虽旧邦,其命维新。

有周不显,帝命不时。文王陟降,在帝左右。

亹亹文王,令聞不已。陳錫哉周,侯文王孙子。

文王孙子,本支百世。凡周之士,不显亦世。

世之不显,厥猶翼翼。思皇多士,生此王国。

王国克生,维周之楨。济济多士,文王以寧。

穆穆文王,於緝熙敬止。假哉天命,有商孙子。

商之孙子,其麗不億。上帝既命,侯于周服。

侯服于周,天命靡常。殷士膚敏,祼將于京。

厥作祼將,常服黼冔。王之藎臣,无念尔祖。

无念尔祖,聿修厥德。永言配命,自求多福。

殷之未喪師,克配上帝。宜鑒于殷,駿命不易。

命之不易,无遏尔躬。宣昭義問,有虞殷自天。

上天之载,无声无臭。儀刑文王,万邦作孚。

大明

明明在下,赫赫在上。天难忱斯,不易维王。天位殷適,使不挾四方。

摯仲氏任,自彼殷商;來嫁于周,曰嬪于京。乃及王季,维德之行。大任有身,生此文王。

维此文王,小心翼翼。昭事上帝,聿懷多福。厥德不回,以受方国。

天監在下,有命既集。文王初载,天作之合。在洽之陽,在渭之涘。文王嘉止,大邦有子。

大邦有子,伣天之妹。文定厥祥,親迎于渭。造舟为梁,不显其光。

有命自天,命此文王。于周于京。纘女维莘,长子维行。篤生武王,保右命尔,燮伐大商。

殷商之旅,其會如林。矢于牧野,维予侯興。上帝臨女,无貳尔心!

牧野洋洋,檀車煌煌,駟騵彭彭。维師尚父,时维鷹揚;涼彼武王,肆伐大商,會朝清明。

绵

绵绵瓜瓞。民之初生,自土沮漆。古公亶父,陶复陶穴,未有家室。

古公亶父,來朝走马,率西水滸,至于岐下。爰及姜女,聿來胥宇。

周原膴膴,堇荼如飴。爰始爰謀,爰契我龜。曰止曰时,筑室于茲。

乃慰乃止,乃左乃右,乃疆乃理,乃宣乃畝。自西徂東,周爰執事。

乃召司空,乃召司徒,俾立室家。其繩則直,縮版以载,作庙翼翼。

捄之陾陾,度之薨薨,筑之登登,削屢馮馮。百堵皆興,鼛鼓弗勝。

乃立皋門,皋門有伉;乃立应門,应門將將。乃立冢土,戎醜攸行。

肆不殄厥慍,亦不隕厥問,柞棫拔矣,行道兌矣。混夷駾矣,维其喙矣。

虞芮質厥成,文王厥厥生。予曰有疏附,予曰有先後,予曰有奔奏,予曰有御侮。

棫樸

芃芃棫樸,薪之槱之。济济辟王,左右趣之。

济济辟王,左右奉璋。奉璋峨峨,髦士攸宜。

淠彼涇舟,烝徒楫之。周王于邁,六師及之。

倬彼云汉,为章于天。周王寿考,遐不作人?

追琢其章,金玉其相。勉勉我王,綱紀四方。

旱麓

瞻彼旱麓,榛楛济济。豈弟君子,干祿豈弟。

瑟彼玉瓚,黃流在中。豈弟君子,福祿攸降。

鳶飞戾天,鱼躍于淵。豈弟君子,遐不作人?

清酒既载,騂牡既備。以享以祀,以介景福。

瑟彼柞棫,民所燎矣。豈弟君子,神所勞矣。

莫莫葛櫑,施于條枚。豈弟君子,求福不回。

思齐

思齐大任,文王之母。思媚周姜,京室之婦。大姒嗣徽音,則百斯男。

惠于宗公,神罔时怨,神罔时恫。刑于寡妻,至于兄弟,以御于家邦。

雍雍在宮,肅肅在庙。不显亦臨,无射亦保。

肆戎疾不殄,烈假不遐。不聞亦式,不諫亦入。

肆成人有德,小子有造。古人之无斁,譽髦斯士。

皇矣

皇矣上帝,臨下有赫;監觀四方,求民之莫。

维此二国,其政不獲;维彼四国,爰究爰度。

上帝耆之,憎其式廓。乃眷西顧,此维与宅。

作之屏之,其菑其翳;修之平之,其灌其栵;

啟之辟之,其檉其椐;攘之剔之,其檿其柘。

帝遷明德,串夷载路。天立厥配,受命既固。

帝省其山,柞棫斯拔,松柏斯兌。帝作邦作對,自大伯王季。

维此王季,因心則友。則友其兄,則篤其慶,载錫之光。受祿无喪,奄有四方。

维此王季,帝度其心,貊其德音。其德克明,克明克類,克长克君。

王此大邦,克順克比。比于文王,其德靡悔。既受帝祉,施于孙子。

帝謂文王:无然畔援,无然歆羨,誕先登于岸。密人不恭,敢距大邦,侵阮徂共。

王赫斯怒,爰整其旅,以按徂旅,以篤周祜,以對于天下。

依其在京,侵自阮疆,陟我高岡。无矢我陵,我陵我阿;无飲我泉,我泉我池!

度其鮮原,居岐之陽,在渭之將。万邦之方,下民之王。

帝謂文王:予懷明德,不大声以色,不长夏以革,不識不知,順帝之則。

帝謂文王:詢尔仇方,同尔兄弟。以尔鉤援,与尔臨沖,以伐崇墉。

臨沖閑閑,崇墉言言,執訊連連,攸馘安安。是類是祃,是致是附,四方以无侮。

臨沖茀茀,崇墉仡仡,是伐是肆,是絕是忽,四方以无拂。

灵台

經始灵台,經之營之。庶民攻之,不日成之。經始勿亟,庶民子來。

王在灵囿,麀鹿攸伏;麀鹿濯濯,白鸟翯翯。王在灵沼,於牣鱼躍。

虡業维樅,賁鼓维鏞。於論鼓鐘,於乐辟癰。

於論鼓鐘,於乐辟癰。鼉鼓逢逢,矇瞍奏公。

下武

下武维周,世有哲王。三後在天,王配于京。

王配于京,世德作求。永言配命,成王之孚。

成王之孚,下土之式。永言孝思,孝思维則。

媚茲一人,应侯順德。永言孝思,昭哉嗣服。

昭茲來許,繩其祖武。於万斯年,受天之祜。

受天之祜,四方來賀。於万斯年,不遐有佐。

文王有声

文王有声,遹駿有声,遹求厥寧,遹觀厥成。文王烝哉!

文王受命,有此武功;既伐于崇,作邑于丰。文王烝哉!

筑城伊淢,作丰伊匹,匪棘其欲,遹追來孝。王后烝哉!

王公伊濯,维丰之垣。四方攸同,王后维翰。王后烝哉!

丰水東注,维禹之績。四方攸同,皇王维辟。皇王烝哉!

鎬京辟癰,自西自東,自南自北,无思不服。皇王烝哉!

考卜维王,宅是鎬京。维龜正之,武王成之。武王烝哉!

丰水有芑,武王豈不仕?詒厥孙謀,以燕翼子。武王烝哉!




大雅·生民之什


生民

厥初生民,时维姜嫄。生民如何?克禋克祀,以弗无子。履帝武敏歆,攸介攸止;载震载夙,载生载育,时维后稷。

誕彌厥月,先生如达。不坼不副,无菑无害。以赫厥灵,上帝不寧。不康禋祀,居然生子。

誕寘之隘巷,牛羊腓字之。誕寘之平林,會伐平林;誕寘之寒冰,鸟覆翼之。鸟乃去矣,后稷呱矣。实覃实訏,厥声载路。

誕实匍匐,克岐克嶷,以就口食。蓺之荏菽,荏菽旆旆,禾役穟穟,麻麥幪幪,瓜瓞唪唪。

誕后稷之穡,有相之道。茀厥丰草,種之黃茂。实方实苞,实種实褎,实发实秀,实堅实好,实穎实栗,即有邰家室。

誕降嘉種,维秬维秠,维穈维芑。恒之秬秠,是獲是畝;恒之穈芑,是任是負,以归肇祀。

誕我祀如何?或舂或揄,或簸或蹂;釋之叟叟,烝之浮浮。载謀载惟,取蕭祭脂,取羝以軷,载燔载烈。以興嗣歲。

卬盛于豆,于豆于登。其香始升,上帝居歆。胡臭亶时。后稷肇祀,庶无罪悔,以迄于今。

行葦

敦彼行葦,牛羊勿踐履。方苞方體,维葉泥泥。戚戚兄弟,莫遠具尔。或肆之筵,或授之几。

肆筵設席,授几有緝御。或獻或酢,洗爵奠斝。醓醢以薦,或幡或炙。嘉肴脾臄,或歌或咢。

敦弓既堅,四鍭既鈞;舍矢既均,序賓以賢。敦弓既句,既挾四鍭;四鍭如樹,序賓以不侮。

曾孙维主,酒醴维醹,酌以大斗,以祈黃耇。黃耇台背,以引以翼。寿考维祺,以介景福。

既醉

既醉以酒,既飽以德。君子万年,介尔景福。

既醉以酒,尔殽既將。君子万年,介尔昭明。

昭明有融,高朗令終。令終有俶,公尸嘉告。

其告维何?籩豆靜嘉。朋友攸攝,攝以威儀。

威儀孔时,君子有孝子。孝子不匱,永錫尔類。

其類维何?室家之壸。君子万年,永錫祚胤。

其胤维何?天被尔祿。君子万年,景命有仆。

其仆维何?厘尔女士。厘尔女士,從以孙子。

鳧鹥

鳧鹥在涇,公尸來燕來寧。尔酒既清,尔殽既馨。公尸燕飲,福祿來成。

鳧鹥在沙,公尸來燕來宜。尔酒既多,尔殽既嘉。公尸燕飲,福祿來为。

鳧鹥在渚,公尸來燕來處。尔酒既湑,尔殽伊脯。公尸燕飲,福祿來下。

鳧鹥在潀,公尸來燕來宗。既燕于宗,福祿攸降。公尸燕飲,福祿來崇。

鳧鹥在亹,公尸來止熏熏。旨酒欣欣,燔炙芬芬。公尸燕飲,无有後艱。

假乐

假乐君子,显显令德。宜民宜人,受祿于天。保右命之,自天申之。

干祿百福,子孙千億。穆穆皇皇,宜君宜王。不愆不忘,率由旧章。

威儀抑抑,德音秩秩。无怨无惡,率由群匹。受福无疆,四方之綱。

之綱之紀,燕及朋友。百辟卿士,媚于天子。不解于位,民之攸墍。《假乐》

公刘

篤公刘,匪居匪康,乃场乃疆,乃積乃倉。乃裹糇糧,于橐于囊,思輯用光。弓矢斯张,干戈戚揚,爰方啟行。

篤公刘,于胥斯原。既庶既繁。既順乃宣,而无詠嘆。陟則在巘,复降在原。何以舟之?维玉及瑤,鞞琫容刀。

篤公刘,逝彼百泉,瞻彼溥原。乃陟南岡,乃覯于京。京師之野,于时處處,于时廬旅。于时言言,于时語語。

篤公刘,于京斯依。蹌蹌济济,俾筵俾几。既登乃依,乃造其曹;執豕于牢,酌之用匏。食之飲之,君之宗之。

篤公刘,既溥既长。既景乃岡,相其陰陽,觀其流泉。其軍三單,度其隰原,徹田为糧。度其夕陽,豳居允荒。

篤公刘,于豳斯館。涉渭为亂,取厲取鍛。止基乃理,爰眾爰有。夾其皇澗,溯其過澗。止旅乃密,芮鞫之即。

泂酌

泂酌彼行潦,挹彼注茲,可以餴饎。豈弟君子,民之父母。

泂酌彼行潦,挹彼注茲,可以濯罍。豈弟君子,民之攸归。

泂酌彼行潦,挹彼注茲,可以濯溉。豈弟君子,民之攸墍。

卷阿

有卷者阿,飄風自南。豈弟君子,來游來歌,以矢其音。

伴奐尔游矣,优游尔休矣。豈弟君子,俾尔彌尔性,似先公酋矣。

尔土宇皈章,亦孔之厚矣。豈弟君子,俾尔彌尔性,百神尔主矣。

尔受命长矣,茀祿尔康矣。豈弟君子,俾尔彌尔性,純嘏尔常矣。

有馮有翼,有孝有德,以引以翼。豈弟君子,四方为則。

颙颙卬卬,如圭如璋,令聞令望。豈弟君子,四方为綱。

鳳皇于飞,翙翙其羽,亦集爰止。藹藹王多吉士,维君子使,媚于天子。

鳳皇于飞,翙翙其羽,亦傅于天。藹藹王多吉人,维君子命,媚于庶人。

鳳皇鳴矣,于彼高岡。梧桐生矣,于彼朝陽。菶菶萋萋,雍雍喈喈。

君子之車,既庶且多;君子之马,既閑且馳。矢詩不多,维以遂歌。

民勞

民亦勞止,汔可械。惠此中国,以綏四方。无縱詭隨,以謹无良。式遏寇虐,憯不畏明。柔遠能邇,以定我王。

民亦勞止,汔可小休。惠此中国,以为民逑。无縱詭隨,以謹惛怓。式遏寇虐,无俾民憂。无棄尔勞,以为王休。

民亦勞止,汔可小息。惠此京師,以綏四国。无縱詭隨,以謹罔極。式遏寇虐,无俾作慝。敬慎威儀,以近有德。
瞻彼中原,其祁孔有。儦儦俟俟,或群或友。悉率左右,以燕天子。

既张我弓,既挟我矢。发彼小豝,殪此大兕。以御宾客,且以酌醴。




小雅·鸿雁之什


鸿雁

鸿雁于飞,肃肃其羽。之子于征,劬劳于野。爰及矜人,哀此鳏寡。

鸿雁于飞,集于中泽。之子于垣,百堵皆作。虽则劬劳,其究安宅?

鸿雁于飞,哀鸣嗷嗷。维此哲人,谓我劬劳。维彼愚人,谓我宣骄。

庭燎

夜如何其?夜未央,庭燎之光。君子至止,鸾声将将。

夜如何其?夜未艾,庭燎晣晣。君子至止,鸾声哕哕。

夜如何其?夜乡晨,庭燎有辉。君子至止,言观其旂。

沔水

沔彼流水,朝宗于海。鴥彼飞隼,载飞载止。嗟我兄弟,邦人诸友。莫肯念乱,谁无父母?

沔彼流水,其流汤汤。鴥彼飞隼,载飞载扬。念彼不迹,载起载行。心之忧矣,不可弭忘。

鴥彼飞隼,率彼中陵。民之讹言,宁莫之惩?我友敬矣,谗言其兴。

鹤鸣

鹤鸣于九皋,声闻于野。鱼潜在渊,或在于渚。乐彼之园,爰有树檀,其下维萚。他山之石,可以为错。

鹤鸣于九皋,声闻于天。鱼在于渚,或潜在渊。乐彼之园,爰有树檀,其下维谷。他山之石,可以攻玉。

祈父

祈父,予王之爪牙。胡转予于恤,靡所止居?

祈父,予王之爪士。胡转予于恤,靡所厎止?

祈父,亶不聪。胡转予于恤?有母之尸饔。

白驹

皎皎白驹,食我场苗。絷之维之,以永今朝。所谓伊人,于焉逍遥?

皎皎白驹,食我场藿。絷之维之,以永今夕。所谓伊人,于焉嘉客?

皎皎白驹,贲然来思。尔公尔侯,逸豫无期?慎尔优游,勉尔遁思。

皎皎白驹,在彼空谷。生刍一束,其人如玉。毋金玉尔音,而有遐心。

黄鸟

黄鸟黄鸟,无集于穀,无啄我粟。此邦之人,不我肯穀。言旋言归,复我邦族。

黄鸟黄鸟,无集于桑,无啄我粱。此邦之人,不可与明。言旋言归,复我诸兄。

黄鸟黄鸟,无集于栩,无啄我黍。此邦之人,不可与处。言旋言归,复我诸父。

我行其野

我行其野,蔽芾其樗。婚姻之故,言就尔居。尔不我畜,复我邦家。

我行其野,言采其蓫。婚姻之故,言就尔宿。尔不我畜,言归斯复。

我行其野,言采其葍。不思旧姻,求尔新特。成不以富,亦祗以异。

斯干

秩秩斯干,幽幽南山。如竹苞矣,如松茂矣。兄及弟矣,式相好矣,无相犹矣。

似续妣祖,筑室百堵,西南其户。爰居爰处,爰笑爰语。

约之阁阁,椓之橐橐。风雨攸除,鸟鼠攸去,君子攸芋。

如跂斯翼,如矢斯棘,如鸟斯革,如翚斯飞,君子攸跻。

殖殖其庭,有觉其楹。哙哙其正,哕哕其冥。君子攸宁。

下莞上簟,乃安斯寝。乃寝乃兴,乃占我梦。吉梦维何?维熊维罴,维虺维蛇。

大人占之:维熊维罴,男子之祥;维虺维蛇,女子之祥。

乃生男子,载寝之床。载衣之裳,载弄之璋。其泣喤喤,朱芾斯皇,室家君王。

乃生女子,载寝之地。载衣之裼,载弄之瓦。无非无仪,唯酒食是议,无父母诒罹。

无羊

谁谓尔无羊?三百维群。谁谓尔无牛?九十其犉。尔羊来思,其角濈濈。尔牛来思,其耳湿湿。

或降于阿,或饮于池,或寝或讹。尔牧来思,何蓑何笠,或负其餱。三十维物,尔牲则具。

尔牧来思,以薪以蒸,以雌以雄。尔羊来思,矜矜兢兢,不骞不崩。麾之以肱,毕来既升。

牧人乃梦,众维鱼矣,旐维旟矣,大人占之;众维鱼矣,实维丰年;旐维旟矣,室家溱溱。




小雅·节南山之什


节南山

节彼南山,维石岩岩。赫赫师尹,民具尔瞻。忧心如惔,不敢戏谈。国既卒斩,何用不监!

节彼南山,有实其猗。赫赫师尹,不平谓何。天方荐瘥,丧乱弘多。民言无嘉,惨莫惩嗟。

尹氏大师,维周之氐;秉国之钧,四方是维。天子是毗,俾民不迷。不吊昊天,不宜空我师。

弗躬弗亲,庶民弗信。弗问弗仕,勿罔君子。式夷式已,无小人殆。琐琐姻亚,则无膴仕。

昊天不佣,降此鞠訩。昊天不惠,降此大戾。君子如届,俾民心阕。君子如夷,恶怒是违。

不吊昊天,乱靡有定。式月斯生,俾民不宁。忧心如酲,谁秉国成?不自为政,卒劳百姓。

驾彼四牡,四牡项领。我瞻四方,蹙蹙靡所骋。

方茂尔恶,相尔矛矣。既夷既怿,如相酬矣。

昊天不平,我王不宁。不惩其心,覆怨其正。

家父作诵,以究王訩。式讹尔心,以畜万邦。

正月

正月繁霜,我心忧伤。民之讹言,亦孔之将。念我独兮,忧心京京。哀我小心,癙忧以痒。

父母生我,胡俾我瘉?不自我先,不自我后。好言自口,莠言自口。忧心愈愈,是以有侮。

忧心惸惸,念我无禄。民之无辜,并其臣仆。哀我人斯,于何从禄?瞻乌爰止?于谁之屋?

瞻彼中林,侯薪侯蒸。民今方殆,视天梦梦。既克有定,靡人弗胜。有皇上帝,伊谁云憎?

谓山盖卑,为冈为陵。民之讹言,宁莫之惩。召彼故老,讯之占梦。具曰予圣,谁知乌之雌雄!

谓天盖高,不敢不局。谓地盖厚,不敢不蹐。维号斯言,有伦有脊。哀今之人,胡为虺蜴?

瞻彼阪田,有菀其特。天之杌我,如不我克。彼求我则,如不我得。执我仇仇,亦不我力。

心之忧矣,如或结之。今兹之正,胡然厉矣?燎之方扬,宁或灭之?赫赫宗周,褒姒灭之!

终其永怀,又窘阴雨。其车既载,乃弃尔辅。载输尔载,将伯助予!

无弃尔辅,员于尔辐。屡顾尔仆,不输尔载。终逾绝险,曾是不意。

鱼在于沼,亦匪克乐。潜虽伏矣,亦孔之炤。忧心惨惨,念国之为虐!

彼有旨酒,又有嘉肴。洽比其邻,婚姻孔云。念我独兮,忧心殷殷。

佌佌彼有屋,蔌蔌方有谷。民今之无禄,天夭是椓。哿矣富人,哀此惸独。

十月之交

十月之交,朔月辛卯。日有食之,亦孔之丑。彼月而微,此日而微;今此下民,亦孔之哀。

日月告凶,不用其行。四国无政,不用其良。彼月而食,则维其常;此日而食,于何不臧。

烨烨震电,不宁不令。百川沸腾,山冢崒崩。高岸为谷,深谷为陵。哀今之人,胡惨莫惩?

皇父卿士,番维司徒,家伯维宰,仲允膳夫,棸子内史,蹶维趣马,禹维师氏。醘妻煽方处。

抑此皇父,岂曰不时?胡为我作,不即我谋?彻我墙屋,田卒汙莱。曰予不戕,礼则然矣。

皇父孔圣,作都于向。择三有事,亶侯多藏。不慭遗一老,俾守我王。择有车马,以居徂向。

黾勉从事,不敢告劳。无罪无辜,谗口嚣嚣。下民之孽,匪降自天。噂沓背憎,职竞由人。

悠悠我里,亦孔之痗。四方有羡,我独居忧。民莫不逸,我独不敢休。天命不彻,我不敢效我友自逸。

雨无正

浩浩昊天,不骏其德。降丧饥馑,斩伐四国。旻天疾威,弗虑弗图。舍彼有罪,既伏其辜。若此无罪,沦胥以铺。

周宗既灭,靡所止戾。正大夫离居,莫知我勚。三事大夫,莫肯夙夜。邦君诸侯,莫肯朝夕。庶曰式臧,覆出为恶。

如何昊天,辟言不信。如彼行迈,则靡所臻。凡百君子,各敬尔身。胡不相畏,不畏于天?

戎成不退,饥成不遂。曾我暬御,惨惨日瘁。凡百君子,莫肯用讯。听言则答,谮言则退。

哀哉不能言,匪舌是出,维躬是瘁。哿矣能言,巧言如流,俾躬处休!

维曰予仕,孔棘且殆。云不何使,得罪于天子;亦云可使,怨及朋友。

谓尔迁于王都。曰予未有室家。鼠思泣血,无言不疾。昔尔出居,谁从作尔室?

小旻

旻天疾威,敷于下土。谋犹回遹,何日斯沮?谋臧不从,不臧覆用。我视谋犹,亦孔之邛。

潝潝訿訿,亦孔之哀。谋之其臧,则具是违。谋之不臧,则具是依。我视谋犹,伊于胡厎。

我龟既厌,不我告犹。谋夫孔多,是用不集。发言盈庭,谁敢执其咎?如匪行迈谋,是用不得于道。

哀哉为犹,匪先民是程,匪大犹是经。维迩言是听,维迩言是争。如彼筑室于道谋,是用不溃于成。

国虽靡止,或圣或否。民虽靡膴,或哲或谋,或肃或艾。如彼泉流,无沦胥以败。

不敢暴虎,不敢冯河。人知其一,莫知其他。战战兢兢,如临深渊,如履薄冰。

小宛

宛彼鸣鸠,翰飞戾天。我心忧伤,念昔先人。明发不寐,有怀二人。

人之齐圣,饮酒温克。彼昏不知,壹醉日富。各敬尔仪,天命不又。

中原有菽,庶民采之。螟蛉有子,蜾蠃负之。教诲尔子,式穀似之。

题彼脊令,载飞载鸣。我日斯迈,而月斯征。夙兴夜寐,毋忝尔所生。

交交桑扈,率场啄粟。哀我填寡,宜岸宜狱。握粟出卜,自何能穀?

温温恭人,如集于木。惴惴小心,如临于谷。战战兢兢,如履薄冰。

小弁

弁彼鸴斯,归飞提提。民莫不穀,我独于罹。何辜于天?我罪伊何?心之忧矣,云如之何?

踧踧周道,鞫为茂草。我心忧伤,惄焉如捣。假寐永叹,维忧用老。心之忧矣,疢如疾首。

维桑与梓,必恭敬止。靡瞻匪父,靡依匪母。不属于毛?不罹于里?天之生我,我辰安在?

菀彼柳斯,鸣蜩嘒嘒,有漼者渊,萑苇淠淠。譬彼舟流,不知所届,心之忧矣,不遑假寐。

鹿斯之奔,维足伎伎。雉之朝雊,尚求其雌。譬彼坏木,疾用无枝。心之忧矣,宁莫之知?

相彼投兔,尚或先之。行有死人,尚或墐之。君子秉心,维其忍之。心之忧矣,涕既陨之。

君子信谗,如或酬之。君子不惠,不舒究之。伐木掎矣,析薪扡矣。舍彼有罪,予之佗矣。

莫高匪山,莫浚匪泉。君子无易由言,耳属于垣。无逝我梁,无发我笱。我躬不阅,遑恤我后。

巧言

悠悠昊天,曰父母且。无罪无辜,乱如此幠。昊天已威,予慎无罪。昊天大幠,予慎无辜。

乱之初生,僭始既涵。乱之又生,君子信谗。君子如怒,乱庶遄沮。君子如祉,乱庶遄已。

君子屡盟,乱是用长。君子信盗,乱是用暴。盗言孔甘,乱是用餤。匪其止共,维王之邛。奕奕寝庙,君子作之。秩秩大猷,圣人莫之。

他人有心,予忖度之。跃跃毚兔,遇犬获之。荏染柔木,君子树之。往来行言,心焉数之。蛇蛇硕言,出自口矣。巧言如簧,颜之厚矣。

彼何人斯?居河之麋。无拳无勇,职为乱阶。既微且尰,尔勇伊何?为犹将多,尔居徒几何?

何人斯

彼何人斯?其心孔艰。胡逝我梁,不入我门?伊谁云从?维暴之云。

二人从行,谁为此祸?胡逝我梁,不入唁我?始者不如今,云不我可

彼何人斯?胡逝我陈?我闻其声,不见其身。不愧于人?不畏于天?

彼何人斯?其为飘风。胡不自北?胡不自南?胡逝我梁?絺搅我心。

尔之安行,亦不遑舍。尔之亟行,遑脂尔车。壹者之来,云何其盱。

尔还而入,我心易也。还而不入,否难知也。壹者之来,俾我絺也。

伯氏吹埙,仲氏吹篪。及尔如贯,谅不我郑出此三物,以诅尔斯。

为鬼为蜮,则不可得。有靦面目,视人罔极。作此好歌,以极反侧。

巷伯

萋兮斐兮,成是贝锦。彼谮人者,亦已大甚!

哆兮侈兮,成是南箕。彼谮人者,谁适与谋。

缉缉翩翩,谋欲谮人。慎尔言也,谓尔不信。

捷捷幡幡,谋欲谮言。岂不尔受?既其女迁。

骄人好好,劳人草草。苍天苍天,视彼骄人,矜此劳人。

彼谮人者,谁适与谋?取彼谮人,投畀豺虎。豺虎不食,投畀有北。有北不受,投畀有昊!

杨园之道,猗于亩丘。寺人孟子,作为此诗。凡百君子,敬而听之。




小雅·谷风之什


谷风

习习谷风,维风及雨。将恐将惧,维予与女。将安将乐,女转弃予。

习习谷风,维风及颓。将恐将惧,置予于怀。将安将乐,弃予如遗。

习习谷风,维山崔嵬。无草不死,无木不萎。忘我大德,思我小怨。

蓼莪

蓼蓼者莪,匪莪伊蒿。哀哀父母,生我劬劳。

蓼蓼者莪,匪莪伊蔚。哀哀父母,生我劳瘁。

瓶之罄矣,维罍之耻。鲜民之生,不如死之久矣。无父何怙?无母何恃?出则衔恤,入则靡至。

父兮生我,母兮鞠我。抚我畜我,长我育我,顾我复我,出入腹我。欲报之德。昊天罔极!

南山烈烈,飘风发发。民莫不穀,我独何害!南山律律,飘风弗弗。民莫不穀,我独不卒!

大东

有饛簋飧,有捄棘匕。周道如砥,其直如矢。君子所履,小人所视。眷言顾之,潸焉出涕。

小东大东,杼柚其空。纠纠葛屦,可以履霜。佻佻公子,行彼周行。既往既来,使我心疚。

有冽氿泉,无浸获薪。契契寤叹,哀我惮人。薪是获薪,尚可载也。哀我惮人,亦可息也。

东人之子,职劳不来。西人之子,粲粲衣服。舟人之子,熊罴是裘。私人之子,百僚是试。

或以其酒,不以其浆。鞙鞙佩璲,不以其长。维天有汉,监亦有光。跂彼织女,终日七襄。

虽则七襄,不成报章。彼牵牛,不以服箱。东有启明,西有长?庚。有捄天毕,载施之行。

维南有箕,不可以簸扬。维北有斗,不可以挹酒浆。维南有箕,载翕其舌。维北有斗,西柄之揭。

四月

四月维夏,六月徂署。先祖匪人,胡宁忍予?

秋日凄凄,百卉具腓。乱离瘼矣,爰其适归?

冬日烈烈,飘风发发。民莫不穀,我独何害?

山有嘉卉,侯栗侯梅。废为残贼,莫知其尤!

相彼泉水,载清载浊。我日构祸,曷云能穀?

滔滔江汉,南国之纪。尽瘁以仕,宁莫我有?

匪鹑匪鸢,翰飞戾天。匪鳣匪鲔,潜逃于渊。

山有蕨薇,隰有杞桋。君子作歌,维以告哀。

北山

陟彼北山,言采其杞。偕偕士子,朝夕从事。王事靡盬,忧我父母。

溥天之下,莫非王土;率土之滨,莫非王臣。大夫不均,我从事独贤。

四牡彭彭,王事傍傍。嘉我未老,鲜我方将。旅力方刚,经营四方。

或燕燕居息,或尽瘁事国;或息偃在床,或不已于行。

或不知叫号,或惨惨劬劳;或栖迟偃仰,或王事鞅掌。

或湛乐饮酒,或惨惨畏咎;或出入风议,或靡事不为。

无将大车

无将大车,祇自尘兮。无思百忧,祇自疧兮。

无将大车,维尘冥冥。无思百忧,不出于颎。

无将大车,维尘雍兮。无思百忧,祇自重兮。

小明

明明上天,照临下土。我征徂西,至于艽野。二月初吉,载离寒暑。心之忧矣,其毒

大苦。念彼共人,涕零如雨。岂不怀归?畏此罪罟!

昔我往矣,日月方除。曷云其还?岁聿云莫。念我独兮,我事孔庶。心之忧矣,惮我

不暇。念彼共人,眷眷怀顾!岂不怀归?畏此谴怒。

昔我往矣,日月方奥。曷云其还?政事愈蹙。岁聿云莫,采萧获菽。心之忧矣,自诒

伊戚。念彼共人,兴言出宿。岂不怀归?畏此反覆。

嗟尔君子,无恒安处。靖共尔位,正直是与。神之听之,式穀以女。

嗟尔君子,无恒安息。靖共尔位,好是正直。神之听之,介尔景福。

鼓钟

鼓钟将将,淮水汤汤,忧心且伤。淑人君子,怀允不忘。

鼓钟喈喈,淮水湝湝,忧心且悲。淑人君子,其德不回。

鼓钟伐鼛,淮有三洲,忧心且妯。淑人君子,其德不犹。

鼓钟钦钦,鼓瑟鼓琴,笙磬同音。以雅以南,以龠不僭。

楚茨

楚楚者茨,言抽其棘,自昔何为?我蓺黍稷。我黍与与,我稷翼翼。

我仓既盈,我庾维亿。以为酒食,以享以祀,以妥以侑,以介景福。

济济跄跄,絜尔牛羊,以往烝尝。或剥或亨,或肆或将。祝祭于祊,祀事孔明。先祖

是皇,神保是飨。孝孙有庆,报以介福,万寿无疆!

执爨 ,为俎孔硕,或燔或炙。君妇莫莫,为豆孔庶。为宾为客,献酬交错。礼仪卒

度,笑语卒获。神保是格,报以介福,万寿攸酢!

我孔 矣,式礼莫愆。工祝致告,徂赉孝孙。苾芬孝祀,神嗜饮食。卜尔百福,如几

如式。既齐既稷,既匡既敕。永锡尔极,时万时亿!

礼仪既备,钟鼓既戒,孝孙徂位,工祝致告,神具醉止,皇尸载起。鼓钟送尸,神保

聿归。诸宰君妇,废彻不迟。诸父兄弟,备言燕私。

乐具入奏,以绥后禄。尔肴既将,莫怨具庆。既醉既饱,小大稽首。神嗜饮食,使君

寿考。孔惠孔时,维其尽之。子子孙孙,勿替引之!

信南山

信彼南山,维禹甸之。畇畇原隰,曾孙田之。我疆我理,南东其亩。

上天同云。雨雪雰雰,益之以霡霂。既优既渥,既沾既足。生我百谷。

疆场翼翼,黍稷彧彧。曾孙之穑,以为酒食。畀我尸宾,寿考万年。

中田有庐,疆场有瓜。是剥是菹,献之皇祖。曾孙寿考,受天之祜。

祭以清酒,从以骍牡,享于祖考。执其鸾刀,以启其毛,取其血膋。

是烝是享,苾苾芬芬。祀事孔明,先祖是皇。报以介福。万寿无疆。

 

 
小雅·甫田之什


甫田 

倬彼甫田,岁取十千。我取其陈,食我农人。自古有年。今适南亩,或耘或耔。黍稷薿薿,攸介攸止,烝我髦士。

以我齐明,与我牺羊,以社以方。我田既臧,农夫之庆。琴瑟击鼓,以御田祖。以祈甘雨,以介我稷黍,以穀我士女。

曾孙来止,以其妇子。馌彼南亩,田畯至喜。攘其左右,尝其旨否。禾易长亩,终善且有。曾孙不怒,农夫克敏。

曾孙之稼,如茨如梁。曾孙之庾,如坻如京。乃求千斯仓,乃求万斯箱。黍稷稻粱,农夫之庆。报以介福,万寿无疆。

大田

大田多稼,既种既戒,既备乃事。以我覃耜,俶载南亩。播厥百谷,既庭且硕,曾孙是若。

既方既皂,既坚既好,不稂不莠。去其螟螣,及其蟊贼,无害我田稚。田祖有神,秉畀炎火。

有渰萋萋,兴雨祈祈。雨我公田,遂及我私。彼有不获稚,此有不敛穧,彼有遗秉,此有滞穗,伊寡妇之利。

曾孙来止,以其妇子。馌彼南亩,田畯至喜。来方禋祀,以其骍黑,与其黍稷。以享以祀,以介景福。

瞻彼洛矣

瞻彼洛矣,维水泱泱。君子至止,福禄如茨。韎韐有奭,以作六师。

瞻彼洛矣,维水泱泱。君子至止,鞸琫有珌。君子万年,保其家室。

瞻彼洛矣,维水泱泱。君子至止,福禄既同。君子万年,保其家邦。

裳裳者华

裳裳者华,其叶湑□兮。我觏之子,我心写兮。我心写兮,是以有誉处兮。

裳裳者华,芸其黄矣。我觏之子,维其有章矣。维其有章矣,是以有庆矣。

裳裳者华,或黄或白。我觏之子,乘其四骆。乘其四骆,六辔沃若。

左之左之,君子宜之。右之右之,君子有之。维其有之,是以似之。

桑扈

交交桑扈,有莺其羽。君子乐胥,受天之祜。

交交桑扈,有莺其领。君子乐胥,万邦之屏。

之屏之翰,百辟为宪。不戢不难,受福不那。

兕觥其觩,旨酒思柔。彼交匪敖,万福来求。

鸳鸯

鸳鸯于飞,毕之罗之。君子万年,福禄宜之。

鸳鸯在梁,戢其左翼。君子万年,

宜其遐福。乘马在厩,摧之秣之。君子万年,福禄艾之。

乘马在厩,秣之摧之。君子万年,福禄绥之。

頍弁

有頍者弁,实维伊何?尔酒既旨,尔肴既嘉。岂伊异人?兄弟匪他。茑与女萝,施于松柏。未见君子,忧心奕奕;既见君子,庶几说怿。

有頍者弁,实维何期?尔酒既旨,尔肴既时。岂伊异人?兄弟具来。茑与女萝,施于松上。未见君子,忧心怲怲;既见君子,庶几有臧。

有頍者弁,实维在首。尔酒既旨,尔肴既阜。岂伊异人?兄弟甥舅。如彼雨雪,先集维霰。死丧无日,无几相见。乐酒今夕,君子维宴。

车舝

间关车之舝兮,思娈季女逝兮。匪饥匪渴,德音来括。虽无好友?式燕且喜。

依彼平林,有集维鷮。辰彼硕女,令德来教。式燕且誉,好尔无射。

虽无旨酒?式饮庶几。虽无嘉肴?式食庶几。虽无德与女?式歌且舞?

陟彼高冈,析其柞薪。析其柞薪,其叶湑兮。鲜我觏尔,我心写兮。

高山仰止,景行行止。四牡騑騑,六辔如琴。觏尔新婚,以慰我心。

青蝇

营营青蝇,止于樊。岂弟君子,无信谗言。

营营青蝇,止于棘。谗人罔极,交乱四国。

营营青蝇,止于榛。谗人罔极,构我二人。

宾之初筵

宾之初筵,左右秩秩。笾豆有楚,殽核维旅。酒既和旨,饮酒孔偕。钟鼓既设,举酬逸逸。大侯既抗,弓矢斯张。射夫既同,献尔发功。

发彼有的,以祈尔爵。籥舞笙鼓,乐既和奏。烝衎烈祖,以洽百礼。百礼既至,有壬有林。锡尔纯嘏,子孙其湛。其湛曰乐,各奏尔能。宾载手仇,室人入又。酌彼康爵,以奏尔时。

宾之初筵,温温其恭。其未醉止,威仪反反。曰既醉止,威仪幡幡。舍其坐迁,屡舞仙仙。其未醉止,威仪抑抑。曰既醉止,威仪抑抑。是曰既醉,不知其秩。

宾既醉止,载号载呶。乱我笾豆,屡舞僛僛。是曰既醉,不知其邮。侧弁之俄,屡舞傞傞。既醉而出,并受其福。

醉而不出,是谓伐德。饮酒孔嘉,维其令仪。凡此饮酒,或醉或否。既立之监,或佐之史。彼醉不臧,不醉反耻。

式勿从谓,无俾大怠。匪言勿言,匪由勿语。由醉之言,俾出童羖。三爵不识,矧敢多又。




小雅·鱼藻之什


鱼藻

鱼在在藻,有颁其首。王在在镐,岂乐饮酒。

鱼在在藻,有莘其尾。王在在镐,饮酒乐岂。

鱼在在藻,依于其蒲。王在在镐,有那其居。

采菽

采菽采菽,筐之莒之。君子来朝,何锡予之?虽无予之?路车乘马。又何予之?玄衮及黼。

觱沸槛泉,言采其芹。君子来朝,言观其旂。其旂淠淠,鸾声嘒嘒。载骖载驷,君子所届。

赤芾在股,邪幅在下。彼交匪纾,天子所予。乐只君子,天子命之。乐只君子,福禄申之。

维柞之枝,其叶蓬蓬。乐只君子,殿天子之邦。乐只君子,万福攸同。平平左右,亦是率从。

汎汎杨舟,绋纚维之。乐只君子,天子葵之。乐只君子,福禄膍之。优哉游哉,亦是戾矣。

角弓

骍骍角弓,翩其反矣。兄弟婚姻,无胥远矣。

尔之远矣,民胥然矣。尔之教矣,民胥效矣。

此令兄弟,绰绰有裕。不令兄弟,交相为愈。

民之无良,相怨一方。受爵不让,至于已斯亡。

老马反为驹,不顾其后。如食宜饇,如酌孔取。

毋教猱升木,如涂涂附。君子有徽猷,小人与属。

雨雪瀌瀌,见晛曰消。莫肯下遗,式居娄骄。

雨雪浮浮,见晛曰流。如蛮如髦,我是用忧。

菀柳

有菀者柳,不尚息焉。上帝甚蹈,无自暱焉。俾予靖之,后予极焉。

有菀者柳,不尚愒焉。上帝甚蹈,无自瘵焉。俾予靖之,后予迈焉。

有鸟高飞,亦傅于天。彼人之心,于何其臻。曷予靖之,居以凶矜。

都人士

彼都人士,狐裘黄黄。其容不改,出言有章。行归于周,万民所望。

彼都人士,台笠缁撮。彼君子女,绸直如发。我不见兮,我心不说。

彼都人士,充耳琇实。彼君子女,谓之尹吉。我不见兮,我心苑结。

彼都人士,垂带而厉。彼君子女,卷发如虿。我不见兮,言从之迈。

匪伊垂之,带则有余。匪伊卷之,发则有旟。我不见兮,云何盱矣。

采绿

终朝采绿,不盈一匊。予发曲局,薄言归沐。

终朝采蓝,不盈一襜。五日为期,六日不詹。

之子于狩,言韔其弓。之子于钓,言纶之绳。

其钓维何?维鲂及鱮。维鲂及鱮,薄言观者。

黍苗

芃芃黍苗,阴雨膏之。悠悠南行,召伯劳之。

我任我辇,我车我牛。我行既集,盖云归哉。

我徒我御,我师我旅。我行既集,盖云归处。

肃肃谢功,召伯营之。烈烈征师,召伯成之。

原隰既平,泉流既清。召伯有成,王心则宁。

隰桑

隰桑有阿,其叶有难。既见君子,其乐如何。

隰桑有阿,其叶有沃。既见君子,云何不乐。

隰桑有阿,其叶有幽。既见君子,德音孔胶。

心乎爱矣,遐不谓矣?中心藏之,何日忘之!

白华

白华菅兮,白茅束兮。之子之远,俾我独兮。

英英白云,露彼菅茅。天步艰难,之子不犹。

滮池北流,浸彼稻田。啸歌伤怀,念彼硕人。

樵彼桑薪,卬烘于煁。维彼硕人,实劳我心。

鼓钟于宫,声闻于外。念子懆懆,视我迈迈。

有鹙在梁,有鹤在林。维彼硕人,实劳我心。

鸳鸯在梁,戢其左翼。之子无良,二三其德。

有扁斯石,履之卑兮。之子之远,俾我疧兮。

绵蛮

绵蛮黄鸟,止于丘阿。道之云远,我劳如何。饮之食之,教之诲之。命彼后车,谓之载之。

绵蛮黄鸟,止于丘隅。岂敢惮行,畏不能趋。饮之食之。教之诲之。命彼后车,谓之载之。

绵蛮黄鸟,止于丘侧。岂敢惮行,畏不能极。饮之食之,教之诲之。命彼后车,谓之载之。

瓠叶

幡幡瓠叶,采之亨之。君子有酒,酌言尝之。

有兔斯首,炮之燔之。君子有酒,酌言献之。

有兔斯首,燔之炙之。君子有酒,酌言酢之。

有兔斯首,燔之炮之。君子有酒,酌言酬之。

渐渐之石

渐渐之石,维其高矣。山川悠远,维其劳矣。武人东征,不遑朝矣。

渐渐之石,维其卒矣。山川悠远,曷其没矣?武人东征,不遑出矣。

有豕白蹢,烝涉波矣。月离于毕,俾滂沱矣。武人东征,不皇他矣。

苕之华

苕之华,芸其黄矣。心之忧矣,维其伤矣!

苕之华,其叶青青。知我如此,不如无生!

牂羊坟首,三星在罶。人可以食,鲜可以饱!

何草不黄

何草不黄?何日不行?何人不将?经营四方。

何草不玄?何人不矜?哀我征夫,独为匪民。

匪兕匪虎,率彼旷野。哀我征夫,朝夕不暇。

有芃者狐,率彼幽草。有栈之车,行彼周道。



 
大雅·文王之什


文王

文王在上,於昭于天,周虽旧邦,其命维新。

有周不显,帝命不时。文王陟降,在帝左右。

亹亹文王,令聞不已。陳錫哉周,侯文王孙子。

文王孙子,本支百世。凡周之士,不显亦世。

世之不显,厥猶翼翼。思皇多士,生此王国。

王国克生,维周之楨。济济多士,文王以寧。

穆穆文王,於緝熙敬止。假哉天命,有商孙子。

商之孙子,其麗不億。上帝既命,侯于周服。

侯服于周,天命靡常。殷士膚敏,祼將于京。

厥作祼將,常服黼冔。王之藎臣,无念尔祖。

无念尔祖,聿修厥德。永言配命,自求多福。

殷之未喪師,克配上帝。宜鑒于殷,駿命不易。

命之不易,无遏尔躬。宣昭義問,有虞殷自天。

上天之载,无声无臭。儀刑文王,万邦作孚。

大明

明明在下,赫赫在上。天难忱斯,不易维王。天位殷適,使不挾四方。

摯仲氏任,自彼殷商;來嫁于周,曰嬪于京。乃及王季,维德之行。大任有身,生此文王。

维此文王,小心翼翼。昭事上帝,聿懷多福。厥德不回,以受方国。

天監在下,有命既集。文王初载,天作之合。在洽之陽,在渭之涘。文王嘉止,大邦有子。

大邦有子,伣天之妹。文定厥祥,親迎于渭。造舟为梁,不显其光。

有命自天,命此文王。于周于京。纘女维莘,长子维行。篤生武王,保右命尔,燮伐大商。

殷商之旅,其會如林。矢于牧野,维予侯興。上帝臨女,无貳尔心!

牧野洋洋,檀車煌煌,駟騵彭彭。维師尚父,时维鷹揚;涼彼武王,肆伐大商,會朝清明。

绵

绵绵瓜瓞。民之初生,自土沮漆。古公亶父,陶复陶穴,未有家室。

古公亶父,來朝走马,率西水滸,至于岐下。爰及姜女,聿來胥宇。

周原膴膴,堇荼如飴。爰始爰謀,爰契我龜。曰止曰时,筑室于茲。

乃慰乃止,乃左乃右,乃疆乃理,乃宣乃畝。自西徂東,周爰執事。

乃召司空,乃召司徒,俾立室家。其繩則直,縮版以载,作庙翼翼。

捄之陾陾,度之薨薨,筑之登登,削屢馮馮。百堵皆興,鼛鼓弗勝。

乃立皋門,皋門有伉;乃立应門,应門將將。乃立冢土,戎醜攸行。

肆不殄厥慍,亦不隕厥問,柞棫拔矣,行道兌矣。混夷駾矣,维其喙矣。

虞芮質厥成,文王厥厥生。予曰有疏附,予曰有先後,予曰有奔奏,予曰有御侮。

棫樸

芃芃棫樸,薪之槱之。济济辟王,左右趣之。

济济辟王,左右奉璋。奉璋峨峨,髦士攸宜。

淠彼涇舟,烝徒楫之。周王于邁,六師及之。

倬彼云汉,为章于天。周王寿考,遐不作人?

追琢其章,金玉其相。勉勉我王,綱紀四方。

旱麓

瞻彼旱麓,榛楛济济。豈弟君子,干祿豈弟。

瑟彼玉瓚,黃流在中。豈弟君子,福祿攸降。

鳶飞戾天,鱼躍于淵。豈弟君子,遐不作人?

清酒既载,騂牡既備。以享以祀,以介景福。

瑟彼柞棫,民所燎矣。豈弟君子,神所勞矣。

莫莫葛櫑,施于條枚。豈弟君子,求福不回。

思齐

思齐大任,文王之母。思媚周姜,京室之婦。大姒嗣徽音,則百斯男。

惠于宗公,神罔时怨,神罔时恫。刑于寡妻,至于兄弟,以御于家邦。

雍雍在宮,肅肅在庙。不显亦臨,无射亦保。

肆戎疾不殄,烈假不遐。不聞亦式,不諫亦入。

肆成人有德,小子有造。古人之无斁,譽髦斯士。

皇矣

皇矣上帝,臨下有赫;監觀四方,求民之莫。

维此二国,其政不獲;维彼四国,爰究爰度。

上帝耆之,憎其式廓。乃眷西顧,此维与宅。

作之屏之,其菑其翳;修之平之,其灌其栵;

啟之辟之,其檉其椐;攘之剔之,其檿其柘。

帝遷明德,串夷载路。天立厥配,受命既固。

帝省其山,柞棫斯拔,松柏斯兌。帝作邦作對,自大伯王季。

维此王季,因心則友。則友其兄,則篤其慶,载錫之光。受祿无喪,奄有四方。

维此王季,帝度其心,貊其德音。其德克明,克明克類,克长克君。

王此大邦,克順克比。比于文王,其德靡悔。既受帝祉,施于孙子。

帝謂文王:无然畔援,无然歆羨,誕先登于岸。密人不恭,敢距大邦,侵阮徂共。

王赫斯怒,爰整其旅,以按徂旅,以篤周祜,以對于天下。

依其在京,侵自阮疆,陟我高岡。无矢我陵,我陵我阿;无飲我泉,我泉我池!

度其鮮原,居岐之陽,在渭之將。万邦之方,下民之王。

帝謂文王:予懷明德,不大声以色,不长夏以革,不識不知,順帝之則。

帝謂文王:詢尔仇方,同尔兄弟。以尔鉤援,与尔臨沖,以伐崇墉。

臨沖閑閑,崇墉言言,執訊連連,攸馘安安。是類是祃,是致是附,四方以无侮。

臨沖茀茀,崇墉仡仡,是伐是肆,是絕是忽,四方以无拂。

灵台

經始灵台,經之營之。庶民攻之,不日成之。經始勿亟,庶民子來。

王在灵囿,麀鹿攸伏;麀鹿濯濯,白鸟翯翯。王在灵沼,於牣鱼躍。

虡業维樅,賁鼓维鏞。於論鼓鐘,於乐辟癰。

於論鼓鐘,於乐辟癰。鼉鼓逢逢,矇瞍奏公。

下武

下武维周,世有哲王。三後在天,王配于京。

王配于京,世德作求。永言配命,成王之孚。

成王之孚,下土之式。永言孝思,孝思维則。

媚茲一人,应侯順德。永言孝思,昭哉嗣服。

昭茲來許,繩其祖武。於万斯年,受天之祜。

受天之祜,四方來賀。於万斯年,不遐有佐。

文王有声

文王有声,遹駿有声,遹求厥寧,遹觀厥成。文王烝哉!

文王受命,有此武功;既伐于崇,作邑于丰。文王烝哉!

筑城伊淢,作丰伊匹,匪棘其欲,遹追來孝。王后烝哉!

王公伊濯,维丰之垣。四方攸同,王后维翰。王后烝哉!

丰水東注,维禹之績。四方攸同,皇王维辟。皇王烝哉!

鎬京辟癰,自西自東,自南自北,无思不服。皇王烝哉!

考卜维王,宅是鎬京。维龜正之,武王成之。武王烝哉!

丰水有芑,武王豈不仕?詒厥孙謀,以燕翼子。武王烝哉!




大雅·生民之什


生民

厥初生民,时维姜嫄。生民如何?克禋克祀,以弗无子。履帝武敏歆,攸介攸止;载震载夙,载生载育,时维后稷。

誕彌厥月,先生如达。不坼不副,无菑无害。以赫厥灵,上帝不寧。不康禋祀,居然生子。

誕寘之隘巷,牛羊腓字之。誕寘之平林,會伐平林;誕寘之寒冰,鸟覆翼之。鸟乃去矣,后稷呱矣。实覃实訏,厥声载路。

誕实匍匐,克岐克嶷,以就口食。蓺之荏菽,荏菽旆旆,禾役穟穟,麻麥幪幪,瓜瓞唪唪。

誕后稷之穡,有相之道。茀厥丰草,種之黃茂。实方实苞,实種实褎,实发实秀,实堅实好,实穎实栗,即有邰家室。

誕降嘉種,维秬维秠,维穈维芑。恒之秬秠,是獲是畝;恒之穈芑,是任是負,以归肇祀。

誕我祀如何?或舂或揄,或簸或蹂;釋之叟叟,烝之浮浮。载謀载惟,取蕭祭脂,取羝以軷,载燔载烈。以興嗣歲。

卬盛于豆,于豆于登。其香始升,上帝居歆。胡臭亶时。后稷肇祀,庶无罪悔,以迄于今。

行葦

敦彼行葦,牛羊勿踐履。方苞方體,维葉泥泥。戚戚兄弟,莫遠具尔。或肆之筵,或授之几。

肆筵設席,授几有緝御。或獻或酢,洗爵奠斝。醓醢以薦,或幡或炙。嘉肴脾臄,或歌或咢。

敦弓既堅,四鍭既鈞;舍矢既均,序賓以賢。敦弓既句,既挾四鍭;四鍭如樹,序賓以不侮。

曾孙维主,酒醴维醹,酌以大斗,以祈黃耇。黃耇台背,以引以翼。寿考维祺,以介景福。

既醉

既醉以酒,既飽以德。君子万年,介尔景福。

既醉以酒,尔殽既將。君子万年,介尔昭明。

昭明有融,高朗令終。令終有俶,公尸嘉告。

其告维何?籩豆靜嘉。朋友攸攝,攝以威儀。

威儀孔时,君子有孝子。孝子不匱,永錫尔類。

其類维何?室家之壸。君子万年,永錫祚胤。

其胤维何?天被尔祿。君子万年,景命有仆。

其仆维何?厘尔女士。厘尔女士,從以孙子。

鳧鹥

鳧鹥在涇,公尸來燕來寧。尔酒既清,尔殽既馨。公尸燕飲,福祿來成。

鳧鹥在沙,公尸來燕來宜。尔酒既多,尔殽既嘉。公尸燕飲,福祿來为。

鳧鹥在渚,公尸來燕來處。尔酒既湑,尔殽伊脯。公尸燕飲,福祿來下。

鳧鹥在潀,公尸來燕來宗。既燕于宗,福祿攸降。公尸燕飲,福祿來崇。

鳧鹥在亹,公尸來止熏熏。旨酒欣欣,燔炙芬芬。公尸燕飲,无有後艱。

假乐

假乐君子,显显令德。宜民宜人,受祿于天。保右命之,自天申之。

干祿百福,子孙千億。穆穆皇皇,宜君宜王。不愆不忘,率由旧章。

威儀抑抑,德音秩秩。无怨无惡,率由群匹。受福无疆,四方之綱。

之綱之紀,燕及朋友。百辟卿士,媚于天子。不解于位,民之攸墍。《假乐》

公刘

篤公刘,匪居匪康,乃场乃疆,乃積乃倉。乃裹糇糧,于橐于囊,思輯用光。弓矢斯张,干戈戚揚,爰方啟行。

篤公刘,于胥斯原。既庶既繁。既順乃宣,而无詠嘆。陟則在巘,复降在原。何以舟之?维玉及瑤,鞞琫容刀。

篤公刘,逝彼百泉,瞻彼溥原。乃陟南岡,乃覯于京。京師之野,于时處處,于时廬旅。于时言言,于时語語。

篤公刘,于京斯依。蹌蹌济济,俾筵俾几。既登乃依,乃造其曹;執豕于牢,酌之用匏。食之飲之,君之宗之。

篤公刘,既溥既长。既景乃岡,相其陰陽,觀其流泉。其軍三單,度其隰原,徹田为糧。度其夕陽,豳居允荒。

篤公刘,于豳斯館。涉渭为亂,取厲取鍛。止基乃理,爰眾爰有。夾其皇澗,溯其過澗。止旅乃密,芮鞫之即。

泂酌

泂酌彼行潦,挹彼注茲,可以餴饎。豈弟君子,民之父母。

泂酌彼行潦,挹彼注茲,可以濯罍。豈弟君子,民之攸归。

泂酌彼行潦,挹彼注茲,可以濯溉。豈弟君子,民之攸墍。

卷阿

有卷者阿,飄風自南。豈弟君子,來游來歌,以矢其音。

伴奐尔游矣,优游尔休矣。豈弟君子,俾尔彌尔性,似先公酋矣。

尔土宇皈章,亦孔之厚矣。豈弟君子,俾尔彌尔性,百神尔主矣。

尔受命长矣,茀祿尔康矣。豈弟君子,俾尔彌尔性,純嘏尔常矣。

有馮有翼,有孝有德,以引以翼。豈弟君子,四方为則。

颙颙卬卬,如圭如璋,令聞令望。豈弟君子,四方为綱。

鳳皇于飞,翙翙其羽,亦集爰止。藹藹王多吉士,维君子使,媚于天子。

鳳皇于飞,翙翙其羽,亦傅于天。藹藹王多吉人,维君子命,媚于庶人。

鳳皇鳴矣,于彼高岡。梧桐生矣,于彼朝陽。菶菶萋萋,雍雍喈喈。

君子之車,既庶且多;君子之马,既閑且馳。矢詩不多,维以遂歌。

民勞

民亦勞止,汔可械。惠此中国,以綏四方。无縱詭隨,以謹无良。式遏寇虐,憯不畏明。柔遠能邇,以定我王。

民亦勞止,汔可小休。惠此中国,以为民逑。无縱詭隨,以謹惛怓。式遏寇虐,无俾民憂。无棄尔勞,以为王休。

民亦勞止,汔可小息。惠此京師,以綏四国。无縱詭隨,以謹罔極。式遏寇虐,无俾作慝。敬慎威儀,以近有德。

民亦勞止,汔可小愒。惠此中国,俾民憂泄。无縱詭隨,以謹醜厲。式遏寇虐,无俾正敗。戎虽小子,而式弘大。

民亦勞止,汔可小安。惠此中国,国无有殘。无縱詭隨,以謹繾綣。式遏寇虐,无俾正反。王欲玉女,是用大諫。

板

上帝板板,下民卒癉。出話不然,为猶不遠。

靡聖管管,不实于亶。猶之未遠,是用大諫。

天之方难,无然憲憲;天之方蹶,无然泄泄。

辭之輯矣,民之洽矣;辭之懌矣,民之莫矣。

我虽異事,及尔同寮。我即尔謀,聽我囂囂。

我言维服,勿以为笑。先民有言:詢于芻蕘。

天之方虐,无然謔謔。老夫灌灌,小子蹻蹻。

匪我言耄,尔用憂謔。多將熇熇,不可救藥。

天之方懠,无为夸毗。威儀卒迷,善人载尸。

民之方殿屎,則莫我敢葵。喪亂蔑資,曾莫惠我師。

天之牖民,如塤如篪,如璋如圭,如取如攜。

攜无曰益,牖民孔易。民之多辟,无自立辟。

價人维藩,大師维垣,大邦维屏,大宗维翰。

懷德维寧,宗子维城。无俾城壞,无獨斯畏。

敬天之怒,无敢戲豫;敬天之渝,无敢馳驅。

昊天曰明,及尔出王;昊天曰旦,及尔游衍。




大雅·荡之什


荡

荡荡上帝,下民之辟。疾威上帝,其命多辟。

天生烝民,其命匪諶。靡不有初,鮮克有終。

文王曰:咨!咨女殷商。曾是強御,曾是掊克,

曾是在位,曾是在服。天降滔德,女興是力。

文王曰:咨!咨女殷商。而秉義類,強御多懟。

流言以對,寇攘式內。侯作侯祝,靡屆靡究。

文王曰:咨!咨女殷商。女炰烋于中国,斂怨以为德。

不明尔德,时无背无側;尔德不明,以无陪无卿。

文王曰:咨!咨女殷商。天不湎尔以酒,不義從式。

既愆尔止,靡明靡晦。式號式呼,俾晝作夜。

文王曰:咨!咨女殷商。如蜩如螗,如沸如羹。

小大近喪,人尚乎由行。內奰于中国,覃及鬼方。

文王曰:咨!咨女殷商。匪上帝不时,殷不用旧。

虽无老成人,尚有典刑。曾是莫聽,大命以傾。

文王曰:咨!咨女殷商。人亦有言:顛沛之揭,

枝葉未有害,本实先撥。殷鑒不遠,在夏後之世!

抑

抑抑威儀,维德之隅。人亦有言:靡哲不愚。庶人之愚,亦職维疾;哲人之愚,亦维斯戾。

无競维人,四方其訓之;有覺德行,四国順之。訏謨定命,遠猶辰告。敬慎威儀,维民之則。

其在于今,興迷亂于政;顛覆厥德,荒湛于酒。女虽湛乐從。弗念厥紹,罔敷求先王,克共明刑。

肆皇天弗尚,如彼泉流,无淪胥以亡。夙興夜寐,灑掃庭內,维民之章。

修尔車马,弓矢戎兵,用戒戎作,用逖蠻方。質尔人民,謹尔侯度,用戒不虞。

慎尔出話,敬尔威儀,无不柔嘉。白圭之玷,尚可磨也;斯言之玷,不可为也。

无易由言,无曰茍矣;莫捫朕舌,言不可逝矣。无言不讎,无德不報。

惠于朋友,庶民小子。子孙繩繩,万民靡不承。視尔友君子,輯柔尔顏,不遐有愆。

相在尔室,尚不愧于屋漏。无曰不显,莫予云覯。神之格思,不可度思,矧可射思?

辟尔为德,俾臧俾嘉。淑慎尔止,不愆于儀。不僭不賊,鮮不为則。

投我以桃,報之以李。彼童而角,实虹小子。荏染柔木,言緡之丝。

溫溫恭人,维德之基。其维哲人,告之話言,順德之行;

其维愚人,覆謂我僭:民各有心。於乎小子!未知臧否。匪手攜之,言示之事;

匪面命之,言提其耳。借曰未知,亦既抱子。民之靡盈,誰夙知而莫成?

昊天孔昭,我生靡乐。視尔夢夢,我心慘慘。誨尔諄諄,聽我藐藐。

匪用为教,覆用为虐。借曰未知,亦聿既耄。

於乎小子!告尔旧止。聽用我謀,庶无大悔。天方艱难,曰喪厥国。

取譬不遠,昊天不忒。回遹其德,俾民大棘。

桑柔

菀彼桑柔,其下侯旬。捋采其刘,瘼此下民。不殄心憂,倉兄填兮;

倬彼昊天,寧不我矜。四牡騤騤,旟旐有翩。亂生不夷,靡国不泯。

民靡有黎,具禍以燼。於乎有哀!国步斯頻。国步蔑資,天不我將;

靡所止疑,云徂何往?君子实维,秉心无競。誰生厲階?至今为梗。

憂心殷殷,念我土宇。我生不辰,逢天僤怒。自西徂東,靡所定處;

多我覯痻,孔棘我圉。为謀为毖,亂況斯削。告尔憂恤,誨尔序爵。

誰能執熱,逝不以濯?其何能淑?载胥及溺。如彼溯風,亦孔之僾;

民有肅心,荓云不逮。好是稼穡,力民代食;稼穡维寶,代食维好。

天降喪亂,滅我立王。降此蟊賊,稼穡卒癢。哀恫中国,具贅卒荒;

靡有旅力,以念穹蒼。维此惠君,民人所瞻。秉心宣猶,考慎其相。

维彼不順,自獨俾臧。自有肺腸,俾民卒狂。

瞻彼中林,甡甡其鹿。朋友已譖,不胥以穀。人亦有言:進退维穀。

维此聖人,瞻言百里;维彼愚人,覆狂以喜。匪言不能,胡斯畏忌。

维此良人,弗求弗迪;维彼忍心,是顧是复。民之貪亂,寧为荼毒!

大風有隧,有空大穀。维此良人,作为式穀;维彼不順,征以中垢。

大風有隧,貪人敗類。聽言則對,誦言如醉。匪用其良,覆俾我悖。

嗟尔朋友!予豈不知而作?如彼飞蟲,时亦弋獲。既之陰女,反予來赫。

民之罔極,職涼善背;为民不利,如云不克。民之回遹,職競用力。

民之未戾,職盜为寇。涼曰不可,覆背善詈。虽曰匪予,既作尔歌。

云汉

倬彼云汉,昭回于天。王曰:於乎!何辜今之人?天降喪亂,饑饉薦臻。靡神不舉,靡愛斯牲。圭璧既卒,寧莫我聽!

旱既太甚,蘊隆蟲蟲。不殄禋祀,自郊徂宮。上下奠瘞,靡神不宗。后稷不克,上帝不臨;耗斁下土,寧丁我躬!

旱既太甚,則不可推。兢兢業業,如霆如雷。周餘黎民,靡有孑遺。昊天上帝,則不我遺。胡不相畏?先祖于摧。

旱既太甚,則不可沮。赫赫炎炎,云我无所。大命近止,靡瞻靡顧。群公先正,則不我助。父母先祖,胡寧忍予?

旱既太甚,滌滌山川。旱魃为虐,如惔如焚。我心憚暑,憂心如薰。群公先正,則不我聞。昊天上帝,寧俾我遁!

旱既太甚,黽勉畏去。胡寧瘨我以旱?憯不知其故。祈年孔夙,方社不莫。昊天上帝,則不我虞。敬恭明神,宜无悔怒。

旱既太甚,散无友紀。鞫哉庶正,疚哉冢宰。趣马師氏,膳夫左右;靡人不周,无不能止。瞻卬昊天,云如何里?

瞻卬昊天,有嘒其星。大夫君子,昭假无贏。大命近止,无棄尔成。何求为我?以戾庶正。瞻卬昊天,曷惠其寧?

崧高

崧高维岳,駿極于天。维岳降神,生甫及申。

维申及甫,维周之翰。四国于蕃,四方于宣。

亹亹申伯,王纘之事。于邑于謝,南国是式。

王命召伯,定申伯之宅。登是南邦,世執其功。

王命申伯,式是南邦,因是謝人,以作尔庸。

王命召伯,徹申伯土田;王命傅御,遷其私人。

申伯之功,召伯是營。有俶其城,寢庙既成,既成藐藐;

王錫申伯,四牡蹻蹻,鉤膺濯濯。

王遣申伯,路車乘马。我圖尔居,莫如南土。

錫尔介圭,以作尔寶。往近王舅,南土是保。

申伯信邁,王餞于郿。申伯還南,謝于誠归。

王命召伯,徹申伯土疆,以峙其粻,式遄其行。

申伯番番,既入于謝,徒御嘽嘽。周邦咸喜,戎有良翰。不显申伯,王之元舅,文武是憲。

申伯之德,柔惠且直。揉此万邦,聞于四国。吉甫作誦,其詩孔碩;其風肆好,以贈申伯。

烝民

天生烝民,有物有則。民之秉彝,好是懿德。天監有周,昭假于下。保茲天子,生仲山甫。

仲山甫之德,柔嘉维則。令儀令色,小心翼翼;古訓是式,威儀是力。天子是若,明命使賦。

王命仲山甫:式是百辟,纘戎祖考,王躬是保,出納王命。王之喉舌,賦政于外,四方爰发。

肅肅王命,仲山甫將之;邦国若否,仲山甫明之。既明且哲,以保其身。夙夜匪解,以事一人。

人亦有言:柔則茹之,刚則吐之。维仲山甫,柔亦不茹,刚亦不吐;不侮矜寡,不畏強御。

人亦有言:德輶如毛,民鮮克舉之,我儀圖之。维仲山甫舉之,愛莫助之。袞職有闕,维仲山甫補之。

仲山甫出祖,四牡業業,征夫捷捷,每懷靡及。四牡彭彭,八鸞鏘鏘,王命仲山甫,城彼東方。

四牡騤騤,八鸞喈喈,仲山甫徂齐,式遄其归。吉甫作誦,穆如清風。仲山甫永懷,以慰其心。

韓奕

奕奕梁山,维禹甸之,有倬其道。韓侯受命,王親命之:纘戎祖考。无廢朕命,夙夜匪解,虔共尔位。朕命不易,干不庭方,以佐戎辟。

四牡奕奕,孔修且张,韓侯入覲,以其介圭,入覲于王。王錫韓侯:淑旗綏章,簟茀錯衡,玄袞赤舄,鉤膺鏤钖,鞹鞃淺幭,鞗革金厄。

韓侯出祖,出宿于屠。显父餞之,清酒百壺。其殽维何?炰鱉鮮鱼。其蔌维何?维筍及蒲。其贈维何?乘马路車。籩豆有且,侯氏燕胥。

韓侯取妻,汾王之甥,蹶父之子。韓侯迎止,于蹶之里。百兩彭彭,八鸞鏘鏘,不显其光。諸娣從之,祁祁如云。韓侯顧之,爛其盈門。

蹶父孔武,靡国不到。为韓姞相攸,莫如韓乐。孔乐韓土,川澤訏訏,魴鱮甫甫,麀鹿噳噳,有熊有羆,有貓有虎。慶既令居,韓姞燕譽。

溥彼韓城,燕師所完。以先祖受命,因时百蠻。王錫韓侯,其追其貊,奄受北国,因以其伯。实墉实壑,实畝实籍。獻其貔皮,赤豹黃羆。

江汉

江汉浮浮,武夫滔滔。匪安匪游,淮夷來求。

既出我車,既設我旟,匪安匪舒,淮夷來鋪。

江汉汤汤,武夫洸洸。經營四方,告成于王。

四方既平,王国庶定。时靡有爭,王心载寧。

江汉之滸,王命召虎,式辟四方,徹我疆土。

匪疚匪棘,王国來極。于疆于理,至于南海。

王命召虎,來旬來宣;文武受命,召公维翰。

无曰:予小子,召公是似。肇敏戎公,用錫尔祉。

厘尔圭瓚,秬鬯一卣,告于文人。錫山土田,

于周受命,自召祖命。虎拜稽首,天子万年。

虎拜稽首,對揚王休。作召公考,天子万寿。

明明天子,令聞不已;矢其文德,洽此四国。

常武

赫赫明明,王命卿士,南仲大祖,大師皇父。

整我六師,以修我戎。既敬既戒,惠此南国。

王謂尹氏,命程伯休父,左右陳行,戒我師旅:

率彼淮浦,省此徐土,不留不處,三事就緒。

赫赫業業,有嚴天子,王舒保作。匪紹匪游,

徐方繹騷。震驚徐方,如雷如霆,徐方震驚。

王奮厥武,如震如怒。進厥虎臣,闞如虓虎。

鋪敦淮濆,仍執醜虜。截彼淮浦,王師之所。

王旅嘽嘽,如飞如翰,如江如汉。如山之苞,

如川之流。绵绵翼翼,不測不克,濯征徐国。

王猶允塞,徐方既來。徐方既同,天子之功。

四方既平,徐方來庭。徐方不回,王曰:還归。

瞻卬

瞻卬昊天,則不我惠。孔填不寧,降此大厲。邦靡有定,

士民其瘵。蟊賊蟊疾,靡有夷屆。罪罟不收,靡有夷瘳。

人有土田,女反有之;人有民人,女覆奪之。此宜无罪,

女反收之;彼宜有罪,女覆說之。哲夫成城,哲婦傾城。

懿厥哲婦,为梟为鴟。婦有长舌,维厲之階。

亂匪降自天,生自婦人。匪教匪誨,时维婦寺。

鞫人忮忒,譖始竟背。豈曰不極?伊胡为慝!

如賈三倍,君子是識。婦无公事,休其蠶織。

天何以刺?何神不富?舍尔介狄,维予胥忌。

不吊不祥,威儀不類。人之云亡,邦国殄瘁。

天之降罔,维其优矣。人之云亡,心之憂矣。

天之降罔,维其幾矣。人之云亡,心之悲矣。

觱沸檻泉,维其深矣。心之憂矣,寧自今矣。

不自我先,不自我後。藐藐昊天,无不克鞏。

无忝皇祖,式救尔後。

召旻

旻天疾威,天篤降喪,瘨我饑饉,民卒流亡。我居圉卒荒。

天降罪罟,蟊賊內訌。昏椓靡共,潰潰回遹,实靖夷我邦。

皋皋訾訾,曾不知其玷。兢兢業業,孔填不寧,我位孔貶。

如彼歲旱,草不潰茂,如彼棲苴。我相此邦,无不潰止。

维昔之富,不如时;维今之疚,不如茲。彼疏斯粺,胡不自替,職兄斯引?

池之竭矣,不云自頻?泉之竭矣,不云自中?溥斯害矣,職兄斯弘,不災我躬?

昔先王受命,有如召公,日辟国百里;今也日蹙国百里。

於乎哀哉!维今之人,不尚有旧。




周頌·清庙之什

清庙

於穆清庙,肅雍显相。济济多士,秉文之德。對越在天,駿奔走在庙。不显不承?无射於人斯。

维天之命

维天之命,於穆不已。於乎不显!文王之德之純。假以溢我,我其收之。駿惠我文王,曾孙篤之。

维清

维清緝熙,文王之典。肇禋。迄用有成,维周之禎。《维清》

烈文

烈文辟公,錫茲祉福,惠我无疆,子孙保之。无封靡于尔邦,维王其崇之。念茲戎功,继序其皇之。无競维人,四方其訓之。不显维德,百辟其刑之。於乎!前王不忘。

天作

天作高山,大王荒之。彼作矣,文王康之。彼徂矣,岐有夷之行。子孙保之。

昊天有成命

昊天有成命,二後受之。成王不敢康,夙夜基命宥密。於緝熙,單厥心,肆其靖之。

我將

我將我享,维羊维牛,维天其右之。儀式刑文王之典,日靖四方。伊嘏文王,既右饗之。我其夙夜,畏天之威,于时保之。

时邁

时邁其邦,昊天其子之,实右序有周。薄言震之,莫不震疊。懷柔百神,及河喬岳。允王维後。

明昭有周,式序在位。载戢干戈,载櫜弓矢。我求懿德,肆于时夏,允王保之。

執競

執競武王,无競维烈。不显成康?上帝是皇。自彼成康,奄有四方,斤斤其明。鐘鼓喤喤,磬筦將將,降福穰穰。降福簡簡,威儀反反。既醉既飽,福祿來反。

思文

思文后稷,克配彼天。立我烝民,莫匪尔極。貽我來牟,帝命率育,无此疆尔界,陳常于时夏。




周颂·臣工之什


臣工

嗟嗟臣工,敬尔在公。王厘尔成,來咨來茹。嗟嗟保介,维莫之春。亦又何求?如何新畬?於皇來牟,將受厥明。明昭上帝,迄用康年。命我眾人,庤乃錢镈,奄觀铚艾。

噫嘻

噫嘻成王,既昭假尔。率时農夫,播厥百穀。駿发尔私,終三十里。亦服尔耕,十千维耦。

振鷺

振鷺于飞,于彼西雍。我客戾止,亦有斯容。在彼无惡,在此无斁。庶幾夙夜,以永終譽。

丰年

丰年多黍多稌,亦有高廩,万億及秭。为酒为醴,烝畀祖妣,以洽百禮。降福孔皆。

有瞽

有瞽有瞽,在周之庭。設業設虡,崇牙樹羽,应田縣鼓,鞉磬柷圉。既備乃奏,簫管備舉。喤喤厥声,肅雍和鳴,先祖是聽。我客戾止,永觀厥成。

潛

猗与漆沮,潛有多鱼。有鳣有鮪,鰷鲿鰋鯉。以享以祀,以介景福。

雍

有來雍雍,至止肅肅。相维辟公,天子穆穆。於薦廣牡,相予肆祀。假哉皇考,綏予孝子。宣哲维人,文武维後。燕及皇天,克昌厥後。綏我眉寿,介以繁祉。既右烈考,亦右文母。

载见

载见辟王,曰求厥章。龙旗陽陽,和鈴央央,鞗革有鸧,休有烈光。率见昭考,以孝以享,以介眉寿。永言保之,思皇多祜。烈文辟公,綏以多福,俾緝熙于純嘏。

有客

有客有客,亦白其马。有萋有且,敦琢其旅。有客宿宿,有客信信。言授之縶,以縶其马。薄言追之,左右綏之。既有淫威,降福孔夷。

武

於皇武王,无競维烈。允文文王,克開厥後。嗣武受之,勝殷遏刘,耆定尔功。




周颂·闵予小子之什


闵予小子

闵予小子,遭家不造,嬛嬛在疚。於乎皇考!永世克孝。

念茲皇祖,陟降庭止。维予小子,夙夜敬止。於乎皇王!继序思不忘。

访落

访予落止,率时昭考。於乎悠哉!朕未有艾,將予就之,继猶判渙。维予小子,未堪家多难。紹庭上下,陟降厥家。休矣皇考,以保明其身。

敬之

敬之敬之,天维显思。命不易哉!无曰:高高在上。陟降厥士,日監在茲。维予小子,不聰敬止。日就月將,學有緝熙于光明。佛时仔肩,示我显德行。

小毖

予其懲,而毖後患。莫予荓蜂,自求辛螫。肇允彼桃蟲,拚飞维鸟。未堪家多难,予又集于蓼。

载芟

载芟载柞,其耕澤澤。千耦其耘,徂隰徂畛。侯主侯伯,侯亞侯旅,侯強侯以。有嗿其馌,思媚其婦,有依其士。

有略其耜,俶载南畝,播厥百穀,实函斯活。驛驛其达,有厭其杰,厭厭其苗,绵绵其麃。载獲济济,有实其積,万億及秭。

为酒为醴,烝畀祖妣,以洽百禮。有飶其香,邦家之光。有椒其馨,胡考之寧?匪且有且,匪今斯今,振古如茲。

良耜

畟畟良耜,俶载南畝,播厥百穀,实函斯活。或來瞻女,载筐及筥。其馕伊黍,其笠伊糾,其镈斯趙,以薅荼蓼。荼蓼朽止,黍稷茂止。獲之挃挃,積之栗栗。其崇如墉,其比如櫛。以開百室。百室盈止,婦子寧止。殺时犉牡,有捄其角。以似以續,續古之人。

丝衣

丝衣其紑,载弁俅俅。自堂徂基,自羊徂牛。鼐鼎及鼒。兕觥其觩,旨酒思柔。不吳不敖,胡考之休?

酌

於鑠王師,遵養时晦。时純熙矣,是用大介。我龙受之,蹻蹻王之造。载用有嗣,实维尔公。允師。

桓

綏万邦,婁丰年,天命匪解。桓桓武王,保有厥士,于以四方,克定厥家。於昭于天,皇以間之。

賚

文王既勤止,我应受之,敷时繹思。我徂维求定,时周之命。於繹思。

般

於皇时周,陟其高山。嶞山喬岳,允猶翕河。敷天之下,裒时之對,时周之命。




鲁颂·駉之什


駉

駉駉牡马,在坰之野。薄言駉者,有驈有皇,有驪有黃,以車彭彭。思无疆,思马斯臧。

駉駉牡马,在坰之野。薄言駉者,有騅有駓,有騂有騏,以車伾伾。思无期,思马斯才。

駉駉牡马,在坰之野。薄言駉者,有驒有駱,有騮有雒。以車繹繹。思无期,思马斯作。

駉駉牡马,在坰之野。薄言駉者,有駰有騢,有駠有鱼,以車繹繹。思无邪,思马斯徂。

有駜

有駜有駜,駜彼乘黃。夙夜在公,在公明明。振振鷺,鷺于下。鼓咽咽,醉言舞。于胥乐兮。

有駜有駜,駜彼乘牡。夙夜在公,在公飲酒。振振鷺,鷺于飞。鼓咽咽,醉言归。于胥乐兮。

有駜有駜,駜彼乘駽。夙夜在公,在公载燕。自今以始,歲其有。君子有穀,詒孙子。于胥乐兮。

泮水

思乐泮水,薄采其芹。鲁侯戾止,言觀其旗。其旗茷茷,鸞声噦噦。无小无大,從公于邁。

思乐泮水,薄采其藻。鲁侯戾止,其马蹻蹻。其马昭昭,其音昭昭。载色载笑,匪怒伊教。

思乐泮水,薄采其茆。鲁侯戾止,在泮飲酒,既飲旨酒,永錫难老。順彼长道,屈此群醜。

穆穆鲁侯,敬明其德。敬慎威儀,维民之則。允文允武,昭假烈祖。靡有不孝,自求伊祜。

明明鲁侯,克明其德,既作泮宮,淮夷攸服。矯矯虎臣,在泮獻馘;淑問如皋陶,在泮獻囚。

济济多士,克廣德心。桓桓于征,狄彼東南。烝烝皇皇,不吳不揚。不告于讻,在泮獻功。

角弓其觩,束矢其搜。戎車孔博,徒御无斁。既克淮夷,孔淑不逆。式固尔猶,淮夷卒獲。

翩彼飞鸮,集于泮林,食我桑黮,懷我好音。憬彼淮夷,來獻其琛:元龜象齿,大賂南金。

閟宮

閟宮有恤,实实枚枚。赫赫姜嫄,其德不回。上帝是依,无災无害;彌月不迟,是生后稷。降之百福,黍稷重穋,稙稚菽麥。奄有下国,俾民稼穡。有稷有黍,有稻有秬。奄有下土,纘禹之緒。后稷之孙,实维大王;居岐之陽,实始翦商。至于文武,纘大王之緒。致天之屆,于牧之野。

无貳无虞,上帝臨女。敦商之旅,克咸厥功。王曰:叔父!建尔元子,俾侯于鲁;大啟尔宇,为周室輔。乃命鲁公,俾侯于東;錫之山川,土田附庸。

周公之孙,莊公之子,龙旗承祀,六轡耳耳。春秋匪解,享祀不忒;皇皇后帝,皇祖后稷,享以騂犧。是饗是宜,降福既多。周公皇祖,亦其福女。

秋而载嘗,夏而楅衡。白牡騂刚,犧尊將將。毛炰胾羹,籩豆大房;万舞洋洋,孝孙有慶。俾尔熾而昌,俾尔寿而臧。保彼東方,鲁邦是常。

不虧不崩,不震不騰。三寿作朋,如岡如陵。公車千乘,朱英綠縢,二矛重弓。公徒三万,貝胄朱綅,烝徒增增。戎狄是膺,荊舒是懲,則莫我敢承。

俾尔昌而熾,俾尔寿而富。黃发台背,寿胥与試。俾尔昌而大,俾尔耆而艾。万有千歲,眉寿无有害。泰山巖巖,鲁邦所詹。奄有龜蒙,遂荒大東,至于海邦。

淮夷來同,莫不率從,鲁侯之功。保有鳧繹,遂荒徐宅,至于海邦。淮夷蠻貊,及彼南夷,莫不率從。莫敢不諾,鲁侯是若。天錫公純嘏,眉寿保鲁;居常与許,复周公之宇。鲁侯燕喜,令妻寿母,宜大夫庶士,邦国是有。

既多受祉,黃发儿齿。徂來之松,新甫之柏,是断是度,是尋是尺。松桷有舄,路寢孔碩。新庙奕奕,奚斯所作。孔曼且碩,万民是若。




商頌


那

猗与那与!置我鞉鼓。奏鼓簡簡,衎我烈祖。汤孙奏假,綏我思成。鞉鼓淵淵,嘒嘒管声。既和且平,依我磬声。於赫汤孙,穆穆厥声。庸鼓有斁,万舞有奕。我有嘉客,亦不夷懌。自古在昔,先民有作。溫恭朝夕,執事有恪。顧予烝嘗,汤孙之將。

烈祖

嗟嗟烈祖!有秩斯祜。申錫无疆,及尔斯所。既载清酤,賚我思成。亦有和羹,既戒既平。鬷假无言,时靡有爭。綏我眉寿,黃耇无疆。約軧錯衡,八鸞鸧鸧,以假以享。我受命溥將。自天降康,丰年穰穰。來假來饗,降福无疆。顧予烝嘗,汤孙之將。

玄鸟

天命玄鸟,降而生商。宅殷土芒芒。古帝命武汤,正域彼四方。方命厥後,奄有九有。商之先後,受命不殆,在武丁孙子。武丁孙子,武王靡不勝。龙旗十乘,大糦是承。邦畿千里,维民所止,肇域彼四海。四海來假,來假祁祁。景員维河,殷受命咸宜,百祿是何。

长发

浚哲维商,长发其祥。洪水芒芒,禹敷下土方。外大国是疆,幅隕既长。有娀方將,帝立子生商。

玄王桓撥,受小国是达,受大国是达。率履不越,遂視既发。相土烈烈,海外有截。

帝命不違,至於汤齐。汤降不迟,聖敬日躋。昭假迟迟,上帝是祗。帝命式于九圍。

受小球大球,为下国綴旒,何天之休?不競不絿,不刚不柔,敷政优优,百祿是遒。

受小共大共,为下国駿厖,何天之龙?敷奏其勇。不震不動,不戁不竦,百祿是總。

武王载旆,有虔秉鉞。如火烈烈,則莫我敢曷。苞有三蘗,莫遂莫达,九有有截。韋顧既伐,昆吾夏桀。

昔在中葉,有震且業。允也天子,降予卿士:实维阿衡,实左右商王。

殷武

撻彼殷武,奮伐荊楚,穼入其阻,裒荊之旅。有截其所,汤孙之緒。

维女荊楚,居国南乡。昔有成汤,自彼氐羌,莫敢不來享,莫敢不來王。曰商是常。

天命多辟,設都于禹之績。歲事來辟,勿予禍適。稼穡匪解。

天命降監,下民有嚴。不僭不濫,不敢怠遑。命于下国,封建厥福。

商邑翼翼,四方之極。赫赫厥声,濯濯厥灵。寿考且寧,以保我後生。

陟彼景山,松柏丸丸。是断是遷,方斫是虔。松桷有梴,旅楹有閑,寢成孔安!

有饛簋飧,有捄棘匕。周道如砥,其直如矢。君子所履,小人所视。眷言顾之,潸焉出涕。

小东大东,杼柚其空。纠纠葛屦,可以履霜。佻佻公子,行彼周行。既往既来,使我心疚。

有冽氿泉,无浸获薪。契契寤叹,哀我惮人。薪是获薪,尚可载也。哀我惮人,亦可息也。

东人之子,职劳不来。西人之子,粲粲衣服。舟人之子,熊罴是裘。私人之子,百僚是试。

或以其酒,不以其浆。鞙鞙佩璲,不以其长。维天有汉,监亦有光。跂彼织女,终日七襄。

虽则七襄,不成报章。彼牵牛,不以服箱。东有启明,西有长?庚。有捄天毕,载施之行。

维南有箕,不可以簸扬。维北有斗,不可以挹酒浆。维南有箕,载翕其舌。维北有斗,西柄之揭。

四月

四月维夏,六月徂署。先祖匪人,胡宁忍予?

秋日凄凄,百卉具腓。乱离瘼矣,爰其适归?

冬日烈烈,飘风发发。民莫不穀,我独何害?

山有嘉卉,侯栗侯梅。废为残贼,莫知其尤!

相彼泉水,载清载浊。我日构祸,曷云能穀?

滔滔江汉,南国之纪。尽瘁以仕,宁莫我有?

匪鹑匪鸢,翰飞戾天。匪鳣匪鲔,潜逃于渊。

山有蕨薇,隰有杞桋。君子作歌,维以告哀。

北山

陟彼北山,言采其杞。偕偕士子,朝夕从事。王事靡盬,忧我父母。

溥天之下,莫非王土;率土之滨,莫非王臣。大夫不均,我从事独贤。

四牡彭彭,王事傍傍。嘉我未老,鲜我方将。旅力方刚,经营四方。

或燕燕居息,或尽瘁事国;或息偃在床,或不已于行。

或不知叫号,或惨惨劬劳;或栖迟偃仰,或王事鞅掌。

或湛乐饮酒,或惨惨畏咎;或出入风议,或靡事不为。

无将大车

无将大车,祇自尘兮。无思百忧,祇自疧兮。

无将大车,维尘冥冥。无思百忧,不出于颎。

无将大车,维尘雍兮。无思百忧,祇自重兮。

小明

明明上天,照临下土。我征徂西,至于艽野。二月初吉,载离寒暑。心之忧矣,其毒

大苦。念彼共人,涕零如雨。岂不怀归?畏此罪罟!

昔我往矣,日月方除。曷云其还?岁聿云莫。念我独兮,我事孔庶。心之忧矣,惮我

不暇。念彼共人,眷眷怀顾!岂不怀归?畏此谴怒。

昔我往矣,日月方奥。曷云其还?政事愈蹙。岁聿云莫,采萧获菽。心之忧矣,自诒

伊戚。念彼共人,兴言出宿。岂不怀归?畏此反覆。

嗟尔君子,无恒安处。靖共尔位,正直是与。神之听之,式穀以女。

嗟尔君子,无恒安息。靖共尔位,好是正直。神之听之,介尔景福。

鼓钟

鼓钟将将,淮水汤汤,忧心且伤。淑人君子,怀允不忘。

鼓钟喈喈,淮水湝湝,忧心且悲。淑人君子,其德不回。

鼓钟伐鼛,淮有三洲,忧心且妯。淑人君子,其德不犹。

鼓钟钦钦,鼓瑟鼓琴,笙磬同音。以雅以南,以龠不僭。

楚茨

楚楚者茨,言抽其棘,自昔何为?我蓺黍稷。我黍与与,我稷翼翼。

我仓既盈,我庾维亿。以为酒食,以享以祀,以妥以侑,以介景福。

济济跄跄,絜尔牛羊,以往烝尝。或剥或亨,或肆或将。祝祭于祊,祀事孔明。先祖

是皇,神保是飨。孝孙有庆,报以介福,万寿无疆!

执爨 ,为俎孔硕,或燔或炙。君妇莫莫,为豆孔庶。为宾为客,献酬交错。礼仪卒

度,笑语卒获。神保是格,报以介福,万寿攸酢!

我孔 矣,式礼莫愆。工祝致告,徂赉孝孙。苾芬孝祀,神嗜饮食。卜尔百福,如几

如式。既齐既稷,既匡既敕。永锡尔极,时万时亿!

礼仪既备,钟鼓既戒,孝孙徂位,工祝致告,神具醉止,皇尸载起。鼓钟送尸,神保

聿归。诸宰君妇,废彻不迟。诸父兄弟,备言燕私。

乐具入奏,以绥后禄。尔肴既将,莫怨具庆。既醉既饱,小大稽首。神嗜饮食,使君

寿考。孔惠孔时,维其尽之。子子孙孙,勿替引之!

信南山

信彼南山,维禹甸之。畇畇原隰,曾孙田之。我疆我理,南东其亩。

上天同云。雨雪雰雰,益之以霡霂。既优既渥,既沾既足。生我百谷。

疆场翼翼,黍稷彧彧。曾孙之穑,以为酒食。畀我尸宾,寿考万年。

中田有庐,疆场有瓜。是剥是菹,献之皇祖。曾孙寿考,受天之祜。

祭以清酒,从以骍牡,享于祖考。执其鸾刀,以启其毛,取其血膋。

是烝是享,苾苾芬芬。祀事孔明,先祖是皇。报以介福。万寿无疆。

 

 
小雅·甫田之什


甫田 

倬彼甫田,岁取十千。我取其陈,食我农人。自古有年。今适南亩,或耘或耔。黍稷薿薿,攸介攸止,烝我髦士。

以我齐明,与我牺羊,以社以方。我田既臧,农夫之庆。琴瑟击鼓,以御田祖。以祈甘雨,以介我稷黍,以穀我士女。

曾孙来止,以其妇子。馌彼南亩,田畯至喜。攘其左右,尝其旨否。禾易长亩,终善且有。曾孙不怒,农夫克敏。

曾孙之稼,如茨如梁。曾孙之庾,如坻如京。乃求千斯仓,乃求万斯箱。黍稷稻粱,农夫之庆。报以介福,万寿无疆。

大田

大田多稼,既种既戒,既备乃事。以我覃耜,俶载南亩。播厥百谷,既庭且硕,曾孙是若。

既方既皂,既坚既好,不稂不莠。去其螟螣,及其蟊贼,无害我田稚。田祖有神,秉畀炎火。

有渰萋萋,兴雨祈祈。雨我公田,遂及我私。彼有不获稚,此有不敛穧,彼有遗秉,此有滞穗,伊寡妇之利。

曾孙来止,以其妇子。馌彼南亩,田畯至喜。来方禋祀,以其骍黑,与其黍稷。以享以祀,以介景福。

瞻彼洛矣

瞻彼洛矣,维水泱泱。君子至止,福禄如茨。韎韐有奭,以作六师。

瞻彼洛矣,维水泱泱。君子至止,鞸琫有珌。君子万年,保其家室。

瞻彼洛矣,维水泱泱。君子至止,福禄既同。君子万年,保其家邦。

裳裳者华

裳裳者华,其叶湑□兮。我觏之子,我心写兮。我心写兮,是以有誉处兮。

裳裳者华,芸其黄矣。我觏之子,维其有章矣。维其有章矣,是以有庆矣。

裳裳者华,或黄或白。我觏之子,乘其四骆。乘其四骆,六辔沃若。

左之左之,君子宜之。右之右之,君子有之。维其有之,是以似之。

桑扈

交交桑扈,有莺其羽。君子乐胥,受天之祜。

交交桑扈,有莺其领。君子乐胥,万邦之屏。

之屏之翰,百辟为宪。不戢不难,受福不那。

兕觥其觩,旨酒思柔。彼交匪敖,万福来求。

鸳鸯

鸳鸯于飞,毕之罗之。君子万年,福禄宜之。

鸳鸯在梁,戢其左翼。君子万年,

宜其遐福。乘马在厩,摧之秣之。君子万年,福禄艾之。

乘马在厩,秣之摧之。君子万年,福禄绥之。

頍弁

有頍者弁,实维伊何?尔酒既旨,尔肴既嘉。岂伊异人?兄弟匪他。茑与女萝,施于松柏。未见君子,忧心奕奕;既见君子,庶几说怿。

有頍者弁,实维何期?尔酒既旨,尔肴既时。岂伊异人?兄弟具来。茑与女萝,施于松上。未见君子,忧心怲怲;既见君子,庶几有臧。

有頍者弁,实维在首。尔酒既旨,尔肴既阜。岂伊异人?兄弟甥舅。如彼雨雪,先集维霰。死丧无日,无几相见。乐酒今夕,君子维宴。

车舝

间关车之舝兮,思娈季女逝兮。匪饥匪渴,德音来括。虽无好友?式燕且喜。

依彼平林,有集维鷮。辰彼硕女,令德来教。式燕且誉,好尔无射。

虽无旨酒?式饮庶几。虽无嘉肴?式食庶几。虽无德与女?式歌且舞?

陟彼高冈,析其柞薪。析其柞薪,其叶湑兮。鲜我觏尔,我心写兮。

高山仰止,景行行止。四牡騑騑,六辔如琴。觏尔新婚,以慰我心。

青蝇

营营青蝇,止于樊。岂弟君子,无信谗言。

营营青蝇,止于棘。谗人罔极,交乱四国。

营营青蝇,止于榛。谗人罔极,构我二人。

宾之初筵

宾之初筵,左右秩秩。笾豆有楚,殽核维旅。酒既和旨,饮酒孔偕。钟鼓既设,举酬逸逸。大侯既抗,弓矢斯张。射夫既同,献尔发功。

发彼有的,以祈尔爵。籥舞笙鼓,乐既和奏。烝衎烈祖,以洽百礼。百礼既至,有壬有林。锡尔纯嘏,子孙其湛。其湛曰乐,各奏尔能。宾载手仇,室人入又。酌彼康爵,以奏尔时。

宾之初筵,温温其恭。其未醉止,威仪反反。曰既醉止,威仪幡幡。舍其坐迁,屡舞仙仙。其未醉止,威仪抑抑。曰既醉止,威仪抑抑。是曰既醉,不知其秩。

宾既醉止,载号载呶。乱我笾豆,屡舞僛僛。是曰既醉,不知其邮。侧弁之俄,屡舞傞傞。既醉而出,并受其福。

醉而不出,是谓伐德。饮酒孔嘉,维其令仪。凡此饮酒,或醉或否。既立之监,或佐之史。彼醉不臧,不醉反耻。

式勿从谓,无俾大怠。匪言勿言,匪由勿语。由醉之言,俾出童羖。三爵不识,矧敢多又。




小雅·鱼藻之什


鱼藻

鱼在在藻,有颁其首。王在在镐,岂乐饮酒。

鱼在在藻,有莘其尾。王在在镐,饮酒乐岂。

鱼在在藻,依于其蒲。王在在镐,有那其居。

采菽

采菽采菽,筐之莒之。君子来朝,何锡予之?虽无予之?路车乘马。又何予之?玄衮及黼。

觱沸槛泉,言采其芹。君子来朝,言观其旂。其旂淠淠,鸾声嘒嘒。载骖载驷,君子所届。

赤芾在股,邪幅在下。彼交匪纾,天子所予。乐只君子,天子命之。乐只君子,福禄申之。

维柞之枝,其叶蓬蓬。乐只君子,殿天子之邦。乐只君子,万福攸同。平平左右,亦是率从。

汎汎杨舟,绋纚维之。乐只君子,天子葵之。乐只君子,福禄膍之。优哉游哉,亦是戾矣。

角弓

骍骍角弓,翩其反矣。兄弟婚姻,无胥远矣。

尔之远矣,民胥然矣。尔之教矣,民胥效矣。

此令兄弟,绰绰有裕。不令兄弟,交相为愈。

民之无良,相怨一方。受爵不让,至于已斯亡。

老马反为驹,不顾其后。如食宜饇,如酌孔取。

毋教猱升木,如涂涂附。君子有徽猷,小人与属。

雨雪瀌瀌,见晛曰消。莫肯下遗,式居娄骄。

雨雪浮浮,见晛曰流。如蛮如髦,我是用忧。

菀柳

有菀者柳,不尚息焉。上帝甚蹈,无自暱焉。俾予靖之,后予极焉。

有菀者柳,不尚愒焉。上帝甚蹈,无自瘵焉。俾予靖之,后予迈焉。

有鸟高飞,亦傅于天。彼人之心,于何其臻。曷予靖之,居以凶矜。

都人士

彼都人士,狐裘黄黄。其容不改,出言有章。行归于周,万民所望。

彼都人士,台笠缁撮。彼君子女,绸直如发。我不见兮,我心不说。

彼都人士,充耳琇实。彼君子女,谓之尹吉。我不见兮,我心苑结。

彼都人士,垂带而厉。彼君子女,卷发如虿。我不见兮,言从之迈。

匪伊垂之,带则有余。匪伊卷之,发则有旟。我不见兮,云何盱矣。

采绿

终朝采绿,不盈一匊。予发曲局,薄言归沐。

终朝采蓝,不盈一襜。五日为期,六日不詹。

之子于狩,言韔其弓。之子于钓,言纶之绳。

其钓维何?维鲂及鱮。维鲂及鱮,薄言观者。

黍苗

芃芃黍苗,阴雨膏之。悠悠南行,召伯劳之。

我任我辇,我车我牛。我行既集,盖云归哉。

我徒我御,我师我旅。我行既集,盖云归处。

肃肃谢功,召伯营之。烈烈征师,召伯成之。

原隰既平,泉流既清。召伯有成,王心则宁。

隰桑

隰桑有阿,其叶有难。既见君子,其乐如何。

隰桑有阿,其叶有沃。既见君子,云何不乐。

隰桑有阿,其叶有幽。既见君子,德音孔胶。

心乎爱矣,遐不谓矣?中心藏之,何日忘之!

白华

白华菅兮,白茅束兮。之子之远,俾我独兮。

英英白云,露彼菅茅。天步艰难,之子不犹。

滮池北流,浸彼稻田。啸歌伤怀,念彼硕人。

樵彼桑薪,卬烘于煁。维彼硕人,实劳我心。

鼓钟于宫,声闻于外。念子懆懆,视我迈迈。

有鹙在梁,有鹤在林。维彼硕人,实劳我心。

鸳鸯在梁,戢其左翼。之子无良,二三其德。

有扁斯石,履之卑兮。之子之远,俾我疧兮。

绵蛮

绵蛮黄鸟,止于丘阿。道之云远,我劳如何。饮之食之,教之诲之。命彼后车,谓之载之。

绵蛮黄鸟,止于丘隅。岂敢惮行,畏不能趋。饮之食之。教之诲之。命彼后车,谓之载之。

绵蛮黄鸟,止于丘侧。岂敢惮行,畏不能极。饮之食之,教之诲之。命彼后车,谓之载之。

瓠叶

幡幡瓠叶,采之亨之。君子有酒,酌言尝之。

有兔斯首,炮之燔之。君子有酒,酌言献之。

有兔斯首,燔之炙之。君子有酒,酌言酢之。

有兔斯首,燔之炮之。君子有酒,酌言酬之。

渐渐之石

渐渐之石,维其高矣。山川悠远,维其劳矣。武人东征,不遑朝矣。

渐渐之石,维其卒矣。山川悠远,曷其没矣?武人东征,不遑出矣。

有豕白蹢,烝涉波矣。月离于毕,俾滂沱矣。武人东征,不皇他矣。

苕之华

苕之华,芸其黄矣。心之忧矣,维其伤矣!

苕之华,其叶青青。知我如此,不如无生!

牂羊坟首,三星在罶。人可以食,鲜可以饱!

何草不黄

何草不黄?何日不行?何人不将?经营四方。

何草不玄?何人不矜?哀我征夫,独为匪民。

匪兕匪虎,率彼旷野。哀我征夫,朝夕不暇。

有芃者狐,率彼幽草。有栈之车,行彼周道。



 
大雅·文王之什


文王

文王在上,於昭于天,周虽旧邦,其命维新。

有周不显,帝命不时。文王陟降,在帝左右。

亹亹文王,令聞不已。陳錫哉周,侯文王孙子。

文王孙子,本支百世。凡周之士,不显亦世。

世之不显,厥猶翼翼。思皇多士,生此王国。

王国克生,维周之楨。济济多士,文王以寧。

穆穆文王,於緝熙敬止。假哉天命,有商孙子。

商之孙子,其麗不億。上帝既命,侯于周服。

侯服于周,天命靡常。殷士膚敏,祼將于京。

厥作祼將,常服黼冔。王之藎臣,无念尔祖。

无念尔祖,聿修厥德。永言配命,自求多福。

殷之未喪師,克配上帝。宜鑒于殷,駿命不易。

命之不易,无遏尔躬。宣昭義問,有虞殷自天。

上天之载,无声无臭。儀刑文王,万邦作孚。

大明

明明在下,赫赫在上。天难忱斯,不易维王。天位殷適,使不挾四方。

摯仲氏任,自彼殷商;來嫁于周,曰嬪于京。乃及王季,维德之行。大任有身,生此文王。

维此文王,小心翼翼。昭事上帝,聿懷多福。厥德不回,以受方国。

天監在下,有命既集。文王初载,天作之合。在洽之陽,在渭之涘。文王嘉止,大邦有子。

大邦有子,伣天之妹。文定厥祥,親迎于渭。造舟为梁,不显其光。

有命自天,命此文王。于周于京。纘女维莘,长子维行。篤生武王,保右命尔,燮伐大商。

殷商之旅,其會如林。矢于牧野,维予侯興。上帝臨女,无貳尔心!

牧野洋洋,檀車煌煌,駟騵彭彭。维師尚父,时维鷹揚;涼彼武王,肆伐大商,會朝清明。

绵

绵绵瓜瓞。民之初生,自土沮漆。古公亶父,陶复陶穴,未有家室。

古公亶父,來朝走马,率西水滸,至于岐下。爰及姜女,聿來胥宇。

周原膴膴,堇荼如飴。爰始爰謀,爰契我龜。曰止曰时,筑室于茲。

乃慰乃止,乃左乃右,乃疆乃理,乃宣乃畝。自西徂東,周爰執事。

乃召司空,乃召司徒,俾立室家。其繩則直,縮版以载,作庙翼翼。

捄之陾陾,度之薨薨,筑之登登,削屢馮馮。百堵皆興,鼛鼓弗勝。

乃立皋門,皋門有伉;乃立应門,应門將將。乃立冢土,戎醜攸行。

肆不殄厥慍,亦不隕厥問,柞棫拔矣,行道兌矣。混夷駾矣,维其喙矣。

虞芮質厥成,文王厥厥生。予曰有疏附,予曰有先後,予曰有奔奏,予曰有御侮。

棫樸

芃芃棫樸,薪之槱之。济济辟王,左右趣之。

济济辟王,左右奉璋。奉璋峨峨,髦士攸宜。

淠彼涇舟,烝徒楫之。周王于邁,六師及之。

倬彼云汉,为章于天。周王寿考,遐不作人?

追琢其章,金玉其相。勉勉我王,綱紀四方。

旱麓

瞻彼旱麓,榛楛济济。豈弟君子,干祿豈弟。

瑟彼玉瓚,黃流在中。豈弟君子,福祿攸降。

鳶飞戾天,鱼躍于淵。豈弟君子,遐不作人?

清酒既载,騂牡既備。以享以祀,以介景福。

瑟彼柞棫,民所燎矣。豈弟君子,神所勞矣。

莫莫葛櫑,施于條枚。豈弟君子,求福不回。

思齐

思齐大任,文王之母。思媚周姜,京室之婦。大姒嗣徽音,則百斯男。

惠于宗公,神罔时怨,神罔时恫。刑于寡妻,至于兄弟,以御于家邦。

雍雍在宮,肅肅在庙。不显亦臨,无射亦保。

肆戎疾不殄,烈假不遐。不聞亦式,不諫亦入。

肆成人有德,小子有造。古人之无斁,譽髦斯士。

皇矣

皇矣上帝,臨下有赫;監觀四方,求民之莫。

维此二国,其政不獲;维彼四国,爰究爰度。

上帝耆之,憎其式廓。乃眷西顧,此维与宅。

作之屏之,其菑其翳;修之平之,其灌其栵;

啟之辟之,其檉其椐;攘之剔之,其檿其柘。

帝遷明德,串夷载路。天立厥配,受命既固。

帝省其山,柞棫斯拔,松柏斯兌。帝作邦作對,自大伯王季。

维此王季,因心則友。則友其兄,則篤其慶,载錫之光。受祿无喪,奄有四方。

维此王季,帝度其心,貊其德音。其德克明,克明克類,克长克君。

王此大邦,克順克比。比于文王,其德靡悔。既受帝祉,施于孙子。

帝謂文王:无然畔援,无然歆羨,誕先登于岸。密人不恭,敢距大邦,侵阮徂共。

王赫斯怒,爰整其旅,以按徂旅,以篤周祜,以對于天下。

依其在京,侵自阮疆,陟我高岡。无矢我陵,我陵我阿;无飲我泉,我泉我池!

度其鮮原,居岐之陽,在渭之將。万邦之方,下民之王。

帝謂文王:予懷明德,不大声以色,不长夏以革,不識不知,順帝之則。

帝謂文王:詢尔仇方,同尔兄弟。以尔鉤援,与尔臨沖,以伐崇墉。

臨沖閑閑,崇墉言言,執訊連連,攸馘安安。是類是祃,是致是附,四方以无侮。

臨沖茀茀,崇墉仡仡,是伐是肆,是絕是忽,四方以无拂。

灵台

經始灵台,經之營之。庶民攻之,不日成之。經始勿亟,庶民子來。

王在灵囿,麀鹿攸伏;麀鹿濯濯,白鸟翯翯。王在灵沼,於牣鱼躍。

虡業维樅,賁鼓维鏞。於論鼓鐘,於乐辟癰。

於論鼓鐘,於乐辟癰。鼉鼓逢逢,矇瞍奏公。

下武

下武维周,世有哲王。三後在天,王配于京。

王配于京,世德作求。永言配命,成王之孚。

成王之孚,下土之式。永言孝思,孝思维則。

媚茲一人,应侯順德。永言孝思,昭哉嗣服。

昭茲來許,繩其祖武。於万斯年,受天之祜。

受天之祜,四方來賀。於万斯年,不遐有佐。

文王有声

文王有声,遹駿有声,遹求厥寧,遹觀厥成。文王烝哉!

文王受命,有此武功;既伐于崇,作邑于丰。文王烝哉!

筑城伊淢,作丰伊匹,匪棘其欲,遹追來孝。王后烝哉!

王公伊濯,维丰之垣。四方攸同,王后维翰。王后烝哉!

丰水東注,维禹之績。四方攸同,皇王维辟。皇王烝哉!

鎬京辟癰,自西自東,自南自北,无思不服。皇王烝哉!

考卜维王,宅是鎬京。维龜正之,武王成之。武王烝哉!

丰水有芑,武王豈不仕?詒厥孙謀,以燕翼子。武王烝哉!




大雅·生民之什


生民

厥初生民,时维姜嫄。生民如何?克禋克祀,以弗无子。履帝武敏歆,攸介攸止;载震载夙,载生载育,时维后稷。

誕彌厥月,先生如达。不坼不副,无菑无害。以赫厥灵,上帝不寧。不康禋祀,居然生子。

誕寘之隘巷,牛羊腓字之。誕寘之平林,會伐平林;誕寘之寒冰,鸟覆翼之。鸟乃去矣,后稷呱矣。实覃实訏,厥声载路。

誕实匍匐,克岐克嶷,以就口食。蓺之荏菽,荏菽旆旆,禾役穟穟,麻麥幪幪,瓜瓞唪唪。

誕后稷之穡,有相之道。茀厥丰草,種之黃茂。实方实苞,实種实褎,实发实秀,实堅实好,实穎实栗,即有邰家室。

誕降嘉種,维秬维秠,维穈维芑。恒之秬秠,是獲是畝;恒之穈芑,是任是負,以归肇祀。

誕我祀如何?或舂或揄,或簸或蹂;釋之叟叟,烝之浮浮。载謀载惟,取蕭祭脂,取羝以軷,载燔载烈。以興嗣歲。

卬盛于豆,于豆于登。其香始升,上帝居歆。胡臭亶时。后稷肇祀,庶无罪悔,以迄于今。

行葦

敦彼行葦,牛羊勿踐履。方苞方體,维葉泥泥。戚戚兄弟,莫遠具尔。或肆之筵,或授之几。

肆筵設席,授几有緝御。或獻或酢,洗爵奠斝。醓醢以薦,或幡或炙。嘉肴脾臄,或歌或咢。

敦弓既堅,四鍭既鈞;舍矢既均,序賓以賢。敦弓既句,既挾四鍭;四鍭如樹,序賓以不侮。

曾孙维主,酒醴维醹,酌以大斗,以祈黃耇。黃耇台背,以引以翼。寿考维祺,以介景福。

既醉

既醉以酒,既飽以德。君子万年,介尔景福。

既醉以酒,尔殽既將。君子万年,介尔昭明。

昭明有融,高朗令終。令終有俶,公尸嘉告。

其告维何?籩豆靜嘉。朋友攸攝,攝以威儀。

威儀孔时,君子有孝子。孝子不匱,永錫尔類。

其類维何?室家之壸。君子万年,永錫祚胤。

其胤维何?天被尔祿。君子万年,景命有仆。

其仆维何?厘尔女士。厘尔女士,從以孙子。

鳧鹥

鳧鹥在涇,公尸來燕來寧。尔酒既清,尔殽既馨。公尸燕飲,福祿來成。

鳧鹥在沙,公尸來燕來宜。尔酒既多,尔殽既嘉。公尸燕飲,福祿來为。

鳧鹥在渚,公尸來燕來處。尔酒既湑,尔殽伊脯。公尸燕飲,福祿來下。

鳧鹥在潀,公尸來燕來宗。既燕于宗,福祿攸降。公尸燕飲,福祿來崇。

鳧鹥在亹,公尸來止熏熏。旨酒欣欣,燔炙芬芬。公尸燕飲,无有後艱。

假乐

假乐君子,显显令德。宜民宜人,受祿于天。保右命之,自天申之。

干祿百福,子孙千億。穆穆皇皇,宜君宜王。不愆不忘,率由旧章。

威儀抑抑,德音秩秩。无怨无惡,率由群匹。受福无疆,四方之綱。

之綱之紀,燕及朋友。百辟卿士,媚于天子。不解于位,民之攸墍。《假乐》

公刘

篤公刘,匪居匪康,乃场乃疆,乃積乃倉。乃裹糇糧,于橐于囊,思輯用光。弓矢斯张,干戈戚揚,爰方啟行。

篤公刘,于胥斯原。既庶既繁。既順乃宣,而无詠嘆。陟則在巘,复降在原。何以舟之?维玉及瑤,鞞琫容刀。

篤公刘,逝彼百泉,瞻彼溥原。乃陟南岡,乃覯于京。京師之野,于时處處,于时廬旅。于时言言,于时語語。

篤公刘,于京斯依。蹌蹌济济,俾筵俾几。既登乃依,乃造其曹;執豕于牢,酌之用匏。食之飲之,君之宗之。

篤公刘,既溥既长。既景乃岡,相其陰陽,觀其流泉。其軍三單,度其隰原,徹田为糧。度其夕陽,豳居允荒。

篤公刘,于豳斯館。涉渭为亂,取厲取鍛。止基乃理,爰眾爰有。夾其皇澗,溯其過澗。止旅乃密,芮鞫之即。

泂酌

泂酌彼行潦,挹彼注茲,可以餴饎。豈弟君子,民之父母。

泂酌彼行潦,挹彼注茲,可以濯罍。豈弟君子,民之攸归。

泂酌彼行潦,挹彼注茲,可以濯溉。豈弟君子,民之攸墍。

卷阿

有卷者阿,飄風自南。豈弟君子,來游來歌,以矢其音。

伴奐尔游矣,优游尔休矣。豈弟君子,俾尔彌尔性,似先公酋矣。

尔土宇皈章,亦孔之厚矣。豈弟君子,俾尔彌尔性,百神尔主矣。

尔受命长矣,茀祿尔康矣。豈弟君子,俾尔彌尔性,純嘏尔常矣。

有馮有翼,有孝有德,以引以翼。豈弟君子,四方为則。

颙颙卬卬,如圭如璋,令聞令望。豈弟君子,四方为綱。

鳳皇于飞,翙翙其羽,亦集爰止。藹藹王多吉士,维君子使,媚于天子。

鳳皇于飞,翙翙其羽,亦傅于天。藹藹王多吉人,维君子命,媚于庶人。

鳳皇鳴矣,于彼高岡。梧桐生矣,于彼朝陽。菶菶萋萋,雍雍喈喈。

君子之車,既庶且多;君子之马,既閑且馳。矢詩不多,维以遂歌。

民勞

民亦勞止,汔可械。惠此中国,以綏四方。无縱詭隨,以謹无良。式遏寇虐,憯不畏明。柔遠能邇,以定我王。

民亦勞止,汔可小休。惠此中国,以为民逑。无縱詭隨,以謹惛怓。式遏寇虐,无俾民憂。无棄尔勞,以为王休。

民亦勞止,汔可小息。惠此京師,以綏四国。无縱詭隨,以謹罔極。式遏寇虐,无俾作慝。敬慎威儀,以近有德。

民亦勞止,汔可小愒。惠此中国,俾民憂泄。无縱詭隨,以謹醜厲。式遏寇虐,无俾正敗。戎虽小子,而式弘大。

民亦勞止,汔可小安。惠此中国,国无有殘。无縱詭隨,以謹繾綣。式遏寇虐,无俾正反。王欲玉女,是用大諫。

板

上帝板板,下民卒癉。出話不然,为猶不遠。

靡聖管管,不实于亶。猶之未遠,是用大諫。

天之方难,无然憲憲;天之方蹶,无然泄泄。

辭之輯矣,民之洽矣;辭之懌矣,民之莫矣。

我虽異事,及尔同寮。我即尔謀,聽我囂囂。

我言维服,勿以为笑。先民有言:詢于芻蕘。

天之方虐,无然謔謔。老夫灌灌,小子蹻蹻。

匪我言耄,尔用憂謔。多將熇熇,不可救藥。

天之方懠,无为夸毗。威儀卒迷,善人载尸。

民之方殿屎,則莫我敢葵。喪亂蔑資,曾莫惠我師。

天之牖民,如塤如篪,如璋如圭,如取如攜。

攜无曰益,牖民孔易。民之多辟,无自立辟。

價人维藩,大師维垣,大邦维屏,大宗维翰。

懷德维寧,宗子维城。无俾城壞,无獨斯畏。

敬天之怒,无敢戲豫;敬天之渝,无敢馳驅。

昊天曰明,及尔出王;昊天曰旦,及尔游衍。




大雅·荡之什


荡

荡荡上帝,下民之辟。疾威上帝,其命多辟。

天生烝民,其命匪諶。靡不有初,鮮克有終。

文王曰:咨!咨女殷商。曾是強御,曾是掊克,

曾是在位,曾是在服。天降滔德,女興是力。

文王曰:咨!咨女殷商。而秉義類,強御多懟。

流言以對,寇攘式內。侯作侯祝,靡屆靡究。

文王曰:咨!咨女殷商。女炰烋于中国,斂怨以为德。

不明尔德,时无背无側;尔德不明,以无陪无卿。

文王曰:咨!咨女殷商。天不湎尔以酒,不義從式。

既愆尔止,靡明靡晦。式號式呼,俾晝作夜。

文王曰:咨!咨女殷商。如蜩如螗,如沸如羹。

小大近喪,人尚乎由行。內奰于中国,覃及鬼方。

文王曰:咨!咨女殷商。匪上帝不时,殷不用旧。

虽无老成人,尚有典刑。曾是莫聽,大命以傾。

文王曰:咨!咨女殷商。人亦有言:顛沛之揭,

枝葉未有害,本实先撥。殷鑒不遠,在夏後之世!

抑

抑抑威儀,维德之隅。人亦有言:靡哲不愚。庶人之愚,亦職维疾;哲人之愚,亦维斯戾。

无競维人,四方其訓之;有覺德行,四国順之。訏謨定命,遠猶辰告。敬慎威儀,维民之則。

其在于今,興迷亂于政;顛覆厥德,荒湛于酒。女虽湛乐從。弗念厥紹,罔敷求先王,克共明刑。

肆皇天弗尚,如彼泉流,无淪胥以亡。夙興夜寐,灑掃庭內,维民之章。

修尔車马,弓矢戎兵,用戒戎作,用逖蠻方。質尔人民,謹尔侯度,用戒不虞。

慎尔出話,敬尔威儀,无不柔嘉。白圭之玷,尚可磨也;斯言之玷,不可为也。

无易由言,无曰茍矣;莫捫朕舌,言不可逝矣。无言不讎,无德不報。

惠于朋友,庶民小子。子孙繩繩,万民靡不承。視尔友君子,輯柔尔顏,不遐有愆。

相在尔室,尚不愧于屋漏。无曰不显,莫予云覯。神之格思,不可度思,矧可射思?

辟尔为德,俾臧俾嘉。淑慎尔止,不愆于儀。不僭不賊,鮮不为則。

投我以桃,報之以李。彼童而角,实虹小子。荏染柔木,言緡之丝。

溫溫恭人,维德之基。其维哲人,告之話言,順德之行;

其维愚人,覆謂我僭:民各有心。於乎小子!未知臧否。匪手攜之,言示之事;

匪面命之,言提其耳。借曰未知,亦既抱子。民之靡盈,誰夙知而莫成?

昊天孔昭,我生靡乐。視尔夢夢,我心慘慘。誨尔諄諄,聽我藐藐。

匪用为教,覆用为虐。借曰未知,亦聿既耄。

於乎小子!告尔旧止。聽用我謀,庶无大悔。天方艱难,曰喪厥国。

取譬不遠,昊天不忒。回遹其德,俾民大棘。

桑柔

菀彼桑柔,其下侯旬。捋采其刘,瘼此下民。不殄心憂,倉兄填兮;

倬彼昊天,寧不我矜。四牡騤騤,旟旐有翩。亂生不夷,靡国不泯。

民靡有黎,具禍以燼。於乎有哀!国步斯頻。国步蔑資,天不我將;

靡所止疑,云徂何往?君子实维,秉心无競。誰生厲階?至今为梗。

憂心殷殷,念我土宇。我生不辰,逢天僤怒。自西徂東,靡所定處;

多我覯痻,孔棘我圉。为謀为毖,亂況斯削。告尔憂恤,誨尔序爵。

誰能執熱,逝不以濯?其何能淑?载胥及溺。如彼溯風,亦孔之僾;

民有肅心,荓云不逮。好是稼穡,力民代食;稼穡维寶,代食维好。

天降喪亂,滅我立王。降此蟊賊,稼穡卒癢。哀恫中国,具贅卒荒;

靡有旅力,以念穹蒼。维此惠君,民人所瞻。秉心宣猶,考慎其相。

维彼不順,自獨俾臧。自有肺腸,俾民卒狂。

瞻彼中林,甡甡其鹿。朋友已譖,不胥以穀。人亦有言:進退维穀。

维此聖人,瞻言百里;维彼愚人,覆狂以喜。匪言不能,胡斯畏忌。

维此良人,弗求弗迪;维彼忍心,是顧是复。民之貪亂,寧为荼毒!

大風有隧,有空大穀。维此良人,作为式穀;维彼不順,征以中垢。

大風有隧,貪人敗類。聽言則對,誦言如醉。匪用其良,覆俾我悖。

嗟尔朋友!予豈不知而作?如彼飞蟲,时亦弋獲。既之陰女,反予來赫。

民之罔極,職涼善背;为民不利,如云不克。民之回遹,職競用力。

民之未戾,職盜为寇。涼曰不可,覆背善詈。虽曰匪予,既作尔歌。

云汉

倬彼云汉,昭回于天。王曰:於乎!何辜今之人?天降喪亂,饑饉薦臻。靡神不舉,靡愛斯牲。圭璧既卒,寧莫我聽!

旱既太甚,蘊隆蟲蟲。不殄禋祀,自郊徂宮。上下奠瘞,靡神不宗。后稷不克,上帝不臨;耗斁下土,寧丁我躬!

旱既太甚,則不可推。兢兢業業,如霆如雷。周餘黎民,靡有孑遺。昊天上帝,則不我遺。胡不相畏?先祖于摧。

旱既太甚,則不可沮。赫赫炎炎,云我无所。大命近止,靡瞻靡顧。群公先正,則不我助。父母先祖,胡寧忍予?

旱既太甚,滌滌山川。旱魃为虐,如惔如焚。我心憚暑,憂心如薰。群公先正,則不我聞。昊天上帝,寧俾我遁!

旱既太甚,黽勉畏去。胡寧瘨我以旱?憯不知其故。祈年孔夙,方社不莫。昊天上帝,則不我虞。敬恭明神,宜无悔怒。

旱既太甚,散无友紀。鞫哉庶正,疚哉冢宰。趣马師氏,膳夫左右;靡人不周,无不能止。瞻卬昊天,云如何里?

瞻卬昊天,有嘒其星。大夫君子,昭假无贏。大命近止,无棄尔成。何求为我?以戾庶正。瞻卬昊天,曷惠其寧?

崧高

崧高维岳,駿極于天。维岳降神,生甫及申。

维申及甫,维周之翰。四国于蕃,四方于宣。

亹亹申伯,王纘之事。于邑于謝,南国是式。

王命召伯,定申伯之宅。登是南邦,世執其功。

王命申伯,式是南邦,因是謝人,以作尔庸。

王命召伯,徹申伯土田;王命傅御,遷其私人。

申伯之功,召伯是營。有俶其城,寢庙既成,既成藐藐;

王錫申伯,四牡蹻蹻,鉤膺濯濯。

王遣申伯,路車乘马。我圖尔居,莫如南土。

錫尔介圭,以作尔寶。往近王舅,南土是保。

申伯信邁,王餞于郿。申伯還南,謝于誠归。

王命召伯,徹申伯土疆,以峙其粻,式遄其行。

申伯番番,既入于謝,徒御嘽嘽。周邦咸喜,戎有良翰。不显申伯,王之元舅,文武是憲。

申伯之德,柔惠且直。揉此万邦,聞于四国。吉甫作誦,其詩孔碩;其風肆好,以贈申伯。

烝民

天生烝民,有物有則。民之秉彝,好是懿德。天監有周,昭假于下。保茲天子,生仲山甫。

仲山甫之德,柔嘉维則。令儀令色,小心翼翼;古訓是式,威儀是力。天子是若,明命使賦。

王命仲山甫:式是百辟,纘戎祖考,王躬是保,出納王命。王之喉舌,賦政于外,四方爰发。

肅肅王命,仲山甫將之;邦国若否,仲山甫明之。既明且哲,以保其身。夙夜匪解,以事一人。

人亦有言:柔則茹之,刚則吐之。维仲山甫,柔亦不茹,刚亦不吐;不侮矜寡,不畏強御。

人亦有言:德輶如毛,民鮮克舉之,我儀圖之。维仲山甫舉之,愛莫助之。袞職有闕,维仲山甫補之。

仲山甫出祖,四牡業業,征夫捷捷,每懷靡及。四牡彭彭,八鸞鏘鏘,王命仲山甫,城彼東方。

四牡騤騤,八鸞喈喈,仲山甫徂齐,式遄其归。吉甫作誦,穆如清風。仲山甫永懷,以慰其心。

韓奕

奕奕梁山,维禹甸之,有倬其道。韓侯受命,王親命之:纘戎祖考。无廢朕命,夙夜匪解,虔共尔位。朕命不易,干不庭方,以佐戎辟。

四牡奕奕,孔修且张,韓侯入覲,以其介圭,入覲于王。王錫韓侯:淑旗綏章,簟茀錯衡,玄袞赤舄,鉤膺鏤钖,鞹鞃淺幭,鞗革金厄。

韓侯出祖,出宿于屠。显父餞之,清酒百壺。其殽维何?炰鱉鮮鱼。其蔌维何?维筍及蒲。其贈维何?乘马路車。籩豆有且,侯氏燕胥。

韓侯取妻,汾王之甥,蹶父之子。韓侯迎止,于蹶之里。百兩彭彭,八鸞鏘鏘,不显其光。諸娣從之,祁祁如云。韓侯顧之,爛其盈門。

蹶父孔武,靡国不到。为韓姞相攸,莫如韓乐。孔乐韓土,川澤訏訏,魴鱮甫甫,麀鹿噳噳,有熊有羆,有貓有虎。慶既令居,韓姞燕譽。

溥彼韓城,燕師所完。以先祖受命,因时百蠻。王錫韓侯,其追其貊,奄受北国,因以其伯。实墉实壑,实畝实籍。獻其貔皮,赤豹黃羆。

江汉

江汉浮浮,武夫滔滔。匪安匪游,淮夷來求。

既出我車,既設我旟,匪安匪舒,淮夷來鋪。

江汉汤汤,武夫洸洸。經營四方,告成于王。

四方既平,王国庶定。时靡有爭,王心载寧。

江汉之滸,王命召虎,式辟四方,徹我疆土。

匪疚匪棘,王国來極。于疆于理,至于南海。

王命召虎,來旬來宣;文武受命,召公维翰。

无曰:予小子,召公是似。肇敏戎公,用錫尔祉。

厘尔圭瓚,秬鬯一卣,告于文人。錫山土田,

于周受命,自召祖命。虎拜稽首,天子万年。

虎拜稽首,對揚王休。作召公考,天子万寿。

明明天子,令聞不已;矢其文德,洽此四国。

常武

赫赫明明,王命卿士,南仲大祖,大師皇父。

整我六師,以修我戎。既敬既戒,惠此南国。

王謂尹氏,命程伯休父,左右陳行,戒我師旅:

率彼淮浦,省此徐土,不留不處,三事就緒。

赫赫業業,有嚴天子,王舒保作。匪紹匪游,

徐方繹騷。震驚徐方,如雷如霆,徐方震驚。

王奮厥武,如震如怒。進厥虎臣,闞如虓虎。

鋪敦淮濆,仍執醜虜。截彼淮浦,王師之所。

王旅嘽嘽,如飞如翰,如江如汉。如山之苞,

如川之流。绵绵翼翼,不測不克,濯征徐国。

王猶允塞,徐方既來。徐方既同,天子之功。

四方既平,徐方來庭。徐方不回,王曰:還归。

瞻卬

瞻卬昊天,則不我惠。孔填不寧,降此大厲。邦靡有定,

士民其瘵。蟊賊蟊疾,靡有夷屆。罪罟不收,靡有夷瘳。

人有土田,女反有之;人有民人,女覆奪之。此宜无罪,

女反收之;彼宜有罪,女覆說之。哲夫成城,哲婦傾城。

懿厥哲婦,为梟为鴟。婦有长舌,维厲之階。

亂匪降自天,生自婦人。匪教匪誨,时维婦寺。

鞫人忮忒,譖始竟背。豈曰不極?伊胡为慝!

如賈三倍,君子是識。婦无公事,休其蠶織。

天何以刺?何神不富?舍尔介狄,维予胥忌。

不吊不祥,威儀不類。人之云亡,邦国殄瘁。

天之降罔,维其优矣。人之云亡,心之憂矣。

天之降罔,维其幾矣。人之云亡,心之悲矣。

觱沸檻泉,维其深矣。心之憂矣,寧自今矣。

不自我先,不自我後。藐藐昊天,无不克鞏。

无忝皇祖,式救尔後。

召旻

旻天疾威,天篤降喪,瘨我饑饉,民卒流亡。我居圉卒荒。

天降罪罟,蟊賊內訌。昏椓靡共,潰潰回遹,实靖夷我邦。

皋皋訾訾,曾不知其玷。兢兢業業,孔填不寧,我位孔貶。

如彼歲旱,草不潰茂,如彼棲苴。我相此邦,无不潰止。

维昔之富,不如时;维今之疚,不如茲。彼疏斯粺,胡不自替,職兄斯引?

池之竭矣,不云自頻?泉之竭矣,不云自中?溥斯害矣,職兄斯弘,不災我躬?

昔先王受命,有如召公,日辟国百里;今也日蹙国百里。

於乎哀哉!维今之人,不尚有旧。




周頌·清庙之什

清庙

於穆清庙,肅雍显相。济济多士,秉文之德。對越在天,駿奔走在庙。不显不承?无射於人斯。

维天之命

维天之命,於穆不已。於乎不显!文王之德之純。假以溢我,我其收之。駿惠我文王,曾孙篤之。

维清

维清緝熙,文王之典。肇禋。迄用有成,维周之禎。《维清》

烈文

烈文辟公,錫茲祉福,惠我无疆,子孙保之。无封靡于尔邦,维王其崇之。念茲戎功,继序其皇之。无競维人,四方其訓之。不显维德,百辟其刑之。於乎!前王不忘。

天作

天作高山,大王荒之。彼作矣,文王康之。彼徂矣,岐有夷之行。子孙保之。

昊天有成命

昊天有成命,二後受之。成王不敢康,夙夜基命宥密。於緝熙,單厥心,肆其靖之。

我將

我將我享,维羊维牛,维天其右之。儀式刑文王之典,日靖四方。伊嘏文王,既右饗之。我其夙夜,畏天之威,于时保之。

时邁

时邁其邦,昊天其子之,实右序有周。薄言震之,莫不震疊。懷柔百神,及河喬岳。允王维後。

明昭有周,式序在位。载戢干戈,载櫜弓矢。我求懿德,肆于时夏,允王保之。

執競

執競武王,无競维烈。不显成康?上帝是皇。自彼成康,奄有四方,斤斤其明。鐘鼓喤喤,磬筦將將,降福穰穰。降福簡簡,威儀反反。既醉既飽,福祿來反。

思文

思文后稷,克配彼天。立我烝民,莫匪尔極。貽我來牟,帝命率育,无此疆尔界,陳常于时夏。




周颂·臣工之什


臣工

嗟嗟臣工,敬尔在公。王厘尔成,來咨來茹。嗟嗟保介,维莫之春。亦又何求?如何新畬?於皇來牟,將受厥明。明昭上帝,迄用康年。命我眾人,庤乃錢镈,奄觀铚艾。

噫嘻

噫嘻成王,既昭假尔。率时農夫,播厥百穀。駿发尔私,終三十里。亦服尔耕,十千维耦。

振鷺

振鷺于飞,于彼西雍。我客戾止,亦有斯容。在彼无惡,在此无斁。庶幾夙夜,以永終譽。

丰年

丰年多黍多稌,亦有高廩,万億及秭。为酒为醴,烝畀祖妣,以洽百禮。降福孔皆。

有瞽

有瞽有瞽,在周之庭。設業設虡,崇牙樹羽,应田縣鼓,鞉磬柷圉。既備乃奏,簫管備舉。喤喤厥声,肅雍和鳴,先祖是聽。我客戾止,永觀厥成。

潛

猗与漆沮,潛有多鱼。有鳣有鮪,鰷鲿鰋鯉。以享以祀,以介景福。

雍

有來雍雍,至止肅肅。相维辟公,天子穆穆。於薦廣牡,相予肆祀。假哉皇考,綏予孝子。宣哲维人,文武维後。燕及皇天,克昌厥後。綏我眉寿,介以繁祉。既右烈考,亦右文母。

载见

载见辟王,曰求厥章。龙旗陽陽,和鈴央央,鞗革有鸧,休有烈光。率见昭考,以孝以享,以介眉寿。永言保之,思皇多祜。烈文辟公,綏以多福,俾緝熙于純嘏。

有客

有客有客,亦白其马。有萋有且,敦琢其旅。有客宿宿,有客信信。言授之縶,以縶其马。薄言追之,左右綏之。既有淫威,降福孔夷。

武

於皇武王,无競维烈。允文文王,克開厥後。嗣武受之,勝殷遏刘,耆定尔功。




周颂·闵予小子之什


闵予小子

闵予小子,遭家不造,嬛嬛在疚。於乎皇考!永世克孝。

念茲皇祖,陟降庭止。维予小子,夙夜敬止。於乎皇王!继序思不忘。

访落

访予落止,率时昭考。於乎悠哉!朕未有艾,將予就之,继猶判渙。维予小子,未堪家多难。紹庭上下,陟降厥家。休矣皇考,以保明其身。

敬之

敬之敬之,天维显思。命不易哉!无曰:高高在上。陟降厥士,日監在茲。维予小子,不聰敬止。日就月將,學有緝熙于光明。佛时仔肩,示我显德行。

小毖

予其懲,而毖後患。莫予荓蜂,自求辛螫。肇允彼桃蟲,拚飞维鸟。未堪家多难,予又集于蓼。

载芟

载芟载柞,其耕澤澤。千耦其耘,徂隰徂畛。侯主侯伯,侯亞侯旅,侯強侯以。有嗿其馌,思媚其婦,有依其士。

有略其耜,俶载南畝,播厥百穀,实函斯活。驛驛其达,有厭其杰,厭厭其苗,绵绵其麃。载獲济济,有实其積,万億及秭。

为酒为醴,烝畀祖妣,以洽百禮。有飶其香,邦家之光。有椒其馨,胡考之寧?匪且有且,匪今斯今,振古如茲。

良耜

畟畟良耜,俶载南畝,播厥百穀,实函斯活。或來瞻女,载筐及筥。其馕伊黍,其笠伊糾,其镈斯趙,以薅荼蓼。荼蓼朽止,黍稷茂止。獲之挃挃,積之栗栗。其崇如墉,其比如櫛。以開百室。百室盈止,婦子寧止。殺时犉牡,有捄其角。以似以續,續古之人。

丝衣

丝衣其紑,载弁俅俅。自堂徂基,自羊徂牛。鼐鼎及鼒。兕觥其觩,旨酒思柔。不吳不敖,胡考之休?

酌

於鑠王師,遵養时晦。时純熙矣,是用大介。我龙受之,蹻蹻王之造。载用有嗣,实维尔公。允師。

桓

綏万邦,婁丰年,天命匪解。桓桓武王,保有厥士,于以四方,克定厥家。於昭于天,皇以間之。

賚

文王既勤止,我应受之,敷时繹思。我徂维求定,时周之命。於繹思。

般

於皇时周,陟其高山。嶞山喬岳,允猶翕河。敷天之下,裒时之對,时周之命。




鲁颂·駉之什


駉

駉駉牡马,在坰之野。薄言駉者,有驈有皇,有驪有黃,以車彭彭。思无疆,思马斯臧。

駉駉牡马,在坰之野。薄言駉者,有騅有駓,有騂有騏,以車伾伾。思无期,思马斯才。

駉駉牡马,在坰之野。薄言駉者,有驒有駱,有騮有雒。以車繹繹。思无期,思马斯作。

駉駉牡马,在坰之野。薄言駉者,有駰有騢,有駠有鱼,以車繹繹。思无邪,思马斯徂。

有駜

有駜有駜,駜彼乘黃。夙夜在公,在公明明。振振鷺,鷺于下。鼓咽咽,醉言舞。于胥乐兮。

有駜有駜,駜彼乘牡。夙夜在公,在公飲酒。振振鷺,鷺于飞。鼓咽咽,醉言归。于胥乐兮。

有駜有駜,駜彼乘駽。夙夜在公,在公载燕。自今以始,歲其有。君子有穀,詒孙子。于胥乐兮。

泮水

思乐泮水,薄采其芹。鲁侯戾止,言觀其旗。其旗茷茷,鸞声噦噦。无小无大,從公于邁。

思乐泮水,薄采其藻。鲁侯戾止,其马蹻蹻。其马昭昭,其音昭昭。载色载笑,匪怒伊教。

思乐泮水,薄采其茆。鲁侯戾止,在泮飲酒,既飲旨酒,永錫难老。順彼长道,屈此群醜。

穆穆鲁侯,敬明其德。敬慎威儀,维民之則。允文允武,昭假烈祖。靡有不孝,自求伊祜。

明明鲁侯,克明其德,既作泮宮,淮夷攸服。矯矯虎臣,在泮獻馘;淑問如皋陶,在泮獻囚。

济济多士,克廣德心。桓桓于征,狄彼東南。烝烝皇皇,不吳不揚。不告于讻,在泮獻功。

角弓其觩,束矢其搜。戎車孔博,徒御无斁。既克淮夷,孔淑不逆。式固尔猶,淮夷卒獲。

翩彼飞鸮,集于泮林,食我桑黮,懷我好音。憬彼淮夷,來獻其琛:元龜象齿,大賂南金。

閟宮

閟宮有恤,实实枚枚。赫赫姜嫄,其德不回。上帝是依,无災无害;彌月不迟,是生后稷。降之百福,黍稷重穋,稙稚菽麥。奄有下国,俾民稼穡。有稷有黍,有稻有秬。奄有下土,纘禹之緒。后稷之孙,实维大王;居岐之陽,实始翦商。至于文武,纘大王之緒。致天之屆,于牧之野。

无貳无虞,上帝臨女。敦商之旅,克咸厥功。王曰:叔父!建尔元子,俾侯于鲁;大啟尔宇,为周室輔。乃命鲁公,俾侯于東;錫之山川,土田附庸。

周公之孙,莊公之子,龙旗承祀,六轡耳耳。春秋匪解,享祀不忒;皇皇后帝,皇祖后稷,享以騂犧。是饗是宜,降福既多。周公皇祖,亦其福女。

秋而载嘗,夏而楅衡。白牡騂刚,犧尊將將。毛炰胾羹,籩豆大房;万舞洋洋,孝孙有慶。俾尔熾而昌,俾尔寿而臧。保彼東方,鲁邦是常。

不虧不崩,不震不騰。三寿作朋,如岡如陵。公車千乘,朱英綠縢,二矛重弓。公徒三万,貝胄朱綅,烝徒增增。戎狄是膺,荊舒是懲,則莫我敢承。

俾尔昌而熾,俾尔寿而富。黃发台背,寿胥与試。俾尔昌而大,俾尔耆而艾。万有千歲,眉寿无有害。泰山巖巖,鲁邦所詹。奄有龜蒙,遂荒大東,至于海邦。

淮夷來同,莫不率從,鲁侯之功。保有鳧繹,遂荒徐宅,至于海邦。淮夷蠻貊,及彼南夷,莫不率從。莫敢不諾,鲁侯是若。天錫公純嘏,眉寿保鲁;居常与許,复周公之宇。鲁侯燕喜,令妻寿母,宜大夫庶士,邦国是有。

既多受祉,黃发儿齿。徂來之松,新甫之柏,是断是度,是尋是尺。松桷有舄,路寢孔碩。新庙奕奕,奚斯所作。孔曼且碩,万民是若。




商頌


那

猗与那与!置我鞉鼓。奏鼓簡簡,衎我烈祖。汤孙奏假,綏我思成。鞉鼓淵淵,嘒嘒管声。既和且平,依我磬声。於赫汤孙,穆穆厥声。庸鼓有斁,万舞有奕。我有嘉客,亦不夷懌。自古在昔,先民有作。溫恭朝夕,執事有恪。顧予烝嘗,汤孙之將。

烈祖

嗟嗟烈祖!有秩斯祜。申錫无疆,及尔斯所。既载清酤,賚我思成。亦有和羹,既戒既平。鬷假无言,时靡有爭。綏我眉寿,黃耇无疆。約軧錯衡,八鸞鸧鸧,以假以享。我受命溥將。自天降康,丰年穰穰。來假來饗,降福无疆。顧予烝嘗,汤孙之將。

玄鸟

天命玄鸟,降而生商。宅殷土芒芒。古帝命武汤,正域彼四方。方命厥後,奄有九有。商之先後,受命不殆,在武丁孙子。武丁孙子,武王靡不勝。龙旗十乘,大糦是承。邦畿千里,维民所止,肇域彼四海。四海來假,來假祁祁。景員维河,殷受命咸宜,百祿是何。

长发

浚哲维商,长发其祥。洪水芒芒,禹敷下土方。外大国是疆,幅隕既长。有娀方將,帝立子生商。

玄王桓撥,受小国是达,受大国是达。率履不越,遂視既发。相土烈烈,海外有截。

帝命不違,至於汤齐。汤降不迟,聖敬日躋。昭假迟迟,上帝是祗。帝命式于九圍。

受小球大球,为下国綴旒,何天之休?不競不絿,不刚不柔,敷政优优,百祿是遒。

受小共大共,为下国駿厖,何天之龙?敷奏其勇。不震不動,不戁不竦,百祿是總。

武王载旆,有虔秉鉞。如火烈烈,則莫我敢曷。苞有三蘗,莫遂莫达,九有有截。韋顧既伐,昆吾夏桀。

昔在中葉,有震且業。允也天子,降予卿士:实维阿衡,实左右商王。

殷武

撻彼殷武,奮伐荊楚,穼入其阻,裒荊之旅。有截其所,汤孙之緒。

维女荊楚,居国南乡。昔有成汤,自彼氐羌,莫敢不來享,莫敢不來王。曰商是常。

天命多辟,設都于禹之績。歲事來辟,勿予禍適。稼穡匪解。

天命降監,下民有嚴。不僭不濫,不敢怠遑。命于下国,封建厥福。

商邑翼翼,四方之極。赫赫厥声,濯濯厥灵。寿考且寧,以保我後生。

陟彼景山,松柏丸丸。是断是遷,方斫是虔。松桷有梴,旅楹有閑,寢成孔安!

幡幡瓠叶,采之亨之。君子有酒,酌言尝之。

有兔斯首,炮之燔之。君子有酒,酌言献之。

有兔斯首,燔之炙之。君子有酒,酌言酢之。

有兔斯首,燔之炮之。君子有酒,酌言酬之。

渐渐之石

渐渐之石,维其高矣。山川悠远,维其劳矣。武人东征,不遑朝矣。

渐渐之石,维其卒矣。山川悠远,曷其没矣?武人东征,不遑出矣。

有豕白蹢,烝涉波矣。月离于毕,俾滂沱矣。武人东征,不皇他矣。

苕之华

苕之华,芸其黄矣。心之忧矣,维其伤矣!

苕之华,其叶青青。知我如此,不如无生!

牂羊坟首,三星在罶。人可以食,鲜可以饱!

何草不黄

何草不黄?何日不行?何人不将?经营四方。

何草不玄?何人不矜?哀我征夫,独为匪民。

匪兕匪虎,率彼旷野。哀我征夫,朝夕不暇。

有芃者狐,率彼幽草。有栈之车,行彼周道。



 
大雅·文王之什


文王

文王在上,於昭于天,周虽旧邦,其命维新。

有周不显,帝命不时。文王陟降,在帝左右。

亹亹文王,令聞不已。陳錫哉周,侯文王孙子。

文王孙子,本支百世。凡周之士,不显亦世。

世之不显,厥猶翼翼。思皇多士,生此王国。

王国克生,维周之楨。济济多士,文王以寧。

穆穆文王,於緝熙敬止。假哉天命,有商孙子。

商之孙子,其麗不億。上帝既命,侯于周服。

侯服于周,天命靡常。殷士膚敏,祼將于京。

厥作祼將,常服黼冔。王之藎臣,无念尔祖。

无念尔祖,聿修厥德。永言配命,自求多福。

殷之未喪師,克配上帝。宜鑒于殷,駿命不易。

命之不易,无遏尔躬。宣昭義問,有虞殷自天。

上天之载,无声无臭。儀刑文王,万邦作孚。

大明

明明在下,赫赫在上。天难忱斯,不易维王。天位殷適,使不挾四方。

摯仲氏任,自彼殷商;來嫁于周,曰嬪于京。乃及王季,维德之行。大任有身,生此文王。

维此文王,小心翼翼。昭事上帝,聿懷多福。厥德不回,以受方国。

天監在下,有命既集。文王初载,天作之合。在洽之陽,在渭之涘。文王嘉止,大邦有子。

大邦有子,伣天之妹。文定厥祥,親迎于渭。造舟为梁,不显其光。

有命自天,命此文王。于周于京。纘女维莘,长子维行。篤生武王,保右命尔,燮伐大商。

殷商之旅,其會如林。矢于牧野,维予侯興。上帝臨女,无貳尔心!

牧野洋洋,檀車煌煌,駟騵彭彭。维師尚父,时维鷹揚;涼彼武王,肆伐大商,會朝清明。

绵

绵绵瓜瓞。民之初生,自土沮漆。古公亶父,陶复陶穴,未有家室。

古公亶父,來朝走马,率西水滸,至于岐下。爰及姜女,聿來胥宇。

周原膴膴,堇荼如飴。爰始爰謀,爰契我龜。曰止曰时,筑室于茲。

乃慰乃止,乃左乃右,乃疆乃理,乃宣乃畝。自西徂東,周爰執事。

乃召司空,乃召司徒,俾立室家。其繩則直,縮版以载,作庙翼翼。

捄之陾陾,度之薨薨,筑之登登,削屢馮馮。百堵皆興,鼛鼓弗勝。

乃立皋門,皋門有伉;乃立应門,应門將將。乃立冢土,戎醜攸行。

肆不殄厥慍,亦不隕厥問,柞棫拔矣,行道兌矣。混夷駾矣,维其喙矣。

虞芮質厥成,文王厥厥生。予曰有疏附,予曰有先後,予曰有奔奏,予曰有御侮。

棫樸

芃芃棫樸,薪之槱之。济济辟王,左右趣之。

济济辟王,左右奉璋。奉璋峨峨,髦士攸宜。

淠彼涇舟,烝徒楫之。周王于邁,六師及之。

倬彼云汉,为章于天。周王寿考,遐不作人?

追琢其章,金玉其相。勉勉我王,綱紀四方。

旱麓

瞻彼旱麓,榛楛济济。豈弟君子,干祿豈弟。

瑟彼玉瓚,黃流在中。豈弟君子,福祿攸降。

鳶飞戾天,鱼躍于淵。豈弟君子,遐不作人?

清酒既载,騂牡既備。以享以祀,以介景福。

瑟彼柞棫,民所燎矣。豈弟君子,神所勞矣。

莫莫葛櫑,施于條枚。豈弟君子,求福不回。

思齐

思齐大任,文王之母。思媚周姜,京室之婦。大姒嗣徽音,則百斯男。

惠于宗公,神罔时怨,神罔时恫。刑于寡妻,至于兄弟,以御于家邦。

雍雍在宮,肅肅在庙。不显亦臨,无射亦保。

肆戎疾不殄,烈假不遐。不聞亦式,不諫亦入。

肆成人有德,小子有造。古人之无斁,譽髦斯士。

皇矣

皇矣上帝,臨下有赫;監觀四方,求民之莫。

维此二国,其政不獲;维彼四国,爰究爰度。

上帝耆之,憎其式廓。乃眷西顧,此维与宅。

作之屏之,其菑其翳;修之平之,其灌其栵;

啟之辟之,其檉其椐;攘之剔之,其檿其柘。

帝遷明德,串夷载路。天立厥配,受命既固。

帝省其山,柞棫斯拔,松柏斯兌。帝作邦作對,自大伯王季。

维此王季,因心則友。則友其兄,則篤其慶,载錫之光。受祿无喪,奄有四方。

维此王季,帝度其心,貊其德音。其德克明,克明克類,克长克君。

王此大邦,克順克比。比于文王,其德靡悔。既受帝祉,施于孙子。

帝謂文王:无然畔援,无然歆羨,誕先登于岸。密人不恭,敢距大邦,侵阮徂共。

王赫斯怒,爰整其旅,以按徂旅,以篤周祜,以對于天下。

依其在京,侵自阮疆,陟我高岡。无矢我陵,我陵我阿;无飲我泉,我泉我池!

度其鮮原,居岐之陽,在渭之將。万邦之方,下民之王。

帝謂文王:予懷明德,不大声以色,不长夏以革,不識不知,順帝之則。

帝謂文王:詢尔仇方,同尔兄弟。以尔鉤援,与尔臨沖,以伐崇墉。

臨沖閑閑,崇墉言言,執訊連連,攸馘安安。是類是祃,是致是附,四方以无侮。

臨沖茀茀,崇墉仡仡,是伐是肆,是絕是忽,四方以无拂。

灵台

經始灵台,經之營之。庶民攻之,不日成之。經始勿亟,庶民子來。

王在灵囿,麀鹿攸伏;麀鹿濯濯,白鸟翯翯。王在灵沼,於牣鱼躍。

虡業维樅,賁鼓维鏞。於論鼓鐘,於乐辟癰。

於論鼓鐘,於乐辟癰。鼉鼓逢逢,矇瞍奏公。

下武

下武维周,世有哲王。三後在天,王配于京。

王配于京,世德作求。永言配命,成王之孚。

成王之孚,下土之式。永言孝思,孝思维則。

媚茲一人,应侯順德。永言孝思,昭哉嗣服。

昭茲來許,繩其祖武。於万斯年,受天之祜。

受天之祜,四方來賀。於万斯年,不遐有佐。

文王有声

文王有声,遹駿有声,遹求厥寧,遹觀厥成。文王烝哉!

文王受命,有此武功;既伐于崇,作邑于丰。文王烝哉!

筑城伊淢,作丰伊匹,匪棘其欲,遹追來孝。王后烝哉!

王公伊濯,维丰之垣。四方攸同,王后维翰。王后烝哉!

丰水東注,维禹之績。四方攸同,皇王维辟。皇王烝哉!

鎬京辟癰,自西自東,自南自北,无思不服。皇王烝哉!

考卜维王,宅是鎬京。维龜正之,武王成之。武王烝哉!

丰水有芑,武王豈不仕?詒厥孙謀,以燕翼子。武王烝哉!




大雅·生民之什


生民

厥初生民,时维姜嫄。生民如何?克禋克祀,以弗无子。履帝武敏歆,攸介攸止;载震载夙,载生载育,时维后稷。

誕彌厥月,先生如达。不坼不副,无菑无害。以赫厥灵,上帝不寧。不康禋祀,居然生子。

誕寘之隘巷,牛羊腓字之。誕寘之平林,會伐平林;誕寘之寒冰,鸟覆翼之。鸟乃去矣,后稷呱矣。实覃实訏,厥声载路。

誕实匍匐,克岐克嶷,以就口食。蓺之荏菽,荏菽旆旆,禾役穟穟,麻麥幪幪,瓜瓞唪唪。

誕后稷之穡,有相之道。茀厥丰草,種之黃茂。实方实苞,实種实褎,实发实秀,实堅实好,实穎实栗,即有邰家室。

誕降嘉種,维秬维秠,维穈维芑。恒之秬秠,是獲是畝;恒之穈芑,是任是負,以归肇祀。

誕我祀如何?或舂或揄,或簸或蹂;釋之叟叟,烝之浮浮。载謀载惟,取蕭祭脂,取羝以軷,载燔载烈。以興嗣歲。

卬盛于豆,于豆于登。其香始升,上帝居歆。胡臭亶时。后稷肇祀,庶无罪悔,以迄于今。

行葦

敦彼行葦,牛羊勿踐履。方苞方體,维葉泥泥。戚戚兄弟,莫遠具尔。或肆之筵,或授之几。

肆筵設席,授几有緝御。或獻或酢,洗爵奠斝。醓醢以薦,或幡或炙。嘉肴脾臄,或歌或咢。

敦弓既堅,四鍭既鈞;舍矢既均,序賓以賢。敦弓既句,既挾四鍭;四鍭如樹,序賓以不侮。

曾孙维主,酒醴维醹,酌以大斗,以祈黃耇。黃耇台背,以引以翼。寿考维祺,以介景福。

既醉

既醉以酒,既飽以德。君子万年,介尔景福。

既醉以酒,尔殽既將。君子万年,介尔昭明。

昭明有融,高朗令終。令終有俶,公尸嘉告。

其告维何?籩豆靜嘉。朋友攸攝,攝以威儀。

威儀孔时,君子有孝子。孝子不匱,永錫尔類。

其類维何?室家之壸。君子万年,永錫祚胤。

其胤维何?天被尔祿。君子万年,景命有仆。

其仆维何?厘尔女士。厘尔女士,從以孙子。

鳧鹥

鳧鹥在涇,公尸來燕來寧。尔酒既清,尔殽既馨。公尸燕飲,福祿來成。

鳧鹥在沙,公尸來燕來宜。尔酒既多,尔殽既嘉。公尸燕飲,福祿來为。

鳧鹥在渚,公尸來燕來處。尔酒既湑,尔殽伊脯。公尸燕飲,福祿來下。

鳧鹥在潀,公尸來燕來宗。既燕于宗,福祿攸降。公尸燕飲,福祿來崇。

鳧鹥在亹,公尸來止熏熏。旨酒欣欣,燔炙芬芬。公尸燕飲,无有後艱。

假乐

假乐君子,显显令德。宜民宜人,受祿于天。保右命之,自天申之。

干祿百福,子孙千億。穆穆皇皇,宜君宜王。不愆不忘,率由旧章。

威儀抑抑,德音秩秩。无怨无惡,率由群匹。受福无疆,四方之綱。

之綱之紀,燕及朋友。百辟卿士,媚于天子。不解于位,民之攸墍。《假乐》

公刘

篤公刘,匪居匪康,乃场乃疆,乃積乃倉。乃裹糇糧,于橐于囊,思輯用光。弓矢斯张,干戈戚揚,爰方啟行。

篤公刘,于胥斯原。既庶既繁。既順乃宣,而无詠嘆。陟則在巘,复降在原。何以舟之?维玉及瑤,鞞琫容刀。

篤公刘,逝彼百泉,瞻彼溥原。乃陟南岡,乃覯于京。京師之野,于时處處,于时廬旅。于时言言,于时語語。

篤公刘,于京斯依。蹌蹌济济,俾筵俾几。既登乃依,乃造其曹;執豕于牢,酌之用匏。食之飲之,君之宗之。

篤公刘,既溥既长。既景乃岡,相其陰陽,觀其流泉。其軍三單,度其隰原,徹田为糧。度其夕陽,豳居允荒。

篤公刘,于豳斯館。涉渭为亂,取厲取鍛。止基乃理,爰眾爰有。夾其皇澗,溯其過澗。止旅乃密,芮鞫之即。

泂酌

泂酌彼行潦,挹彼注茲,可以餴饎。豈弟君子,民之父母。

泂酌彼行潦,挹彼注茲,可以濯罍。豈弟君子,民之攸归。

泂酌彼行潦,挹彼注茲,可以濯溉。豈弟君子,民之攸墍。

卷阿

有卷者阿,飄風自南。豈弟君子,來游來歌,以矢其音。

伴奐尔游矣,优游尔休矣。豈弟君子,俾尔彌尔性,似先公酋矣。

尔土宇皈章,亦孔之厚矣。豈弟君子,俾尔彌尔性,百神尔主矣。

尔受命长矣,茀祿尔康矣。豈弟君子,俾尔彌尔性,純嘏尔常矣。

有馮有翼,有孝有德,以引以翼。豈弟君子,四方为則。

颙颙卬卬,如圭如璋,令聞令望。豈弟君子,四方为綱。

鳳皇于飞,翙翙其羽,亦集爰止。藹藹王多吉士,维君子使,媚于天子。

鳳皇于飞,翙翙其羽,亦傅于天。藹藹王多吉人,维君子命,媚于庶人。

鳳皇鳴矣,于彼高岡。梧桐生矣,于彼朝陽。菶菶萋萋,雍雍喈喈。

君子之車,既庶且多;君子之马,既閑且馳。矢詩不多,维以遂歌。

民勞

民亦勞止,汔可械。惠此中国,以綏四方。无縱詭隨,以謹无良。式遏寇虐,憯不畏明。柔遠能邇,以定我王。

民亦勞止,汔可小休。惠此中国,以为民逑。无縱詭隨,以謹惛怓。式遏寇虐,无俾民憂。无棄尔勞,以为王休。

民亦勞止,汔可小息。惠此京師,以綏四国。无縱詭隨,以謹罔極。式遏寇虐,无俾作慝。敬慎威儀,以近有德。

民亦勞止,汔可小愒。惠此中国,俾民憂泄。无縱詭隨,以謹醜厲。式遏寇虐,无俾正敗。戎虽小子,而式弘大。

民亦勞止,汔可小安。惠此中国,国无有殘。无縱詭隨,以謹繾綣。式遏寇虐,无俾正反。王欲玉女,是用大諫。

板

上帝板板,下民卒癉。出話不然,为猶不遠。

靡聖管管,不实于亶。猶之未遠,是用大諫。

天之方难,无然憲憲;天之方蹶,无然泄泄。

辭之輯矣,民之洽矣;辭之懌矣,民之莫矣。

我虽異事,及尔同寮。我即尔謀,聽我囂囂。

我言维服,勿以为笑。先民有言:詢于芻蕘。

天之方虐,无然謔謔。老夫灌灌,小子蹻蹻。

匪我言耄,尔用憂謔。多將熇熇,不可救藥。

天之方懠,无为夸毗。威儀卒迷,善人载尸。

民之方殿屎,則莫我敢葵。喪亂蔑資,曾莫惠我師。

天之牖民,如塤如篪,如璋如圭,如取如攜。

攜无曰益,牖民孔易。民之多辟,无自立辟。

價人维藩,大師维垣,大邦维屏,大宗维翰。

懷德维寧,宗子维城。无俾城壞,无獨斯畏。

敬天之怒,无敢戲豫;敬天之渝,无敢馳驅。

昊天曰明,及尔出王;昊天曰旦,及尔游衍。




大雅·荡之什


荡

荡荡上帝,下民之辟。疾威上帝,其命多辟。

天生烝民,其命匪諶。靡不有初,鮮克有終。

文王曰:咨!咨女殷商。曾是強御,曾是掊克,

曾是在位,曾是在服。天降滔德,女興是力。

文王曰:咨!咨女殷商。而秉義類,強御多懟。

流言以對,寇攘式內。侯作侯祝,靡屆靡究。

文王曰:咨!咨女殷商。女炰烋于中国,斂怨以为德。

不明尔德,时无背无側;尔德不明,以无陪无卿。

文王曰:咨!咨女殷商。天不湎尔以酒,不義從式。

既愆尔止,靡明靡晦。式號式呼,俾晝作夜。

文王曰:咨!咨女殷商。如蜩如螗,如沸如羹。

小大近喪,人尚乎由行。內奰于中国,覃及鬼方。

文王曰:咨!咨女殷商。匪上帝不时,殷不用旧。

虽无老成人,尚有典刑。曾是莫聽,大命以傾。

文王曰:咨!咨女殷商。人亦有言:顛沛之揭,

枝葉未有害,本实先撥。殷鑒不遠,在夏後之世!

抑

抑抑威儀,维德之隅。人亦有言:靡哲不愚。庶人之愚,亦職维疾;哲人之愚,亦维斯戾。

无競维人,四方其訓之;有覺德行,四国順之。訏謨定命,遠猶辰告。敬慎威儀,维民之則。

其在于今,興迷亂于政;顛覆厥德,荒湛于酒。女虽湛乐從。弗念厥紹,罔敷求先王,克共明刑。

肆皇天弗尚,如彼泉流,无淪胥以亡。夙興夜寐,灑掃庭內,维民之章。

修尔車马,弓矢戎兵,用戒戎作,用逖蠻方。質尔人民,謹尔侯度,用戒不虞。

慎尔出話,敬尔威儀,无不柔嘉。白圭之玷,尚可磨也;斯言之玷,不可为也。

无易由言,无曰茍矣;莫捫朕舌,言不可逝矣。无言不讎,无德不報。

惠于朋友,庶民小子。子孙繩繩,万民靡不承。視尔友君子,輯柔尔顏,不遐有愆。

相在尔室,尚不愧于屋漏。无曰不显,莫予云覯。神之格思,不可度思,矧可射思?

辟尔为德,俾臧俾嘉。淑慎尔止,不愆于儀。不僭不賊,鮮不为則。

投我以桃,報之以李。彼童而角,实虹小子。荏染柔木,言緡之丝。

溫溫恭人,维德之基。其维哲人,告之話言,順德之行;

其维愚人,覆謂我僭:民各有心。於乎小子!未知臧否。匪手攜之,言示之事;

匪面命之,言提其耳。借曰未知,亦既抱子。民之靡盈,誰夙知而莫成?

昊天孔昭,我生靡乐。視尔夢夢,我心慘慘。誨尔諄諄,聽我藐藐。

匪用为教,覆用为虐。借曰未知,亦聿既耄。

於乎小子!告尔旧止。聽用我謀,庶无大悔。天方艱难,曰喪厥国。

取譬不遠,昊天不忒。回遹其德,俾民大棘。

桑柔

菀彼桑柔,其下侯旬。捋采其刘,瘼此下民。不殄心憂,倉兄填兮;

倬彼昊天,寧不我矜。四牡騤騤,旟旐有翩。亂生不夷,靡国不泯。

民靡有黎,具禍以燼。於乎有哀!国步斯頻。国步蔑資,天不我將;

靡所止疑,云徂何往?君子实维,秉心无競。誰生厲階?至今为梗。

憂心殷殷,念我土宇。我生不辰,逢天僤怒。自西徂東,靡所定處;

多我覯痻,孔棘我圉。为謀为毖,亂況斯削。告尔憂恤,誨尔序爵。

誰能執熱,逝不以濯?其何能淑?载胥及溺。如彼溯風,亦孔之僾;

民有肅心,荓云不逮。好是稼穡,力民代食;稼穡维寶,代食维好。

天降喪亂,滅我立王。降此蟊賊,稼穡卒癢。哀恫中国,具贅卒荒;

靡有旅力,以念穹蒼。维此惠君,民人所瞻。秉心宣猶,考慎其相。

维彼不順,自獨俾臧。自有肺腸,俾民卒狂。

瞻彼中林,甡甡其鹿。朋友已譖,不胥以穀。人亦有言:進退维穀。

维此聖人,瞻言百里;维彼愚人,覆狂以喜。匪言不能,胡斯畏忌。

维此良人,弗求弗迪;维彼忍心,是顧是复。民之貪亂,寧为荼毒!

大風有隧,有空大穀。维此良人,作为式穀;维彼不順,征以中垢。

大風有隧,貪人敗類。聽言則對,誦言如醉。匪用其良,覆俾我悖。

嗟尔朋友!予豈不知而作?如彼飞蟲,时亦弋獲。既之陰女,反予來赫。

民之罔極,職涼善背;为民不利,如云不克。民之回遹,職競用力。

民之未戾,職盜为寇。涼曰不可,覆背善詈。虽曰匪予,既作尔歌。

云汉

倬彼云汉,昭回于天。王曰:於乎!何辜今之人?天降喪亂,饑饉薦臻。靡神不舉,靡愛斯牲。圭璧既卒,寧莫我聽!

旱既太甚,蘊隆蟲蟲。不殄禋祀,自郊徂宮。上下奠瘞,靡神不宗。后稷不克,上帝不臨;耗斁下土,寧丁我躬!

旱既太甚,則不可推。兢兢業業,如霆如雷。周餘黎民,靡有孑遺。昊天上帝,則不我遺。胡不相畏?先祖于摧。

旱既太甚,則不可沮。赫赫炎炎,云我无所。大命近止,靡瞻靡顧。群公先正,則不我助。父母先祖,胡寧忍予?

旱既太甚,滌滌山川。旱魃为虐,如惔如焚。我心憚暑,憂心如薰。群公先正,則不我聞。昊天上帝,寧俾我遁!

旱既太甚,黽勉畏去。胡寧瘨我以旱?憯不知其故。祈年孔夙,方社不莫。昊天上帝,則不我虞。敬恭明神,宜无悔怒。

旱既太甚,散无友紀。鞫哉庶正,疚哉冢宰。趣马師氏,膳夫左右;靡人不周,无不能止。瞻卬昊天,云如何里?

瞻卬昊天,有嘒其星。大夫君子,昭假无贏。大命近止,无棄尔成。何求为我?以戾庶正。瞻卬昊天,曷惠其寧?

崧高

崧高维岳,駿極于天。维岳降神,生甫及申。

维申及甫,维周之翰。四国于蕃,四方于宣。

亹亹申伯,王纘之事。于邑于謝,南国是式。

王命召伯,定申伯之宅。登是南邦,世執其功。

王命申伯,式是南邦,因是謝人,以作尔庸。

王命召伯,徹申伯土田;王命傅御,遷其私人。

申伯之功,召伯是營。有俶其城,寢庙既成,既成藐藐;

王錫申伯,四牡蹻蹻,鉤膺濯濯。

王遣申伯,路車乘马。我圖尔居,莫如南土。

錫尔介圭,以作尔寶。往近王舅,南土是保。

申伯信邁,王餞于郿。申伯還南,謝于誠归。

王命召伯,徹申伯土疆,以峙其粻,式遄其行。

申伯番番,既入于謝,徒御嘽嘽。周邦咸喜,戎有良翰。不显申伯,王之元舅,文武是憲。

申伯之德,柔惠且直。揉此万邦,聞于四国。吉甫作誦,其詩孔碩;其風肆好,以贈申伯。

烝民

天生烝民,有物有則。民之秉彝,好是懿德。天監有周,昭假于下。保茲天子,生仲山甫。

仲山甫之德,柔嘉维則。令儀令色,小心翼翼;古訓是式,威儀是力。天子是若,明命使賦。

王命仲山甫:式是百辟,纘戎祖考,王躬是保,出納王命。王之喉舌,賦政于外,四方爰发。

肅肅王命,仲山甫將之;邦国若否,仲山甫明之。既明且哲,以保其身。夙夜匪解,以事一人。

人亦有言:柔則茹之,刚則吐之。维仲山甫,柔亦不茹,刚亦不吐;不侮矜寡,不畏強御。

人亦有言:德輶如毛,民鮮克舉之,我儀圖之。维仲山甫舉之,愛莫助之。袞職有闕,维仲山甫補之。

仲山甫出祖,四牡業業,征夫捷捷,每懷靡及。四牡彭彭,八鸞鏘鏘,王命仲山甫,城彼東方。

四牡騤騤,八鸞喈喈,仲山甫徂齐,式遄其归。吉甫作誦,穆如清風。仲山甫永懷,以慰其心。

韓奕

奕奕梁山,维禹甸之,有倬其道。韓侯受命,王親命之:纘戎祖考。无廢朕命,夙夜匪解,虔共尔位。朕命不易,干不庭方,以佐戎辟。

四牡奕奕,孔修且张,韓侯入覲,以其介圭,入覲于王。王錫韓侯:淑旗綏章,簟茀錯衡,玄袞赤舄,鉤膺鏤钖,鞹鞃淺幭,鞗革金厄。

韓侯出祖,出宿于屠。显父餞之,清酒百壺。其殽维何?炰鱉鮮鱼。其蔌维何?维筍及蒲。其贈维何?乘马路車。籩豆有且,侯氏燕胥。

韓侯取妻,汾王之甥,蹶父之子。韓侯迎止,于蹶之里。百兩彭彭,八鸞鏘鏘,不显其光。諸娣從之,祁祁如云。韓侯顧之,爛其盈門。

蹶父孔武,靡国不到。为韓姞相攸,莫如韓乐。孔乐韓土,川澤訏訏,魴鱮甫甫,麀鹿噳噳,有熊有羆,有貓有虎。慶既令居,韓姞燕譽。

溥彼韓城,燕師所完。以先祖受命,因时百蠻。王錫韓侯,其追其貊,奄受北国,因以其伯。实墉实壑,实畝实籍。獻其貔皮,赤豹黃羆。

江汉

江汉浮浮,武夫滔滔。匪安匪游,淮夷來求。

既出我車,既設我旟,匪安匪舒,淮夷來鋪。

江汉汤汤,武夫洸洸。經營四方,告成于王。

四方既平,王国庶定。时靡有爭,王心载寧。

江汉之滸,王命召虎,式辟四方,徹我疆土。

匪疚匪棘,王国來極。于疆于理,至于南海。

王命召虎,來旬來宣;文武受命,召公维翰。

无曰:予小子,召公是似。肇敏戎公,用錫尔祉。

厘尔圭瓚,秬鬯一卣,告于文人。錫山土田,

于周受命,自召祖命。虎拜稽首,天子万年。

虎拜稽首,對揚王休。作召公考,天子万寿。

明明天子,令聞不已;矢其文德,洽此四国。

常武

赫赫明明,王命卿士,南仲大祖,大師皇父。

整我六師,以修我戎。既敬既戒,惠此南国。

王謂尹氏,命程伯休父,左右陳行,戒我師旅:

率彼淮浦,省此徐土,不留不處,三事就緒。

赫赫業業,有嚴天子,王舒保作。匪紹匪游,

徐方繹騷。震驚徐方,如雷如霆,徐方震驚。

王奮厥武,如震如怒。進厥虎臣,闞如虓虎。

鋪敦淮濆,仍執醜虜。截彼淮浦,王師之所。

王旅嘽嘽,如飞如翰,如江如汉。如山之苞,

如川之流。绵绵翼翼,不測不克,濯征徐国。

王猶允塞,徐方既來。徐方既同,天子之功。

四方既平,徐方來庭。徐方不回,王曰:還归。

瞻卬

瞻卬昊天,則不我惠。孔填不寧,降此大厲。邦靡有定,

士民其瘵。蟊賊蟊疾,靡有夷屆。罪罟不收,靡有夷瘳。

人有土田,女反有之;人有民人,女覆奪之。此宜无罪,

女反收之;彼宜有罪,女覆說之。哲夫成城,哲婦傾城。

懿厥哲婦,为梟为鴟。婦有长舌,维厲之階。

亂匪降自天,生自婦人。匪教匪誨,时维婦寺。

鞫人忮忒,譖始竟背。豈曰不極?伊胡为慝!

如賈三倍,君子是識。婦无公事,休其蠶織。

天何以刺?何神不富?舍尔介狄,维予胥忌。

不吊不祥,威儀不類。人之云亡,邦国殄瘁。

天之降罔,维其优矣。人之云亡,心之憂矣。

天之降罔,维其幾矣。人之云亡,心之悲矣。

觱沸檻泉,维其深矣。心之憂矣,寧自今矣。

不自我先,不自我後。藐藐昊天,无不克鞏。

无忝皇祖,式救尔後。

召旻

旻天疾威,天篤降喪,瘨我饑饉,民卒流亡。我居圉卒荒。

天降罪罟,蟊賊內訌。昏椓靡共,潰潰回遹,实靖夷我邦。

皋皋訾訾,曾不知其玷。兢兢業業,孔填不寧,我位孔貶。

如彼歲旱,草不潰茂,如彼棲苴。我相此邦,无不潰止。

维昔之富,不如时;维今之疚,不如茲。彼疏斯粺,胡不自替,職兄斯引?

池之竭矣,不云自頻?泉之竭矣,不云自中?溥斯害矣,職兄斯弘,不災我躬?

昔先王受命,有如召公,日辟国百里;今也日蹙国百里。

於乎哀哉!维今之人,不尚有旧。




周頌·清庙之什

清庙

於穆清庙,肅雍显相。济济多士,秉文之德。對越在天,駿奔走在庙。不显不承?无射於人斯。

维天之命

维天之命,於穆不已。於乎不显!文王之德之純。假以溢我,我其收之。駿惠我文王,曾孙篤之。

维清

维清緝熙,文王之典。肇禋。迄用有成,维周之禎。《维清》

烈文

烈文辟公,錫茲祉福,惠我无疆,子孙保之。无封靡于尔邦,维王其崇之。念茲戎功,继序其皇之。无競维人,四方其訓之。不显维德,百辟其刑之。於乎!前王不忘。

天作

天作高山,大王荒之。彼作矣,文王康之。彼徂矣,岐有夷之行。子孙保之。

昊天有成命

昊天有成命,二後受之。成王不敢康,夙夜基命宥密。於緝熙,單厥心,肆其靖之。

我將

我將我享,维羊维牛,维天其右之。儀式刑文王之典,日靖四方。伊嘏文王,既右饗之。我其夙夜,畏天之威,于时保之。

时邁

时邁其邦,昊天其子之,实右序有周。薄言震之,莫不震疊。懷柔百神,及河喬岳。允王维後。

明昭有周,式序在位。载戢干戈,载櫜弓矢。我求懿德,肆于时夏,允王保之。

執競

執競武王,无競维烈。不显成康?上帝是皇。自彼成康,奄有四方,斤斤其明。鐘鼓喤喤,磬筦將將,降福穰穰。降福簡簡,威儀反反。既醉既飽,福祿來反。

思文

思文后稷,克配彼天。立我烝民,莫匪尔極。貽我來牟,帝命率育,无此疆尔界,陳常于时夏。




周颂·臣工之什


臣工

嗟嗟臣工,敬尔在公。王厘尔成,來咨來茹。嗟嗟保介,维莫之春。亦又何求?如何新畬?於皇來牟,將受厥明。明昭上帝,迄用康年。命我眾人,庤乃錢镈,奄觀铚艾。

噫嘻

噫嘻成王,既昭假尔。率时農夫,播厥百穀。駿发尔私,終三十里。亦服尔耕,十千维耦。

振鷺

振鷺于飞,于彼西雍。我客戾止,亦有斯容。在彼无惡,在此无斁。庶幾夙夜,以永終譽。

丰年

丰年多黍多稌,亦有高廩,万億及秭。为酒为醴,烝畀祖妣,以洽百禮。降福孔皆。

有瞽

有瞽有瞽,在周之庭。設業設虡,崇牙樹羽,应田縣鼓,鞉磬柷圉。既備乃奏,簫管備舉。喤喤厥声,肅雍和鳴,先祖是聽。我客戾止,永觀厥成。

潛

猗与漆沮,潛有多鱼。有鳣有鮪,鰷鲿鰋鯉。以享以祀,以介景福。

雍

有來雍雍,至止肅肅。相维辟公,天子穆穆。於薦廣牡,相予肆祀。假哉皇考,綏予孝子。宣哲维人,文武维後。燕及皇天,克昌厥後。綏我眉寿,介以繁祉。既右烈考,亦右文母。

载见

载见辟王,曰求厥章。龙旗陽陽,和鈴央央,鞗革有鸧,休有烈光。率见昭考,以孝以享,以介眉寿。永言保之,思皇多祜。烈文辟公,綏以多福,俾緝熙于純嘏。

有客

有客有客,亦白其马。有萋有且,敦琢其旅。有客宿宿,有客信信。言授之縶,以縶其马。薄言追之,左右綏之。既有淫威,降福孔夷。

武

於皇武王,无競维烈。允文文王,克開厥後。嗣武受之,勝殷遏刘,耆定尔功。




周颂·闵予小子之什


闵予小子

闵予小子,遭家不造,嬛嬛在疚。於乎皇考!永世克孝。

念茲皇祖,陟降庭止。维予小子,夙夜敬止。於乎皇王!继序思不忘。

访落

访予落止,率时昭考。於乎悠哉!朕未有艾,將予就之,继猶判渙。维予小子,未堪家多难。紹庭上下,陟降厥家。休矣皇考,以保明其身。

敬之

敬之敬之,天维显思。命不易哉!无曰:高高在上。陟降厥士,日監在茲。维予小子,不聰敬止。日就月將,學有緝熙于光明。佛时仔肩,示我显德行。

小毖

予其懲,而毖後患。莫予荓蜂,自求辛螫。肇允彼桃蟲,拚飞维鸟。未堪家多难,予又集于蓼。

载芟

载芟载柞,其耕澤澤。千耦其耘,徂隰徂畛。侯主侯伯,侯亞侯旅,侯強侯以。有嗿其馌,思媚其婦,有依其士。

有略其耜,俶载南畝,播厥百穀,实函斯活。驛驛其达,有厭其杰,厭厭其苗,绵绵其麃。载獲济济,有实其積,万億及秭。

为酒为醴,烝畀祖妣,以洽百禮。有飶其香,邦家之光。有椒其馨,胡考之寧?匪且有且,匪今斯今,振古如茲。

良耜

畟畟良耜,俶载南畝,播厥百穀,实函斯活。或來瞻女,载筐及筥。其馕伊黍,其笠伊糾,其镈斯趙,以薅荼蓼。荼蓼朽止,黍稷茂止。獲之挃挃,積之栗栗。其崇如墉,其比如櫛。以開百室。百室盈止,婦子寧止。殺时犉牡,有捄其角。以似以續,續古之人。

丝衣

丝衣其紑,载弁俅俅。自堂徂基,自羊徂牛。鼐鼎及鼒。兕觥其觩,旨酒思柔。不吳不敖,胡考之休?

酌

於鑠王師,遵養时晦。时純熙矣,是用大介。我龙受之,蹻蹻王之造。载用有嗣,实维尔公。允師。

桓

綏万邦,婁丰年,天命匪解。桓桓武王,保有厥士,于以四方,克定厥家。於昭于天,皇以間之。

賚

文王既勤止,我应受之,敷时繹思。我徂维求定,时周之命。於繹思。

般

於皇时周,陟其高山。嶞山喬岳,允猶翕河。敷天之下,裒时之對,时周之命。




鲁颂·駉之什


駉

駉駉牡马,在坰之野。薄言駉者,有驈有皇,有驪有黃,以車彭彭。思无疆,思马斯臧。

駉駉牡马,在坰之野。薄言駉者,有騅有駓,有騂有騏,以車伾伾。思无期,思马斯才。

駉駉牡马,在坰之野。薄言駉者,有驒有駱,有騮有雒。以車繹繹。思无期,思马斯作。

駉駉牡马,在坰之野。薄言駉者,有駰有騢,有駠有鱼,以車繹繹。思无邪,思马斯徂。

有駜

有駜有駜,駜彼乘黃。夙夜在公,在公明明。振振鷺,鷺于下。鼓咽咽,醉言舞。于胥乐兮。

有駜有駜,駜彼乘牡。夙夜在公,在公飲酒。振振鷺,鷺于飞。鼓咽咽,醉言归。于胥乐兮。

有駜有駜,駜彼乘駽。夙夜在公,在公载燕。自今以始,歲其有。君子有穀,詒孙子。于胥乐兮。

泮水

思乐泮水,薄采其芹。鲁侯戾止,言觀其旗。其旗茷茷,鸞声噦噦。无小无大,從公于邁。

思乐泮水,薄采其藻。鲁侯戾止,其马蹻蹻。其马昭昭,其音昭昭。载色载笑,匪怒伊教。

思乐泮水,薄采其茆。鲁侯戾止,在泮飲酒,既飲旨酒,永錫难老。順彼长道,屈此群醜。

穆穆鲁侯,敬明其德。敬慎威儀,维民之則。允文允武,昭假烈祖。靡有不孝,自求伊祜。

明明鲁侯,克明其德,既作泮宮,淮夷攸服。矯矯虎臣,在泮獻馘;淑問如皋陶,在泮獻囚。

济济多士,克廣德心。桓桓于征,狄彼東南。烝烝皇皇,不吳不揚。不告于讻,在泮獻功。

角弓其觩,束矢其搜。戎車孔博,徒御无斁。既克淮夷,孔淑不逆。式固尔猶,淮夷卒獲。

翩彼飞鸮,集于泮林,食我桑黮,懷我好音。憬彼淮夷,來獻其琛:元龜象齿,大賂南金。

閟宮

閟宮有恤,实实枚枚。赫赫姜嫄,其德不回。上帝是依,无災无害;彌月不迟,是生后稷。降之百福,黍稷重穋,稙稚菽麥。奄有下国,俾民稼穡。有稷有黍,有稻有秬。奄有下土,纘禹之緒。后稷之孙,实维大王;居岐之陽,实始翦商。至于文武,纘大王之緒。致天之屆,于牧之野。

无貳无虞,上帝臨女。敦商之旅,克咸厥功。王曰:叔父!建尔元子,俾侯于鲁;大啟尔宇,为周室輔。乃命鲁公,俾侯于東;錫之山川,土田附庸。

周公之孙,莊公之子,龙旗承祀,六轡耳耳。春秋匪解,享祀不忒;皇皇后帝,皇祖后稷,享以騂犧。是饗是宜,降福既多。周公皇祖,亦其福女。

秋而载嘗,夏而楅衡。白牡騂刚,犧尊將將。毛炰胾羹,籩豆大房;万舞洋洋,孝孙有慶。俾尔熾而昌,俾尔寿而臧。保彼東方,鲁邦是常。

不虧不崩,不震不騰。三寿作朋,如岡如陵。公車千乘,朱英綠縢,二矛重弓。公徒三万,貝胄朱綅,烝徒增增。戎狄是膺,荊舒是懲,則莫我敢承。

俾尔昌而熾,俾尔寿而富。黃发台背,寿胥与試。俾尔昌而大,俾尔耆而艾。万有千歲,眉寿无有害。泰山巖巖,鲁邦所詹。奄有龜蒙,遂荒大東,至于海邦。

淮夷來同,莫不率從,鲁侯之功。保有鳧繹,遂荒徐宅,至于海邦。淮夷蠻貊,及彼南夷,莫不率從。莫敢不諾,鲁侯是若。天錫公純嘏,眉寿保鲁;居常与許,复周公之宇。鲁侯燕喜,令妻寿母,宜大夫庶士,邦国是有。

既多受祉,黃发儿齿。徂來之松,新甫之柏,是断是度,是尋是尺。松桷有舄,路寢孔碩。新庙奕奕,奚斯所作。孔曼且碩,万民是若。




商頌


那

猗与那与!置我鞉鼓。奏鼓簡簡,衎我烈祖。汤孙奏假,綏我思成。鞉鼓淵淵,嘒嘒管声。既和且平,依我磬声。於赫汤孙,穆穆厥声。庸鼓有斁,万舞有奕。我有嘉客,亦不夷懌。自古在昔,先民有作。溫恭朝夕,執事有恪。顧予烝嘗,汤孙之將。

烈祖

嗟嗟烈祖!有秩斯祜。申錫无疆,及尔斯所。既载清酤,賚我思成。亦有和羹,既戒既平。鬷假无言,时靡有爭。綏我眉寿,黃耇无疆。約軧錯衡,八鸞鸧鸧,以假以享。我受命溥將。自天降康,丰年穰穰。來假來饗,降福无疆。顧予烝嘗,汤孙之將。

玄鸟

天命玄鸟,降而生商。宅殷土芒芒。古帝命武汤,正域彼四方。方命厥後,奄有九有。商之先後,受命不殆,在武丁孙子。武丁孙子,武王靡不勝。龙旗十乘,大糦是承。邦畿千里,维民所止,肇域彼四海。四海來假,來假祁祁。景員维河,殷受命咸宜,百祿是何。

长发

浚哲维商,长发其祥。洪水芒芒,禹敷下土方。外大国是疆,幅隕既长。有娀方將,帝立子生商。

玄王桓撥,受小国是达,受大国是达。率履不越,遂視既发。相土烈烈,海外有截。

帝命不違,至於汤齐。汤降不迟,聖敬日躋。昭假迟迟,上帝是祗。帝命式于九圍。

受小球大球,为下国綴旒,何天之休?不競不絿,不刚不柔,敷政优优,百祿是遒。

受小共大共,为下国駿厖,何天之龙?敷奏其勇。不震不動,不戁不竦,百祿是總。

武王载旆,有虔秉鉞。如火烈烈,則莫我敢曷。苞有三蘗,莫遂莫达,九有有截。韋顧既伐,昆吾夏桀。

昔在中葉,有震且業。允也天子,降予卿士:实维阿衡,实左右商王。

殷武

撻彼殷武,奮伐荊楚,穼入其阻,裒荊之旅。有截其所,汤孙之緒。

维女荊楚,居国南乡。昔有成汤,自彼氐羌,莫敢不來享,莫敢不來王。曰商是常。

天命多辟,設都于禹之績。歲事來辟,勿予禍適。稼穡匪解。

天命降監,下民有嚴。不僭不濫,不敢怠遑。命于下国,封建厥福。

商邑翼翼,四方之極。赫赫厥声,濯濯厥灵。寿考且寧,以保我後生。

陟彼景山,松柏丸丸。是断是遷,方斫是虔。松桷有梴,旅楹有閑,寢成孔安!

民亦勞止,汔可小愒。惠此中国,俾民憂泄。无縱詭隨,以謹醜厲。式遏寇虐,无俾正敗。戎虽小子,而式弘大。

民亦勞止,汔可小安。惠此中国,国无有殘。无縱詭隨,以謹繾綣。式遏寇虐,无俾正反。王欲玉女,是用大諫。

板

上帝板板,下民卒癉。出話不然,为猶不遠。

靡聖管管,不实于亶。猶之未遠,是用大諫。

天之方难,无然憲憲;天之方蹶,无然泄泄。

辭之輯矣,民之洽矣;辭之懌矣,民之莫矣。

我虽異事,及尔同寮。我即尔謀,聽我囂囂。

我言维服,勿以为笑。先民有言:詢于芻蕘。

天之方虐,无然謔謔。老夫灌灌,小子蹻蹻。

匪我言耄,尔用憂謔。多將熇熇,不可救藥。

天之方懠,无为夸毗。威儀卒迷,善人载尸。

民之方殿屎,則莫我敢葵。喪亂蔑資,曾莫惠我師。

天之牖民,如塤如篪,如璋如圭,如取如攜。

攜无曰益,牖民孔易。民之多辟,无自立辟。

價人维藩,大師维垣,大邦维屏,大宗维翰。

懷德维寧,宗子维城。无俾城壞,无獨斯畏。

敬天之怒,无敢戲豫;敬天之渝,无敢馳驅。

昊天曰明,及尔出王;昊天曰旦,及尔游衍。




大雅·荡之什


荡

荡荡上帝,下民之辟。疾威上帝,其命多辟。

天生烝民,其命匪諶。靡不有初,鮮克有終。

文王曰:咨!咨女殷商。曾是強御,曾是掊克,

曾是在位,曾是在服。天降滔德,女興是力。

文王曰:咨!咨女殷商。而秉義類,強御多懟。

流言以對,寇攘式內。侯作侯祝,靡屆靡究。

文王曰:咨!咨女殷商。女炰烋于中国,斂怨以为德。

不明尔德,时无背无側;尔德不明,以无陪无卿。

文王曰:咨!咨女殷商。天不湎尔以酒,不義從式。

既愆尔止,靡明靡晦。式號式呼,俾晝作夜。

文王曰:咨!咨女殷商。如蜩如螗,如沸如羹。

小大近喪,人尚乎由行。內奰于中国,覃及鬼方。

文王曰:咨!咨女殷商。匪上帝不时,殷不用旧。

虽无老成人,尚有典刑。曾是莫聽,大命以傾。

文王曰:咨!咨女殷商。人亦有言:顛沛之揭,

枝葉未有害,本实先撥。殷鑒不遠,在夏後之世!

抑

抑抑威儀,维德之隅。人亦有言:靡哲不愚。庶人之愚,亦職维疾;哲人之愚,亦维斯戾。

无競维人,四方其訓之;有覺德行,四国順之。訏謨定命,遠猶辰告。敬慎威儀,维民之則。

其在于今,興迷亂于政;顛覆厥德,荒湛于酒。女虽湛乐從。弗念厥紹,罔敷求先王,克共明刑。

肆皇天弗尚,如彼泉流,无淪胥以亡。夙興夜寐,灑掃庭內,维民之章。

修尔車马,弓矢戎兵,用戒戎作,用逖蠻方。質尔人民,謹尔侯度,用戒不虞。

慎尔出話,敬尔威儀,无不柔嘉。白圭之玷,尚可磨也;斯言之玷,不可为也。

无易由言,无曰茍矣;莫捫朕舌,言不可逝矣。无言不讎,无德不報。

惠于朋友,庶民小子。子孙繩繩,万民靡不承。視尔友君子,輯柔尔顏,不遐有愆。

相在尔室,尚不愧于屋漏。无曰不显,莫予云覯。神之格思,不可度思,矧可射思?

辟尔为德,俾臧俾嘉。淑慎尔止,不愆于儀。不僭不賊,鮮不为則。

投我以桃,報之以李。彼童而角,实虹小子。荏染柔木,言緡之丝。

溫溫恭人,维德之基。其维哲人,告之話言,順德之行;

其维愚人,覆謂我僭:民各有心。於乎小子!未知臧否。匪手攜之,言示之事;

匪面命之,言提其耳。借曰未知,亦既抱子。民之靡盈,誰夙知而莫成?

昊天孔昭,我生靡乐。視尔夢夢,我心慘慘。誨尔諄諄,聽我藐藐。

匪用为教,覆用为虐。借曰未知,亦聿既耄。

於乎小子!告尔旧止。聽用我謀,庶无大悔。天方艱难,曰喪厥国。

取譬不遠,昊天不忒。回遹其德,俾民大棘。

桑柔

菀彼桑柔,其下侯旬。捋采其刘,瘼此下民。不殄心憂,倉兄填兮;

倬彼昊天,寧不我矜。四牡騤騤,旟旐有翩。亂生不夷,靡国不泯。

民靡有黎,具禍以燼。於乎有哀!国步斯頻。国步蔑資,天不我將;

靡所止疑,云徂何往?君子实维,秉心无競。誰生厲階?至今为梗。

憂心殷殷,念我土宇。我生不辰,逢天僤怒。自西徂東,靡所定處;

多我覯痻,孔棘我圉。为謀为毖,亂況斯削。告尔憂恤,誨尔序爵。

誰能執熱,逝不以濯?其何能淑?载胥及溺。如彼溯風,亦孔之僾;

民有肅心,荓云不逮。好是稼穡,力民代食;稼穡维寶,代食维好。

天降喪亂,滅我立王。降此蟊賊,稼穡卒癢。哀恫中国,具贅卒荒;

靡有旅力,以念穹蒼。维此惠君,民人所瞻。秉心宣猶,考慎其相。

维彼不順,自獨俾臧。自有肺腸,俾民卒狂。

瞻彼中林,甡甡其鹿。朋友已譖,不胥以穀。人亦有言:進退维穀。

维此聖人,瞻言百里;维彼愚人,覆狂以喜。匪言不能,胡斯畏忌。

维此良人,弗求弗迪;维彼忍心,是顧是复。民之貪亂,寧为荼毒!

大風有隧,有空大穀。维此良人,作为式穀;维彼不順,征以中垢。

大風有隧,貪人敗類。聽言則對,誦言如醉。匪用其良,覆俾我悖。

嗟尔朋友!予豈不知而作?如彼飞蟲,时亦弋獲。既之陰女,反予來赫。

民之罔極,職涼善背;为民不利,如云不克。民之回遹,職競用力。

民之未戾,職盜为寇。涼曰不可,覆背善詈。虽曰匪予,既作尔歌。

云汉

倬彼云汉,昭回于天。王曰:於乎!何辜今之人?天降喪亂,饑饉薦臻。靡神不舉,靡愛斯牲。圭璧既卒,寧莫我聽!

旱既太甚,蘊隆蟲蟲。不殄禋祀,自郊徂宮。上下奠瘞,靡神不宗。后稷不克,上帝不臨;耗斁下土,寧丁我躬!

旱既太甚,則不可推。兢兢業業,如霆如雷。周餘黎民,靡有孑遺。昊天上帝,則不我遺。胡不相畏?先祖于摧。

旱既太甚,則不可沮。赫赫炎炎,云我无所。大命近止,靡瞻靡顧。群公先正,則不我助。父母先祖,胡寧忍予?

旱既太甚,滌滌山川。旱魃为虐,如惔如焚。我心憚暑,憂心如薰。群公先正,則不我聞。昊天上帝,寧俾我遁!

旱既太甚,黽勉畏去。胡寧瘨我以旱?憯不知其故。祈年孔夙,方社不莫。昊天上帝,則不我虞。敬恭明神,宜无悔怒。

旱既太甚,散无友紀。鞫哉庶正,疚哉冢宰。趣马師氏,膳夫左右;靡人不周,无不能止。瞻卬昊天,云如何里?

瞻卬昊天,有嘒其星。大夫君子,昭假无贏。大命近止,无棄尔成。何求为我?以戾庶正。瞻卬昊天,曷惠其寧?

崧高

崧高维岳,駿極于天。维岳降神,生甫及申。

维申及甫,维周之翰。四国于蕃,四方于宣。

亹亹申伯,王纘之事。于邑于謝,南国是式。

王命召伯,定申伯之宅。登是南邦,世執其功。

王命申伯,式是南邦,因是謝人,以作尔庸。

王命召伯,徹申伯土田;王命傅御,遷其私人。

申伯之功,召伯是營。有俶其城,寢庙既成,既成藐藐;

王錫申伯,四牡蹻蹻,鉤膺濯濯。

王遣申伯,路車乘马。我圖尔居,莫如南土。

錫尔介圭,以作尔寶。往近王舅,南土是保。

申伯信邁,王餞于郿。申伯還南,謝于誠归。

王命召伯,徹申伯土疆,以峙其粻,式遄其行。

申伯番番,既入于謝,徒御嘽嘽。周邦咸喜,戎有良翰。不显申伯,王之元舅,文武是憲。

申伯之德,柔惠且直。揉此万邦,聞于四国。吉甫作誦,其詩孔碩;其風肆好,以贈申伯。

烝民

天生烝民,有物有則。民之秉彝,好是懿德。天監有周,昭假于下。保茲天子,生仲山甫。

仲山甫之德,柔嘉维則。令儀令色,小心翼翼;古訓是式,威儀是力。天子是若,明命使賦。

王命仲山甫:式是百辟,纘戎祖考,王躬是保,出納王命。王之喉舌,賦政于外,四方爰发。

肅肅王命,仲山甫將之;邦国若否,仲山甫明之。既明且哲,以保其身。夙夜匪解,以事一人。

人亦有言:柔則茹之,刚則吐之。维仲山甫,柔亦不茹,刚亦不吐;不侮矜寡,不畏強御。

人亦有言:德輶如毛,民鮮克舉之,我儀圖之。维仲山甫舉之,愛莫助之。袞職有闕,维仲山甫補之。

仲山甫出祖,四牡業業,征夫捷捷,每懷靡及。四牡彭彭,八鸞鏘鏘,王命仲山甫,城彼東方。

四牡騤騤,八鸞喈喈,仲山甫徂齐,式遄其归。吉甫作誦,穆如清風。仲山甫永懷,以慰其心。

韓奕

奕奕梁山,维禹甸之,有倬其道。韓侯受命,王親命之:纘戎祖考。无廢朕命,夙夜匪解,虔共尔位。朕命不易,干不庭方,以佐戎辟。

四牡奕奕,孔修且张,韓侯入覲,以其介圭,入覲于王。王錫韓侯:淑旗綏章,簟茀錯衡,玄袞赤舄,鉤膺鏤钖,鞹鞃淺幭,鞗革金厄。

韓侯出祖,出宿于屠。显父餞之,清酒百壺。其殽维何?炰鱉鮮鱼。其蔌维何?维筍及蒲。其贈维何?乘马路車。籩豆有且,侯氏燕胥。

韓侯取妻,汾王之甥,蹶父之子。韓侯迎止,于蹶之里。百兩彭彭,八鸞鏘鏘,不显其光。諸娣從之,祁祁如云。韓侯顧之,爛其盈門。

蹶父孔武,靡国不到。为韓姞相攸,莫如韓乐。孔乐韓土,川澤訏訏,魴鱮甫甫,麀鹿噳噳,有熊有羆,有貓有虎。慶既令居,韓姞燕譽。

溥彼韓城,燕師所完。以先祖受命,因时百蠻。王錫韓侯,其追其貊,奄受北国,因以其伯。实墉实壑,实畝实籍。獻其貔皮,赤豹黃羆。

江汉

江汉浮浮,武夫滔滔。匪安匪游,淮夷來求。

既出我車,既設我旟,匪安匪舒,淮夷來鋪。

江汉汤汤,武夫洸洸。經營四方,告成于王。

四方既平,王国庶定。时靡有爭,王心载寧。

江汉之滸,王命召虎,式辟四方,徹我疆土。

匪疚匪棘,王国來極。于疆于理,至于南海。

王命召虎,來旬來宣;文武受命,召公维翰。

无曰:予小子,召公是似。肇敏戎公,用錫尔祉。

厘尔圭瓚,秬鬯一卣,告于文人。錫山土田,

于周受命,自召祖命。虎拜稽首,天子万年。

虎拜稽首,對揚王休。作召公考,天子万寿。

明明天子,令聞不已;矢其文德,洽此四国。

常武

赫赫明明,王命卿士,南仲大祖,大師皇父。

整我六師,以修我戎。既敬既戒,惠此南国。

王謂尹氏,命程伯休父,左右陳行,戒我師旅:

率彼淮浦,省此徐土,不留不處,三事就緒。

赫赫業業,有嚴天子,王舒保作。匪紹匪游,

徐方繹騷。震驚徐方,如雷如霆,徐方震驚。

王奮厥武,如震如怒。進厥虎臣,闞如虓虎。

鋪敦淮濆,仍執醜虜。截彼淮浦,王師之所。

王旅嘽嘽,如飞如翰,如江如汉。如山之苞,

如川之流。绵绵翼翼,不測不克,濯征徐国。

王猶允塞,徐方既來。徐方既同,天子之功。

四方既平,徐方來庭。徐方不回,王曰:還归。

瞻卬

瞻卬昊天,則不我惠。孔填不寧,降此大厲。邦靡有定,

士民其瘵。蟊賊蟊疾,靡有夷屆。罪罟不收,靡有夷瘳。

人有土田,女反有之;人有民人,女覆奪之。此宜无罪,

女反收之;彼宜有罪,女覆說之。哲夫成城,哲婦傾城。

懿厥哲婦,为梟为鴟。婦有长舌,维厲之階。

亂匪降自天,生自婦人。匪教匪誨,时维婦寺。

鞫人忮忒,譖始竟背。豈曰不極?伊胡为慝!

如賈三倍,君子是識。婦无公事,休其蠶織。

天何以刺?何神不富?舍尔介狄,维予胥忌。

不吊不祥,威儀不類。人之云亡,邦国殄瘁。

天之降罔,维其优矣。人之云亡,心之憂矣。

天之降罔,维其幾矣。人之云亡,心之悲矣。

觱沸檻泉,维其深矣。心之憂矣,寧自今矣。

不自我先,不自我後。藐藐昊天,无不克鞏。

无忝皇祖,式救尔後。

召旻

旻天疾威,天篤降喪,瘨我饑饉,民卒流亡。我居圉卒荒。

天降罪罟,蟊賊內訌。昏椓靡共,潰潰回遹,实靖夷我邦。

皋皋訾訾,曾不知其玷。兢兢業業,孔填不寧,我位孔貶。

如彼歲旱,草不潰茂,如彼棲苴。我相此邦,无不潰止。

维昔之富,不如时;维今之疚,不如茲。彼疏斯粺,胡不自替,職兄斯引?

池之竭矣,不云自頻?泉之竭矣,不云自中?溥斯害矣,職兄斯弘,不災我躬?

昔先王受命,有如召公,日辟国百里;今也日蹙国百里。

於乎哀哉!维今之人,不尚有旧。




周頌·清庙之什

清庙

於穆清庙,肅雍显相。济济多士,秉文之德。對越在天,駿奔走在庙。不显不承?无射於人斯。

维天之命

维天之命,於穆不已。於乎不显!文王之德之純。假以溢我,我其收之。駿惠我文王,曾孙篤之。

维清

维清緝熙,文王之典。肇禋。迄用有成,维周之禎。《维清》

烈文

烈文辟公,錫茲祉福,惠我无疆,子孙保之。无封靡于尔邦,维王其崇之。念茲戎功,继序其皇之。无競维人,四方其訓之。不显维德,百辟其刑之。於乎!前王不忘。

天作

天作高山,大王荒之。彼作矣,文王康之。彼徂矣,岐有夷之行。子孙保之。

昊天有成命

昊天有成命,二後受之。成王不敢康,夙夜基命宥密。於緝熙,單厥心,肆其靖之。

我將

我將我享,维羊维牛,维天其右之。儀式刑文王之典,日靖四方。伊嘏文王,既右饗之。我其夙夜,畏天之威,于时保之。

时邁

时邁其邦,昊天其子之,实右序有周。薄言震之,莫不震疊。懷柔百神,及河喬岳。允王维後。

明昭有周,式序在位。载戢干戈,载櫜弓矢。我求懿德,肆于时夏,允王保之。

執競

執競武王,无競维烈。不显成康?上帝是皇。自彼成康,奄有四方,斤斤其明。鐘鼓喤喤,磬筦將將,降福穰穰。降福簡簡,威儀反反。既醉既飽,福祿來反。

思文

思文后稷,克配彼天。立我烝民,莫匪尔極。貽我來牟,帝命率育,无此疆尔界,陳常于时夏。




周颂·臣工之什


臣工

嗟嗟臣工,敬尔在公。王厘尔成,來咨來茹。嗟嗟保介,维莫之春。亦又何求?如何新畬?於皇來牟,將受厥明。明昭上帝,迄用康年。命我眾人,庤乃錢镈,奄觀铚艾。

噫嘻

噫嘻成王,既昭假尔。率时農夫,播厥百穀。駿发尔私,終三十里。亦服尔耕,十千维耦。

振鷺

振鷺于飞,于彼西雍。我客戾止,亦有斯容。在彼无惡,在此无斁。庶幾夙夜,以永終譽。

丰年

丰年多黍多稌,亦有高廩,万億及秭。为酒为醴,烝畀祖妣,以洽百禮。降福孔皆。

有瞽

有瞽有瞽,在周之庭。設業設虡,崇牙樹羽,应田縣鼓,鞉磬柷圉。既備乃奏,簫管備舉。喤喤厥声,肅雍和鳴,先祖是聽。我客戾止,永觀厥成。

潛

猗与漆沮,潛有多鱼。有鳣有鮪,鰷鲿鰋鯉。以享以祀,以介景福。

雍

有來雍雍,至止肅肅。相维辟公,天子穆穆。於薦廣牡,相予肆祀。假哉皇考,綏予孝子。宣哲维人,文武维後。燕及皇天,克昌厥後。綏我眉寿,介以繁祉。既右烈考,亦右文母。

载见

载见辟王,曰求厥章。龙旗陽陽,和鈴央央,鞗革有鸧,休有烈光。率见昭考,以孝以享,以介眉寿。永言保之,思皇多祜。烈文辟公,綏以多福,俾緝熙于純嘏。

有客

有客有客,亦白其马。有萋有且,敦琢其旅。有客宿宿,有客信信。言授之縶,以縶其马。薄言追之,左右綏之。既有淫威,降福孔夷。

武

於皇武王,无競维烈。允文文王,克開厥後。嗣武受之,勝殷遏刘,耆定尔功。




周颂·闵予小子之什


闵予小子

闵予小子,遭家不造,嬛嬛在疚。於乎皇考!永世克孝。

念茲皇祖,陟降庭止。维予小子,夙夜敬止。於乎皇王!继序思不忘。

访落

访予落止,率时昭考。於乎悠哉!朕未有艾,將予就之,继猶判渙。维予小子,未堪家多难。紹庭上下,陟降厥家。休矣皇考,以保明其身。

敬之

敬之敬之,天维显思。命不易哉!无曰:高高在上。陟降厥士,日監在茲。维予小子,不聰敬止。日就月將,學有緝熙于光明。佛时仔肩,示我显德行。

小毖

予其懲,而毖後患。莫予荓蜂,自求辛螫。肇允彼桃蟲,拚飞维鸟。未堪家多难,予又集于蓼。

载芟

载芟载柞,其耕澤澤。千耦其耘,徂隰徂畛。侯主侯伯,侯亞侯旅,侯強侯以。有嗿其馌,思媚其婦,有依其士。

有略其耜,俶载南畝,播厥百穀,实函斯活。驛驛其达,有厭其杰,厭厭其苗,绵绵其麃。载獲济济,有实其積,万億及秭。

为酒为醴,烝畀祖妣,以洽百禮。有飶其香,邦家之光。有椒其馨,胡考之寧?匪且有且,匪今斯今,振古如茲。

良耜

畟畟良耜,俶载南畝,播厥百穀,实函斯活。或來瞻女,载筐及筥。其馕伊黍,其笠伊糾,其镈斯趙,以薅荼蓼。荼蓼朽止,黍稷茂止。獲之挃挃,積之栗栗。其崇如墉,其比如櫛。以開百室。百室盈止,婦子寧止。殺时犉牡,有捄其角。以似以續,續古之人。

丝衣

丝衣其紑,载弁俅俅。自堂徂基,自羊徂牛。鼐鼎及鼒。兕觥其觩,旨酒思柔。不吳不敖,胡考之休?

酌

於鑠王師,遵養时晦。时純熙矣,是用大介。我龙受之,蹻蹻王之造。载用有嗣,实维尔公。允師。

桓

綏万邦,婁丰年,天命匪解。桓桓武王,保有厥士,于以四方,克定厥家。於昭于天,皇以間之。

賚

文王既勤止,我应受之,敷时繹思。我徂维求定,时周之命。於繹思。

般

於皇时周,陟其高山。嶞山喬岳,允猶翕河。敷天之下,裒时之對,时周之命。




鲁颂·駉之什


駉

駉駉牡马,在坰之野。薄言駉者,有驈有皇,有驪有黃,以車彭彭。思无疆,思马斯臧。

駉駉牡马,在坰之野。薄言駉者,有騅有駓,有騂有騏,以車伾伾。思无期,思马斯才。

駉駉牡马,在坰之野。薄言駉者,有驒有駱,有騮有雒。以車繹繹。思无期,思马斯作。

駉駉牡马,在坰之野。薄言駉者,有駰有騢,有駠有鱼,以車繹繹。思无邪,思马斯徂。

有駜

有駜有駜,駜彼乘黃。夙夜在公,在公明明。振振鷺,鷺于下。鼓咽咽,醉言舞。于胥乐兮。

有駜有駜,駜彼乘牡。夙夜在公,在公飲酒。振振鷺,鷺于飞。鼓咽咽,醉言归。于胥乐兮。

有駜有駜,駜彼乘駽。夙夜在公,在公载燕。自今以始,歲其有。君子有穀,詒孙子。于胥乐兮。

泮水

思乐泮水,薄采其芹。鲁侯戾止,言觀其旗。其旗茷茷,鸞声噦噦。无小无大,從公于邁。

思乐泮水,薄采其藻。鲁侯戾止,其马蹻蹻。其马昭昭,其音昭昭。载色载笑,匪怒伊教。

思乐泮水,薄采其茆。鲁侯戾止,在泮飲酒,既飲旨酒,永錫难老。順彼长道,屈此群醜。

穆穆鲁侯,敬明其德。敬慎威儀,维民之則。允文允武,昭假烈祖。靡有不孝,自求伊祜。

明明鲁侯,克明其德,既作泮宮,淮夷攸服。矯矯虎臣,在泮獻馘;淑問如皋陶,在泮獻囚。

济济多士,克廣德心。桓桓于征,狄彼東南。烝烝皇皇,不吳不揚。不告于讻,在泮獻功。

角弓其觩,束矢其搜。戎車孔博,徒御无斁。既克淮夷,孔淑不逆。式固尔猶,淮夷卒獲。

翩彼飞鸮,集于泮林,食我桑黮,懷我好音。憬彼淮夷,來獻其琛:元龜象齿,大賂南金。

閟宮

閟宮有恤,实实枚枚。赫赫姜嫄,其德不回。上帝是依,无災无害;彌月不迟,是生后稷。降之百福,黍稷重穋,稙稚菽麥。奄有下国,俾民稼穡。有稷有黍,有稻有秬。奄有下土,纘禹之緒。后稷之孙,实维大王;居岐之陽,实始翦商。至于文武,纘大王之緒。致天之屆,于牧之野。

无貳无虞,上帝臨女。敦商之旅,克咸厥功。王曰:叔父!建尔元子,俾侯于鲁;大啟尔宇,为周室輔。乃命鲁公,俾侯于東;錫之山川,土田附庸。

周公之孙,莊公之子,龙旗承祀,六轡耳耳。春秋匪解,享祀不忒;皇皇后帝,皇祖后稷,享以騂犧。是饗是宜,降福既多。周公皇祖,亦其福女。

秋而载嘗,夏而楅衡。白牡騂刚,犧尊將將。毛炰胾羹,籩豆大房;万舞洋洋,孝孙有慶。俾尔熾而昌,俾尔寿而臧。保彼東方,鲁邦是常。

不虧不崩,不震不騰。三寿作朋,如岡如陵。公車千乘,朱英綠縢,二矛重弓。公徒三万,貝胄朱綅,烝徒增增。戎狄是膺,荊舒是懲,則莫我敢承。

俾尔昌而熾,俾尔寿而富。黃发台背,寿胥与試。俾尔昌而大,俾尔耆而艾。万有千歲,眉寿无有害。泰山巖巖,鲁邦所詹。奄有龜蒙,遂荒大東,至于海邦。

淮夷來同,莫不率從,鲁侯之功。保有鳧繹,遂荒徐宅,至于海邦。淮夷蠻貊,及彼南夷,莫不率從。莫敢不諾,鲁侯是若。天錫公純嘏,眉寿保鲁;居常与許,复周公之宇。鲁侯燕喜,令妻寿母,宜大夫庶士,邦国是有。

既多受祉,黃发儿齿。徂來之松,新甫之柏,是断是度,是尋是尺。松桷有舄,路寢孔碩。新庙奕奕,奚斯所作。孔曼且碩,万民是若。




商頌


那

猗与那与!置我鞉鼓。奏鼓簡簡,衎我烈祖。汤孙奏假,綏我思成。鞉鼓淵淵,嘒嘒管声。既和且平,依我磬声。於赫汤孙,穆穆厥声。庸鼓有斁,万舞有奕。我有嘉客,亦不夷懌。自古在昔,先民有作。溫恭朝夕,執事有恪。顧予烝嘗,汤孙之將。

烈祖

嗟嗟烈祖!有秩斯祜。申錫无疆,及尔斯所。既载清酤,賚我思成。亦有和羹,既戒既平。鬷假无言,时靡有爭。綏我眉寿,黃耇无疆。約軧錯衡,八鸞鸧鸧,以假以享。我受命溥將。自天降康,丰年穰穰。來假來饗,降福无疆。顧予烝嘗,汤孙之將。

玄鸟

天命玄鸟,降而生商。宅殷土芒芒。古帝命武汤,正域彼四方。方命厥後,奄有九有。商之先後,受命不殆,在武丁孙子。武丁孙子,武王靡不勝。龙旗十乘,大糦是承。邦畿千里,维民所止,肇域彼四海。四海來假,來假祁祁。景員维河,殷受命咸宜,百祿是何。

长发

浚哲维商,长发其祥。洪水芒芒,禹敷下土方。外大国是疆,幅隕既长。有娀方將,帝立子生商。

玄王桓撥,受小国是达,受大国是达。率履不越,遂視既发。相土烈烈,海外有截。

帝命不違,至於汤齐。汤降不迟,聖敬日躋。昭假迟迟,上帝是祗。帝命式于九圍。

受小球大球,为下国綴旒,何天之休?不競不絿,不刚不柔,敷政优优,百祿是遒。

受小共大共,为下国駿厖,何天之龙?敷奏其勇。不震不動,不戁不竦,百祿是總。

武王载旆,有虔秉鉞。如火烈烈,則莫我敢曷。苞有三蘗,莫遂莫达,九有有截。韋顧既伐,昆吾夏桀。

昔在中葉,有震且業。允也天子,降予卿士:实维阿衡,实左右商王。

殷武

撻彼殷武,奮伐荊楚,穼入其阻,裒荊之旅。有截其所,汤孙之緒。

维女荊楚,居国南乡。昔有成汤,自彼氐羌,莫敢不來享,莫敢不來王。曰商是常。

天命多辟,設都于禹之績。歲事來辟,勿予禍適。稼穡匪解。

天命降監,下民有嚴。不僭不濫,不敢怠遑。命于下国,封建厥福。

商邑翼翼,四方之極。赫赫厥声,濯濯厥灵。寿考且寧,以保我後生。

陟彼景山,松柏丸丸。是断是遷,方斫是虔。松桷有梴,旅楹有閑,寢成孔安!

日辟国百里;今也日蹙国百里。

於乎哀哉!维今之人,不尚有旧。




周頌·清庙之什

清庙

於穆清庙,肅雍显相。济济多士,秉文之德。對越在天,駿奔走在庙。不显不承?无射於人斯。

维天之命

维天之命,於穆不已。於乎不显!文王之德之純。假以溢我,我其收之。駿惠我文王,曾孙篤之。

维清

维清緝熙,文王之典。肇禋。迄用有成,维周之禎。《维清》

烈文

烈文辟公,錫茲祉福,惠我无疆,子孙保之。无封靡于尔邦,维王其崇之。念茲戎功,继序其皇之。无競维人,四方其訓之。不显维德,百辟其刑之。於乎!前王不忘。

天作

天作高山,大王荒之。彼作矣,文王康之。彼徂矣,岐有夷之行。子孙保之。

昊天有成命

昊天有成命,二後受之。成王不敢康,夙夜基命宥密。於緝熙,單厥心,肆其靖之。

我將

我將我享,维羊维牛,维天其右之。儀式刑文王之典,日靖四方。伊嘏文王,既右饗之。我其夙夜,畏天之威,于时保之。

时邁

时邁其邦,昊天其子之,实右序有周。薄言震之,莫不震疊。懷柔百神,及河喬岳。允王维後。

明昭有周,式序在位。载戢干戈,载櫜弓矢。我求懿德,肆于时夏,允王保之。

執競

執競武王,无競维烈。不显成康?上帝是皇。自彼成康,奄有四方,斤斤其明。鐘鼓喤喤,磬筦將將,降福穰穰。降福簡簡,威儀反反。既醉既飽,福祿來反。

思文

思文后稷,克配彼天。立我烝民,莫匪尔極。貽我來牟,帝命率育,无此疆尔界,陳常于时夏。




周颂·臣工之什


臣工

嗟嗟臣工,敬尔在公。王厘尔成,來咨來茹。嗟嗟保介,维莫之春。亦又何求?如何新畬?於皇來牟,將受厥明。明昭上帝,迄用康年。命我眾人,庤乃錢镈,奄觀铚艾。

噫嘻

噫嘻成王,既昭假尔。率时農夫,播厥百穀。駿发尔私,終三十里。亦服尔耕,十千维耦。

振鷺

振鷺于飞,于彼西雍。我客戾止,亦有斯容。在彼无惡,在此无斁。庶幾夙夜,以永終譽。

丰年

丰年多黍多稌,亦有高廩,万億及秭。为酒为醴,烝畀祖妣,以洽百禮。降福孔皆。

有瞽

有瞽有瞽,在周之庭。設業設虡,崇牙樹羽,应田縣鼓,鞉磬柷圉。既備乃奏,簫管備舉。喤喤厥声,肅雍和鳴,先祖是聽。我客戾止,永觀厥成。

潛

猗与漆沮,潛有多鱼。有鳣有鮪,鰷鲿鰋鯉。以享以祀,以介景福。

雍

有來雍雍,至止肅肅。相维辟公,天子穆穆。於薦廣牡,相予肆祀。假哉皇考,綏予孝子。宣哲维人,文武维後。燕及皇天,克昌厥後。綏我眉寿,介以繁祉。既右烈考,亦右文母。

载见

载见辟王,曰求厥章。龙旗陽陽,和鈴央央,鞗革有鸧,休有烈光。率见昭考,以孝以享,以介眉寿。永言保之,思皇多祜。烈文辟公,綏以多福,俾緝熙于純嘏。

有客

有客有客,亦白其马。有萋有且,敦琢其旅。有客宿宿,有客信信。言授之縶,以縶其马。薄言追之,左右綏之。既有淫威,降福孔夷。

武

於皇武王,无競维烈。允文文王,克開厥後。嗣武受之,勝殷遏刘,耆定尔功。




周颂·闵予小子之什


闵予小子

闵予小子,遭家不造,嬛嬛在疚。於乎皇考!永世克孝。

念茲皇祖,陟降庭止。维予小子,夙夜敬止。於乎皇王!继序思不忘。

访落

访予落止,率时昭考。於乎悠哉!朕未有艾,將予就之,继猶判渙。维予小子,未堪家多难。紹庭上下,陟降厥家。休矣皇考,以保明其身。

敬之

敬之敬之,天维显思。命不易哉!无曰:高高在上。陟降厥士,日監在茲。维予小子,不聰敬止。日就月將,學有緝熙于光明。佛时仔肩,示我显德行。

小毖

予其懲,而毖後患。莫予荓蜂,自求辛螫。肇允彼桃蟲,拚飞维鸟。未堪家多难,予又集于蓼。

载芟

载芟载柞,其耕澤澤。千耦其耘,徂隰徂畛。侯主侯伯,侯亞侯旅,侯強侯以。有嗿其馌,思媚其婦,有依其士。

有略其耜,俶载南畝,播厥百穀,实函斯活。驛驛其达,有厭其杰,厭厭其苗,绵绵其麃。载獲济济,有实其積,万億及秭。

为酒为醴,烝畀祖妣,以洽百禮。有飶其香,邦家之光。有椒其馨,胡考之寧?匪且有且,匪今斯今,振古如茲。

良耜

畟畟良耜,俶载南畝,播厥百穀,实函斯活。或來瞻女,载筐及筥。其馕伊黍,其笠伊糾,其镈斯趙,以薅荼蓼。荼蓼朽止,黍稷茂止。獲之挃挃,積之栗栗。其崇如墉,其比如櫛。以開百室。百室盈止,婦子寧止。殺时犉牡,有捄其角。以似以續,續古之人。

丝衣

丝衣其紑,载弁俅俅。自堂徂基,自羊徂牛。鼐鼎及鼒。兕觥其觩,旨酒思柔。不吳不敖,胡考之休?

酌

於鑠王師,遵養时晦。时純熙矣,是用大介。我龙受之,蹻蹻王之造。载用有嗣,实维尔公。允師。

桓

綏万邦,婁丰年,天命匪解。桓桓武王,保有厥士,于以四方,克定厥家。於昭于天,皇以間之。

賚

文王既勤止,我应受之,敷时繹思。我徂维求定,时周之命。於繹思。

般

於皇时周,陟其高山。嶞山喬岳,允猶翕河。敷天之下,裒时之對,时周之命。




鲁颂·駉之什


駉

駉駉牡马,在坰之野。薄言駉者,有驈有皇,有驪有黃,以車彭彭。思无疆,思马斯臧。

駉駉牡马,在坰之野。薄言駉者,有騅有駓,有騂有騏,以車伾伾。思无期,思马斯才。

駉駉牡马,在坰之野。薄言駉者,有驒有駱,有騮有雒。以車繹繹。思无期,思马斯作。

駉駉牡马,在坰之野。薄言駉者,有駰有騢,有駠有鱼,以車繹繹。思无邪,思马斯徂。

有駜

有駜有駜,駜彼乘黃。夙夜在公,在公明明。振振鷺,鷺于下。鼓咽咽,醉言舞。于胥乐兮。

有駜有駜,駜彼乘牡。夙夜在公,在公飲酒。振振鷺,鷺于飞。鼓咽咽,醉言归。于胥乐兮。

有駜有駜,駜彼乘駽。夙夜在公,在公载燕。自今以始,歲其有。君子有穀,詒孙子。于胥乐兮。

泮水

思乐泮水,薄采其芹。鲁侯戾止,言觀其旗。其旗茷茷,鸞声噦噦。无小无大,從公于邁。

思乐泮水,薄采其藻。鲁侯戾止,其马蹻蹻。其马昭昭,其音昭昭。载色载笑,匪怒伊教。

思乐泮水,薄采其茆。鲁侯戾止,在泮飲酒,既飲旨酒,永錫难老。順彼长道,屈此群醜。

穆穆鲁侯,敬明其德。敬慎威儀,维民之則。允文允武,昭假烈祖。靡有不孝,自求伊祜。

明明鲁侯,克明其德,既作泮宮,淮夷攸服。矯矯虎臣,在泮獻馘;淑問如皋陶,在泮獻囚。

济济多士,克廣德心。桓桓于征,狄彼東南。烝烝皇皇,不吳不揚。不告于讻,在泮獻功。

角弓其觩,束矢其搜。戎車孔博,徒御无斁。既克淮夷,孔淑不逆。式固尔猶,淮夷卒獲。

翩彼飞鸮,集于泮林,食我桑黮,懷我好音。憬彼淮夷,來獻其琛:元龜象齿,大賂南金。

閟宮

閟宮有恤,实实枚枚。赫赫姜嫄,其德不回。上帝是依,无災无害;彌月不迟,是生后稷。降之百福,黍稷重穋,稙稚菽麥。奄有下国,俾民稼穡。有稷有黍,有稻有秬。奄有下土,纘禹之緒。后稷之孙,实维大王;居岐之陽,实始翦商。至于文武,纘大王之緒。致天之屆,于牧之野。

无貳无虞,上帝臨女。敦商之旅,克咸厥功。王曰:叔父!建尔元子,俾侯于鲁;大啟尔宇,为周室輔。乃命鲁公,俾侯于東;錫之山川,土田附庸。

周公之孙,莊公之子,龙旗承祀,六轡耳耳。春秋匪解,享祀不忒;皇皇后帝,皇祖后稷,享以騂犧。是饗是宜,降福既多。周公皇祖,亦其福女。

秋而载嘗,夏而楅衡。白牡騂刚,犧尊將將。毛炰胾羹,籩豆大房;万舞洋洋,孝孙有慶。俾尔熾而昌,俾尔寿而臧。保彼東方,鲁邦是常。

不虧不崩,不震不騰。三寿作朋,如岡如陵。公車千乘,朱英綠縢,二矛重弓。公徒三万,貝胄朱綅,烝徒增增。戎狄是膺,荊舒是懲,則莫我敢承。

俾尔昌而熾,俾尔寿而富。黃发台背,寿胥与試。俾尔昌而大,俾尔耆而艾。万有千歲,眉寿无有害。泰山巖巖,鲁邦所詹。奄有龜蒙,遂荒大東,至于海邦。

淮夷來同,莫不率從,鲁侯之功。保有鳧繹,遂荒徐宅,至于海邦。淮夷蠻貊,及彼南夷,莫不率從。莫敢不諾,鲁侯是若。天錫公純嘏,眉寿保鲁;居常与許,复周公之宇。鲁侯燕喜,令妻寿母,宜大夫庶士,邦国是有。

既多受祉,黃发儿齿。徂來之松,新甫之柏,是断是度,是尋是尺。松桷有舄,路寢孔碩。新庙奕奕,奚斯所作。孔曼且碩,万民是若。




商頌


那

猗与那与!置我鞉鼓。奏鼓簡簡,衎我烈祖。汤孙奏假,綏我思成。鞉鼓淵淵,嘒嘒管声。既和且平,依我磬声。於赫汤孙,穆穆厥声。庸鼓有斁,万舞有奕。我有嘉客,亦不夷懌。自古在昔,先民有作。溫恭朝夕,執事有恪。顧予烝嘗,汤孙之將。

烈祖

嗟嗟烈祖!有秩斯祜。申錫无疆,及尔斯所。既载清酤,賚我思成。亦有和羹,既戒既平。鬷假无言,时靡有爭。綏我眉寿,黃耇无疆。約軧錯衡,八鸞鸧鸧,以假以享。我受命溥將。自天降康,丰年穰穰。來假來饗,降福无疆。顧予烝嘗,汤孙之將。

玄鸟

天命玄鸟,降而生商。宅殷土芒芒。古帝命武汤,正域彼四方。方命厥後,奄有九有。商之先後,受命不殆,在武丁孙子。武丁孙子,武王靡不勝。龙旗十乘,大糦是承。邦畿千里,维民所止,肇域彼四海。四海來假,來假祁祁。景員维河,殷受命咸宜,百祿是何。

长发

浚哲维商,长发其祥。洪水芒芒,禹敷下土方。外大国是疆,幅隕既长。有娀方將,帝立子生商。

玄王桓撥,受小国是达,受大国是达。率履不越,遂視既发。相土烈烈,海外有截。

帝命不違,至於汤齐。汤降不迟,聖敬日躋。昭假迟迟,上帝是祗。帝命式于九圍。

受小球大球,为下国綴旒,何天之休?不競不絿,不刚不柔,敷政优优,百祿是遒。

受小共大共,为下国駿厖,何天之龙?敷奏其勇。不震不動,不戁不竦,百祿是總。

武王载旆,有虔秉鉞。如火烈烈,則莫我敢曷。苞有三蘗,莫遂莫达,九有有截。韋顧既伐,昆吾夏桀。

昔在中葉,有震且業。允也天子,降予卿士:实维阿衡,实左右商王。

殷武

撻彼殷武,奮伐荊楚,穼入其阻,裒荊之旅。有截其所,汤孙之緒。

维女荊楚,居国南乡。昔有成汤,自彼氐羌,莫敢不來享,莫敢不來王。曰商是常。

天命多辟,設都于禹之績。歲事來辟,勿予禍適。稼穡匪解。

天命降監,下民有嚴。不僭不濫,不敢怠遑。命于下国,封建厥福。

商邑翼翼,四方之極。赫赫厥声,濯濯厥灵。寿考且寧,以保我後生。

陟彼景山,松柏丸丸。是断是遷,方斫是虔。松桷有梴,旅楹有閑,寢成孔安!

