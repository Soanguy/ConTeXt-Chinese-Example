\usemodule[memos][
    themecolor=cyan,
    layout=moderate,
    hdrstyle=foemarginalt,
    fontstyle=ss,
    fontsize=10pt,
    lineheight=1.75\bodyfontsize, 
    ]
\start
\language[en]
\startsectionlevel[title={Expanding TeX's \letterbackslash{}newif},reference={expanding-texs-newif}]

%https://mht.wtf/post/tex/

\startsectionlevel[title={Introduction},reference={introduction}]

Like most of my colleagues, I use LaTeX to write papers, reports, notes,
or what have you. In fact, I think all of the places that I regularly
write supports some variable subset of LaTeX. Also like most of my
colleagues, I'm not a TeXnician. I'm not proud to be ignorant in this
regard, but there's only so many hours in a day, and the gains from
properly learning a huge ecosystem like LaTeX seems minuscule compared
to the initial buy-in cost.

Still, I was curious.

LaTeX and TeX, tomato tomato? Here's how I see it. If LaTeX is like
C++20 --- big, complex, confusing, full of cruft, but still very popular
--- then TeX is like C89 --- small,
simpler\high{\goto{1}[url(https://mht.wtf/post/tex/\#user-content-fn-simple)]},
confusing, a child of its time, and often neglected.

There's a certain pleasure in going far enough down the stack that the
systems you are using becomes simple enough to reason about on a deep
level. It's the feeling you might get sitting down one afternoon trying
to write some assembly after a long week of debugging consistency errors
in your sharded database across multiple kubernetes
clusters\high{\goto{2}[url(https://mht.wtf/post/tex/\#user-content-fn-kube)]}.
No magic, no need to constantly search for other people who's had the
same problems you're dealing with on StackOverflow. It's just you and
the CPU, and likely the Intel Instruction Set Manual or something as big
and scary. I wanted that, but with typesetting.

This was my romantic motivation to dig into TeX and try to see whether
it really is rewarding to step back a few decades to avoid the
complexity of newer and bigger typesetting systems. I bought the
TeXbook, and read it from start to finish. Well, some paragraphs are
marked with \quotation{dangerous bends}, signalling that the content
covered or the background assumed for those paragraphs are more
advanced. I read the single bends, but skipped the double bends, at
least most of the time.

Somewhere in the book I found the definition of~\type{\newif}, a macro
that's used to define conditionals, which you can later query, and
branch on. Booleans, in other words. I read it, and really didn't
understand a single thing, and I figured that if I can manage to sit
down and figure out what on earth this macro is doing and why, then I've
had a good taste of what it's like digging down this low in the world of
TeX.

This post is the result of that process.

\startsectionlevel[title={How Do I Write
TeX?},reference={how-do-i-write-tex}]

This is not really as obvious as it might sound. After all, TeX produces
a document, but when playing with macros we really want to see what
forms expand to, which macros are defined, and so on. I have to say
upfront that the method I used here probably wasn't the ideal, because I
just started used~\type{tex}~(or sometimes~\type{pdftex}, for the
purposes of this post they seem to be exactly the same), and started
writing. The repl doesn't support~\type{readline}~bindings or arrow
keys, or clicking to move the cursor, so if I wanted to add something in
the middle of a line, I had to hold backspace all the way back to where
I wanted to go and write out the rest of the expression. Sometimes I
pasted back and forth from a text editor, which worked okay.

Here's exactly how I got
started\high{\goto{3}[url(https://mht.wtf/post/tex/\#user-content-fn-arch)]}.

\starttyping
/h/martin$ tex
This is TeX, Version 3.141592653 (TeX Live 2021/Arch Linux) (preloaded format=tex)
**\relax  % don't read input from a file

*\tracingall=1                 % Give us lots of output
{vertical mode: \tracingstats}
{\tracingpages}
{\tracingoutput}
{\tracinglostchars}
{\tracingmacros}
{\tracingparagraphs}
{\tracingrestores}
{\showboxbreadth}
{\showboxdepth}
{the character =}
{horizontal mode: the character =}
{blank space  }

*\message{This will show somewhere}    % some sample message
{\message}
This will show somewhere               % here's the things you wrote above
{blank space  }

*\def\mymacro{from the macro}  % Make a new macro
{\def}
{blank space  }

*\message{\mymacro}         % \message will expand the macro
{\message}

\mymacro ->from the macro   % \mymacro is expanded to `from the macro`
from the macro              % ... and we get the fully expanded form out.
{blank space  }

*
\stoptyping

Input lines start with a~\type{*}. It's very useful to
set~\type{\tracingall=1}, which makes TeX output a bunch of things some
of which you care about. Note that I've changed up the formatting of the
output throughout this post so that it's easier to see what's going on.

Another quick note: I didn't want to spend hours write an intro to TeX
as well as whatever this is, so if you have never written a line of TeX
or LaTeX, this might be difficult to follow. If you've written some
LaTeX, and maybe defined your own simple macros, I think you'll be fine.

\stopsectionlevel

\stopsectionlevel

\startsectionlevel[title={The Goal},reference={the-goal}]

This is the definition we'll unravel, copied verbatim from The TeXbook.

\starttyping
\outer\def\newif#1{\count@=\escapechar \escapechar=-1
  \expandafter\expandafter\expandafter
   \def\@if#1{true}{\let#1=\iftrue}%
  \expandafter\expandafter\expandafter
   \def\@if#1{false}{\let#1=\iffalse}%
  \@if#1{false}\escapechar=\count@} % the condition starts out false
\def\@if#1#2{\csname\expandafter\if@\string#1#2\endcsname}
{\uccode`1=`i \uccode`2=`f \uppercase{\gdef\if@12{}}} % `if` is required
\stoptyping

Don't despair if this is nonsense: the whole point of this post is to
explain what's going on, and to get some better idea of how real and
(somewhat) involved TeX macros work.

\stopsectionlevel

\startsectionlevel[title={How TeX Reads
Tokens},reference={how-tex-reads-tokens}]

To start on the right foot, let's make sure that we properly understand
how TeX reads tokens. A token is the input \quotation{unit} that TeX
reads when it reads a document. For instance if you were to
write~\type{Let $n=\numb$ be a number.}~then this will be transformed
into a queue of tokens from which we will read one at a time. Exactly
how the tokens are split up is not crucial to understanding, but in this
example it looks something like this:

\starttyping
tokens = ['L', 'e', 't', ' ', $, 'n', '=', \numb, $, ...]
\stoptyping

Notice three things. First, a letter is a token in of itself and we do
not have one \quotation{word} be a token Second,~\type{$}~is not the
character~\type{'$'}, but the special begin/end math mode token. If we
were to write~\type{\$}~we would get the character token~\type{'$'}.
Third, the whole macro~\type{\numb}~is one single token. When you hear
\quotation{token}, think \quotation{input unit}.

So how does TeX read the tokens? One mental model is like this:

\starttyping
while tokens is not empty
    t <- pop(tokens)
    if shouldexpand(t)
        exp <- expand(t)
        tokens.push(ex)
    else
        process(t)
\stoptyping

Some tokens, like the~\type{\newif}~token we will figure out in this
post, expand, and the expansion is another list of tokens, some of which
might be regular character tokens, and some of which might be other
tokens that also expand. Therefore when we expand a token we will push
the result back onto the front of the queue.

Note that when we expand a macro that takes arguments,
like~\type"\def\paren#1{(#1)}"~the expansion of~\type{\paren}~will pop
more tokens from the queue, and then push the tokens of the expanded
form back onto the queue.

What does it mean to \quotation{process} a token? For a character, this
basically means to write that character at the current position on the
page\high{\goto{4}[url(https://mht.wtf/post/tex/\#user-content-fn-writechar)]}.
For a macro definition like~\type"\def\bob{123}"~it means to make the
definition and storing it somewhere in memory so that if you ever
encounter a~\type{\bob}~token you know that it expands to the three
tokens~\type{1},\type{2},\type{3}.

\startsectionlevel[title={A Short Example},reference={a-short-example}]

Let~\type"\def\A{a} \def\B{\A b} \def\C{\B\B}"~and the input token queue
be~\type{[\C]}. To make sure we understand how this works, let's
manually expand this whole thing. The left column is the token queue,
and the left side of the queue is the front, which is the place at which
we will be working. The right column explains what we're about to do.

Tokens \letterbar{} Current action
:------\letterbar{}:--------------~\type{[\C]}~\letterbar{}
take~\type{\C}~out of the front of the
queue~\type{[]}~\letterbar{}~\type{\C}expands to~\type{\B\B}, which we
push back~\type{[\B \B]}~\letterbar{} take the
first~\type{\B}~out~\type{[\B]}~\letterbar{}~\type{\B}~expands
to~\type{\A b``[\A b \B]}~\letterbar{}~\type{\A}~is taken out, expanded
to~\type{a}~and pushed back~\type{[ a b \B]}~\letterbar{}~\type{a}~is
taken out and processed, because it doesn't
expand.~\type{[ b \B]}~\letterbar{}~\type{b}~is taken out and
processed.~\type{[\B]}~\letterbar{} you get the
idea\ldots{}~\type{[\A b]}~\letterbar{}~\type{[ a b]}~\letterbar{}~\type{[ b]}~\letterbar{}~\type{[]}~\letterbar{}

The end result of this execution is that we have sent the
tokens~\type{a},~\type{b},~\type{a},~\type{b}~to the processing part of
TeX.

\stopsectionlevel

\stopsectionlevel

\startsectionlevel[title={A Primer on
Catcodes},reference={a-primer-on-catcodes}]

We need to know one more thing about tokens, or rather how the
characters of your input are split into them. Each character have a~{\em
category code}, or catcode for short. Catcodes decide how to group and
split characters into a token. There is a character code for letters
(11), a code for space (10), and one for math shift (3) (there are also
others). This way TeX knows that in the input~\type{let $}~consists of
three characters, one space, and one \quotation{math shift}. This is
also how TeX figures out when the name of a macro ends and new tokens
begin, as in~\type{\hey3}: here we have one token with catcode 0 (the
escape character~\type{\}), three of catcode 11, and one of catcode 12
(\quotation{others}, which include numbers). The name of a macro is only
letters, so this way TeX knows that~\type{\hey}~is a macro
and~\type{3}~is just the next token in the queue.

But catcodes can be changed. Why is this useful? Well, if we would like
to make some macros that another user wouldn't accidentally redefine we
have it include a character that, by default, isn't allowed to be in its
name, like~\type{@}. The catcode of~\type{@}~is 12, and so the
input~\type{\h@}~will be read as two tokens~\type{\h}~and~\type{'@'}.
However, if we change the catcode of~\type{@}~to 11 it's as
if~\type{@}~is just a regular letter, and~\type{\h@}~will be read as a
single token~\type{\h@}.

This is how we change the catcode of~\type{@}~to 11 and then back to 12:

\starttyping
*\catcode`\@=11  % Category 11 consists of regular letters
*\catcode`\@=12  % Category 12 consists of "other characters"
\stoptyping

\stopsectionlevel

\startsectionlevel[title={Some Not So Bad
Macros},reference={some-not-so-bad-macros}]

We need to know about a few other macros that~\type{\newif}~uses
internally. Most of these are pretty straight forward.

\startsectionlevel[title={\type{\string}},reference={string}]

Takes an argument and replaces it by the non-expanded token
list.~\type{\string\foo}~expands to the four tokens~\type{\ f o o}, no
matter what the macro~\type{\foo}~would expand to. A crucial detail
which we will come back to is that the tokens~\type{\string}~produces
will get catcode 12 (unless it's a space).

\stopsectionlevel

\startsectionlevel[title={\type{\escapechar}},reference={escapechar}]

The character which is used when a control sequence is outputted as
text. Normally set to~\type{\}. If this is set to for instance~\type{@},
then~\type{\string\foo}~would expand to the four
tokens~\type{@ f o o}~instead.

\stopsectionlevel

\startsectionlevel[title={\type{\uccode}},reference={uccode}]

Short for uppercase code. This allows one to set the uppercase character
code for another letter. Usually this would
be~\type{\uccode}x=\type{X \uccode}X=\type{X}~and so on, but this, like
most things in TeX, can be changed, and changes, like most things in
TeX, are local to the current group.

\stopsectionlevel

\startsectionlevel[title={\type{\csname}~and~\type{\endcsname}},reference={csname-and-endcsname}]

Read and expand everything up until the matching~\type{\endcsname}. The
expansion result should be a list of character tokens, and this list
will be made into a single control sequence token. If this is currently
not defined it will be defined to~\type{\relax}.

For instance~\type{\csname hello\endcsname}~will expand to the single
token~\type{\hello}~and make the macro~\type{\hello}~expand
to~\type{\relax}. More
interestingly,~\type"\def\inner{hello}\csname\inner\endcsname"will do
the same: Here the~\type{inner}~macro expands to the list of
tokens~\type{h e l l o}, and the~\type{csname}pair of macros expand this
macro, effectively replacing it with~\type{\csname hello\endcsname}.

\stopsectionlevel

\startsectionlevel[title={\type{\gdef}},reference={gdef}]

Normally definitions made with~\type{\def}~are local to your scope, just
like in most programming languages. However, sometimes we want to define
global macros, and~\type{gdef}~does exactly this. When a macro is
defined with~\type{\gdef}~it is as if it was defined in the top level
scope.

\starttyping
{
    \def\inner{hello}
    \inner  % expands to  h e l l o
}
\inner   % this doesn't work, because \inner is no longer defined

{
    \gdef\inner{hello}
    \inner  % expands to  h e l l o
}
\inner % also expands to  h e l l o
\stoptyping

\stopsectionlevel

\startsectionlevel[title={\type{\outer}},reference={outer}]

This is a safety measure that you put before a~\type{\def}~which ensures
that this macro is not allowed to be an argument, in the parameter text,
or in the replacement text of another macro.

\stopsectionlevel

\stopsectionlevel

\startsectionlevel[title={The~\type{\expandafter}~Macro},reference={the-expandafter-macro}]

Now that we've seen a few simple macros we turn to one that is slightly
less simple. The~\type{\expandafter}~macro first reads the very next
token in the queue without expanding it. Then, it'll read~{\em and
expand}~the next token after that. Last, it will put the first token
back in front, without expanding it. Here's a small example of how it
runs:

\starttyping
*\def\first{first}
*\def\second{second}
*\expandafter\first\second
{\expandafter}

\second ->SECOND

\first ->FIRST
{the letter F}
*
\stoptyping

Here the output shows that~\type{\second}~is expanded
before~\type{\first}, and that the first token that we process
is~\type{f}. Note that the second form is only~{\em expanded}~and not
actually processed, so the following does~{\bf not}~work:

\starttyping
*\expandafter\first\def\first{another first!}
\stoptyping

The second term, the~\type{\def}~will be expanded, but it will not
\quotation{run}, so when~\type{\expandafter}~later
expands~\type{\first}~it will still have the same value as before, for
instance not to be defined.

Due to how TeX expansion rules work, a macro doesn't have to have all of
it's arguments in place when you use it;
currying\high{\goto{5}[url(https://mht.wtf/post/tex/\#user-content-fn-curry)]}~is
in a sense possible. We can use~\type{\expandafter}~to use this fact if
the first token expands to a curried macro, and the first token in
the~{\em expansion}~of the second token is the argument we want to give
to the curried form.

Here's an example. Say we have a macro~\type{\twoarray}~that takes two
things and wraps them in square brackets divided by a comma, as well as
a macro~\type{\tuple}~that expands to two tokens~\type{4}~and~\type{5}.
If we want to have~\type{\twoarray}~wrap the two tokens
from~\type{\tuple}, it doesn't work out of the box:

\starttyping
*\def\twoarray#1#2{[ #1 , #2 ]}
*\def\tuple{4 5}
*\twoarray\tuple X  % X is just a placeholder for whatever's next; we don't want it.
[ 4 5 , X ]
% This does not work because `\twoarray` will read two tokens, `\tuple` and `X`

*\expandafter\twoarray\tuple X
[ 4 , 5 ] X
% This does work because `\tuple` is expanded before `\twoarray`, and so the token
% queue when we process `\twoarray` is  `4 5 X`
\stoptyping

\startsectionlevel[title={Chaining},reference={chaining}]

So what happens when we chain multiple~\type{\expandafter}s together?
Let's work it out with some notation: dashes under a line
means~\type{\expandafter}~is skipping that line, and it's expanding the
token above the hat~\type{^}. Primed~\type{a'}~letters means expanded.

\starttyping
*\expandafter a  b  c  d ...
%             -  ^
% token list: a  b' c  d
\stoptyping

With two~\type{\expandafter}s this becomes

\starttyping
*\expandafter \expandafter a  b  c  d ...
%             ------------ ^
% token list:  \expandafter a' b  c  d
*\expandafter  a' b  c  d ...
%              -  ^
% token list:  a' b' c  d
\stoptyping

It undid itself! The expansion order was~\type{a}~and then~\type{b}.
Let's try three expands in a row. Now we're getting somewhere, because
when expanding the second token that~\type{\expandafter}~finds, we might
end up reading~{\em additional}~tokens,~{\em if}~that token takes
arguments. In this case this token is~\type{\expandafter}, which does
indeed take two arguments!

\starttyping
*\expandafter \expandafter \expandafter a  b  c  d ...
%             ------------     ^^^
%                          [eat 2 arguments]
*             \expandafter       a  b'        c  d ...
% This is just the first example again.
% token list:  a  b'' c  d ...
\stoptyping

and we're again back to having the expansion order
of~\type{a}~and~\type{b}~flipped. Despite this though, they are not
identical, because~\type{expandafter}~does not expand a form until it
only expands to itself, but only once. We can think of regular expansion
as taking out the next token in the queue and if it is expandable we
push back the expansion onto the queue.

Let's get concrete. As a warm up, here is the easy case where the two
forms~{\em are}~identical, namely when expanding once is fully expanded.
The list of~\type{\A ->a}~beneath each input line is the evaluation
sequence such that the macro~\type{\A}~expands to the token~\type{a}.

\starttyping
*\def\A{a}\def\B{b}\def\C{c}

*\A\B\C
\A ->a   \B ->b   \C ->c
*\expandafter\A\B\C
\B ->b   \A ->a   \C ->c
*\expandafter\expandafter\A\B\C
\A ->a   \B ->b   \C ->c
*\expandafter\expandafter\expandafter\A\B\C
\B ->b   \A ->a   \C ->c
\stoptyping

Note that just like we said above, the first and third lines are the
same, and the second and fourth are the same.

Next we make it slightly more interesting by expanding macros which body
is another macro:

\starttyping
*\def\AA{\A}\def\BB{\B}\def\CC{\C}

*\AA\BB\CC
\AA ->\A   \A ->a     \BB ->\B    \B ->b   \CC ->\C   \C ->c
*\expandafter\AA\BB\CC
\BB ->\B   \AA ->\A   \A ->a      \B ->b   \CC ->\C   \C ->c
*\expandafter\expandafter\AA\BB\CC
\AA ->\A   \BB ->\B   \A ->a      \B ->b   \CC ->\C   \C ->c
*\expandafter\expandafter\expandafter\AA\BB\CC
\BB ->\B   \B ->b     \AA ->\A    \A ->a   \CC ->\C   \C ->c
\stoptyping

The four lines have all distinct orders on which macros are expanded
when, in contrast with the last example. With four~\type{expandafter}s
we are back to as if we had none.

What if we had~\type{\AAA}~and friends?

\type{\meaning\noexpand\foo}

\stopsectionlevel

\stopsectionlevel

\startsectionlevel[title={Start Actually
Expanding~\type{\newif}},reference={start-actually-expanding-newif}]

If you've made it this far, good job! I realize this is a fair amount of
prerequisites before getting to the point of the post.

Here's the definition of~\type{\newif}~again, but formatted a little
differently:

\starttyping
\outer\def\newif#1{
    \count@=\escapechar
    \escapechar=-1
    \expandafter\expandafter\expandafter \def\@if#1{true}{\let#1=\iftrue}%
    \expandafter\expandafter\expandafter \def\@if#1{false}{\let#1=\iffalse}%
    \@if#1{false} % the condition starts out false
    \escapechar=\count@
}
\def\@if#1#2{\csname\expandafter\if@\string#1#2\endcsname}
{
    \uccode`1=`i
    \uccode`2=`f
    \uppercase{\gdef\if@12{}}
} % `if` is required
\stoptyping

Let's do this in parts, starting with the bottom group, then the
middle~\type{\def}, and then move on to the actual~\type{\newif}. Note
that only the first form is the actual body of~\type{\newif}~and that
the bottom group and the~\type{\def}~in the middle is just part of the
one-time setup. We'll start with the bottom group.

\startsectionlevel[title={The Bottom
Group},reference={the-bottom-group}]

\starttyping
{
    \uccode`1=`i
    \uccode`2=`f
    \uppercase{\gdef\if@12{}}
} % `if` is required
\stoptyping

Recall from before that the~\type{\uccode}~macro sets the character code
of the uppercase version of a character, so we can for instance change
the uppercase of~\type{g}~to be~\type{H}~by
writing~\type{\uccode}g=\type{H}. In our snippet we are setting the
uppercase version of the numbers~\type{1}~and~\type{2}~to
be~\type{i}~and~\type{f}. Yes really. Also recall that the change is
local to the current group, so this change will be undone after the
third macro.

So we've changed the uppercase of~\type{1}~and~\type{2}, and next we're
uppercasing a~\type{gdef}~which name is~\type{if@12}.

Let's make this slightly easier by only having one character we
uppercase

\starttyping
*{\uccode`1=`M \uppercase{\gdef\bob1{bob}}}
*\bob
\bob M->BOB
\stoptyping

Notice that the name of the macro is just~\type{\bob},
not~\type{\bob1}~or~\type{\bobM}.

\startsectionlevel[title={A note about more advanced parameter
texts},reference={a-note-about-more-advanced-parameter-texts}]

TeX allows us to ensure that there are other tokens in the argument list
of a macro expansion, or that the arguments are delimited by certain
tokens. For instance consider the following:

\starttyping
*\def\commasep#1,#2{(#1, #2)}
*\message{\commasep 1 2 3 , 9 8 7}
(1 2 3 ,9) 8 7
\stoptyping

We see that the first argument was not in fact just the first token, but
all tokens up until we hit~\type{,}which we had after the~\type{#1}~in
the parameter text. The last argument however, was just the next token.

We can also do this:

\starttyping
*\def\mfirst m#1{(#1)}
*\message{\mfirst a a}
! Use of \mfirst doesn't match its definition.
<*> \mfirst a
              a
*\message{\mfirst m a}
(a)
\stoptyping

Here we've said that we need an~\type{m}~before we get the next token as
the first argument to the macro. If the next token is not an~\type{m},
like in the first attempt, we error. It is basically a very simple
version of pattern matching.

\stopsectionlevel

\startsectionlevel[title={Back to Bob},reference={back-to-bob}]

In our definition of~\type{\bob}~we have ensured that the parameter text
should end with an uppercase~\type{1}, which was~\type{M}. There is a
problem though:

\starttyping
*\bob M
! Use of \bob doesn't match its definition.
<*> \bob M

?
\stoptyping

The reason this doesn't work is that while the uppercase of~\type{1}~is
temporarily set to~\type{M}~and the macro really does expect to be
called as~\type{\bob M}, the~\type{M}~we send in now has the wrong
character code: it's a letter and not a number. We can temporarily
change this in a group, and it will work.

\starttyping
*{\catcode`M=12 \bob M}
{begin-group character {}
{entering simple group (level 1)}
{\catcode}
{changing \catcode77=11}
{into \catcode77=12}

\bob M->BOB
{the letter B}
{end-group character }}
{restoring \catcode77=11}
{leaving simple group (level 1)}
{blank space  }

*
\stoptyping

Now we are ready to understand the current snippet

\starttyping
{\uccode`1=`i \uccode`2=`f \uppercase{\gdef\if@12{}}} % `if` is required
\stoptyping

This will define a macro~\type{\if@}~that ensures that the first two
tokens after it is~\type{i}~and~\type{f}~with category code~\type{12}.
Also note that it will expand to nothing, but it will eat the matched
tokens in the parameter list. In other words:

\starttyping
*\def\eat h{H} \message{\eat hello}
Hello
\stoptyping

The~\type{h}~is eaten and replaced with the body of the macro,~\type{H},
and the rest of the tokens~\type{ello}~are just characters so nothing is
done to them, and the result is~\type{Hello}.

To summarize, we've now globally defined a macro~\type{if@}~which
ensures that when applied the next two tokens in the token list will be
two tokens with catcode 12 that is~\type{i}~and~\type{f}, and these
tokens will be taken out of the token list.

\stopsectionlevel

\stopsectionlevel

\startsectionlevel[title={The
Middle~\type{\def}},reference={the-middle-def}]

Moving on to this part:

\starttyping
\def\@if#1#2{\csname\expandafter\if@\string#1#2\endcsname}
\stoptyping

Let's peel the onion. We've got a~\type{csname}/\type{endcsname}~pair,
so the output of the function will be a control sequence name, which
will, unless already defined, be defined to expand to~\type{\relax}. The
name will be the result of~\type{\expandafter\if@\string#1#2}; the
arguments passed to~\type{\@if}~(the~\type{def}we're looking at) will
thus be sent to~\type{\if@}, but the first argument will be eaten
by~\type{\string}~first. We just learned that the only thing
that~\type{\if@}~does is to ensure that the first two tokens given
are~\type{i f}~of catcode 12. And it just so happen that the tokens that
we get from expanding~\type{\string}~are exactly of catcode 12!

Let's try to expand~\type"\@if{ifeven}{true}":

\starttyping
\@if{ifeven}{true}
\csname \expandafter\if@\string{i f e v e n}{t r u e}\endcsname
\csname \if@ i f e v e n {t r u e}\endcsname
\csname e v e n {t r u e}\endcsname
\csname e v e n t r u e\endcsname   % csname doesn't care about grouping
eventrue
\stoptyping

The result is a single control sequence token with the
name~\type{eventrue}. That's it! As long as the~\type{\string}~expansion
of the first argument starts with~\type{i f}~we will get a control
sequence token that is the concatenation of the two arguments.

\stopsectionlevel

\startsectionlevel[title={The
First~\type{\def}},reference={the-first-def}]

Phew, back at the top. Here it is, once more:

\starttyping
\outer\def\newif#1{
    \count@=\escapechar
    \escapechar=-1
    \expandafter\expandafter\expandafter \def\@if#1{true}{\let#1=\iftrue}%
    \expandafter\expandafter\expandafter \def\@if#1{false}{\let#1=\iffalse}%
    \@if#1{false} % the condition starts out false
    \escapechar=\count@
}
\stoptyping

We're almost there; it's just a matter of piecing together some of the
parts that we've already unravelled. First we can note that we are
temporarily setting~\type{\escapechar}~to be~\type{-1}~and then
restoring it at the end. There are two questions we can answer here: (1)
why do we set it, and (2) why can't we group it instead?

\startlists
\item
  We want the argument to~\type{\newif}~to be a control sequence,
  like~\type{\newif\ifred}, and we also need to check that the given
  control sequence starts with~\type{if}, which we do
  in~\type{\if@}~through the~\type{\string}~macro. If naively
  applied,~\type{\string\ifred}~would expand to~\type{\ i f r e d}, but
  we need it to be~\type{i f r e d}. By setting~\type{\escapechar=-1}~we
  make~\type{\string}~output nothing for~\type{\}, and we are good.
\item
  Had we used grouping the~\type{\def}s we have inside would be local to
  the group and effectively destroyed by the time we are done
  expanding~\type{\newif}. If we were to use~\type{\gdef}~then all
  defined macros with~\type{\newif}~would have to be global. This way we
  can have the user define~\type{\newif}s that are local to their
  groups.
\stoplists

That only leaves three lines in the macro body, and two of them are of
the same form. From earlier we remember that
three~\type{\expandafter}~would expand the second token in the token
list twice. Let's assume~\type{#1 = \ifred}. With the total form

\starttyping
\expandafter\expandafter\expandafter \def \@if \ifred {true} {\let \ifred = \iftrue}
\stoptyping

we would first expand~\type{\@if}, which will eat two
tokens,~\type{#1}~and~\type"{true}"~and be replaced with the body of the
macro, as seen above. Then we need a second expansion to expand
the~\type{csname}~pair, and this will expand to the control sequence
token~\type{redtrue}. This would be put back in the token queue,

\starttyping
\expandafter \def \csname \expandafter\if@\string\ifred{true}\endcsname{\let \ifred = \iftrue}
\def \redtrue{\let \ifred = \iftrue}
\stoptyping

and at the end we have a familiar form. The same happens with
the~\type{false}~variant. The next line is then ran:

\starttyping
\@if\ifred{false} % expand:
\csname \expandafter\if@\string\ifred{true}\endcsname  % eval the csname pair
\redfalse  % we just defined this macro
\let\ifred=\iffalse  % run this
\stoptyping

At last, we restore~\type{\escapechar}~to whatever it was initially.

\stopsectionlevel

\stopsectionlevel

\startsectionlevel[title={In Conclusion},reference={in-conclusion}]

Taking it all together, running~\type{\newif\ifred}~expands to this:

\starttyping
% In the preamble we have the forms
\def\@if#1#2{\csname\expandafter\if@\string#1#2\endcsname}
{\uccode`1=`i \uccode`2=`f \uppercase{\gdef\if@12{}}} % `if` is required

% The user writes
\newif\ifred
% .. which expands to
\count@=\escapechar
\escapechar=-1
\expandafter\expandafter\expandafter \def\@if\ifred{true}{\let\ifred=\iftrue}
\expandafter\expandafter\expandafter \def\@if\ifred{false}{\let\ifred=\iffalse}
\@if\ifred{false}
\escapechar=\count@
% ... which is basically the same as
\def\redtrue{\let\ifred=\iftrue}
\def\redfalse{\let\ifred=\iffalse}
\redfalse
\stoptyping

and that's it! So hey, we had to peel a few
onions\high{\goto{6}[url(https://mht.wtf/post/tex/\#user-content-fn-onion)]},
but in the end we managed to unravel the mystery and really understand
what's going on in~\type{\newif}; it turns out it's quite a lot, though
the main functionality seems that we don't have to write these three
lines every time we want to define a new conditional, but that only one
suffices.

If you want to know more \quotation{real} definition and edge cases,
check out~\goto{this
site}[url(https://www.tug.org/utilities/plain/cseq.html)]; I went back
and forth on that and in the TeXbook when writing this post, and having
a searchable index of basically the entire language is, well,
indispensable. Of course, if you don't know much about TeX from before I
can only assume that the reference will be hard to dig into.

Notes, comments, questions, and tomatoes can be sent to my~\goto{public
inbox}[url(mailto:https://lists.sr.ht/~mht/public-inbox)].

Hope you learned something, and thanks for reading.

\stopsectionlevel

\startsectionlevel[title={Footnotes},reference={footnotes}]

\startlists
\item
  I couldn't call C89 or TeX simple in good
  faith.~\goto{↩}[url(https://mht.wtf/post/tex/\#user-content-fnref-simple)]
\item
  I don't know what I'm talking about here; can you
  tell?~\goto{↩}[url(https://mht.wtf/post/tex/\#user-content-fnref-kube)]
\item
  btw I use
  arch~\goto{↩}[url(https://mht.wtf/post/tex/\#user-content-fnref-arch)]
\item
  This isn't really how it works, but for the purposes of this post we
  might as well pretend it
  is.~\goto{↩}[url(https://mht.wtf/post/tex/\#user-content-fnref-writechar)]
\item
  This example is more close to destructuring, but I didn't want to get
  in the weeds of constructing an example that looked more like
  currying. Here's a sketch: you can have a macro in the body of another
  macro~\type{\func #1 x y}~such that~\type{#1}~expands to another
  macro. If we~\type{\expandafter}~the~\type{#1}~here we might get
  something like~\type{\func u v w x y}~and so we've effectively
  constructed a
  function~\type{f(g) = h(g(), x, y)}.~\goto{↩}[url(https://mht.wtf/post/tex/\#user-content-fnref-curry)]
\item
  Something something crying when peeling an
  onion.~\goto{↩}[url(https://mht.wtf/post/tex/\#user-content-fnref-onion)]
\stoplists

\stopsectionlevel

\stopsectionlevel
\stop
\page
\setuplayout[default]