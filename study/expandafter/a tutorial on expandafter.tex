\usemodule[memos][
    themecolor=cyan,
    layout=moderate,
    hdrstyle=foemarginalt,
    fontstyle=ss,
    fontsize=10pt,
    lineheight=1.75\bodyfontsize, 
    ]
\definesectionlevels[article][
  {title},
  {subject},
  {subsubject},
  {subsubsubject},
  {subsubsubsubject},
  {subsubsubsubsubject}]
\starttext

%https://switwu.github.io/2024-01-06-expandafter/#more-theory-multiple-expandafters

\startsectionlevel[article][title={\letterbackslash{}expandafter},reference={expandafter}]

这是 Stephan v. Bechtolsheim 于 1988 年在 TUGboat
上发表的一篇关于 \type{\expandafter} 的\goto{文章}[url(https://tug.org/TUGboat/tb09-1/tb20bechtolsheim.pdf)],存档于此处。
另外,本文中存在一些技术性的问题,请读者留心观察,我会在最后一节指出。

\startsectionlevel[article][title={Introduction},reference={introduction}]

I have found from my own experience teaching TEX courses
that \type{\expandafter} is one of the instructions that people have
difficulty understanding. After starting with a little theory, we will
present a number of examples showing different applications of this
instruction. Later we will also deal with multiple \type{\expandafter}s.

This article is condensed from a draft of my book, \emph{Another Look
at TEX}. See the end of this article for more information about the
book.

\stopsectionlevel

\startsectionlevel[article][title={The theory behind it},reference={the-theory-behind-it}]

\type{\expandafter} is an instruction that reverses the order of
expansion. It is not a typesetting instruction, but an instruction that
influences the expansion of macros. The term \emph{expansion}means
the \emph{replacement} of the macro and its arguments, if there are any,
by the \emph{replacement text} of the macro. Assume you define a
macro \type{\xx} as follows: \type"\def\xx{ABC}"; then the replacement
text of \type{\xx} is \type{ABC}, and the \emph{expansion
of} \type{\xx} is \type{ABC}.

As a control sequence, \type{\expandafter} can be followed by a number
of argument tokens. Assuming that the tokens in the following list have
been defined previously

\type{\expandafter} $⟨tokene⟩$ $⟨token1⟩$ $⟨token2⟩$ \type{...} $⟨tokenn⟩$ \type{...}

then the following rules describe the execution of \type{\expandafter}:

\startitemize[n,packed][stopper=.]
\item
  $⟨tokene⟩$, the token immediately following \type{\expandafter}, is
  saved without expansion.
\item
  Now $⟨token1⟩$, which is the token after the saved $⟨tokene⟩$, is
  analyzed. The following cases can be distinguished:
  \startitemize[n,packed][stopper=.]
  \item
    $⟨token1⟩$ is a \emph{macro}: The macro $⟨token1⟩$ will be expanded. In
    other words, the macro and its arguments, if any, will be replaced
    by the replacement text. After this TEX will {\bf not} look at the
    first token of this new replacement text to expand it again or
    execute it. See 3 instead! Examples 1--6 and others below fall into
    this category.
  \item
    $⟨token1⟩$ is \emph{primitive}: Here is what we can say about this
    case: Normally a primitive can not be expanded so
    the \type{\expandafter} has no effect; see Example 7. But there are
    exceptions:
    \startitemize[n,packed][stopper=.]
    \item
      $⟨token1⟩$ is another \type{\expandafter}: See the section on
      \quotation{Multiple \type{\expandafter}s} later in this article,
      and also look at Example 9.
    \item
      $⟨token1⟩$ is \type{\csname}: TEX will look for
      matching \type{\endcsname}, and replace the text between
      the \type{\csname} and the \type{\endcsname} by the token
      resulting from this operation. See Example 11.
    \item
      $⟨token1⟩$ is an opening curly brace which leads to the opening
      curly brace temporarily being suspended. This is listed as a
      separate case because it has some interesting applications; see
      Example 8.
    \item
      $⟨token1⟩$ is \type{\the}: the \type{\the} operation is performed,
      which involves reading the token after \type{\the}.
    \stopitemize
  \stopitemize
\item
  $⟨tokene⟩$ is stuck back in front of the tokens generated in the
  previous step, and processing continues with $⟨tokene⟩$.
\stopitemize

\stopsectionlevel

\startsectionlevel[article][title={Example 1 and 2: Macros and \type{\expandafter}},
                   reference={example-1-and-2-macros-and-expandafter}]

In these examples, $⟨token1⟩$ is a macro. Assuming the following two macro
definitions (Example 1):

\startverbatim
\def\xx [#1]{...}
\def\yy{[ABC]}
\stopverbatim


We would like to call \type{\xx} with \type{\yy}'s replacement text as
its argument. This is not directly possible (\type{\xx\yy}), because
when TEX reads \type{\xx} it will try to find \type{\xx}'s argument {\bf
without} expansion. So \type{\yy} will {\bf not} be expanded, and
because TEX expects square brackets containing the argument
of \type{\xx} on the main token list, it will report an error stating
that \quotation{\type{the use of \xx doesn't match its definition}}.

On the other hand \type{\expandafter\xx\yy} will work. Now,
before \type{\xx} is expanded, the expansion of \type{\yy} will be
performed, and so \type{\xx} will find \type{[ABC]} on the main token
list as its argument.

Now assume the following additional macro definition is given (Example
2):

\startverbatim
\def\zz {\yy}
\stopverbatim


Observe that \type{\expandafter\xx\zz} will not work,
because \type{\zz} is replaced by its replacement text which
is \type{\yy}. But then \type{\yy} is not expanded any further.
Instead \type{\xx} will be substituted back in front of \type{\yy}. In
other words the expansion in an \type{\expandafter} case is {\bf
not} \quotation{pushed all the way}; to accomplish a complete expansion,
one should use \type{\edef}, where further expansion can be prevented
only with \type{\noexpand}. This example (using \type{\zz} as an
argument to \type{\xx}), which would not work with \type{\expandafter},
but work with \type{\edef}:

\startverbatim
% Equivalent to "\def\temp{\xx[ABC]}".
\edef\temp{\noexpand\xx\zz}
\temp
\stopverbatim


As a side remark: Example 1 can also be programmed {\em
without} \type{\expandafter}, by using \type{\edef}:

\startverbatim
% Equivalent to "\def\temp{\xx[ABC]}".
\edef\temp{\noexpand\xx\yy}
\temp
\stopverbatim


\stopsectionlevel

\startsectionlevel[article][title={Example 3},reference={example-3}]

In this example, $⟨token1⟩$ is a definition.

\startverbatim
\def\xx{\yy}
\expandafter\def\xx{This is fun}
\stopverbatim


\type{\expandafter} will temporarily suspend \type{\def}, which
causes \type{\xx} being replaced by its replacement text which
is \type{\yy}. This example is therefore equivalent to

\startverbatim
\def\yy{This is fun}
\stopverbatim


\stopsectionlevel

\startsectionlevel[article][title={Example 4 and 5: 
                   Using \type{\expandafter} to pick out arguments},
reference={example-4-and-5-using-expandafter-to-pick-out-arguments}]

Assume the following macro definitions \type{\Pick...} of two macros,
which both have two arguments and which print only either the first
argument or the second one. These macros can be used to pick out parts
of some text stored in another macro.

\startverbatim
% \PickFirstOfTwo
% This macro is called with two
% arguments, but only the first
% argument is echoed. The second
% one is dropped.
% #1: repeat this argument
% #2: drop this argument
\def\PickFirstOfTwo #1#2{#1}

% \PickSecondOfTwo
% ===============
% #1 and #2 of \PickFirstOfTwo
% are reversed in their role.
% #1: drop this argument
% #2: repeat this argument
\def\PickSecondOfTwo #1#2{#2}
\stopverbatim


Here is an application of these macros (Example 4 and 5) where one
string is extracted from a set of two strings.

\startverbatim
% Define macro \a. In practice, \a
% would most likely be defined
% by \read, or a mark.
\def\a{{First part}{Second part}}

% Example 4: Generates "First part"
% Pick out first part of \a.
\expandafter\PickFirstOfTwo\a

% Example 5: Generates "Second part".
% Pick out second part of \a.
\expandafter\PickSecondOfTwo\a
\stopverbatim


Let us analyze Example 4: \type{\PickFirstOfTwo} is saved because of
the \type{\expandafter} and \type{\a}is expanded
to \type"{First part}{Second part}". The two strings inside curly braces
generated this way form the arguments of \type{\PickFirstOfTwo}, which
is re-inserted in front of \type"{First part}{Second part}". Finally,
the macro call to \type{\PickFirstOfTwo} will be executed, leaving
only \type{First part} on the main token list.

Naturally the above \type{\Pick...} macros could be extended to pick
out x arguments from y arguments, where x≤y, to offer a theoretical
example.

\stopsectionlevel

\startsectionlevel[article][title={Example 6: \type{\expandafter} and \type{\read}},
                   reference={example-6-expandafter-and-read}]

The \type{\expandafter} can be used in connection with \type{\read},
which allows the user to read information from a file, typically line by
line. Assume that a file being read in by the user contains one number
per line. Then an instruction
like \type{\read\stream} to \type{\InLine} defines \type{\InLine} as the
next line from the input file. Assume, as an example, the following
input file:

\startverbatim
  12
  13
  14
\stopverbatim


Then the first execution of \type{\read\stream} to \type{\InLine} is
equivalent to \type"\def\InLine{12␣}", the second one
to \type"\def\InLine{13␣}", and so forth. The space ending the
replacement text of \type{\InLine} comes from the end-of-line character
in the input file.

This trailing space can be taken out by defining another
macro \type{\InLineNoSpace} with otherwise the same replacement text.
The space contained in the replacement text of \type{\InLine}matches the
space which forms the delimiter of the first parameter
of \type{\temp} in the following. Here, the macro \type{\readn} reads
one line from the input file and defines the
macro \type{\InLineNoSpace} as that line without the trailing space:

\startverbatim
\newread\stream
\def\readn{%
  \read\stream to \InLine
  % \temp is a macro with one
  % parameter, delimited by a space.
  \def\temp ##1␣{%
    \def\InLineNoSpace{##1}}%
  \expandafter\temp\InLine
}
\stopverbatim


\stopsectionlevel

\startsectionlevel[article][title={Example 7: Primitives and \type{\expandafter}},
                   reference={example-7-primitives-and-expandafter}]

Most primitives trigger no actions by TEX because in general, primitives
can not be expanded (a few primitives are treated differently). In this
sense, characters are primitives also. Let's look at an example:

\startverbatim
\expandafter AB
\stopverbatim


Character \type{A} is saved. Then TEX tries to expand, but {\em
not} print \type{B}, because \type{B} can not be expanded.
Finally, \type{A} is put back in front of the \type{B}; in other words,
the two characters are printed in the given order, and we may as well
have omitted the \type{\expandafter}. So what's the point
here? \type{\expandafter} reverses the order of \emph{expansion}, not of
execution.

\stopsectionlevel

\startsectionlevel[article][title={Example 8: \type{\expandafter} to 
                   temporarily suspend an opening curly brace},
reference={example-8-expandafter-to-temporarily-suspend-an-opening-curly-brace}]

\type{\expandafter} can be used to \emph{temporarily suspend an opening
curly brace}. In the following case this is done to load a token
register.

\startverbatim
% Allocate token registers \ta and \tb.
\newtoks\ta
\newtoks\tb

% (1) Initialize \ta to "\a\b\c".
\ta = {\a\b\c}
% (2) \tb now contains "\a\b\c".
\tb = \expandafter{\the\ta}
% (3) \tb now contains "\the\ta".
\tb = {\the\ta}
% (4) \tb now contains "\a\b\c"
\tb = \ta
\stopverbatim


In (1) we load the token register \type{\ta}. In (2)
the \type{\expandafter} temporarily causes the opening curly brace to be
hidden, so that \type{\the\ta} is evaluated, resulting in the generation
of a copy of \type{\ta}'s content. In (3) \type{\ta} is not copied
because when a token register is loaded, the new content of it, enclosed
within curly braces, is not expanded. Finally, (4) is equivalent to (2),
without using \type{\expandafter}.

\stopsectionlevel

\startsectionlevel[article][title={More theory: Multiple \type{\expandafter}s},
                   reference={more-theory-multiple-expandafters}]

Sometimes one needs to reverse the expansion, not of two, but of three
tokens. This can be done. Assume
that \type{\a}, \type{\b} and \type{\c} are macros without parameters
and \type{\ex}i stands for the ii-th \type{\expandafter}. Then

\type{\ex}\low{1} \type{\ex}\low{2} \type{\ex}\low{3} \type{\a} \type{\ex}\low{4} \type{\b} \type{\c}

reverses the expansion of all three tokens. Here is what happens:

\startitemize[n,packed][stopper=.]
\item
  \type{\ex}\low{1} is executed causing \type{\ex}\low{2} to be saved.
  Then TEX looks at \type{\ex}\low{3}.
\item
  \type{\ex}\low{3} is another \type{\expandafter}: {\bf TEX will continue
  performing \type{\expandafter}s!}
\item
  \type{\a} is saved (the list of saved tokens is
  now \type{\ex}\low{2} - \type{\a}).
\item
  \type{\ex}\low{4} is now executed: \type{\b} is saved (the saved token list
  is now \type{\ex}\low{2} - \type{\a} - \type{\b}), and \type{\c}is expanded.
\item
  Now \emph{all} saved tokens \type{\ex}\low{2} - \type{\a} - \type{\b} are
  inserted back in front of the replacement text of \type{\c}.
\item
  The first token on the main token list which resulted from the
  operation of the previous step is \type{\ex}\low{2},
  another \type{\expandafter}, which causes \type{\a} to be saved again,
  and \type{\b} to be replaced by its replacement text.
\item
  Finally \type{\a} is inserted back into the main token list and
  replaced by its replacement text. The execution continues with the
  first token of the replacement text of \type{\a}.
\stopitemize

One could summarize the net effect of this sequence
of \type{\expandafter}s and other tokens as follows (this is important
for Example 8): \emph{in front of the expansion of \type{\c} the
expansion of \type{\b} is inserted and in turn the expansion
of \type{\a} is inserted in front of that}.

The above example can be \quotation{expanded} to reverse the expansion
of any number of tokens. Very rarely the reversed expansion of four
tokens \type{\a}, \type{\b}, \type{\c} and \type{\d} is needed. Assume
that all four tokens are macros without parameters. Here is how this is
done:

\startverbatim
\let\ex = \expandafter
% 7 \expandafters
\ex\ex\ex\ex\ex\ex\ex\a
% 3 \expandafters
\ex\ex\ex\b
% 1 \expandafter
\ex\c
\d
\stopverbatim


In general, to reverse the expansion
of nn tokens $⟨token1⟩$ \ldots{} $⟨tokenn⟩$, the ii-th token has to be
preceded by $2^{n-i}-1$ \type{\expandafter}s.

\stopsectionlevel

\startsectionlevel[article][title={Example 8: 
                   Forcing the partial expansion of token lists of \type{\write}s},
reference={example-8-forcing-the-partial-expansion-of-token-lists-of-writes}]

\type{\expandafter} can be used to force the expansion of the first
token of a delayed \type{\write}. Remember that unless \type{\write} is
preceded by \type{\immediate}, the expansion of the token list of
a \type{\write} is delayed until \type{\write} operation is really
executed, as side effect of the \type{\shipout}instruction in the output
routine. So, when given the
instruction \type"\write\stream{\x\y\z}", TEX will try to
expand \type{\x}, \type{\y} and \type{\z} when the \type{\shipout} is
performed, not when this \type{\write}instruction is given.

There are applications where we have to expand the first token
(\type{\x} in our example) immediately, in other words, at the time
the \type{\write} instruction is given, {\bf not} when
the \type{\write} instruction is later actually performed as side effect
of \type{\shipout}. This can be done by:

\startverbatim
\def\ws{\write\stream}
\let\ex = \expandafter
\ex\ex\ex\ws\ex{\x\y\z}
\stopverbatim


Going back to our explanation of
multiple \type{\expandafter}s: \type{\ws} corresponds
to \type{\a}, \type"{" to \type{\b}, and \type{\x} to \type{\c}. In
other words \type{\x} will be expanded (!!), and \type"{" will be
inserted back in front of it (it cannot be expanded).
Finally, \type{\ws} will be expanded into \type{\write\stream}.
Now \type{\write} will be performed and the token list of
the \type{\write} will be saved without expansion. But observe
that \type{\x} was already expanded. \type{\y} and \type{\z}, on the
other hand, will be expanded when the
corresponding \type{\shipout} instruction is performed.

\stopsectionlevel

\startsectionlevel[article][title={Example 9: Extracting a substring},
                   reference={example-9-extracting-a-substring}]

Assume that a macro \type{\xx} (without parameters) expands to text
which contains the two tokens \type{\aaa} and \type{\bbb} embedded in it
somewhere. You would like to extract the tokens
between \type{\aaa} and \type{\bbb}. Here is how this could be done:

\startverbatim
% Define macro to extract substring
% from \xx.
\def\xx{This is fun\aaa TTXXTT
        \bbb That's it}
% Define macro \extract with three
% delimited parameters.
% Delimiters are \aaa, \bbb, and \Del.
% Macro prints substring contained
% between \aaa and \bbb.
\def\extract #1\aaa#2\bbb#3\Del{#2}
% Call macro to extract substring
% from \xx.
% Prints "TTXXTT".
\expandafter\extract\xx\Del
% which is equivalent to:
\extract This is fun\aaa TTXXTT
     \bbb That's it\Del
\stopverbatim


In a \quotation{real life example} \type{\xx} would be defined through
some other means like a \type{\read}. There is no reason to go to that
much trouble just to print \type{TTXXTT}.

\stopsectionlevel

\startsectionlevel[article][title={Example 10: Testing on the presence of a
substring\goto{}[url(https://switwu.github.io/2024-01-06-expandafter/\#example-10-testing-on-the-presence-of-a-substring)]},reference={example-10-testing-on-the-presence-of-a-substring}]

Now let us solve the following problem: We would like to test whether or
not a macro's replacement text contains a specific substring. In our
example, we will test for the presence of \type{abc} in \type{\xx}'s
replacement text. For that purpose we define a
macro \type{\@TestSubS} as follows: (\type{\@Del}is a delimiter):

\startverbatim
\def\@TestSubS #1abc#2\@Del{...}
\stopverbatim


Now look at the following source:

\startverbatim
\def\xx{AABBCC}
% #1 of \@TestSubS is AABBCC
\expandafter\@TestSubS\xx abc\@Del
\def\xx{AABBabcDD}
% #1 of \@TestSubS is AABB
\expandafter\@TestSubS\xx abc\@Del
\stopverbatim


Observe that

\startitemize[n,packed][stopper=.]
\item
  If \type{\xx} {\bf does not} contain the substring \type{abc} we are
  searching for, then \type{#1} of \type{\@TestSubS}becomes the same
  as \type{\xx}.
\item
  In case \type{\xx} {\bf does} contain the substring \type{abc},
  then \type{#1} of \type{\@TestSubS} decomes that part
  of \type{\xx} which occurs before the \type{abc} in \type{\xx}.
\stopitemize

We can now design \type{\IfSubString}. It is a simple extension of the
above idea, with a test added at the end to see whether or
not \type{#1} of \type{\@TestSubS} is the same as \type{\xx}.

\startverbatim
\catcode`@ = 11
% This conditional is needed because
% otherwise we would have to call the
% following macro \IfNotSubString.
\newif\if@TestSubString
% \IfSubString
% ============
% This macro evaluates to a conditional
% which is true iff #1's replacement
% text contains #2 as substring.
% #1: Some string
% #2: substring to test for whether it
%     is in #1 or not.
\def\IfSubString #1#2{%
    \edef\@MainString{#1}%
    \def\@TestSubS ##1#2##2\@Del{%
         \edef\@TestTemp{##1}}%
    \expandafter\@TestSubS
         \@MainString#2\@Del
    \ifx\@MainString\@TestTemp
         \@TestSubStringfalse
    \else
         \@TestSubStringtrue
    \fi
    \if@TestSubString
}
\catcode`@ = 12
\stopverbatim


\stopsectionlevel

\startsectionlevel[article][title={Example 11: \type{\expandafter} and \type{\csname}},
                   reference={example-11-expandafter-and-csname}]

A character string enclosed
between \type{\csname} and \type{\endcsname} expands to the token formed
by the character string. \type{\csname a?a-4\endcsname}, for instance,
forms the token \type{\a?a-4}. If you wanted to use this token in a
macro definition you have to do it the following way:

\startverbatim
\expandafter\def\csname a?a-4\endcsname{...}
\stopverbatim


The effect of the \type{\expandafter} is of course to
give \type{\csname} a chance to form the requested token rather than
defining a new macro called \type{\csname}.

\stopsectionlevel

\startsectionlevel[article][title={Summary},reference={summary}]

These examples have shown some typical applications
of \type{\expandafter}. Some were presented to \quotation{exercise your
brain a little bit}. I recommend that you take the examples and try them
out; there is very little input to enter. I also encourage you to tell
Barbara Beeton or me what you think about tutorials in TUGboat. There
are many more subjects which could be discussed and which may be of
interest to you.

This article is, as briefly mentioned in the introduction, an adaptation
of a section of my book, \emph{Another Look At TEX}, which I am currently
finishing. The book, now about 800 pages long, grew out of my teaching
and consulting experience. The main emphasis of the book is to give
concrete and useful examples in all areas of TEX. It contains, to give
just one example, 100 (!!) \type{\halign} tables. In this book you
should be able to find an answer to almost any TEX problem.

\stopsectionlevel

\startsectionlevel[article][title={注},reference={注}]

作者在文中提到的 Another Look At TEX 这本书应该又名为 TEX in
Practice,分四卷:

\startitemize[packed]
\item
  Volume I: Basics
\item
  Volume II: Paragraphs, Math and Fonts
\item
  Volume III: Tokens, Macros
\item
  Volume IV: Output Routines, Tables
\stopitemize

本文章所讲述的内容位于第三卷之中。下面指出文章中的问题:

\startitemize[n,packed][stopper=.]
\item
  在 第二节 中,如果 token1 为花括号的话,TEX 会试图展开 \type"{",然而任何 character token
  都是不可展开的,所以在这种情况下 \type{\expandafter} 没有任何作用。
  如果tokene 为 \type"{" 的话,TEX 会尝试展开花括号后面的 token (以及该 token 可能附带的参数),
  一个经典的例子是 \type"\uppercase\expandafter{\romannumeral3}",
  TEX 先将 \type"{" 后面的 \type{\romannumeral3} 展开为 \type{iii},
  然后执行 \type"\uppercase{iii}" 得到 \type{III}。
\item
  同样在第二节中,作者的叙述容易误导读者以为可以展开的 primitives 只有
   \type{\the}、\break\type{\csname} 和 \type{\expandafter},
  事实上,所有的 {\em conditionals} (包括 \type{\if} 系列、\type{\else}、\type{\fi})
  以及 \type{\number} 都是可展开的。
\stopitemize

\stopsectionlevel

\stopsectionlevel

\stoptext