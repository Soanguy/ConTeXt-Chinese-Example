\usemodule[memos][paperdesign=kindle,hdrstyle=fctext]

\definelayer[lay:title][x=0mm, y=0mm,width=\paperwidth, height=\paperheight]

\setlayer[lay:title][hoffset=0cm, voffset=0cm]{\externalfigure[imgs/小王子(艾柯譯)/Frontcover.jpg][height=\paperheight]}

\startsetups setups:title
   \setupbackgrounds[page][background=lay:title]
\stopsetups

\setupmakeup[page][before=\setups{setups:title},]
\startmakeup[page]
\stopmakeup

\startfrontmatter
\resetuserpagenumber
\starttitle[title={版权信息}]
书名:小王子

作者:【法】安东尼·德·圣艾修伯里

译者:艾柯
\stoptitle

\starttitle[title={献给------里昂·维德}]

我希望孩子们能够原谅我,把这本书献给一个大人。

首先,我有一个相当重要的理由:这个大人是我世界上最好的朋友;另一个理由是:他了解每件事,即使是有关孩子的书也一样。我还有第三个理由:这个大人住在法国,他既饥饿又寒冷,他真的需要一些鼓励和安慰。

如果我所说的这些理由都还不够好的话,就让我把这本书献给小时候的他吧。

所有的大人都曾经是小孩子,虽然,只有少数人记得。所以,我将我的献辞更改为:

献给小时候的里昂·维德

忘记朋友是一件悲哀的事,不是每个人都会有朋友的。

审判自己要比审判别人难得多了。

在那些爱慕虚荣的人眼里,别人都成了他们的崇拜者。

人群里也是很寂寞的。

真正重要的东西是肉眼无法看见的。

人们拼命挤进快速火车,却不知道自己在寻找些什么。
\stoptitle
\stopfrontmatter

\startbodymatter
\resetuserpagenumber
\starttitle[title={小王子}]
\stoptitle
\starttitle[title={1}]

我六岁的时候,在一本书上看到一幅扣人心弦的图画。那是一本描述原始森林的书,叫做《生命中的真实故事》,画的是一条蟒蛇正在吞食一只大野兽的情形。喏,就是下面的这幅图。

书上这样写道:“蟒蛇连嚼都不嚼,就把捉到的动物囫囵吞下去,肚子撑得无法动弹,直到它们睡上六个月的长觉,把所有吃进去的食物消化完为止。”

{\startalignment[center]
 \placefigeasy[][imgs/小王子(艾柯譯)/figure_0007_0001.jpg][maxwidth=\textwidth,maxheight=\textheight,location={middle,none}]{}
 \stopalignment}

从此,我对丛林的种种奇事产生了无穷无尽的遐想,于是,我拿起彩色铅笔,勾勒了一幅画------我的一号作品,就是下面的这个样子:

{\startalignment[center]
 \placefigeasy[][imgs/小王子(艾柯譯)/figure_0008_0002.jpg][maxwidth=\textwidth,maxheight=\textheight,location={middle,none}]{}
 \stopalignment}

我把这幅“杰作”拿给大人们看,还问他们有没有被这张画吓到。

他们却回答说:“一顶帽子有什么可怕的?”

我画的根本就不是帽子,而是一条蟒蛇正在消化它吃进去的大象呀!于是,我又画了一张蟒蛇内部的透视图,这样,大人们就可以看懂了。唉,大人最麻烦了,老是要我们给他们解释。我的二号作品是像下页上这样子的:

大人们看完后,劝我别再画这些蟒蛇图了,不论是外观的还是透视的,要我把心思好好地放在念地理、历史、算术和文法上。这也就是为什么我在六岁时,就放弃成为画家这个美好职业的原因。我对一号作品和二号作品的失败感到沮丧。这些大人从来就不主动去了解任何事情;而对我们小孩来说,老是要跟他们再三解释,实在是太累了。

既然如此,我只好放弃画画,选择了另一个行业,学习驾驶飞机去了。现在,我已经飞遍了大半个地球。的确,地理学帮了我大忙,我一眼就可以分辨出中国和美国亚利桑那州。如果夜航时迷了路,这样的本领很管用。

在这样的职业生涯里,我跟很多重要人物打过交道。我花了许多时间跟很多大人接触,也曾经很仔细地观察过他们,但是,我对他们的看法并没有多大改变。

无论何时,每当我遇到一个看起来头脑清楚的人,我就会拿出一直保存着的一号作品给他看。我的确想知道他是否有足够的理解能力。可是,不论男女,看过之后都说:“这是一顶帽子。”

{\startalignment[center]
 \placefigeasy[][imgs/小王子(艾柯譯)/figure_0010_0003.jpg][maxwidth=\textwidth,maxheight=\textheight,location={middle,none}]{}
 \stopalignment}

于是,我就不会再跟他说什么蟒蛇、原始雨林或星星了。我只会谈他们能够理解的事,譬如桥牌、高尔夫球、政治,还有领带等等。这样一来,这个人就会非常高兴,以为遇到了一个通情达理、善解人意的人。

\reference[Chapter1_2.xhtml]{}

\stoptitle

\starttitle[title={2}]

就这样,我一个人孤独地生活着,找不到一个真正可以交心谈心的人。直到六年前,我在撒哈拉沙漠发生了航行意外,才改变了一切。那次,我的飞机引擎出了点毛病,被迫在撒哈拉沙漠降落,当时既没有修理技师在旁边,也没有路人经过,我只能自己动手,试着修修看。可是我随身带的水只够饮用一星期了,能否支撑下去还是个问题。

第一天晚上我就睡在这远离人间烟火的大沙漠上,比大海中伏在小木排上的遇难者还要孤独得多。所以,你就可以想象,第二天早上,当我被那个奇怪又微弱的声音吵醒时,我是多么的惊讶。那个细小的声音说:

{\startalignment[center]
 \placefigeasy[][imgs/小王子(艾柯譯)/figure_0012_0004.jpg][maxwidth=\textwidth,maxheight=\textheight,location={middle,none}]{}
 \stopalignment}

“劳驾\ldots{}\ldots{}请给我画一只羊吧!”

“什么?”

“帮我画一只绵羊。”

我像是被雷电击中了一样,一跃而起。我使劲揉了揉眼睛,定睛一看,那是一个十分奇特的小家伙,正睁大眼睛注视着我。

这儿有一张他的画像,是我后来尽了最大努力画的。不过,他本人要比这张画像好看多了。然而这并不是我的错。早在六岁那年,我的绘画天赋就被那些大人们给毁了,除了大蟒蛇的外观图和透视图之外,我就没画过别的东西。

我目瞪口呆地站着,看着这个突然出现的、像幻影一样的小家伙。别忘了,我当时可是被迫降在方圆千里、人迹罕至的沙漠上!而眼前这个小人儿看起来既不像是在沙漠里迷了路,也不像疲惫、饥饿、口渴或害怕的人,况且从他的身上一点儿都看不出迷路孩子的迹象。当我在惊讶之余能说出话时,我问他:

“你在这儿做什么?”

他并没有回答我的问题,而是缓缓地重复着他刚才所说的话,仿佛在说一件很重要的事:“拜托!帮我画一只绵羊。”

当一个人被某种神秘力量震慑住时,他绝对不敢不服从。在荒无人烟的沙漠上,又面临死亡威胁的情况下,尽管这样的举动使我感到十分荒诞,我还是从口袋里掏出一张纸和一支笔。这个时候,我才想起,我把主要精力都花在地理、历史、算术和文法这些科目上了,于是我告诉这个小家伙(有些别扭地)我不知道该怎么画。他回答说:

“不要紧,帮我画一只绵羊。”

可是我从没画过绵羊呀!只画过那两张画。于是,我就给他画了其中的一张,即没有打开肚子的蟒蛇。“不是,不是,我不要这种蟒蛇把大象吃进去的图。”这小家伙所说的话让我目瞪口呆。他接着说:“蟒蛇这东西太危险,大象又太大了。我是从很小的地方来的,每样东西都很小。我要的是一只绵羊,帮我画一只绵羊。”

我只好照他的意思画了一只羊。

他仔细地看了我的画,然后说:“不好,这只绵羊太瘦弱,再重新画一张吧!”

{\startalignment[center]
 \placefigeasy[][imgs/小王子(艾柯譯)/figure_0016_0006.jpg][maxwidth=\textwidth,maxheight=\textheight,location={middle,none}]{}
 \stopalignment}

于是我又画了一张。

这次,他温和且腼腆地笑了,又说:

“你画的不是小羊,是一只公羊,你看,还有犄角呢!”

于是我再画了一张。

{\startalignment[center]
 \placefigeasy[][imgs/小王子(艾柯譯)/figure_0016_0007.jpg][maxwidth=\textwidth,maxheight=\textheight,location={middle,none}]{}
 \stopalignment}

但是这张画也和先前那几张的命运一样,再次遭到了否定。

“这只太老了!我要一只可以活得久一点的羊。”

终于,我开始沉不住气,我急着想拆开飞机的引擎。于是就随便画了这张图,丢下一句话:

{\startalignment[center]
 \placefigeasy[][imgs/小王子(艾柯譯)/figure_0017_0008.jpg][maxwidth=\textwidth,maxheight=\textheight,location={middle,none}]{}
 \stopalignment}

“这是装羊的箱子,你要的那只羊在里面。”

然而,我却很惊讶地看到我的小评论家的脸上,闪露出欣喜的光芒。

{\startalignment[center]
 \placefigeasy[][imgs/小王子(艾柯譯)/figure_0018_0009.jpg][maxwidth=\textwidth,maxheight=\textheight,location={middle,none}]{}
 \stopalignment}

“这正是我想要的!你说,这只羊需不需要喂很多草给它吃呢?”

“什么意思?”

“因为在我的家乡,每样事物都很小。”

“箱子里已经有足够的草了,”我说,“我帮你画了一只非常小的绵羊。”

他弯下头来看着画:

“没那么小!你看!它睡着了。”

就这样,我认识了小王子。

\reference[Chapter1_3.xhtml]{}

\stoptitle

\starttitle[title={3}]

我费了好长时间才弄清楚他是从哪里来的。虽然小王子问了我一大堆问题,可是,对于我提出的问题,他好像压根儿没有听见似的,我只能从他偶然间所说的事情中,一点一滴拼凑出他的身世来历。例如,当他第一次瞅见我的飞机时(我就不画我的飞机了,因为这种图画对我来说太复杂),就问我道:

“这是什么玩意儿呀?”

“这不是‘玩意儿',它能飞,是飞机,是我的飞机。”

我很骄傲地告诉他,我可是一个飞行员。没想到他马上大叫说:

“什么!你是从天上掉下来的?”

“是啊。”我谦虚地回答。

“哇!好神奇啊!”

小王子说着,迸发出一连串清脆悦耳的笑声,这使我有点不高兴。我希望他能够以严肃的态度,对待我的不幸。然后他又说:

“这么说,你也是从天上来的?!你住在哪一个星球?”

这句话犹如一道亮光,仿佛让我瞥见了他隐约中闪现出的秘密。我赶忙问他:“你是从另一个星球来的?”

他没有回答,只是盯着我的飞机,轻轻地摇着头:“从你乘的这玩意儿来看,不可能来自太远的地方,对不对?”

然后他就陷入了沉思中。过了一会儿,他把我画的羊从口袋里掏出来,像宝贝似的,从头到尾看了一遍。

相信你们都不难想象,听了他那句无意中透露的“其他星球”的话后,我会是怎样的惊讶吧!我竭力从中探听他的来历:“我的小人儿,你从哪儿来?你的家在哪儿?你要把你的绵羊牵到哪儿去?”

他黯然不答,然后才说:“你给了我一个箱子,太好了。夜里可以给绵羊做屋子了。”

“那当然。如果你听话的话,我再给你画一根绳子,白天可以拴住它。喔,对了!还要有一根柱子才可以。”

小王子被这个说法吓了一大跳:

“把它拴起来!多么奇怪的想法!”

“要是你不拴住它的话,它就会到处跑,会走丢的。”

小人儿又发出了清脆的笑声:

“你说它会跑到哪儿去呢?”

“哪儿都可能啊,到处乱跑。”

谁知道,小王子郑重其事地说:

“没关系,我来的地方每样东西都好小!”

接着,他又略带伤感地说:

“一直朝前走,也走不了多远。”

\reference[Chapter1_4.xhtml]{}

\stoptitle

\starttitle[title={4}]

就这样,我有了第二项重大发现:小王子住的那个星球不会比一栋房子大多少!

这倒并没有使我感到太奇怪。我早知道,除了那些已经被命名的大行星,像地球、木星、火星、金星之外,还有数以百计的行星存在,有些星球甚至小到用望远镜都很难看到。天文学家发现这种小行星的时候,就给它编个号码,当做它的名字。例如“小行星325号”。

我有很充分的理由相信,小王子所来自的那个星球是小行星B612。这颗小行星只在1909年被一位土耳其的天文学家透过望远镜看到过一次。

{\startalignment[center]
 \placefigeasy[][imgs/小王子(艾柯譯)/figure_0023_0010.jpg][maxwidth=\textwidth,maxheight=\textheight,location={middle,none}]{}
 \stopalignment}

那位土耳其学者曾将自己的发现向国际天文协会做了报告,以充分的证据来证明自己的观测结果。但没有人愿意相信一个穿着土耳其服装的天文学家的话。

{\startalignment[center]
 \placefigeasy[][imgs/小王子(艾柯譯)/figure_0024_0011.jpg][maxwidth=\textwidth,maxheight=\textheight,location={middle,none}]{}
 \stopalignment}

大人们都是这个样子,以衣冠取人。

值得庆幸的是,为了维护小行星B612的声誉,土耳其的独裁者下令他的人民都要改穿欧洲式服装,违抗命令的人可得砍头哪!因此,1920年,这位天文学家穿着光鲜亮丽的西服,为他的发现再一次在国际会议上发表演说。这一回,大家都相信了他所说的话。

我之所以不厌其烦地叙述有关这颗小行星的事,还一再提到B612这个编号,无非是想迎合大人们喜欢数字的习性。当你告诉大人们你新交了一位朋友时,他们从不会向你提出一些实质性问题,绝对不会问:“他的声音好不好听?他最喜欢的游戏是什么?有没有收集蝴蝶?”他们只会问:“他几岁?有几个兄弟姊妹?体重多少?他父亲赚多少钱?”他们以为只有知道了这些数字才算是了解别人。

{\startalignment[center]
 \placefigeasy[][imgs/小王子(艾柯譯)/figure_0025_0012.jpg][maxwidth=\textwidth,maxheight=\textheight,location={middle,none}]{}
 \stopalignment}

如果你对大人们说:“有一次,我看到一栋漂亮的、粉红色的砖砌房子,窗户上缀满了天竺葵,鸽子在屋顶上栖息。”他们绝对无法想象那栋房子到底是什么样子。你必须对他们说:“我看到一栋价值十万法郎的房子!”那么,他们就会惊叹:“哇!多美的房子呀!”

就是这样。要是你告诉他们:“确实有个小王子存在着。他面带笑容,非常可爱,想要一只羊;一个人要一只羊,就是他存在的证明。”他们会不以为意,然后耸耸肩,把你当成一个不懂事的孩子!但是,如果你跟他们说:“他来自小行星B612。”他们就会相信,而且不会再拿一些杂七杂八的问题来烦你。

大人们就是这个样子。小孩子除了忍耐,对此毫无办法。

当然,懂得生活的人才不会在乎那些编号呢!所以,我想用童话故事的方式,开始叙述这个故事。我会这样说:

“很久以前,有个小王子,住在一个比他自己大不了多少的小行星上;而且,他希望有一个朋友。”对于那些真正懂得生活的人来说,这样说本身就很真实。

我不喜欢别人漫不经心地读我的书。提起这些往事,我还挺伤心的。自从我的朋友带着他的小羊离开,到现在已经六年了。我尝试着在这里写下关于他的故事,是因为我永远不会忘记他。忘记朋友是一件悲哀的事,不是每个人都会有朋友的。如果我把他忘了,那我就可能会变得跟那些除了数字之外,对什么事都漠不关心的大人一样。为了这个缘故,我买了几支铅笔和一盒颜料。以我现在的年纪来说,开始学习画图是很困难的,尤其是在自从六岁那年画过大蟒蛇以后,再没有拿过画笔的情况下。当然,我会尽自己所能画得更逼真些,但我自己也没有把握。也许有一张画得还过得去,另一张就不一定行了。还有身材大小,我也记得不太真切,有的图画得太高,有的又画得太矮了。对他衣服的颜色我也拿不准,仅能凭记忆尽可能地东拼西凑。说不定可能会在某些重要细节上出错,对这一点,请你们一定要谅解。我的这个朋友从来不对我解释任何事,也许他以为我像他一样。但是,很不幸,我根本“不能”透过盒子看到里面的羊。也许,我有点像大人,我大概是变老了。

\reference[Chapter1_5.xhtml]{}

\stoptitle

\starttitle[title={5}]

日子就这样一天天地过去,通过和小王子的交谈,我渐渐知道了一些关于他所在的小行星的事情以及他的出走、他的旅行。所有这一切都是他平时不注意时流露出来的。就这样,到了第三天,我就了解到关于猴面包树的事情。

这一回还真亏了小羊。因为小王子对一件事产生了极大的怀疑,突然问我:“羊吃灌木,这是真的吗?”

“对啊!”

“哦!那很好!”

我不明白羊吃小灌木这件事为什么如此重要。可小王子又说道:

{\startalignment[center]
 \placefigeasy[][imgs/小王子(艾柯譯)/figure_0029_0013.jpg][maxwidth=\textwidth,maxheight=\textheight,location={middle,none}]{}
 \stopalignment}

“这么说,它们也吃猴面包树喽?”

我向小王子解释说,猴面包树可不是小灌木,而是像教堂那么大的大树,即便是带回一群大象,也啃不了一棵猴面包树。

一群大象啃猴面包树这种想法使小王子发笑:

“那可得把这些大象一只叠一只地垒起来。”

但是他又很聪明地继续说:

“猴面包树在长大之前,开始也是小小的。”

“不错。可是为什么你想叫你的羊去吃小猴面包树呢?”

他说:“嘿!那还用说吗!”似乎这是再明白不过的事情。而要我自己想明白这个问题,确实要绞尽脑汁了。

{\startalignment[center]
 \placefigeasy[][imgs/小王子(艾柯譯)/figure_0030_0014.jpg][maxwidth=\textwidth,maxheight=\textheight,location={middle,none}]{}
 \stopalignment}

原来,在小王子的星球上就像其他所有星球一样,长着有用的植物和不好的植物,所以也就有好的种子跟坏的种子。然而,种子是看不到的,它们埋藏在土地深处,直至其中一颗会突然苏醒,把头伸出地面。刚开始时,这颗小种子会羞怯地伸伸懒腰;然后,毫无威胁性地,朝太阳伸出一株青嫩可爱、娇小玲珑、不伤人的幼苗来。如果长出来的是萝卜或玫瑰的嫩芽,就可以让它们自由生长。可是,如果是一棵坏苗,一经认出,就必须立刻把它拔掉。

而在小王子的星球上,就有一些可怕的植物种子------猴面包树种子。它们密布在星球的土壤里,而且一旦它们占有地盘,就不可能把它们彻底清除掉,猴面包树会将根钻入泥土里,覆盖整个星球。况且,如果星球太小而猴面包树太多的话,就会把星球给挤爆了\ldots{}\ldots{}

“这是个纪律问题。”小王子后来向我解释道,“一个人每天早上盥洗结束后,就得认真打扫自己的星球,必须按规定及时拔掉猴面包树苗。这种树苗小的时候与玫瑰苗差不多,一旦可以把它们区别开来,就要立刻把它拔掉。这种工作虽然很无聊,却相当简单。”

有一天,他建议我尽力尝试画一张漂亮的图画,给地球上的孩子们看,好让他们了解这一切。“如果有一天他们去旅行的话,可能会有用。”

他又说:“有时候,把工作拖到以后再做也没关系,但是,如果遇到猴面包树的话,就非造成大灾难不可。我知道曾经在一个行星上,住着一个懒骨头,他忘了拔那三株猴面包树苗,结果\ldots{}\ldots{}”

{\startalignment[center]
 \placefigeasy[][imgs/小王子(艾柯譯)/figure_0032_0015.jpg][maxwidth=\textwidth,maxheight=\textheight,location={middle,none}]{}
 \stopalignment}

于是,根据小王子的说明,我把这个星球画了下来。我素来不喜欢装腔作势,摆出一副道学士的面孔教训人,但许多人对猴面包树的危害性认识不足,这对一个要漫游小行星的人来说,是多么大的风险啊。因此这一回,我打破了自己不喜欢教训人的惯例,对大家说:“孩子们,要当心那些猴面包树呀!”

为了向朋友发出警告,让他们提防早已威胁到我们的危险------而他们和我一样,对这样的危险一无所知------我对这幅画下足了工夫。只要能使大伙儿对此有所警觉,我即使费再大的心思去做这件事也是值得的。现在,你可以了解为什么在这本书里,找不出任何一张像猴面包树这样令人印象深刻的画了。别的画我也曾经试图画得好些,却没成功。而当我画猴面包树时,有一种急切的心情在激励着我。

\reference[Chapter1_6.xhtml]{}

\stoptitle

\starttitle[title={6}]

啊!我的小王子\ldots{}\ldots{}就这样,一点一滴地,我逐渐懂得了你那忧郁的小生命。长久以来,你唯一的乐趣只是欣赏落日。这是我在第四天早晨知道的,当你说出:

“我喜欢看夕阳。我们一起去看太阳下山吧\ldots{}\ldots{}”

“可是,我们必须要等\ldots{}\ldots{}”

“等什么?”

“等太阳落山哪!”

起初,你看起来好像很惊讶,然后,又自我解嘲地说:

“我总以为自己还在家里。”

确实,大家都知道,美国的正午时分,正是法国夕阳西下的时候。如果能在一分钟内赶到法国,你就可以看到落日了,可惜法国太远了。但是,在你的小行星上,只要把椅子向后挪几步,就可以随时随地看到落日的余晖了。

“有一天,我看了四十三次落日!”

过了一会儿,你又说:

“你知道------当人感觉忧伤时,就会喜欢看落日\ldots{}\ldots{}”

“你那时很忧伤吗?就是你看了四十三次落日的那天?”

小王子没有回答。

{\startalignment[center]
 \placefigeasy[][imgs/小王子(艾柯譯)/figure_0035_0016.jpg][maxwidth=\textwidth,maxheight=\textheight,location={middle,none}]{}
 \stopalignment}

\reference[Chapter1_7.xhtml]{}

\stoptitle

\starttitle[title={7}]

第五天,我发现了小王子身世的另一个秘密------再次感谢那只羊。好像默默地思索了很长时间以后,得出了什么结果一样,他突然没头没脑地问我:

“羊会吃花儿吗?就像吃灌木丛一样?”

“它碰到什么吃什么。”

“连有刺的花都吃吗?”

“有刺的花儿也吃。”

“那刺还有什么用呢?”

我不知道该怎么回答。那时候,我正忙着将一个卡在引擎上的螺丝拆下来。我发现,飞机损坏的情形很严重,而且,更让我担心的是,饮用水已经所剩不多了。

“那刺还有什么作用呢?”

小王子一旦提出了问题,就绝不放弃;而我正为了螺丝生气,于是不假思索地回答他:

“那些刺儿毫无用处,花儿长刺只能害人!”

“噢!”

沉默了一会儿,他悻悻地说:“我不信你说的话!花儿弱不禁风,花儿天真无邪,她们自顾不暇呢。她们身上长了刺,是为了给自己壮胆,为了保护自己\ldots{}\ldots{}”

我没有答话,当时我在想:“如果螺丝还不松动的话,我就一锤子敲碎它。”

小王子的话再次打断了我的思路:

“你真的认为花儿\ldots{}\ldots{}”

“算了吧,算了吧!我什么也不认为!我只是随便说说。你没看到我正在忙着要紧的事吗?”

他瞪着我,愣住了。

“要紧的事!?”

他看着我,我蹲在那个在他看来丑得要命的东西前面,手握着锤子,手指上沾满了油污\ldots{}\ldots{}

“你跟那些大人没什么两样!”

听了这话,我觉得有点惭愧。然而,他又毫不留情地说:

“你什么都分不清,你把什么都混在一起!”

他生气地摇晃着脑袋,金黄色的头发随风飘动着。

“我知道有一个星球上,住着一位红脸绅士。他从没闻过花香,也没看过星星,更没爱过别人。他除了算数以外,就没做过别的事,他跟你一样,整天不停地唠叨:‘我很认真!真的,我在做要紧的事!'而且,自以为是,神气活现的!他根本就算不上是一个人,只能算是一个------蘑菇!”

“一个什么?”

“一个蘑菇!”

小王子气得脸色发白。

“几百万年来,花儿生来就有刺,就像几百万年来羊都在吃花儿一样。难道了解花儿的身上为什么会有这些没用的刺不重要吗?难道羊和花之间的战争不重要吗?这些事难道不比那个臃肿的红脸绅士的数字更重要吗?如果我知道一朵花儿------人世间唯一的花儿,只长在我的小行星上,别的地方都不存在,在一天早晨,被一只小羊糊里糊涂地毁掉了,难道这样的事也不重要吗?”

他脸色渐渐转红,然后又接着说:

“如果有人钟爱着一朵独一无二、盛开在浩瀚星海里的花儿,那么,当他抬头仰望繁星时,便会心满意足。他会告诉自己:‘我心爱的花儿在那里,在那颗遥远的星星上。'可是,如果羊把花儿吃掉了,那么,对他来说,所有的星光便会在刹那间黯淡无光!而你却认为这不重要!”

他突然泣不成声,无法再说下去了。

夜幕降临,黑暗翩然而至。我放下手中的工具,把锤子、螺丝以及饥饿和死亡全抛在脑后,一切对我来说都已不再重要。在地球上,在我的行星上,有一位需要安慰的小王子。我将他抱在怀里,轻轻地摇着他,对他说:“你心爱的那朵花儿不会有危险的,我给你的小羊画个口罩;我给你的花画个护栏\ldots{}\ldots{}我\ldots{}\ldots{}”

我不知道该对他说些什么,只觉得自己很笨拙,不懂得该怎样抚慰他,不知道该如何才能再次回到与他心灵相通的地方。眼泪就是这么奇妙的东西。

{\startalignment[center]
 \placefigeasy[][imgs/小王子(艾柯譯)/figure_0040_0017.jpg][maxwidth=\textwidth,maxheight=\textheight,location={middle,none}]{}
 \stopalignment}

\reference[Chapter1_8.xhtml]{}

\stoptitle

\starttitle[title={8}]

很快地,我对花儿有了更多的了解。在小王子生活的行星上,长着很朴素的单瓣花儿,这些花儿非常小,一点儿也不占地方,也不扰人,在草地上朝开暮落。然而有一天,不知从哪里吹来一颗种子,小王子非常仔细地观察它,因为这株花芽跟他以前看过的嫩芽都不一样,这可能是新品种的猴面包树。

没多久,这一小株植物就停止生长,开始准备开花了。看到这棵苗上长出了一个很大很大的花蕾,小王子感觉到这个花蕾中一定会出现一个奇迹。然而,这朵花儿却仍旧在她绿莹莹的屋子里精心打扮。她谨慎地选择颜色,并且逐一地调整花瓣的角度,她不想像罂粟花儿那样,冒出来就皱皱的。她要让自己带着光艳夺目的丽姿来到世间。

噢!是的!她妩媚极了!总而言之,她花了好几天时间做准备。然后,在一个阳光明媚的早晨,她突然掀开了面纱,悄悄地绽放了。

她已经煞费苦心地打扮了这么长时间,却打着哈欠说:

“啊\ldots{}\ldots{}嗯\ldots{}\ldots{}我还没睡醒哪!我的花瓣儿还是乱糟糟的。”

但是小王子依然忍不住赞叹说:“您真美啊!”

“可不是嘛?”花朵回答,“我是在太阳升起的时刻诞生的呀!”

小王子看出了这花儿不太谦虚,可是她确实风姿绰韵,千娇百媚。

“我想现在该是早餐时间了。”她随后说,“劳驾,给我来些水好吗?”

小王子为自己的疏忽感到非常惭愧,于是走出去,打来了一壶清凉的水,浇灌着花儿。

不久,她爱慕虚荣的性情开始折磨他。譬如,有一天,她跟小王子讲起身上长的四根刺:

{\startalignment[center]
 \placefigeasy[][imgs/小王子(艾柯譯)/figure_0044_0019.jpg][maxwidth=\textwidth,maxheight=\textheight,location={middle,none}]{}
 \stopalignment}

“老虎要来张牙舞爪就让它们来吧,我可一点儿也不怕它们。”

“我的行星上没有半只老虎,”小王子不以为然地说,“再说,老虎也不喜欢吃草。”

“我才不是草呢!”花儿柔声柔气地说。

“对不起\ldots{}\ldots{}”

“我不怕老虎,不过,我可没办法忍受风。你有没有屏风?”

{\startalignment[center]
 \placefigeasy[][imgs/小王子(艾柯譯)/figure_0045_0020.jpg][maxwidth=\textwidth,maxheight=\textheight,location={middle,none}]{}
 \stopalignment}

“不能忍受风,对于一株植物来说还真是不幸。”小王子心里想着,“这朵花儿可真不是等闲之辈。”

“晚上,我想请你为我准备一个玻璃罩。你这儿可真冷,一点儿都不干净,我从前住的地方\ldots{}\ldots{}”

她没有说下去。她来的时候只不过是一粒种子,哪里见过什么别的世界。她发现自己说漏了嘴,马上假装咳嗽,以便转移小王子的注意力。她说道:

“屏风的事怎么样了?”

“我正要去找屏风,是您和我说话的呀!”

于是,她干脆再多咳几下,存心让他内疚不安。

小王子怜香惜玉,但她的做作令他生疑。他是个做事一丝不苟的人,因此常把无关紧要的闲话当真,难免会招来不少麻烦。

“我真不该听她的话,”有一天他告诉我,“人不应该听花儿说些什么的,只要观赏她们,闻闻花香就够了。那朵花儿使整个行星芳香四溢,可我不会享受它。关于老虎爪子的事,本应该使我产生同情,却反而惹我恼火。”

{\startalignment[center]
 \placefigeasy[][imgs/小王子(艾柯譯)/figure_0046_0021.jpg][maxwidth=\textwidth,maxheight=\textheight,location={middle,none}]{}
 \stopalignment}

他对我倾吐着自己内心的秘密。

“我那时什么也不懂!我应该根据她的行为,而不是根据她的话来判断她。她香气四溢,让我的生活更加芬芳多彩,我真不该离开她的\ldots{}\ldots{}我早该猜到,在她那可笑的伎俩后面是缱绻柔情啊。花朵是如此地天真无邪!可是,我毕竟是太年轻了,不知该如何去爱她。”

{\startalignment[center]
 \placefigeasy[][imgs/小王子(艾柯譯)/figure_0047_0022.jpg][maxwidth=\textwidth,maxheight=\textheight,location={middle,none}]{}
 \stopalignment}

\reference[Chapter1_9.xhtml]{}

\stoptitle

\starttitle[title={9}]

我想小王子大概是利用一群候鸟迁徙的机会跑出来的。在他出发的那天早上,他把他的星球收拾得整整齐齐,把它上头的活火山打扫得干干净净------他有两座活火山,做早餐很方便;他还有一座死火山,但是,就像小王子说的:“谁也不知道会发生什么事!”为了预防万一,他把死火山也给清理了。火山爆发就和烟囱着火一样,打扫干净了,它们就可以慢慢地有规律地燃烧,而不会突然爆发。当然,在地球上人太小,不能打扫火山,所以火山给我们带来很多很多麻烦。

小王子闷闷不乐地拔除了最后几棵猴面包树苗,他知道自己不会再回来了。那天早晨,他觉得所有的家务活干起来都特别亲切。当最后一次给花儿浇水,准备给她盖上罩子时,他觉得自己的眼泪都要掉下来了。

{\startalignment[center]
 \placefigeasy[][imgs/小王子(艾柯譯)/figure_0049_0023.jpg][maxwidth=\textwidth,maxheight=\textheight,location={middle,none}]{}
 \stopalignment}

{\startalignment[center]
 \placefigeasy[][imgs/小王子(艾柯譯)/figure_0050_0024.jpg][maxwidth=\textwidth,maxheight=\textheight,location={middle,none}]{}
 \stopalignment}

“再见啦!”他对花儿说。

可花儿并没有回答他。

“再见!”他再次道别。

花儿咳了一声,但她并没有感冒。

“我一直都很蠢,”她终于说话了,“我觉得很抱歉,希望你能快乐幸福。”

他觉得很惊讶,花儿竟然没有责怪他。他困惑地拿着玻璃罩站在那儿,无法理解她这种温柔恬静。

“我是爱你的,真的!”花儿告诉他,“你不知道,是我的错\ldots{}\ldots{}那不重要,可是你------你一直都跟我一样蠢。快乐起来吧\ldots{}\ldots{}把玻璃罩放到旁边去,我不再需要它了。”

“要是风来了怎么办?”

“我并非如此地弱不禁风\ldots{}\ldots{}夜晚的凉风对我倒是有一些好处呢。我是一朵花儿啊。”

“要是有虫子和野兽呢?”

“哦,如果我想跟蝴蝶交朋友的话,当然就得忍耐两三只毛毛虫的拜访。我听说蝴蝶长得很漂亮。况且,如果没有蝴蝶,没有毛毛虫,还会有谁来看我呢?你离我那么远\ldots{}\ldots{}至于大动物,我才不怕呢,我有我的利爪啊。”

她天真地伸出身上的四根刺,然后说:

“不要这样呆呆地站着啦,这样子真让我受不了!如果你要走,就赶快走吧!”

她不愿让小王子看见她哭。她是一朵很骄傲的花呢!

\reference[Chapter1_10.xhtml]{}

\stoptitle

\starttitle[title={10}]

小王子发现自己身处小行星区------小星球编号325、326、327、328、329、330,他开始一个一个拜访它们,一方面让自己有事情做,一方面还可以增长见识。

第一个行星上只住了一个国王。这位国王身着代表皇家身份的紫色貂皮长袍,坐在一个看起来样式简单却透着威严的宝座上。

“啊,来了一个臣民。”国王看到小王子便喊叫起来。

“他从没见过我,怎么会认得我呢?”小王子想。

他哪里知道,在那些国王的眼里,世界是非常简单的:所有的人都是臣民。

{\startalignment[center]
 \placefigeasy[][imgs/小王子(艾柯譯)/figure_0053_0025.jpg][maxwidth=\textwidth,maxheight=\textheight,location={middle,none}]{}
 \stopalignment}

“靠近一点,让我好好看看你。”国王说。他觉得自己终于成为某个人的国王了,所以相当得意。

小王子四处张望着,想找地方坐下,然而,整个星球都被国王那件巨大的貂皮长袍给盖住了。所以,他只好继续挺直站着,但他实在太疲倦了,忍不住打了个哈欠。

“在国王面前打哈欠是相当失礼的。”国王说,“我禁止你打哈欠。”

小王子抱歉地说:“我不是有意的。我实在忍不住了,我长途跋涉,没有睡觉。”

“既然如此,”国王对他说,“我命令你打哈欠。我已经好久好久没看过人打哈欠了。打哈欠可是很少见的。过来!再打个哈欠!这是命令。”

“这太可怕了\ldots{}\ldots{}我现在没法再打了\ldots{}\ldots{}”小王子心慌意乱,说话都结结巴巴的。

“嗯!嗯!”国王又说了,“那么,我就命令你一会儿打哈欠,一会儿不打哈欠。”

他嘟嘟囔囔,显得有点儿恼怒。

身为国王,最关心的是别人对他权威的尊重,他不允许别人违抗他的命令。他是一个典型的专制君王,但同时也是个善良之辈,所以下达的命令都合乎情理。

“如果我命令一个将军变成一只海鸟,”他常常说,“但是他不服从我的命令,那并不是将军的错,而是我的过错。”

“我可以坐下吗?”小王子怯怯地问。

“我命令你坐下。”国王一边回答,一边把他威风的貂皮斗篷推到一边。

然而,小王子觉得很疑惑:这个行星这么小,国王能统治什么?

“陛下,”小王子说,“对不起,我想问个问题,但是\ldots{}\ldots{}”

“我命令你问。”国王急忙抢着说。

“陛下您统治什么呢?”

“统治一切。”国王非常简单明了地回答。

“一切?”

国王做了个手势,把他的行星、其他行星及所有星星全都概括起来。

“所有这一切?”小王子问道。

“一切。”国王肯定地回答。

原来他不仅是一个星球的君主,而且是整个宇宙的君主。

“那么,星星都服从您吗?”

“那当然!”国王对他说,“它们立即就得服从。我是不允许无纪律的。”

这样的力量使小王子惊叹不已。如果他也掌握这样的权力,那么一天之内他就可以观赏到不止四十三次,而是七十二次,甚至一百次,或两百次日落了,而且还不用挪动椅子。想到他那颗被遗弃的小行星,小王子忽然伤感起来。于是,他鼓足勇气大胆地向国王提出了一个请求。

“我想看日落,请求您命令太阳落山吧。”

“如果我命令一个将军像蝴蝶一样从一朵花儿飞到另一朵花儿,或是叫他写出悲剧,或是叫他变成一只海鸟,而将军却不遵照命令的话,是他不对还是我不对?”

“是您错了。”小王子不假思索便脱口而出。

“一点也不错,”国王接着说,“向每个人提出的要求应该是他们所能做到的。权威首先应该建立在理性的基础上。如果命令你的老百姓去投海,他们非得起来革命不可。我的命令是合理的,所以我有权要求别人服从。”

“那我的落日呢?”小王子说,他从来不会忘记自己所提出的问题。

“你会看到你要求的落日的,我会发布命令。要科学管理一个国家,下命令应等时机成熟。”

“什么时候时机才成熟呢?”小王子问。

“嗯,嗯,首先,我要查看大皇历。嗯,嗯,要等到今晚10时40分左右。你会看到他们是怎样执行我的命令的。”

小王子打了个哈欠。他对于他将等不到日落感到很遗憾,而且,他已经感觉有点无聊了。

“在这里我没有别的事可做,”他对国王说,“我要走了。”

“不要走。”这位因为刚刚有了一个臣民而十分骄傲自得的国王说道,“别走,我任命你为大臣。”

“什么大臣?”

“嗯,司法大臣!”

“但这儿无人可审呀!”

“这可不一定,”国王说,“我还没有游完我的领土呢!我年纪大了,这里也没地方可以放马车,走路又太累。”

“噢!可是我已经看过了。”小王子说道,并探身朝星球的那一侧看了看,那边也没有一个人\ldots{}\ldots{}

“这样的话,”国王回答,“那你就审判你自己好了。这可是最困难的。审判自己要比审判别人难得多了。如果你能很成功地做好这件事,你就会成为一个真正的智者。”

“我吗?”小王子说,“随便在什么地方都可以审判自己,没有必要留在这里。”

“我想,”国王又说,“在我的星球上有一只老鼠,每天晚上都能听到它的声音。这样吧,你可以审判它呀!你可以偶尔判它死刑,它的死活就看你如何审判了。不过,每次你都必须再赦免它,因为,这是唯一的一只老鼠,得对它宽大点。”

“我不喜欢判任何人或任何东西死刑,”小王子说,“再说我要走了。”

国王大叫:“不!”

小王子已经做好走的准备,但他不愿伤老国王的心,就对老国王说:“如果陛下希望令行禁止,您可以向我下达合理的命令。例如说,可以命令我在一分钟之内离开。我认为时机已经成熟了。”

国王没有回话,小王子叹口气,迟疑了一下,然后离开了。

国王急忙大叫:“我封你为大使。”

国王摆出一副很权威的样子。

小王子一面赶路,一面自言自语:“大人们真怪。”

{\startalignment[center]
 \placefigeasy[][imgs/小王子(艾柯譯)/figure_0059_0026.jpg][maxwidth=\textwidth,maxheight=\textheight,location={middle,none}]{}
 \stopalignment}

\reference[Chapter1_11.xhtml]{}

\stoptitle

\starttitle[title={11}]

第二颗星球上住着一个爱慕虚荣的人。

“啊!太好了!有一个仰慕者来拜访我了!”看见小王子,这位爱慕虚荣的人大老远就高声喊。

在那些爱慕虚荣的人眼里,别人都成了他们的崇拜者。

“您好!”小王子说,“您戴的帽子好奇怪哟!”

“那是给人行礼用的。”爱慕虚荣的人回答说,“别人向我喝彩时,我就脱帽致意。只可惜这条路从来没人走过。”

“这样子啊?”小王子说。他不知道这个男人在讲些什么。

“快点拍手。”这个爱慕虚荣的人告诉他。

小王子于是拍手。爱慕虚荣的人举起帽子,态度谦恭地行礼。

“这比拜访那位国王好玩儿多了。”小王子心想,于是又多拍了几次手,爱慕虚荣的人又举帽子行礼。

{\startalignment[center]
 \placefigeasy[][imgs/小王子(艾柯譯)/figure_0061_0027.jpg][maxwidth=\textwidth,maxheight=\textheight,location={middle,none}]{}
 \stopalignment}

玩了五分钟,小王子就开始厌倦这种单调的把戏了。

“要怎么做,你才会把帽子丢掉呢?”小王子问。

这个爱慕虚荣的人没有听见他的话,因为凡是爱慕虚荣的人只听得见赞美声。

“你真的很崇拜我吗?”他问小王子。

“崇拜的意思是什么啊?”

“崇拜的意思,就是你认为我是这个行星上最帅、最会穿衣服、最有钱,而且是最聪明的人。”

“可是,你的行星上就只有你一个人啊!”

“你就帮我一个忙,崇拜我一下嘛。”

“好吧!”小王子轻轻地耸了耸肩,“那我就崇拜你吧。可这有什么能使你如此高兴的呢?”

小王子说完就走了。

“大人还真的是很奇怪!”小王子一边赶路一边对自己说。

\reference[Chapter1_12.xhtml]{}

\stoptitle

\starttitle[title={12}]

第三个星球上住着一位酒鬼。这是一次相当短暂的拜访,却使他心情沮丧了好一阵子。

“你在这儿做什么呀?”他看到有个人静静地坐在那里,面前有一大堆空酒瓶子,一旁还有许多装满的瓶装酒。

“喝酒啊!”酒鬼阴沉忧郁地说。

“为什么要喝酒?”小王子追问道。

“为了忘却。”酒鬼回答。

“忘掉什么呢?”小王子动了恻隐之心。

“忘掉我的羞愧。”酒鬼忏悔着,垂下头来。

“为什么羞愧?”小王子很想帮助他。

{\startalignment[center]
 \placefigeasy[][imgs/小王子(艾柯譯)/figure_0064_0028.jpg][maxwidth=\textwidth,maxheight=\textheight,location={middle,none}]{}
 \stopalignment}

“因为我喝酒!”酒鬼醉倒了,进入无声的沉静中。

小王子怅然地走开了,心中满是疑惑。

“大人们真是不可理喻。”他自言自语,叹息着又踏上了旅途。

\reference[Chapter1_13.xhtml]{}

\stoptitle

\starttitle[title={13}]

第四个星球是属于一个商人的。这个商人忙得不可开交,所以,小王子到达时,他甚至连头也没有抬。“早安!”小王子和他打招呼,“你的烟熄灭了。”

“三加二等于五;五加七等于十二;十二加三等于十五。你早!十五加七等于二十二;二十二加六等于二十八。我没时间再点烟了。二十六加五等于三十一。哇!总共是五亿一百六十二万二千七百三十一。”

“五亿什么?”小王子问。

“什么?你还在这里啊?五亿一百万?我也不知道是什么了。我有太多的工作要做\ldots{}\ldots{}我可是很忙的人!我绝不能虚度生命\ldots{}\ldots{}二加五得七\ldots{}\ldots{}”

“五亿一百万什么?”小王子重复问道。只要他问了问题,就绝对不放弃。

商人终于抬起头。

{\startalignment[center]
 \placefigeasy[][imgs/小王子(艾柯譯)/figure_0066_0029.jpg][maxwidth=\textwidth,maxheight=\textheight,location={middle,none}]{}
 \stopalignment}

“我在这行星上已经住四十四年了,四十四年中,只被打断过三次。第一次是二十二年前,不知从哪儿来了一只甲虫,制造出惊人的声音,害我算错了四次。第二次是十一年前,我风湿病发作。我的运动量不够,我没时间闲晃,我可是个严肃的人,真的。第三次,嗯,就是现在!我刚刚说到哪里了?五亿一百万?”

“五亿一百万什么?”

这个商人终于明白,他若不回答这个问题便休想有片刻安宁。

“那些小东西,就是你常在天空看到的那种。”

“苍蝇吗?”

“不,不是,是小小的、发光的东西。”

“萤火虫吗?”

“噢,不是。是那种小小的、闪闪发光的东西,那些让懒惰的人做白日梦的东西啦。不过,我是个很严肃的人,才没时间做白日梦呢!”

“啊,是星星吗?”

“没错!就是星星。”

“你要拿这五亿颗星星做什么?”

“是五亿一百六十二万二千七百三十一颗星星。我很认真地计算过,而且结果是相当精确的,我可是一个相当严肃的人哟!”

“你拿这些星星做什么?”

“我要它做什么?”

“是啊。”

“什么也不做。它们都是属于我的。”

“哦,你拥有这些星星,是吗?”

“那当然。”

“可是,我刚遇到一个国王,他\ldots{}\ldots{}”

“国王并没有拥有权,国王统治它们,这不是一码事!”

“可是,对你来说拥有这些星星有什么用呢?”

“这可以使我变得富有。”

“有钱有什么用?”

“那我就可以买更多星星啊,如果有人再发现星星的话。”

“这个人的想法跟那个酒鬼一样。”小王子心里这样想着。

但是无论如何,小王子还是继续问问题:

“你怎么能拥有这些星星呢?”

“那么你说说,谁可以拥有它们呢?”商人有点烦了,没好气地反问道。

“我不知道,没有人吧?!”

“那就对了!它们是我的,因为我是第一个想到拥有它们的人。”

“是这样子的吗?是这样吗?”

“当然喽。如果你发现一颗没有主人的钻石,它就是你的;如果你发现一个无人的岛,那个岛也就是你的了;当你比别人早一点想到任何创意,你如果去申请专利,那就成了你的了。现在这些星星归我所有,因为之前从来没人想过要拥有星星呀!”

“这倒也是。”小王子说道,“可是你用它们来干什么?”

“我管理它们。我正在重复计算它们的数量。”商人回答,“这相当困难,不过我是一个相当严肃认真的人。”

小王子仍然不满意。

“如果我有一条围巾,我就会把它围在脖子上;如果我有一朵花儿,我可以把它摘下来带走。可是,你不能把星星摘下来。”

“我不能摘星星,但我能把它们存入银行。”

“那是什么意思?”

“意思就是,我可以把多少星星写在一张纸上,然后把这张纸锁在抽屉里。”

“就这样吗?”

“这样就够了。”商人说道。

“真好玩。”小王子想,“还挺有诗意的呢,但算不上是严肃正经的事儿。”对于什么事情是重要的,小王子跟大人的观点非常不一样。

他继续说:“我呢,我自己拥有一朵花儿,我每天给她浇水;我有三座火山,我每个礼拜都会清理一次,我也清理死火山,以防万一。对我的花儿和火山来说,被我拥有是有用的。可是,你对星星好像没有用嘛?!”

商人张口结舌,无言以对。于是小王子就走了。

“大人们真的非常特别。”小王子这样想着,继续他的旅程。

\reference[Chapter1_14.xhtml]{}

\stoptitle

\starttitle[title={14}]

第五颗星球,真的是很奇特。它是所有行星中最小的,只能容得下一盏灯和一个灯夫。

小王子无法理解,在星际里,在这样一个没人居住也没有房子的星球上,要一盏灯和一个灯夫有什么用?

但他仍然对自己说:

“这个灯夫也许很可笑,但是,比起那个国王、酒鬼、爱慕虚荣的人和商人好多了。至少,他的工作还有点意义。当他点亮路灯的时候,就好像另一颗星星或花朵诞生了一样;当他熄掉路灯时,就像是送星星或花朵回去睡觉。这活儿充满诗意,既然充满诗意,那就是有益的工作。”

于是,他怀着崇敬的心情,登上了这个星球。

“早上好,你刚才为什么把路灯熄了呢?”

“这是规定呀!”灯夫回答,“早上好。”

“什么规定?”

“把灯熄掉呀!晚安。”他又点燃了路灯。

“那么为什么你又把灯点亮呢?”

“规定呀。”灯夫回答。

“我不懂。”小王子说道。

“用不着明白,”灯夫说,“规定就是规定。早上好。”

随后他又熄了灯,然后拿出一条红方格的手帕擦一擦前额。

“这活儿可把我累苦了。以前还说得过去,早上熄灯,晚上点灯,剩下的时间,白天我就休息,夜晚我就睡觉。”

“现在的规定改了?”

“规定倒没有改,”点灯人说,“倒霉就倒霉在这儿!行星一年比一年转得快,而规定却没有改变!”

“结果呢?”小王子问。

“结果现在每分钟转一圈,我连一秒钟的休息时间都没有了。每分钟我就要点一次灯,熄一次灯!”

{\startalignment[center]
 \placefigeasy[][imgs/小王子(艾柯譯)/figure_0073_0030.jpg][maxwidth=\textwidth,maxheight=\textheight,location={middle,none}]{}
 \stopalignment}

“真好玩!你这里每天只有一分钟长?”

“一点都不好玩!”灯夫说道,“我们已经聊了一个月了。”

“一个月?”

“没错,一个月。三十分钟,就等于三十天了。晚安!”

灯夫又把灯点上。

小王子看着他,觉得自己挺喜欢这个忠于职守的灯夫。他想起以前要欣赏落日时,只需挪动座椅就行。于是他想帮助灯夫。

“你知道吗?”小王子说,“有一个方法当你想休息的时候就可以让你休息。”

“我一直都想休息。”灯夫说,“一个人是不可能同时努力工作和偷懒的。”

小王子接着说:

“你的行星这么小,走三步就可以绕一圈了,你只要放慢脚步,太阳就老在你的头顶上。你想休息的时候,就往前走------那样,你的白天要多长有多长。”

“这办法帮不了我多大忙,我生来就喜欢睡觉。”灯夫说。

“真不走运。”小王子说。

“真不走运。”点灯人说,“早上好。”

他又熄灭了路灯。

小王子继续踏上他的旅程,他独自想着:“国王、爱慕虚荣的人、酒鬼、商人一定都会看不起灯夫。然而,他却是我觉得唯一不愚蠢的人。也许因为他是唯一不为自己而忙碌的人吧!”

他感慨地叹了口气,自言自语道:

“他是唯一一个我想跟他做朋友的人。可是,他的行星实在太小了,住不下两个人。”

小王子没有勇气承认的是:他留恋这颗令人赞美的星星,是因为在那里二十四小时之内有一千四百四十次日落!

\reference[Chapter1_15.xhtml]{}

\stoptitle

\starttitle[title={15}]

第六个星球比刚才那个要大上十倍。上面住着一位老先生,他正在写一本大部头的书。

他远远看到小王子走来,便嚷嚷开了:“哇!来了一位探险家。”

小王子这次长途跋涉,走了不少路。他走到桌边坐了下来,有点气喘吁吁。

“你从哪儿来?”老先生问道。

“那本厚厚的书是什么呢?”小王子问,“你在这儿做什么?”

“我是地理学家。”老先生说道。

“什么是地理学家?”

“地理学家就是一种学者,他知道哪里有海洋,哪里有江河、城市、山脉、沙漠。”

“那可真有意思。”小王子说,“总算碰上一个有真正职业的人了!”他在地理学家所住的星球四面逛了一遍,发现这是他迄今为止所见到的最壮丽的一个星球了。

“您的星球真美呀。上面有海洋吗?”

“我无法知道。”地理学家说道。

“哦!”小王子说,“那有山脉吗?”

“我无法知道。”地理学家回答。

“那\ldots{}\ldots{}城市、河流、沙漠呢?”

“我也无法知道。”

“可是,你是一个地理学家啊!”

“没错,”地理学家说道,“但我不是探险家啊。我这儿根本没有探险家。地理学家是不应该去计算和探测城镇、河流、高山、大海、大洋、沙漠的。地理学家的工作太重要了,根本没有时间到外面闲逛!他绝不能离开办公室,不过,他会从探险家那儿接收信息来作为研究材料。他会问他们问题,记下他们的经历。要是某个探险家提供的旅游记录够有趣的话,那么地理学家就会对他的品行和背景进行调查。”

“为什么?”

“一个说谎的探险家会让地理学家写的书很惨,饮酒过量的探险家也是一样。”

“这又是为什么?”小王子好奇地问。

“因为喝醉了酒的人会把一个看成两个;那么,地理学家就会把只有一座山的地方写成两座山。”

“我认识一个人,他就是那种差劲的探险家。”小王子说。

“这是可能的。因此,如果探险家的品德足够好,我才会去调查他的发现。”

“你亲自去实地调查吗?”

“不,那太复杂了。不过,我会要求探险家提供证据。例如,如果他发现了一座大山,就要求他带来一些大石头。”

地理学家突然兴奋起来。

“你是个来自远方的探险家,请你把你的那个星球给我描述一番吧!”

{\startalignment[center]
 \placefigeasy[][imgs/小王子(艾柯譯)/figure_0079_0031.jpg][maxwidth=\textwidth,maxheight=\textheight,location={middle,none}]{}
 \stopalignment}

地理学家打开他的登记簿,把铅笔削尖。

他先用铅笔记下探险家的口述。探险家提供证据后,再用钢笔誊写。

“好啦!请谈吧!”地理学家说。

“哦,我住的那个地方不是很有趣。”小王子说,“那里所有的东西都是小小的。星球上一共有三座火山:两座是活火山,一座已经熄灭了。不过,谁知道将来会怎么样。”

“你当然不会知道。”地理学家说。

“我还有一朵花儿。”

“我对花儿不感兴趣。”地理学家说。

“为什么不?那是我那个星球上最美的东西啊!”

“因为花儿不被列入记录,”地理学家说,“花儿的生命是短暂的。”

“什么叫短暂?”

“地理学书籍是所有书中最严肃的。”地理学家说道,“这类书是从不会过时的。很少会发生一座山变换了位置,或者一个海洋干涸的现象。我们记载的是永恒的东西。”

“但是熄灭的火山也可能会再复苏的。”小王子打断了地理学家。

“什么叫短暂?”

“火山是熄灭了的也好,苏醒的也好,这对我们这些人来讲都是一回事。”地理学家说,“对我们来说,重要的是山。山是不会变换位置的。”

“但是,‘短暂'是什么意思?”小王子再次问道。他一旦提出一个问题就绝不放过。

“意思就是很快就会消失。”

“我的花儿是很快就会消失的吗?”

“那当然。”

“我的花儿的生命是短暂的,”小王子自言自语地说,“而且她只有四根刺来保护自己!可我却把她孤零零地独自留在家里!”

这是小王子第一次对自己的离开感到懊悔。不过,很快地,他又重新振作起来。

“您能否指点我,我该去哪儿访问?”

“地球,”地理学家回答,“它的名声不错。”

于是,小王子离开了,心中想念着他的花儿。

\reference[Chapter1_16.xhtml]{}

\stoptitle

\starttitle[title={16}]

第七颗星球就是地球。

地球可真不是个普通的星球!它上面有一百一十一位国王(当然,也包括黑人国王在内)、七千个地理学家、九十万个商人、七百五十万个酒鬼,还有三亿一千一百万个自负的人。这些人全部加起来,大约有二十亿个大人。

为了让你对地球的大小有个概念,我想告诉你,在发明电灯之前,如果要使全球七大洲的灯全都亮起来的话,就得让四十六万二千五百一十一个灯夫来点灯。远远望去,真是光彩夺目,无比绚烂,就像歌剧院的芭蕾舞动作一样有条不紊。

首先上场点灯的应该是新西兰和澳大利亚的灯夫,点着了灯,随后他们就去睡觉了;接着,就换中国和西伯利亚的灯夫登上舞台,他们很快就退到舞台后面去了;然后,就轮到俄罗斯和印度的灯夫;接着是非洲和欧洲;再来是南美洲;然后是北美洲的灯夫。而且,没有人会出错,或是把出场的顺序搞错,这真是一个奇观啊!

{\startalignment[center]
 \placefigeasy[][imgs/小王子(艾柯譯)/figure_0083_0032.jpg][maxwidth=\textwidth,maxheight=\textheight,location={middle,none}]{}
 \stopalignment}

只有负责照料北极和南极那盏孤零零的路灯的两名点灯人,过得比谁都安逸,因为,他们两个人一年只需忙碌两次就够了。

\reference[Chapter1_17.xhtml]{}

\stoptitle

\starttitle[title={17}]

当你试着要诙谐一点时,有时可以撒点无关紧要的小谎。关于点灯人,我说得并非十分明白。可能会让那些不清楚地球现状的人,产生一些误解。人们在地球上所占据的空间其实很小,如果地球上二十亿人成排紧密地站在一起,就像参加聚会那样,他们很容易就可挤进一个边长二十英里的正方形广场。也就是说可以把全部的人类,都挤到一个太平洋的小岛上。

当然,大人们是不会相信这些的。他们会觉得自己占了很大的空间,他们把自己看得像猴面包树那样大得不得了。那你就建议他们算一算。他们不是喜欢数字吗?他们会高兴的。

但你本人无须浪费时间去过问这件多余的事,这毫无必要。我想,你是会相信的。

小王子一登陆就很惊讶,他没有看到任何人。当一个月色的环状物在沙间移动时,他已经开始猜想:自己是不是登陆错星球了。

“晚安。”小王子并不指望得到回答。

“晚安。”一条蛇回应说。

“我在哪一个星球?”小王子问。

“在地球上,在非洲。”蛇回答。

“啊!怎么,难道说地球上没有人吗?”

“这里是沙漠,沙漠中没有人。地球是很大的。”蛇说。

小王子在一块石头上坐下来,仰望天空。

“我在想,星星们闪闪发亮是不是为了让每个人找到回家的路,”他说,“看,我的那颗星星,它恰好就在头上,却距离如此遥远!”

“它很美。”蛇说,“你到这里来干什么呢?”

“我和一朵花儿之间有点问题。”小王子说。

“噢!”蛇说。

于是他们都沉默下来。

“所有的人都到哪儿去了?”小王子终于又开了腔,“在沙漠上,真有点孤独。”

“人群里也是很寂寞的。”蛇说。

小王子久久地凝视着它。

“你是个奇怪的动物,细得像个手指头。”小王子终于又开始说话了。

“可是,我比国王的手指头还要有力。”蛇说。

小王子微笑了。

{\startalignment[center]
 \placefigeasy[][imgs/小王子(艾柯譯)/figure_0086_0033.jpg][maxwidth=\textwidth,maxheight=\textheight,location={middle,none}]{}
 \stopalignment}

“你才没有什么力呢!你又没有脚,也走不了多远。”

“我能带你去任何船只都无法到达的地方。”蛇说。

它把自己缠在小王子的脚踝上,就像一只金镯子一样。

“地球上无论是谁被我碰到,都会被我送回老家,”它继续说道,“可是,你是如此纯洁,又来自别的星球。”

小王子默默无言。

“我为你感到难过------在这个花岗岩组成的地球上,你是如此脆弱,”蛇说道,“如果你发现自己很想家的话,我可以帮助你,我可以\ldots{}\ldots{}”

“噢!我完全明白你的意思,”小王子说,“但你为什么要说些令人费解的话呢?”

“才不呢!我专门解答问题的。”蛇说。

于是,他俩又陷入沉默。

\reference[Chapter1_18.xhtml]{}

\stoptitle

\starttitle[title={18}]

小王子穿过沙漠,他只遇到一朵花儿,一朵有着三枚花瓣的花,一朵很不起眼的小花儿。

“早安。”小王子说。

“早安。”花儿说。

“请问人都到哪儿去了?”小王子十分礼貌地问道。

花儿曾看到一支商队经过。

“人吗?我想大概有六七个吧,几年前看到过他们,但我不知道在哪儿能找到他们,风把他们吹散了。他们没有根,活得很辛苦。”

“再见了。”小王子说。

“再见。”花儿说。

\reference[Chapter1_19.xhtml]{}

\stoptitle

\starttitle[title={19}]

小王子爬上了一座高山。过去,他所知道的山,就是他那座只有膝盖高的火山,他还很习惯把死火山当做凳子呢!他心想:“登上这样的一座高山,我就可以一眼望尽整个星球和全人类了。”

然而,他看到的只是悬崖峭壁。

“哈喽!”他并不指望有任何响应。

“哈喽------哈喽------哈喽”,一阵阵回音传来。

“你是谁?”小王子问。

“你是谁?你是谁?你是谁?”回音回答。

“做我的朋友吧!我好寂寞。”他喊道。

“我好寂寞------我好寂寞------我好寂寞。”依旧是回音。

于是小王子想着:“好奇怪的星球!这里又干燥、又陡峭,充满咸味,而且人们一点想象力也没有,他们只会重复你对他们说的话而已。在我的家乡,我的花儿总是先开口说话。”

{\startalignment[center]
 \placefigeasy[][imgs/小王子(艾柯譯)/figure_0091_0035.jpg][maxwidth=\textwidth,maxheight=\textheight,location={middle,none}]{}
 \stopalignment}

\reference[Chapter1_20.xhtml]{}

\stoptitle

\starttitle[title={20}]

在沙漠、岩石、雪地上行走了很长时间以后,小王子终于发现了一条大路。只要有路,就会通向人类居住的地方。

“哈喽!”他说。

他走过一个盛开着玫瑰的花园。

“哈喽!”玫瑰花儿们说。

小王子注视着她们,她们看上去和他的花儿非常相似。

“你们是谁?”他惊奇地问她们。

“我们是玫瑰花儿呀。”玫瑰们说。

“啊!”小王子说。

他感到很伤心。他的花儿曾经告诉他,她是全宇宙仅有的一朵玫瑰。而这里,单是这个花园里就有五千朵玫瑰花,长得跟他的花儿一模一样!

“她一定会气得要命,如果她看到这幅景象的话,”他想,“她准会咳得很厉害,而且会装出一副快死的样子,以免被嘲笑。而我就得装出照顾她的样子------因为如果我不这样做的话,她真的会因为羞辱而自杀。”

然后,他对自己说:“我还以为我有一朵独一无二的花儿呢,我有的只是一朵普通的花儿。这朵花儿,再加上三座只有我膝盖那么高的火山,而且其中一座还可能是永远熄灭了的。这些根本就不能让我成为一个了不起的王子啊!”于是,他趴在草地上伤心地哭泣起来。

{\startalignment[center]
 \placefigeasy[][imgs/小王子(艾柯譯)/figure_0093_0036.jpg][maxwidth=\textwidth,maxheight=\textheight,location={middle,none}]{}
 \stopalignment}

\reference[Chapter1_21.xhtml]{}

\stoptitle

\starttitle[title={21}]

这时,一只狐狸出现了。

“早上好。”狐狸说。

“早上好。”小王子很有礼貌地回答。他转过头去,但什么也没看到。

“我在这儿,在苹果树下。”那个声音说。

“你是谁?”小王子问,又接着说,“你看起来好漂亮。”

“我是一只狐狸。”狐狸说道。

“来和我一起玩吧,”小王子提议,“我现在很伤心。”

“我不能和你玩,”狐狸回答,“我还没有被驯服。”

“啊!对不起。”小王子说。

{\startalignment[center]
 \placefigeasy[][imgs/小王子(艾柯譯)/figure_0095_0037.jpg][maxwidth=\textwidth,maxheight=\textheight,location={middle,none}]{}
 \stopalignment}

他想了一想,之后说:

“驯服?什么叫驯服?”

“原来你不是这里的人。”狐狸说,“你在找什么?”

“我在找人,”小王子说,“驯服是什么意思?”

狐狸说:“人类有枪,他们会打猎,这真是讨厌。但他们也养鸡,这是他们的可取之处。你在找鸡吗?”

“不是,”小王子说,“我在找朋友。驯服是什么意思?”

“这是常常被人们遗忘的事情,”狐狸说道,“它的意思就是‘建立关系'。”

“建立关系?”

“没错,”狐狸说,“对我而言,你不过是个小男孩,就像其他千万个小男孩一样。我不需要你,你也同样不需要我。对你来说,我也不过是只狐狸,就跟其他千万只狐狸一样。然而,如果你驯服了我,我们将会彼此需要,对我而言,你将是宇宙间唯一的了;我对你来说,也是世界上唯一的了。”

“我有点儿明白了,”小王子说,“有一朵花儿我想她已经驯服我了。”

“很有可能,”狐狸说道,“地球上什么事情都可能发生。”

“噢,这不是地球上的事。”小王子说。

狐狸很好奇:“在另一个星球上吗?”

“是的。”

“那个星球上有猎人吗?”

“没有。”

“哇,多有趣呀!那么,有鸡吗?”

“没有。”

“没有十全十美的事。”狐狸说着,叹了口气。

但狐狸又把话题拉回来:

“我的生活很单调乏味。我捕捉鸡,而人又捕捉我。所有的鸡全都一样,所有的人也全都一样。我已经厌烦了。不过,如果你驯养我,那我的生命就会充满阳光,你的脚步声会变得跟其他人的不一样。其他人的脚步声会让我迅速躲到地底下,而你的脚步声则会像音乐一样,把我召唤出洞穴。然后,你看,看到那边的麦田了吗?我不吃面包,麦子对我来说一点意义也没有。麦田无法让我产生联想,这实在很可悲。但是,你有一头金黄色的头发,如果你驯养我,那该会有多么美好啊!金黄色的麦子会让我想起你,我甚至会喜欢上风在麦穗间吹拂的声音。”

狐狸停止了说话,凝视着小王子。

“求求你驯养我吧!”

“我很想,”小王子说,“可我没有太多时间。我想去交朋友,还有,想弄懂很多事情。”

“你只能了解你所驯养的东西。”狐狸说,“人类不再有时间去了解事情了,他们总是到商店里买现成的东西。但是,却没有一家商店贩卖友谊,所以,人类没有真正的朋友。如果你想要一个朋友,就驯养我吧!”

{\startalignment[center]
 \placefigeasy[][imgs/小王子(艾柯譯)/figure_0098_0038.jpg][maxwidth=\textwidth,maxheight=\textheight,location={middle,none}]{}
 \stopalignment}

“那我要做些什么呢?”小王子问。

“需要有非常的耐心。”狐狸回答,“首先,你必须离我稍远一点,就那样,远远地坐在那边的草地上。我会用眼角不经意地瞟你,这时,你什么也不要说。言语可是会导致误会的。然后,你可以一天天地向我靠近\ldots{}\ldots{}”

第二天,小王子又来了。

“你要能在每天同一个时间来就更好了。”狐狸说道,“比如说你在下午四点来,从三点钟开始,我就开始感觉很快乐,时间越临近,我就感到越来越快乐。到了四点钟,我就会坐立不安,我发现了幸福的价值。但是如果你随便什么时候来,我就不知道在什么时候准备好迎接你的心情了。仪式还是需要的。”

“什么是仪式?”小王子问道。

“这是另一件经常被人们遗忘的事情。”狐狸说,“它就是使某一天与其他日子不同,使某一时刻与其他时刻不同的东西。比如说,那些猎人就有一种仪式,他们每星期四都和村子里的姑娘们跳舞。于是,星期四就是一个美好的日子!我可以一直散步到葡萄园去。如果猎人们任何时候都在跳舞,天天又全都一样,那我也就没有假日了。”

{\startalignment[center]
 \placefigeasy[][imgs/小王子(艾柯譯)/figure_0100_0039.jpg][maxwidth=\textwidth,maxheight=\textheight,location={middle,none}]{}
 \stopalignment}

就这样,小王子驯服了狐狸,在他离开的时刻快要到来时。

“啊!”狐狸说,“我一定会哭的。”

“这就是你自己的错了,”小王子说,“我不想伤害你,是你要我驯养你的。”

“对啊。”狐狸说。

“可是,你快哭出来了!”小王子说。

“当然喽。”狐狸说。

“你根本没有得到什么好处!”

“不,我得到了好处!现在我拥有麦子的颜色了。”

然后,狐狸又接着说:“再去看看那些玫瑰花吧!你一定会明白,你的那朵是世界上独一无二的玫瑰。你回来和我告别时,我会再赠送给你一个秘密的。”

于是,小王子又去看那些玫瑰花儿。

“你们一点儿也不像我的玫瑰。”他告诉她们,“你们现在什么都不是。没有人驯养过你们;你们也没驯养过任何人。你们就和我的狐狸以前那样,她曾经和其他千百只狐狸一样,但现在她是我的朋友了。从此以后,她是世界上独一无二的狐狸。”

玫瑰们一点都不喜欢他所说的话。

“你们很美,”他继续往下说,“但是很空虚。没有人会为你们而死,没错,一般的过路人,可能会认为我的玫瑰和你们很像,但她只要一朵就胜过你们全部,因为她是我灌溉的那朵玫瑰花儿;她是我放在玻璃罩下面,让我保护她不被风吹袭,而且为她打死毛毛虫(只留两三只变成蝴蝶)的玫瑰;因为,她是那朵我愿意倾听她发牢骚、吹嘘,甚至沉默的那朵玫瑰;因为,她是我的玫瑰。”

然后,小王子又回到狐狸身边。

“再见了。”他说。

“再见,”狐狸说,“我的秘密其实很简单:只有用心灵才能看得清事物的本质;真正重要的东西是肉眼无法看见的。”

“真正重要的东西是肉眼无法看见的。”小王子重复着狐狸的话,以便能把它记在心间。

“因为你把时间投注在你的玫瑰花儿身上,所以,她才会如此重要。”

“因为我把时间投注在我的玫瑰花儿身上\ldots{}\ldots{}”小王子说着,以免自己忘记。

“人类已经忘记这个简单的真理了。”狐狸说,“不过,你不可以忘记,你必须对那些你所驯养的东西负责。你必须对你的玫瑰花儿负责任。”

{\startalignment[center]
 \placefigeasy[][imgs/小王子(艾柯譯)/figure_0103_0040.jpg][maxwidth=\textwidth,maxheight=\textheight,location={middle,none}]{}
 \stopalignment}

“我必须对我的玫瑰花儿负责任。”小王子重复着这些话,他要让这些话深深地印在自己的脑海中。

\reference[Chapter1_22.xhtml]{}

\stoptitle

\starttitle[title={22}]

“早上好!”小王子打着招呼。

“早上好!”扳道工回应着。

“你在这儿干什么?”小王子问他。

“我为人们扳道,发车。”扳道工回答。

这时,一列华丽的快车飞驰而过,响声如雷,将扳道工的小屋震得颤动起来。

“跑得那么急!”小王子说,“他们要去哪里?”

“这个嘛,连列车长也不一定说得清楚呀!”扳道工回答说。

第二列崭新的快车又从另一个方向飞驰而过。

“他们已经回来了吗?”小王子还在探究。

“这可不是先前的那一列,”扳道工回答,“它们正好是对开列车。”

“他们不满意原先住的地方吗?”小王子问道。

“人总是这山望着那山高的。”扳道工说。

第三列灯火通明的火车也轰隆隆地响起来了。

“他们在追第一批旅客,对不对?”小王子问。

“他们什么人也不追。他们在里面睡觉,打哈欠。”扳道工说,“只有孩子们把鼻子贴在玻璃窗上往外看。”

“是呀,看来只有孩子们才知道他们在追寻什么。”小王子说,“他们为一个布娃娃花费不少时间,这个布娃娃就成了很重要的东西,如果有人夺走他们的布娃娃,他们就哭泣。”

“孩子们真幸运。”扳道工说。

\reference[Chapter1_23.xhtml]{}

\stoptitle

\starttitle[title={23}]

“早上好!”小王子在打着招呼。

“早上好!”商人回答。

那是一位推销止渴丸的商人。一个星期吃一颗,你就不会觉得口渴了。

“你为什么要卖这种药丸呢?”小王子问道。

“为了节省更多的时间。”商人说,“专家们计算过,用这种药丸每个星期能节省五十三分钟。”

“那么,这五十三分钟用来做什么呢?”

“随你便吧\ldots{}\ldots{}”

“我若有五十三分钟的空闲,我就会悠闲地逛到清冽的泉边去。”

\reference[Chapter1_24.xhtml]{}

\stoptitle

\starttitle[title={24}]

这是我的飞机在沙漠失事的第八天。我在听小王子讲生意人的故事时,喝完了最后一滴水。

“唉!”我对小王子说,“你的回忆确实非常动人,现在的问题是我不但没有修好飞机,而且身边再没有水可喝了。要是我也能悠闲地逛到清冽的泉边去,我当然也会非常高兴的。”

“我的狐狸朋友\ldots{}\ldots{}”小王子还有话要对我说。

“我的小人儿,别再说狐狸的事了!”

“为什么?”

“因为,我们快要渴死了。”

他根本就不了解我的意思,继续说:

“有个朋友是很好的,即使你快死了。像我,我就真的很高兴我有一个狐狸朋友。”

“他根本一点儿危机意识都没有。”我想,“他根本就没有渴过,也没饿过,他所需要的只是一点阳光就可以了。”

但是,他凝视着我,好像猜透了我的心思:

“我也很渴,我们去找井吧\ldots{}\ldots{}”

我疲倦地耸耸肩。在浩瀚无垠的沙漠里,去找一口井,真荒谬。然而,我们还是出发了。

我们沉默地走了好几个小时之后,天黑了下来,星星也出来了。我渴得有点神智不清,望着天上一颗颗明亮的星星,仿佛自己正置身于梦中。小王子说的话似乎在我的脑海中跳跃。

“你也渴了,是吗?”我问他。

他没有回答我的问题,只是说:

“水对心灵也会有好处\ldots{}\ldots{}”

我不懂他的话是什么意思,只好沉默。我知道不应该去问他。

他累了,坐了下来。我也在他身旁坐下。沉默了一会儿后,他又说道:

“因为有一朵我们看不到的花儿,星星才显得如此美丽。”

“当然。”我说。

然后我便望着月光下绵延起伏的沙丘,不再多说什么。

“沙漠是如此美丽。”小王子说。

这倒是真的。我一直很喜欢沙漠。你可以坐在沙丘上,虽然看不到任何东西,也听不到任何声音,却有一种说不出的东西在那幽幽深处散发着光芒。

“沙漠美丽,是因为沙漠的某处隐藏着一口井。”小王子说。

我很惊讶,突然明白了为什么沙漠放着光芒。当我还是个小男孩的时候,我住在一栋古老的房子里。有一个传说,说那里埋有宝藏。当然,没有人能找到宝藏,可能也没有人看到过宝藏。但是,那房子却因此笼罩了一层魔力。我家的地心深处,隐藏了一个秘密。

“是啊,”我对小王子说,“不管是房子、星星,或是沙漠,都因为看不见的东西而显得美丽!”

“真高兴你同意我的狐狸的看法。”他说。

小王子睡着了,我抱起他,继续朝前走。我的内心有种深深的感动,仿佛自己抱着的就是一件非常脆弱的珍宝。我甚至觉得,在这个世界上再也找不到比这更纤弱的宝物了。月光下我看着那苍白的前额、紧闭的眼睛和那在微风中轻轻颤动的鬓发,此刻,我对自己说:“我见到的不过是具躯壳,而那最重要的部分,光靠肉眼是看不见的\ldots{}\ldots{}”

他双唇微张,嘴边荡漾着朦胧的笑意。

我喃喃自语:“熟睡中的这位小王子让我如此感动的,是他对那朵花儿忠贞的爱\ldots{}\ldots{}那朵玫瑰的影子,照亮了他整个生命,如同一盏明灯,甚至当他熟睡的时候\ldots{}\ldots{}”

而我竟觉得他更脆弱了。我心中产生了一股要保护好他的冲动,仿佛他本身就是一朵微风吹过就可能熄灭的火花儿\ldots{}\ldots{}

{\startalignment[center]
 \placefigeasy[][imgs/小王子(艾柯譯)/figure_0110_0041.jpg][maxwidth=\textwidth,maxheight=\textheight,location={middle,none}]{}
 \stopalignment}

就这样,走着走着,拂晓的时候,我们找到了水井。

\reference[Chapter1_25.xhtml]{}

\stoptitle

\starttitle[title={25}]

“人们拼命挤进快速火车,”小王子说,“却不知道自己在寻找些什么。于是他们变得忧虑、烦躁,在原地打转\ldots{}\ldots{}”

然后,他又补充道:

“这没有必要。”

我们所找到的井不像沙漠中的井。沙漠里的井只是地上的洞,我们的井却像村子里的井。可是,四周并没有村子啊,我想,我一定是在做梦。

“真是奇怪了,”我告诉小王子,“打水用具全都在:辘轳、水桶、绳子\ldots{}\ldots{}”

他笑着,拉起绳子,转动辘轳,当风儿再度吹动时,辘轳发出的声音,像是久未使用的风向标。

{\startalignment[center]
 \placefigeasy[][imgs/小王子(艾柯譯)/figure_0112_0042.jpg][maxwidth=\textwidth,maxheight=\textheight,location={middle,none}]{}
 \stopalignment}

“听!”小王子说,“我们叫醒这口井了,它正在唱歌呢。”

我不想让小王子太累。

“让我来吧!”我说,“这对你来说太重了。”

我慢慢地把水桶提到井栏上,把它稳稳地放在那里。我的耳朵里回响着辘轳的歌声。在晃荡的水面上,闪烁着粼粼的波光。

“我好想喝这水,”小王子说道,“给我喝一点儿。”

这时我才明白他要寻找的是什么!

我把水桶举到他唇边,他闭着眼睛啜饮。这是怎样的飨宴啊!这水的确比一般水更富有营养,它是来自于星空下的漫步、辘轳的歌声和我端水的辛劳呢!它就像是一件礼物一样。当我还是个小男孩的时候,圣诞树上的光、子夜弥撒的音乐和温柔的笑脸,给我收到的礼物增添了许多光辉。

“人类在一个花园里种了五千朵玫瑰,”小王子说,“然而,却仍找不到自己真正追寻的东西。”

“是啊,他们是找不到的。”我回答。

“可是他们一直追寻的东西,其实就在一朵花儿上或几滴水中。”

“一点不错。”我回答道。

然后,小王子接着说:

“肉眼是盲目的,我们必须要用心才能看得到\ldots{}\ldots{}”

我喝过水,呼吸也顺畅多了。旭日把沙漠染成了蜂蜜色,这个颜色让我感到很快乐。然而,为什么我又如此忧伤呢?

“你可要履行诺言啊。”小王子温柔地对我说,他又靠近我坐着。

“什么诺言?”

“你知道的呀,就是给我的羊画个口罩\ldots{}\ldots{}我要对我的花儿负责呀\ldots{}\ldots{}”

我从口袋中拿出我的画稿。小王子看看那些画,笑着说:

“你画的猴面包树,有点像白菜。”

“啊!”

我还一直自以为画得不错呢!

“你画的狐狸耳朵有点像犄角,而且太长了!”然后,他又笑了。

“这不公平,我的小人儿,我唯一会画的,只有蟒蛇的外观图和透视图。”

“噢,没关系的,”他说,“小孩子会看懂的。”

然后,我用铅笔画了一个口罩,把它交给他的时候,我的心揪得紧紧的。

“你有什么计划瞒着我吗?”我问道。

他没有回答我,只说:

“你知道吗,明天就是我到地球的周年纪念日了。”

接着,沉默了一会儿,他又说道:

“当时我就落在附近\ldots{}\ldots{}”

此时,他的面颊绯红。

不知为什么,我有一种说不出来的惆怅悲哀。突然间我想到一个问题:

“那么,八天前的那个早晨,我在那个人迹罕至的地方遇见你,并不是偶然?!你是在走回你降落地点的半路上,是吗?”

小王子的脸又红了。

我又犹豫着问:“是为了纪念降落周年?”

小王子的脸再次红了。

他没有回答我的问题。但是,脸红就表示“是”了,不是吗?

“噢,”我说,“我怕\ldots{}\ldots{}”

但是他却回答我:

“现在,你该回去工作了,快回去修飞机吧。我会一直在这里等,等你明天傍晚再回来\ldots{}\ldots{}”

可是我放不下心。我想起了那只狐狸,如果你被驯养的话,难免要掉泪了。

{\startalignment[center]
 \placefigeasy[][imgs/小王子(艾柯譯)/figure_0116_0043.jpg][maxwidth=\textwidth,maxheight=\textheight,location={middle,none}]{}
 \stopalignment}

\reference[Chapter1_26.xhtml]{}

\stoptitle

\starttitle[title={26}]

水井边有片老旧的、残缺的石墙。第二天晚上我弄完飞机赶过来的时候,远远地就看到小王子坐在墙上,两只脚晃来晃去。我听到他说:

“你不记得啦,”他说,“不是这里呀。”一定有别的声音在跟他讲话,因为他答道:

“是,是!是今天没错,但是地点不对呀。”

我继续朝着墙走去,仍然没看到或听到任何人的声音。但是,小王子又说话了:

“\ldots{}\ldots{}当然,你会看到我留在沙地上的脚印,看到它是从哪儿开始的,你在那儿等我就行了,今夜我会去那儿的。”

我离那片墙只有二十米远,可是却始终看不到任何东西。

沉默了一会儿之后,小王子又开口了:

“你的毒液很毒,是吗?你肯定那不会让我痛苦太久吗?”

我停了下来,虽然听不明白他在讲什么,但心里却有一种说不出的滋味。

{\startalignment[center]
 \placefigeasy[][imgs/小王子(艾柯譯)/figure_0118_0044.jpg][maxwidth=\textwidth,maxheight=\textheight,location={middle,none}]{}
 \stopalignment}

“现在,你走开吧。”小王子说,“我要下来了。”

我低头看着墙脚,吓了一跳。在那儿,竖起来正对着小王子的,是一条黄色的蛇,三十秒内就能让人致命。我从口袋里掏出左轮手枪,奔跑过去。蛇一听到我的声音,便溜进沙堆里,就像一条潺流的小溪在慢慢地游移,然后,他带着一丝金属般的声响,溜进了石缝中。

我跑到墙边时,正好把我的小王子接在怀里。他的脸色像雪一样惨白。

“这是怎么回事,噢,你怎么和蛇也谈起心来了?”

我解开他一直围着的金色围巾,用一块湿布在他的太阳穴上沾了些水,并给他一些水喝。现在,我不敢再问他任何问题了。他忧伤地看着我,双手围着我的脖子。我感觉到他的心跳声,就像一只遭受枪伤,生命垂危的鸟儿的心跳声一样。

“我很高兴你终于排除了引擎故障。”他说,“这样,你就可以回家了。”

“你怎么知道的?”

我本来就想告诉他,我的活儿干得很顺手,超出我的想象。

他没有回答我的问题,接着说:

“我今天也要回家了。”

然后,他忧伤地说:

“路好远也很困难\ldots{}\ldots{}”

我清楚地知道,不寻常的事就要发生了。我用双臂紧紧地搂住他,就像搂着一个可爱的婴儿那样。对我来说,这就好像他要一头掉进无底的深渊里,而我却无法拉住他。

他的表情非常严肃,深邃而迷茫。

“我有你画的羊,还有羊的小屋。而且,还有口罩\ldots{}\ldots{}”

然后,他带着忧伤的神情笑了。

我等待了很长时间,终于看到他渐渐恢复了红润。

“小人儿,你受惊了\ldots{}\ldots{}”

他真的很害怕,但他静静地笑了。

“今天晚上我会更害怕\ldots{}\ldots{}”

再一次,我的心由于这种无能为力的感觉打了个寒颤。我知道,我无法忍受今后再也听不到他的笑声。对我来说,他的笑声就像沙漠中的喷泉。

“小人儿,”我说,“我想再继续听到你的笑声。”

可是,他却说:

“到今天夜里,正好是一年了,我会在我一年前降落地点的上空,找到我的那颗星星\ldots{}\ldots{}”

“小人儿,求求你告诉我,蛇、见面的地方,还有星星,都只是一场噩梦,对不对?”

但是,他并没有回答我的问题,他说:

“真正重要的东西是看不见的\ldots{}\ldots{}”

“没错,我知道\ldots{}\ldots{}”

“就像我的花儿一样。如果你爱上了某个星球上的一朵花儿,那么,只要在夜晚仰望星空,就会觉得满天的繁星都像一朵朵盛开的花儿\ldots{}\ldots{}”

“是的\ldots{}\ldots{}”

“就像水一样。因为辘轳和绳子,使得你让我喝的水有如音乐一般。你记得吗,它是如此甜美?”

“是的\ldots{}\ldots{}”

“夜晚,你抬头望着天空,寻找我的那颗星星。我的那颗太小了,无法指给你看。这样更好\ldots{}\ldots{}你就把我的星星看做是万千星星中的一颗吧,这样你就会爱看所有的星星,他们都会变成你的朋友。另外,我还要送给你一件礼物\ldots{}\ldots{}”

然后,他又笑了。

“噢,小人儿,亲爱的小人儿!我多么喜欢听到你的笑声!”

“对啊,这笑声就是我的礼物。它会像我们喝的水一样\ldots{}\ldots{}”

“你想说明什么呢?”

“星星对每个人的意义是不一样的。对旅行的人来说,星星可以指引方向;对有些人来说,星星只是一些小光点;对专家来说,星星是研究对象;对我遇到的商人来说,星星是黄金。然而,所有的星星都是沉默的。你的星星将和别人的星星都不一样。”

“你的意思是\ldots{}\ldots{}”

“我会住在这其中的一颗星星上面,在某一颗星星上微笑着,每当夜晚你仰望星空的时候,就会像是看到所有的星星都在微笑一般!”

于是,他又笑了。

“当你抚平你的忧伤时(时间会缓解任何忧伤),你将是我永远的朋友,你要和我一起笑。而且,有时候,当你为了与我一同欢笑而打开窗户时,你的朋友一定会因为你看着天空微笑而感到很惊讶。到时候,你就可以告诉他们,‘没错,星星常让我笑!'然后,他们会认为你疯了。这是我跟你开的小玩笑\ldots{}\ldots{}”

他又笑了。

“这就好像我给了你很多会笑的小铃铛,而不是小星星一样\ldots{}\ldots{}”

说着,他又笑了。然而他的笑容不久又蒙上了一种凝重。

“听着,今晚不要来!”

“我不会丢下你一个人的。”我说。

“那时候我看起来会很痛苦,一副快死掉的样子。事情看起来就会像那样子,所以我不要你来,也不要你看,不要来\ldots{}\ldots{}”

“我不会丢下你一个人的。”

他露出忧心忡忡的神色。

“我告诉你这些也是因为蛇的缘故。别让它咬了你,蛇是很坏的,它随意咬人\ldots{}\ldots{}”

“我不会丢下你一个人的。”

但是,突然间,他平静下来:

“对呀!它没有足够的毒液可以咬第二口。”

那天晚上,我没有看到他出发,他是悄悄走的。当我追上他时,他正迅速而坚定地向前走着,他对我说的只是:“噢!你来了。”

他心神不定地紧握我的手。

{\startalignment[center]
 \placefigeasy[][imgs/小王子(艾柯譯)/figure_0124_0045.jpg][maxwidth=\textwidth,maxheight=\textheight,location={middle,none}]{}
 \stopalignment}

“你不该来的,你会很难过的,看到我那副快死的样子。虽然那不是真的\ldots{}\ldots{}”

我没有说话。

“你知道的,路途太远了,我不能带着这副躯壳呀,那太重了。”

我没有说话。

“那只是一副老旧的躯壳而已,你没有必要为老旧的躯壳而哀伤\ldots{}\ldots{}”

我还是沉默不语。

他有点儿泄气,但马上又振作起来:

“想起你,我会很幸福的,你知道,我也会看着星星啊。所有的星星都将会成为有着生锈辘轳的井,所有的星星都会流出水来让我喝\ldots{}\ldots{}”

我依旧沉默。

“那该会多么有趣呀!你会拥有五亿个小铃铛;我会拥有五亿口井\ldots{}\ldots{}”

然后,他也沉默下来,泪水爬满了他的脸。

“就是这里了,让我自己走吧。”

他坐了下来,显得很害怕。然后说:

“你知道的,我得对我的花儿负责。她是如此脆弱!如此天真无邪!她只带着四根一点用也没有的刺,来保护自己,对抗周围的环境\ldots{}\ldots{}”

{\startalignment[center]
 \placefigeasy[][imgs/小王子(艾柯譯)/figure_0126_0046.jpg][maxwidth=\textwidth,maxheight=\textheight,location={middle,none}]{}
 \stopalignment}

我也坐了下来,因为我再也站不住了。他说:

“现在就这样了\ldots{}\ldots{}”

他又迟疑了一会儿,然后站起来,往前踏了一步,而我却动弹不得。一道黄色的闪光接近他的脚踝,有一阵子他待在原地不动。他没有尖叫,轻轻地倒下了,像一棵树那样,毫无声息地缓缓倒在一片沙地上。

\stoptitle

\starttitle[title={27}]

这一切发生在六年前,我从未向别人讲起这个故事。我的同伴都很高兴地发现我还活着,我却很难过,只是告诉他们:“我累了。”

渐渐地,我的忧伤稍稍缓解了。我知道他肯定也已回到自己那个小小的星球上,那是因为,天亮后我再也找不到他的躯体\ldots{}\ldots{}每天夜里,我总喜欢凝神聆听那五亿颗小铃铛似的可爱的星星的声音\ldots{}\ldots{}

然而,有一件极不寻常的事\ldots{}\ldots{}我帮小王子画口罩时,忘记了画一条带子。这样,他是永远也无法把口罩套到羊的嘴上的。所以,我不断地在想:他的星球上发生了什么事?羊是否已经把花儿给吃了\ldots{}\ldots{}

有时候,我对自己说:“当然不会!小王子每晚都把玫瑰花儿罩在玻璃罩里,而且他会小心翼翼地看着他的羊\ldots{}\ldots{}”然后,我就会觉得快乐了,所有的星星也都温柔地笑起来。

有时,我又会想:“万一有疏忽,后果就会很严重!如果,有一天他忘记帮花罩上玻璃罩,或者不小心让羊跑了出来\ldots{}\ldots{}”于是,所有的小铃铛就变成了眼泪\ldots{}\ldots{}

我们面临一个真正的难题,那就是:对于喜爱小王子的人们来说,世界的改变都维系在一件事上,那就是在某个地方,一个没有人知道的某个地方,会不会有一只我们不知道的羊,吃掉一朵玫瑰\ldots{}\ldots{}

抬头仰望天空,问问你自己:羊是否已经吃掉那朵花儿了?然后,你就会看到每件事都变了\ldots{}\ldots{}

没有大人会了解这是一件多么重要的事情!

对我来说,这是世界上最美丽、也最令人哀伤的情景。它与前一页画的是同一个地方,我再画一次,为的是让你看清楚。这儿就是小王子出现,同时也是他最后消失的地方。

仔细看这幅画,如果有一天到非洲旅行,你可以在沙漠中,再次认出这个地方来。如果你刚好经过那儿,请不要匆匆走过,请在这颗星星底下等待一会儿。如果出现了一个笑着的小人儿走向你,如果他有着一头金发,而且从不回答问题,你将会知道他是谁。然后,如果你心地善良,就不要让我活在悲惨之中!请立刻写信告诉我,告诉我:他回来了。

{\startalignment[center]
 \setupcaption[textfig][location={bottom,center},width=\textwidth]
 \placefigeasy[][imgs/小王子(艾柯譯)/figure_0131_0048.jpg][maxwidth=\textwidth,maxheight=\textheight,location={middle}]{安东尼·德·圣艾修伯里 \\ ANTOINE DE SAINT EXUPERY \\(1900.6.29-1944.7.31)}
 \stopalignment}

\stoptitle
\stopbodymatter

\starttitle[title={1}]

飞机飞入云层。

飞机上只有两个人,我身边的男人驾驶着这架老旧的侦察机,而我是不占空间的存在。

下面就是隆河河谷,今天和那天一样,晴空万里,艳阳高照,离开和相逢都是个好天气,他脸上一直带着微笑,看上去平静而美好,就像回到家乡,遥远星球上的小王子。

我看到窗外的白云,大朵大朵,厚重起伏。

他说的那天是1944年7月31日,距今天已有61年。

他的死,是法国文学史上的一个传奇。

我想告诉他,直到今天,人们依旧在探索他神秘的死亡,不时有人会声称打捞到飞机残骸什么的,渐渐不了了之。

最后,我依然选择沉默。

我现在可以看到这个侦察机的真身,它的机型和洛克希德P-38型闪电机相似。我这种军事盲会记得这么清楚,完全是因为这事和安东尼·德·圣艾修伯里有关。当人们关注喜欢的人的时候,会很自然地记住与他有关的细节。记忆会随着喜爱之心扩张或萎缩。

1944年7月31日,就在这里------地中海的上空,安东尼驾驶的侦察机被纳粹德军击中,坠海。

“那天我再次偏离了航道,大约一两分钟的时间,你知道,这玩意儿很过瘾。”他说:“当我体验到飞行带来的喜悦以后,我就无法停止。即使母亲非常不喜欢我从事如此危险的工作,我依然违背了她。我喜欢机械和飞行,是从小就有的爱好。”

安东尼·德·圣艾修伯里,世界上第一代飞行员,他的飞行技术和他的文字技巧一样高超。他热爱祖国,无惧死亡。然而,就在1944年6月29日,他在执行任务时擅自改变航道,飞越圣拉斐尔,到妹妹嘉布丽尔居住的亚盖城堡上空探视,为此,他归营后受到了惩戒处分。

“乡愁”------缠绕在安东尼心里的情结,当我再次翻开《小王子》时,我被书里浓溢的忧伤所打动。因为简单而充满智慧的小王子,他并不快乐,为了他遥远的玫瑰,还有,在地球的沙漠里看星空。星星闪亮耀眼,故乡却始终遥不可及。

“我们现在去圣摩里斯村------回到我的童年。”他说。

法国南部连绵的海岸线,蔚蓝的闪着光亮的海水,太阳仿佛整个儿融入到大海里。

我感觉,我可以肯定,飞机飞行的高度很低,绝对低于飞行安全标准的6000公尺,如同那日坠海前一样,安东尼想快点看到熟悉的童年王国。他是不愿意长大的,因此,不是我们的小王子不愿意长大,而是安东尼,他不愿意让小王子长大。

当孩提时的无限眼界变得狭隘,当他发现自己再也找不回心里真实的世界,长大于是便成了无可奈何的事,进入成人的世界,正是他对生命最无奈的妥协。

他给母亲的信中曾这样写道------“我不确定告别童年以后的我是否真的活过。”

这个浑身沁满了乡愁的男人,他对童年时光如此眷恋不舍。即使因此曾被德国的战斗机发现也依然如故。

\reference[Chapter2_2.xhtml]{}

\stoptitle

\starttitle[title={2}]

然后我们看见圣摩里斯城堡。

1900年的6月29日,安东尼出生于法国里昂市。这下我们可以理解为什么在1944年6月29日,他会违反部队的纪律去看自己的亲人了,那一天是他的生日。安东尼第一次踏足这个城堡,是在他出生后的第六周,他在家族的礼拜堂接受洗礼。他的家族一直保持着极虔诚的宗教信仰,这对安东尼的成长影响极为深远。

安东尼并不是父母唯一的孩子,他还有两个姐姐,一个弟弟和一个妹妹。大姐玛德莲,二姐西梦,弟弟方素华。嘉布丽尔,安东尼的小妹妹,1904年,她还在母亲肚子里的时候,他们的父亲尚·圣艾修伯里子爵过世。

城堡卧房的东面,是偌大的花园,花园里的菩提树是安东尼唯一的敌人,他对这些花过敏,因此每到开花季节,他就必须回到室内,安心又不甘心地读书。

{\startalignment[center]
 \placefigeasy[][imgs/小王子(艾柯譯)/figure_0136_0049.jpg][maxwidth=\textwidth,maxheight=\textheight,location={middle,none}]{}
 \stopalignment}
儿时的安东尼与兄弟姐妹(右二:安东尼)

相对于城堡的华宅正厅,孩子们更喜欢自在神秘的花园,这里才是他们的王国:菩提树和冷杉木林立的花园里,有柔软的草坪,灿烂的阳光,沐浴在阳光里的鲜花。他们在花园玩躲雨的游戏,最后一个躲雨的人会被封为“爱克林骑士”,这个封号会保留到第二次暴雨来临前。这种儿童的喜悦之心,安东尼曾把它写进《战斗的飞行员》里。长大后他在躲避雷雨时依然想起,根本不曾忘记。

城堡座落在茱拉山麓的普吉峰下,风景如画。而圣修伯里家族只在春、夏和初秋入住圣摩里斯城堡,因此安东尼的回忆多是温暖而柔嫩的颜色。柔软的大草坪,开满花的花园。安东尼的二姐西梦曾写了一本回忆录《花园里的五个孩子》。这样的名字,说明无论是安东尼还是他的兄弟姐妹,都对城堡的花园有浓厚的感情。他们因此记忆清晰。

他们在花园里饲养宠物,由性格柔和的大姐玛德莲来照顾。而安东尼最常做的事,是在母亲玛丽的身边画画、读书,听母亲讲家族的历史、先辈的光荣事迹以及圣经上的故事,并且对此百听不厌。

像童话里描写的一样,这座城堡里生活着很多人,有城堡的女主人(严厉而受人尊重的提考德伯爵夫人),神父(公正、学识渊博的蒙特梭),女管家(有爱心而又善良的茱喜),仆人领班(为了爱情多愁善感的西普林),厨子(经常因为来不及上菜怠慢贵宾而被伯爵夫人叱责)和一群忠于伯爵夫人的仆人。

现在小镇圣摩里斯因为安东尼·德·圣艾修伯里而声名鹊起,镇中的中央广场也因此更名为圣艾修伯里广场。这里和其他法国南部充满风情的小镇一样,随着时代进步而被开发出来,只有房子仍是明显的茱拉山麓建筑,保持着原来的风貌------屋顶非常陡峭,被涂成深灰色。这里依然是雪季漫长,积雪很深。季节成为唯一难以被时光动摇的东西。

安东尼小时候,圣摩里斯还是个与世隔绝的清幽小镇,交通闭塞,人们自给自足,几百年来,生活形态一成不变,他们已经习惯依循城堡天主的教导生活。在如此单纯的氛围里,安东尼即使很小也能感受到贵族世家的优越感。而后来,安东尼明白,这里所有的一切终将败亡,他可以听见时代如列车前进时发出的隆隆响声,强大到他无法装作不知,这使他感觉到矛盾和苦恼。

心底对旧日时光的眷恋,对家族荣耀的眷恋,心性里的天真完美和固执,这一切加深了安东尼对童年时代的追忆与怀念,对安东尼来说,长大成人是无可救赎的原罪。

\reference[Chapter2_3.xhtml]{}

\stoptitle

\starttitle[title={3}]

1904年3月,安东尼三岁多的时候,他的父亲尚·圣艾修伯里子爵在陪夫人玛丽回拉摩探亲时,在普罗旺斯的拉福车站突然中风,等到医生赶来时,已经不治而亡。当时的大报中,只有保守的天主教报纸《十字报》报导了这个消息,令人遗憾的是,它并没有提到死者的姓名,而是以方斯科伦布家族的女婿代替。

这无形中透露出一个信息------尚·圣艾修伯里生前并无卓越的成就,他曾经担任部队的军官和家族公司的业务员,是悠闲而有风度的绅士阶级。当世袭制度几乎被摧毁殆尽时,各大家族之间,旧贵族与新兴的工商阶级新贵之间,为了保护各自的利益而展现出前所未有的包容心和团结,他们通常选择用联姻的方式来巩固自己的势力。这种力量强大而顽固,多数人愿意服从,安东尼父母的结合正是属于这种情况。而安东尼自己却因为拒绝家族指派的婚姻,被视为叛逆者,直到他的作品获得了巨大成功,赢得崇高的声誉以后,才被家族重新接受认可。

尚在被派往乡下驻防时受到欢迎,33岁那年,他和当地另一贵族之女玛丽结婚,婚后即远离父兄搬离到南部的圣摩里斯生活。由此可以感觉到他性格里有隐忍坚强的一面。

{\startalignment[center]
 \placefigeasy[][imgs/小王子(艾柯譯)/figure_0140_0050.jpg][maxwidth=\textwidth,maxheight=\textheight,location={middle,none}]{}
 \stopalignment}
安东尼的母亲玛丽

{\startalignment[center]
 \placefigeasy[][imgs/小王子(艾柯譯)/figure_0140_0051.jpg][maxwidth=\textwidth,maxheight=\textheight,location={middle,none}]{}
 \stopalignment}
母亲玛丽、安东尼与弟弟方素华

{\startalignment[center]
 \placefigeasy[][imgs/小王子(艾柯譯)/figure_0140_0052.jpg][maxwidth=\textwidth,maxheight=\textheight,location={middle,none}]{}
 \stopalignment}
圣摩里斯

尚的一生波澜不惊,与妻子玛丽感情深厚,身后留下了五个孩子。他的人生还算完满,唯一的遗憾是,他过世得太早,使他的孩子们缺少父爱。三岁就失去父亲的安东尼,在未来成长的道路上缺乏一位有力的引导者------英雄的男性形象。这种心理上的隐疾在他成年后反而凸显出来,后来安东尼一度有忧郁症的倾向。

安东尼反省自己的成长历程,一切似乎和父亲没有直接的关联。他在《风沙星辰》里提到巴巴里酋长父亲的死,写道:“是他教我认识死亡,并使我不得不关注他的脸,因为他始终未曾阖眼。”这句话,已足以让人感觉到他内心对父爱的陌生和失落。

然而童年的安东尼并不孤独,他有相爱的兄弟姐妹,有关爱他的母亲和长辈,生活幸福无忧。尤其是他的母亲玛丽,她温柔善良,多才多艺,非常宠爱孩子,愿意倾听安东尼的想法,尽力去满足他。安东尼的一生都对母亲抱有热烈而持久的爱,从来不曾停歇。直到他中年,死前的前几日,在写给母亲的信里,依然这样要求------“亲吻我,如同我内心深处亲吻你一般。”

他在写给母亲的信中透露出,他非常需要母亲的爱和鼓励。我们可以感觉到安东尼心里一直保持着对童稚的渴望。如同他年幼时,喜欢叫醒兄弟姐妹们,让家人欣赏他的作品和表演。渴望得到家族成员的赞赏,这点成了他创作的动机,后来他的亲友都晓得,为了讨他欢心,即使批评也得小心措辞。

从现存的照片来看,当年的玛丽小姐年轻貌美,风情万种。她的娘家方斯科伦布家族是书香世家,自18世纪被封为贵族起,家族中出了好几位画家、音乐家、作家、艺术品收藏家以及科学家,在埃克斯地区留下了丰富的文化艺术遗产,而她的夫族圣艾修伯里家族则世代尚武。安东尼允文允武,或许正是由于他继承了父族与母族的双重熏陶。

父亲死后,母亲带着孩子们回到埃克斯的城堡投奔亲戚,就是先前提到的提考德伯爵夫人。安东尼此后一直在女性当家的大宅院里生活,直到他九岁。

安东尼是父亲的长子,自他懂事之日起,家族中人便一再给他灌输一个观念------你将来会成为一家之主。所以安东尼总是想尽力照顾家族中的每一位成员,履行好自己的责任。他的姐姐西梦回忆,自己1912年和安东尼去附近爬山,西梦弄丢了一只表,安东尼执意回去寻找,他走了很远的路,直到深夜才拖着疲惫的身影回家。因为没有找到表,安东尼很自责,他对自己让姐姐失望感到深深的惭愧。

强烈的信念衍生出的强烈责任感伴随了安东尼的一生,促使他性格的一个重要方面------固执形成。

我们看《小王子》,小王子去拜访的那些人:国王、爱慕虚荣的人、酒鬼、商人、点灯的灯夫、地理学家,他们形形色色,外貌、习惯、职业、生活的环境各不相同,然而有一点他们是相同的,这是一群固执的人。

创作《小王子》时,安东尼流亡于美国,人生陷入低潮,而且进退维谷。于是在这本书里,他回归童年之心,重新看待曾经有过的执著和烦恼,获得前所未有的平静和智慧,书中这些人,除了代表人世间不同性格的人之外,也是安东尼对自己精神世界的透视和深省。

安东尼的祖父费南·圣艾修伯里是《小王子》里地理学家的原型。他墨守成规,不肯改变自己的想法,思想上的保守使这位固执的老人无法接受媳妇的做法:鼓励孩子们追求自我,宽以待人,文化生活与宗教信仰并重兼修。圣艾修伯里家族是彻底的教会支持者和保皇党。

安东尼九岁时,家族认为他必须接受严格的教育,为将来做好一家之主铺路。当时的最好选择是继承家族的传统投身军旅,在这种严格限制下,祖父和叔叔罗杰替安东尼选择了勒曼的一家教会学校就读,对于此事,安东尼最亲爱的母亲无法发表意见,她成了外人。

法国的法律认为,感性的需求并不重要,妇女没有能力单独做出决定,若出现宗教、心灵层次的问题,安东尼必须遵从叔叔罗杰的指示,他实际上的监护人是以祖父为首的圣艾修伯里宗亲。

于是,安东尼在一夕之间被带离圣摩里斯城堡,告别了阴柔甜美的女性世界,来到法国西面的勒曼市和祖父生活在一起,进入一个以严苛出名的教会学校,纯男性的世界。

“我并不喜欢这里,”安东尼说,“学校的气氛刻板严肃,就像,你知道,对!那只被拴起来的小羊,对!我们就是一大群羊,而神父负责驯养我们。在那里,我没有一点笑容,我的学习并不优秀,无论班级人数多少,我总是倒数几名。学习中有很多是我不感兴趣的科目。我最拿手的科目是法文,但在老师眼中依然很差。”

我们看到了勒曼市的圣十字圣母学校。从1909年10月7日到1915年转到瑞士弗莱堡的圣约翰学校,安东尼在这里度过了并不快乐的四年。

首先是离开母亲的孤单,母亲除了在他入学时帮他办理入学证明时来过,多数时间与他分离。安东尼是个纤细敏感的人,被迫离开圣摩里斯充满欢乐的大家庭之后,他一直落落寡欢。其次是学校令人难以忍受的严苛环境,学校施行全军事化教育,其严苛程度和正规的军队训练一般无二。为了避免受到惩罚,年幼的孩子们必须学会忍受寒冷、冻疮的折磨,眼睁睁地看着午餐冷掉,暖瓶里的水结成冰块,而当时安东尼只有九岁。

我知道安东尼没有骗我,在一些珍贵的照片里,他和同学们全都表情严肃,丧失了甜美的想象和笑容。他们是一群被选择的人。这些孩子背负着大人的希望,家族的理想,追随着前辈的足迹来到这里,然而对多数人来说,只是迫不得已的接受,并不是因为喜欢或者希望。

“我记得一个人,他待我不同。”安东尼说,接着他微笑起来,温柔将眼中的忧伤之色覆盖,很快,他充满感情地回忆起奥古斯汀·劳南神父。

那是一个表情严肃,留着平头的男人,他总穿着黑衣,对待学生非常严格,他的学生叫他------“恺撒”,这个绰号无损安东尼对他的温柔回忆。在学校里,虽然别的老师也表扬过安东尼的作文,但只有劳南神父一直把安东尼的作文当做法文课的范本。他非常欣赏安东尼的文章。1931年,当安东尼以《夜航》获得生平第一座文学奖时,他专程回到勒曼,感谢劳南神父当年的鼓励和赏识,而在此之前,从1915年离开勒曼以后,安东尼就没有再回去过。

对当年的安东尼来说,求学的经历实在不堪回首,只有劳南神父的鼓励让他倍受感动。1913至1914年间,12到14岁,是安东尼的叛逆期,他和祖父的关系变得紧张。尽管他曾把祖父当成绝对的权威,但随着年龄的增长,安东尼越来越不喜欢祖父的专制。

远离母亲和祖父生活在一起的日子,安东尼是苦闷的。他只能用消极的方式抵抗,学习成绩越来越差,接近退学的边缘。只有作文课是他喜欢的,无人可倾诉,纸笔便成了他心灵的避风港,得到别人的鼓励和赏识也因此显得重要。

在此期间,安东尼写得最好的文章是《漂泊万里的帽子》,13岁的他已经懂得发挥想象力,充分利用自己的知识来开展故事,他描写了一顶大礼帽从风光无限到最后沦落非洲成为酋长的头巾的传奇经历,这篇作文超过一千字,叙述有致,笔法老道,几乎就像是大人写的。

而劳南神父收藏的安东尼的另一篇作文《蚂蚁的葬礼》已经遗失,只有同学记得零星大概:“送葬的队伍被水滴挡住去路,公蚁用绿草造了一座便桥。”这篇文章里安东尼已经透露出对机械发明的浓厚兴趣。

“在这个让人不快的地方,我获得了一些宝贵的经验:第一,即使身处逆境也会出现患难与共的好友。永远不要对自己的处境绝望;第二,你必须学会忍受别人的嘲笑,我指的是身体。他们与生俱来,不可更改,比如我的大脚丫子和翘鼻子。”在我们将要离开勒曼的时候,安东尼说。

是的,安东尼在学校里获得了两个奇怪的绰号:“大鞋子”和“圆月弯刀”,分别是针对他的大脚丫子和奇怪的往上翘的鼻子。对此他并不喜欢,只是后来不显露出自己的不快乐罢了。

只是有一日我们学会宽容,因此能够平静接受。事实上没有人喜欢被取笑,无论善意还是恶意,如花被修剪或攀折,一样都是伤害。

必须要提及的是安东尼的母亲玛丽,在安东尼最孤独的时候,她的心一直未曾离开过他,当安东尼因为学习成绩太差而遭到学校责罚时,她曾不止一次写信给学校,请求取消对孩子留校查看的处罚。她对孩子倾尽母爱,简直无微不至。

安东尼感激母亲对他的关爱,玛丽成为他心里完美无瑕的女性形象,直到他成年后,爱慕和寻找的一直是像玛丽这样温柔善良的女子。

\reference[Chapter2_4.xhtml]{}

\stoptitle

\starttitle[title={4}]

卡拉斯,希腊神话里用蜡将鸟的羽翼沾在双肩上的人,与父逃亡,因为飞近太阳,蜡融翼落,坠海而死。

传说,很久以前,人是可以飞的,但是人类狡猾而奸诈,神担心如果人拥有这最后一种能力,有一天连他都无法制约,所以便收回收这种能力,而把它赋予弱小的鸟类。

从此之后,鸟翱翔于天,而人只能仰望。神要人学会仰望。

但是,千万年以后,人类终于征服了属于神的天空。1903年,美国的莱特兄弟缔造了人类飞行于空中的传奇,也因此在世界范围内掀起巨大的飞行热。无数青年为此热血沸腾,人生为此改变,甚至付出生命也在所不惜。

安东尼·德·圣艾修伯里,他也是生活在那个年代,卷入那场巨大漩涡的人。因为是心甘情愿地沉溺,即使死亡也无须被拯救。

我看着像孩子一样欢欣喜悦的安东尼,他驾驶飞机,如同摆弄玩具,飞机的确是他的玩具。我们成年后对所做的事情缺乏耐力和决心,往往是因为我们已经失去了对它的喜爱和渴望,你可曾像孩子渴求玩具一样渴求你的工作?

你工作,只是为了生存或者需要,并不会对它有眷恋。

可是,这个男人,他始终保持着对飞行的喜爱,他爱它多年如一。时间让他学会冷静飞行,他不再兴奋和激动。但是现在你坐在他身边,你靠近他,依然可以感觉到他的幸福和喜悦。

安东尼第一次飞行的具体日期已无从考证,但应该是在他12岁时无疑。当时的飞行非常危险,报纸上每天都会报道飞行失事的消息,这也是玛丽一直不希望儿子成为飞行员的重要原因。但是,在12岁那年,尽管知道这么做很危险,而且会受到大人的斥责和惩罚,他还是坐进了驾驶舱。

让安东尼第一次尝到飞行快感的是飞行员加伯列·罗布鲁斯基,为了隐瞒波兰籍身份,他化名萨维兹。他当时只有23岁,和他的哥哥皮耶都是热爱飞行的天才。兄弟俩共拥有四架单羽机,而安东尼乘坐的是第三架。两年以后,他们的飞机坠毁,当时的安东尼在勒曼上学,可是,安东尼对这两兄弟的印象非常深刻。他非常难过。为此,他善解人意的母亲写了一封慰问信给罗布鲁斯基夫人,说:“您的两位公子对小儿极为亲切,这起不幸的意外,让他感到非常难过。”另外还随信附上一封安东尼签名的慰问信,并提出请神父为两兄弟举行弥撒。

飞机继续飞行,安东尼表示要带我去瑞士弗莱堡的圣约翰学校。我看见他的脸,神色温柔安静,飞行时对死亡的恐惧已经消失,我在想,那日坠机前,他是否也是这样平静?

“我已经准备好迎接死亡,并且我心甘情愿接受死亡。”安东尼在坠机前就对朋友这样说。这是安东尼的各种个人简历中总会被人浓墨重彩地强调的一笔。以等待故友的心情迎接死亡,自然表现出安东尼的豁达和平静,但是因何会如此?

并不是安东尼生来便无惧生死,他是从弟弟方素华的死亡中学会了接受命运。1917年,安东尼17岁时,弟弟方素华因为急性关节炎去世,在去世前的20分钟里,方素华对哥哥说他快要死了,但没有一丝痛苦,他立了遗嘱,把自己心爱的物品交给哥哥安东尼。

方素华的死,改变了安东尼对生死的看法,他开始相信宿命。1940年,安东尼的飞机受到飞弹攻击,生死一线之际,他想起方素华最后说的几句话:“没办法,我就是快死了,什么时候离开人世不是我能决定的,这是身体决定的。”

由最亲近的人身上目睹死亡,了解躯体只是暂时寄居于世,安东尼仿若新生。不朽就是朽,不死就是死,永生存在于肉体之内,不死也可以死去。

安东尼的飞机降落在瑞士,我们来到弗莱堡的圣约翰学校。1915年,安东尼罹患了严重的贫血症,此时一战已经全面展开,安东尼的叔叔罗杰战死沙场,法国已经无法保证安东尼的安全,为了保证圣艾修伯里家族七百多年的血脉不断,祖父决定让安东尼从勒曼退学,改上另一所天主教学校。

在安东尼进入圣约翰之前,母亲已经为他做了多方打探,家族经过慎重讨论才决定让安东尼和弟弟方素华就读于此。1915年至1917年,安东尼在这里住读了两年。

圣约翰与圣十字圣母学校完全不同,它风景优美,学风开放,老师多鼓励学生参加运动,循循善诱,不厌其烦。尽管安东尼的学习成绩仍然不是太好,可是他的身心却得到了全面的放松,他渐渐摆脱了耶稣会的阴影和祖父的控制。在勒曼时的满脸忧郁已经被阳光灿烂的笑容代替,直至成年成名以后,安东尼仍不无感激地回忆起圣约翰学校老师对自己的教导,称他们无怨无悔,除了获得服务别人的满足感之外,别无所求。

与对勒曼学校的森冷印象不同,安东尼对圣约翰学校满心怀念。在他的第一本小说《南方邮件》里,他就借主人翁的身份回到学校,缅怀了过去的美好时光。

如果说,在圣约翰还有什么伤心记忆的话,那应该就是弟弟方素华参加学校的旅行,丢失了外套,由此患上急性关节炎,最后因此病而去世。

弟弟的死,塑造了安东尼的宿命论,然而他可以接受命运,却无法排遣忧伤。弟弟的死,完全抹杀了安东尼在圣约翰两年的快乐时光,他需要用一辈子的时间来疗伤。

\reference[Chapter2_5.xhtml]{}

\stoptitle

\starttitle[title={5}]

1914年第一次世界大战爆发,当时安东尼尚小,只有14岁。为了安全起见,家人很快将他转入瑞士弗莱堡的圣约翰学校。但是到了1917年,战火仍在持续,德国重新占了上风,战争比众人预计得更为漫长。此时安东尼已经17岁,在爱国激情的感召下,他和其他的热血青年一样,准备随时响应国家的征召,披甲上阵。

圣艾修伯里家族有尚武报国的光荣传统,安东尼的叔叔罗杰曾担任法陆军步兵团的军官,而且家族中进入陆军的人数较多,如此,安东尼也势必应该追随长辈们的脚步加入步兵团。但是,家族宗亲经商议,最终决定将安东尼送入海军服役。因为海军的伤亡率最低,并且海军一直是最坚定的保皇势力,此种决定和当年他们决定将安东尼送入勒曼的教会学校读书时,动机如出一辙。

于是,你该猜到我们将去哪里?不错。我们的下一个目的地是位于巴黎圣米榭大道的圣路易预科学校。

“你在这里将看到我是如何散漫、放荡不羁,我依旧不是个好学生。”安东尼说,“这是我第一次尝到巴黎浮华生活的滋味,当时,即使是处于德军疲劳轰炸的危险中,巴黎依然是夜夜笙歌,歌舞升平。”

在此期间,安东尼的不良行为包括:喜欢搞恶作剧,放鞭炮扰乱课堂,打电话给消防队谎称学校着火,行为放荡并四处结交女友。

有一次我从学校的排水沟中爬到校外,险些被逮个正着\ldots{}\ldots{}安东尼笑起来,显然,回忆这段往事让他想起自己曾经年少轻狂。从那次以后,安东尼就想自己在校外租住公寓。这样行动会更加自由,生活上的各种需求也能充分满足。

在巴黎生活的那段时间,安东尼由祖父手中继承遗产,母亲也有产业,他同时受到亲友们的照顾,生活无忧,自由挥霍。安东尼表示,即使德军攻占巴黎,他也不打算离开。

然而像安东尼一样的贵族子弟们并没有等到为国效力的机会。1918年11月11日,一战结束,这使很多心怀壮志的年轻人感到失望。他们并未因和平的到来而深感欣慰,反倒是对不能一圆英雄梦感到无比遗憾。安东尼正是如此,所以二战时,他积极投身到抗击纳粹的战斗中。

{\startalignment[center]
 \placefigeasy[][imgs/小王子(艾柯譯)/figure_0156_0053.jpg][maxwidth=\textwidth,maxheight=\textheight,location={middle,none}]{}
 \stopalignment}
安东尼与飞机同行

与大战擦身而过,使得安东尼对加入海军越来越没兴趣。1919年6月,他未能通过海军的入学考试。但是,他对此毫不在乎。巧合的是,一战结束后,安东尼生命中最重要的两位保皇派人士相继辞世,祖父逝于1919年,提考德伯爵夫人逝于1920年。这两位家族核心人物的去世,让安东尼获得更多自由,他后来加入空军。他一直向往飞向蓝天。

“遇上苏多神父是我的幸运,他是一位传奇英雄,曾担任陆军的随军神父驻守战场,并在战壕里对着德军唱圣诞歌,而德军的神父以德语回应,那是1917年的圣诞节。我想,每个人都厌恶战争,渴望和平,渴望与家人团聚,拥有礼物。”在离开圣路易预科学校时,安东尼说。

我们必须提及墨里斯·苏多神父。多年以来,他一直是安东尼精神上的知己。安东尼的一生中,影响其人格发展的神父有好几位,他即是其中之一。在圣路易预科学校寄宿时,安东尼渐渐失去信仰,他向神父告解,说他对教义产生强烈质疑并怀疑夺走弟弟的上帝,苏多神父细心听他倾诉,并引导他趋于平静。不仅如此,1926年,他还帮安东尼找到最喜爱的飞行员的工作,帮助他摆脱平庸乏味的生活。

{\startalignment[center]
 \placefigeasy[][imgs/小王子(艾柯譯)/figure_0157_0054.jpg][maxwidth=\textwidth,maxheight=\textheight,location={middle,none}]{}
 \stopalignment}
安东尼与战友

即使离开学校,安东尼也一直与苏多神父保持着深厚的感情。1931年,安东尼与康苏珞结婚,他请求神父担任他的证婚人。

\reference[Chapter2_6.xhtml]{}

\stoptitle

\starttitle[title={6}]

我回忆与安东尼在一起的情景,多数时间我们是在不同地点之间辗转,如鸟儿迁徙,飞于空中。

我当庆幸与他的相遇是在今时今日。如果是在几十年前,我们这次旅程毫无安全可言。

安东尼提出要带我去他正式学习飞行的地方,我同意了。因为除了身边的这个男人,一架毫无先进性可言的侦察机,身边绵延的云层以及穿透云层的阳光,我发现我已与这个世界毫无联系,听天由命。所幸不久我们即安全到达史特拉斯堡的纽霍夫机场。

1921年4月9日,安东尼应征入伍,军阶是二等兵,担任地勤修护人员,此时,他仍与飞行无缘。军队规定飞行官必须在军官学校签定三年服役期才能进行飞行任务,显然安东尼并不符合这一条件。不过天无绝人之路,这个滑头的小子很快发现制度存在漏洞------当时的规定不严,只要拥有民航机的驾照,即使是二等兵也能参加飞行训练。安东尼决定在报到之前实现他的计划。

在入伍的前两个月,我前往史特拉斯堡的纽霍夫机场,找到一位东方航空公司的飞行员,他叫罗勃·埃比,是前德军飞行员,经验丰富,做事谨慎。我获得上级指挥官的默许,缴了2000法郎的学费,拜埃比为师,可怜的他对此毫不知情,我的家人也将这件事称为“史特拉斯堡密谋”。多年之后,安东尼提起自己的教练,仍是充满尊敬之心和幽默感。

安东尼首次单飞即遭遇险情,但他沉着应对的不俗表现,令埃比对他赞赏有加。埃比对徒弟的评价是:能力过人,反应敏捷,判断果决。这些说明安东尼的确具备成为优秀飞行员的素质,但在史特拉斯堡时,安东尼的某些行为暴露了他一个一生为之困扰的毛病,安东尼一生都在追求独立自主和享受他的贵族背景带来的好处之间挣扎,他的性格充满矛盾。

埃比对安东尼的某些少爷行径不敢苟同,初次见面,埃比就被他的不修边幅吓坏了!服役期间,安东尼要求母亲给他寄很多零用钱,在史特拉斯堡租住单独的公寓,他学开飞机,偶尔出勤,生活得舒适自在。而当时的二等兵都挤在军队安排的小小公寓里。此外,安东尼上课时不愿帮忙处理例行事务,有时还忍不住利用头衔和关系让自己在空军里占点便宜。埃比称呼他为“伯爵少爷”。

“我是于1921年8月2日开始自己的飞行事业的,我记得很清楚。”安东尼说:“12月23日我通过考核,成为正式飞行官。如果没有埃比的指导和训练,我不可能实现飞向蓝天的梦想。”

“你的母亲呢?她一如既往地支持你吗?”我问,我想起玛丽,安东尼柔弱坚毅的母亲,她应该是非常不安的,但是她如同圣母一样慈和,有求必应。

果然,安东尼告诉我,玛丽答应了他的要求,强忍着内心的痛苦和不安帮助儿子实现理想,她汇钱到史特拉斯堡帮儿子支付学费,换来了20年担惊受怕的日子。

1922年10月20日,安东尼晋升少尉。但是在1923年1月,安东尼在布杰坠机险些丧命。康复以后,他本想与空军签约,成为一名职业军官,但是爱情的突然光临,路易斯·薇摩韩的出现,使得安东尼改变了人生计划。

\reference[Chapter2_7.xhtml]{}

\stoptitle

\starttitle[title={7}]

安东尼一生的最后一部作品《小王子》是献给妻子康苏珞的情书,而他的第一本书《南方邮件》则是献给未婚妻路易斯·薇摩韩的爱的礼物。

安东尼第一次遇见路易斯是战后待在波许维的那段日子,在贵族的聚会中,他认识了美丽活泼的路易斯。路易斯的性格可爱迷人,拥有无数追求者,她的才华及美貌令安东尼深深倾倒。此后两人订婚,但很快分手。安东尼为之伤心了很久。除康苏珞之外,路易斯是安东尼一生中另一位为之钟情的女性。

路易斯·薇摩韩,《南方邮件》女主角的原型。如果要谈谈安东尼第一次失败的婚约,他与路易斯之间纠缠反复的恋情,我们势必要提及这部小说。

解除婚约六年后,安东尼带着《南方邮件》的初稿来到路易斯位于巴黎的家中,因为路易斯外出,安东尼留了信给她,信中提到《南方邮件》这本书,他带着道歉的口吻写道:“我想为你写点东西,想跟你讨论这本书,如果你觉得可行的话,我或许可以将这本书献给你。”他觉得当年的自己,害羞而不知所谓,令路易斯感到失望,对此他很抱歉。然而事实并非如此,路易斯也是爱好文艺的才女,她留下的很多文章显示,她对安东尼并没有安东尼对她那样情深难忘,即使谈及也只是轻描淡写地一带而过。

她甚至将安东尼最重要的礼物------《南方邮件》的原始手稿出售,至于安东尼送给她的订婚戒指,也被她送人,对于路易斯来说,安东尼只是她生命的过客,两人有没有结果都无关紧要。

安东尼曾借《南方邮件》替自己疗伤,书中糅合了浓烈的爱情和悲壮的历险故事,男主角空中英雄伯尼是他的化身,而娇生惯养的女主角珍妮维则是路易斯。安东尼深爱路易斯,可惜,他并不知道两人无疾而终的原因何在。即使他深入探讨了自己的感情生活,做了彻底的反省,他还是不知道分手的真正原因是------他和路易斯的性格南辕北辙。

路易斯喜欢安逸的生活,热闹频繁地交际,喜欢像猫儿一样被娇宠,待在有暖炉和薰鱼的房子里。而安东尼渴望飞行,喜欢待在空旷沙漠里的寂静。除此之外,路易斯只当他是普通朋友,订婚也是她一时兴起闹着玩的,而薇摩韩家也认为挥霍无度的路易斯和花钱如流水的安东尼不是好的伴侣。

尽管分手,安东尼对路易斯的爱意依然丝毫未减。“我想你心里很清楚,你可以要求我做任何事,无论牺牲多大,无论何时,我都愿意。我写这段话并不是要你感激我,因为你永远不会真的需要我,但是在你孤单寂寞时,想想我这段话,也许能够减少孤单的感觉。虽然你伤了我的心,但我原谅你的所作所为。”他这样表白。

我之所以写出这段话,是想让人感受到安东尼的善良和忧郁。我们的小王子,即使他受了再大的委屈也总是能够体谅别人。

“我是男人,你是女人。”后来,他对路易斯说。他明白,相遇太早,两人都不懂事,一切还来不及发生。

\reference[Chapter2_8.xhtml]{}

\stoptitle

\starttitle[title={8}]

1933年,安东尼从空军部队退役,为了获得伊人芳心,他没有继续和军队签约,而是去做了一名办公族,从事办公室内勤工作。他希望这份象征安稳生活的工作,能让路易斯对他有所改观,拉近彼此的距离。然而事实上爱情无法勉强,无论安东尼如何努力,1933年秋天,路易斯还是不辞而别。安东尼伤心之余,开始写第一本小说,祭奠自己的爱情。

此时安东尼的创作观开始改变,他讨厌虚无缥缈、言之无物的文学作品,不再写浪漫诗篇,也不再刻意讲究文体、格式等写作理论。

“我认为作家笔下必须言之有物,否则就别写了。我渴望观察、诠释真实的生活,就像驾驶飞机从天空中鸟瞰大地一样,心境应如蓝天一样广阔,而思想应如无垠的大地,起伏连绵,辽阔深远。”安东尼说。

不久,他决定将我带离巴黎,也带离他生命中那段枯燥乏味的生活。但在离开之前,他回忆了在浪漫之都平静刻板的工作和生活的细节。

{\startalignment[center]
 \placefigeasy[][imgs/小王子(艾柯譯)/figure_0165_0055.jpg][maxwidth=\textwidth,maxheight=\textheight,location={middle,none}]{}
 \stopalignment}
安东尼在美国广播电台播出文稿

后来,安东尼换了一份汽车推销员的工作,需要到乡下开拓业务。但是苦闷依旧,他用两句话形容这样的生活:“我的生活充满了蜿蜒小路,必须加速离开;日子就在一家家歪歪斜斜的旅店之间悄然溜走。我觉得心情很差。”

唯一值得开心的是,工作之余,安东尼开始学习当一名作家。他开始和一些文学界的朋友接触,探讨文学,这些人后来给了他很大帮助,使得《南方邮件》能够顺利出版。安东尼并没有虚度光阴,到1926年4月,原本只见大纲的《南方邮件》已经成为一本短篇小说。1929年4月,这部小说正式出版。

敏感的安东尼习惯向家人和朋友倾诉痛苦,母亲由此得知儿子生活苦闷。她竭尽所能帮他摆脱现状,再也顾不得自己的担心,转而支持孩子的理想------成为一名飞行员。玛丽亲自出马请苏多神父帮忙。神父与拉特柯尔航空公司的总经理贝波·马西迷交情深厚,在苏多神父的帮助下,安东尼获得面试的机会,马西迷原本只想让安东尼处理行政业务,但安东尼坚持从事危险性极高的飞行工作,马西迷只好答应帮忙。

马西迷打电话给公司的业务经理迪狄耶·道哈特,道哈特有权挑选飞行员,他坚守原则、不容妥协的领导风格让安东尼深深着迷,他是安东尼心中景仰的英雄之一,日后成为安东尼的小说《夜航》中的英雄李维耶的原型。另一个人让安东尼景仰的英雄是亨利·居劳梅。和这些刚毅果敢的英雄们在一起生活工作,不但实现了安东尼的理想,也塑造了他坚毅不屈的性格。

{\startalignment[center]
 \placefigeasy[][imgs/小王子(艾柯譯)/figure_0166_0056.jpg][maxwidth=\textwidth,maxheight=\textheight,location={middle,none}]{}
 \stopalignment}
不同版本的《夜航》

在正式驾驶飞机之前,必须先做单调的调试工作,当时的飞机故障频出,飞行员必须仔细研究飞机的每个部件,对此安东尼毫无怨言,乐在其中,他感觉自己离理想越来越近。生活中唯一让他难过不安的是对母亲的歉意。他晓得母亲现在深感孤寂,无人陪伴。

大姐马德莲从小为癫痫所苦,此时已经病逝。西梦只身前往西南半岛,从事图书管理工作。小妹妹嘉布丽尔三年前结婚,搬到地中海边的亚盖城堡。母亲现在孤单一人。

安东尼有意请母亲来度假,不过玛丽更想专心地投入慈善工作以及整理关于孩子的回忆。安东尼对一家人的离散感到怅然若失。童年的时光一去不回,幸而此时有飞行可以弥补遗憾。自从当上了空中邮递员,安东尼犹如重获新生。

\reference[Chapter2_9.xhtml]{}

\stoptitle

\starttitle[title={9}]

夜间飞行,如同渡过苍茫大海,城市是岛屿,而我们无法降落,那里并非着陆点。他带我穿越无边繁华,黑暗中只有微弱月光相伴。而后我们看见满天星斗,清亮夺目到无法言喻。月光皎洁,沙子看上去像碎钻。

到达撒哈拉沙漠,我对安东尼表示由衷的感谢,感谢他带我来这儿。从我看过《小王子》就一直向往于此。我们降落在一个高原。远古时代的撒哈拉沙漠是海床,这里堆积了一山的贝壳。若你看过高原,你将不愿离开。

高原上有很多黑色石头,在月光下莹然生辉。安东尼拾起一块石头,坐下来,说:“它们是几万年前掉落到地球上的陨石,它们给了我创作小王子的灵感。小王子是天外来客,这些陨石也是。它们远离故乡,终身怀想。”

安东尼的诉说仍在继续\ldots{}\ldots{}

1928年,安东尼调至珠壁角基地任经理。这是一块荒凉的不毛之地,只有西班牙驻军和强悍、难以驯服的沙漠族人------摩尔族在此落脚。除此之外,仅剩满目风沙与人做伴。

当时,法国与西班牙争夺西非地区的控制权,斗争如火如荼地由地面延续到天空,开辟空中航线成为焦点。拉特柯尔一方面与西班牙政府斡旋,一方面将安东尼派往珠壁角,期望他与当地的西班牙总督搞好关系。虽然安东尼不懂西班牙语,但公司认为他的贵族背景有利于交涉。对此安东尼并无怨言,虽然他的很多朋友及伙伴都反映他个性散漫,但安东尼其实是个很遵从领导安排的人。基地经理的责任之一是救援飞行员,安东尼生平第一次展示了自己的领导才能。

“我干得还不赖。”安东尼笑得很可爱,因为回忆起令人开心的往事而眉飞色舞。“我曾带人去‘敌人的地盘'夺回失事的飞机。当时阿拉伯游击队就在附近,子弹从头顶上咻咻地飞过,我在枪林弹雨中昂首阔步,觉得自己勇敢极了。但后来,我明白真正的勇气并不是莽撞任性,而是面临危险时的永不绝望。”

另一次为同僚称赞的救援行动历时三个月,安东尼寻找两名失踪的飞行员汉尼和塞赫,过程充分展现了安东尼的救援技术、毅力和交涉能力,远远超过众人原本的期待。

珠壁角荒凉寂寞。但安东尼富于童心和幻想力,他为人随和开朗,与这里的西班牙驻军、孩子以及酋长们相处融洽。他常给孩子们巧克力和糖果,孩子们称他为“鸟群酋长”。安东尼还邀请过几位酋长上天旅行,除了减少他们的敌意之外,也借此压压这些地头蛇的威风。

“我好比在‘驯服'他们。”安东尼笑着说。谈到驯服,安东尼想起他在沙漠里邂逅的那只体形比猫还小的小狐狸。安东尼说:“我非常想和它做朋友,但这个小家伙野性难驯,吼叫声像狮子。”在《小王子》里,这只耳朵又尖又长的小狐狸也现了身。狐狸出现时的主题是:两个人各自希望被对方驯服,由此衍生出感情。

{\startalignment[center]
 \placefigeasy[][imgs/小王子(艾柯譯)/figure_0170_0057.jpg][maxwidth=\textwidth,maxheight=\textheight,location={middle,none}]{}
 \stopalignment}
《小王子》手稿

头顶星河斑斓,我们坐在沙漠里,看见一颗颗流星划过天际,缤纷如雨,心中的喜悦惊叹到无法言说。神奇的沙漠夜空让人心醉神迷,我和安东尼当年一样,舍不得离开,唯愿时间停伫千年。

撒哈拉沙漠生活清苦,然而却让人备感幸福,荒漠生活的洁净无瑕将人带离尘世,靠近遥远的童年时光。1928年离开此地前,安东尼如同离开天堂。他预感悲伤会接踵而至。此后他一直想念在这里的生活,于书中不断回忆、描摹。

{\startalignment[center]
 \placefigeasy[][imgs/小王子(艾柯譯)/figure_0171_0058.jpg][maxwidth=\textwidth,maxheight=\textheight,location={middle,none}]{}
 \stopalignment}

\reference[Chapter2_10.xhtml]{}

\stoptitle

\starttitle[title={10}]

1928年安东尼回到法国,在西非的出色表现使得他再次被委以重任------派往布宜诺斯艾利斯担任运输经理,开辟南美航线,这次风险更大。1929年春天,公司先将他送往不列塔尼半岛列斯特港的海军空中导航学校,接受李纳马克斯·夏辛开办的进修课程。安东尼的理论学科成绩优异,但是实物演练却屡犯严重错误。此时安东尼成绩一般的原因主要是,挂念即将出版的新书《南方邮件》,他常在夜间校稿,导致白天上课分神。

安东尼拿到了及格分数,可惜在国防部的强力干涉下,并未获得进阶导航证书。督察认为,如果每个学生都顺利过关,证书将失去价值,安东尼惨遭淘汰。

尽管如此,夏辛对安东尼的评价甚高,他觉得安东尼的数学非常好,而且能不断想到新点子改进导航工作,这期间的不少发明日后还获得了专利。

1929年10月,安东尼到布宜诺斯艾利斯报到,他的工作包括负责开拓世界最南端的飞行路线------巴塔哥尼亚与赋格山的定期航线。安东尼在南美的飞行时数超过非洲,情况也更为惊险,他将写作搁置一边,与心中的英雄居劳梅、梅莫兹一起并肩作战,专心开辟南美航线。他们同历甘苦,日后安东尼仍不断回味大家在一起的纯洁情谊。他是个非常恋旧的人。

“在离开阿根廷之前,我遇上了康苏珞。至今我仍深爱着她,即使我将消失于这世间时,她也不会从我心中凋零。我记得她乌黑、深邃的双眼,迷人的笑容。她声音柔细,我渴望保护她。”安东尼默默地说。飞机在布宜诺斯艾利斯的上空低低盘旋,久久不愿离去。

安东尼与妻子康苏珞·索馨·圣多瓦邂逅于1930年,他们在一场专门招待法国作家和艺术家的演讲会上相识。不得不承认,康苏珞,这个身材娇小的南美美女具有惊人的魅力,她活力四射,多才善谈,无论在何地,她都能获得关注及喜爱,无数男士拜倒在她的石榴裙下。

安东尼此时生活正渐有起色,在非洲及南美的杰出表现,为他赢得“无所畏惧”的美名。飞行者的传奇经历,作家以及贵族的头衔,使得安东尼很能获得女子的芳心,对康苏珞也不例外。不过他追求康苏珞的过程并不顺利,因为康苏珞并不是寻常女子,她经历丰富,在遇上安东尼之前,已经有过两次婚姻。遇见安东尼时,她顶着名作家葛梅兹·卡利约遗孀的身份出现在高级交际圈里,如鱼得水。无疑,她是一个非常独立,个性复杂的迷人女子。

两个人的性格差异造就了驯服的欲望,爱情随之衍生,安东尼在康苏珞的爱中孕育出20世纪最伟大的作品《小王子》,并在康苏珞的鼓励和支持下写出《风沙星辰》《夜航》等著作,成为法国最著名的作家。但更多的时候两个人都被爱恨交织的感情所折磨。深深相爱的两个人免不了被彼此坚硬的性情刺伤,在十几年的婚姻生活中各自尝尽人生百味。

“我初次见到康苏珞时,并未赶上去赞美一番,我变得拙嘴笨舌,说的话听起来好像在抱怨她太矮太瘦,天知道,其实她是那样娇小可人,令人着迷。”安东尼笑起来,闭着眼靠在椅背上吁了口气,然后睁开眼接着说:“那次她吻我的时候,就在你现在坐的位置上。她安慰我说,我一点也不丑。而在此之前,我跟她道歉说,我是一只大笨熊。我觉得自己很丑,当然,这只是我的小伎俩,我经常利用她心软的弱点,占点小便宜。”

我看着安东尼,突然觉得自己看到的是个调皮滑头的大男孩。

为了弥补康苏珞,安东尼决定邀请她到布宜诺斯艾利斯的天空兜风。在飞机上,这个家伙突然命令康苏珞吻他。康苏珞拒绝,理由是西班牙女子只吻心上人,安东尼听了脸色一沉,马上熄掉引擎,宣布他将伤心地坠入海中,他的脸上挂着两行泪痕,楚楚可怜的样子使得康苏珞于心不忍,一时心软,在他的脸上亲了一下。

安东尼因为这一吻而疯狂地爱上康苏珞。但康苏珞依然拒绝他,她说自己更愿意保留个人的生活空间。回到法国不久,她甚至告诉朋友这段恋情已经结束,后来的生活显示,康苏珞的顾虑非常正确。虽然安东尼是她见过的最体贴的男人,但安东尼也有霸道、不讲理的一面,他的占有欲极强。安东尼渴望康苏珞随时陪伴他,尤其当他结束长途飞行时。终其一生,安东尼霸道、苛求别人的态度从未改变,不止是康苏珞,安东尼的朋友也饱受他霸道个性的折磨。只要他有兴致,无论何时都会打电话给朋友,请他们听他朗读文稿。

有时候,康苏珞一听到安东尼的要求,就必须抛下所有的朋友和正在进行的活动飞奔到他身边,因为安东尼坚持真正厮守的爱情是两个人心里只有彼此,容不下其他事物。

{\startalignment[center]
 \placefigeasy[][imgs/小王子(艾柯譯)/figure_0176_0059.jpg][maxwidth=\textwidth,maxheight=\textheight,location={middle,none}]{}
 \stopalignment}
美国的康苏珞

一别六个星期之后,两人在西班牙重逢,康苏珞意识到自己渴望再见到安东尼。随后她与安东尼在尼斯度过了温馨甜美的数月,期间安东尼潜心创作《夜航》,康苏珞给他鼓励支持,两人感情如胶似漆。

在离别的这段时间,安东尼陪伴在母亲身边,告诉母亲他爱上一个南美女子,希望母亲支持他的选择------和康苏珞结婚意味着安东尼将承受来自圣艾修伯里家族的压力,因为他背弃了贵族之间通婚的传统做法,最重要的是只有这样才能挽救圣摩里斯城堡的财务危机。玛丽尊重儿子的选择,不过她希望儿子和康苏珞结束未婚同居的生活,尽快结婚。

1931年4月20日,安东尼和康苏珞在亚盖教堂举行婚礼,在苏多神父的福证之下,结为夫妻。同年底,安东尼出版第二本小说《夜航》,书一推出即获好评。不过对于安东尼来说,他更在意康苏珞是否喜欢,为了她,安东尼展现出前所未有的勤奋,以挖掘自己的潜力,他要向心上人证明自己有成为名作家的能力,并且会超越康苏珞的前夫葛梅兹·卡利约。

\reference[Chapter2_11.xhtml]{}

\stoptitle

\starttitle[title={11}]

1931年,安东尼与康苏珞婚后不久,康苏珞即饱受“折磨”,当热恋的光环褪去,做飞行员的妻子并不是好的选择。你必须无时无刻地为他担心,直至他返航为止,而下一次的焦虑又将伴随下一次的飞行开始。

康苏珞在卡萨布兰卡的日子很不好过,这里没人说西班牙语,她成了不折不扣的外地人,安东尼不在身边,当地的法国圈子成为康苏珞唯一的心灵寄托。但很不幸的是,安东尼讨厌法国人在卡萨布兰卡营造的虚浮社交生活。如此,康苏珞好像在夹缝中生存。

而且此时航空公司内部不和,不断爆发丑闻,入不敷出,濒临破产,康苏珞不得不为两人日后的生活担忧。事已至此,她只能寄希望于安东尼绝佳的文笔,希望他在文坛闯出名气以后,放弃危险的飞行事业,不过更不幸的是,安东尼一生挚爱飞行,历经劫难也从未打算放弃。

1931年,《夜航》获得“女性文学大奖”,安东尼尚未来得及仔细品尝成功带来的喜悦,就因为公司内部派系斗争被同事打击得一塌糊涂。《夜航》的出版,让反对道哈特的同事觉得安东尼是“拥道哈特”派。他们对这本书并不买账,仿佛安东尼描写同事的英雄事迹是一种犯罪。这让素来在意评价的安东尼伤透了心。

空中邮局里的坏消息仍不断传出,安东尼的飞行事业每况愈下。一方面他和康苏珞同甘共苦,另一方面他和康苏珞又都没有什么经济头脑,一个挥霍成性,一个不在乎金钱。虽然他们都不在乎钱,可经济的拮据还是给生活带来不少摩擦和烦恼。

尽管两人爱意不减,但却聚少离多,直到1933年,安东尼调往圣拉斐尔工作,两人才有较长时间相聚。然而,相聚的快乐很快被噩梦取代,1933年12月21日,安东尼发生严重坠机意外,随后被公司以驾驶技术拙劣为由解雇。

安东尼失业了,其后整整一年里,面对巨大的生活压力,他不断为报纸写专栏文章、编写剧本,但这些不能从根本上改变他们的窘境,除了康苏珞挥霍之外,安东尼自己对付没钱的办法就是花更多的钱,1934年到1935年间,他又买下两架飞机,因此欠下一大堆账单。

安东尼一生不改散漫天真的个性。1935年12月底,他决定驾驶飞机由巴黎到西贡,预备打破飞行纪录赢得15万法郎的奖金。不过,他准备不充分,在飞行时又因天气预报有误,飞机在沙漠中坠毁。这期间的故事被他写入《风沙星辰》。安东尼和助手在沙漠中流浪,1936年1月2日获救。安东尼深爱康苏珞,在沙漠中他想起康苏珞渴求的眼神,抵御了死神的诱惑。康苏珞得知安东尼获救,高兴得昏了过去,醒来后和朋友大肆庆祝。即使矛盾不断,他们仍深爱对方。这种深刻复杂的感情衍生出小王子与玫瑰的爱。

安东尼与康苏珞从1936年开始分居,在那之前婚姻已出现状况,有钱虽然可望好转,然而症结不在这里。在安东尼的飞行生涯中,康苏珞有三次险些成为寡妇,长久的担忧使她越来越忧郁,开始靠酒精麻醉自己,和朋友在一起让她获得心灵的安慰。安东尼则认为康苏珞不够关心他,因为她不曾全心为他付出。虽然安东尼喜欢小孩,可惜,四处漂泊的职业,姐姐和弟弟的早逝带来的不安以及对彼此情感的怀疑,使他们未及孕育子女。

\reference[Chapter2_12.xhtml]{}

\stoptitle

\starttitle[title={12}]

1936年到1937年,安东尼奉派到西班牙担任战地记者,亲历战争和死亡,这让安东尼开始深思整个人类文明以及生命的种种意义,安东尼认为:“来自灵魂深处的呼喊可以免除人类被自己的愚昧蒙蔽,前提是人类愿意敞开心胸,发现存在于小我世界之外的大我共同目标。”此时的安东尼扬弃从小被灌输的宗教政治观,试图寻找一种更真实平和的信仰。

自战地回来后,安东尼接受《巴黎晚报》约稿,准备写八篇关于西班牙内战的文章,但他最终只交了四篇,安东尼对作品一丝不苟、精益求精的态度,深深地折磨着自己和编辑。除了这个原因,1938年,安东尼再次发生严重坠机意外。

1938年2月,安东尼向政府申请开拓巴黎到巴塔哥尼亚的航线,这项行动在红颜知己尼莉的帮助下获得政府支持。安东尼的动机只是为了逃避恼人的家务事,只有飞行能让他忘却凡尘俗事。

飞机在瓜地马拉坠毁,安东尼和助手严重受伤,此次意外带来的身体伤害超过以往任何一次,给安东尼的生活带来沉重负担。安东尼被送到医院救治,康苏珞赶来,她照顾安东尼,拒绝签署截肢同意书,为安东尼保住左手。随后安东尼被送往纽约接受进一步治疗,那是他第一次踏上美国的土地。法国内战爆发后,安东尼不忍同胞相残,流亡美国,在此完成《小王子》。

在漫长的康复时间里,安东尼得以专心地修改《风沙星辰》的原稿,交出近乎完美的作品。《风沙星辰》于1939年出版,这本书让安东尼红遍美国,获得巨大反响。《风沙星辰》为安东尼带来名誉和财富,也就此奠定了他法国当代杰出作家的地位,不过,事业的成功无法舒缓情感的压力,安东尼和康苏珞仍未复合。

《风沙星辰》出版后6个月,1939年9月3日,二战正式爆发,安东尼响应国家征召,担任空军上尉。安东尼拒绝后勤行政工作,通过朋友帮助,他被分派到侦察单位的高级空照部队2/33大队。安东尼与同胞们同甘共苦,拒绝受到优待,他多次出动,出色完成任务,获得大家尊重。此间的事,启发了他创作《战斗的飞行员》的灵感。

尽管军士们骁勇善战,渴望为国效力,但当时的法国军事高层一味保存实力,指挥失策,战败势不可免。1940年5月开始,法国战败,全国陷入恐慌。

法国的失败让安东尼感到失望和耻辱,他对法国军事高层完全丧失信心,从6月停战协议生效到11月这段时间,安东尼所在的2/33大队效忠贝当政府,安东尼为了义气也拒绝加入“自由法国”运动,他并不赞同戴高乐的政治观点。而且安东尼从骨子里认定,只有美国参战才能改变局势,拯救法国,其他都是虚言。

1940年,戴高乐的“自由法国”和贝当的维琪政权正式对决,法国上演内战。

这一年是多事之秋。少年时代对他影响深远的两位神父也先后离世。先是劳南神父退休后一病不起,苏多神父得知法国战败的消息后,也在做教学弥撒时骤逝。截止1940年,安东尼在2/33大队同一小队的23名战友中,已有7人丧生。亲友的相继辞世,让安东尼在国难当头之时饱受家愁的煎熬。

12月1日,安东尼的一生挚友居劳梅,驾驶运输机在地中海被意大利军队误击,坠海而死。安东尼为之痛心疾首。他再也不想待在这个伤心地。12月,有出版商邀请安东尼去纽约促销《风沙星辰》,并商讨新书事宜,安东尼决定前往美国。

在美国的生活虽然优渥,但安东尼仍难排遣忧国忧民之心,他不断与好友李奥·维斯通信,以获得国内的消息,不止是文学、哲学,法国的未来也是他和朋友探讨争辩的话题。安东尼在美国时文学成就巨大,他娴熟运用以前积累的生活经历,把它们编织成完美的文学名著。

1941年的大部分时间里,安东尼都在完成《战斗的飞行员》,他面临截稿的压力,而且他一直在设法安排康苏珞离开法国。这两件事花了他将近一年的时间。

康苏珞的到来,并不意味着幸福和平静再次光临,他和康苏珞已深深互相伤害,无法弥补。

尽管他们到死依然深爱彼此,却犹如刺猬爱上玫瑰,无法靠近。

1943年,安东尼带着对妻子难解的爱和绝望离开。他只给她留下《小王子》。

\reference[Chapter2_13.xhtml]{}

\stoptitle

\starttitle[title={13}]

我要求安东尼告诉我,他是如何完成《小王子》的。我告诉他那是我最喜爱的作品。

“你我因此而结缘,我应当满足你的要求。”安东尼微笑着说,很快他将我带到洛杉机医院。1941年夏天,安东尼在此住院。

我透过他的回忆,看见一幅清晰的画面:偌大的隔离病房,空荡荡的,安东尼躺在病床上看《安徒生童话》。每当沮丧失落时,他习惯在童年回忆中寻找慰籍。他想起小时候躺在母亲寝室的空床上,母亲对自己细心照顾。他开始静心体悟生命的种种甘辛甜美以及无止境的辽阔寂寞。《小王子》由此酝酿产生。

治疗告一段落后,安东尼租了一间公寓,正式开始创作。接下来的几个星期里,安东尼用小孩的眼光看待整个世界,疾病、政治问题、夫妻争执、写作压力错综交会,但安东尼决意将一切抛之脑后,只安心构建完美无瑕的梦想国度。

安东尼对《小王子》极其用心,1942年的大部分时间里,他全心投入这部作品。安东尼此时的女友,纽约女记者席薇雅·汉弥尔顿建议安东尼自己画插图,安东尼采纳了她的意见,并将原稿赠送给她。这份原稿后来被汉弥尔顿出售给纽约的摩根图书馆。

在撰写《小王子》期间。安东尼和康苏珞决定给彼此最后一次机会,两人再度住在一起,讨论爱与忠贞的话题。此时,安东尼的婚外情已经走到尽头,情敌远在法国,这让康苏珞很安心。《小王子》里爱的承诺和忠贞不二的誓言令她感到快乐欣慰。安东尼在创作中听取康苏珞的意见,尤其是在插图方面,因为康苏珞是很有成就的画家。而小王子素雅的装束和黄色的头巾也很容易让人想到康苏珞。安东尼后来在书信里告诉妻子,他最大的遗憾就是未能将《小王子》题献给她。

不幸的是,夫妻俩的矛盾并未因《小王子》的诞生而消减,这本书只不过拖延了出问题的时间而已。1942年圣诞节前后,安东尼与康苏珞大吵一架,两人关系再度濒临破裂。《小王子》交到出版商手中几个星期后,夫妻关系陷入低迷,安东尼写信告诉康苏珞,唯有一死才能使他内心重获平静。

1943年4月8日,安东尼离开美国去北非向法国空军报到的前两天,与康苏珞发生严重争执。原因是康苏珞花太多钱购置衣服,经济紧张,使得安东尼无法订购空军飞行官的装备,只能凑合穿上歌剧院的道具服装------有点可笑的蓝色制服。

这一别,竟成永诀。

二战结束后,亲戚们为了安东尼的版税问题争得面红耳赤,康苏珞被迫出示安东尼的书信表明《小王子》是安东尼献给自己的情书。康苏珞将这些书信保存完好,即使晚年贫病困窘也不出售。她的行为表明她和安东尼一样珍视守护彼此的爱。

{\startalignment[center]
 \placefigeasy[][imgs/小王子(艾柯譯)/figure_0187_0060.jpg][maxwidth=\textwidth,maxheight=\textheight,location={middle,none}]{}
 \stopalignment}
永远的安东尼

一切正如小王子所说:“我应该以她的行为,而不是她的言语来评断她的一切。她用身体将我包围,照亮我的生命。我不应该离她而去。我早该猜到,在她并不高明的把戏背后隐藏着最深的温柔,花朵的心思总叫人猜不透。我太年轻了。不知道如何爱她。”然而在当时,连小王子自己也不能真正了解话中的深意。

还有,玫瑰也会挂念小王子,一直想念。

\reference[Chapter2_14.xhtml]{}

\stoptitle

\starttitle[title={14}]

安东尼离开美国前往大西洋基地报到。此时,安东尼已经43岁,以飞行员的年龄来说,安东尼太老了,他的身体以及体力都无法适应飞行任务,可是安东尼德高望重,而且非常倔强,大家无法阻止他。不过,即使不是因为种种原因让安东尼有了赴死之心,在美国加入战争以后,他也不可能袖手旁观,在美国过悠闲的生活。安东尼的爱国之心,势必让他决意为法国流尽最后一滴血。

13个月后,1944年7月31日,安东尼驾驶侦察机前往法国南部进行长程空照任务。任务预定四个小时结束,但迟迟不见安东尼返航。到了下午2时,大家已经完全放弃希望。

在地中海上空俯瞰蓝色无垠的大海,我突然落下泪来,那一天,自从他从这里消失不见,世间再没有安东尼·德·圣艾修伯里这个人。

众人纷纷揣测他的去向,母亲玛丽一直认为儿子未死,他只是遁世远修去了。但无论她如何思念爱子,也无法阻挡死神的到来。这个人,他真的,真的,消失在这个世间了。

在此地,我重临安东尼的死亡现场,见证他消失。我观望他走向死亡,如水融入大海。

我们,无能为力,无法挽救,无法拒绝------这人世间最后的归宿。

那天,在我睡觉的时候,看见一个男人走进来。虽然房间里很暗,但我还是能感觉到他身材高大,比一般的中国男人要魁梧。

他看起来像一个站在我家门口,等待被邀请的谨慎的客人。我感觉他没有恶意,因此稍稍放心。

我伸手打开灯,看清了他的脸。他脸上有个很清晰的特征------一个往上翘的鼻子。还有,和一般的不速之客不同,他穿着笨重的飞行服。

“你好。我飞行经过这里,看见小王子在你的屋子里,所以来看一看。”

“\ldots{}\ldots{}小王子\ldots{}\ldots{}你是说《小王子》?”我拿起桌子上的书。很奇怪我突然能听懂法语,天知道我连英语都很烂。

“我叫安东尼·德·圣艾修伯里。”他自我介绍说。

《小王子》之父。哦,不。你就是小王子。

之后,他邀我开始旅行。

我说过:“他看起来好像死了,但这不是真的。”说完,安东尼·德·圣艾修伯里就像小王子一样消失了。

直到今天,除了在梦里,我从没有遇见他。但我相信他的话,小王子并没有死,他只是暂时离开。

\stoptitle