\usemodule[memos][%mode=kindle,
    themecolor=green,
    latinfontname=cormorant,
    layout=moderate,
    lineheight=1.5\bodyfontsize,
    chapterstyle=madsen,
    ]
%\setupheadertexts[\framed[frame=off,bottomframe=on,
%                          align=low,
%                          height=\headerheight,width=max]
%                         {\pagenumber \ \limitatetext{王阳明图传}{6cm}{...}\par}]
%             [][][\framed[frame=off,bottomframe=on,
%                          align={flushright,low},
%                          height=\headerheight,width=max]
%                         {\limitatetext{\getmarking[chapter]}{6cm}{...} \ \pagenumber\par}]
\definefontsize [e]
\definebodyfontenvironment [default][e=2.488]
\definesectionlevels[preface]  [title,subject,subsubject,subsubsubject]
\definesectionlevels[default]  [chapter,section,subsection,subsubsection]
\definesectionlevels[appendix] [title,subject,subsubject,subsubsubject]
\setuphead          [subject]  [style=hwb,align=center]
\setupcaption                  [width=max]
\setupdelimitedtext [quotation][style=hw,
                                before=\blank[halfline],after=\blank[halfline]]
\definebar          [topnote]  [color=lightgray,rulethickness=1,
                                offset=0.5,order=background,continue=yes]
\setupfootnotes     [way=bysection]
\def\topnote#1{\startbar[topnote]{\em (#1)}\stopbar}
\setupcaptions[width=.8\textwidth]
\protected\def\buffernote[#1]{\footnote{\inlinebuffer[#1]}}

\startbuffer[1]
钱德洪《征宸濠反间遗事》,《王阳明全集》卷三十九,上海古籍出版社,2014年,第1632页。
\stopbuffer
\startbuffer[2]
许自昌《樗斋漫录》卷六,参见《相与校对〈忠义水浒传〉》,《冯梦龙集笺注》,天津古籍出版社,2006年,第250页。
\stopbuffer
\startbuffer[3]
《三教偶拈叙》,本书,第214页。
\stopbuffer
\startbuffer[4]
本书,第200、202页。
\stopbuffer
\startbuffer[5]
本书,第14页。
\stopbuffer
\startbuffer[6]
本书,第5页。
\stopbuffer
\startbuffer[7]
《中兴实录叙》,《甲申纪事》,《冯梦龙全集》第15册,第247---248页。
\stopbuffer
\startbuffer[8]
《三教偶拈叙》,本书,第214---215页。
\stopbuffer
\startbuffer[9]
《〈醒世恒言〉序》,《冯梦龙全集》第3册,第1页。
\stopbuffer
\startbuffer[10]
《王阳明全集》卷三十三《年谱一》,第1348页。
\stopbuffer
\startbuffer[11]
本书,第20页。
\stopbuffer
\startbuffer[12]
杜维明《宋明儒学思想之旅---青年王阳明(1472---1509)》,《杜维明文集》第三卷,武汉出版社,2002年,第63---64页。
\stopbuffer
\startbuffer[13]
《三教偶拈》,《冯梦龙全集》第10册,第204页。
\stopbuffer
\startbuffer[14]
董穀《斩蛟》,《碧里杂存》,明刻本,第15---16页。
\stopbuffer
\startbuffer[15]
本书,第129页。
\stopbuffer
\startbuffer[16]
《三教偶拈》,《冯梦龙全集》第10册,第194---197页。
\stopbuffer
\startbuffer[17]
本书,第24页。
\stopbuffer
\startbuffer[18]
参见中田胜《\quota{王阳明先生出身靖乱录}考》,二松学舍大学《人文论丛》(23),1982年,第1---10页。
\stopbuffer
\startbuffer[19]
《节本明儒学案》,商务印书馆,1916年,第351---353页。
\stopbuffer
\startbuffer[20]
李贽《大孝一首》,《为黄安二上人三首》,《焚书》卷二,中华书局,1975年,第80页。
\stopbuffer
\startbuffer[21]
《王心斋全集》,江苏教育出版社,2001年,第109页。
\stopbuffer
\startbuffer[22]
《前言》,《王阳明大传:知行合一的心学智慧》,第3页。
\stopbuffer
\startbuffer[23]
梁启超《王阳明知行合一之教》,《饮冰室文集》之四十三,《饮冰室合集》第5册,中华书局,1989年,第23---24页。
\stopbuffer
\startbuffer[24]
《王阳明全集》卷二十四,第1010页。
\stopbuffer
\startbuffer[25]
《〈警世通言〉叙》,《冯梦龙全集》第2册,第663页。
\stopbuffer
\startbuffer[26]
《三教偶拈》,《冯梦龙全集》第10册,第146页。
\stopbuffer
\startbuffer[27]
《三教偶拈》,《冯梦龙全集》第10册,第150---151页。
\stopbuffer
\startbuffer[28]
《〈古今小说〉序》,《冯梦龙全集》第1册,第2---3页。
\stopbuffer
\startbuffer[29]
本书,第108页。
\stopbuffer
\startbuffer[30]
钱德洪《瑞云楼后记》,《徐爱钱德洪董澐集》,凤凰出版社,2007年,第170页。
\stopbuffer
\startbuffer[31]
本书,第46页。
\stopbuffer
\startbuffer[32]
陆深《海日先生行状》,《王阳明全集》卷三十八,第1550---1551页。
\stopbuffer
\startbuffer[33]
冯梦龙《庄子休鼓盆成大道》,《警世通言》卷二,《冯梦龙全集》第2册,第13页。
\stopbuffer

\startbuffer[1-1]
\quota{须知规矩出方圆},《王阳明全集》作\quota{须从规矩出方圆}。见《王阳明全集》卷二十《别诸生》,上海古籍出版社,2014年,第872页。按:《靖乱录》所引阳明诗与《王阳明全集》常有微小差异,盖所据版本不同。以下不再出校。
\stopbuffer
\startbuffer[1-2]
\quota{道学},指宋代出现的传承尧、舜、文王、周公、孔子、孟子之道,重视理气心性、天德王道的儒学。与后来出现的\quota{理学}一词含义大致相同。
\stopbuffer
\startbuffer[1-3]
\quota{五教},出自《尚书·舜典》:\quota{帝曰:契,百姓不亲,五品不逊,汝作司徒,敬敷五教,在宽。}五教指父义、母慈、兄友、弟恭、子孝,亦指父子有亲,君臣有义,夫妇有别,长幼有序,朋友有信。
\stopbuffer
\startbuffer[1-4]
王华(1446---1522),字德辉,晚号海日翁。关于拾金囊一事,杨一清与陆深均有记录,但与冯梦龙所言稍异。据杨一清《海日先生墓志铭》:六岁,与群儿戏水滨,见一客来濯足,已大醉,去,遗其所提囊。取视之,数十金也。公度其醒必复来,恐人持去,以投水中坐守之。少顷,其人果号而至。公迎谓曰:\quota{求尔金邪?}为指其处。其人喜,以一锭为谢,却不受。(《世德纪》,《王阳明全集》卷三十八,第1534页。)又见陆深《海日先生行状》:其人喜跃,以一金谢。先生笑却之曰:\quota{不取尔数十金,乃取尔一金乎?}客且惭且谢,随至先生家,无少长咸遍拜而去。(《世德纪》,《王阳明全集》卷三十八,第1545页。)
\stopbuffer
\startbuffer[1-5]
竹轩翁即王华的父亲王伦,据魏瀚《竹轩先生传》:\quota{先生名伦,字天叙,以字行。性爱竹,所居轩外环植之,日啸咏其间,视纷华势利,泊如也。客有造竹所者,辄指告之曰:\quota{此吾直谅多闻之友,何可一日相舍耶?}学者因称曰竹轩先生。}(《王阳明全集》卷三十八,第1530页。)
\stopbuffer
\startbuffer[1-6]
\quota{苍头},原指士卒的皂巾,后代指奴仆。
\stopbuffer
\startbuffer[1-7]
据陆深《海日先生行状》:\quota{十四岁时,尝与亲朋数人读书龙泉山寺。寺旧有妖为祟。数人者皆富家子,素豪侠自负,莫之信;又多侵侮寺僧,僧甚苦之。信宿妖作,数人果有伤者。寺僧因复张皇其事,众皆失气,狼狈走归。先生独留居如常,妖亦遂止。僧咸以为异。每夜分,辄众登屋号笑,或瓦石撼卧榻,或乘风雨雷电之夕,奋击门障。僧从壁隙中窥,先生方正襟危坐,神气自若,辄又私相叹异。然益多方试之,技殚,因从容问曰:\quota{向妖为祟,诸人皆被伤,君能独无恐乎?}先生曰:\quota{吾何恐?}僧曰:\quota{诸人去后,君更有所见乎?}先生曰:\quota{吾何见?}僧曰:\quota{此妖但触犯之,无得遂已者,君安得独无所见乎?}先生笑曰:\quota{吾见数沙弥为祟耳。}诸僧相顾色动,疑先生已觉其事,因佯谓曰:\quota{此岂吾寺中亡过诸师兄为祟邪?}先生笑曰:\quota{非亡过诸师兄,乃见在诸师弟耳。}僧曰:\quota{君岂亲见吾侪为之?但臆说耳。}先生曰:\quota{吾虽非亲见,若非尔辈亲为,何以知吾之必有见邪?}寺僧因具言其情,且叹且谢曰:\quota{吾侪实欲以此试君尔。君天人也,异时福德何可量?}至今寺僧犹传其事。}(《世德纪》,《王阳明全集》卷三十八,第1546 页。)
\stopbuffer
\startbuffer[1-8]
据钱德洪:\quota{海日公夫人郑,妊先生既弥十四月,岑夜梦五色云中,见神人绯袍玉带,鼓吹导前,送儿授岑曰:\quota{与尔为子。}岑辞曰:\quota{吾已有子,吾媳妇事吾孝,愿得佳儿为孙。}神人许之。忽闻啼声,惊寤,起视中庭,耳中金鼓声隐隐归空,犹如梦中。盖成化壬辰九月三十日亥时也。}(《瑞云楼后记》,《徐爱 钱德洪 董澐集》,凤凰出版社,2007年,第170页。)
\stopbuffer
\startbuffer[1-9]
黄绾《阳明先生行状》:\quota{六岁不言。一日,有僧过之,摩其顶曰:\quota{有此宁馨儿,却叫坏了。}龙山公悟,改今名,遂言,颖异顿发。}(《王阳明全集》卷三十八,第1554页。)宁馨儿,吴方言,意思是\quota{这样的孩子},有赞美之意。语出《晋书·王衍传》:\quota{何物老妪,生宁馨儿!}
\stopbuffer
\startbuffer[1-10]
\quota{高真},道教中称得道之人为\quota{高真}。
\stopbuffer
\startbuffer[1-11]
\quota{蔽月山房},疑为\quota{水月山房},据《金山志》卷四:\quota{水月山房,额在客堂后院地上。}(束景南《阳明佚文辑考编年》,上海古籍出版社,2012年,第13页。)
\stopbuffer
\startbuffer[1-12]
\quota{楮},落叶乔木,皮可做纸,代指祭祀用的纸钱。
\stopbuffer
\startbuffer[1-13]
\quota{韬钤},原指《六韬》及《玉钤篇》两兵书,后泛指兵书。
\stopbuffer
\startbuffer[1-14]
《梦中绝句》,《王阳明全集》卷二十,第877页。诗中\quota{云埋铜柱}\quota{六字题文},据说马援在平定交趾后,在国境立一铜柱,上面题有六字\quota{铜柱折,交趾灭。}《大明一统志》卷九十对此有记载。此诗为阳明破断藤峡前所作,诗前有序,\quota{此予十五岁时梦中所作。今拜伏波祠下,宛如梦中。兹行殆有不偶然者,因识其事于此。}正德十一年丙子(1516),王阳明四十五岁时,升南赣佥都御史以后,有诗:\quota{五月南征想伏波},\quota{一战功成未足云}。(《喜雨三首》,《王阳明全集》卷二十,第821---822页。)据《铜柱梦》:\quota{阳明先生既受广西田州之命,自言曰:\quota{吾少时常梦至马伏波庙,题之云:铜柱折,交趾灭,拜表归来白如雪。又梦题诗曰:拜表归来马伏波,早年兵法鬓毛皤。云埋铜柱雷轰折,六字铭文永不磨。不意今有此行。}乃嘉靖四年秋也。逾年,功成而疾亟矣。屡表乞致,不许,遂促归,至南雄府青龙铺水西驿而卒。事闻,上怒,爵阴遂尼至今。梦之验也如此。}(董穀《碧里杂存》,明刻本,第21页。)
\stopbuffer
\startbuffer[1-15]
终军(约前133---前112),字子云,西汉济南人。据《汉书》载:\quota{少好学,以辩博能属文闻于郡中,年十八,选为博士弟子。}后请缨出使南越,\quota{军遂往说越王,越王听许,请举国内属}。\quota{越相吕嘉不欲内属,发兵攻杀其王及汉使者,皆死。}\quota{军死时年二十余,故世谓之\quota{终童}。}(《汉书》卷六十四下,中华书局,1962年,第2814---2821页。)
\stopbuffer
\startbuffer[1-16]
据清毛奇龄:\quota{游居庸是偶然事,或意有所在。而行状及年谱皆云:时有石和尚、刘千斤之乱,公欲作疏奏诸朝,请自讨之。公父禁之,乃相度形势,出游居庸。则可笑之甚。按石和尚、刘千斤在成化二年(1466)作乱,越一年,遂平。又越五年至八年,而公始生。是作疏讨贼,皆公前世事也。且公父海日公登成化十七年(1481)进士,此时亦未能有修撰官居早在京邸。又况石、刘之乱,只在河阳、南阳间,与居庸无涉。初不意门人黄绾作行状,德洪作年谱而诞罔无理至于此。}(《王文成传本》卷一,清刻本,第2页。)
\stopbuffer
\startbuffer[1-17]
据《铁柱老僧》:\quota{阳明先生壮年受室时,以妇翁宦江西,因往焉。一日,独游铁柱观,至一静室中,见一老僧坐,与语相得。僧乃出书一编,授先生而别,且曰:\quota{三十年后再相见。}后平宸濠,入洪都,复往游焉。老僧尚在,以诗遗先生曰:\quota{三十年前曾见君,再来消息我先闻。君于生死轻毫末,谁把纲常任半分?穷海也知钦令德,老天应未丧斯文。东归若到武夷去,千载香灯锁白云。}先生亦有和章,今失记。昔所授编,亦竟不知何书也。}(董穀《碧里杂存》,第20---21页。)据《年谱》,铁柱老僧为一道士:\quota{孝宗弘治元年戊申,先生十七岁,在越。七月,亲迎夫人诸氏于洪都。外舅诸公养和为江西布政司参议,先生就官署委禽。合卺之日,隅闲行入铁柱宫,遇道士趺坐一榻,即而叩之,因闻养生之说,遂相与对坐忘归。诸公遣人追之,次早始还。}(《王阳明全集》卷三十三,第1347页。)阳明四十八岁,正德十四年,阳明平宸濠,入南昌,与\quota{三十年后再相见}\quota{三十年前曾见君}相符。
\stopbuffer
\startbuffer[1-18]
事见《年谱》弘治元年戊申(1488),先生十七岁。《年谱》所记与此有差异:\quota{只\quota{不要字好}一念,亦是不敬}一句,《年谱》作\quota{乃知古人随时随事只在心上学,此心精明,字好亦在其中矣}。(《王阳明全集》,第1347---1348页。)按:据《年谱》,此处王阳明是肯定明道先生(程颢,1032---1085)的意思。\quota{某写字时甚敬,非是要字好,只此是学。}见《二程遗书》卷三,《二程集》,中华书局,2004年,第60页。
\stopbuffer
\startbuffer[1-19]
如前所记,王阳明十二岁时,惟以为圣贤是人生第一等事。娄谅正是以此接引阳明:\quota{还广信,谒一斋娄先生。异其质,语以所当学,而又期以圣人,为可学而至,遂深契之。}(黄绾《阳明先生行状》,《王阳明全集》卷三十八,第1555页。)娄谅之学,\quota{以收放心为居敬之门,以何思何虑、勿助勿忘为居敬之要}。\quota{文成年十七,亲迎过信,从先生问学,深相契也,则姚江之学,先生为发端也。}(《明儒学案》卷二,《黄宗羲全集》第7册,浙江古籍出版社,2005年,第38页。)娄谅师从吴与弼,吴与弼号康斋,\quota{刻苦奋励,多从五更枕上汗流泪下得来。及夫得之而有以自乐,则又不知足之蹈之、手之舞之。盖七十年如一日,愤乐相生,可谓独得圣贤之心精者}。(《师说》,《黄宗羲全集》第7册,第11页。)吴与弼门下三大弟子:娄谅、胡居仁、陈献章。胡居仁主敬,敬义夹持,为吴与弼守\quota{愤}之门户者;陈献章受学吴与弼后,归家即绝意科举,筑春阳台,数年静坐其中,虽家人罕见其面,虚静中养出端倪,有得于似初春的氤氲一气,由敬畏到洒落,为吴与弼守\quota{乐}之门户者。胡居仁所訾者,亦唯陈献章与娄谅为最,谓两人皆是陷入异教去,但正是陈、娄二人,展开了明代学术的转型。\quota{椎轮为大辂之始,层冰为积水所成,微康斋,焉得有后时之盛哉?}(《明儒学案》卷一,《黄宗羲全集》第7册,第1页。)
\stopbuffer
\startbuffer[1-20]
\quota{蘧伯玉年五十,而知四十九年非。}(《淮南子·原道训》。)据《论语·宪问》:\quota{蘧伯玉使人于孔子,孔子与之坐而问焉,曰:\quota{夫子何为?}对曰:\quota{夫子欲寡其过而未能也。}}
\stopbuffer
\startbuffer[1-21]
据陆深《海日先生行状》:\quota{庚戌正月下旬,竹轩之讣始至,号恸屡绝。即日南奔,葬竹轩于穴湖山,遂庐墓下。墓故虎穴,虎时时群至。先生昼夜哭其傍,若无睹者。久之益驯,或傍庐卧,人畜一不犯,人以为异。}(《世德纪》,《王阳明全集》卷三十八,第1548页。)据此,竹轩翁当卒于浙江家中,龙山公在京得闻讣。
\stopbuffer
\startbuffer[1-22]
据《年谱》,王阳明再次落第的原因是才高遭忌:\quota{明年春,会试下第,缙绅知者咸来慰谕。宰相李西涯戏曰:\quota{汝今岁不第,来科必为状元,试作来科状元赋。}先生悬笔立就。诸老惊曰:\quota{天才!天才!}退有忌者曰:\quota{此子取上第,目中无我辈矣。}及丙辰会试,果为忌者所抑。}(《王阳明全集》卷三十三,第1349页。)
\stopbuffer
\startbuffer[1-23]
威宁伯指王越,据《王越传》:\quota{王越,字世昌,浚人。长身,多力善射,涉书史,有大略。}\quota{卒于甘州,赠太傅,谥襄敏。越姿表奇伟,议论颷举,久历边陲,身经十余战,知敌情伪及将士勇怯,出奇制胜,动有成算。奖拔士类,笼罩豪俊,用财若流水,以故人乐为用。}(《明史》卷一百七十一,《明史》,中华书局,1974年,第4571---4576页。)
\stopbuffer
\startbuffer[1-24]
《陈言边务疏》,《王阳明全集》卷九,第316---322页。
\stopbuffer
\startbuffer[1-25]
据《年谱》:\quota{闻地藏洞有异人,坐卧松毛,不火食,历岩险访之。正熟睡,先生坐傍抚其足。有顷醒,惊曰:\quota{路险何得至此!}因论最上乘曰:\quota{周濂溪、程明道是儒家两个好秀才。}后再至,其人已他移,故后有会心人远之叹。}(《王阳明全集》卷三十三,第1351页。)此处所言\quota{朱考亭是个讲师,只未到最上一乘},《年谱》中无此语。据阳明所言:\quota{赖天之灵,因有所觉,始乃沿周、程之说求之,而若有得焉。}(《别湛甘泉序》,《王阳明全集》卷七,第257页。)\quota{朱子所谓\quota{格物}云者,在即物而穷其理也。即物穷理,是就事事物物上求其所谓定理者也。是以吾心而求理于事事物物之中,析\quota{心}与\quota{理}为二矣。}\quota{夫析心与理而为二,此告子\quota{义外}之说,孟子之所深辟也。}(《答顾东桥书》,《王阳明全集》卷二,第50---51页。)由此可见,地藏洞异人对于周濂溪、程明道的推崇与阳明是一致的,对于朱子的贬低亦有所据。
\stopbuffer
\startbuffer[1-26]
正德十五年(1520),王阳明《重游化城寺二首》之\quota{会心人远空遗洞,识面僧来不记名},可与此诗相印证。
\stopbuffer
\startbuffer[1-0]
据毛奇龄所言:\quota{公晚爱会稽山阳明洞,名因号阳明子。按会稽山即苗山,并无洞壑。凡禹井、禹穴、阳明洞类,只是石罅,并无托足处。旧诬以道人授书洞中,固大妄。今作传者且曰讲学阳明洞,则妄极矣。}(《王文成传本》卷一,第1---2页。)阳明洞是在冯梦龙所讲的四明山,还是在毛奇龄所言的会稽山?由此引发出阳明洞是否属实及阳明之号的问题。
冯梦龙对于四明山的描述多引自梅福《四明山记》,明代戴洵亲自勘验,有《四明辨并诗》(《明文海》卷一百一十三)。据《年谱》,正德八年(1513),王阳明\quota{从上虞入四明,观白水,寻龙谷之源},遂自宁波还余姚。(《王阳明全集》卷三十三,第1363页。)白水即白水冲,与龙谷均在四明山。据《云笈七签》,道家三十六小洞天中第九为四明山洞,周回一百八十里,名曰丹山赤水天,在越州上余县;第十为会稽山洞,周回三百五十里,名曰阳明洞,在古越州山阴县。(《洞天福地》,《云笈七签》卷二十七,中华书局,2003年,第613页。)道家的阳明洞在会稽山,而非在四明山。
据黄绾《阳明先生行状》,\quota{养病归越,辟阳明书院}(《阳明先生行状》,《王阳明全集》卷三十八,第1556页),邹守益谓\quota{辟阳明洞旧基为书屋},黄绾与邹守益均为阳明重要弟子,此言可信度高。禹井、禹穴、阳明洞俱在会稽山,毛奇龄所言\quota{禹井、禹穴、阳明洞只是石罅,并无托足处},阳明洞的大小并不重要,关键在于王阳明是否在阳明洞附近造书屋、建书院。
陈来先生认为下文提及的五云门为绍兴府的正东门,与山阴为邻的萧山湘湖\quota{去阳明洞方数十里}等证据,足以证明阳明洞在会稽山无疑。(陈来《有无之境:王阳明哲学的精神》,人民出版社,1991年,第369页。)\quota{已卜居萧山之湘湖,去阳明洞方数十里耳。书屋亦将落成,闻之喜极。诚得良友相聚会,共进此道,人间更复有何乐!}(《与王纯甫》,《王阳明全集》卷四,第174页。)此信下注明\quota{壬申}(1512),距阳明建阳明洞已过去十二年,在此之前,阳明应居于阳明洞的旧基。此处阳明本人所讲是建设书屋,由此更印证了黄绾与邹守益之言。
另据王华的门人陆深所言:\quota{正德壬申秋,以使事之余,迂道拜先生于龙山里第。扁舟载酒,相与游南镇诸山,乃休于阳明洞天之下,执手命之曰:\quota{此吾儿之志也。大业日远,子必勉之。}}(《海日先生行状》,《王阳明全集》卷三十八,第1554页。)王华与陆深所游阳明洞天应该是新落成的书屋或书院,而真正讲学的盛况则发生在嘉靖癸未(1522)至丁亥(1527):\quota{先生初归越时,朋友踪迹尚寥落,既后,四方来游者日进。癸未年已后,环先生而居者比屋,如天妃、光相诸刹,每当一室,常合食者数十人;夜无卧处,更相就席;歌声彻昏旦。南镇、禹穴、阳明洞诸山远近寺刹,徙足所到,无非同志游寓所在。先生每临讲座,前后左右环坐而听者,常不下数百人,送往迎来,月无虚日;至有在侍更岁,不能遍记其姓名者。}(《传习录下》,《王阳明全集》卷三,第134页。)阳明苦心经营的阳明洞真正发挥作用,也完成了王阳明早年在狱中\quota{幽哉阳明麓,可以忘吾老}(《读易》,《王阳明全集》卷十九,第747页)的夙愿。综上,阳明洞在会稽山,王阳明在阳明洞附近造书屋、建书院,1512年基本建成,至1522年,阳明洞真正发挥了讲学的作用。阳明洞是王阳明的精神寄托,是阳明讲学的重要场所。
另据王阳明《和九柏老仙诗》,署名\quota{弘治辛酉(1501)仲冬望日,阳明山人王守仁识}。游九华在辟阳明洞之前一年,王阳明便有\quota{阳明山人}的称号。(束景南《阳明佚文辑考编年》,上海古籍出版社,2012年,第110页。)王阳明可能很早就心系阳明洞,冯梦龙所讲的因隐居阳明洞而号\quota{阳明}是不准确的。
\stopbuffer
\startbuffer[1-27]
据王畿言阳明:\quota{筑洞天精庐,日夕勤修炼习伏藏,洞悉机要,其于彼家所谓见性抱一之旨,非惟通其义,盖已尽得其髓矣。自谓尝于静中,内照形躯如水晶宫,忘己忘物、忘天忘地,与空虚同体,光耀神奇,恍惚变幻,似欲言而忘其所以言,乃真境象也。}(《滁阳会语》,《王畿集》卷二,凤凰出版社,2007年,第33页。)
\stopbuffer
\startbuffer[1-28]
王思舆是阳明的早期道友,王思舆名文辕,号黄轝子,山阴人,习静隐居,励志力行,与阳明为莫逆。成化、弘治间,学者守成说,不敢有私议朱子者,故不见信于时,惟阳明与之为友,独破旧说,盖有所本云。及阳明先生领南赣之命,见黄轝子,黄轝子欲试其所得,每撼激之动,语人曰:\quota{伯安自此可胜大事矣。盖其平生经世之志,于此见焉。}其后黄轝子殁,阳明方讲良知之学,人多非议之,叹曰:\quota{使黄轝子在,于吾言必相契矣。}(见季本《王思舆传》,《季彭山先生文集》,《北京图书馆古籍珍本丛刊》第106册,书目文献出版社,1998年,第896页。)
\stopbuffer
\startbuffer[1-29]
出自瞿佑(1347---1433)《西湖四时曲》,瞿佑为钱塘人,字宗吉,号存斋,有《存斋诗集》《剪灯新话》等著作。
\stopbuffer
\startbuffer[1-30]
此为林逋所作《西湖》,《林和靖集》卷二,后一句又作\quota{细风斜雨不堪听}。林逋\quota{字君复,钱塘人,隐西湖之孤山,真宗闻其名,诏长吏岁时劳问,和靖其赐谥也。}(《四库提要·林和靖集》。)
\stopbuffer
\startbuffer[1-31]
\quota{嗟予瞻眺门墙外,何能仿佛窥室堂?也来攀附摄遗迹,三千之下,不知亦许再拜占末行。}(《泰山高次王内翰司献韵》,《王阳明全集》卷十九,第743页。)
\stopbuffer
\startbuffer[1-32]
此处当来自前文阳明在阳明洞中行导引之法时之所悟:\quota{此孝弟一念,生于孩提,此念若可去,断灭种性矣。此吾儒所以辟二氏。}
\stopbuffer
\startbuffer[1-33]
出自朱子《行宫便殿奏札二》,《晦庵朱先生文公文集》卷十四。原文为:\quota{此循序致精,所以为读书之法也。}\quota{此居敬持志,所以为读书之本也。}(《朱子全书》第12册,上海古籍出版社、安徽教育出版社,2002年,第669---670页。)
\stopbuffer
\startbuffer[1-34]
据《文简穆玄庵先生孔晖》:\quota{穆孔晖字伯潜,号玄庵,山东堂邑人。弘治乙丑进士,由庶吉士除简讨。为刘瑾所恶,调南京礼部主事。瑾败,复官。历司业、侍讲、春坊庶子、学士、太常寺卿。嘉靖己亥八月卒,年六十一,赠礼部右侍郎,谥文简。阳明主试山东,取先生为第一。}\quota{盖先生学阳明而流于禅,未尝经师门之锻炼,故阳明集中未有问答。}(《明儒学案》卷二十九,《黄宗羲全集》第7册,第739页。)
\stopbuffer
\startbuffer[1-35]
\quota{某幼不问学,陷溺于邪僻者二十年,而始究心于老、释。赖天之灵,因有所觉,始乃沿周、程之说求之,而若有得焉。顾一二同志之外,莫予翼也,岌岌乎仆而后兴。晩得友于甘泉湛子,而后吾之志益坚,毅然若不可遏,则予之资于甘泉多矣。}(《别湛甘泉序》,《王阳明全集》卷七,第258页。)
\stopbuffer
\startbuffer[1-36]
王阳明《别湛甘泉序(壬申)》:\quota{晩得友于甘泉湛子,而后吾之志益坚,毅然若不可遏,则予之资于甘泉多矣。}\quota{吾与甘泉友,意之所在,不言而会;论之所及,不约而同;期于斯道,毙而后已者。}(《王阳明全集》卷七,第257---258页。)另据湛若水所记:\quota{正德丙寅,始归正于圣贤之学。会甘泉子于京师,语人曰:\quota{守仁从宦三十年,未见此人。}甘泉子语人亦曰:\quota{若水泛观于四方,未见此人。}遂相与定交讲学。}(《阳明先生墓志铭》,《王阳明全集》卷三十八,第1539页。)
\stopbuffer
\startbuffer[2-1-3]
日本翻刻本此处有眉批:\quota{曰仁即徐爱字,此为二人者,误矣。}冯梦龙此段话存疑有三:其一,徐爱即徐曰仁,为阳明妹婿;其二,蔡宗应为蔡宗兖;其三,冀元亨、蒋信、刘观时并非至浙江问学。
徐爱(1487---1517),字曰仁,号横山,余姚横河马堰人,正德二年,在山阴师从阳明。蔡宗兖(1474---1549),字希渊(颜),号我斋,山阴白洋人。\quota{时闻先师倡道阳明山中,乃偕守忠往受业焉。因与余姚徐君曰仁为三友,刊落繁芜,学务归一。}(《奉议大夫按察司提学佥事蔡公墓志铭》,《季彭山先生文集》,《北京图书馆古籍珍本丛刊》第106册,第890页。)朱节,字守忠(中),号白浦,浙江山阴人。据王阳明《别三子序(丁卯)》,\quota{盖自近年而又得蔡希颜、朱守忠于山阴之白洋,得徐曰仁于余姚之马堰。曰仁,予妹婿也。希颜之深潜,守中之明敏,曰仁之温恭,皆予所不逮。}(《王阳明全集》卷七,第252页。)另据《明儒学案》,\quota{正德丁卯(1507),徐横山、蔡我斋、朱白浦三先生举于乡,别文成而北}。\quota{盖三先生皆以丁卯来学,文成之弟子未之或先者也。癸酉(1513),三先生从文成游四明山。我斋自永乐寺返,白浦自妲溪返,横山则同入雪窦,春风沂水之乐,真一时之盛事也。横山为弟子之首,遂以两先生次之。}(《提学蔡我斋先生宗兖、御史朱白浦先生节》,《明儒学案》卷十一,第252页。)
冀元亨、蒋信、刘观时皆属楚中王门,三人并非于杭州从学阳明。据《明儒学案》,楚中王门\quota{道林、闇斋、刘观时出自武陵,故武陵之及门,独冠全楚。观徐曰仁《同游德山诗》,王文鸣应奎、胡珊鸣玉、刘瓛德重、杨初介诚、何凤韶汝谐、唐演汝渊、龙起霄正之,尚可考也。然道林实得阳明之传。}(《楚中王门学案》,《明儒学案》卷二十八,《黄宗羲全集》第7册,第727页。)考徐爱《同游德山诗序》:\quota{正德乙亥(1515)春正月壬午,与予同游德山者十有四人。杜世荣仁夫则浙人,余皆武陵人士也。}(《横山遗集》卷上,《徐爱钱德洪董澐集》,凤凰出版社,2007年,第66页。)据此,楚中学者大规模的从学阳明在正德十年乙亥。
\quota{冀元亨,字惟乾,号闇斋,楚之武陵人。阳明谪龙场,先生与蒋道林往师焉。从之庐陵,逾年而归。}(《孝廉冀闇斋先生元亨》,《楚中王门学案》,《明儒学案》卷二十八,《黄宗羲全集》第7册,第737页。)\quota{阳明在龙场,见先生之诗而称之,先生遂与闇斋师事焉。}(《佥宪蒋道林先生信》,《楚中王门学案》,《明儒学案》卷二十八,《黄宗羲全集》第7册,第729页。)据赴谪诗(正德丁卯,即1507年),《广信元夕蒋太守舟中夜话》(《王阳明全集》卷十九,第758页),阳明赴龙场之前已与蒋信相识。
\stopbuffer
\startbuffer[2-2-7]
关于阳明赴谪途中的传奇,毛奇龄则予以否决,认为王阳明\quota{时径之龙场,而谱状乃尽情诳诞,举凡遇仙遇佛,无可乘间摭入者,皆举而摭之。于此二十年前、三十年后,开关闭关,随意胡乱。亦思行文说事,俱有理路。浙江一带水与福建武夷、江西鄱阳俱隔仙霞、常玉诸岭峤,而岭表车筏尤且更番叠换,并非身跨鱼鳖可泛泛而至其地者。即浙可通海,然断无越温、台、鄞、鄮,不驾商舶得由海入闽之理。且阳明亦人耳,能出游魂,附鬼伥,朝游丹山,暮飞铁柱,何荒唐也!}(《王文成传本》卷一,第4页。)毛奇龄是从现实的情况出发,进而否定阳明传说的种种奇异之处。
赴谪途中的传奇,可能是王阳明为摆脱刘瑾追杀所使用的计谋,据其好友湛若水回忆:人或告曰:\quota{阳明公至浙,沉于江矣,至福建始起矣。登鼓山之诗曰:\quota{海上曾为沧水使,山中又拜武夷君。}有征矣。}甘泉子闻之,笑曰:\quota{此佯狂避世也。}故为之作诗,有云:\quota{佯狂欲浮海,说梦痴人前。}及后数年,会于滁,乃吐实。彼夸虚执有以为神奇者,乌足以知公者哉?(《阳明先生墓志铭》,《王阳明全集》卷三十八,第1504页。)由此可见,沉江、浮海之说可能确实肇始于阳明本人,以此摆脱刘瑾的追杀。冈田武彦总结为:\quota{王阳明为了从刺客手中逃脱,曾假装跳钱塘江自杀,并隐匿在家乡附近的一座山中,后瞅准时机,经过广信府,前往龙场。}(《王阳明大传:知行合一的心学智慧》,重庆出版社,2015年,第232页。)另据《玉山东岳庙遇旧识严星士》,\quota{行藏无用君平卜,请看沙边鸥鹭群}(《王阳明全集》卷十九,第758页),以及藏有《明夷》卦的《泛海》诗等,透露出神奇荒诞背后的某些事实依据。总之,赴谪途中的传奇既不可尽信,亦不可像毛奇龄那样一概否决。
\stopbuffer
\startbuffer[2-3-9]
按冯梦龙所言,王阳明赴龙场路线依次为:北京、杭州、广信、铅山、福建北、武夷山、铅山、上饶、南京、龙场。
黄绾《阳明先生行状》,\quota{公行至钱塘,度或不免,乃托为投江,潜入武夷山中,决意远遁}。\quota{尝有诗云:\quota{海上曾为沧水使,山中又拜武夷君。}遂由武夷至广信,溯彭蠡,历沅、湘,至龙场。}(《王阳明全集》卷三十八,第1556---1557页。)\quota{海上曾为沧水使,山中又拜武夷君。}又作\quota{海上真为沧水使,山中又遇武夷君}。(《王阳明全集》卷十九,第757页。)按此,王阳明赴龙场路线为:北京、钱塘、武夷山、广信、彭蠡、沅江、湘江、龙场。
据《年谱》:\quota{先生至钱塘,瑾遣人随侦。先生度不免,乃托言投江以脱之。因附商船游舟山,偶遇飓风大作,一日夜至闽界。比登岸,奔山径数十里,夜扣一寺求宿,僧故不纳。趋野庙,倚香案卧,盖虎穴也。夜半,虎绕廊大吼,不敢入。黎明,僧意必毙于虎,将收其囊;见先生方熟睡,呼始醒,惊曰:\quota{公非常人也!不然,得无恙乎?}邀至寺。寺有异人,尝识于铁柱宫,约二十年相见海上;至是出诗,有\quota{二十年前曾见君,今来消息我先闻}之句。与论出处,且将远遁。其人曰:\quota{汝有亲在,万一瑾怒,逮尔父,诬以北走胡,南走粤,何以应之?}因为蓍,得《明夷》,遂决策返。先生题诗壁间曰:\quota{险夷原不滞胸中,何异浮云过太空。夜静海涛三万里,月明飞锡下天风。}因取间道,由武夷而归。时龙山公官南京吏部尚书,从鄱阳往省。十二月返钱塘,赴龙场驿。}(《王阳明全集》卷三十三,第1353---1354页。)按此,王阳明赴龙场路线为:北京、钱塘、福建、武夷山、鄱阳、南京、钱塘、龙场。
据王阳明《咎言》:\quota{正德丙寅冬十一月,守仁以罪下锦衣狱。}(《王阳明全集》卷十九,第732页。)王阳明下锦衣狱出狱后,赴谪诗次序如下:《赴谪次北新关喜见诸弟》《卧病静慈写怀》《移居胜果寺二首》《泛海》《武夷次壁间韵》《草萍驿次林见素韵奉寄》《玉山东岳庙遇旧识严星士》《广信元夕蒋太守舟中夜话》《夜泊石亭寺用韵呈陈娄诸公因寄储柴墟都宪及乔白岩太常诸友》《过分宜望钤冈庙》《袁州府宜春台四绝》《萍乡道中谒濂溪祠》《宿萍乡武云观》《醴陵道中风雨夜宿泗州寺次韵》《长沙答周生》《陟湘于迈岳麓是尊仰止先哲因怀友生丽泽兴感伐木寄言二首》《游岳麓书事》(\quota{醴陵西来涉湘水,信宿江城沮风雨。})《天心湖阻泊既济书事》(\quota{挂席下长沙,瞬息百余里。舟人共扬眉,予独忧其驶。日暮入沅江,抵石舟果圮。})(《王阳明全集》卷十九,第756---765页。)由《广信元夕蒋太守舟中夜话》可知,至广信时值正月十五,后续赴谪路线应发生在丁卯,依次为江西分宜、宜春、萍乡,湖南醴陵、长沙、沅江,至龙场。陈来先生认为这一路线并无《年谱》与冯梦龙所言的至南京省亲之迹。(陈来《有无之境:王阳明哲学的精神》,第345页。)
王阳明在赴龙场前经过广信,娄谅于弘治辛亥(1491)五月二十七日去世,不可能在正德年间再与阳明会面。但娄谅之子娄忱亦热衷于讲学,从游者甚众。王阳明在广信时正值正月十五,是很有可能去拜访他的启蒙恩师的故地,或许当时接待他的正是娄忱,或者是娄谅的其他弟子。
\stopbuffer
\startbuffer[2-1]
《乞宥言官去权奸以章圣德疏(正德元年,时官兵部主事)》,《王阳明全集》卷九,第323---324页。疏中并未提刘瑾,但\quota{去权奸}似指刘瑾而言。据杨一清《海日先生墓志铭》:\quota{明年改元,丙寅,瑾贼窃柄,士夫侧足立,争奔走其门,求免祸。公独不往。瑾衔之。时伯安为兵部主事,疏瑾罪恶。瑾矫诏执之,几毙廷杖,窜南荒以去。瑾复移怒于公。}(《王阳明全集》卷三十八,第1535页。)王世贞考辨说:\quota{当时王公止是救南京给事中戴铣等,初与瑾无深仇,何必作此狡狯?}\quota{《双溪杂记》言:王伯安奏刘瑾,被挞几死,谪龙场驿丞,以此名闻天下。杨文襄公作《王海日公华墓志铭》,其说亦同而加详。考之国史与《王文成公年谱》《行状》《文集》,止是救南京给事中戴铣等忤刘瑾,下狱杖谪,本无所谓劾瑾也。夫以杨文襄之在吏部,用文成为属,王恭襄之在本兵,与文成若一人,而卤莽乃尔,安在其为野史家乘耶?}(王世贞《史乘考误八》,《弇山堂别集》卷二十七,中华书局,1985年,第480---481页。)
\stopbuffer
\startbuffer[2-2]
王阳明先是在静慈寺(净慈寺),后移居胜果寺。据《卧病静慈写怀》:\quota{卧病空山春复夏,山中幽事最能知。雨晴阶下泉声急,夜静松间月色迟。把卷有时眠白石,解缨随意濯清漪。吴山越峤俱堪老,正奈燕云系远思!}《移居胜果寺二首》有云:\quota{半空虚阁有云住,六月深松无暑来。病肺正思移枕簟,洗心兼得远尘埃。}\quota{病余岩阁坐朝曛,异景相新得未闻。}(《王阳明全集》卷十九,第756---757页。)
\stopbuffer
\startbuffer[2-3]
\quota{较},通\quota{校}。
\stopbuffer
\startbuffer[2-4]
王世贞录此诗及其事:\quota{寓杭州胜果寺。一夕,梦使者持书二缄付伯安,启之,一书\quota{沧浪之水清兮,可以濯我缨。伍员名。}一画水上覆一舟,后题\quota{屈平},止二字。既觉,越三日,昼见二军校至,有旨:\quota{赐汝溺,不可缓。}窘迫之,伯安恳告校曰:\quota{少间须臾,留诗于世,以俟命绝。}乃以纸展几上,题一律云:\quota{学道无成岁月虚,天乎至此复何如。身曾许国生无补,死不忘亲痛有余。自信孤忠悬日月,岂知余骨葬江鱼?百年臣子悲何极,日夜潮声泣子胥。}更有《告终词》一篇,不及录。书罢,为二校面缚,挟至江边投之。伯安初入水,即得物负之,不能沉,漂荡凡七昼夜,所见如画中。伯安惊慌,莫知所之。舟偶及岸,见一老人率四卒来,云:\quota{汝何致此狼狈?吾当为汝解缚登岸。}伯安拜谢,因问老人曰:\quota{此当何处?}老人曰:\quota{福建界也。}伯安告曰:\quota{愿公护某至彼。}老人曰:\quota{此去福建尚远,不能猝达,当送君往广信。}乃命四卒共往,舁之去如飞,不半日已抵广信矣。老人复在彼,率诣僧寺,僧闻其名,延款甚恭。伯安问僧曰:\quota{老人在何处?请来同坐。}又谓僧曰:\quota{我馁甚,乞饭少许。}且嘱先饭四卒。僧觅之,皆不见。询僧:\quota{自岸至此,为程几何?}僧曰:\quota{千里。}曰:\quota{自辰及午,迅速若是,信为神祐也。}食罢,僧达郡邑,皆馆谷之。即移文浙省,差人迎候,恍惚若梦寐中。人谓伯安志慕神仙,故堕此福地也。}\quota{异人所赠诗,后六句予能记之,云:\quota{君将性命轻毫发,谁把纲常重昆仑?寰海已知夸令德,皇天终不丧斯文。武夷山下经行处,好把椒浆荐夕曛。}疑亦王公所托言也。}(《史乘考误八》,《弇山堂别集》卷二十七,第479---480页。)
\stopbuffer
\startbuffer[2-5]
《泛海》,《王阳明全集》卷十九,第757页。诗中藏有\quota{明}\quota{夷},或与道士卜《明夷》卦有关。
\stopbuffer
\startbuffer[2-6]
冯梦龙与邹守益均提到和尚视虎穴,道士久候。黄绾则将和尚与道士合为一人,并未提及二十年南昌道士事:\quota{瑾怒未释。公行至钱塘,度或不免,乃托为投江,潜入武夷山中,决意远遁。夜至一山庵投宿,不纳。行半里许,见一古庙,遂据香案卧。黎明,道士特往视之,方熟睡。乃推醒曰:\quota{此虎狼穴也,何得无恙?}因诘公出处,公乃吐实。道士曰:\quota{如公所志,将来必有赤族之祸。}公问:\quota{何以至此?}道士曰:\quota{公既有名朝野,若果由此匿迹,将来之徒假名以鼓舞人心,朝廷寻究汝家,岂不致赤族之祸?}公深然其言。}(黄绾《阳明先生行状》,《王阳明全集》卷三十八,第1556---1557页。)
\stopbuffer
\startbuffer[2-7]
\quota{近世士夫亦有稍知求道者,皆因实德未成而先揭标榜,以来世俗之谤,是以往往隳堕无立,反为斯道之梗。诸友宜以是为鉴,刊落声华,务于切己处着实用力。前在寺中所云静坐事,非欲坐禅入定。盖因吾辈平日为事物纷拏,未知为己,欲以此补小学收放心一段功夫耳。}(《与辰中诸生》,《王阳明全集》卷四,第 162 页。)
\stopbuffer
\startbuffer[2-8]
\quota{凿石为椁},当据黄绾《阳明先生行状》:\quota{公于一切得失荣辱皆能超脱,惟生死一念,尚不能遣于心,乃为石椁,自誓曰:\quota{吾今惟俟死而已,他复何计?}日夜端居默坐,澄心精虑,以求诸静一之中。一夕,忽大悟,踊跃若狂者。}(《王阳明全集》卷三十八,第1557页。)石椁当指天然洞穴玩易窝,王阳明龙场悟道得益于《周易》,无论是在狱中、赴谪途中,还是龙场,《周易》成为王阳明的精神支柱:如狱中诗:\quota{囚居亦何事?省愆惧安饱。瞑坐玩羲《易》,洗心见微奥。}(《读易》,《王阳明全集》卷十九,第747页。)赴谪诗:\quota{羊肠亦坦道,太虚何阴晴?灯窗玩古《易》,欣然获我情。起舞还再拜,圣训垂明明。拜舞讵逾节?顿忘乐所形。敛衽复端坐,玄思窥沉溟。寒根固生意,息灰抱阳精。}(《杂诗三首》之三,《王阳明全集》卷十九,第760页。)龙场《玩易窝记》:\quota{阳明子之居夷也,穴山麓之窝而读《易》其间。始其未得也,仰而思焉,俯而疑焉,函六合,入无微,茫乎其无所指,孑乎其若株。其或得之也,沛兮其若决,瞭兮其若彻,菹淤出焉,精华入焉,若有相者而莫知其所以然。其得而玩之也,优然其休焉,充然其喜焉,油然其春生焉。精粗一,外内翕,视险若夷,而不知其夷之为阨也。于是阳明子抚几而叹曰:\quota{嗟乎!此古之君子所以甘囚奴,忘拘幽,而不知其老之将至也夫!吾知所以终吾身矣。}名其窝曰\quota{玩易}。}(《王阳明全集》卷三十八,第988---989页。)
\stopbuffer
\startbuffer[2-9]
据《孟子·尽心上》:孟子曰:\quota{人之所不学而能者,其良能也;所不虑而知者,其良知也。孩提之童,无不知爱其亲者;及其长也,无不知敬其兄也。亲亲,仁也;敬长,义也。无他,达之天下也。}
\stopbuffer
\startbuffer[2-10]
据《年谱》,\quota{三年戊辰,先生三十七岁,在贵阳}。\quota{是年始悟格物致知。}\quota{时瑾憾未已,自计得失荣辱皆能超脱,惟生死一念尚觉未化,乃为石椁自誓曰:\quota{吾惟俟命而已!}日夜端居澄默,以求静一;久之,胸中洒洒。而从者皆病,自析薪取水作糜饲之;又恐其怀抑郁,则与歌诗;又不悦,复调越曲,杂以诙笑,始能忘其为疾病夷狄患难也。因念:\quota{圣人处此,更有何道?}忽中夜大悟格物致知之旨,寤寐中若有人语之者,不觉呼跃,从者皆惊。始知圣人之道,吾性自足,向之求理于事物者,误也。}(《王阳明全集》卷三十三,第1354页。)阳明龙场悟道主要在于格物致知,次年提出知行合一,龙场悟道时并未提出良知与致良知。据钱德洪记载,王阳明自\quota{征宁藩后,专发致良知宗旨,则益明切简易矣。}(《与滁阳诸生书并问答语》跋语,《王阳明全集》卷二十六,第1083页。)正德十六年,阳明五十岁,\quota{是年先生始揭致良知之教}。(《王阳明全集》卷三十三,第1411页。)
\stopbuffer
\startbuffer[2-11]
据《五经臆说序》,\quota{夫说凡四十六卷,《经》各十,而《礼》之说尚多缺,仅六卷云}。(《王阳明全集》卷二十二,第966页。)《五经臆说》以心解经,如解释《春秋》\quota{元年春王正月}为\quota{元也者,在天为生物之仁,而在人则为心},解释《周易》之《晋》卦为\quota{心之德本无不明也,故谓之明德}。(《五经臆说十三条》,《王阳明全集》卷二十六,第1076---1079页。)
\stopbuffer
\startbuffer[2-12]
\quota{癸巳},应为\quota{己巳}。
\stopbuffer
\startbuffer[2-13]
此处引文主要出自《传习录上》(《王阳明全集》卷一,第4---5页),为正德七年(1512)与徐爱论学所述。《年谱》将阳明与徐爱论学的这段话放置在正德四年,王阳明与席书论朱陆异同之后。(《王阳明全集》卷三十三,第1355页。)冯梦龙或据此,而将王阳明与徐爱论学之语误作与席书论学。
\stopbuffer
\startbuffer[2-14]
\quota{寅宾堂},应作\quota{宾阳堂}。王阳明《宾阳堂记》云:\quota{传之堂东向曰\quota{宾阳},取《尧典》\quota{寅宾出日}之义。}(《王阳明全集》卷二十三,第986---987页。)黄绾《阳明先生行状》亦称\quota{宾阳堂}。(《王阳明全集》卷三十八,第1557页。)
\stopbuffer
\startbuffer[2-15]
《睡起写怀》,《王阳明全集》卷十九,第793页。据《年谱》,\quota{师昔还自龙场,与门人冀元亨、蒋信、唐愈贤等讲学于龙兴寺,使静坐密室,悟见心体}。(《王阳明全集》卷三十六,第1476页。)
\stopbuffer
\startbuffer[2-16]
\quota{囹圄},监狱。此据《年谱》:\quota{稽国初旧制,慎选里正三老,坐申明亭,使之委曲劝谕。民胥悔胜气嚣讼,至有涕泣而归者。由是囹圄日清。}(《王阳明全集》卷三十三,第1356页。)又见黄绾《阳明先生行状》:\quota{庚午,升庐陵知县。比至,稽国初旧制,慎选里正三老,委以词讼,公坐视其成,囹圄清虚。}(《王阳明全集》卷三十八,第1558页。)
\stopbuffer
\startbuffer[2-17]
《别方叔贤四首》之三,《王阳明全集》卷二十,第797页。
\stopbuffer
\startbuffer[2-18]
《传习录上》,《王阳明全集》卷一,第12页。
\stopbuffer
\startbuffer[2-19]
如《龙蟠山中用韵》:\quota{无奈青山处处情,村沽日日办山行。真惭廪食虚官守,只把山游作课程。}《琅琊山中三首》:\quota{《六经》散地莫收拾,丛棘被道谁刊删?已矣驱驰二三子,凤图不出吾将还。}(《王阳明全集》卷二十,第803---804页。)
\stopbuffer
\startbuffer[2-20]
据《年谱》:\quota{滁山水佳胜,先生督马政,地僻官闲,日与门人游遨琅琊、瀼泉间。月夕,则环龙潭而坐者数百人,歌声振山谷。诸生随地请正,踊跃歌舞。旧学之士皆日来臻。于是从游之众自滁始。}(《王阳明全集》卷三十三,第1363页。)
\stopbuffer
\startbuffer[2-21]
《滁阳别诸友》,《王阳明全集》卷二十,第809页。
\stopbuffer
\startbuffer[2-22]
\quota{朱虎},应为\quota{朱箎}。
\stopbuffer
\startbuffer[2-23]
《示弟立志说》作于正德十年乙亥(1515)。(《王阳明全集》卷七,第289页。)
\stopbuffer

\startbuffer[3-1]
据正德十年(1515)八月《乞养病疏》:\quota{顷来南都,寒暑失节,病遂大作。且臣自幼失母,鞠于祖母岑,今年九十有六,耄甚不可迎侍,日夜望臣一归为诀。臣之疾痛,抱此苦怀,万无生理。}(《王阳明全集》卷九,第325页。)又据正德十一年(1516)十月时升南赣佥都御史作《辞新任乞以旧职致仕疏》:\quota{臣自幼失慈,鞠于祖母岑,今年九十有七,旦暮思臣一见为诀。去岁乞休,虽迫疾病,实亦因此。臣敢辄以蝼蚁苦切之情控于陛下,冀得便道先归省视岑疾,少伸反哺之私,以俟矜允之命。臣衷情迫切,不自知其触昧条宪。臣不胜受恩感激,渎冒战惧,哀恳祈望之至!}(《王阳明全集》卷九,第330页。)
\stopbuffer
\startbuffer[3-2]
阳明正德十一年(1516)十月,归省至越时,王思舆语季本曰:\quota{阳明此行,必立事功。}本曰:\quota{何以知之?}曰:\quota{吾触之不动矣。}(《年谱》,《王阳明全集》卷三十三,第1365页。)
\stopbuffer
\startbuffer[3-3]
此事见《年谱》,《王阳明全集》卷三十三,第1366页。
\stopbuffer
\startbuffer[3-4]
此事见《年谱》,《王阳明全集》卷三十三,第1366页。
\stopbuffer
\startbuffer[3-5]
《十家牌法告谕各府父老子弟》《案行各分巡道督编十家牌》,《王阳明全集》卷十六,第587---590页。
\stopbuffer
\startbuffer[3-6]
据《选拣民兵》:\quota{夫事缓则坐纵乌合,势急乃动调狼兵,一皆苟且之谋,此岂可常之策?古之善用兵者,驱市人而使战,假闾戍以兴师。岂以一州八府之地,遂无奋勇敢战之夫?事豫则立,人存政举。}\quota{为此案仰四省各兵备官,于各属弩手、打手、机快等项,挑选骁勇绝群、胆力出众之士,每县多或十余人,少或八九辈;务求魁杰异材,缺则悬赏召募。大约江西、福建二兵备,各以五六百名为率;广东、湖广二兵备,各以四五百名为率。中间若有力能扛鼎、勇敌千人者,优其廪饩,署为将领。召募犒赏等费,皆查各属商税赃罚等银支给。}\quota{所募精兵,专随各兵备官屯扎,别选素有胆略属官员分队统押。}\quota{则各县屯戍之兵,既足以护防守截;而兵备募召之士人,可以应变出奇。}(《王阳明全集》卷十六,第585---586页。)
\stopbuffer
\startbuffer[3-7]
\quota{纪庸},应为\quota{纪镛},见《闽广捷音疏》,《王阳明全集》卷九,第336页。
\stopbuffer
\startbuffer[3-8]
据《剿捕漳寇方略牌》:\quota{于公文至日,便可扬言。本院新有明文,谓:天气向暖,农务方新,兼之山路崎险,林木蓊翳,若雨水洊至,瘴雾骤兴,军马深入,实亦非便。莫若于要紧地方,量留打手机兵,操练隄备。其余军马,逐渐抽回;待秋收之后,风气凉冷,然后三省会兵齐进。或宣示远近,或晓谕下人,此声既扬,却乃大飨军士,阳若犒劳给赏,为散军之状;实则感激众心,作兴士气。}(《王阳明全集》卷十六,第591页。)
\stopbuffer
\startbuffer[3-9]
据《钦奉敕谕切责失机官员通行各属》:\quota{参看各官顿兵不进,致此败衄,显是不奉节制,故违方略,正宜协愤同奋,因败求胜,岂可辄自退阻,倚调狼兵,坐失机会。}(《王阳明全集》卷十六,第599页。)又据《案行漳南道守巡官戴罪督兵剿贼》:\quota{今据前因,参照指挥高伟既奉差委督哨,自合与覃桓等相度机宜,协谋并进;若乃孤军轻率,中贼奸计,虽称督兵救援,先亦颇有斩获,终是功微罪大,难以赎准。广东通判陈策,指挥黄春,千百户陈洪、郑芳等,既与覃桓等面议夹攻,眼见摧败,略不应援,挫损军威,坏事非细,俱属违法。各该领兵守备、兵备、守巡等官,督提欠严,亦属有违,合就通行参究;但在紧急用人之际,姑且记罪,查勘督剿。}(《王阳明全集》卷十六,第594页。)
\stopbuffer
\startbuffer[3-10]
王阳明少年时便留意八阵图,据黄绾《阳明先生行状》,\quota{钦差督造威宁伯王公坟于河间,驭役夫以什伍之法,暇即演八阵图,识者已知其有远志。}(《王阳明全集》卷三十八,第1556页。)
\stopbuffer
\startbuffer[3-11]
参见《回军上杭》《喜雨三首》,《王阳明全集》卷二十,第821---822页;《时雨堂记》,《王阳明全集》卷二十三,第994页;《书察院行台壁》,《王阳明全集》卷二十四,第1010页。
\stopbuffer
\startbuffer[3-12]
《兵符节制》,《王阳明全集》卷十六,第601---602页。据《管子》:\quota{管子对曰:\quota{昔者,圣王之治其民也,参其国而伍其鄙,定民之居,成民之事,以为民纪,谨用其六秉,如是而民情可得,而百姓可御。}桓公曰:\quota{六秉者何也?}管子曰:\quota{杀、生、贵、贱、贫、富,此六秉也。}桓公曰:\quota{参国奈何?}管子对曰:\quota{制国以为二十一乡,商工之乡六,士农之乡十五。公帅十一乡,高子帅五乡,国子帅五乡,参国故为三军。公立三官之臣,市立三乡,工立三族,泽立三虞,山立三衡。制五家为轨,轨有长。十轨为里,里有司。四里为连,连有长。十连为乡,乡有良人。三乡一帅。}桓公曰:\quota{五鄙奈何?}管子对曰:\quota{制五家为轨,轨有长。六轨为邑,邑有司。十邑为率,率有长。十率为乡,乡有良人。三乡为属,属有帅。五属一大夫,武政听属,文政听乡,各保而听,毋有淫佚者。}}(《小匡第二十》,《管子》卷八。)
\stopbuffer
\startbuffer[3-13]
《谕俗四条》,《王阳明全集》卷二十四,第1010---1011页。
\stopbuffer
\startbuffer[3-14]
据《训蒙大意示教读刘伯颂等》:\quota{大抵童子之情,乐嬉游而惮拘检,如草木之始萌芽,舒畅之则条达,摧挠之则衰痿。今教童子,必使其趋向鼓舞,中心喜悦,则其进自不能已。譬之时雨春风,沾被卉木,莫不萌动发越,自然日长月化;若冰霜剥落,则生意萧索,日就枯槁矣。故凡诱之歌诗者,非但发其志意而已,亦所以泄其跳号呼啸于咏歌,宣其幽抑结滞于音节也;导之习礼者,非但肃其威仪而已,亦所以周旋揖让而动荡其血脉,拜起屈伸而固束其筋骸也。}(《王阳明全集》卷二,第99页。)
\stopbuffer
\startbuffer[3-15]
据《申明赏罚以励人心疏》:\quota{特敕兵部俯采下议,特假臣等令旗令牌,使得便宜行事。如是而兵有不精,贼有不灭,臣等亦无以逃其死。}(《王阳明全集》卷九,第345页。)据《攻治盗贼二策疏》:\quota{乞要申明赏罚,假臣等令旗令牌,使得便宜行事,庶几举动如意,而事功可成。}(《王阳明全集》卷九,第349页。)阳明请示王琼:\quota{惟望老先生授之以成妙之算,假之以专一之权,眀之以赏罚之典。生虽庸劣,无能为役,敢不鞭策驽钝,以期无负推举之盛心。}(《与王晋溪司马》,《王阳明全集》卷二十七,第1104页。)
\stopbuffer
\startbuffer[3-16]
《添设清平县治疏》,《王阳明全集》卷九,第353---356页。
\stopbuffer
\startbuffer[3-17]
阳明提督军务、给旗牌,得以应对随时可能出现的宁藩叛乱:\quota{先宁藩之未变,朝廷固已阴觉其谋,故改臣以提督之任,假臣以便宜之权,使据上游以制其势。故臣虽仓卒遇难,而得以从宜调兵,与之从事。当时帷幄谋议之臣,则有若大学士杨廷和等,该部调度之臣,则有若尚书王琼等,是皆有先事御备之谋,所谓发纵指示之功也。}(《辞封爵普恩赏以彰国典疏》,《王阳明全集》卷十三,第503页。)
\stopbuffer
\startbuffer[3-18]
据《横水桶冈捷音疏》:\quota{议得桶冈、横水、左溪诸贼,荼毒三省,其患虽同,而事势各异。以湖广言之,则桶冈诸巢为贼之咽喉,而横水、左溪诸巢为之腹心;以江西言之,则横水、左溪诸巢为贼之腹心,而桶冈诸巢为之羽翼。今不先去横水、左溪腹心之患,而欲与湖广夹攻桶冈,进兵两寇之间,腹背受敌,势必不利。今议者纷纷,皆以为必须先攻桶冈,而湖广克期乃在十一月初一日,贼见我兵未集,而师期尚远,且以为必先桶冈,势必观望未备。今若出其不意,进兵速击,可以得志。已破横水、左溪,移兵而临桶冈,破竹之势,蔑不济矣。}(《王阳明全集》卷十,第381页。)
\stopbuffer
\startbuffer[3-19]
参见《征剿横水桶冈分委统哨牌》,《王阳明全集》卷十六,第609---616页。其中\quota{石人坑}应为\quota{石坑};\quota{黄雀坳}应为\quota{黄径坳}。
\stopbuffer
\startbuffer[3-20]
阳明在战场上表现出超常的精力,据后学所言,与其有特殊的休养生息的方法有关。王慎中问王畿:\quota{先师在军中四十日未尝睡,有诸?}王畿曰:\quota{然。此原是圣学。古人有息无睡,故曰:\quota{向晦入燕息。}世人终日扰扰,全赖后天渣滓厚味培养,方彀一日之用。夜间全赖一觉熟睡方能休息。不知此一觉熟睡,阳光尽为阴浊所陷,如死人一般。若知燕息之法,当向晦时,耳无闻,目无见,口无吐纳,鼻无呼吸,手足无动静,心无私累,一点元神,与先天清气相依相息,如炉中种火相似,比之后天昏气所养,奚啻什百。是谓通乎昼夜之道而知。}(《三山丽泽录》,《王畿集》卷一,凤凰出版社,2007年,第13页。)
\stopbuffer
\startbuffer[3-21]
参见《年谱》,《王阳明全集》卷三十三,第1376页。
\stopbuffer
\startbuffer[3-22]
参见《浰头捷音疏》,《王阳明全集》卷十一,第400---401页。
\stopbuffer
\startbuffer[3-23]
\quota{牛马驴一百八只},应为\quota{牛马骡六百八只匹}。参见《横水桶冈捷音疏》,《王阳明全集》卷十,第387页。另据《平茶寮碑》:\quota{凡破巢八十有四,擒斩三千余,俘三千六百有奇。释其胁从千有余众,归流亡,使复业。}(《王阳明全集》卷二十五,第1043---1044页。)
\stopbuffer
\startbuffer[3-24]
\quota{九十三人}应为\quota{四十余人},参见《浰头捷音疏》,《王阳明全集》卷十一,第402页。另见《平浰头碑》:\quota{丁未,破三浰,乘胜归北。大小三十余战,灭巢三十有八,俘斩三千余。}(《王阳明全集》卷二十五,第1044页。)
\stopbuffer
\startbuffer[3-25]
阳明征剿横水、桶冈、浰头时,是带病上阵的。据正德十三年(1518)《乞休致疏》:\quota{自去岁二月往征闽寇,五月旋师;六月至于九月,俱有地方之警;十月攻横水,十一月破桶冈,十二月旋师;未几,今年正月又复出剿浰贼。前后一岁有余,往来二三千里之内,上下溪涧,出入险阻,皆扶病从事。}\quota{但惟臣病月深日亟,百疗罔效,潮热咳嗽,疮疽痈肿,手足麻痹,已成废人。}(《王阳明全集》卷十一,第392---393页。)
\stopbuffer
\startbuffer[3-26]
参见《浰头捷音疏》,《王阳明全集》卷十一,第402---403页。
\stopbuffer
\startbuffer[3-27]
参见《犒赏新民牌》,《王阳明全集》卷三十,第1198---1199页;《浰头捷音疏》,《王阳明全集》卷十一,第405页。
\stopbuffer
\startbuffer[3-28]
参见《添设和平县治疏》,《王阳明全集》卷十一,第407---412页;《再议平和县治疏》,《王阳明全集》卷十一,第423---426页。按:和平县后改称平和县。
\stopbuffer
\startbuffer[3-29]
参见《辞免升荫乞以原职致仕疏》,《王阳明全集》卷十一,第417---419页。正德十四年正月十四日再上《乞放归田里疏》,恳请致仕,均未允。(《王阳明全集》卷十一,第431---433页。)
\stopbuffer
\startbuffer[3-30]
《大学古本序》作于正德十八年戊寅(1518)。(《王阳明全集》卷七,第270---271页。)据《朱子晚年定论》序,该文作于正德(1515)乙亥冬十一月。(《王阳明全集》卷三,第145页。)
\stopbuffer
\startbuffer[3-31]
徐爱为正德三年(1508)进士,自正德七年开始记录阳明讲学语录,正德十三年(1518)卒。据《年谱》,正德十三年,\quota{八月门人薛侃刻《传习录》。侃得徐爱所遗《传习录》一卷,序二篇,与陆澄各录一卷,刻于虔}。(《王阳明全集》卷三十三,第1385页。)
\stopbuffer
\startbuffer[3-32]
正德六年辛未(1511),邹守益\quota{会试第一。先是,文成王公移令庐陵,先生慕而谒之,一见期许。是岁,王公以吏部主事司分校},\quota{遂冠南宫}。正德十四年己卯(1519),邹守益\quota{就质王公于虔台},王阳明为其开示:\quota{致知者,致吾心之良知于事事物物也。致吾心之良知于事事物物,则事事物物皆得其理矣。独即所谓良知也。慎独者,所以致其良知也。戒谨恐惧,所以慎其独也。《大学》《中庸》之旨一也。}邹守益豁然有悟,遂执贽师事焉。(耿定向《东廓邹先生传》,《耿天台先生文集》卷十四,明万历刻本,第10---11页。)
\stopbuffer
\startbuffer[3-33]
《三箴》,见《王阳明全集》卷二十五,第1046---1047页。
\stopbuffer
\startbuffer[3-34]
\quota{读书讲学,此最吾所宿好,今虽干戈扰攘中,四方有来学者,吾亦未尝拒之。}(《赣州书示四侄正思等》,《王阳明全集》卷二十六,第1088页。)
\stopbuffer
\startbuffer[3-35]
《与杨仕德薛尚谦》,《王阳明全集》卷四,第188页。此书作于正德十二年(1517)。
\stopbuffer
\startbuffer[3-36]
《传习录下》,《王阳明全集》卷三,第121页。阳明与仙家亦有一则传奇,据《虔台梦》:\quota{阳明先生在赣州,都府军令甚严,宿卫之士无敢偶语离次者。一夕,于中夜,卫士偶见府门洞开,有一道流自外至,长髯蕉扇,俨如洞宾。一童子执纱灯前导以入,门复闭。久之,开门送出,长揖别去,甚速,不知所之。见者惊愕,门如故。天明,遂相传言。自守巡以下皆知之。已而守巡入揖,先生遂自言梦纯阳真人来访,\quota{吾问:\quota{如何谓之仙?}彼曰:\quota{非儒之至者,不足以称真仙。}吾又问:\quota{如何谓之儒?}曰:\quota{非仙之至者,不足以言真儒。}良久别去。}守巡乃敢言夜来卫士所见,始知纯阳之果至也。}(董穀《碧里杂存》,第15页。)
\stopbuffer

\startbuffer[4-1]
\quota{秦荣}应为\quota{秦溁}。据《年谱》,\quota{初,宁献王臞仙传惠、靖、康三王,康王久无子,宫人南昌冯氏以成化丁酉生濠。康王梦蛇入宫,啖人殆尽,心恶之,欲弗举,以内人争免,遂匿优人家,与秦溁同寝处。稍长,淫宫中。康王忧愤且死,不令入诀。弘治丙辰袭位,通书史歌词。}(《王阳明全集》卷三十四,第1391页。)
\stopbuffer
\startbuffer[4-2]
\quota{胡十三}应为\quota{吴十三},\quota{闵廿四}又作\quota{闵念四}。据《擒获宸濠捷音疏》:\quota{兼又招纳叛亡,诱致剧贼渠魁如吴十三、凌十一之属,牵引数千余众,召募四方武艺骁勇、力能拔树排关者亦万有余徒。}(《王阳明全集》卷十二,第449页。)另据《案行浙江按察司交割逆犯暂留养病》:\quota{擒获宁王宸濠及逆党李士实、刘养正、王春等,贼首吴十三、凌十一、闵念四、吴国七、闵念八等。}(《王阳明全集》卷十七,第654页。)
\stopbuffer
\startbuffer[4-3]
\quota{媒孽},媒为酒母,孽为曲。酝酿之意,比喻构陷诬害,酿成其罪。出自《汉书·司马迁传》:\quota{今举事一不当,而全躯保妻子之臣随而媒孽其短,仆诚私心痛之。}臣瓒曰:\quota{媒谓遘合会之,孽谓为生其罪舋也。}颜师古曰:\quota{媒如媒娉之媒,孽如曲孽之孽。一曰齐人谓曲饼为媒也。}(《汉书》卷六十二,中华书局,1962年,第 2729---2731 页。)
\stopbuffer
\startbuffer[4-4]
冀元亨,湖广常德府武陵县举人,王阳明谪戍龙场,冀元亨曾从讲学。阳明至南赣,延之教子。\quota{偶值宸濠饰诈要名,礼贤求学,本职因使本生乘机往见宸濠,冀得因事纳规,开陈大义,沮其邪谋;如其不可劝喻,亦因得以审察动静,知其叛逆迟速之机,庶可密为御备。本生既与相见,议论大相矛盾,宸濠以本职所遣,一时虽亦含忍遣发,而毒怒不已,阴使恶党,四出访缉,欲加陷害;本生素性愿恪,初不之知,而本职风闻其说,当遣密从间道潜回常德,以避其祸。}(《咨六部伸理冀元亨》,《王阳明全集》卷十七,第673页。)另据《书佛郎机遗事》:\quota{初,予尝使门人冀元亨者因讲学说濠以君臣大义,或格其奸。濠不怿,已而滋怒,遣人阴购害之。冀辞予曰:\quota{濠必反,先生宜早计。}遂遁归。}(《王阳明全集》卷二十四,第1015页。)
\stopbuffer
\startbuffer[4-5]
娄妃为阳明的启蒙老师娄谅之女,据《年谱》:\quota{娄为谅女,有家学,故处变能自全。}(《王阳明全集》卷三十四,第1398 页。)
\stopbuffer
\startbuffer[4-6]
\quota{陈金},应为\quota{秦金}。事见钱德洪《征宸濠反间遗事》,《王阳明全集》卷三十九,第1627页。
\stopbuffer
\startbuffer[4-7]
宸濠一疑,为阳明赢得了战机。据龙光对钱德洪所言,\quota{昔夫子写杨公火牌将发时,雷济问曰:\quota{宁王见此恐未必信。}曰:\quota{不信,可疑否?}对曰:\quota{疑则不免。}夫子笑曰:\quota{得渠一疑,彼之大事去矣。}既而叹曰:\quota{宸濠素行无道,残害百姓,今虽一时从逆者众,必非本心,徒以威劫利诱,苟一时之合耳。纵使奋兵前去,我以问罪之师徐蹑其后,顺逆之势既判,胜负预可知也。但贼兵早越一方,遂破残一方民命。虎兕出柙,收之遂难。为今之计,只是迟留宸濠一日不出,则天下实受一日之福。}}(钱德洪《征宸濠反间遗事》,《王阳明全集》卷三十九,第1630页。)
\stopbuffer
\startbuffer[4-8]
第五哨\quota{章丘门}应为\quota{章江门},第六哨\quota{李缉}应为\quota{李楫}。据王阳明《牌行各哨统兵官进攻屯守》,分军为十二哨,其中第七哨分列\quota{李美}与\quota{余恩};冯梦龙将\quota{余恩}列为第八哨,以后依次相差一哨。(《王阳明全集》卷十七,第642---643页。)另见《江西捷音疏》,《王阳明全集》卷十二,第441---442页。
\stopbuffer
\startbuffer[4-9]
《告七门从逆军民》,《王阳明全集》卷十七,第645页。
\stopbuffer
\startbuffer[4-10]
\quota{夫子应变之神真不可测。时官兵方破省城,忽传令造免死木牌数十万,莫知所用。及发兵迎击宸濠于湖上,取木牌顺流放下。时贼兵既闻省城已破,胁从之众俱欲逃窜无路,见水浮木牌,一时争取,散去不计其数。二十五日,贼势尚锐,值风不便,我兵少挫。夫子急令斩取先却者头。知府伍文定等立于铳炮之间,方奋督各兵,殊死抵战。贼兵忽见一大牌书:\quota{宁王已擒,我军毋得纵杀!}一时惊扰,遂大溃。}(钱德洪《征宸濠反间遗事》,《王阳明全集》卷三十九,第1631页。)
\stopbuffer
\startbuffer[4-11]
《书佛郎机遗事》,《王阳明全集》卷二十四,第1014---1015页。
\stopbuffer
\startbuffer[4-12]
《鄱阳战捷》,《王阳明全集》卷二十,第830页。
\stopbuffer
\startbuffer[4-1-0]
王阳明遇北风之事见《飞报宁王谋反疏》,\quota{方尔回程,随有兵卒千余已夹江并进,前来追臣。偶遇北风大作,臣亦张疑设计,整舟安行,兵不敢逼,幸而获免。}(《王阳明全集》卷十二,第434页。)据《咨两广总制都御史杨共勤国难》,\quota{于六月初九日自赣启行,于本月十五日行至丰城县地名黄土脑},\quota{各官竞阻本职,不宜轻进。本职自顾单旅危途,势难复进,方尔回程,随有兵卒千余已夹江并进来追,偶遇北风大作,本职亦张疑设计,整舟安行,兵不敢逼,幸而获免。}(《王阳明全集》卷十七,第634页。)一为向皇帝的奏折,一为与同僚的咨文,王阳明均称是偶遇北风。

据龙光所述,\quota{是年六月十五日,公于丰城闻宸濠之变。时参谋雷济、萧禹在侍,相与拜天誓死,起兵讨贼。欲趋还吉安,南风正急,舟不能动。又痛哭告天,顷之,得北风。宸濠追兵将及,潜入小渔船,与济等同载,得脱免。}(钱德洪《征宸濠反间遗事》,《王阳明全集》卷三十九,第1626页。)

与阳明随行的雷济叙述此事较详:\quota{夫子昔在丰城闻变,南风正急,拜天哭告曰:\quota{天若悯恻百万民命,幸假我一帆风!}须臾风稍定,顷之,舟人欢噪回风。济、禹取香烟试之舟上,果然。久之,北风大作。宸濠追兵将及时,夫人、公子在舟。夫子呼一小渔船自缚,敕令济、禹持米二斗,脔鱼五寸,与夫人为别。将发,问济曰:\quota{行备否?}济、禹对曰:\quota{已备。}夫子笑曰:\quota{还少一物。}济、禹思之不得。夫子指船头罗盖曰:\quota{到地方无此,何以示信?}于是又取罗盖以行。
明日至吉安城下,城门方戒严,舟不得泊岸。济、禹揭罗盖以示,城中遂欢庆曰:\quota{王爷爷还矣。}乃开门罗拜迎入。于是济、禹心叹危迫之时,暇裕乃如此。}(钱德洪《征宸濠反间遗事》,《王阳明全集》卷三十九,第1631页。)事后,阳明曾有诗追忆拜北风:《丰城阻风(前岁遇难于此,得北风幸免)》:\quota{北风休叹北船穷,此地曾经拜北风。勾践敢忘尝胆地?齐威长忆射钩功。桥边黄石机先授,海上陶朱意颇同。况是倚门衰白甚,岁寒茅屋万山中。}(《王阳明全集》卷二十,第846页。)阳明晚年平思田之乱时,《重登黄土脑》:\quota{一上高原感慨重,千山落木正无穷。前途且与停西日,此地曾经拜北风。剑气晩横秋色净,兵声寒带暮江雄。水南多少流亡屋,尚诉征求杼轴空。}(《王阳明全集》卷二十,第877页。)
\stopbuffer
\startbuffer[5-1]
《书草萍驿二首》之一,《王阳明全集》卷二十,第830页。
\stopbuffer
\startbuffer[5-2]
《重游化城寺二首》之一,《王阳明全集》卷二十,第851页。
\stopbuffer
\startbuffer[5-3]
《有僧坐岩中已三年诗以励吾党》,《王阳明全集》卷二十,第854页。
\stopbuffer
\startbuffer[5-4]
\quota{皆由世儒认理为内}原作\quota{皆由世儒认理为外},据《王阳明全集》改。(《答罗整庵少宰书》,《王阳明全集》卷二,第84---89页。)
\stopbuffer
\startbuffer[5-5]
《青原山次黄山谷韵》,《王阳明全集》卷二十,第858---859页。
\stopbuffer
\startbuffer[5-6]
朱宸濠诬陷冀元亨,借以攻击阳明:\quota{后宸濠既败,痛恨本职起兵攻剿,虽反噬之心无所不至,而天理公道所在,无因得遂其奸;乃以本生系本职素所爱厚之人,辄肆诋诬,谓与同谋,将以泄其仇愤。且本生既与同谋,则宸濠举叛之日,本生何故不与共事,却乃反回常德,聚徒讲学?宸濠素所同谋之人如李士实、刘养正、王春之流,宸濠曾不一及,而独口称本生与之造始,此其挟仇妄指,盖有不待辩说行道之人皆能知者。}\quota{乃今身陷俘囚,妻子奴虏,家业荡尽,宗族遭殃。信奸人之口,为叛贼泄愤报仇,此本职之所为痛心刻骨,日夜冤愤而不能自已者也。本职义当与之同死。}(《咨六部伸理冀元亨》,《王阳明全集》卷十七,第673---674页。)
\stopbuffer
\startbuffer[5-7]
冀元亨之死令阳明十分痛心:\quota{复有举人冀元亨者,为臣劝说宁濠,反为奸党抅陷,竟死狱中。以忠受祸,为贼报仇,抱冤赍恨,实由于臣。虽尽削臣职,移报元亨,亦无以赎此痛。此尤伤心惨目,负之于冥冥之中者。}(《辞封爵普恩赏以彰国典疏》,《王阳明全集》卷十三,第504页。)另见《仰湖广布按二司优恤冀元亨家属》,《王阳明全集》卷十七,第685页。
\stopbuffer
\startbuffer[6-0-1]
王艮是泰州学派的开创者,\quota{阳明先生之学,有泰州、龙溪而风行天下,亦因泰州、龙溪而渐失其传。}(《明儒学案》卷三十二,《黄宗羲全集》第7册,第820页。)泰州学派、浙中王门最能鼓舞人心、传播最快,但衰落也最快。

阳明收王艮是阳明学发展过程中的重要事件。据《明儒学案》:\quota{王艮字汝止,号心斋,泰州之安丰场人。}\quota{先生虽不得专功于学,然默默参究,以经证悟,以悟释经,历有年所,人莫能窥其际也。一夕梦天堕压身,万人奔号求救,先生举臂起之,视其日月星辰失次,复手整之。觉而汗溢如雨,心体洞彻。记曰:正德六年间,居仁三月半。自此行住语默皆在觉中,乃按《礼经》制五常冠、深衣、大带、笏板,服之。曰:\quota{言尧之言,行尧之行,而不服尧之服,可乎?}时阳明巡抚江西,讲良知之学。大江之南,学者翕然信从,顾先生僻处,未之闻也。有黄文刚者,吉安人,而寓泰州,闻先生论,诧曰:\quota{此绝类王巡抚之谈学也。}先生喜曰:\quota{有是哉!虽然,王公论良知,艮谈格物,如其同也,是天以王公与天下后世也;如其异也,是天以艮与王公也。}即日启行,以古服进见,至中门,举笏而立。阳明出迎于门外。始入,先生据上坐,辩难久之,稍心折,移其坐于侧。论毕,乃叹曰:\quota{简易直截,艮不及也。}下拜自称弟子。退而绎所闻,间有不合,悔曰:\quota{吾轻易矣。}明日入见,且告之悔。阳明曰:\quota{善哉!子之不轻信从也。}先生复上坐,辩难久之,始大服,遂为弟子如初。阳明谓门人曰:\quota{向者吾擒宸濠,一无所动,今却为斯人动矣!}}(《明儒学案》卷三十二,《黄宗羲全集》第7册,第828---829页。)
\stopbuffer
\startbuffer[6-0-2]
霍韬认为阳明破八寨、断藤峡有八善,但是成功后也遭遇了类似擒宸濠一样的不公正待遇:

\quota{臣等是以叹服王守仁能体陛下之仁,以怀绥田州、思恩向化之民;又能体陛下之义,以讨服八寨、断藤峡梗化之贼也。仁义之用,两得之也。

谨按王守仁之成功有八善焉:乘湖兵归路之便,则兵不调而自集,一也。因田州、思恩效命之助,则劳而不怨,二也。机出意外,贼不及遁,所诛者真,积年渠恶,非往年滥杀报功者比,三也。因归师讨逆贼,无粮运之费,四也。不役民兵,不募民马,一举成功,民不知扰,五也。平八寨,平断藤峡,则极恶者先诛,其细小巢穴可渐施德化,使去贼从良,得抚剿之宜,六也。八寨不平,则西而柳、庆,东而罗旁、绿水、新宁、恩平之贼合数千里,共为窟穴,虽调兵数十万,费粮数百万,未易平伏。今八寨平定,则诸贼可以渐次抚剿,两广良民可渐安生业,纾圣明南顾之忧,七也。韩雍虽平断藤峡贼矣,旋复有贼者,实当尔时未及区画其地,为经久图,俾余贼复据为巢穴故也。今五十年生聚,则贼复炽盛也亦宜。若八寨乃百六十年所不能诛之剧贼,山川天险尤难为功,今守仁既平其巢窟,即徙建城邑以镇定之,则恶贼失险,后日固不能为变,逋贼来归,不日且化为良民矣。诛恶绥良,得民父母之体,八也。}

\quota{先是正德十四年,宸濠谋反江西,两司俯首从贼,惟王守仁同御史伍希儒、谢源誓心效忠。不幸奸臣张忠、许泰等欲掩王守仁之功以为己有,乃扬诸人曰:\quota{王守仁初同贼谋。}及公论难掩,乃又曰:\quota{宸濠金帛俱王守仁、伍希儒、谢源满载以去。}当时大学士杨廷和,尚书乔宇,亦忌王守仁之功,遂不与辨白而黜伍希儒、谢源,俾落仕籍。王守仁不辨之谤,至今未雪,可谓黯哑之冤矣。}(《地方疏》,《王阳明全集》卷三十九,第1623---1625页。)
\stopbuffer
\startbuffer[6-1]
《乞便道归省疏》,《王阳明全集》卷十三,第501---502页。
\stopbuffer
\startbuffer[6-2]
据陆深《海日先生行状》:先生素闻宁濠之恶,疑其乱,尝私谓所亲曰:\quota{异时天下之祸,必自兹人始矣。}令家人卜地于上虞之龙溪,使其族人之居溪傍者买田筑室,潜为栖遁之计。至是正德己卯,宁濠果发兵为变。远近传闻骇愕,且谓新建公亦以遇害,尽室惊惶,请徙龙溪。先生曰:\quota{吾往岁为龙溪之卜,以有老母在耳。今老母已入土,使吾儿果不幸遇害,吾何所逃于天地乎?}饬家人勿轻语动。已而新建起兵之檄至,亲朋皆来贺,益劝先生宜速逃龙溪。咸谓新建既与濠为敌,其势必阴使奸人来不利于公。先生笑曰:\quota{吾儿能弃家杀贼,吾乃独先去以为民望乎?祖宗德泽在天下,必不使残贼覆乱宗国,行见其败也。吾为国大臣,恨已老,不能荷戈首敌。倘不幸胜负之算不可期,犹将与乡里子弟共死此城耳。}因使趣郡县宜急调兵粮,且禁讹言,勿令揺动。乡人来窃视先生,方晏然如平居,亦皆稍稍复定。不旬月,新建捷至,果如先生所料。亲朋皆携酒交庆。先生曰:\quota{此祖宗深仁厚泽,渐渍人心,纪纲法度,维持周密,朝廷威灵,震慑四海,苍生不当罹此荼毒。故旬月之间,罪人斯得,皆天意也。岂吾一书生所能办此哉?然吾以垂尽之年,幸免委填沟壑;家门无夷戮之惨;乡里子弟又皆得免于征输调发;吾儿幸全首领,父子相见有日;凡此皆足以稍慰目前者也。}诸亲友咸喜,极饮尽欢而罢。(《王阳明全集》卷三十八,第1551---1552页。)
\stopbuffer
\startbuffer[6-3]
据《乞便道归省疏》:\quota{顾臣父既老且病,顷遭谗抅之厄,危疑震恐,汹汹朝夕,常有父子不及相见之痛。今幸脱洗殃咎,复睹天日,父子之情固思一见颜面,以叙其悲惨离隔之怀,以尽菽水欢欣之乐。}(《王阳明全集》卷十三,第501---502页。)
\stopbuffer
\startbuffer[6-4]
《归兴二首》之一,《王阳明全集》卷二十,第863---864页。
\stopbuffer
\startbuffer[6-5]
得旨:\quota{江西反贼剿平,地方安定,各该官员功绩显著,你部里既会官集议,分别等第明白,王守仁封伯爵,给与诰券,子孙世世承袭,照旧参赞机务,钦此。}\quota{王守仁封新建伯,奉天翊卫推诚宣力守正文臣,特进光禄大夫柱国,还兼南京兵部尚书,照旧参赞机务,岁支禄米一千石,三代并妻一体追封,钦此。}(《辞封爵普恩赏以彰国典疏》,《王阳明全集》卷十三,第503页。)
\stopbuffer
\startbuffer[6-6]
\quota{属纩},丧礼,代指\quota{临终}。据《礼记·丧大记》:\quota{属纩以俟绝气。}郑玄注:\quota{纩,今之新绵,易动摇,置口鼻之上以为候。}(《礼记注疏》卷四十四。)
\stopbuffer
\startbuffer[6-7]
金克厚为浙江台州仙居人,温雅忠厚。据《年谱》:\quota{门人子弟纪丧,因才任使。以仙居金克厚谨恪,使监厨。克厚出纳品物惟谨,有不慎者追还之,内外井井。}金克厚与钱德洪同贡于乡,连举进士,谓洪曰:\quota{吾学得司厨而大益,且私之以取科第。先生尝谓学必操事而后实,诚至教也。}(《王阳明全集》卷三十五,第1418页。)
\stopbuffer
\startbuffer[6-8]
\quota{谦之},原作\quota{尚谦},误。阳明弟子中,薛侃,字尚谦,广东揭阳人,为闽粤王门之代表人物;邹守益,字谦之,江西安福人,为江右王门的领袖。此下所引诗乃阳明为邹守益所作,见《次谦之韵》,《王阳明全集》卷二十,第864页。
\stopbuffer
\startbuffer[6-9]
何秦为阳明的重要弟子。时人语阳明弟子\quota{江有何、黄,浙有钱、王}。(《江右王门学案四》,《明儒学案》卷十九,《黄宗羲全集》第7册,第521页。)何、黄指何秦与黄弘纲,属江右王门;钱、王指钱德洪与王畿,属浙中王门,钱德洪与王畿二人于天泉桥向阳明请教四有、四无的问道,史称天泉问道。
\stopbuffer
\startbuffer[6-10]
\quota{黄弘纲},原作\quota{黄竹纲},误。
\stopbuffer
\startbuffer[6-11]
《论语·里仁》:\quota{子曰:\quota{君子喻于义,小人喻于利。}}
\stopbuffer
\startbuffer[6-12]
王畿即王龙溪,是浙中王门的领军人物。据《明儒学案》:魏良器,字师颜,号药湖。洪都从学之后,随阳明至越。时龙溪为诸生,落魄不羁,每见方巾中衣往来讲学者窃骂之。居与阳明邻,不见也。先生多方诱之。一日,先生与同门友投壶雅歌,龙溪过而见之,曰:\quota{腐儒亦为是耶!}先生答曰:\quota{吾等为学,未尝担板,汝自不知耳。}龙溪于是稍相昵就,已而有味乎其言,遂北面阳明。绪山临事多滞,则戒之曰:\quota{心何不洒脱?}龙溪工夫懒散,则戒之曰:\quota{心何不严慄?}其不为姑息如此。\ldots{}\ldots{}龙溪与先生最称莫逆,然龙溪之玄远,不如先生之浅近也。(《明儒学案》卷十九,《黄宗羲全集》第7册,第535---536页。)
\stopbuffer
\startbuffer[6-13]
\quota{复初书院},原作\quota{复古书院},误。据邹守益《广德州新修复初书院记》:\quota{嘉靖丙戌秋七月,新作复初书院成。}\quota{若知复初之义乎?天地之中,而民实受之。}《六经》、圣人之学的主旨在于复元气,复天地之中。(《邹守益集》卷六,凤凰出版社,2007年,第315---317页。)另据《复初书院讲章》:\quota{书院告成,以复初为第一义。}(《邹守益集》卷十五,第722页。)阳明赞之曰:\quota{书院记文,整严精确,迥尔不群,皆是直写胸中实见,一洗近儒影响雕饰之习,不徒作矣。}(《寄邹谦之》,《王阳明全集》卷六,第228页。)
\stopbuffer
\startbuffer[6-14]
据《田石谣》:\quota{阳明先生既平田州之乱。先是,田州有一巨石,谓之田石,侧卧江浒。旧有童谣云:\quota{田石倾,田州兵。田石平,田州宁。}岑猛闻而恶之,乃夜遣人平之。明复如故。如是再三,终欹侧也。自先生定乱之后,此石平矣。先生自往观之,命洗剔苔秽,见有古刻\quota{新建伯}三大字于其上,亦异矣。先生遂续加九字,并刻之云:\quota{嘉靖岁戊子春,新建伯王守仁。}因奏改为田宁府云。}(董穀《碧里杂存》,第14---15页。)
\stopbuffer
\startbuffer[6-15]
《谒伏波庙二首》之一,《王阳明全集》卷二十,第878页。
\stopbuffer
\startbuffer[6-16]
较,同\quota{校}。据《处置平复地方以图久安疏》:\quota{田州新服,用夏变夷,宜有学校。但疮痍逃窜之余,尚无受廛之民,焉有入学之士。况斋膳廪饩,俱无所出,即欲建学,亦为徒劳。然风化之原,终不可缓。臣等议欲于附近府州县学教官之内,令提学官选委一员,暂领田州学事,听各学生徒之愿改田州府学及各处儒生之愿来田州附籍入学者,皆令寄名其间。所委教官,时至其地相与讲肄游息,或于民间兴起孝弟,或倡远近举行乡约,随事开引,渐为之兆。俟休养生息一二年后,流移尽归,商旅凑集,民居已觉既庶,财力渐有可为,则如学校及阴阳医学之类,典制之所宜备者,皆听该府官以次举行上请,然后为之设官定制。如此,则施为有渐而民不知扰,似亦招徕填实之道,鼓舞作新之机也。均乞圣明裁处。}(《王阳明全集》卷十四,第545页。)
\stopbuffer
\startbuffer[6-17]
据嘉靖戊子(1528)阳明在信中所言:\quota{自入广来,精神顿衰。虽因病患侵凌,水土不服,要亦中年以后之人,其势亦自然至此,以是怀归之念日切。诚恐坐废日月,上无益于国家,下无以发明此学,竟成虚度此生耳,奈何奈何!}(《与黄宗贤》第五书,《王阳明全集》卷二十一,第917页。)
\stopbuffer
\startbuffer[6-18]
有一则关于阳明的传说,所对应的即是此时此地。明人魏浚(1553---1625,万历三十二年进士)所著的《峤南琐记》载:\quota{王伯安平思、田、八寨,即乞病归。至南安,小憩一佛寺。寺有靖室,乃前老僧示寂处。老僧化时戒其徒岁加封识,不许开户,伯安固强开之。中有书云:\quota{五十七年王守仁,启吾钥,拂吾尘,问公欲识前程事,开门即是闭门人。}伯安愕然,数日遂卒。}(魏浚:《峤南琐记》卷下,收入《四库存目丛书·子部》第243册,第558页。)蒋一葵(万历二十二年举人)所记与此稍有差异:\quota{王阳明尝游僧寺,见一室封锁甚密,欲开视之,寺僧不可,云:\quota{中有入定僧,闭门五十年矣。}阳明固开视之,见龛中坐一僧,俨然如生,貌酷肖己。先生曰:\quota{此岂吾之前身乎?}既而见壁间一诗云:\quota{五十年前王守仁,开门原是闭门人。精灵剥后还归复,始信禅门不坏身。}先生怅然久之,建塔以瘗而去。}(蒋一葵:《尧山堂外纪》卷九十《国朝》,收入《续修四库全书·子部》第1195册,第116页。)明末遗民张怡所著《玉光剑气集》卷三十《杂记》所载与此基本相同,其后又云\quota{此事载阳明本传,徐华亭删去,为从祀计也}。按,徐华亭即徐阶,松江华亭人,受业于聂豹,官至大学士,有力推动了阳明学的兴盛。从祀,指奏请以王阳明等从祀孔庙事。
\stopbuffer
\startbuffer[6-19]
据黄绾《阳明先生行状》,\quota{十月初十日,复上疏乞骸骨,就医养病。因荐林富自代。又一月,乃班师。至大庾岭,谓布政使王公大用曰:\quota{尔知孔明之所以付托姜维乎?}大用遂领兵拥护,为敦匠事。廿九日至南康县,将属纩,家童问何所嘱。公曰:\quota{他无所念,平生学问方才见得数分,未能与吾党共成之,为可恨耳!}遂逝。}(《王阳明全集》卷三十八,第1579页。)
\stopbuffer
\startbuffer[6-20]
王阳明去世后,\quota{江西男妇皆缟衣匍匐,攀舟而号}。\quota{特是部使丧归,例有赠恤,而前不予祭,后不予葬。詹事黄绾上言:\quota{守仁在前朝颇效微功,今平蛮甫竣,舆疾办事,而客死道路,妻孥孱弱,(子三岁。)家僮载骨,藁葬空山,鬼神有知,亦无不忍。}少师徐阶尝为江西督学使,深知公冤,有云:\quota{以死勤事,则祀之。今勤事之人以尸归国,而不令所司奠一杯,尚望祀乎?}}(毛奇龄《王文成传本》卷二,第12---13页。)其中黄绾上疏为《明是非定赏罚疏》,\quota{家僮载骨}后为\quota{藁殡空山,见者为之流涕,闻者为之酸心。若使鬼神有知,亦当为之夜哭矣。}(《黄绾集》卷三十二,上海古籍出版社,2014年,第628页。)
\stopbuffer
\startbuffer[6-21]
\quota{绯袍玉带}神人与王阳明出生时岑夫人梦\quota{衣绯腰玉}神人呼应。
\stopbuffer
\startbuffer[6-22]
据《明通鉴》卷六十八:\quota{万历十二年(1584),诏以陈献章、胡居仁、王守仁从祀孔庙。大学士申时行等言:\quota{守仁言致知出于《大学》,良知出于《孟子》,陈献章言主静,则为沿袭宋儒周敦颐、程颐。且孝弟出处如献章,文章功业如守仁,纯言笃行如胡居仁,故三人并宜从祀。}从之,乃以三人并从祀两庑,列于薛瑄之次。}(夏燮《明通鉴》,岳麓书社,1999年,第2418页。)
\stopbuffer
\startbuffer[6-23]
据《年谱》,十一月十一日葬先生于洪溪。(《王阳明全集》,第1466页。)
\stopbuffer
\starttext

\definecover[王阳明图传]
\setupcover [王阳明图传] [
  list=book,
  cover=flyleaf,
  book={王阳明图传},
  bookstyle=\hwe,
  bookalign=center]
\setuptextbackground [王阳明图传] [backgroundcolor=transparent]
\makecover           [王阳明图传]

\startfrontmatter
\completecontent
\stopfrontmatter
\startbodymatter
\externalfigure[imgs/王阳明图传/cover00152.jpeg]

\startsectionlevel[preface][title={前言}]%

  \startsectionlevel[preface][title={一}]%

王阳明(1472---1529),名守仁,字伯安,号阳明子,浙江余姚人,明代大儒,以\quota{立德、立功、立言}真三不朽著称。本书是一本独特的王阳明传,将明代冯梦龙的《皇明大儒王阳明先生出身靖乱录》(以下简称《靖乱录》)、邹守益的《王阳明先生图谱》(以下简称《图谱》)合而为一,并加详注而成。冯梦龙文笔超绝,《靖乱录》是推动阳明传奇故事广泛流传的重要媒介。《靖乱录》原文包括正文和眉批,本书予以标点分章分段,并将《图谱》中全部29幅图分别插入正文相应的位置。除此之外,本书还增加了大量注释和三幅王阳明手迹。注释主要分五类:校勘记、生僻词语注释、引文查证、传奇故事增补、征引相关史料及研究成果以考证相关史实或提供其他说法。王阳明的三幅墨宝置于相关的正文之后,并加释读。

阳明学以\quota{致良知}为宗旨,良知是真知,致良知是真行,知行合一,方是真学问。本书生动展现了王阳明波澜壮阔的一生,以及阳明学千锤百炼始成金的历程:少年王阳明以立志成圣贤为人生第一等事,龙场悟道确立学旨,通过知县庐陵、平山贼匪患、靖宁藩叛乱等实践和发展阳明学。事功是良知学的试金石,用兵屡建奇功基于\quota{学问纯笃,养得此心不动};\buffernote[1]涵育内在道德之本与外在事功协同促进,成就一代大儒。

《图谱》的作者是王阳明的弟子邹守益。邹守益(1491---1562),字谦之,号东廓,江西吉安府安福县人。王阳明的主要事功在江西完成,一生精神俱在江右。邹守益是从学王阳明最早者之一,是江右王门领袖;在学问上深得阳明真传,在培养弟子方面亦卓有建树,其子邹善,其孙邹德涵、邹德溥、邹德泳均是阳明后学中重要人物。江右王门能够兴盛不衰,其中邹氏家族起了重要作用。嘉靖丁巳(1557),来自浙中王门的学使王宗沐视学吉安时,与邹守益共倡讲会,并为《图谱》作序。王宗沐认为,孔子的内在精神固然是根本,但其肖像形貌亦可以引发学子信心、坚定其信念,\quota{三千笃信,沦浃肌髓,一旦泰山颓坏,众志茕然,如孺子之丧慈母,无所依归},\quota{苟一有所存焉,亦足以收其将散之心,而植其未废之教}。这同样适用于王阳明:

\startquotation
阳明王先生天挺间出\ldots{}\ldots{}功业、理学盖宇宙百世师矣。当时及门之士,相与依据尊信,不啻三千之徒。今没才三十年,学亦稍稍失指趣。高弟安成东廓邹公辈相与绘图勒石,取先生平生经历之所及,与功用之大,谱而载焉。\ldots{}\ldots{}余少慕先生,十四岁游会稽,而先生已没。两官先生旧游之地,凡事先生者皆问而得概焉,然不若披图而溯之为尤详也。以余之尤有待于是,则后世可知,而邹公之意远矣。
\stopquotation

由序言可知《图谱》之由来,同时也显示出《图谱》中图像的真实性与重要性,是后学追思王阳明的重要依据。王宗沐的盛赞反映出邹守益苦心作《图谱》的深远意义,这当然不限于在王阳明去世三十年后,直至今日,《图谱》仍是研究王阳明图像和事迹的重要文献。本书图谱部分据1941年影印本《王阳明先生图谱》,其后跋云:\quota{其《图谱》一册,亦当时弟子所辑述者,先生事略具载于斯。明嘉靖间曾刊而行之。三百年来,遗书零落},由程守中寻得善本影印而成。

\startplacefigure[location={none},title={}]%
  \externalfigure[imgs/王阳明图传/image00114.jpeg][normal-img-60][width=.6\textwidth]
\stopplacefigure

此次整理的《靖乱录》底本为《古本小说丛刊》第四辑(中华书局,1990年)所收之影印本,原书不分章,本书分六章,补加标题:

第一章《早年传奇》:自王阳明之父王华事迹开始,至弘治十八年(1505)王阳明与湛若水定交,包括王阳明的家世、瑞云而生、梦马伏波、铁柱宫遇高道、结庐阳明洞等早年传奇故事。

第二章《龙场悟道》:自正德元年(1506)至正德九年(1514),包括王阳明上疏忤旨、谪戍贵州、龙场悟道、起复庐陵等经历。

第三章《破山中贼》:自正德十年(1515)至正德十三年(1518),叙述扫除南、赣匪患事迹。

第四章《靖宁藩乱》:正德十四年(1519),记载平定宸濠叛乱始末。

第五章《忠泰之变》:自正德十四年八月上疏谏止武宗皇帝亲征,至正德十五年(1520)九月再至南昌。讲述王阳明靖宁藩乱后遭小人诋毁及拈出\quota{良知}的过程。君子以小人为砺石,王阳明拈出\quota{致良知},阳明学宣告成功。

第六章《此心光明》:自正德十五年于南昌收王艮开始,至王阳明去世。王阳明晚年平广西思、田叛乱,打八寨、断藤峡,易如反掌;弟子盈门,所收王艮日后开创别开生面、盛极一时的泰州学派,邹守益、陈九川、何秦、黄弘纲、聂豹等成为江右王门的中流砥柱,王畿、董澐等奠基浙中王门\ldots{}\ldots{}此心光明复何言,须至下丹圣果圆。

本书的传奇故事既有政途的艰险坎坷,也有军事的波澜壮阔;既有在百死千难、遭遇迫害时的不屈不挠,也有在靖乱时的当机立断、痛快淋漓。中间如梅花间竹,穿插了许多奇遇异闻。而贯穿所有传奇经历的主线则是阳明先生修身讲学、圆证良知的全过程。只有抓住这条主线,才能了解所有这些百回千折、辉煌功业不过是修行途中的风景、功到自然成的结果。当今社会以科技为第一生产力,在王阳明的世界中,良知是第一生产力。良知如春,事功如花,春到花自开;阳明学如日月,日月行而百化兴。良知不仅是内在的道德源泉,而且能够锻炼强大的内心,激发创造力,展现生命的盎然春意;致良知不仅可以养成圣贤人格,而且能够转化应用到深度思辨的哲学、超越生死的宗教、直透性灵的文学、全心为民的政治、战无不胜的军事、探索钻研的科学等领域,润沃身心,促进个人发展,成就辉煌事业。

  \stopsectionlevel

  \startsectionlevel[preface][title={二}]%

冯梦龙与阳明后学颇有渊源,冯梦龙酷嗜泰州学派的李贽之学,奉为蓍蔡。李贽批点《忠义水浒传》,其门人携至吴中,冯梦龙与袁宏道门人袁叔度见而爱之,相对再三,精书妙补。\buffernote[2]阳明后学李贽的童心说、袁宏道的性灵说对冯梦龙的文学创作有重要的影响。

《靖乱录》是冯著《三教偶拈》的一种,据序言:\quota{宋之崇儒讲学,远过汉唐,而头巾习气刺于骨髓,国家元气日以耗削。}宋儒严于理气心性之辨,但若雕琢章句、空发议论,不向躬行践履、治国为民处着力,不免有头巾腐儒之讥、空谈误国之嫌。如同打蛇打七寸、杀人向咽喉处着刀一样,阳明学直接面向当下的问题,在为政、军事中致良知。冯梦龙作《靖乱录》的动机正是有鉴于此:\quota{偶阅《王文成公年谱》,窃叹谓文事武备,儒家第一流人物。暇日演为小传,使天下之学儒者,知学问必如文成,方为有用。}\buffernote[3]对比冯梦龙所看到的儒者:\quota{堪笑伪儒无用处,一张利口快如风。}\quota{今日讲坛如聚讼,惜无新建作明师。}\buffernote[4]这种讲学像打官司的空谈风气在明末死灰复燃,他由此呼唤像王阳明一样的真儒来主持正学,挽救大厦将倾的明朝。

王阳明十四岁时习学弓马,留心兵法,多读韬钤之书,尝曰:\quota{儒者患不知兵。仲尼有文事,必有武备。区区章句之儒,平时叨窃富贵,以词章粉饰太平,临事遇变,束手无策,此通儒之所羞也。}\buffernote[5]冯梦龙赞誉王阳明:\quota{真个是卷舒不违乎时,文武惟其所用,这才是有用的学问,这才是真儒。}\buffernote[6]像王阳明这样的文武兼备之才,又何尝不是国家危亡时所祈盼的呢?《靖乱录》当作于甲申前一年,当时境况如何呢?

\startquotation
今日流贼之乱,从古未有。然起于何地?纵自何人?炎炎燎原,必有燃始。当事者从不究极于此,其可怪一也。守土之臣,不能战则守,不能守则死。今贼来则逃,贼退复往,甚则仓皇而走,仍然梱载而归。互相弥缝,恬不知耻,其可怪二也。兵部务精,以众相夸。纪律无闻,羁縻从事。官兵所至,行居觳觫。民之畏兵,甚于畏贼,其可怪三也。饷不核旧,耑务撮新。奸胥之腹,茹而不吐。贪吏之橐,结而不开。民已透输,官乃全欠,其可怪四也。京师天府,固于磐石。游骑一临,不攻自下。百官不效一筹,羽林不发一矢,其可怪五也。衣冠济济,声气相高。脚色纷纷,跪拜恐后。举天下科甲千百之众,而殉难才二十人,其可怪六也。\buffernote[7]
\stopquotation

王阳明为政务在开导人心,使百姓安居乐业,从根本上铲除流贼根苗,其靖乱功绩显赫,多是主动出击,哪可能有\quota{贼来则逃,贼退复往}的懦弱之兵呢?阳明治军有方,纪律严明,军民一心,与明末的怪相形成了鲜明对比,如有像阳明一样的真儒当政,必无此六怪。冯梦龙为王阳明作传的动机密切关联着他所处的时代巨变,反映出当时士人评判学术、探究国家衰败原因、积极救亡图存的深刻思考和真切期盼。

\startplacefigure[location={none},title={}]%
  \externalfigure[imgs/王阳明图传/image00115.jpeg][normal-img-60][width=.6\textwidth]
\stopplacefigure

  \stopsectionlevel

  \startsectionlevel[preface][title={三}]%

《三教偶拈》于儒释道三教代表人物各选其一:道教为许逊,题名《许真君旌阳宫斩蛟记》;佛教是道济,题名《济颠罗汉净慈寺显圣记》。许逊、道济、王阳明共同的特征是:有济世的情怀,能够为百姓做事,深受人民爱戴。

三教合一是明季思想发展的主流,冯梦龙以济世作为儒释道的根基:\quota{是三教者,互相讥而莫能相废,吾谓得其意皆可以治世。}\quota{于释教,吾取其慈悲;于道教,吾取其清净;于儒教,吾取其平实。所谓得其意皆可以治世者,此也。}\buffernote[8]三教在治世上各有所长,取三者之长以济世:\quota{崇儒之代,不废二教,亦谓导愚适俗,或有藉焉,以二教为儒之辅可也。}\buffernote[9]这表明冯梦龙以儒家为统领的思想倾向。

王阳明出入释道,其中与道教的关系尤为密切。王阳明少年立志成圣,经娄谅\quota{语宋儒格物之学,谓圣人必可学而至,遂深契之}。\buffernote[10]九华山地藏洞老道点拨王阳明,直透性灵,备言佛老之要,渐及于儒,曰:\quota{周濂溪、程明道是儒者两个好秀才。}又曰:\quota{朱考亭是个讲师,只未到最上一乘。}\buffernote[11]《阳明先生年谱》中有大体相同的记述,但并未提及朱熹(考亭)。业师杜维明先生注意到,周濂溪(周敦颐)与程明道(程颢)的\quota{精神努力在许多方面预示着守仁后半生的精神取向。还应该指出,许多思想史家论证说,守仁实际上属于程明道传统,而朱熹则是程颐(伊川)的真正的继承者。在儒学传统中,把周濂溪和程明道挑出来给予特别关注,并不是信口开河。它有一个审慎的意图,即降低朱熹的重要性。的确,据这段轶事的另一种叙述,朱熹实际上被看做儒学的讲师。这一陈述隐含着对这位宋代大儒的严厉批评。讲师可能善于辞令,口若悬河,但由于他的内心经验的品质仍然离儒学的典范教师的理想有差距,所以他还不能以他的全部身心来讲学}。\buffernote[12]朱熹为讲师之说与冯梦龙对于宋儒的头巾之讥相呼应。在冯梦龙看来,程颐与朱熹可能在境界修为方面尚有欠缺,加之打着程朱理学幌子的末流好以纯儒自居,排斥释道,自然与二氏存有隔阂。地藏洞老道为阳明指出了学习的典范,周敦颐正是从内在的孔颜乐处启发程颢,阳明由此践行,同样是儒家的好秀才,这对于阳明反思朱子学,开创学问新气象有重要导向作用。

《许真君旌阳宫斩蛟记》又见于冯梦龙的《警世通言》第四十卷《旌阳宫铁树镇妖》。王阳明与许逊的传说有些联系,如《许真君旌阳宫斩蛟记》结尾:\quota{正德戊寅年间,宁府阴谋不轨,亲诣其宫,真君降箕笔云:三三两两两三三,杀尽江南一檐耽。荷叶败时黄菊绽,大明依旧镇江山。后来果败。诸灵验不可尽述。}\buffernote[13]许逊灵验之符的传说表明朱宸濠不得人心,失道寡助,注定失败。用王阳明比附许逊,表明后学对于王阳明的极度尊崇,如阳明后学董穀以阳明靖宸濠之乱比附许逊镇蛟:

\startquotation
嘉靖八年春,金华举人范信,字成之,谓余言:宁王初反时,飞报到金华,知府某不胜忧惧,延士大夫至府议之,范时亦在座。有赵推官者,常州人也,言于知府曰:\quota{公不须忧虑,阳明先生决擒之矣。}袖出旧书一小编,乃《许真君斩蛟记》也。卷末有一行云:\quota{蛟有遗腹子贻于世,落于江右,后被阳明子斩之。}既而不数日,果闻捷音。范语如此。余后于白玉蟾《修真十书》,始知真人斩蛟之事甚详。其略云:真人既制蛟于牙城南井,仍铸铁柱镇之,其柱出井数尺,下施八索,钩锁地脉。祝之曰:\quota{铁柱若亚,其妖再兴,吾当复出。铁柱若正,其妖永除。}由是水妖顿息,都邑无虞。复虑后世奸雄窃发,故因铁柱再记云:\quota{地胜人心善,应不出奸雠。纵有兴谋者,终须不到头。}又曰:\quota{吾没后一千二百四十年间,此妖复出,为民害。豫章之境,五陵之内,当有地仙八百人出而诛之。}真人生于吴赤乌二年正月二十八日,至晋宁康三年八月朔,年一百三十六岁,拔宅上升云。余考传记,旌阳存日至今正德己卯,大约适当一千二百四十年之数。且所记铁柱井,今在洪都南城铁柱观中,而真人亦有庙在省城,其有功于南昌甚大。观江西士人言:宁王初生时,见有白龙自井中出,入于江。非定数而何哉?\buffernote[14]
\stopquotation

王阳明主要的事功在江西完成,江西百姓得以安居乐业,自然对他推崇备至。江西阳明学发展兴盛,阳明的传奇故事在民间广为传播,由此再结合流传已久的许真君斩蛟传说,自然成为民间创作题材。由《靖乱录》可知,阳明酷爱剑法,心仪马援,曾在铁柱宫学静坐之法,这与许真君有些相似;豫章蛟龙为害百姓,无恶不作,将朱宸濠比作蛟龙的遗腹子亦在情理之中。\quota{康王梦蟒蛇一条,飞入宫中,将一宫之人登时啖尽,又张口来啮康王。}\buffernote[15]正如康王所梦,朱宸濠果然败家叛乱,不仅劫持鄱阳湖中的来往船只,而且殃及南昌、九江等地百姓,朱宸濠何尝不是一个恶蛟呢?

许逊铁柱中所记的\quota{纵有兴谋者,终须不到头}暗合降箕笔云\quota{荷叶败时黄菊绽,大明依旧镇江山。}《许真君旌阳宫斩蛟记》与董穀所记传说有呼应之处:蛟龙被许逊追杀时,化作美少男子逃至长沙,与刺史贾玉之女成婚,生三子。后为许逊识破,拔剑诛三子。许逊欲杀贾玉之女时,贾玉夫妇跪请哀告,未杀。由此可呼应董穀所言蛟有遗腹子之事。许逊在长沙擒蛟后,\quota{遂锁了孽龙,径回豫章。于是,驱使神兵铸铁为树,置之郡城南井中。下用铁索钩锁,镇其地脉,牢系孽龙于树。且祝之曰:\quota{铁树开花,其妖若兴,吾当复出。铁树居正,其妖永除,水妖屏迹,城邑无虞。}}\buffernote[16]这与董穀所记许逊言\quota{其妖再兴,吾当复出}如出一辙。

《济颠罗汉净慈寺显圣记》讲的是济公的故事。王阳明与道济也有一些关联,道济显圣的净慈寺,正是阳明上疏忤刘瑾、遭廷杖、出狱后的养病之地。王阳明擒获朱宸濠献俘后,亦养病于西湖净慈寺。当然,最有趣的是王阳明于西湖虎跑泉点化那位闭目塞听、枯坐三年的小僧,以禅机喝之曰:\quota{这和尚终日口巴巴说甚么?终日眼睁睁看甚么?}\buffernote[17]眉批:\quota{绝妙禅理!}这真似道济再现。

\stopsectionlevel

\startsectionlevel[preface][title={四}]%

《靖乱录》传入日本,经翻印后,流传甚广,同时也为研究者所重视。\buffernote[18]国内《靖乱录》清代以来少见流传,2007年凤凰出版社《冯梦龙全集》收入此书,但较少为人所注意。这种现象也映射出中日阳明学发展的差异,梁启超读《泰州学案》按语谓:

\startquotation
日本自幕府之末叶,王学始大盛,其著者曰大平中斋、曰吉田松阴、曰西乡南洲、曰江藤新平,皆为维新史上震天撼地人物。其心得及其行事,与泰州学派盖甚相近矣。井上哲次郎一书曰《日本阳明派之哲学》。其结论云:\quota{王学入日本则成为一日本之王学,成活泼之事迹,留赫奕之痕迹,优于支那派远甚。}嘻!此殆未见吾泰州之学风焉尔。抑泰州之学,其初起气魄虽大,然终不能敌一般舆论,以致其传不能永,则所谓活泼赫奕者,其让日本专美亦宜。接其传而起其衰,则后学之责也。\buffernote[19]
\stopquotation

泰州学派中敢于赤手搏龙蛇之士充分发扬了阳明的豪杰气概,勇于担当,洋溢着良知学刚健活泼的精神,展现了凤凰翔于千仞之上的气象。虽然在阳明后学展开过程中,泰州学派末流猖狂自恣,流弊甚多,但是阳明学内在的精神气质依然不减,李贽盛赞泰州学派:\quota{盖心斋真英雄,故其徒亦英雄也。}\quota{一代高似一代,所谓大海不宿死尸,龙门不点破额,岂不信乎!}\buffernote[20]据《心斋先生弟子师承》:\quota{计得诸贤四百八十七人,可谓盛矣。上自师保公卿,中及疆吏司道牧令,下逮士庶樵陶农吏,几无辈无之。考诸贤所出之地,几无省无之。}\buffernote[21]阳明后学还有江右王门、浙中王门、止修学派等,中国阳明后学成活泼之事迹、留赫奕之痕迹者如过江之鲫。为什么初起气魄大,其传不能永呢?梁启超的话题关联着阳明学发展的困境。阳明学至三传弟子后的衰落与首辅张居正禁学、魏忠贤迫害正义之士等外部高压的政治环境有关,随着清廷确立,阳明学也几乎成为绝响。学难以讲,道如何修?同时,内部学派分化,良知异见层出,这需要深入分析阳明学发展的内在理路以及流弊产生的根源。只有对流弊有深刻的警醒,才能有效保证真精神的顺畅发用。当然,学术本身具有\quota{自愈}功能,明末清初的大思想家方以智、王夫之、黄宗羲等对此都有深刻的反思。吾辈今日\quota{宜接其传而起其衰},既要深入发掘阳明学的内核精神,又要澄汰其流弊,既需要在义理上把握辨析,也需要以儒家典范为榜样,在其事迹中体会学习,汲取内在的精神力量,转化成自我人格培养、增进事业的源泉。

本次整理的《靖乱录》参校本为日本弘毅馆翻刻本,分为上中下三卷,每卷字数相当,但从内容的角度意义不大。其优点是字迹特别是眉批较清晰,但较之于原本增加了一些翻刻错误。

笔者在研究阳明学过程中,只关注过少量翻译的日本阳明学研究成果。业师杜维明先生曾多次说起冈田武彦先生,这次修订正值冈田先生的《王阳明大传:知行合一的心学智慧》中文版出版,得以参考,笔者近来又购得一套《冈田武彦全集》,其中还有《王阳明全集抄评释》等,冈田先生的研究方法及深耕易耨的治学精神值得学习。他在书中大量征引《靖乱录》,并予以说明:\quota{可能有读者会问,《皇明大儒王阳明先生出身靖乱录》是传记小说,其中肯定会有虚构的成分,为什么还要引用呢?这是因为通过阅读这样的小说,读者可以更容易理解阳明思想的精髓。阳明思想中最出彩的\quota{体认},其实是一种情感。}\buffernote[22]冯梦龙作《靖乱录》时,将自己的真实情感灌注其中,读者也应怀着一种全身投入的真情,方能与之共鸣,心灵得以滋养。梁启超也曾讲道:王阳明学说宗旨的形成,\quota{他自己必几经实验,痛下苦功,见得真切,终能拈出来}。阳明学是修养身心、磨炼人格的学问。如果对此把握不准,则学生以\quota{吃书}为职业,学校和教师只是在\quota{贩卖}知识。\buffernote[23]诚若如此,何谈有本有根的真学问?更不用说事功与济世了。

\stopsectionlevel

\startsectionlevel[preface][title={五}]%

正德十二年丁丑(1512),王阳明作《谕俗四条》,以劝善为首:

\startquotation
为善之人,非独其宗族亲戚爱之,朋友乡党敬之,虽鬼神亦阴相之。为恶之人,非独其宗族亲戚恶之,朋友乡党怨之,虽鬼神亦阴殛之。故\quota{积善之家,必有余庆,积不善之家,必有余殃}。\buffernote[24]
\stopquotation

王阳明注意到,\quota{破山中贼}后,要从根源上铲除匪患,必须进一步\quota{破心中贼}:注重教化百姓,移风易俗,消除作恶的根源。\quota{积善之家,必有余庆,积不善之家,必有余殃。}出自《易传·文言》,用今天通俗的话来讲,就是家庭成员共同积攒、放大正能量,则家庭兴旺,事业发达,而笼罩着负能量的家庭则相反,这是儒学在民间发挥宗教信仰功能的理论基石。劝人积善是冯梦龙小说的一贯主旨,他说:\quota{六经《语》《孟》,谈者纷如,归于令人为忠臣、为孝子},\quota{为树德之士、为积善之家,如是而已矣},\quota{而通俗演义一种,遂足以佐经书史传之穷}。\buffernote[25]这在《三教偶拈》中也有集中体现:道济之父李茂春为人纯厚;汉代兰期勉修孝行,感动仙灵:\quota{夫孝至于天,日月为之明;孝至于地,万物为之生;孝至于民,王道为之成。}\quota{上至天子,下至庶人,孝道所至,异类皆应。}\buffernote[26]许逊的祖父许琰为人仁慈,罄家资救饥荒。值黄巾大乱,许都遭大荒,许逊的父亲许肃将自己的仓谷尽数周给各乡。\quota{有监察神将许氏世代积善,奏知玉帝,若不厚报,无以劝善。}玉帝准奏,宣玉洞天仙身变金凤,口衔宝珠,下降许肃家投胎。有诗为证:\quota{御殿亲传玉帝书,祥云蔼蔼凤衔珠。试看凡子生仙种,积善之家庆有余。}\buffernote[27]由此形象诠释了《孝经·感应章》\quota{孝悌之至,感于神明,光于四海,无所不通}。

与高深严肃的典籍相比,小说自有独特的作用:小说\quota{资于通俗者多},\quota{试今说话人当场描写,可喜可愕,可悲可涕,可歌可舞。}\quota{虽日诵《孝经》《论语》,其感人未必如是之捷且深也。噫,不通俗而能之乎?}\buffernote[28]孝子贤臣的鲜活事迹是普及教化、移风易俗的最佳方式,这种方式深入人心,易晓易记,能在民间迅速传播。

《靖乱录》通过一些具体的事例展现孝忠等儒家核心价值,劝善止恶,如其中的两个反面典型:讲到阳明杀池仲容后,冯梦龙言:\quota{人恶人怕天不怕,人善人欺天不欺。善恶到头终有报,只争来早与来迟。}\buffernote[29]这不正是积善有余庆、积不善有余殃的另一种表达吗?朱宸濠作恶多端,不但自己身败名裂,而且导致整个宁王世家至此终结。

王阳明一家是《靖乱录》中为善的典型,阳明之父王华积德行善,不但自己得中状元,也因此感得神仙送子。冯梦龙讲述王华拒绝\quota{欲借人间种}的美姬,并与王华中状元一事相关联,由此暗喻不邪淫、不贪色、不为恶,能够感神明、得功名、增福报。并记述阳明出生时,岑夫人梦神人衣绯腰玉,于云中鼓吹,送一小儿来家。积善有余庆,得贵子通常是余庆的重要特征,并伴有吉祥的征兆。但这些神异的事迹并非冯梦龙杜撰,而均有相关史传来源。王华的事迹见于《王阳明全集》卷三十八《海日先生行状》,王阳明降生的传说则阳明年谱、传记均有记载。而阳明弟子钱德洪所记尤为详尽:\quota{岑夫人夜梦五色云中,见神人绯袍玉带,鼓吹导前,送儿授岑曰:\quota{与尔为子。}岑辞曰:\quota{吾已有子,吾媳妇事吾孝,愿得佳儿为孙。}神人许之。}\buffernote[30]王华孝友出于天性,以身示教;子承父德,由此烘托王阳明的忠孝。王阳明上疏救戴铣遭廷杖,王华时为礼部侍郎,在京,喜曰:\quota{吾子得为忠臣,垂名青史,吾愿足矣。}眉批:\quota{是父是子。}\buffernote[31]王阳明靖宁藩叛乱时,王华不惧危祸,\quota{咸谓新建既与濠为敌,其势必阴使奸人来不利于公。先生笑曰:\quota{吾儿能弃家杀贼,吾乃独先去以为民望乎?祖宗德泽在天下,必不使残贼覆乱宗国,行见其败也。吾为国大臣,恨已老,不能荷戈首敌。倘不幸胜负之算不可期,犹将与乡里子弟共死此城耳}}。\buffernote[32]王华的善义之举使得阳明能够不分心于家事,专注于靖乱,阳明的成功离不开家人的支持。冯梦龙在《靖乱录》中以王华开端,并在后来的行文中插入王华的事迹,父子一体,父子一心,彰显积善之家有余庆的宗旨。

冯梦龙认为\quota{儒释道三教虽殊,总抹不得孝弟二字}。\buffernote[33]在儒家看来,求忠臣于孝子之门,孝亲很容易转化成爱国,这种爱是一种充满深情的、积极的责任担当。当王阳明在平漳南匪寇后,主动乞假令旗令牌,使得便宜行事,由此进一步袭破山贼、平定宁藩叛乱。晚年平思、田叛乱后,看到八寨、断藤峡等处的山贼据险作乱,又因湖广归师之便,密授方略而袭平。其忠孝之心并不会因为刘瑾等奸佞的迫害而屈服,反而经过百死一生的磨炼而龙场悟道;也不会因为妒贤嫉功的江彬、许泰等小人的猜忌而损减,反而促成拈出\quota{良知},学问修养更加纯熟。像王阳明这样的孝子忠臣,又何尝不是冯梦龙在国家危亡时强烈呼唤的呢?阳明学是中国文化宝库中瑰丽的明珠,积淀在民族记忆深处。虽然几经摧折,但笔者坚信阳明学在中国的根仍然活着,复兴的基础依旧雄厚。如果读者在阅读中有所启发,能结合自己所处的境遇,活学活用,推动当下的事业的开展、社会的进步,方不愧真正有用的学问,才可以说是把握了阳明学的真精神。

\startalignment[flushright]
\hw\hfill 张昭炜\\
    \hfill 2017年5月\\
\stopalignment

  \stopsectionlevel

\stopsectionlevel


\startsectionlevel[default][title={早年传奇}]%

如今且说\quota{道学}二字,道乃道理,学乃学问。有道理,便有学问,不能者待学而能,不知者待问而知,问总是学,学总是道,故谓之道学。诗曰:

\startverse[leftalign=yes,sample={绵绵圣学已千年,两字良知是口传。},]%
绵绵圣学已千年,两字良知是口传。\\
欲识浑沦无斧凿,须知规矩出方圆。\\
不离日用常行内,直造先天未画前。\\
握手临歧更何语?殷勤莫愧别离筵!\buffernote[1-1]
\stopverse

这首诗,乃是国朝一位有名的道学\buffernote[1-2]先生别门生之作。那位道学先生姓王,双名守仁,字伯安,学者称为阳明先生,乃浙江省绍兴府余姚县人也。

如今且说\quota{道学}二字,道乃道理,学乃学问。
有道理,便有学问,不能者待学而能,不知者待问而知,问总是学,学总是道,故谓之道学。
且如鸿蒙之世,茹毛饮血,不识不知,此时尚无道理可言,安有学问之名?
自伏羲始画八卦,制文字,泄天地之精微,括人事之变化,于是学问渐兴。
据古书所载:黄帝学于太真,颛帝学于录图,帝喾学于赤松子,尧学于君畴,
舜学于务成昭,禹学于西王国,汤学于伊尹,文王学于时子思,武王学于尚父,成王学于周公。
这几个有名的帝王,天纵聪明,何所不知,何所不能?
只为道理无穷,不敢自足,所以必须资人讲解,此乃道学渊源之一派也。
自周室东迁,教化渐衰,处士横议,天生孔圣人出来,删述六经,表章五教\buffernote[1-3],
上接文、武、周公之脉,下开百千万世之绪,此乃帝王以后第一代讲学之祖。

汉儒因此立为经师:《易经》有田何、丁宽、孟喜、梁丘贺等,
《书经》有伏胜、孔安国、刘向、欧阳高等,
《诗经》有申培、毛公、王吉、匡衡等,
《礼经》有大戴、小戴、后苍、高堂生等,
《春秋》有公羊氏、穀梁氏、董仲舒、眭弘等。各执专经,聚徒讲解。
当时明经行修者,荐举为官,所以人务实学,风俗敦厚。

及唐以诗赋取士,理学遂废,惟有昌黎伯韩愈,独发明道术,为一代之大儒。
至宋太祖崇儒重道,后来真儒辈出,为濂洛关闽之传。
濂以周茂叔为首,洛以二程为首,关以张横渠为首,闽以朱晦庵为首,于是理学大著。
许衡、吴澄当胡元腥世,犹继其脉。

迄于皇明,薛瑄、罗伦、章懋、蔡清之徒,皆以正谊明道、清操劲节相尚,生为名臣,没载祀典,
然功名事业,总不及阳明先生之盛。
即如讲学一途,从来依经傍注,惟有先生揭\quota{良知}二字为宗,直抉千圣千贤心印,开后人多少进修之路。
只看他一生行事,横来竖去,从心所欲,勘乱解纷,无不底绩,都从良知挥霍出来。
真个是卷舒不违乎时,文武惟其所用,这才是有用的学问,这才是真儒。
所以国朝道学,公论必以阳明先生为第一。有诗为证:

\startverse[align=center]
世间讲学尽皮肤,虚誉虽隆实用无。\\
养就良知满天地,阳明才是仲尼徒。
\stopverse

且说阳明先生之父,名华,字德辉,别号龙山公,自幼警敏异常。
六岁时,与群儿戏于水滨,望见一醉汉濯足于水中而去。
公先到水次,见一布囊,提之颇重,意其中必有物,
知是前醉汉所遗,酒醒必追寻至此,犹恐为他儿所见,乃潜投于水中。
群儿至,问:\quota{汝投水是何物?}公谬对曰:\quota{石块耳。}群儿戏罢,将晩餐,拉公同归。
公假称腹痛不能行,独坐水次而守之。少顷,前醉汉酒醒,悟失囊,号泣而至。

公起迎,问曰:\quota{汝求囊中物耶?}醉汉曰:\quota{然。童子曾见之否?}
公曰:\quota{吾恐为他人所取,为汝藏于水中。汝可自取。}
醉汉取囊,解而视之,内裹白金数锭,分毫不动。
醉汉大惊曰:\quota{闻古人有还金之事,不意出自童子。}
简一小锭为谢曰:\quota{与尔买果饵吃。}
公笑曰:\quota{吾家岂乏果饵而需尔金耶?}
奔而去。归家,亦绝不言于父母。\buffernote[1-4]
\topnote{还金,廉士所能也。出于六岁儿,异矣。尤异处在保全此金以待客。}

年七岁,母岑夫人授以句读。
值邑中迎春,里中儿皆欢呼出观,公危坐,读书不辍。
岑夫人怜之,谓曰:\quota{儿可出外暂观,再读不妨。}
公拱手对曰:\quota{观春不若观书也。}
岑夫人喜曰:\quota{是儿他日成就殆不可量!}
自此,送乡塾就学,过目辄不忘。同学小儿所读书,经其耳,无不成诵。

年十一,从里师钱希宠,初习对句,辄工。
月余,学为诗,又月余,学为文,出语惊人。为文两月,同学诸生虽年长,无出其右者。
钱师惊叹曰:\quota{一岁之后,吾且无以教汝矣!}
值新县令出外拜客,仆从甚盛,在塾前喝道而过。
同学生停书,争往出观,公据案朗诵不辍,声琅琅达外。
钱师止之曰:\quota{汝不畏知县耶?}公对曰:\quota{知县亦人耳,吾何畏?况读书未有罪也。}
钱师语其父竹轩翁曰:\quota{令公子德器如此,定非常人!}\buffernote[1-5]

年十四,学成,假馆于龙泉寺。
寺有妖祟,每夜出,抛砖弄瓦,往时借寓读书者咸受惊恐,或发病,不敢复居。
公独与一苍头\buffernote[1-6]寝处其中,寂然无声。
僧异之,乘其夜读,假以猪尿泡涂灰粉,画眉眼其上,用芦管透入窗棂,嘘气涨泡,如鬼头形。
僧口作鬼声,欲以动公。公取床头小刀刺泡,泡气泄。
僧拽出,公投刀,复诵读如常,了不为异。闻者皆为缩舌。\buffernote[1-7]

\startplacefigure[location={middle,top},
                  title={十二年丙申,先生五岁,尚未能言。
                         一日,与群儿戏,见一异人过,熟目之而去。
                         先生追蹑里许,异人愕然,还见竹轩翁曰:\quota{好个小孩儿,可惜叫破了。}}]%
  \externalfigure[imgs/王阳明图传/image00116.jpeg][normal-img-60][width=.6\textwidth]
\stopplacefigure

娶夫人郑氏于成化七年,怀娠凡十四月,岑夫人梦神人衣绯腰玉,于云中鼓吹,送一小儿来家。
比惊醒,闻啼声,侍女报郑夫人已产儿,儿即阳明先生也。\buffernote[1-8]
竹轩公初取名曰\quota{云},乡人因指所生楼曰\quota{瑞云楼}。
云五岁,尚不能言。一日,有神僧过之,闻奶娘呼名,僧摩其顶曰:\quota{好个小儿,可惜道破了。}
竹轩翁疑梦不当泄,乃更名守仁。是日遂能言。\buffernote[1-9]

且祖父所读书,每每口诵,讶问曰:\quota{儿何以能诵?}
对曰:\quota{向时虽不言,然闻声已暗记矣。}
其神契如此。

有富室闻龙山公名,迎至家园馆谷。忽一夜,有美姬造其馆,华惊避。
美姬曰:\quota{勿相讶,我乃主人之妾也。因主人无子,欲借种于郎君耳。}
公曰:\quota{蒙主人厚意留此,岂可为此不肖之事?}
姬即于袖中出一扇,曰:\quota{此主人之命也。郎君但看扇头字,当知之。}
公视扇面,果主人亲笔,书五字,曰:\quota{欲借人间种。}
公援笔添五字于后,曰:\quota{恐惊天上神。}
厉色拒之。姬怅怅而去。

公既中乡榜,明年会试,前富室主人延一高真\buffernote[1-10]
设醮祈嗣,高真伏坛,遂睡去,久而不起。既醒,主人问其故。

\startplacefigure[location={middle,top},
                  title={十五年己亥,先生八岁,大父竹轩翁授以《曲礼》,过目成诵。
                         一日,忽诵竹轩翁所尝读书,翁惊问之。
                         曰:\quota{闻公公读时,吾言虽不能出口,已默记矣。}}]%
  \externalfigure[imgs/王阳明图传/image00117.jpeg][normal-img-60][width=.6\textwidth]
\stopplacefigure

高真曰:\quota{适梦捧章至三天门,遇天上迎状元榜,久乃得达,故迟迟耳。}主人问:\quota{状元为谁?}
高真曰:\quota{不知姓名。但马前有旗二面,旗上书一联云:\quota{欲借人间种,恐惊天上神。}}主人默然大骇。
时成化十七年辛丑之春也。未几,会试报至,公果状元及第。阳明先生时年十岁矣。

次年壬寅,公在京师,迎养其父竹轩翁,翁因携先生同往。过金山寺,竹轩公与客酣饮,拟作诗,未成。先生在旁索笔。
竹轩翁曰:\quota{孺子亦能赋耶?}先生即书四句云:

\startverse[align=center]
金山一点大如拳,打破维扬水底天。\\
醉倚妙高楼上月,玉箫吹彻洞龙眠。
\stopverse

坐客惊异,咸为起敬。少顷,游蔽月山房\buffernote[1-11],竹轩公曰:\quota{孺子还能作一诗否?}

先生应声吟曰:

\startverse[align=center]
山近月远觉月小,便道此山大于月。\\
若人有眼大如天,还见山小月更阔。
\stopverse

\startplacefigure[location={middle,top},
                  title={十八年壬寅,竹轩公以龙山公辛丑及第,携先生之京。
                         过金山,与客酣饮,拟赋金山诗。先生即应声曰:
                        \quota{金山一点大如拳,打破维扬水底天。醉倚妙高台上月,玉箫吹彻洞龙眠。}
                         客欲试之,命赋蔽月山房,即随应曰:
                        \quota{山近月远觉月小,便道此山大于月。若人有眼大如天,还见山小月更阔。}}]%
  \externalfigure[imgs/王阳明图传/image00118.jpeg][normal-img-60][width=.6\textwidth]
\stopplacefigure

坐客谓竹轩翁曰:\quota{令孙声口,俱不落凡。想他日定当以文章名天下。}
先生曰:\quota{文章小事,何足成名?}众益异之。

十二岁,在京师就塾师,不肯专心诵读。
每潜出,与群儿戏,制大小旗帜,付群儿持立四面,自己为大将,居中调度,左旋右转,略如战阵之势。
龙山公出,见之,怒曰:\quota{吾家世以读书显,安用是为?}
先生曰:\quota{读书有何用处?}
龙山公曰:\quota{读书则为大官,如汝父中状元,皆读书力也。}
先生曰:\quota{父中状元,子孙世代还是状元否?}
龙山公曰:\quota{止我一世耳。汝若要中状元,还是去勤读。}
先生笑曰:\quota{只一代,虽状元,不为希罕。}
父益怒朴责之。
先生又尝问塾师曰:\quota{天下何事为第一等人?}
塾师曰:\quota{嵬科高第,显亲扬名,如尊公,乃第一等人也。}
先生吟曰:\quota{嵬科高第时时有,岂是人间第一流?}
塾师曰:\quota{据孺子之见,以何事为第一?}
先生曰:\quota{惟为圣贤方是第一!}
龙山公闻之,笑曰:\quota{孺子之志,何其奢也!}\topnote{世爵、大儒福德,俱自幼而定。}

先生一日出游市上,见卖雀儿者,欲得之,卖雀者不肯与,先生与之争。
有相士号麻衣神相,一见先生,惊曰:\quota{此子他日大贵,当建非常功名!}
乃自出钱,买雀以赠先生,因以手抚其面曰:孺子记吾言:

\startverse[leftalign=yes,sample={须至上丹台,其时结圣胎;}]%
须拂领,其时入圣境;\\
须至上丹台,其时结圣胎;\\
须至下丹田,其时圣果圆。
\stopverse

又嘱曰:\quota{孺子当读书自爱,吾所言将来必有应验。}言讫遂去。
先生感其言,自此潜心诵读,学问日进。

十三岁,母夫人郑氏卒。先生居丧,哭泣甚哀。
父有所宠小夫人,待先生不以礼。
先生游于街市,见有缚鸮鸟一只求售者,先生出钱买之,复怀银五钱,赠一巫妪,
授以口语:\quota{见庶母,如此恁般。}
先生归,将鸮鸟潜匿于庶母床被中。
母发被,鸮冲出,绕屋而飞,口作怪声。
小夫人大惧,开窗逐之,良久方去。
俗忌野鸟入室,况鸮乃恶声之鸟,见者以为不祥。
又伏于被中,曲房深户,重帷锦衾,何自而入,岂不是大怪极异之事?
先生闻房中惊诧之声,佯为不知,入问其故,小夫人述言有此怪异。
先生曰:\quota{何不召巫者询之?}
小夫人使人召巫妪,巫妪入门,便言家有怪气。
既见小夫人,又言:\quota{夫人气色不佳,当有大灾晦至矣。}
小夫人告以发被得鸮鸟之异,巫妪曰:\quota{老妇当问诸家神。}
即具香烛,命小夫人下拜。
索钱楮\buffernote[1-12],焚讫,妪即谬托郑夫人附体,
言曰:\quota{汝待我儿无礼,吾诉于天曹,将取汝命!适怪鸟,即我所化也。}
小夫人信以为真,跪拜无数,伏罪悔过,言:\quota{此后再不敢!}
良久,媪苏曰:\quota{适见先夫人,意色甚怒,将托怪鸟啄尔生魂,幸夫人许以改过,方才升屋檐而去。}
小夫人自此待先生加意有礼。
先生尚童年,其权术已不测如此矣。

先生十四岁,习学弓马,留心兵法,多读韬钤\buffernote[1-13]之书,尝曰:

\startquotation
儒者患不知兵。仲尼有文事,必有武备。
区区章句之儒,平时叨窃富贵,以词章粉饰太平,临事遇变,束手无策,此通儒之所羞也。
\stopquotation

十五岁,从父执\topnote{父辈谓之父执}。游居庸三关,慨然有经略四方之志。
一日,梦谒伏波将军\topnote{汉马援封伏波将军}庙,。赋诗曰:

\startverse[leftalign=yes,sample={卷甲归来马伏波,早年兵法鬓毛皤。}]%
卷甲归来马伏波,早年兵法鬓毛皤。\\
云埋铜柱雷轰折,六字题文尚不磨。
\buffernote[1-14]
\stopverse

其时地方水旱,盗贼乘机作乱,
畿内有石英、王勇,陕西有石和尚、刘千斤,屡屡攻破城池,劫掠府库,官军不能收捕。
先生言于龙山公:\quota{欲以诸生上书,请效终军故事。\buffernote[1-15]
愿得壮卒万人,削平草寇,以靖海内。}
龙山公曰:\quota{汝病狂耶!书生妄言取死耳!}
先生乃不敢言,于是益专心于学问。\buffernote[1-16]

\startplacefigure[location={middle,top},
                  title={尝梦王威宁伯遗以弓矢宝剑,后差筑威宁坟,其家果赠,如所梦。
                         梦南征,谒马伏波庙,题诗曰:
                        \quota{卷甲归来马伏波,早年兵法鬓毛皤。云埋铜柱雷轰折,六字题文尚不磨。}
                        及征田州,果验。}]%
  \externalfigure[imgs/王阳明图传/image00119.jpeg][normal-img-60][width=.6\textwidth]
\stopplacefigure

弘治元年,先生十七岁,归余姚,遂往江西就亲,所娶诸氏夫人,乃江西布政司参议诸养和公之女也。
既成婚官署中,一日信步出行,至许旌阳铁柱宫,于殿侧遇一道者,庞眉皓首,盘膝静坐。
先生叩曰:\quota{道者何处人?}
道者对曰:\quota{蜀人也,因访道侣至此。}
先生问:\quota{其寿几何?}
对曰:\quota{九十六岁矣。}
问其姓,对曰:\quota{自幼出外,不知姓名。人见我时时静坐,呼我曰\quota{无为道者}。}
先生见其精神健旺,声如洪钟,疑是得道之人,因叩以养生之术。
道者曰:\quota{养生之诀,无过一静。老子清净,庄生逍遥,惟清净而后能逍遥也。}
因教先生以导引之法。先生恍然有悟,乃与道者闭目对坐,如一对槁木,不知日之已暮,并寝食俱忘之矣。
诸夫人不见先生归署,言于参议公,使衙役遍索,不得。
至次日天明,始遇之于铁柱宫中,隔夜坐处尚未移动也。衙役以参议命促归,先生呼道者与别。
道者曰:\quota{珍重珍重。二十年后,当再见于海上也。}\buffernote[1-17]
先生回署。署中蓄纸最富,先生日取学书,纸为之空,书法大进。
先生自言:\quota{吾始学书,对模古帖,止得字形。其后不轻落纸,凝思于心,久之,始通其法。
明道程先生有曰:\quota{吾作字甚敬,非是要字好,只此是学。}
夫既不要字好,所学何事?只\quota{不要字好}一念,亦是不敬。}\buffernote[1-18]
闻者叹服。\topnote{便能参驳先儒,识见超异。}

明年己酉,先生十八岁。是冬,与诸夫人同返余姚。
行至广信府上饶县,谒道学娄一斋,名谅。
语以宋儒格物致知之义,谓圣人必可学而至。\buffernote[1-19]
先生深以为然。自是,奋然有求为圣贤之志。
平日好谐谑豪放,此后每每端坐省言,
曰:\quota{吾知过矣。蘧伯玉行年五十,而知四十九之非,何其晩也!}\buffernote[1-20]

弘治五年壬子,先生年二十一岁,竹轩翁卒于京师,龙山公奉其丧以归。\buffernote[1-21]
是秋,先生初赴乡试场中。
夜半,巡场者见二巨人:一衣绯,一衣绿,东西相向立,
大声言曰:\quota{三人好做事。}言讫,忽不见。
及发榜,先生与孙忠烈燧、胡尚书世宁同举。
其后,宁王宸濠之变,胡发其奸,孙死其难,先生平其乱,人以为\quota{三人好做事},此其验也。

明年癸丑春,会试下第。
宰相李西涯讳东阳,时方为文章主盟,服先生之才,戏呼为来科状元。
丙辰,再会试,复被黜落。\buffernote[1-22]
同寓友人以不第为耻,先生曰:\quota{世情以不得第为耻,吾以不得第动心为耻。}
友人服其涵养。时龙山公已在京任,先生遂寓京中。

明年丁巳,先生年二十六岁。边任报紧急,举朝仓皇,推择将才,莫有应者。
先生叹曰:\quota{武举之设,仅得骑射击刺之士,而不可以收韬略统驭之才。平时不讲将略,欲备仓卒之用,难矣!}
于是留情武事。凡兵家秘书,莫不精研熟讨。每遇宾客宴会,辄聚果核为阵图,指示开阖进退之方。
一夕,梦威宁伯王越,解所佩宝剑为赠。既觉,喜曰:\quota{吾当效威宁,以斧钺之任,垂功名于竹帛,吾志遂矣!}

弘治十二年己未,先生中会试第二名,时年二十八岁,廷试二甲,
以工部观政进士受命,往浚县督造威宁伯坟。\buffernote[1-23]
先生一路不用肩舆,日惟乘马,偶因过山,马惊,先生坠地吐血,从人进轿,先生仍用马。盖以此自习也。
既见威宁子弟,问先大夫用兵之法,其家言之甚悉。
先生即以兵法部署造坟之众,凡在役者更番休息,用力少,见功多,工得速完。
其家致金帛为谢,先生固辞不受,后乃出一宝剑相赠,曰:\quota{此先大夫所佩也。}
先生喜其与梦相符,遂受之。\topnote{如此,方用世实。}

\startplacefigure[location={middle,top},
                  title={上边务八事:
                         曰蓄材以备急,曰舍短以用长,曰简师以省费,曰屯田以足食,
                         曰行法以振威,曰敷恩以激怒,曰捐小以全大,曰严守以乘弊。
                         其言今之大患,
                         在于为大臣者外托慎重老成,而内为固禄希宠之计;
                         为左右者内挟交蟠壅蔽,而外肆招权纳贿之恶。
                         忧时者谓之迂狂,进言者目以浮躁。尤为剀切。}]%
  \externalfigure[imgs/王阳明图传/image00120.jpeg][normal-img-60][width=.6\textwidth]
\stopplacefigure

复命之日,值星变,达虏方犯边,朝廷下诏求直言。
先生上言边务八策,言极剀切。\buffernote[1-24]
明年,授官刑部主事。又明年,奉命审录江北,多所平反,民称不冤。
事毕,遂游九华山,历无相、化城诸寺,到必经宿。
时道者蔡蓬头踞坐堂中,衣服敞陋,若颠若狂。
先生心知其异人也,以客礼致敬,请问:\quota{神仙可学否?}蔡摇首曰:\quota{尚未尚未。}
有顷,先生屏去左右,引至后亭,再拜,复叩问之。蔡又摇首曰:\quota{尚未尚未。}
先生力恳不已,蔡曰:\quota{汝自谓拜揖尽礼,我看你一团官相,说甚神仙!}
先生大笑而别。
游至地藏洞,闻山岩之巅有一老道,不知姓名,坐卧松毛,不餐火食。
先生欲访之,乃悬崖扳木而上,直至山巅。
老道踡足熟睡,先生坐于其傍,以手抚摩其足。
久之,老道睡方觉,见先生,惊曰:\quota{如此危险,安得至此?}
先生曰:\quota{欲与长者论道,不敢辞劳也。}
因备言佛老之要,渐及于儒,曰:\quota{周濂溪、程明道是儒者两个好秀才。}
又曰:\quota{朱考亭是个讲师,只未到最上一乘。}\buffernote[1-25]
先生喜其谈论,盘桓不能舍。次日再往访之,其人已徙居他处矣。有诗为证:

\startverse[leftalign=yes,sample={路入岩头别有天,松毛一片自安眠。}]%
路入岩头别有天,松毛一片自安眠。\\
高谈已散人何处,古洞荒凉散冷烟。
\buffernote[1-26]
\stopverse

弘治十五年,先生至京复命。京中诸名士俱以古文相尚,立为诗文之社,来约先生。
先生叹曰:\quota{吾焉能以有限精神,作此无益之事乎?}
遂告病归余姚,筑室于四明山之阳明洞。
洞在四明山之阳,故曰阳明。山高一万八千丈,周二百一十里,《道经》第九洞天也。
为峰二百八十有二,其中峰曰芙蓉峰,有汉隶刻石于上,曰\quota{四明山心}。
其右有石窗,四面玲珑如户牖,通日月星辰之光。
先生爱其景致,隐居于此,因自号曰\quota{阳明}。\buffernote[1-0]
思铁柱宫道者之言,乃行神仙导引之术。
月余,觉阳神自能出入,未来之事,便能前知。\buffernote[1-27]
一日静坐,谓童子曰:\quota{有四位相公来此相访,汝可往五云门迎之。}

\startplacefigure[location={middle,top},
                  title={时太原乔宇,广信江俊,河南李梦阳、何景明,
                           姑苏顾璘、徐禛卿,山东边贡,泰州储瓘,
                           俱以才名相知,为古诗文。
                           先生一日叹曰:\quota{吾安能以有限精神为无用虚文?}
                           遂告病归,辟阳明洞旧基为书屋,究仙经秘旨。
                           久之,忽能预知。王思裕等四人自云门来访,
                           先生命仆买果肴以候,历语其过涧摘桃花踪迹,
                           四人以为得道。久之,悟曰:\quota{此弄精魂,非道也。}}]%
  \externalfigure[imgs/王阳明图传/image00121.jpeg][normal-img-60][width=.6\textwidth]
\stopplacefigure

童子方出五云门,果遇王思舆等四人,乃先生之友也。\buffernote[1-28]
童子述先生遣迎之意。四人见先生,问曰:\quota{子何以预知吾等之至?}先生笑曰:\quota{只是心清。}
四人大惊异。述于朋辈,朋辈惑之。往往有人来叩先生以吉凶之事,先生言多奇中。
忽然悟曰:\quota{此簸弄精神,非正觉也。}
遂绝口不言,思脱离尘网,超然为出世之事。
惟祖母岑太夫人与父龙山公在念,不能忘情,展转踌躇,
忽又悟曰:\quota{此孝弟一念,生于孩提,此念若可去,断灭种性矣。此吾儒所以辟二氏。}
乃复思三教之中,惟儒为至正,复翻然有用世之志。
明年,迁寓于钱塘之西湖。怎见得西湖景致好处?有《四时望江南词》为证:

\startverse[leftalign=yes,sample={花底管弦公子宴,水边罗绮丽人行,十里按歌声。}]%
西湖景,春日最宜晴。\\
花底管弦公子宴,水边罗绮丽人行,十里按歌声。

西湖景,夏日正堪游。\\
金勒马嘶垂柳岸,红妆人泛采莲舟,惊起水中鸥。

西湖景,秋日更宜观。\\
桂子冈峦金谷富,芙蓉洲渚彩云间,爽气满前山。

西湖景,冬日转清奇。\\
赏雪楼台评酒价,观梅园圃订春期,共醉太平时。
\buffernote[1-29]
\stopverse

又有林和靖先生《咏西湖》诗一首:

\startverse[leftalign=yes,sample={混元神巧本无形,幻出西湖作画屏。}]%
混元神巧本无形,幻出西湖作画屏。\\
春水净于僧眼碧,晩山浓似佛头青。\\
栾栌粉堵摇鱼影,兰社烟丛阁鹭翎。\\
往往鸣榔与横笛,斜风细雨不须听。
\buffernote[1-30]
\stopverse

那西湖,又有十景,那十景?

\startquotation
苏堤春晓、平湖秋月、曲院风荷、断桥残雪、雷峰夕照、

南屏晩钟、雨峰出云、三潭印月、柳浪闻莺、花港观鱼。
\stopquotation

\startplacefigure[location={middle,top},
                  title={十七年甲子春,居阳明洞。
                         夏,山东聘主考试,梓文咸出先生手笔。
                         展胸中素蕴,一洗陈言虚套之习,五策举可措诸用,海内传以为式。
                         登泰山,作《泰山高》,有:\quota{瞻眺门墙,仿佛室堂。三千之下,许占末行。}
                         \buffernote[1-31]}]%
  \externalfigure[imgs/王阳明图传/image00122.jpeg][normal-img-60][width=.6\textwidth]
\stopplacefigure


先生寓居西湖,非关贪玩景致。
那杭州乃吴越王钱氏及故宋建都之地,名山胜水,古刹幽居,多有异人栖止。先生遍处游览,冀有所遇。
一日,往虎跑泉游玩,闻有禅僧坐关三年,终日闭目静坐,不发一语,不视一物。
先生往访,以禅机喝之曰:\quota{这和尚终日口巴巴说甚么?终日眼睁睁看甚么?}
其僧惊起作礼,谓先生曰:\quota{小僧不言不视,已三年于兹。檀越却道口巴巴说甚么,眼睁睁看甚么,此何说也?}
先生曰:\quota{汝何处人?离家几年了?}
僧答曰:\quota{某河南人,离家十余年矣。}
先生曰:\quota{汝家中亲族还有何人?}
僧答曰:\quota{止有一老母,未知存亡。}
先生曰:\quota{还起念否?}
僧答曰:\quota{不能不起念也。}
先生曰:\quota{汝既不能不起念,虽终日不言,心中已自说着;终日不视,心中已自看着了。}\topnote{绝妙禅理。}
僧猛省,合掌曰:\quota{檀越妙论,更望开示!}
先生曰:\quota{父母天性,岂能断灭?你不能不起念,便是真性发现。
        虽终日呆坐,徒乱心曲。俗语云:\quota{爹娘便是灵山佛,不敬爹娘敬甚人?}}
        \buffernote[1-32]
言之未毕,僧不觉大哭起来,曰:\quota{檀越说得极是,小僧明早便归家,省吾老母。}
次日,先生再往访之,寺僧曰:\quota{已五鼓负担还乡矣。}
先生曰:\quota{人性本善,于此僧可验也。}
于是益潜心圣贤之学。读朱考亭语录,反复玩味,又读其《上宋光宗疏》,
有曰:\quota{居敬持志,为读书之本;循序致精,为读书之法。}\buffernote[1-33]
掩卷叹曰:\quota{循序致精,渐渍洽浃,使物理与吾心混合无间,方是圣贤得手处。}
于是从事于格物致知。每举一事,旁喻曲晓,必穷究其归,至于尽处。

弘治十七年甲子,山东巡按御史陆偁重先生之名,遣使致聘,迎主本省乡试。
先生应聘而往,得穆孔晖为解元,后为名臣。\buffernote[1-34]是省全录,皆出先生之手。
其年九月,改兵部武选司主事。
先生往京都赴任,谓学者溺于词章记诵之末,不知身心之学为何等,于是首倡讲学之事。
闻者兴起,于是从学者众,先生俨然以师道自任。
同辈多有议其好名者,惟翰林学士湛甘泉讳若水。
深契之,一见定交,终日相与谈论,号为莫逆。\buffernote[1-36]

\startplacefigure[location={middle,top},title={十八年乙丑,甘泉湛公若水为庶吉士,先生一见定交,以倡圣学为志。尝赠甘泉有:\quota{幼不学问,陷溺于邪僻者二十年,而始究心于老、释。赖天之灵,始沿周、程求之,若有得焉。自得友于湛子而后志益坚,毅然若不可遏。}\buffernote[1-35]徐爱,字曰仁,居余姚马堰,娶先生女弟,受学甚蚤。沉潜而笃信,记《传习录》示同志。年三十二以没。尝梦瞿昙拊其背曰:\quota{子与颜子同德,亦与颜子同寿。}}]%
  \externalfigure[imgs/王阳明图传/image00123.jpeg][normal-img-60][width=.6\textwidth]
\stopplacefigure

\stopsectionlevel


\startsectionlevel[default][title={龙场悟道}]%

弘治十八年,孝宗皇帝宴驾。武宗皇帝初即位,宠任阉人刘瑾等八人,号为八党。那八人?

\startquotation
刘瑾、谷大用、马永成、张永、魏彬、罗祥、丘聚、高凤。
\stopquotation

这八人自幼随侍武宗皇帝,在于东宫游戏,因而用事,刘瑾尤得主心。
阁老刘健与台谏合谋去之,机不早断,以致漏泄。
刘瑾与其党泣诉于上前,武宗皇帝听其言,反使刘瑾掌司礼监,斥逐刘健,杀忠直内臣王岳。
由是权独归瑾,票拟任意,公卿侧目。

正德元年,南京科道官戴铣、薄彦徽等上疏,言:\quota{皇上新政,宜亲君子,远小人,不宜轻斥大臣,任用阉寺。}
刘瑾票旨:\quota{铣等出言狂妄,纽解来京勘问。}
先生目击时事,满怀忠愤,抗疏救之。略曰:

\startquotation
臣闻君仁则臣直,今铣等以言为责,其言如善,自宜嘉纳;即其未善,亦宜包容,以开忠谠之路。
今赫然下令,远事拘囚,在陛下不过少事惩创,非有意怒绝之也。下民无知,妄生疑惧,臣窃惜之!
自是而后,虽有上关宗社安危之事,亦将缄口不言矣。
伏乞追回前旨,俾铣等仍旧供职,明圣德无我之公,作臣子敢言之气。\buffernote[2-1]
\stopquotation

疏既入,触瑾怒,票旨下先生于诏狱,廷杖四十。
瑾又使心腹人监杖,行杖者加力,先生几死而苏,谪贵州龙场驿驿丞。
龙山公时为礼部侍郎,在京,喜曰:\quota{吾子得为忠臣,垂名青史,吾愿足矣。}\topnote{是父是子。}

\startplacefigure[location={middle,top},title={正德元年丙寅,时宦官刘瑾窃权纳贿,南京给事中戴铣等劾之,瑾差官校械锦衣狱。先生抗疏《乞宥言官去权奸以彰圣德》,具言:耳目不可使壅塞,手足不可使痿痹,在廷群臣岂无忧国爱君之心,而莫敢言者,惧以罪。铣等罪之,自今而后,虽有上关宗社危疑不制之事,陛下孰从而闻之?瑾怒,下于狱,矫诏廷杖五十,毙而复苏,谪贵州龙场驿驿丞。}]%
  \externalfigure[imgs/王阳明图传/image00124.jpeg][normal-img-60][width=.6\textwidth]
\stopplacefigure

明年,先生将赴龙场。瑾遣心腹人,一路尾其后,伺察其言动。
先生既至杭州,值夏月天暑,先生又积劳致病,乃暂息于胜果寺。\buffernote[2-2]
妹婿徐曰仁来访,首拜门生听讲,又同乡徐爱、蔡宗、朱节、冀元亨、蒋信、刘观时等,
皆来执贽问道,先生乐之。\buffernote[2-1-3]

居两月余,忽一日午后,方纳凉于廊下,苍头皆出外,
有大汉二人矮帽窄衫,如官较\buffernote[2-3]状,腰悬刀刃,口吐北音,从外突入,谓先生曰:\quota{官人是王主事否?}
先生应曰:\quota{然。}二较曰:\quota{某有言相告。}即引出门外,挟之同行。
先生问:\quota{何往?}二较曰:\quota{但前行便知。}
先生方在病中,辞以不能步履。二较曰:\quota{前去亦不远,我等左右相扶可矣。}
先生不得已,任其所之。约行三里许,背后复有二人追逐而至。先生顾其面貌,颇似相熟。
二人曰:\quota{官人识我否?我乃胜果寺邻人沈玉、殷计也。
      素闻官人乃当世贤者,平时不敢请见,适闻有官较挟去,恐不利于官人,特此追至,看官人下落耳。}
二较色变,谓沈、殷二人曰:\quota{此朝廷罪人,汝等何得亲近?}
沈、殷二人曰:\quota{朝廷已谪其官矣,又何以加罪乎?}二较扶先生又行,沈、殷亦从之。
天色渐黑,至江头一空室中,二较密谓沈、殷二人曰:\quota{吾等实奉主人刘公公之命,来杀王公。
汝等没相干人,可速去,不必相随也。}
沈玉曰:\quota{王公今之大贤,令其死于刃下,不亦惨乎!且遗尸江口,必累地方,此事决不可行!}
二较曰:\quota{汝言亦是。}乃于腰间解青索一条,长丈余,授先生曰:\quota{听尔自缢,何如?}
沈玉又曰:\quota{绳上死与刀下死,同一惨也。}
二较大怒,各拔刀在手,厉声曰:\quota{此事不完,我无以复命,亦必死于主人之手!}
殷计曰:\quota{足下不必发怒,令王公夜半自投江中而死,既令全尸,又不累地方,足下亦可以了事归报,岂不妙哉?}
二较相对低语。少顷,乃收刀入鞘,曰:\quota{如此庶几可耳。}
沈玉曰:\quota{王公命尽此夜,吾等且沽酒共饮,使其醉而忘。}
二较亦许之,乃锁先生于室中。
先生呼沈、殷二人,曰:\quota{我今夕固必死,当烦一报家人,收吾尸也。}
二人曰:\quota{欲报尊府,必得官人手笔,方可准信。}
先生曰:\quota{吾袖中偶有素纸,奈无笔何?}
二人曰:\quota{吾当于酒家借之。}
沈玉与一较同往市中沽酒,殷计与一较守先生于门外。
少顷,沽酒者已至,一较启门,身边各带有椰瓢,沈玉满斟送先生,不觉泪下。
先生曰:\quota{我得罪朝廷,死自吾分,吾不自悲,汝何必为我悲乎?}
引瓢一饮而尽。殷计亦献一瓢,先生复饮之。
先生量不甚弘,辞曰:\quota{吾不能饮矣。既有高情,幸转进于远客,吾尚欲作家信也。}
沈玉以笔授先生,先生出纸于袖中,援笔写诗一首,诗曰:

\startverse[leftalign=yes,sample={学道无成岁月虚,天乎至此欲何如。}]%
学道无成岁月虚,天乎至此欲何如。\\
生曾许国惭无补,死不忘亲恨有余。\\
自信孤忠悬日月,岂论遗骨葬江鱼?\\
百年臣子悲何极,日夜潮声泣子胥。\buffernote[2-4]
\stopverse

先生吟兴未已,再作一:

\startverse[leftalign=yes,sample={敢将世道一身担,显被生刑万死甘。}]%
敢将世道一身担,显被生刑万死甘。\\
满腹文章宁有用,百年臣子独无惭。\\
涓流裨海今真见,片雪填沟旧齿谈。\\
昔代衣冠谁上品?状元门第好奇男。\\
\stopverse

二诗之后,尚有绝命辞。甚长,不录。纸后作篆书十字云:

\startquotation
阳明已入水,沈玉、殷计报。
\stopquotation

二较本不通文理,但见先生手不停挥,相顾惊叹,以为天才。
先生且写且吟,四人互相酬劝,各各酩酊。

将及夜半,云月朦胧,二较带着酒兴,逼先生投水。
先生先向二较谢其全尸之德,然后径造江岸,
回顾沈、殷二人曰:\quota{必报我家,必报我家。}\topnote{以且写且吟劝酒,饮者不得不醉矣。}
言讫,从沙泥中步下江来。二较一来多了几分酒,二来江滩潮湿,不便相从。
乃立岸上,远而望之,似闻有物堕水之声,谓先生已投江矣。一响之后,寂然无声。
立了多时,放心不下,遂步步挣下滩来,见滩上脱有云履一双,又有纱巾浮于水面,曰:\quota{王主事果死矣。}
欲取二物以去。沈玉曰:\quota{留一物在,使来早行人人见之,知王公堕水。
传说至京都,亦可作汝等证见也。}\topnote{沈玉尽通。何处无义士。}
二较曰:\quota{言之有理。}遂弃履,只捞纱巾带去,各自分别。
至是夜,苍头回胜果寺,不见先生,问之主僧,亦云不知。
乃连夜提了行灯,各处去找寻了一回,不见一些影响。

其年丁卯,乃是乡试之年,先生之弟守文在省应试,仆人往报守文。守文言于官,命公差押本寺僧四出寻访,恰遇沈、殷二人亦来寻守文报信。守文接了绝命词及二诗,认得果其兄亲笔,痛哭了一场。未几,又有人拾得江边二履报官。官以履付守文,众人轰传,以为先生真溺死矣。守文送信家中,合家惊惨,自不必说。
龙山公遣人到江边遗履之处,命渔舟捞尸,数日无所得。
门人闻者无不悼惜,惟徐爱言先生必不死,
曰:\quota{天生阳明,倡千古之绝学,岂如是而已耶?}\topnote{孔子于颜回,师信其弟;徐爱于阳明,弟信其师。}

\startplacefigure[location={middle,top},title={正德二年丁卯春,先生以被罪未敢归家,留寓钱塘胜果寺养病。瑾怒未得逞,遣四人谋致之死。一旦,挟先生至山顶,吐实曰:\quota{我辈观公动止,何忍加害?公必有良策,使我得反报。}先生曰:\quota{吾欲遁世久矣。明日吊我于江之滨。}夜留题于壁,从间道登海舟。从者求弗得,与乡人沿哭于江。海舟,绍兴采柴者,往返如期。是夜,飘入闽中,备海兵捕之,微服奔岸,乞食于僧寺。}]%
  \externalfigure[imgs/王阳明图传/image00125.jpeg][normal-img-60][width=.6\textwidth]
\stopplacefigure

却说先生果然不曾投水,他算定江滩是个绝地,没处走脱,二较必然放心。
他有酒之人,怎走得这软滩?以此独步下来,脱下双履,留做证见,又将纱巾抛弃水面,却取石块向江心抛去。
黄昏之后,远观不甚分明,但闻扑通声响,不知真假,便认做了事。
不但二较不知,连沈玉、殷计亦不知其未死也。
先生却沿江滩而去,度其已远,藏身于岸坎之下。次日,趁个小船,船子怜其无履,以草履赠之。
七日之后,已达江西广信府。行至铅山县,其夜复搭一船。
一日夜,到一个去处,登岸问之,乃是福建北界矣。舟行之速,疑亦非人力所及。
巡海兵船见先生状貌不似商贾,疑而拘之。\topnote{对小人讲,不托之神,无以起其敬畏。}
先生曰:\quota{我乃兵部主事王守仁也。因得罪朝廷受廷杖,贬为贵州龙场驿驿丞。
自念罪重,欲自引决,投身于钱塘江中,遇一异物,鱼头人身,自称巡江使者,言奉龙王之命,前来相迎。
我随至龙宫,龙王降阶迎接,言我异日前程尚远,命不当死,以酒食相待,即遣前使者送我出江,
仓卒之中,附一舟至此,送我登岸,舟亦不见矣。不知此处离钱塘有多少程途?我自江中至此,才一日夜耳。}
兵士异其言,亦以酒食款之,即驰一人往报有司。

先生恐事涉官府,不能脱身,捉空潜遁,从山径无人之处狂奔三十余里,至一古寺。
天已昏黑,乃叩寺投宿。寺僧设有禁约,不留夜客歇宿。寺傍有野庙久废,虎穴其中。
行客不知,误宿此庙,遭虎所啖。
次早,寺僧取其行囊自利,以为常事。
先生既不得入寺,乃就宿野庙之中,饥疲已甚,于神案下熟寝。
夜半,群虎绕庙环行,大吼,无敢入者,天明寂然。
寺僧闻虎声,以为夜来借宿之客已厌虎腹,相与入庙,欲简其囊。
先生梦尚未醒,僧疑为死人,以杖微击其足,先生蹶然而起。
僧大惊曰:\quota{公非常人也!不然,岂有入虎穴而不伤者乎?}
先生茫然不知,问:\quota{虎穴安在?}僧答曰:\quota{即此神座下是矣。}
僧心中惊异,反邀先生过寺朝餐。
餐毕,先生偶至殿后,先有一老道者打坐,见先生来,即起相讶曰:\quota{贵人还识无为道者否?}
先生视之,乃铁柱宫所见之道者,容貌俨然如昨,不差毫发。
道者曰:\quota{前约二十年后相见于海上,不欺公也。}
先生甚喜,如他乡遇故知矣。
因与对坐,问曰:\quota{我今与逆瑾为难,幸脱余生。将隐姓潜名,为避世之计。不知何处可以相容?望乞指教。}
道者曰:\quota{汝不有亲在乎?万一有人言汝不死,逆瑾怒,逮尔父,诬以北走胡,南走越,何以自明?
汝进退两无据矣。}
因出一书示先生,乃预写就者。诗曰:

\startverse[leftalign=yes,sample={二十年前已识君,今来消息我先闻。}]%
二十年前已识君,今来消息我先闻。\\
君将性命轻毫发,谁把纲常重一分?\\
寰海已知夸令德,皇天终不丧斯文。\\
英雄自古多磨折,好拂青萍建大勋。\\
\stopverse

先生服其言,且感其意,乃决意赴谪。索笔,题一绝于殿壁。诗曰:

\startverse[leftalign=yes,sample={险夷原不滞胸中,何异浮云过太空!}]%
险夷原不滞胸中,何异浮云过太空!\\
夜静海涛三万里,月明飞锡下天风。\buffernote[2-5]
\stopverse

\startplacefigure[location={middle,top},title={先是,余姚诸公养和为江西参议,先生就婚贰室,尝遇一道士于玄妙观,约二十年后再见海上。至是僧延入寺,道士迎笑曰:\quota{候此久矣!}出诗以赠,有\quota{二十年前曾见君,今来消息我先闻}之句。先生告以将远遁。曰:\quota{尔有亲在,万一瑾怒,追尔父,诬尔北走胡,南走越,则族且赤矣。}先生瞿然,乃相与斋戒揲蓍,得箕子之《明夷》,遂决策从上饶以归。\buffernote[2-6]}]%
  \externalfigure[imgs/王阳明图传/image00126.jpeg][normal-img-60][width=.6\textwidth]
\stopplacefigure

先生辞道者,欲行。道者曰:\quota{吾知汝行资困矣。}乃于囊中出银一锭为赠。\buffernote[2-2-7]
先生得此盘缠,乃从间道游武夷山,出铅山,过上饶,复晤娄一斋。
一斋大惊曰:\quota{先闻汝溺于江,后又传有神人相救,正未知虚实。今日得相遇,乃是斯文有幸!}
先生曰:\quota{某幸而不死,将往谪所,但恨未及一见老父之面,恐彼忧疑成病,以此介介耳。}
娄公曰:\quota{逆瑾迁怒于尊大人,已改官南京宗伯矣。此去归途便道,可一见也。}
先生大喜。娄公留先生一宿,助以路费数金。
先生径往南京,省觐龙山公。父子相见,出自意外,如枯木再花,不胜之喜。
居数日,不敢久留,即辞往贵州,赴龙场驿驿丞之任。携有仆从三人,始成行李模样。\buffernote[2-3-9]

龙场地在贵州之西北,宣慰司所属,万山丛棘中,蛇虺成堆,魍魉昼见,瘴疠蛊毒,苦不可言。
夷人语言,又皆鴂舌难辩。居无宫室,惟累土为窟,寝息其中而已。
夷俗尊事蛊神,有中土人至,往往杀之以祀神,谓之祈福。
先生初至,夷人欲谋杀先生,卜之于神,不吉。
夜梦神人,告曰:\quota{此中土圣贤也。汝辈当小心敬事,听其教训。}
一夕而同梦者数人,明旦转相告语。
于是有中土往年亡命之徒能通夷语者,夷人央之通语于先生,日贡食物,亲近欢爱如骨肉。
先生乃教之范木为墼\low{\hw 音激},。架木为梁,刈草为盖,建立屋宇。人皆效之,于是一方有栖息之所。夷人又以先生所居湫隘卑湿,别为之伐木构室,宽大其制。于是有寅宾堂、何陋轩、君子亭、玩易窝,统名曰龙冈书院。翳之以桧竹,莳之以卉药。先生日夕吟讽其中,渐与夷语相习,乃教之以礼义孝悌,亦多有他处夷人特来听讲,先生息心开导,略无倦怠之色。\topnote{龙场之谪,先生之不幸,贵州之大幸也。}

\startplacefigure[location={middle,top},title={三年戊辰四月,萍乡谒濂溪祠,游岳麓,得霁,作《吊屈平赋》,泛沅湘,道常德、辰州以入龙场。诸生冀元亨等谒先生讲学,及归,复留虎溪龙兴寺。尝贻书诸生曰:近世士大夫亦知求道,而实德未成,先标榜以来谤,宜刊落声华,于切己处着实用力。前所云静坐,非欲坐禅入定,因平日为事物纷拏,欲以此补小学收放心一段工夫。\buffernote[2-7]}]%
  \externalfigure[imgs/王阳明图传/image00127.jpeg][normal-img-60][width=.6\textwidth]
\stopplacefigure

久之,得家信,言逆瑾闻先生不死,且闻父子相会于南都,益大恚忌,矫旨,勒龙山公致仕还乡。
先生曰:\quota{瑾怒尚未解也。得失荣辱,皆可付于度外。惟生死一念,自省未能超脱。}
乃于居后凿石为椁,昼夜端坐其中,胸中洒然,若将终身,夷狄、患难俱忘之矣。\buffernote[2-8]
仆人不堪其忧,每每患病,先生辄宽解之。又或歌诗制曲,相与谐笑,以适其意。
因思:\quota{设使古圣人当此,必有进于此者,吾今终未能免\quota{排遣}二字,吾于格致工夫未到也。}
忽一夕,梦谒孟夫子。孟夫子下阶迎之,先生鞠躬请教。孟夫子为讲\quota{良知}一章,\buffernote[2-9]
千言万语,指证亲切,梦中不觉叫呼,仆从伴睡者俱惊醒。自是,胸中始豁然大悟。
叹曰:\quota{圣贤左右逢源,只取用此\quota{良知}二字。
所谓格物,格此者也;所谓致知,致此者也。
不思而得,得甚么?不勉而中,中甚么?总不出此良知而已。
惟其为良知,所以得不由思,中不由勉\low{\hw 或作:得不繇思,中不繇勉}。
若舍本性自然之知,而纷逐于闻见,纵然想得着,做得来,亦如取水于支流,终未达于江海,
不过一事一物之知,而非原原本本之知。
试之变化,终有窒碍,不由我做主。必如孔子\quota{从心不踰矩},方是良知满用。
故曰:\quota{无入而不自得焉。}如是,又何有穷通荣辱死生之见得以参其间哉?}\buffernote[2-10]
于是,默记\quota{五经}以自证其旨,无不吻合,因著《五经臆说》。\buffernote[2-11]
水西安宣慰闻先生之名,遣使馈米肉,又馈鞍马金帛,先生俱辞不受。
夷人传说,益加敬礼。时正德三年,先生三十七岁事也。

\startplacefigure[location={middle,top},title={龙场,古夷蔡之外。先生至,无可居。茇于丛棘,继迁东峰,就石穴。从者俱病,自折薪取水为糜以调之。夷人卜蛊神进毒,神曰:\quota{天人也,彼不害尔,尔何为害彼?}乃相率罗拜。先生和易诱谕之,遂伐木架屋,作寅宾堂\buffernote[2-14]、何陋轩、君子亭、玩易窝,为终身计。诸生闻之,皆亦来集。稍暇,作《五经臆说》,体验探求,一夕如神启,悟致知格物之旨,证诸六经四子,沛然若决江河。}]%
  \externalfigure[imgs/王阳明图传/image00128.jpeg][normal-img-60][width=.6\textwidth]
\stopplacefigure

\startplacefigure[location={middle,top},title={何陋轩记:昔孔子欲居九夷,人以为陋。孔子曰:\quota{君子居之,何陋之有?}守仁以罪谪龙场。龙场,古夷蔡之外,于今为要绥,而习类尚因其故。人皆以为予自上国而往,将陋其地,弗能居也。而予处之旬月,安而乐之,求其所谓甚陋者而莫得。独其结题鸟言,山棲羝服,无轩裳宫室之观、文仪揖让之缛。然此犹淳庞质素之遗焉。盖古之时,法制未备,则有然矣,不得以为陋也。夫爱憎面背,乱白黝,浚奸穷黠,外良而中螫,诸夏盖不免焉。若是而彬郁其容,宋甫鲁掖,折旋矩矱,将无为陋乎?夷之人乃不能此,其好言恶詈,直情率遂,则有矣。世徒以言辞物采之眇而陋之,吾不谓然也。始予至,无室以止,处于丛棘之间,则郁也。迁于东峰,就石穴而居之,又阴以湿。龙场之民,老稚日来,视予喜,不予陋,益孚比。予尝圃于丛棘之后,民谓予之乐之也,相与伐木阁之材,就其地为轩以居予。予因而翳之以桧竹,莳之以卉药;列堂阶,辨室奥;编琴图史,讲}]%
  \externalfigure[imgs/王阳明图传/image00130.jpeg][normal-img-60][width=.6\textwidth]
\stopplacefigure

\startplacefigure[location={middle,top},title={续前文:诵游适之道略具。学士之来游者,亦稍稍而集于是。人之及吾轩者,若观于通都焉,而予亦忘予之居夷也。因轩扁曰\quota{何陋},以信孔子之言。嗟夫!诸夏之盛,其典章礼乐,历圣修而传之,夷不能有也,则谓之陋固宜。于后蔑道德而专法令,搜抉钩絷之术穷,而狡匿谲诈,无所不至,浑朴尽矣。夷之民方若未琢之璞、未绳之木,虽粗砺顽梗,而椎斧尚有施也,安可以陋之?斯孔子所谓\quota{欲居}也欤?虽然,典章文物则亦胡可以无讲?今夷之俗,崇巫而事鬼,渎礼而任情,不中不笄,卒未免于陋之名,则亦不讲于是耳。然此无损于其质也。诚有君子而居焉,其化之也盖易。而予非其人也,记之以俟来者。弟守仁谪居龙场,久而乐之,聊寄此以慰舜功年丈远怀。}]%
  \externalfigure[imgs/王阳明图传/image00129.jpeg][normal-img-60][width=.6\textwidth]
\stopplacefigure

明年癸巳\buffernote[2-12],贵州提学副使席书,号元山,亦究心于理学。
素重先生之名,特遣人迎先生入于省城,叩以:\quota{致知力行是一层工夫,还是两层工夫?}
先生曰:\quota{知行本自合一,不可分为两事。就如称其人知孝知弟,必是已行过孝弟之事,方许能知。
又如知痛,必然已自痛了;知寒,必然已自寒了。知是行的主意,行是知的工夫。
古人只为世人贸贸然胡乱行去,所以先说个知,不是画知行为二也。若不能行,仍是不知。}\buffernote[2-13]
席公大服,乃建立贵阳书院,身率合省诸生以师礼事之,有暇即来听讲。先生乃大畅良知之说。

正德五年,安化王寘鐇反,以诛刘瑾为名。
朝廷遣都御史杨一清、太监张永率师讨之,未至,而寘鐇已为指挥使仇钺用谋擒缚。
一清因献俘,阴劝张永以瑾恶密奏,永从之。武宗皇帝听张永之言,族瑾家,并诛其党张文冕等。
凡因瑾得官者尽皆罢斥,召复直谏诸臣,先生得升庐陵县知县。
临行之际,缙绅士民送者数千人,俱依依不舍。过常德、辰州,一路讲学,从游者甚众。
有《睡起写怀》诗为证:

\startverse[leftalign=yes,sample={险夷原不滞胸中,何异浮云过太空!}]%
红日熙熙春睡醒,江云飞尽楚山青。\\
闲观物态皆生意,静悟天机入窅冥。\\
道在险夷随地乐,心忘鱼鸟自流形。\\
未须更觅羲皇事,一曲沧浪击壤听。\buffernote[2-15]
\stopverse

先生时年三十九岁。
既至庐陵,为政不事刑威,惟以开导人心为本。慎选里正三老,坐申明亭,凡来讼者,使之委曲劝谕。
百姓有盛气而来,涕泣而归者。由是囹圄日清,风俗大变。\buffernote[2-16]
城中失火,先生公服下拜,天为之反风。乃令城市各辟火巷,火患永绝。
\topnote{如此邑令,地方幸矣,朝廷积谷助饷将何从出?呜呼!此今日所以无贤令也。}

\startplacefigure[location={middle,top},title={五年庚午,先生自宜春,以三月至庐陵。谕父老子弟以诚信相处,凡诉者周悉区画,俾咸得其平。督庠序,以崇正学,选教读以教童稚。暇则练民壮,亲肄以技艺。民病疫,命医携药户疗之。邑南左隅及钟楼前两遇火,亲临拜以祷,反风而灭。}]%
  \externalfigure[imgs/王阳明图传/image00131.jpeg][normal-img-60][width=.6\textwidth]
\stopplacefigure

是冬,入觐,馆于大兴隆寺,与湛甘泉、储柴墟讳巏。等讲致良知之旨。
进士黄宗贤等闻其说而叹服,遂执贽称门生听讲。
十二月,升南京刑部主事。湛甘泉恐废讲聚,言于冢宰杨一清。
明年正月,即调北京吏部验封司主事。
时有吏部郎中方叔贤讳献夫,位在先生之上,闻先生论学有契,遂下拜,事以师礼。先生赠以诗云:

\startverse[leftalign=yes,sample={休论寂寂与惺惺,不妄由来即性情。}]%
休论寂寂与惺惺,不妄由来即性情。\\
却笑殷勤诸老子,翻从知见觅虚灵。\buffernote[2-17]
\stopverse

是年十月,升文选司员外。明年三月,升考功司郎中。弟子益进,
如穆孔晖、冀元亨、顾应祥、郑一初、王道、梁谷、
万潮、陈鼎、魏廷霖、萧鸣凤、林达、黄绾、应良,皆一时之表表者,余人不可尽述。
徐爱等亦至京师,一同受业。
先生尝言:\quota{格物是诚意的工夫,明善是诚身的功夫,穷理是尽性的功夫,
道问学是尊德性的功夫,博文是约礼的功夫,惟精是惟一的功夫。}
\buffernote[2-18]\topnote{良知一以贯之,正此谓也。}
诸如此类,乍闻之,亦自骇然。其后思之既久,转觉亲切不可移动。
十二月,升南京太仆寺少卿,驻扎滁州,专督马政,便道归省。
未几,至滁州。门人从者颇众。地僻官闲,日与门人游遨琅琊山在州城。
瀼泉即六一泉。之间。月夕,则环龙潭在龙蟠山。
而坐者数百人,歌声振谷。诸生随地请益,先生就眼前点化,各有所得。\buffernote[2-19]
于是,从游益盛。\buffernote[2-20]

正德九年四月,升南京鸿胪寺卿。滁阳诸友送至江浦,不忍言别,遂各赁居,候先生渡江。
先生以诗促之使归,诗曰:

\startverse[leftalign=yes,sample={君不见尧羮与舜墙,又不见孔与跖对面不相识?}]%
滁之水,入江流。江潮日复来滁州。\\
相思若潮水,来往何时休?\\
空相思,亦何益?欲慰相思情,不如崇令德。\\
掘地见泉水,随处无弗得,\\
何必驱驰为?千里远相即。\\
君不见尧羮与舜墙,又不见孔与跖对面不相识?\\
逆旅主人多殷勤,出门转盼成路人。\buffernote[2-21]
\stopverse

五月,至南京,徐爱等相从,
又有黄宗明、薛侃、陆澄、季本、萧惠、饶文璧、朱虎\buffernote[2-22]等二十余人,一同受业。

\startplacefigure[location={middle,top},title={九年甲戌夏,升南京鸿胪寺卿。弟守文来学,作《立志说》。\buffernote[2-23]门人季本、薛侃、黄宗明、马明衡、周绩、陆澄、刘晓、郭庆、栾惠等,讲学日众。}]%
  \externalfigure[imgs/王阳明图传/image00132.jpeg][normal-img-60][width=.6\textwidth]
\stopplacefigure

\stopsectionlevel

\startsectionlevel[default][title={破山中贼}]%

正德十年,先生念祖母岑太夫人年九十有六,思一修觐,乃上疏请告,不允。\buffernote[3-1]
时汀漳各郡皆有巨寇,兵部尚书王琼特举先生之才,升都察院左佥都御史,巡抚南、赣、汀、漳等处。
先生因得归省岑太夫人及龙山公。\buffernote[3-2]
正德十二年正月,赴任南赣。道经吉安府万安县,适遇流贼数百,肆劫商舟。
舟人惊惧,欲回舟避之,不敢复进。先生不许,乃集数十舟,联络为阵势,扬旗鸣鼓,若将进战者。
贼见军门旗号,知是抚院,大惊,皆罗拜于岸上,号呼曰:\quota{某等饥荒流民,求爷赈济活命!}
先生命将船从容泊岸,使中军官传令,
谕之曰:\quota{巡抚老爷知汝等迫于饥寒,一到赣后,即差官抚插,宜散归候赈。若更聚劫乡村,王法不宥。}
贼俱解散。\buffernote[3-3]
既抵赣,即行牌所属,分别赈济,招抚流民,置二匣于台前,榜曰:求通民情,愿闻己过。

\startplacefigure[location={middle,top},title={十一年丙子秋,升左佥都御史,巡南、赣、汀、漳、雄、韶、惠、潮、郴等处。\ldots{}\ldots{}先生初过万安,贼正劫百家滩,跳号无忌。下车,首督四省兵备,挑选民兵,别其饶贫。悬赏招募异材操演\ldots{}\ldots{}亲教以八阵法技艺。\buffernote[3-10]}]%
  \externalfigure[imgs/王阳明图传/image00133.jpeg][normal-img-60][width=.6\textwidth]
\stopplacefigure

因漳贼詹师富、温火烧等连年寇盗,其势方炽,移文湖广、福建、广东三省,克期进剿。
赣民多受贼贿,为之耳目。官府举动,贼已先觉。
先生访知军门有一老隶奸狡尤甚,忽召入卧室,
谓之曰:\quota{有人告你通贼,你罪在必死。若能改过,悉列通贼诸奸民告我,我当赦汝之命。}
老隶叩头,悉吐其实,备开奸民姓名。先生俱密拿正法。\buffernote[3-4]
又严行十家牌法。其法十家共一牌,开列各户籍贯、姓名、年貌、行业,日轮一家,沿门诘察。
遇面生可疑之人,即时报官。如或隐匿,十家连坐。所属地方,一体遵行。\buffernote[3-5]
\topnote{此弭盗良法,惜无善用之者。}
又以向来远调狼达上军,动经岁年,糜费巨万,骄横难制,有损无益。
乃使各省兵备官,令府州县挑选本地真正骁勇,每县多者十人,少者七八人。
大约江西、福建二省,各以五六百名为率;广东、湖广二省,以四五百名为率,
其间有魁杰出群、通晓韬略者,署为将领。所募骁勇,随各兵备官屯扎训练。
无事拨守城隘,有事应变出奇。\buffernote[3-6]到任十余日,调度略毕,即议进兵。
兵次长富村,遇贼,大战,斩获颇多。贼奔至象湖山拒守,我兵追至地名莲花石,与贼对垒。
会指挥覃桓率广东兵到,与贼战,小胜,遂进前合围。贼见势急,溃围而出。
覃桓马蹶,为贼所杀,县丞纪庸\buffernote[3-7]亦同时被害。
诸将气沮,谓:\quota{贼未可平,请调狼兵,俟秋再举。}
先生阳听其说,进屯汀州府上杭县,宣言:\quota{大犒三军,暂且退师蓄锐,俟狼兵齐集征进。}\buffernote[3-8]
密遣义官曾崇秀觇贼虚实,回言:\quota{贼还据象湖,只等官军一退,复出劫掠。}
先生乃责各军以失律之罪,使尽力自效。\buffernote[3-9]
分兵为二路,俱于二月廿九晦日,出其不意,衔枚并进,直捣象湖,夺其隘口。
众贼失险,复据上层,峻壁四面,滚木礌石,以死拒战。先生亲督兵士,奋勇攻之。
自辰至午,呼声震地。三省奇兵从间道攀崖附木,四面蚁集,贼惊溃奔走。官军乘胜追剿,贼兵大败。
先生乃分遣福建佥事胡琏、参政陈策、副使唐泽等,率本省兵攻长富村;
广东佥事顾应祥、都指挥杨懋等,率本省兵攻水竹大重坑;先生自提江西兵,往来接应。
不一月,福建兵攻破长富村巢穴三十余处,广东兵攻破水竹大重坑巢穴一十三处,
斩首从贼詹师富、温火烧等七千余名,俘获贼属及辎重无算。漳、南数十年之寇,至是悉平。
以二月出师,四月班师,成功未有如此之速者。

先生驻军上杭,久旱不雨。师至之日,一雨三日。百姓歌舞于道。
先生因名行台之堂曰\quota{时雨堂},取\quota{王师若时雨}之义也。\buffernote[3-11]
先生谓:\topnote{此即管子内政遗制,治军之法,莫妙于此。要在实实行之耳。}

\startquotation
习战之方,莫要于行伍;治众之法,莫先于分数。
每每调集各兵,二十五人编为一伍,伍有小甲;五十人为一队,队有总甲;
二百人为一哨,置哨长一人,协哨二人;四百人为一营,置营官一人,参谋二人;
一千二百人为一阵,阵有偏将;二千四百人为一军,军有副将。
偏将无定员,临事而设。小甲选于各伍中,总甲又选于小甲中,哨长选于千百户义官中。
副将得以罚偏将,偏将得以罚营官,营官得以罚哨长,哨长得以罚总甲,总甲得以罚小甲,小甲得以罚伍兵。
务使上下相维,如身臂使指,自然举动齐一,治众如寡。编选既定,每伍给一牌,备列同伍姓名,谓之伍符。
每队各置两牌,编立字号,一付总甲,一藏本院,谓之队符。
每哨各置两牌,编立字号,一付哨长,一藏本院,谓之哨符。
每营各置两牌,编立字号,一付营官,一藏本院,谓之营符。
凡遇征调,发符比号而行,以防奸伪。\buffernote[3-12]
\stopquotation

\startplacefigure[location={middle,top},title={先生修濂溪书院,创射圃,辟讲堂,门人周冲、周仲、郭治、刘魁、欧阳德、何秦、黄弘纲、刘肇衮、王学益等四至。立十家牌,联保甲,奸伪无所容。屡谕父老子弟孝亲敬长,奉法修睦,息讼罢争。作《谕俗四条》,\buffernote[3-13]修社学,择教读刘伯颂等教童子歌诗习礼,\buffernote[3-14]选耆儒袁庆奇主之,弦歌盈街市。}]%
  \externalfigure[imgs/王阳明图传/image00134.jpeg][normal-img-60][width=.6\textwidth]
\stopplacefigure

又疏请申明赏罚:\topnote{尤为要着。}

\startquotation
兵士临阵退缩者,领兵官即军前斩首;领兵官不用命者,总兵官即军前斩首。
其有擒斩功次,不论尊卑,一体升赏。生擒贼从勘明,决不待时。
夫盗贼之日滋,由招抚之太滥;招抚之太滥,由兵力之不足;兵力之不足,由赏罚之不行。
乞假臣等,以令旗令牌,使得便宜行事。\buffernote[3-15]
\stopquotation

又议割南靖、漳浦之地,建立县治于大洋陂,又添立巡简司,协同镇压。
兵部王琼以先生之言为然,覆奏俱依拟,赐县名曰清平,\buffernote[3-16]
改巡抚为提督军务,给旗牌,假便宜。\buffernote[3-17]仍论平漳寇功,加俸一级。
先生益得发舒其志。
再说南赣西接湖广桂阳,有桶冈、横水诸贼巢;东接广东龙川,有浰头诸贼巢。
横水贼首谢志珊、桶冈贼首蓝天凤、浰头贼首池仲容,俱僭号称王,伪署官职,
拥众据险,出入无常。屡调狼兵进讨,不能取胜。
谢志珊自号征南王,闻督府方讨漳寇,乃大修战具,并造吕公车若干,欲乘隙先破南康,乘虚入广。
时湖广巡抚都御史陈金,疏请三省之师夹攻桶冈。
先生曰:\quota{桶冈、横水、左溪诸贼,荼毒三省,其患虽同,而事势各异。
论湖广,则桶冈为腹心之疾;论江西,则横水为腹心之疾。
今不去江西腹心之疾,而欲与湖广夹攻桶冈,失缓急之宜矣。
湖广克期以十一月朔日会集,今尚在十月。横水贼闻湖广合剿之信,必谓我先攻桶冈。
又见我兵未集,师期尚远,必不准备。若出其不意,进兵疾击,可以得志。
已破横水,移兵桶冈,此破竹之势也。}\buffernote[3-18]
先生恐征横水时浰头贼乘机扰乱,乃为告谕一通,具述利害,
遣报效生员黄表、义民周祥等招抚池仲容等,劝之立功自赎,且各赐银布,以安其心。
一时,贼党见谕词诚恳,莫不感动。
酋长黄金巢、刘逊、刘粗眉、温仲秀等随黄表等各引部下出投,情愿杀贼立功。
先生用好言抚慰,选其精壮五百人为兵,随军征进,余老弱散遣之。
先生已定出师之期,预先分定哨道,密授方略,那几处哨道?

\startquotation
一、江西都司都指挥许清,率兵一千,自南康县所溪入,攻白蓝,与本院会于横水。\\
一、赣州府知府刑珣,率兵一千,自上犹县石人坑入,协攻白蓝,会于横水。\\
一、南赣守备郏文,率兵一千,自大庾县义安入,合攻左溪,会于横水。\\
一、汀州府知府唐淳,率兵一千,自大庾县聂都入,合攻左溪,会于横水。\\
一、南安府知府季敩,率兵一千,自大庾县稳下入,合攻左溪,会于横水。\\
一、南康县县丞舒富,率兵一千,自上犹县金坑入,径攻左溪,会于横水。\\
一、赣州卫指挥余恩,率兵一千,自上犹县独孤岭入,径攻左溪,会于横水。\\
一、宁都县知县王天与,率兵一千,自上犹县官隘员坑入,进屯横水。\\
一、吉安府知府伍文定,率兵一千,搜剿稽芜等处贼巢,进屯横水。\\
一、广东潮州府程乡县知县张戬,率兵一千,搜剿黄雀坳等贼巢,进屯横水。
\stopquotation

\startplacefigure[location={middle,top},title={横水大贼首谢志珊僭称征南王,桶冈蓝天凤称盘皇子孙,及鸡湖、双坝等巢,为乱湖广。夹攻以十一月攻桶冈,而横水、左溪观望未备。先生密布诸哨,各给旗号军令,以十月初七日分道并进\ldots{}\ldots{}}]%
  \externalfigure[imgs/王阳明图传/image00135.jpeg][normal-img-60][width=.6\textwidth]
\stopplacefigure

分拨十路军马,限定十月初七日,各哨齐发。
又拨兵备副使杨璋、分守参议黄宏监督各营官兵往来给饷。\buffernote[3-19]
先生暗谕:本院标下将领,同时进发。
号令虽出,衙门中寂然无闻。先生在赣院,左有旁门,通射圃,暇即与诸生讲学其中,或习射,每至夜分而散。
次早,则诸生入院揖谢,以此为常。出兵之前一日,与诸生夜坐谈论。诸生以先生坐久,请休息。\buffernote[3-20]
先生乃回院。及明旦,诸生集于院门,欲进谢,守门者辞曰:\quota{公进院未几,即领兵出城去,不知何往。
度此际可行二十余里矣。}其神机不测如此。\topnote{举动门且贼,安之必成功。}

先生于十月初九日兵至南康。有人出首义官李正岩、医官刘福泰,素与贼通者。
先生召二人至,以首状示之。二人力辩无有,先生曰:\quota{即有之,姑释汝罪。}乃皆留于幕下,戴罪立功。
景晩,李正岩、刘福泰禀有机密事求见,先生召入,密叩之。
二人齐声禀称:\quota{欲攻桶冈,必经由十八面地方,此乃第一险要去处。
乱山环拱,岭峻道狭,从来官军不能入。今有木工张保,久在蛮中,凡建立栅寨,皆出其手。
要知地利,非得此人不可。}先生问:\quota{张保何在?}
二人曰:\quota{某等蒙老爷不杀之恩,誓欲报效。天幸遇着张保,已拘留在辕门之外。未奉呼唤,不敢擅自引入。}
先生即令二人出外,同张保入见,务要隐密,不得声张其事。当下,李、刘二人引张保直至后堂,叩头。
先生曰:\quota{闻蛮贼建立栅寨,皆出汝手,汝罪当死!}
张保连连叩头,答曰:\quota{小人手艺为活,误入贼穴,一时贪生怕死,受其驱使,实非得已。}
先生曰:\quota{我且不计较汝。但彼立寨之处,必然选择险要,汝在彼中,亦必备知,
        可细细开明左右前后大小出入之道。贼破之日,一例叙功。}
张保欣然,遂请求笔砚。先生分付李、刘二人监押,教他安坐开写。自己退回卧房,使亲随门子以酒食劳之。
张保感激,即备细开出某贼寨在某山,某处是进路,某处是退路,某处山头与某寨相对,路平路险,
如何上山,如何下山,恰像写卖山文契的,四址分明,滴水不漏。门子禀道:\quota{木工开写已完。}
先生复出召见,亲自收取,看了一遍,再把好言抚慰,即留三人于内堂厢房安歇。次早,皆授义官名色。
初十日,兵进至南坪地方,使李正岩、刘福泰引着间谍,四路分探,
回报:\quota{众贼不虞官兵猝至,各巢皆鸣锣聚众,往来呼噪,为分头御敌之计,势甚张皇。
各险隘皆设有滚木礌石,已做准备。}先生乃乘夜疾进。

十一日,离贼巢三十里下寨,使人伐木立栅,开堑设堠,示以久屯之形。
使报效听选官雷济、义民萧廋,分率乡兵及樵竖善登山者四百人,各给旗一面,
赍铳、炮、钩、镰、枪,使由间道攀崖悬壁而上,分伏各山顶高处,预堆积茅草,
约定次日官军进攻各山头,将旗竖立,举炮燃火相应。

十二日,官军至十八面隘。贼方据险迎敌,忽闻远近山顶炮声如雷,烟焰四起,官军呼噪奋勇,炮箭齐发。
贼惊皇失措,以为巢穴已破,遂弃险奔溃。
先生预遣千户陈伟、高睿分率壮士数十悬崖而上,夺其险隘,尽发其木石。官军乘胜急进,呼声震天。
指挥谢昶、冯廷瑞由间道先入,放火焚贼巢。贼退无所据,乃大败,四散奔走,遂连破长龙、十八面隘等七巢。
贼首谢志珊与萧贵模计议,谓横水居众险之中,可倚以自固。及闻官军四进,仓卒分众阨险出御。
见横水烟焰障天,铳炮之声揺撼山谷,心胆愈裂,弃险而逃。
时各哨官兵陆续俱到,刑珣兵破磨刀坑等三巢,王天与破樟木坑等二巢,许清破鸡湖等三巢,俱至横水来会。
唐淳破羊牯脑等三巢,并破左溪大巢,郏文破狮寨等三巢,余恩破长流坑等三巢,
舒富破箬坑等三巢,季敩破上西峰等三巢,俱至左溪,守巡各官亦随后而至。
是日,斩大贼钟明贵、陈曰能等数人,从贼首级千余。其自相蹂践堕崖填谷而死者不计其数。
贼于入路皆刊崖倒树,设阱埋签。
官军昼夜涉深涧,蹈丛棘,遇险绝,则挂绳于崖树,鱼贯而上,猿擘而下,往往失堕深谷,不死为幸。
各兵至横水左溪者,皆疲困不能驱逐。会日暮,传令收兵屯扎。
至次日,大雾咫尺不辨,先生令各营休兵享士,使乡导数十分探溃贼何在,并未破巢穴动静。
连日雾雨,至十五日,尚濛濛不开。
各乡导回报,言诸贼预于各山绝险崖壁立寨为退保计,亦有并聚于未破各巢者。
诸将皆曰:\quota{会剿桶冈,期在十一月朔,日已迫矣,奈何?}
先生曰:\quota{此去桶冈尚百余里,山路绝崄,三日方达。
若此处之贼未能扫尽,而移兵桶冈,瞻前顾后,备多力分,非计之得也。}
适搜山者擒一贼至,问之,乃是桶冈贼遣至横水探信者,姓钟,名景。
先生曰:\quota{吾兵所向皆克,灭桶冈只待旦夕。汝若肯留吾麾下效用,当赦汝罪。}
钟景叩头愿降。先生因叩桶冈地利。钟景言之甚详,兼能识横水各巢路道。
先生遂解其缚,赐以酒食,留于帐下。于是传令各营,皆分兵为奇正二哨,一攻其前,一袭其后,冒雾疾趋。

十六日,刑珣攻破旱坑等二巢,季敩同郏文攻破稳下等二巢。
十七日,唐淳攻破丝茅坝巢。
十八日,许清攻破朱雀坑等四巢。
十九日,余恩攻破梅坑等二巢。
二十日,刑珣又破白封龙等二巢,王天与破黄泥坑。
二十二日,舒富破白水洞巢。是日,伍文定、张戬兵亦至。
二十四日,伍文定破寨下巢,张戬破杞州坑巢。
二十五日,张戬又破朱坑巢,伍文定破杨家山巢。
二十六日,季敩又破李坑巢,许清又破川坳巢。
二十七日,郏文又攻破长河洞巢,俘斩无数。
谢志珊谋遁桶冈,被刑珣活捉解来。先生奉新奏准事例,即命于辕门枭首。
临刑,先生问曰:\quota{汝一介小民,何得聚众如此之多?}
志珊曰:\quota{此事亦非容易!某平日见世上有好汉,决不肯轻易放过,必多方钩致,与为相识,
或纵其饮,或周其乏。待其感德,然后吐实告之,无不乐从矣。
负千觔气力者五十余人,今俱被杀,束手就缚,乃明天子之洪福也,又何尤哉?}因瞑目受刑。
先生他日述此事于门人,曰:\quota{吾儒一生求朋友之益,亦当如此。}\buffernote[3-21]
后人论此语,不但学者求朋友当如此,虽吏部尚书为天下求才,亦当如此。有诗四句云:

\startverse[leftalign=yes,sample={同志相求志自同,岂容当面失英雄?}]%
同志相求志自同,岂容当面失英雄?\\
秉铨谁是怜才者?不及当年盗贼公。
\stopverse

考陆天池《史余》上说:先生微服与木工同入贼寨,自称工师,兼通地理。贼喜其辩说,礼为上客。
先生周行其穴,密籍其险要可藏之处,绐贼,以五百人随出,约伏官军营侧,克期出兵为应。贼从其计。
先生至军中,悉配其人于四郊,各不相通。自选精卒千人诈降,密携火器,埋之贼境,又辞归。
至期,率兵数万而进。贼启关出迎,洞中火炮大发。精卒从后夹击,贼惶惑不能支,遂大败。
平贼后,取五百人者,剜其目睛,而全其命。
今按先生《年谱》:自起兵至平贼,才二十日耳,如疾雷迅霆,安得有许多曲折?
且自称工师,往来诱敌,旷日持久,亦非万全之策。此乃小说家传言之妄,当以《年谱》为据。

再说是日诛了谢志珊,诸将遂请乘胜进攻桶冈。先生询访钟景等,已知地势之详,
谓诸将曰:\quota{桶冈天险四塞,其出入之路,惟锁匙龙、葫芦洞、茶坑、十八磊、新池五处。
然皆架栈梯壑,一人守之,千人难过。止有上章一路稍平,然非半月不可达。奔驰之际,彼已知备矣。
莫若移屯近地,休兵养威,谕以祸福。彼见吾兵累胜,必惧而请服,如其迟疑,当进而袭之。}
乃遣戴罪义官李正岩、医官刘福泰,并降贼钟景,于二十八夜往桶冈,招安蓝天凤等。
如果愿降,待以不死。期定于十一月初一日上午,至锁匙龙送款。

话分两头,却说浰头贼首池仲容,绰号池大鬓,原是龙川县大户出身,
因被仇家告害,官府不明,一时气愤,与其弟仲宁、仲安聚起家丁庄户,杀了仇家一十一口,
遂招集亡命,占住三浰落草。屡败官军,渐渐势大,自号金龙霸王。
伪造符印,以兵力胁远近居民,壮者收为部下,富者借贷银米,稍有违抗,焚杀无遗。
龙川大姓卢珂、郑志高、陈英三人,颇有本事,各聚众千余,保守乡村。
仲容欲招至入伙,卢珂等不从,互相仇杀。
先生檄岭东兵备道,先招卢珂等三家。三家遂奉约束,愿出力剿贼。
遂留本村,与龙川县协同备御,仲容深恨之。及黄金巢等出降,众贼俱有纳款之意,惟池仲容不肯,
谓众贼曰:\quota{我等作贼,已非一年;官府来招,亦非一次。其言未足凭信!
且待黄金巢等到官后,果无他说,我等遣人出投,亦未为晩。}
及闻十月十二日官兵已破横水,仲容始有惧色。适先生又使黄金巢等作书往招。
仲容乃谓其党高飞甲曰:\quota{官军既破横水,必乘胜直捣桶冈,次即及浰头矣,奈何?}
高飞甲曰:\quota{前督抚曾遣人来招安,且闻黄金巢等已蒙署官录用,不若亦遣一人出投,
一则缓其来攻,二则窥觑虚实。若官军势果强盛,招安果系实情,又作计较。
不然,留仲安在彼处,亦好潜为内应,一面拨人守险,多备木石,以防掩袭。}
仲容以为然,乃遣其弟仲安率老弱二百余人,往至横水投降,情愿随众立功。\buffernote[3-22]
时横水贼已全平矣。先生谓曰:\quota{汝既是真心纳降,本院即日加兵桶冈,汝可引本部兵往上新地屯扎。
如桶冈贼奔逸到彼,用心截杀,将首级来献,便算你功。}
那上新、中新、下新三巢,是桶冈西路,去浰头甚远。
先生故意调开,使其难归,外示委用,以安其心,此是先生妙计。

再说李正岩等至桶冈,先述督抚兵威,后述招抚之期。蓝天凤大喜,情愿就抚。
方召其党商议此事,横水贼萧贵模逃入桶冈,来见天凤,
曰:\quota{征南王不知守险,使官军潜入内地,是以溃败。若加意隄防,虽有百万之众,岂能飞入?
今锁匙龙各隘地皆绝险,其所收横水余兵,尚有千余,足可助桶冈为守。奈何自就死地,如猪羊入屠人之手乎?}
天凤意不能决,乃令各寨头目俱至锁匙龙聚议。
先生遣县丞舒富率数百人,逼锁匙龙下寨,连连遣使,催取天凤等款状。
一面密使刑珣兵入茶坑,伍文定兵入西山界,唐淳兵入十八磊,张戬兵入葫芦洞,立限三十日,乘夜各至分地。
是夜大雨,不得进。初一日早,雨犹未止,各军冒雨而入。
天凤见屡使催款,正在商量,又见大雨,料难进兵,防备就懈弛了。
忽闻四路兵已大进,惊曰:\quota{王公用兵真如神矣!}急收拾兵众千人,据内隘绝壁,隔水为阵,以拒官军。
邢珣率兵渡水前击,张戬之兵冲其右,伍文定又自戬兵之右,悬崖而下,绕贼傍合攻。
贼不能支,且战且却。及午雨止,各兵奋击,贼大败。
王天与、舒富两路兵闻官军已入前山,亦从锁匙龙并登,各军乘胜奋击。
贼悉望十八磊奔逃,正遇唐淳之兵严阵以待,又大战一场。会日暮暂息,贼犹扼险相持。
次早,诸军复合势剿杀,贼遂大败。凡破十三巢,擒斩无数。
初五日至十三日,陆续又破上新、中新、下新等十巢,斩萧贵模于阵。
蓝天凤率败兵,欲于桶冈后山乘飞梯直入范阳大山,却先被官军把守,前后困围,计无复之,乃投崖而死,
枭其首以献。岩谷溪壑之间,僵尸填满,于是桶冈之贼略尽。

据先生报二处捷数目:

\startquotation
捣过巢穴共八十四处,\\
擒斩大贼首谢志珊、蓝天凤等八十六名颗,\\
从贼首级三千一百六十八名颗,\\
俘获贼属二千三百三十六名口,\\
夺回被虏男妇八十三名口,\\
牛马驴一百八只,\\
赃仗二千一百三十一件,\\
金银一百一十三两八钱一分。\buffernote[3-23]
\stopquotation

时湖广军门已遣参将史春统兵前来会剿,行至郴州,接得先生钧牌,知会桶冈贼巢俱已荡平,不必复劳远涉。
史春大惊曰:\quota{向议三省合剿,打帐一年,尚恐未能尽殄。今王督院之兵朝去夕平,如扫秋叶,真天人也!}
先生奏凯班师,百姓扶老携幼,手香罗拜,言:\quota{今日方得安枕而卧。}
所经州县关隘,各立生祠。远乡之民肖像于家堂供养,岁时尸祝。先生谓:
横水、桶冈各贼寨,散在大犹、庾岭之间,地方窎远,号令不及。
议割三县之地,建立县治,及增添三处巡司,设关保障。
疏上,悉依议,赐县名曰\quota{崇义},附江西南安府,赐敕奖谕。

浰头贼闻桶冈复破,愈加恐惧,乃分兵为守隘拒敌之计。
先生先谕黄金巢等,密遣部下,散归贼巢左近,俟官兵一到,即据险遏贼。再谕卢珂、郑志高等,用心提备。
然后遣生员黄表、义民周祥等,赍牛酒复至浰头,赏劳各酋长,并诘其分兵守隘之故。
池仲容无词可解,乃诈称:\quota{龙川义民卢珂、郑志高素有仇怨,今不时引兵相攻,若一撤备,必被掩袭。
某等所以密为之防,非敢抗官兵也。}
遂遣其党鬼头王,随黄表等回报,请宽其期,当悉众出投,尽革伪号,止称新民。
先生阳信其言,遂移檄龙川,使察卢珂等擅兵仇杀之实,
谓鬼头王曰:\quota{卢珂等本院已行察去讫,如情罪果真,本院当遣大军往讨,但须假道浰头。
汝等既降,先为我伐木开道,以候官军,不日征进。}
鬼头王回报,池仲容且喜且惧:所喜者,督院嗔怪卢珂等,堕其术中;所惧者,恐其取道浰头,不是好意。
复遣鬼头王来谢,且禀称:\quota{卢珂等某自当悉力捍御,不敢动劳官军。}
恰遇卢珂、郑志高、陈英亲到督院具状,辩明其事。
状中备述池仲容等平昔僭号设官,今又点集兵众,号召远姓各巢贼酋,授以总兵都督等伪官,准备抗拒官军。
先生大怒曰:\quota{池仲容已自投招,便是一家。
汝挟仇,擅自仇杀,罪已当死。又造此不根之言,乘机诬陷,欲掩前罪,本院如见肺肝。
那池仲容方遣其弟池仲安领兵报效,诚心归附,岂有复行抗拒之事?}
遂扯碎其状,诧之使出:\quota{再来渎扰,必斩!}
却教心腹参谋,密向他说:\quota{督府知汝忠义,适来佯怒,欲哄诱浰头自来。
你须是再告,告时受杖三十,暂系数旬,方遂其计。}
卢珂等依言,又来告辩。先生益怒,喝令缚珂等斩首来报,标下众将俱为叩头讨饶。
先生怒犹未解,将卢珂责三十板,喝令监候。
池仲安等在幕下,闻珂等首辩,心怀惊惧。及见先生两次发怒,然后大喜,率其党欢呼罗拜,争诉珂等罪恶。
先生曰:\quota{本院已体访明白,汝可开列恶款来,待我审实后,当尽收家属处斩,以安地方。}
仲安益大喜,作家书,付鬼头王,回报其兄仲容去讫。
卢珂等既入监,先生又使心腹参随,只说要紧人犯在监,不放心,教他巡阅。
却暗地致督府之意,安慰珂等,说:\quota{事成之日,当有重用。你可密地分付家属,整顿人马,伺候军令差遣。}
珂等感泣曰:\quota{督府老爷为地方除害。若用我之时,虽肝脑涂地,亦无所恨。}
先生又使生员黄表、听选官雷济安慰池仲容,说:\quota{督府已知卢珂等仇杀之情,汝等勿以此怀疑。}
仲容大排筵席,管待黄表、雷济二人,
坐中夸:\quota{督府用兵如神,更兼宽宏大量,来者不拒,黄金巢等俱授有官职。你等若到麾下,自当题请重用。}
仲容拱手曰:\quota{全仗先生们提挈。}
黄表因私谓所亲信贼酋曰:\quota{卢珂等说令兄恶迹多端,无非是妒忌之意。虽然督府不信,令兄处也该自去投诉。}
仲宁唯唯。言于仲容,仲容迟疑不行。

十二月二十日,先生大军已还南赣,各路军马俱已散遣,回归本处。先生乃张乐设饮,大享将士。示谕城中云:

\startquotation
督抚军门示:向来贼寇抢攘,时出寇掠,官府兴兵转饷,骚扰地方,民不聊生。
今南安贼巢尽皆扫荡,而浰头新民又皆诚心归化,地方自此可以无虞。民久劳苦,亦宜暂休息为乐。
乘此时和年丰,听民间张灯鼓乐,以彰一时太平之盛。
\stopquotation

先生又曰:\quota{乐户多住龟角尾,恐有盗贼藏匿,仰悉迁入城中,以清奸薮。}
于是街巷俱燃灯鸣鼓,倡优杂沓,游戏为乐。
先生又呼池仲安至前,谓曰:\quota{汝兄弟诚心向化,本院深嘉。
闻卢珂党与最众,虽然,本身被系,其党怀怨,或掩尔不虞,事不可知。
今放尔暂归浰头,帮助尔兄防守。传语尔兄,小心严备,不可懈弛失事。}
仲安叩头感谢。先生又使指挥俞恩护送仲安,并赍新历,颁赐诸酋。诸酋大喜,盛筵设款。
仲安又述督府散兵安民及遣归协守之意,无不以手加额,踊跃谢天。
时黄表、雷济尚留寨内,会饮中间,仲容说道:\quota{我等若早遇督府,归正久矣。}
表、济曰:\quota{尔辈新民,不知礼节。今官府所以安辑劳来尔等甚厚,况且遣官颁历,奈何安坐而受之?
论礼亦当亲往一谢。}
余恩曰:\quota{此言甚当!况卢珂等日夜哀诉,说你谋反有据。官府若去拘他,他断然拒命不来。
何不试拘对理,看他来与不来,即此可证反情之实。}
仲容曰:\quota{若督府来唤对理,岂有不去之理?}
表、济又曰:\quota{今若不待拘唤,竟往叩谢,须便就诉明卢珂等罪恶,官府必益信尔无他,珂等诈害是实,杀之必矣。}
所亲信贼酋亦从中力劝。仲容以为然,
乃谓其众曰:\quota{若要伸,先用屈。输得自己,赢得他人。赣州伎俩,亦须亲往勘破。}
遂定计,选麾下好汉,并所亲信者共九十三人,亲至赣州,来见督府。
仲宁、仲安留于本寨。\buffernote[3-24]余恩等先驰归报。
先生乃密遣人传谕属县:\quota{勒兵分哨付本院,不时檄到即发。}
又遣千户孟俊,先至龙川,督集卢珂、郑志高、陈英三家兵众。
又以路从浰巢经过,恐其起疑,于是另写一牌,牌上开写:
卢珂等擅兵仇陷过恶,仰龙川县密拘三家党属,解至本院问究。
却将真牌藏于贴肉秘处。孟俊行至浰头,贼党一路盘问,俊出牌袖中示之,
故意嘱他:\quota{此官府秘密事情,万勿泄漏!}
贼皆罗拜,争献酒肉,为之向导,送出浰巢。一路上其党自相传说,无不欢喜。
孟俊到了龙川,方出真牌,部勒三家兵众。巢中诸贼传闻,皆以为拘捕其党,并不他疑。
仲容等到于赣州,正似猪羊近屠户之家,一步步来寻死地。
仲容把一行人众营于教场,单引亲信数人进院参谒。先生用好言抚慰,问:\quota{此来许多人众?}
仲容禀曰:\quota{随从不过九十余人。}
先生曰:\quota{既是九十余人,必须拣个极宽的去处安顿方好。}问中军官:\quota{何处最为宽闲?}
中军官禀道:\quota{惟有祥符寺地最宽敞,房屋亦俱整齐。}
先生曰:\quota{就引至祥符寺居住罢。}又问:\quota{众人今在何处?}
中军官不等仲容开口,便禀道:\quota{众人见屯教场。}
先生伪变色曰:\quota{尔等皆我新民,不来见我,而营于教场,莫非疑心本院么?}
仲容惶恐,叩首曰:\quota{就空地暂息,听老爷发放,岂有他意?}
先生曰:\quota{本院今日与你洗雪,复为良民,也非容易。你若悔过自新,学好做人,本院还有扶持你处。}
仲容叩谢而出。

既至祥符寺,见宫室整洁,又有参随数人为馆伴,赐以米薪酒肉,
标下各官俱来相拜,各有下程相送,欢若同僚,喜出望外。时乃闰十二月二十三日也。
参随等日导众贼游行街市,见各营官军果然散归,街市上张灯设戏,宴饮嬉游,信以为督府不复用兵矣。
又密赂狱卒,私往觇卢珂等动静,果然械系深固。
狱卒又说:\quota{官府已行牌,拘其家属,一同究问,不日取斩。}
仲容大喜曰:\quota{吾事今日始得万全也!}
先生复制长衣油靴,分给众贼,使参随教之习礼。一日,又漫给布帛,未曾开明分别赏赐,于是老少互争。
参随禀知,先生曰:\quota{本院多事,未及细开,何不教他开一花名手本?下次照依次序给赏,老少不乱,岂不便乎?}
仲容依言,开手本送上,于是尽得其九十三人名姓。
过五日,仲容等辞归,先生曰:\quota{自此至浰有八九日程途,即今往,不能到家过岁矣。
新春少不得又来贺节,多了一番跋踄。况赣州今岁灯事颇盛,在此亦不寂寞,何不以正月回去?}
贼中少年喜观灯,且得游于娼家,参随复借贷银钱,诸贼皆欣然忘归。

至元旦,随班入贺行礼。下午,仲容复入辞,先生曰:\quota{汝谒正,尚未犒赏,奈何就去?
初二日本院尚未得暇,初三日当有薄犒。}
次日,令有司送酒于寺馆,参随官携妓女陪侍,众贼欢饮竟日。
预悬牌于辕门,牌上写道:
浰头新民池仲容等,次日齐赴军门领赏,照依花名次序,不许搀前哗乱。
领赏过,三叩头即出,齐赴兵备道叩谢,事毕径回,不必又辞。
本院参随官抄写牌面,与众贼看了,无不欢喜。
是夜,先生密谕守备郏文,令拨经战甲士六百人,分作二十队,伏于射圃。
候本院犒赏贼酋,每五名一班,鼓吹送出院门,过射圃,则以甲士一队擒而杀之。
大约六人制一人,度无不胜。事了之后,只用一人在龙县丞处回话。
龙县丞者名光,原是正途出身,为吉安县丞。因不善逢迎,上司不喜,要赶逐他。
太守伍文定察其人可用,言其冤于先生,留作参随。
先生又召龙光,分付:\quota{汝可引甲士一队,妆做衙门公役,各藏暗器,立于大门照墙之下。
如贼党中有强力难制者,你令手下甲士上前相帮。
若了事时,你便遥立屏墙,使我望见,以慰我心。倘有他变,趋入报我。}
又分付有司预备花红、羊豕、坛酒、历日、银两之类。院内军将随常排列,自有规矩。
亦密谕中军官:\quota{只等本院号令,一齐下手。}

至初三日侵早,军门上已吹打过二次,各官俱集。
池仲容引着九十三人,都穿着军门颁赐长衣油靴,整整齐齐,来至院前,见巡捕官在院门上结彩,
问其缘故,答道:\quota{今日老爷犒赏新民,乃是地方吉庆之事,如何不挂彩?}
须臾,屠户牵许多猪羊来到。参随指与仲容道:\quota{这都是你们的赏物。}众贼预先欢喜。
须臾,三通吹打,放铳开门,文武属官进院作揖。仲容等亦随入叩头。
礼毕,先生先唤池仲容到前,说:\quota{你自头目,倡率归顺,与众不同。}
将案上大葵花银杯赐酒三大杯,草花一对,红绢二段缠身,犒银三两,大馍馍一盘,羊肉豕肉各五斤,酒二坛。
分付:\quota{你且站在一边,看本院赏完众人,拨门上家丁一名,送你归寺。}仲容复叩头称谢。
此时天门、二门两班乐人大吹大擂,阶下屠户杀猪宰羊,论斤分剁,好不热闹。
仲容双花双红,立于泊水檐下,何等荣耀!便似新得了科第一般,不胜之喜。
众贼候赏的一个个伸头舒颈,在阶下专听唱名。先生将花名手本付与中军,
分付道:\quota{依次唱名,每五名做一班,鼓乐导出。也教百姓看见,晓得从顺的好处,四方传说。}
中军官领诺,手执手本,高唱某某,众贼答应。每五名做一字脆着,每名草花一对,红布一匹,
都是中军官与他插缠,亦各赐热酒二杯,犒赏银一两,大馍馍十枚,羊肉豕肉各一斤,酒一小坛。
贼人要将馍馍银封置于袖中,中军官道:\quota{你若藏了,不见督府老爷的恩典。
须是放在外面,教众百姓们大家观看。}乃教他将衣兜子兜起馍馍,右手抱着酒坛,手中就捻着银封,
左手提着猪羊肉,东脚门进,西脚门出。
刚到射圃前,那三十名甲士先在那里挨次伺候,六人伏侍一个,已自众寡不敌,
况且没心人对了有心人,双手又拿着许多赏物,身上穿着长衣,又被红布缠住,脚下油靴底滑,许多不方便。
虽有强悍有本事的,也减了数分。不消得十分费力,便都了当,就将五个银封缴到龙县丞处为信。
这里杀人,里面热闹之际,那得知道?一五一十,只管送将出来。
龙县丞在屏墙下,数过第十七队,已了过八十五人矣,算道:\quota{院内连池仲容只有九人,不足为虑。}
乃走入院门,意欲回复。先生遥见龙光走进,疑外厢有变,注目视之,
见龙光行步甚缓,知其无他,心下方才安稳。龙县丞步至堂,取茶一瓯,送至先生案前,密禀曰:\quota{都了却!}
先生以头麾去。中军官又唤五名,已跪下领赏。
先生曰:\quota{汝等俱是少年后辈,前日何得与年长者争赏?须绑出捆打二十,以示教诲。}
因指未赏者三人曰:\quota{汝亦是争赏者,且只教诲你八个人。}
中军官及两班勇士一齐上前绑缚。池仲容色变,肚中如七八个吊桶,一上一落,好不安稳。
一时在他矮檐下,怎敢不低头?先生见各贼绑完,唤池仲容到前,说:\quota{汝虽投顺,去后难保其心。}
仲容方欲启口分辨,先生喝声:\quota{中军官,也与我绑着!}
就于袖中出卢珂等首状,当面逐款质问:\quota{伪檄上金龙霸王印信,从何而来?}
仲容顿口无言,惟有叩头请死。
先生命押付辕门,同八人斩首号令。仲容到辕门之外,方知领赏众贼俱已杀完,悔之无及,瞑目受刑。正是:

\startverse[leftalign=yes,sample={人恶人怕天不怕,人善人欺天不欺。}]%
人恶人怕天不怕,人善人欺天不欺。\\
善恶到头终有报,只争来早与来迟。
\stopverse

先生用计,不动声色,除了积年的反贼,满城官吏士民无不称快。
犒贼之物,一毫不失,即以赏有功甲士。狱中放出卢珂、郑志高、陈英,厚加赏赐,不在话下。
时日已过午,先生退堂,一个头旋,昏倒在地。左右慌忙扶起,呕吐不止。众官俱至私衙问安。
先生曰:\quota{连日积劳所致,非他病也。幸食薄粥,稍静坐片时,安然如故矣。}\buffernote[3-25]

是夜,先生发檄催各路兵,期定本月初七日,于三浰相会,一同捣巢。那几路?

\startquotation
从广东惠州府龙川县入者,共三路:
知府陈祥,兵从和平都入;指挥姚玺,兵从乌虎镇入;千户孟俊,兵从平地水入。

从江西赣州府龙南县入者,共四路:指挥余恩,兵从高沙堡入;推官危寿,兵从南平入;
                           知府邢珣,兵从太平堡入;指挥郏文,兵从冷水径入。

从赣州府信丰县入者,共二路:知府季斅,兵从黄田冈入;县丞舒富,兵从乌径入。

先生自率帐下官兵,从龙南冷水径直捣下浰大巢。\buffernote[3-26]
\stopquotation

却说巢中诸贼,先前得池仲容书信,说赣州兵俱已散归,督府待之甚厚,不日诛卢珂等。
传去各巢,人人信以为真,各自安居,不做准备。初闻官兵四路并进,怪仲容无信到,尚不以为然。
比及打听得实,官兵已至龙子岭,去贼巢甚近了,一时惊惶失措,乃悉其精锐,据险设伏,并势迎敌。
官军聚为三冲,犄角而前。指挥余恩兵首先遇贼,百长王受奋勇前进,与贼大战,约莫三十余合,贼兵稍却。
王受追赶里许,贼伏四起,将王受围困垓心,左冲右突,不能出去。
忽闻东角头鼓噪之声,一队官军杀将入来,乃是惠州府推官危寿部下义官叶芳也。
伏兵见有接应,正欲分兵迎敌,千户孟俊兵又从冈后杀到,横冲贼伏,与王受合兵。
三路军马同时剿杀,呼声震天,贼大奔溃。官军乘胜逐北,三浰大巢俱不能守。
各路兵闻大巢已破,心胆益壮,各自奋勇立功,连破五花障、白沙、赤唐等巢穴十一处,斩级无数。
其夜,败贼复奔铁石障、尺八岭等巢穴。
次早,先生传令各哨官兵:探贼所往,分投急击。

初九日,知府陈祥破铁石障巢,斩池仲宁,获金龙霸王伪印及违禁旗炮各物。
于是,复克羊角山等巢穴二十三处,擒斩更多。
各巢奔散之贼,其精悍者尚有八百多人,高飞甲等率之,复哨聚于九连山。
那九连山高有百仞,横亘数百余里,俱是顽石卓立,四面抖绝,止东南崖壁之下一条线路可通。
贼又将木石堆积崖上,只等我兵到时,发石滚木,百无一全。
先生传选精锐七百人,将所获贼人号衣穿着,假作奔溃之贼,乘夜直冲崖下涧道而过。
贼认做各巢败散之党,于崖上招呼。我兵亦佯与呼应,贼遂不疑。我兵已度险,遂扼断其后路。
次日黎明,我兵放起炮来,贼方知是官军并势来攻。我兵所据,反在贼崖上面。从上击下,贼不能支,遂退。
高飞甲与池仲安商议,分队潜遁。先生预令各哨官兵四路埋伏,贼遇伏辄败,又杀五百余人。
池仲安中箭而死。高飞甲率残党三百余人,分逃上下坪、黄田坳等处。
各哨官兵复约会搜捕,见贼便杀,高飞甲亦为守备郏文所斩。
有名贼徒剿灭殆尽,惟张仲全等二百余人,聚于九连谷口,呼号痛哭,
自言:\quota{本是龙川良民,被池仲容等迫胁在此,与他搬运木石。
只因贪恋残生,受其驱役,并不曾见阵厮杀,求开生路。}

\startplacefigure[location={middle,top},title={回军龙南诗: 百里妖氛一战清,万峰雷雨洗回兵。未能干羽苗顽格,深愧壶浆父老迎。莫倚谋攻为上策,须还内治是先声。功微不愿封侯赏,但乞蠲输绝横征。甲马新从鸟道回,览奇还更陟崔嵬。寇平渐喜流移复,春暖兼欣农务开。两窦高明行日月,九关深黑闭风雷。投簮最好支茅地,恋土犹怀旧钓台。洞府人寰此最佳,当年空自费青鞋。麾幢旖旎悬仙仗,台殿高低接纬阶。天巧固应非斧凿,化工无乃太安排?欲将点瑟携童冠,就揽春云结小斋。阳明山人旧有居,此地阳明景不如。但在乾坤皆逆旅,曾留信宿即吾庐。行窝巳许人先号,别洞何妨我借书。他日巾车还旧隐,应怀兹土复乡闾。春山随处款归程,古洞幽虚道意生。涧壑风泉时远近,石门萝月自分明。林僧住久炊遗火,野老忘机罢席争。习静未缘成久歇,却惭尘土逐虚名。回军龙南道中,小憇玉石岩,用韵书此,阳明山人王守仁伯安识。}]%
  \externalfigure[imgs/王阳明图传/image00136.jpeg][normal-img-60][width=.6\textwidth]
\stopplacefigure

先生遣报效生员黄表往验,果然俱是老弱,且从贼未久,其情可怜。
乃使赣州邢知府往抚其众,籍其名数,安插于白沙地方,复为良民。\buffernote[3-27]
此番用兵,自正月初七日起,至三月初八日止,通计两月内:

\startquotation
捣过巢穴三十八处,\\
斩大贼首二十九名颗,次贼首三十八名颗,从贼二千零六名颗,\\
俘获贼属男妇八百九十名口,夺获牛马一百二十二只匹,器械赃仗二千八百七十件,赃银七十两六钱六分。
\stopquotation

先生上疏奏捷,请于和平峒添设县治,以扼三省之冲。得旨,准添设,名和平县。\buffernote[3-28]
升先生都察院右副都御史,荫一子锦衣卫,世袭千户。辞免,不允。时正德十三年也。\buffernote[3-29]

\startplacefigure[location={middle,top},title={十四年己卯,先生至虔台,作《三箴》自儆。\buffernote[3-33]干戈倥偬中,日出射圃切磋歌诗习射,若事无。\buffernote[3-34]门人王思、李中、邹守益、郭持平、杨凤、杨鸾、梁焯及冀元亨等偕至军中。致书杨士德、薛尚谦曰:\quota{破山中贼易,破心中贼难。区区剪除草窃,何足为异?诸君扫荡心腹之寇,以收廓清平定,此诚大丈夫不世伟绩!}\buffernote[3-35]}]%
  \externalfigure[imgs/王阳明图传/image00137.jpeg][normal-img-60][width=.6\textwidth]
\stopplacefigure

诸贼既平,地方安靖,乃得专意于讲学。
大修濂溪书院,将《古本大学》《朱子晩年定论》付梓,凡听教者悉赠之。\buffernote[3-30]
时门人徐爱亦举进士,刻先生平昔问答行于世,命曰《传习录》。\buffernote[3-31]
海内读其书,无不想慕其人也。江西名士邹守益等执贽门下,\buffernote[3-32]
生徒甚盛。先生尝论三教同异,曰:

\startquotation
仙家说到虚,圣人岂能于虚上加一毫实?佛家说到无,圣人岂能于无上加一毫有?
但仙家说虚,从养生来;佛家说无,从出离生死苦海来,却于本体上加却这些子意在。
良知之虚,便是天之太虚;良知之无,便是太虚之无形。
日月风雷、山川民物,凡有象貌形色,皆在太虚无形中发用流行,未尝为天障碍。
圣人只是顺其良知之发用,天地万物皆在于我。\buffernote[3-36]
\stopquotation

正是:

\startverse[leftalign=yes,sample={道在将兴逢圣世,文当未丧出明师。}]%
道在将兴逢圣世,文当未丧出明师。\\
人人有个良知体,不遇先生总不知。
\stopverse

\stopsectionlevel

\startsectionlevel[default][title={靖宁藩乱}]%

话分两头。却说江西南昌府宗藩宁王,乃是太祖高皇帝第十七子,名权,初封大宁,因号宁王。
高皇帝诸子中,只有燕王善战,宁王善谋,故封于北边,以捍御北虏。
后燕王将起兵靖难,以大宁降胡所聚,以计劫宁王,与之同事,富贵共之。
后燕王既登大宝,改元永乐,是为成祖文皇帝。以大宁故地,置朵颜三卫,欲封宁王于川广。
宁王自择苏、杭二处请封,文皇帝不许。宁王大恚,遂出飞旗,令有司治驰道。
文皇怒,宁王不自安,屏去从人,独携老监数人,自南京竟走至江西省城,称病,卧于城楼之上。
布按三司奏闻,文皇帝不得已,以南昌封之,仍号宁王。数传至于臞仙,修真好道,礼贤下士,号为贤藩。

臞仙传惠王,惠王传靖王,靖王传康王。康王中年无子,悦院妓冯针儿,留侍宫中,呼为冯娘娘。
针儿有娠,康王梦蟒蛇一条,飞入宫中,将一宫之人登时啖尽,又张口来啮康王。
康王大叫一声,猛然惊醒,侍儿报冯娘娘已生世子矣。康王恶其不祥,命勿留养,遂匿于优人秦荣之家。
既长,归宫,康王心终不喜,临薨时,不令入诀。\buffernote[4-1]
濠性聪慧,通诗史,善为歌词,然轻佻无威仪,喜兵嗜利。既袭位,愈益骄横。
术士李自然言其有天子骨相,渐有异志。辇金于都下,先结交内侍李广。
正德初,又结交刘瑾等八党,为之延誉。又贿买诸生,举其孝行,朝廷赐玺书褒奖。
又谋广其府基,故意于近处放火延烧,假意救灭,拆毁其房,然后抑价以买其地。
又置庄于赵家园地方,多侵民业,民不能堪。每收租时,立寨聚众相守。
又畜养大盗胡十三、凌十一、闵廿四等,于鄱阳湖中劫掠客商货物,预蓄军资。\buffernote[4-2]
先是,胡世宁为江西兵备副使,洞察其恶,乃上疏奏闻,语甚激切。
宸濠亦奏世宁离间骨肉,辇金遍赂用事太监及当道大臣。
都察院副都御史丛兰尤与濠密,反劾世宁狂率,拿送锦衣卫,谪戍沈阳,于是宸濠得志。
凡仕江右者,俱厚其交际之礼,朝中权贵无不结交。
又这人于各处访求名士,聘为门客。锦衣千户朱宁者,小名福宁儿,云南李巡简家生子也。
太监钱能镇守云南,因以为养子,名钱宁。
因刘瑾得引见武宗皇帝,伏侍踢毬,以柔佞得幸,赐姓朱,冒功拜官。
宁转荐伶人臧贤,亦得宠。二人招权纳贿,家累巨万,宸濠俱结为心腹。
武宗皇帝屡幸臧贤之家,贤于家中造成复壁,外为木橱,橱门用锁,门内潜通密室。
每每驾到,预藏宁府使者于复壁中窃听,一言一动,无不悉知。

安福县举人刘养正,字子吉,幼举神童,既中举不第,不复会试,制隐士服,以诗文自高。
三司抚按折节其门,以得见为幸。濠以厚币招致,岁时馈问不绝,遂与濠昵。
李士实由翰林官至侍郎致仕,与濠为儿女亲家。士实颇有权术,以姜子牙、诸葛孔明自负,濠用为谋主。
又以丞奉刘吉、术士李自然、徐卿等,党与甚众。因武宗皇帝无子,濠谋以其子二哥为皇嗣。
朱宁、臧贤与诸大阉力任其事。朝中六部九卿、科道官员亦多有为之左右者,因其事重大,未敢发言。
李士实为濠谋,通于兵部尚书陆完,题复宁府护卫,一面使南京镇守太监毕真倡率南边官员人等,保举宁王孝行。
及陆完改吏部,王琼代为兵部尚书。
琼策濠必反,谓陆完曰:\quota{祖宗革去护卫,所以杜藩王不轨之谋,正是保全他处。
宁王再三要复护卫,不知他要兵马何用?异日恐有他变,必累及公矣。}
陆完大悔,写书于濠,欲其自以己意缴还护卫。濠不从,借护卫为名,公然招募勇健,朝夕在府中使枪弄棒。
先生闻濠反谋,乃因其贺节之礼,使门人冀元亨往谢。元亨字惟乾,钱塘举人,为人忠信可托。
先生聘为公子正宪之师,故特遣行,使探听宁王举动。
却说宸濠有意结交先生,闻元亨是先生门人,甚加礼貌,渐渐言及于外事。
元亨佯为不知,与谈致知格物之学,欲以开导宁王,止其邪心。
濠大笑曰:\quota{人痴乃至此耶!}立与绝。元亨归赣,述于先生。
先生曰:\quota{汝祸在此矣。汝留此,宁王必并媒孽\buffernote[4-3]及我。}
遂遣人卫之归家。\buffernote[4-4]
\topnote{牛头不对马嘴,亦见元亨迂腐一斑。若阳明处此,必别有一番化导开发处。}

再说宁府典宝阎顺、内官陈宣、刘良,见濠所为不法,私诣京师出首。朱宁与陆完隐其事,使人报濠。
濠疑承奉周仪所使,假装强盗,尽杀其家,又杀典仗查武等数百人。复辇金京师,遍赂权要,求杀阎顺等。
顺等亡命远方,乃免。于是,逆谋益急。宁王之妃娄氏,素有贤德。\buffernote[4-5]
生下三子,大哥、三哥、四哥,宁王最敬重之。
娄妃察宸濠有不轨之志,乃于饮宴中间,使歌姬进歌劝酒,欲以讽之,曲名《梧叶儿》,云:

\startverse[leftalign=yes,sample={一霎时转眼故人稀,渐渐的朱颜易改,}]%
争甚么名和利?问甚么咱共伊?\\
一霎时转眼故人稀,渐渐的朱颜易改,\\
看看的白发来催,提起时好伤悲。\\
赤紧的可堪,当不住白驹过隙。
\stopverse

宸濠听此词,有不悦之色。
娄妃问曰:\quota{殿下对酒不乐,何也?}
宸濠曰:\quota{我之心事,非汝女流所知。}
娄妃陪脸笑曰:\quota{殿下贵为亲王,锦衣玉食,享用非常。若循理奉法,永为国家保障,世世不失富贵。
此外,更有何心事?}
宸濠带了三分酒意,叹口气道:\quota{汝但知小享用之乐,岂知有大享用之乐哉?}
娄妃曰:\quota{愿闻如何是大享用、小享用。}
宸濠曰:\quota{大享用者,身登九五之尊,治临天下,玉食万方。
吾今位不过藩王,治不过数郡,此不过小享用而已,岂足满吾之愿哉?}
娄妃曰:\quota{殿下差矣!天子总揽万几,晏眠早起,劳心焦思,内忧百姓之失所,外愁四夷之未服。
至于藩王,衣冠宫室,车马仪仗,亚于天子,有丰享之奉,无政事之责,是殿下之乐过于天子也。
殿下受藩镇之封,更思越位之乐。窃恐志大谋疏,求福得祸,那时悔之晩矣!}
宸濠勃然变色,掷杯于地而起。有诗为证:

\startverse[leftalign=yes,sample={造谋越位费心机,逆耳忠言苦执迷。}]%
造谋越位费心机,逆耳忠言苦执迷。\\
天位岂容侥幸取?一朝势败悔时迟。
\stopverse

娄妃复戒其弟娄伯将,勿从王为逆。伯将亦不听。
宸濠起造阳春书院,僭号离宫,用酖酒毒死巡抚王哲,守臣无不悚惧。
讽有司参谒,俱用朝服。各官惧其势焰,亦多从之。
时鄱阳湖中屡屡失盗,尽知是宁府窃养,吞声莫诉。娄妃屡谏,不听。
兵部尚书王琼预忧其变,督责各抚臣,训兵修备。又以承奉周仪等之死,责江西抚臣严捕盗贼。
南昌府获盗一伙,内有凌十一。
有人认得是宁府中亲信之人,抚台孙燧密闻于王琼,宸濠使其党于狱中强劫以去。
叛谋益急,约定八月乡试时,百官皆进科场,然后举兵。
王琼闻凌十一被劫,怒曰:\quota{有此贼,正好做宁府反叛证见,如何容他劫去了?}责令有司,立限缉获。
濠恐事泄,复讽南昌诸生,颂己贤孝,迫挟抚按具奏,为之解释。
按察副使许逵劝发兵围宁府,搜获劫盗,若拿出一二人,究出谋叛之情,请旨迫夺,免得养成其患。
燧犹豫不决,被濠屡次催促,巡抚孙燧不得已,随众署名,乃别奏濠不法事,列款有据。
濠亦虑及此,预布心腹勇健,假装响马于北京一路,但有江西章奏,尽行劫去。
燧七次奏本,都被拦截,不得上闻,止有保举孝行的表章。濠使心腹林华同赉上京,直达天聪。
时江彬新得宠幸,冒功封平虏伯。太监张忠与朱宁有隙,遂附江彬,每欲发宁王之事,以倾朱宁,未得其便。
及保奏表至,武宗皇帝问于张忠曰:\quota{保官好升他官职,保亲王意欲何为?}
忠对曰:\quota{王上更无进步,其意未可测也。}

先是,宸濠结交臧贤,伪使伶人秦荣就学音乐,谢以万金及金丝宝壶一把。
忽一日,武宗皇帝驾幸臧贤家,贤注酒献上。
武宗皇帝见壶,惊曰:\quota{此壶光泽巧丽,我宫中亦无此好物,汝何从得此?}
臧贤恃上之爱宠,且欲表宸濠之情,遂以实对曰:\quota{不敢隐瞒,赖万岁洪福,此乃宁殿下所赐也。}
武宗皇帝曰:\quota{宁叔有此好物,何不献我,乃赐汝耶?}
其时,优人中有小刘者,亦新得宠,独未得濠贿赂,心中怏怏。及大驾回宫,又夸金壶之美。
小刘笑曰:\quota{宁殿下不思爷爷物足矣,爷爷尚思宁殿下乎?昨保举贤孝,爷爷岂遂忘之?
今朱宁、臧贤日夕与宁府交通,所得宝货无算,藏纳奸细于京中,不计其数。外人无不知,独爷爷不知耳。}
武宗皇帝遂疑臧贤。有旨,遣太监萧疏搜索贤家。又降旨:各藩使人,无事不许擅留京师。
试御史萧淮遂直攻宁王,并参李士实、毕真等。给事中徐之鸾、御史沈灼等连章复上,朝廷准奏。
念亲亲之情,不忍加兵,遣驸马都尉崔元、都御史颜颐寿及太监赖义往谕,革其护卫。
宁府心腹林华先在复壁中,听知金壶之语,用心打探。
及闻京师挨缉奸细,又有诏使遣至江西,遂于会同馆取快马,昼夜奔驰,在路才十八日,便至南昌。

其日乃是六月十三日,正宸濠诞辰,诸司入贺,濠张宴款待。林华候至席散,方才禀奏。
濠谓李士实、刘养正等曰:\quota{凡抄解宫眷,始用驸马亲臣。今诏使远来,事可疑矣。
若待科场之事,恐诏使先到,便难措手,今当如何?}
养正曰:\quota{事急矣!明旦,诸司谢酒,便当以兵威胁之。}
士实曰:\quota{须是假传太后密旨,如此恁般,方好。}
商量停当,时闵廿四、凌十一、吴十三等,亦以贺寿毕集。夜传密信,令各饬兵伺候。
及旦,诸司入谢,礼毕。濠出坐,立于露台之上,
诈言于众曰:\quota{昔孝宗皇帝为太监李广所误,抱养民间子。我祖宗不血食者,今十四年矣。
太后有密旨,命寡人发兵讨罪,共伸大义,汝等知否?}
巡抚孙燧挺身出曰:\quota{既然太后有旨,请出观之。}
濠大声曰:\quota{不必多言!我今往南京去,汝愿保驾否?}
燧曰:\quota{天无二日,民无二王,这才是大义,此外非某所知。}
濠戟手怒曰:\quota{汝既举保我孝行,如何又私遣人诬奏我谋为不轨?如是反覆,岂知大义?}叱左右:\quota{与我绑了!}
按察副使许逵从下大呼曰:\quota{孙都御史乃钦差大臣,汝反贼敢擅杀耶?}濠怒,喝令并缚之。
逵顾燧曰:\quota{我欲先发,公不听我言。今果受制于人,尚何言哉?}
因大骂:\quota{宸濠逆贼,今日汝杀我等,天兵一到,你全家受戮,只在早晩!}
濠令较尉火信拽出于惠民门,斩首示众。比及娄妃闻信,急使内侍传救,已无及矣。
阳明先生有《哭孙许二公诗》二首,其一云:

\startverse[leftalign=yes,sample={丢下乌纱做一场,男儿谁敢堕纲常。}]%
丢下乌纱做一场,男儿谁敢堕纲常。\\
肯将言语阶前屈,硬着肩头剑下亡。\\
万古朝端名姓重,千年地里骨头言。\\
史官谩把《春秋》笔,好好生生断几行。
\stopverse

其二云:

\startverse[leftalign=yes,sample={天翻地覆片时间,取义成仁死不难。}]%
天翻地覆片时间,取义成仁死不难。\\
苏武坚持西汉节,天祥不受大元官。\\
忠心贯日三台见,心血凝冰六月寒。\\
卖国欺君李士实,九泉相见有何颜?
\stopverse

时佥事潘鹏,自为御史时,先受宁王贿赂,与之交通。至是,率先叩头,呼万岁。
参政王伦、季敩\low{\hw 敩为南安知府,从先生平贼有功,升参政。}惧祸,亦相继拜伏。
布政使梁宸、按察使杨璋、副使唐锦、都指挥马骥,各各以目相视,不敢出声。
濠大喝曰:\quota{顺我者生,逆我者死!}四人不觉屈膝。
镇守太监王宏、巡按御史王金、奉差主事马思聪、金山、布政使胡濂、
参政程杲、刘斐、参议许效廉、黄宏、佥事赖凤、佥书郏文、\low{\hw 以指挥,从先生征贼有功,升今任。}
都指挥许清、白昂初皆不屈。濠令系狱三日,俟其改口愿附,方释之。
惟马思聪与黄宏终不肯服,不食而死,真忠臣也。
濠即日伪置官属,以吉暨、涂钦、万锐等为御前太监,尊李士实为太师,
刘养正为国师,刘吉为监军都御史,参政王纶授兵部尚书,季敩等各加伪职,
大盗闵廿四、吴十三、凌十一等俱授都指挥等官。
南昌知府郑瓛、知县陈大道俱愿降,复职管事如故。
其时有瑞州知府,姓王,名以方,湖广黔阳人,素知宸濠必叛,练卒葺城,为守御计。
宸濠慕其才能,屡次遣人送礼,欲招致之,以方拒而不受。
至是,适有公事到于省城,逆党擒送宁府。宸濠命降,以方不从,系之于狱。
宸濠又传檄远近,革去正德年号,拟改\quota{顺德}二字。只待南京正位,即便改元。又造伪檄,指斥乘舆,极其丑诋。
时濠畜养死士二万,招诱四方盗贼渠魁四万余,又分遣心腹娄伯将、王春等肆出收兵,
合护卫党与并胁从之人,共六七万余人,军势甚盛。

又用江西布政司印信公文,差人遍行天下布政司,告谕亲王三司等官举兵之意,一面修理战具。
此一场,闹动了江西省城百姓。后人有诗叹云:

\startverse[leftalign=yes,sample={宁藩妄想动兵戎,枉使机关指日穷。}]%
宁藩妄想动兵戎,枉使机关指日穷。\\
可叹古今兴废迹,鄱阳湖水血流红。
\stopverse

是时,福州三卫军人进贵等聚众鼓噪,朝廷命阳明先生往勘。
先生以六月初九日启行,亦要赶十三日与宁王拜寿,此乃是常规。
临发时,参随官龙光等取敕印作一扛,留于后堂。轿出,仓卒封门,忘其所以。
行至吉安,先生登崖取敕印,方省不曾带来,乃发中军官,转回赣州取扛,以此沿途迟留,待扛至方行。
六月十四日午后,刚刚行至丰城,此正孙都堂、许副使遇害之日也。
若非忘记敕印,迟此数日,亦在入谢班中,同与孙、许之难矣,岂非天乎?
正是万般皆是命,果然半点不由人。

却说丰城县离省城仅一百二十里,宁王杀害守臣不过半日,便有报到丰城了。
知县顾佖谒见先生,将省中之事禀知,兼述所传闻之语:\quota{宁府已发兵千余,邀取王都堂,未知果否?}
先生分付顾佖:\quota{你自去保守地方。那宁王反情,京师久已知道,不日大兵将至。
可安慰百姓,不必忧虑,本院亦即日起兵来矣。}顾佖辞去。
先生急召龙光,问曰:\quota{闻顾知县语否?}光对曰:\quota{未闻。}
先生曰:\quota{宁王反矣。}龙光惊得目睁口呆。
先生曰:\quota{事已至此,惟走为上策。自此西可入瑞州,到彼传檄起兵讨贼,别无他策。}
分付管船的快快转船,连夜行去。
艄子听说反了宁王,心胆俱裂,意不愿行,
来禀道:\quota{来时顺风顺水,今转去是上水,又是大南风,甚逆,难以移动。
便要行,且待来早看风色如何。}
先生命取瓣香,亲至船头焚香,望北再拜,曰:\quota{皇天若哀悯生灵,许王守仁匡扶社稷,愿即反风。
若天心助逆,生民合遭涂炭,守仁愿先溺水中,不望余生矣。}
言与泪下,从者俱感动。祝罢,南风渐息。须臾,樯竿上小旗飘扬,已转北风。
艄子又推天晩,不行。先生大怒,拔剑欲斩之,众参随跪劝,乃割其一耳。
于是张帆而上。行不上二十里,日已西沉。
先生见船大行迟,使参随潜觅渔舟。先生微服过舟,惟龙光、雷济相从,止带敕印随身,其衣冠仪仗并留大船。
分付参随萧禹在内,随后而至。渔舟惯在波浪出入,拽起蓬来,梭子般去了。\buffernote[4-1-0]
却说宸濠打听南赣军门起马牌,是六月初六日发的。旧规三日前发牌,算定初九日准行,如何还不见到?
难道径偷过了,或者半途晓得风声,走转去了,也不可知。此人是经济之才,若得他相助,大事可就。
遂分付内官喻才,以划船数十只追之,行至地名黄土脑,\low{\hw 属丰城县。}已及大船,拿住萧禹。
禹曰:\quota{王都爷已去久矣,拿我何益?}喻才乃取其衣冠,回复宁王去了。正是:
鳌鱼脱却金钩去,摆尾揺头再不来。

\startplacefigure[location={middle,top},title={先生奉命查处福建乱军,以六初九发赣州,十五日至丰城黄土脑。知县顾佖遣史走报:宁府宸濠杀都御史孙燧、副使许逵,并缚各官,拘印信,放囚劫库,直取南京北上。先生吁于天,誓死讨贼。祷北风,风果北,觅渔舟以遁。濠兵追之不及。}]%
  \externalfigure[imgs/王阳明图传/image00138.jpeg][normal-img-60][width=.6\textwidth]
\stopplacefigure

先生乘渔舟,径至临江,有司俱不知。先生使龙光登崖,索取轿伞。
临江知府戴德孺急来迎接,款留先生入城调度。
先生曰:\quota{临江,大江之滨,与省城相近,且居道路之冲,不可居也。}
德孺曰:\quota{闻宁王兵势甚盛,何以御之?}
先生曰:\quota{濠出上策,乘其方锐之气,出其不意,直趋京师,则宗社危矣。
若出中策,则径攻南京,大江南北,亦被其害。
但据江西省城,则勤王之师四集,鱼游釜中,不死何为?此下策矣。}
德孺曰:\quota{以老大人明见度之,当出何策?}
先生曰:\quota{宁王未经战阵,中情必怯。若伪为兵部恣文:\quota{发兵攻南昌。}
彼必居守,不敢远出。旬日之间,王师四集,破之必矣!}\topnote{便已算定宁王了,办贼何有?}
德孺请先生更船,先生辞之,止取黄伞以行。至新淦,于船中张伞。
知县李美有将才,素练士卒,有精兵千余。至是,来迎先生,固请登城。
先生曰:\quota{汝意甚善,然弹丸之地,不堪用武。}
李美具站船,先生始更舟,先后共行四昼夜,方至吉安。
知府伍文定闻先生至,大喜,急来谒见。先生欲暂回南、赣征兵。
伍文定曰:\quota{本府兵粮俱已勉力措置,亦须老大人发号施令,不必又回,稽误时日。}
\topnote{绝妙一个大帮手。若暂回南赣,局面又当一变矣。}

先生乃驻扎吉安。上疏告宁府之变,请命将出师,以解东西倒悬之苦。
并请留两广差满御史谢源、任希儒军前纪功。
一面请致仕卿官王懋中等,与知府伍文定及门人卿官邹守益等一同商议,
遵便宜之制,传檄四方,暴濠之罪状,征各郡兵勤王。
又遣龙光于安福,取刘养正家小至吉安城中,厚其供给,遗书养正,以疑宁贼之心。
又访着李士实家属,谬托腹心,语之曰:\quota{吾只应敕旨聚兵为名而已,宁王事成败未卜,吾安得遽与为敌乎?}
\topnote{王公精于用间。}
又令参随雷济,假作南、赣打来报单,内开报兵部准令:
许泰、郄永,分领边军四万,从凤阳;
刘晖、桂勇,分领京边官军四万,从徐、淮,水陆并进;
王守仁,领兵二万,杨旦等,领兵八万,
陈金等,领兵六万,分道夹攻南昌。
原奉机密敕旨,各军缓缓而行,只等宸濠出城,前后遮击,务在必获。\buffernote[4-6]
又伪作两广机密火牌,内云:
都御史颜咨奉兵部咨,率领狼达官兵四十八万,前往江西公干。
先生又自作文书,各处投递,说:
各路军马,俱于南昌取齐。本省各府县速调集军马,刻期接应。
又于丰城县张疑兵,作为接济官兵之状。
又取新洤优人十余名,各将约会公文一角,
并抄报,卑火牌缝于衣袂之中,厚赐路费,纵之南行,被宁府伏路小军所获,解至王府。
原来李士实、刘养正等果劝宸濠由蕲黄直趋北京,不然,亦须先据南京。根本既定,方可号召天下。
宸濠初意欲听其谋,因搜优人身伴,见了督府公文,
以为王师大集,旦暮且至,遂不敢出城,但多备滚木礌石,为守城之计。\topnote{不出王公所料。}
李士实复言于宸濠曰:\quota{朝廷方遣驸马,安得遽发边兵?此必守仁缓兵之计也。
王负反叛之名,不务风驰雷击,而困守一隅,徐待四方兵集,必无幸矣。
宜分兵一支,打九江府,若得此郡,内有二卫军,足可调用。再分兵一支,打南康府。
殿下亲率大军直趋南京,先即大位。天下之贪富贵者翕然来归,大业指日可定也。}
宸濠意尚犹豫,\buffernote[4-7]
一面打探官军消息,一面先遣闵廿四、吴十三等,各帅万人,夺官民船装载,顺流去打南康。
知府陈霖遁走,城遂陷。进攻九江府,知府汪颖、知县何士凤及兵备副使曹雷亦遁,九江百姓开门以纳贼兵。
闵廿四、吴十三分兵屯守,飞报捷音。
宸濠大喜曰:\quota{出兵才数日,连得二郡,又添许多钱粮军马,吾事必成矣!}
遂遣贼将徐九宁守九江,陈贤守南康,俱冒伪太守之号。闵廿四、吴十三撤回,随大军征进。
因遣使四出,招谕府属各县降者,复官如故。
恰好打探官军的回报道:\quota{火牌报单都是军门假造出来的,各路军马并无消息。
王都堂安坐吉安府中,闻说已发牌属郡,约会军马,尚未见到。}
宸濠谓投降参政季敩曰:\quota{汝曾与王守仁同在军中,能为我往吉安招降守仁,汝功不浅。}
季敩不敢推托,即同南昌府学教授赵承芳及旗较等十二人,赍伪檄榜文,来谕吉安府,并说先生归顺宁王。
先生先有文移:各路领哨官把守信地,如有宁府人等经过,不拘何人,即行绑送军门勘究。
敩等行至墨潭地方,被领哨官阻住。
季敩喝曰:\quota{我乃本省参政,汝何人?敢来拦截。}领哨官曰:\quota{到此何事?}
季敩曰:\quota{有宁府檄文在此。}
旗较将檄文牌面与领哨官观看,领哨官遂将旗较拿住,季敩慌忙回船逃去。
领哨官晓得参政是个大官,不敢轻动,止将旗较五名连檄榜解至军门来。
先生问:\quota{季敩何在?}领哨官曰:\quota{已逃矣。}
先生叹曰:\quota{忠臣孝子与叛臣贼子,只在一念之间。季敩向日立功讨贼,便是忠臣;
今日奉贼驱使,便是叛臣。为舜为跖,毫厘千里,岂不可惜?}
先生欲将旗较斩首,思量恐有用他之处,乃发临江府监候。遂将伪檄具疏驰奏。略曰:

\startquotation
陛下在位一年,屡经变难,民心骚动,尚尔巡游不已,致使宗室谋动干戈。
且今天下之觊觎,岂特一宁王?天下之奸雄,岂特在宗室?言及至此,懔骨寒心。
昔汉武帝有轮台之悔,而天下向治;唐德宗下奉天之诏,而士民感泣。
伏望皇上痛自克责,易辙改弦,罢黜奸谀,以回天下豪杰之心;绝迹巡游,以杜天下奸雄之望,则太平尚可图。
臣不胜幸甚!
\stopquotation

知府伍文定请先生出兵征进,先生曰:\quota{彼气方锐,未可急攻。
必示以自守不出之形,诱其离穴,然后尾其后而图之。
先复省城,以捣其巢,彼闻,必回兵来援,我因邀而击之。兵法所谓致人而不致于人也。}
乃敛兵自守,使人打听南昌消息。

再说娄伯将回进贤家中募兵,知县刘源清捕而斩之,尽召城外巨室入城,垒其三门,誓众死守。
又贼党有船数只,为首者自称七殿下,往龙津夺运船。
驲丞孙天佑禀余干知县马津,津使率兵拒战,射杀数人,七殿下麾舟急退。
又贼党袁义官自上流募兵百余,还过龙津,亦被天佑追杀,焚其船。濠怒,将先取进贤、余干,然后东下。
李士实曰:\quota{若大事既定,彼将焉逃?}
濠乃止。于是,二府之民不尽从贼,皆二县三人之力也。
再说季敩自墨潭逃回,来见宁王,述旗较被擒之事。宸濠大怒,乃问王守仁出兵消息。
季敩惧罪,乃答曰:\quota{王守仁只可自守,安敢与殿下作敌?}
濠信之。以王师未集,乃伏兵万余,命宜春王栱樤,同其子三哥、四哥,
与伪大监万锐等,分付坚守省城,多设灰瓶、火炮、滚粪、石弩之类,又伏兵一枝于城外,以防突城。
自与娄妃及世子大哥、宗室栱栟、刘养正、李士实、杨璋、潘鹏等,择七月初二日,发兵东下。
伪封宗弟宸澅为九江王,使率百舟前导。
是早,宸濠入宫,请娄妃登舟。娄妃尚未知其意,问曰:\quota{殿下邀妾何往?}
宸濠曰:\quota{近日太后娘娘有旨,许各亲王往南京祭祖。我同汝一往,不久便回。}
娄妃半信半疑,只得随行。
濠登舟之时,设坛祭江,命斩端州知府王以方,以之代牲。
方奠牲之时,几案忽折,以方头足自跳跃覆地,宸濠命弃之于江。
舟始发,天忽变,云色如墨,疾风暴雨,雷电大作。前舟宸澅被霆震而死,濠意不乐。
\topnote{人心示儆〔如〕此,濠是〔时〕束身归罪,尚可自活。以骑虎不下,自取覆灭。愚哉!}
李士实曰:\quota{事已至此,殿下能住手否?天道难测,不足虑也。}
濠索酒痛饮,即醉卧于椅上,梦见揽镜,其头尽白如霜,猛然惊醒。
唤术士徐卿问之,卿叩首称贺曰:\quota{殿下贵为亲王,而梦头白,乃皇字也。此行取大位必矣!}
时兵众有六七万人,号为十万,尽夺官民船只装载,旌旗蔽江而下,相连六十余里,有诗为证:

\startverse[leftalign=yes,sample={杀气凄凄红日蔽,金鼓齐鸣震天地。}]%
杀气凄凄红日蔽,金鼓齐鸣震天地。
艨艟压浪鬼神惊,旌旆凌空彪虎聚。
流言管蔡似波翻,争锋楚汉如儿戏。
难将人力胜天心,一朝扫尽英雄气。
\stopverse

贼兵一路攻掠沿江各县,将及安庆,知降佥事潘鹏安庆人,先遣鹏持伪檄往安庆谕降。
太守张文锦召都指挥杨锐问计,锐曰:\quota{王都堂前有牌面来,分付紧守信地,大兵不日且至。
今潘鹏来谕降,当力拒之。}
杨锐登城楼,谓潘鹏曰:\quota{佥事乃国家宪臣,奈何为反贼奴隶传语?宁王有本事,来打安庆城便了。}
\topnote{正欲激逆濠之怒。}
潘鹏曰:\quota{汝且开城门,放我进来,有话商量。}
杨锐曰:\quota{要开门,除是逆濠自来。}遂弯弓搭箭,欲射潘鹏。潘鹏羞惭满面而退,回报宸濠。
宸濠怒曰:\quota{谅一个安庆,有甚难打?}
李士实谏曰:\quota{殿下速往南都正位,何愁安庆不下?}宸濠默然。
船过安庆城下,杨锐曰:\quota{若宁王直走南京,便成大势,当以计留之。}
乃建旗四隅,大书\quota{剿逆贼}三字,濠闻而恶之。
锐又使军士及百姓环立城头,辱骂:\quota{宸濠反贼,不日天兵到来,全家剿灭!}千反贼万反贼的骂。
\topnote{守安庆,留宁王,全得杨锐力。后王公叙功疏,独不及锐,何耶?}
宸濠在舟中听得外面喧嚷,问其缘故。
潘鹏曰:\quota{此即指挥杨锐使军民辱骂殿下。}
宸濠大怒曰:\quota{我且攻下安庆,杀了杨锐,然后往南京未迟。}
乃掠其西郭,遂围正观、集贤二门。濠乘黄舰,泊黄石矶,亲自督战。
\topnote{潘鹏欲借宁贼以报锐仇,不知反中锐计也。}
安庆城池坚固,又兼张文锦和杨锐料理已久,多积炮石及守城之器。
军卫卒不满百人,乘城者皆民兵,阖户调发,老弱妇女,亦令馈饷。
登城者必带石块一二,石积如山。又暑渴,置釜于城上,煮茶以饮之。
贼攻城,辄投石击之,或沃以沸汤,贼不敢近。
贼拥云楼,瞷城中,将乘城,城中造飞楼数十,从高射贼,贼多死。
夜复募死士缒城,焚其楼。贼又置云梯数十,广二丈,高于城外,蔽以板,前后有门,中伏兵。
城上束藁沃膏,燃其端,俟梯至,投其中,燥木着火即燎,贼多焚死。
锐又射书贼营,谕令解散。贼兵转相传语,多有逃去者。锐又募死士,夜劫其营,贼众大扰,至晓始定。
濠问篙工曰:\quota{此地何名?}对曰:\quota{黄石矶也。}
\quota{黄石矶}音声与\quota{王失机}相近。濠恶其言,拔剑斩之,谓其徒曰:\quota{一个安庆,且不能克,安望金陵哉?}
于是亲自运土填堑,期在必克。
话分两头,再说先生所差探听南昌消息的,引着安庆逃回被掳船户,
一同回报:\quota{打听得宁王于七月初二日起大兵,从水路而下,见今围住安庆城攻打,势甚危急。
其南昌守备甚固,闻说城外又有伏兵,未知何处。}
先生发放船户,重赏探子,着再去打探伏兵的实信回话。众将请救安庆。
先生曰:\quota{今九江、南康皆为贼所据,而南昌城中精悍尚且万余,食货重积。
我兵若抵安庆,贼必回军死斗。安庆之兵仅足自守,必不能援我于湖中。
南昌之兵绝我粮道,而九江、南康之贼合势挠摄,四方之援又不可望,大事去矣。
今各郡官兵渐次齐集,先声所加,城中必已震慑。因而并力以攻省城,其势必下。
既破南昌,贼先丧胆,彼欲归救根本,则安庆之围自解,而濠亦可擒矣。}
\topnote{权衡利害,如指诸掌。此孙子救韩直走魏都之计。}

\startplacefigure[location={middle,top},title={比出师,聚柴围公署。诸夫人问故。曰:\quota{我若兵败,即以焚汝。}夫人惊泣,以大义折之而行。}]%
  \externalfigure[imgs/王阳明图传/image00139.jpeg][normal-img-60][width=.6\textwidth]
\stopplacefigure

遂以本月十三日,自吉安起马,与诸将刻期于十五日齐会于临江府漳澨地方。于是,各属府县兵将并至。
初欲登台誓师,先生以积劳病发,勉强书一牌,呼知府伍文定、邢珣、徐琏、戴德孺四人授之。
牌上写云:伍不用命者斩队将,队将不用命者斩副将,副将不用命者斩主将。
先生曰:\quota{军中无戏言。此是实语,不相诳也。}
文定等皆暗暗吐舌。大军行至丰城,南昌府推官徐文英因查盘在外,独不与难。
奉新知县刘守绪皆引兵壮来会,悉留军前听用。先生病亦稍可,乃分军为十三哨,各示以进攻屯守之宜。

\startquotation
第一哨:吉安府知府伍文定,统部下官军兵快四千四百二十一员名,进攻广润门;
       就留兵防守本门,直入布政司屯兵,分兵把守王府内门。\\
第二哨:赣州府知府邢珣,统部下官军兵快三千一百三十余员名,进攻顺化门;
       就留兵防守本门,直入镇守府屯兵。\\
第三哨:袁州府知府徐琏,统部下官军兵快三千五百三十员名,进攻惠民门;
       就留兵防守本门,直入按察司察院屯兵。\\
第四哨:临江府知府戴德孺,统部下官军兵快三千六百七十五员名,进攻永和门;
       就留兵防守本门,直入都察院提学分司屯兵。\\
第五哨:瑞州府通判胡尧元、童琦,统部下官军兵快四千员名,进攻章丘门;
       就留兵防守本门,直入南昌卫前屯兵。\\
第六哨:泰和县知县李缉,统部下官军兵快一千四百九十二员名,夹攻广润门;直入王府西门屯兵。\\
第七哨:新淦县知县李美,统部下官军兵快二千员名,进攻德胜门;就留兵防守本门,直入王府东门屯兵。\\
第八哨:中军赣州卫都指挥余恩,统部下官军兵快四千六百七十员名,进攻进贤门;直入都司屯兵。\\
第九哨:宁都县知县王天与,统部下官军兵快一千余员名,夹攻进贤门;就留兵防守本门,直入钟楼下屯兵。\\
第十哨:吉安府通判谈储,统部下官军兵快一千五百七十六员名,夹攻德胜门;直入南昌左卫屯兵。\\
第十一哨:万安县知县王冕,统部下官军兵快一千二百五十七员名,夹攻进贤门;就把守本门,直入阳春书院屯兵。\\
第十二哨:吉安府推官王暐,统部下官军兵快一千余员名,夹攻顺化门;直入南、新二县儒学屯兵。\\
第十三哨:抚州府通判邹琥、傅南乔,统部下官军三千余员名,夹攻德胜门;就留兵防守本门,随于城外天宁寺屯兵。\\\buffernote[4-8]
\stopquotation

先生分拨已定,期定十九日至市汊。
二十日黎明,各至信地。临发,绑不用命者数人,斩首以狥,各军无不股慄。
不知所斩者,乃密取临江府监候赍伪檄之旗较也。先生权术不测,类如此。

再说宸濠攻打安庆十有八日,城中随机应变,并无挫折。
宸濠正在心焦,忽接得南昌告急文书,说:\quota{王都堂大军已至丰城,将及省下。城中军民震骇,乞作急分兵归援。}
宸濠大惊,便欲解围而归。李士实曰:\quota{若殿下一回,则军心离矣。}
宸濠曰:\quota{南昌我之根本,如何不救?}
刘养正亦曰:\quota{今安庆音问不通,破在旦夕。
得了安庆,以为屯止之所,然后调集南康、九江之兵,齐救省城,官军见我兵势浩大,不战而退矣。}
濠张目视曰:\quota{汝家属受王守仁供养,欲以南昌奉之耶?}二人乃不敢复言。
先生先遣探卒打探得南昌伏兵千余,在新旧坟厂地方。
乃使奉新县知县刘守绪同千户徐诚,领精兵四百,从间道袭之,出其不意。
伏兵一时溃散,齐奔南昌城来。城中骤闻王都堂兵至,杀散伏兵,人人惊骇,传相告语,俱怀畏避之意。
二十五日五更,各哨俱照依派定信地进发。先生复申明约束:
一鼓附城,再鼓登城,三鼓不克,诛其伍,四鼓不进,诛其将。
各哨统兵官知先生军令严肃,一闻鼓声,呼噪并进。伍文定兵梯縆先登,守贼军士见军势大,皆倒戈狂走。
城中喊声大振,四下鼎沸,砍开城门,各路兵俱入,遂擒宜春王栱樤及宁王之子三哥、四哥并伪太监万锐等,
共千有余人。宫眷情急,纵火自焚。可怜眷属百数,化作一阵烟灰,哀哉!
火势猛烈,延烧居民房屋。先生统大队军兵入城,传令各官,分道救火,抚慰居民。
火熄后,伍文定等都来参见,将捉到人犯押跪堂下。
先生审明发监,封其府库,搜获原收大小衙门印信九十六颗,人心始安。
于是,胁从官胡濂、\low{\hw 原布政。}刘斐、\low{\hw 原参政。}许效廉、\low{\hw 原参议。}
唐锦、\low{\hw 原副使。}赖凤\low{\hw 原佥事。}及南昌知府郑瓛,
同知何继周、通判张元澄、南昌知县陈大道、新建知县郑公奇,皆自投首,先生俱安慰之。有诗为证:

\startverse[leftalign=yes,sample={皖城方逞螳螂臂,谁料洪都巢已倾。}]%
皖城方逞螳螂臂,谁料洪都巢已倾。\\
赫赫大功成一鼓,令人千载羡文成。
\stopverse

先生又打探得宁王已解安庆之围,移兵于沅子港,先分兵二万,遣凌十一、闵廿四分率之,疾趋南昌,
自帅大军随后而进,时乃二十二日也。先生闻报,大集众将问计。
众皆曰:\quota{贼势强盛,今既有省城可守,且宜敛兵入城,坚壁观衅,俟四方援兵至,然后图之。}
先生笑曰:\quota{不然。贼势虽强,未逢大敌,惟以爵赏诱人而已。今进不得逞,退无所归,其气已消沮。
若出奇兵击其惰归,一挫其锐,将不战自溃,所谓先人有夺人之心也。}
\topnote{王公遇事,真是看得彻底,所以动必有功。}
适抚州知府陈槐、进贤知县刘源清各引兵来助战。
先生乃遣伍文定、邢珣、徐琏、戴德孺各领兵五百,分作四路并进。
又遣余恩,以兵四百,往来于鄱阳湖上,诱致贼兵。
又遣陈槐、胡尧元、童琦、谈储、王暐、徐文英、李美、李楫、王冕、王轼、刘守绪、刘源清等,
各引兵百余,四面张疑设伏,候文定等交锋,然后合击。
分布已定,乃开仓,大赈城中军民人等。又虑宗室郡王将军,或为内应生变,亲自慰谕,以安其心。出告示云:

\startquotation
督府示谕:省城七门内外军民杂役人等,除真正造逆不赦外,其原役宁府被胁伪授指挥、千百户、较尉等官,
及南昌前卫一应从乱杂色人役,家属在省城者,仰各安居乐业,毋得逃窜。
父兄子弟有能寄信本犯,迁善改过,擒获正恶,诣军门报捷者,一体论功给赏。
逃回投首者,免其本罪。其有收藏军器,许尽数送官。各宜悔过,毋取灭亡。特示。\buffernote[4-9]
\stopquotation

写下二十余通,发去城内城外居民及乡导人等,于七门内外,各处粘贴传布,以解散其党。
二十三日,濠先锋凌十一、闵廿四已至樵舍,风帆蔽江,前后数十里。我兵奉军令,乘夜趋进。
伍文定以正兵当其前,余恩继其后,邢珣引兵绕出贼背,徐琏、戴德孺分左右翼,各自攻击,以分其势。
二十四日早,北风大起,贼兵鼓噪,乘风而前,直逼黄家渡,离南昌仅三十里。
伍文定之兵才战,即佯为败走。余恩复战,亦佯退。贼得志,各船争前趋利,前后不相连。
邢珣兵从后而进,直贯其中,贼船大乱。伍文定、余恩督兵乘之,徐琏、戴德孺合势夹攻,
四面伏兵,纷纷扰扰,呼噪而至,满湖都是官军,正没摆布那一头处。
凌十一、闵廿四不过江湖行劫,几会曾见这等战阵,心胆俱落,急教回船,贼兵遂大溃。
官军追赶十余里,擒斩二千余级,凌十一中箭落水,贼徒死于水者万数。
闵廿四引着残卒数千,退保八字脑,手下兵士渐渐逃散。宸濠闻败,大惧,尽发九江、南康守城之兵以益师。
先生探听的实,曰:\quota{贼兵已撤,二郡空虚矣。不复九江,则南兵终不敢越九江以援我;
不复南康,则我兵亦不能逾南康以蹑贼。}
乃遣抚州知府陈槐领兵四百,合饶州知府林瑊兵,往攻九江。
适建昌知府曾玙兵亦到,即遣玙率兵四百,合广信知府周朝佐兵,往取南康。
二十五日,宸濠立赏格以激励将士。当先冲锋者,赏银千两,对阵受伤者,赏银百两,传令并力大战。
其日北风更大,贼船乘风奋击。伍文定率兵打头阵,因风势不顺,被杀数十人。
先生望见官军将有退却之意,急取令牌,将剑付中军官,令斩取领兵官伍文定头示众,
且暗嘱云:\quota{若能力战,姑缓之。}
文定见牌,大惊,亲握军器立于船头,督率军士,施放铳炮。风逆,火燎其须,不顾,军士皆拼命死战。
邢珣等兵俱至,一齐放炮,炮声如雷震天,将宸濠副舟击破,闵廿四亦被炮打死。
濠大骇,将船移动,贼遂溃败,擒斩复二千余,溺死无算。
濠乃聚兵屯于樵舍,连舟结为方阵,四面应敌。尽出金银,赏犒将士,约来日决一死敌。
先生乃密为火攻之具,使邢珣击其左,徐琏、戴德孺击其右,余恩等各官分兵四面暗伏,只望见火发,一齐合战。
二十六日早,宸濠方朝群臣,责备诸将不能力战,以致连败。
喝教先将三司各官杨璋、潘鹏等十余人绑起,责他坐观成败,全不用心,欲斩之以立法。
璋等立辩求免。正在争论之际,忽闻四下喊声大举。
伍文定引着官军,用小船载荻,乘风纵火,火烈风猛,延烧贼船。但见:

\startverse[leftalign=yes,sample={众将惊惶,各各魄散魂消,投戈弃甲。}]%
浓烟蔼蔼,青波上罩万道乌云;\\
紫焰烘烘,绿水中布千层赤雾。\\
三军慌乱,个个心惊胆裂,撇鼓丢锣;\\
众将惊惶,各各魄散魂消,投戈弃甲。\\
舴艋艨艟,一霎时变成煨烬;\\
旗旛剑戟,须臾顷化作灰尘。\\
分明赤壁遇周瑜,好似咸阳逢项羽。
\stopverse

各路伏兵望见火光,并力杀来。贼舟四面皆火,栱栟二人被火焚烧,奔出船舱,为官军所杀。
王春、吴十三亦被擒获。先生使人持大牌晓谕各军,牌上写云:
逆濠已擒,诸军勿得纵杀。愿降者听。\topnote{诡言得濠,以壮我而怠敌,此先生妙用。}
各军闻之,信以为然,勇气百倍。濠军莫不丧气,争觅小舟逃命。\buffernote[4-10]
宸濠知事不济,亦欲谋遁,与娄妃泣别曰:\quota{昔人亡国,因听妇人之言。我为不听贤妃之言,以至如此。}
娄妃哽咽,不能出声,但云:\quota{殿下保重,勿以妾为念。}
言毕,与宫娥数人跳下湖中而死。宸濠心如刀刺。
万锐觅得划船来到,濠变服,同锐下了划船,冒着兵戈而走,还带有宫女四人。
万安县知县王冕受先生密计,假装渔船数只,散伏芦苇。望见划船有些跷蹊,慌忙摇拢来看。
宁王认是渔船,唤曰:\quota{渔翁渡我,当有厚报。}

\startplacefigure[location={middle,top},title={二十六日,官军用火攻之。濠方朝群臣,责诘所执三司,铳声四起,四顾无兵,乃与妃嫔泣别。妃嫔争赴水死,三司走官军舟得免。万安兵遂执宁王。知县王冕执先登之旗以献俘。先生命二指挥夹扶之,置于清军察院。观者堵立,欢声动天地。}]%
  \externalfigure[imgs/王阳明图传/image00140.jpeg][normal-img-60][width=.6\textwidth]
\stopplacefigure

濠既下渔船,船上一声哨子,众船皆至。宸濠自知不免,亦投于水。
逢浅处,立水中不死。军士用长篙挽其衣而执之。
是时,伍文定、邢珣等乘胜杀入,先擒世子大哥及宫眷等,
其伪党李士实、刘养正、刘吉、屠钦、熊琼、卢𤤾、卢璜、丁馈、秦荣、
葛江、刘勋、何镗、吴国七、火信、喻才、李自然、徐卿等数百人,前后俱被擒获,无一漏者。
复执胁从王宏、\low{\hw 原镇守太监。}王金、\low{\hw 原巡按。}杨璋、\low{\hw 原按察使。}
金山、\low{\hw 原主事。}程果、\low{\hw 原参政。}潘鹏、\low{\hw 原佥事。}
梁宸、\low{\hw 原布政使。}郏文、马骥、白昂\low{\hw 俱指挥。}等。王纶、季敩赴水死。
擒斩共三千余人,落水者二万有余,衣甲、器械、财物与浮尸横十余里。
复分兵搜剿零贼于昌邑、吴城各处,擒斩殆尽。
湖口县知县章玄梅迎先生坐于城中,察院王冕解宸濠入城献功。
濠望见远近街衢行伍整肃,笑曰:\quota{此是我家事,何劳王都堂这等费心?}
既见先生,遂拱手曰:\quota{濠做差了事,死自甘心。但娄妃每每苦谏勿叛,乃贤妃也。已投水而死,望善葬之。}
先生即遣中军官同宫监一人前往识认,只见渔舟载有一尸,周身衣服,皆用线密密缝紧。
渔人疑有宝货在身,正欲搜简,就被宫监认出是娄妃,取来盛殓,埋葬于湖口县之城外,至今称为贤妃墓。
是日,众官俱来相见。先生下堂,执伍文定之手,曰:\quota{今番破贼,足下之功居多。本院即当首列,必有不次之擢。}
文定曰:\quota{仗圣天子洪福、老大人妙算,知府何功之有?}
先生曰:\quota{斩阵先登,人所共知,不必过谦。}
其余邢珣、余恩等,各以温言慰劳。众人各欢喜而退。

\startplacefigure[location={middle,top},title={见素林公俊闻濠变,即夜遣人范锡为佛郎机铳,并抄火药方,遣两仆自莆阳裹粮,从间道冒暑昼夜行三千余里,手书勉先生竭力讨贼。比至,则濠已擒七日。先生感激泣下,以世之当其任者犹畏难巧避,而公忠诚出天性,老而弥笃,身退而忧愈深,节愈励。为《书佛郎机遗事》,\buffernote[4-11]命同志和之。}]%
  \externalfigure[imgs/王阳明图传/image00142.jpeg][normal-img-60][width=.6\textwidth]
\stopplacefigure

次日,先生正在军中整理军务,中军官报单,报道:
知府陈槐、曾玙等分兵攻南康、九江,贼兵出战,俱为官军所败。
陈槐阵上斩了徐九宁,知县何士凤开门以迎王师,将城中余贼尽行诛剿。
南康百姓闻官军薄城,共杀陈贤,二郡悉平。
于是,贼党俱尽。
按:宸濠自六月十四日举逆,至七月十六日被获,前后共四十二日。
先生自七月十三日于吉安起马,至二十六日成功,才十有四日耳。
自古勘定祸乱,未有如此之神速者。但见成功之易,不知先生擘画之妙也。
是日,门生邹守益入见,贺曰:\quota{且喜老师成百世之功,名扬千载。}
先生曰:\quota{功何敢言?且喜昨晩沉睡。}
盖自闻报后,晓夜焦劳,至是,始得安枕矣。
先生口占一律,云:

\startverse[leftalign=yes,sample={甲马秋惊鼓角风,旌旗晓拂阵云红。}]%
甲马秋惊鼓角风,旌旗晓拂阵云红。\\
勤王敢在汾淮后,恋阙真随江汉东。\\
群丑漫劳同吠犬,九重端合是飞龙。\\
涓埃未尽酬沧海,病懒先须伴赤松。\buffernote[4-12]
\stopverse

是日,先生传令班师,暂回省城。城中听知王师凯旋,军民聚观者不下万数。
宸濠坐在小轿之中,其余贼党俱各囚车锁押。前后军兵拥卫,一个个枪刀出鞘,盔甲鲜明。
才至中街,两傍看者欢声如沸,莫不以手加额,曰:\quota{我等今日方脱倒悬之苦,皆王都爷之赐也。}
先生到察院下马,大会众官商议。
除将宁王并世子郡王、将军、仪宾,伪授太师、国师、元帅、都督、指挥等官,
各分别收监候解,其胁从等官,并各宗室,别行另奏。将擒斩俘获功次,发纪功。
御史谢源、伍希儒审验明白,造册。先生于三十日上捷报,据册开:

\startquotation
生擒首贼,一百零四名。\\
生擒从贼,六千一百七十五名。\low{\hw 内审放胁从一千一百九十三名。}\\
斩获贼级,共四千四百五十九颗。\\
俘获贼属男妇,二百三十八名口,宫人四十三名。\\
夺回被胁被掳官民人等,三百八十四员名口。\\
招抚畏服投首,一百九十三位名。\\
夺获符验一道,金玺二颗,金册二副,印信关防一百零六颗。\\
金并首饰,六百二十三两一钱二分;银首饰器皿,八万三千八百九十七两一钱五分零。\\
赃仗,一千八百九十件。器械,一千一百九十九件。\\
牛,三十头;马,一百九匹;驴骡,十三头;鹿,三只。\\
烧毁贼船,七百四十三只。
\stopquotation

后人有诗一绝,诵先生之功云:

\startverse[leftalign=yes,sample={指挥谈笑却莱夷,千古何人似仲尼?}]%
指挥谈笑却莱夷,千古何人似仲尼?\\
旬日之间除叛贼,真儒作用果然奇。
\stopverse

\stopsectionlevel

\startsectionlevel[default][title={忠泰之变}]%

话分两头。却说兵部尚书王琼,见先生所上宁王反叛两次表章,疏请五府六部大臣会议于左顺门。
诸臣中也有曾受宁王贿赂,与他暗通的;也有见宁王势大,怕他成事的,一个个徘徊观望,尚不敢斥言濠反。
王琼正色言曰:\quota{竖子素行不义,今仓卒造乱,自取灭亡耳。
都御史王守仁据上游,必能了贼,不日当有捷报至也。其请京军,特张威耳。}
乃顷刻覆了十三本,首请削宸濠属籍,正名为贼,布告天下。
但有忠臣义士,能倡义旅,擒反贼宸濠者,封以侯爵。先将通贼逆党朱宁、臧贤拿送法司正罪。
又传檄南京、两广、浙江、江西各路军马,分据要害,一齐剿杀。朝廷差安边伯许泰总督军务,充总兵官;
平虏伯江彬、太监张忠、魏彬俱为提督官;左都督刘翚为总兵官;太监张永赞画机密,并体勘濠反逆事情。
兵部侍郎王宪督理粮饷,前往江西征讨。行至临清地方,闻江西有捷报,宁王已擒。
许泰、江彬、张忠等耻于无功,乃密疏请御驾亲征,顺便游览南方景致。
武宗皇帝大喜,遂自称为总督军务威武大将军总兵官、后军都督府太师镇国公,往江西亲征。
廷臣力谏,不听,有被杖而死者。车驾遂发,大学士梁储、蒋冕扈从。
九月十一日,先生南昌起马,将宸濠一班逆党囚禁。先期遣官上疏,略云:

\startquotation
逆濠睥睨神器,阴谋久蓄。招纳叛亡,探辇毂之动静,日无停迹,广置奸细。臣下之奏白,百不一通。
发谋之始,逆料大驾必将亲征,先于沿途伏有奸党,为博浪、荆轲之谋。今逆不旋踵,遂已成擒。
法宜解赴阙下,式昭天讨。欲令部下各官押解,恐旧所潜布乘隙窃发,或致意外之虞,臣死有余憾。
况平贼献俘,国家常典,亦臣子常职。
臣谨于九月十一日亲自量带官军,将濠并宫眷、逆贼、情重罪犯潜解赴阙。
\topnote{疏中便暗止圣驾南巡,王公用心之密如此。}
\stopquotation

先生行至常山草萍铺,闻有御驾亲征之事,大惊曰:\quota{东南民力已竭,岂堪骚扰?}
即索笔题诗于壁上,传谕次早兼程而进。诗曰:

\startverse[leftalign=yes,sample={一战功成未足奇,亲征消息尚堪危。}]%
一战功成未足奇,亲征消息尚堪危。\\
边烽西北方传警,民力东南已尽疲。\\
万里秋风嘶甲马,千山晓日渡旌旗。\\
小臣何事驱驰急?欲请回銮罢六师。\buffernote[5-1]
\stopverse

时圣驾已至淮、徐。许泰、张忠、刘翚等见先生疏到,密奏曰:

\startquotation
陛下御驾亲征,无贼可擒,岂不令天下人笑话?且江南之游,以何为名?
今逆贼党与俱尽釜中之鱼,宜密谕王守仁释放宁王于鄱阳湖中,待御驾到,亲擒之。
他日史书上传说陛下英武,也教扬名万代。
\topnote{佞臣一言逢迎,不顾国家利害。如用其言,即宁贼死灰未必再燃,而将士解体尽矣。}
\stopquotation

武宗皇帝原是好顽耍的,听他邪说,果然用威武大将军牌面,遣锦衣千户追取宸濠。
先生行至严州,接了牌面。或言:\quota{威武大将军即一今上也,牌到,与圣旨一般,礼合往迎。}
先生曰:\quota{大将军品级不过一品,文武官僚不相统属,我何迎为?}众皆曰:\quota{不迎必得罪。}
先生曰:\quota{人子于父母乱命,不可告语,当涕泣随之,忍从谀乎?}
三司官苦苦相劝,先生不得已,令参随负敕印出,同迎以入。
中军禀问:\quota{锦衣奉御差至此,当送何等样程仪?}\topnote{以敕印抵当牌,亦一策也。}
先生曰:\quota{不过五金。}中军官曰:\quota{恐彼怒,不纳,奈何?}
先生曰:\quota{由他便了。}锦衣千户果然大怒,麾去不受。次日,即来辞别。
先生握其手曰:\quota{下官在正德初年,下锦衣狱甚久,贵衙门官相处极多,看来未见有轻财重义如公者。
昨薄物出区区鄙意,只求礼备。闻公不纳,令我惶愧。
下官无他长,单只会做几篇文字,他日当为公表章其事,令后世锦衣知有公也。}
锦衣唯唯,不能出一语,竟别去。先生竟不准其牌,不把宸濠与他。
锦衣星夜回报,许泰、江彬等大怒,遂造谤言,
说:\quota{先生先与宁王交通,曾遣门人冀元亨往见宁王,许他借兵三千,后见事势无成,然后袭取宁王以掩己罪。}
\topnote{此谤流传至今,尚有疑者,谗言可畏如此。}
太监张永素知先生之忠,力为辩雪,且请先行查访。先生至杭州,张永先在,先生与永相见,
永曰:\quota{泰、彬等诽谤老先生,只因先生献捷太早,阻其南行,以此不悦。}
\topnote{王公免祸,全得张永之力。}
先生曰:\quota{西民久遭濠毒,今经大乱,继以旱灾,困苦已极。
若边军又到,责以供饷,穷迫所激,势必逃聚山谷为乱。奸党群应,土崩之势成矣。更思兴兵伐之,不亦难乎?}
张永深以为然,徐曰:\quota{本监此出,正为群小蛊惑圣听,欲于中调获,非掩功也。
但皇上圣意,亦耻巡游无名。老先生但将顺天意,犹可挽回几分。苟逆之,徒激群小之怒,何救于大事?}
先生曰:\quota{老公所见甚明,下官不愿居功,情愿都让他们。容下官乞休而去,足矣。}
乃以宸濠及逆党交付张永,遂上疏乞休,屏去人从,养病于西湖之净慈寺。
张永在武宗皇帝面前备言王守仁尽心为国之忠,江西反侧未安,全赖弹压,不可听其休致自便。
诸奸捕冀元亨,付南京法司,备极拷掠,并无一语波及先生,奸谋乃沮。
忠、泰等又密奏:\quota{宁王余党尚多,臣等愿亲往南昌搜捕,以张天威。}
武宗皇帝复许之。比及先生赴南昌任,忠、泰等亦至。带令北军二万,填街塞巷。
许泰、江彬、张忠坐了察院,妄自尊大。先生往拜之,泰等看坐于傍,令先生坐。
先生佯为不知,将傍坐移下,自踞上坐,使泰、彬等居主位。泰、彬等且愧且怒,以语讽刺先生。
先生以交际事体谕之,然后无言。
先生退,谓门人邹守益等曰:\quota{吾非争一主也,恐一屈体于彼,便当受其节制,举动不得自由耳。}
泰、彬等托言搜捕余党,扳害无辜富室,索诈贿赂,满意方释。
又纵容北军占居民房,抢掠市井财物,向官府索粮要赏。
或呼名谩骂,或故意冲导,欲借此生衅,与先生大闹一场,就好在皇上面前谤毁。
先生全不计较,务待以礼。预令市人移居乡村,以避其诈害,仅以老羸守家。
先生自出金帛,不时慰犒北军,病者为之医药,死者为之棺殓,边军无不称颂王都堂是好人。
泰、彬等怪先生买了军心,严禁北军不许受军门犒劳。先生乃传示内外:
北军离家苦楚,尔居民当敦主客之礼。百姓遇边军,皆致敬,或献酒食。
北军人人知感,不复行抢夺之事。
时十一月冬至将近,先生示谕:
百姓新遭濠乱,横死甚多,深为可悯。今冬节在迩,凡丧家俱具奠如礼,如在官人役,给假三日。
于是,居民家家上坟酬酒,哀哭之声,远近相接。北军闻之,无不思家,至于泣下,皆向本官叩头求归。
分明是:楚歌一夜起,吹散八千兵。

张忠、许泰、刘翚等自恃北人所长在于骑射,度先生南人,决未习学。
一日,托言演武,欲与先生较射。先生谦谢不能,再四强之。
先生曰:\quota{某书生,何敢与诸公较艺?诸公请先之。}
刘翚以先生果不习射矣,意气甚豪,谓许泰、张忠曰:\quota{吾等先射一回,与王老先生看。}
军士设的于一百二十步外。三人雁行叙立,张忠居中,许泰在左,刘翚在右,各逞精神施设。
北军与南军分列两边,抬头望射。一个个弓弯满月,箭发流星,每一发矢,叫声\quota{着}。
一会,箭九枝都射完了。单只许泰一箭射在鹄上,张忠一箭射着鹄角,刘翚射个空回。
他三个都是北人,惯习弓矢,为何不能中的?一来欺先生不善射,心满气骄了;
二来立心要在千人百眼前逞能炫众,就有些患得患失之心,矜持反太过,一箭不中,便着了忙,所以中的者少。
三人射毕,自觉出丑,面有愧色,说道:\quota{咱们自从跟随圣驾,久不曾操弓执矢,手指便生疏了。
必要求老先生射一回赐教。}
先生复谦让,三人越发相强,务要先生试射,射而不中,自家便可掩饰其惭。
先生被强不过,顾中军官取弓箭来,举手对泰、彬等曰:\quota{下官初学,休得见笑。}
先生独立在射棚之中,三位武官太监环立于傍,光着六只眼睛含笑观看。
先生神闲气定,左手如托泰山,右手如抱婴儿,飕的一箭,正中红心,北军连声喝采,
都道:\quota{好箭!射的准!射的准!}泰、彬等心中已自不快了,还道:\quota{是偶然幸中。}
先生一连又发两矢,箭箭俱破的。北军见先生三发三中,都道:\quota{咱们北边到没有恁般好箭。}欢呼动地。
泰等便执住先生之手,说道:\quota{到是老先生久在军中,果然习熟。已见所长,不必射了。}遂不乐而散。
是夜,刘翚私遣心腹窥探北军口气,
一个个都道:\quota{王都堂做人又好,武艺又精,咱们服事得这一位老爷,也好建功立业,不枉为人一世。}
刘翚闻之,一夜不睡。次早,见许泰、张忠,曰:\quota{北军俱归附王守仁矣!奈何?}
泰、忠乃商议班师。前后杀害良民数百,皆评为逆党,取首级论功。北军离了西江省城,百姓始复归乐业。
时武宗皇帝大驾自淮阳至京口,馆于前大学士杨一清之家。
泰等来见,但云逆党已尽,遂随驾渡江,驻跸南都,游览江山之胜。
三人乘间谗谤先生,说:\quota{他专兵得众,将来必有占据江西之事。}赖张永一力周旋,上信永言,付之不问。
泰等又遣心腹屡矫伪旨,来召先生。只要先生起马,将近南都,遂以擅离地方驾罪。先生知其伪,竟不赴。
正德十五年正月,先生尚留省城,泰等三人因侍宴武宗皇帝,言及天下太平,
三人同声对曰:\quota{只江西王守仁早晩必反,甚是可忧。}
武宗皇帝问曰:\quota{汝谓王守仁必反,以何为验?}
三人曰:\quota{他兵权在手,人心归向。去岁臣等带领边兵至省城,他又私恩小惠,买转军心。
若非臣等速速班师,连北军多归顺他了。皇爷若不肯信,只须遣诏召之,他必不来。}
武宗皇帝果然遣诏,召先生面见。张永重先生之品,又怜先生之忠,密地遣人星夜驰报先生,尽告以三人之谋。

先生得诏,即日起马,行至芜湖。张忠闻先生之来,恐面召时有所启奏,复遣人矫旨止之。
先生留芜湖半月,进退维谷,不得已,入九华山,每日端坐草庵中。
一日,微服重游化城寺,至地藏洞。思念二十七岁时,于此洞见老道,共谈三教之理。
今年四十九岁,不觉相隔二十二年矣。功名羁绊,不得自由,进不得面见圣上,扫除奸佞;
退不得归卧林泉,专心讲学。不觉凄然长叹,取笔砚题诗一首,诗曰:

\startverse[leftalign=yes,sample={爱山日日望山晴,忽到山中眼自明。}]%
爱山日日望山晴,忽到山中眼自明。\\
鸟道渐非前度险,龙潭更比旧时清。\\
会心人远空遗洞,识面僧来不记名。\\
莫谓中丞喜忘世,前途风浪苦难行。\buffernote[5-2]
\stopverse

\startplacefigure[location={middle,top},title={欲见上于南京,不得入,遂游九华山。与诸生江学曾、施宗道、陶埜及寺僧历探云峰、化城之胜,咏赋盈壁。许泰谤先生谋反,上默不应,夜遣亲信觇之,见游山题诗,归以实奏,乃复有巡抚江西之命。}]%
  \externalfigure[imgs/王阳明图传/image00143.jpeg][normal-img-60][width=.6\textwidth]
\stopplacefigure

又见山岩中有僧危坐,问:\quota{何时到此?}僧答曰:\quota{已三年矣。}
先生曰:\quota{吾儒学道之人,肯如此精专凝静,何患无成?}复吟一诗云:

\startverse[leftalign=yes,sample={莫怪岩僧木石居,吾侪真切几人如。}]%
莫怪岩僧木石居,吾侪真切几人如。\\
经营日夜身心外,剽窃糠粃齿颊余。\\
俗学未堪欺老衲,昔贤取善及陶渔。\\
年来奔走成何事?此日斯人亦启予。\buffernote[5-3]
\stopverse

\startplacefigure[location={middle,top},title={恶,纵有铁船还未牢。秦鞭驱之不能动,奡力何所施其篙。我欲乘之访蓬岛,雷师鼓舵虹为缫。弱流万里不胜芥,复恐驾此成徒劳。世路难行每如此,独立斜阳首重搔。阳明山人书于铜陵舟次,时正德庚辰春分献俘还自南都。}]%
  \externalfigure[imgs/王阳明图传/image00144.jpeg][normal-img-60][width=.6\textwidth]
\stopplacefigure

\startplacefigure[location={middle,top},title={铜陵观铁船: 录寄士洁侍御道契,见行路之难也。青山滚滚如奔涛,铁船何处来停桡。人间刳木宁有此,疑是仙人之所操。仙人一去已千载,山头日日长风号。船头出土尚仿佛,后冈有石云船稍。我行过此费忖度,昔人用心无己忉。由来风波平地}]%
  \externalfigure[imgs/王阳明图传/image00145.jpeg][normal-img-60][width=.6\textwidth]
\stopplacefigure

张忠等既阻先生之行,反奏先生不来朝谒。
武宗皇帝问于张永,永密奏曰:\quota{王守仁已到芜湖,为彬等所拒。
彼忠臣也,今闻众人争功,有谋害之意,欲弃其官入山修道。
此人若去,天下忠臣更无肯为朝廷出力者矣。}
武宗皇帝感动,遂降旨,命先生兼江西巡抚,刻期速回理事。
先生遂于二月还南昌,以祖母岑太夫人鞠育之恩,临终不及面诀,乃三疏请归省葬,俱不允。
六月,复还赣州。过泰和,少宰罗整庵讳钦顺,弘治癸丑榜眼。以书问学。先生告以:

\startquotation
学无内外。格物者,格其心之物也;正心者,正其物之心也。
   以理之凝聚而言,则谓之性;     以其主宰而言,则谓之心;
以其主宰之发动而言,则谓之意;以其发动之明觉而言,则谓之知;
以其明觉之感应而言,则谓之物。
故就物而言,谓之格;就知而言,谓之致;就意而言,谓之诚;就心而言,谓之正。
所谓穷理以尽性,其功一也。天下无性外之理,即无性外之物。
学之不明,皆由世儒认理为内,认物为外,将反观内省与讲习讨论分为两事,所以有朱、陆之岐。
然陆象山之致知,未尝专事于内;朱晦庵之格物,未尝专事于外也。\buffernote[5-4]
\stopquotation

整庵深叹服焉。

\startplacefigure[location={middle,top},title={六月,按吉安,吉安乡士夫趋而会,乃宴于文山祠。复偕佥事李素及伍希儒、邹守益游青原山。推官王暐具碑以请和黄山谷韵,亲登于石。\buffernote[5-5]论抗许泰等及驭边兵颠末,曰:\quota{这一段劳苦,更胜起义师时。}}]%
  \externalfigure[imgs/王阳明图传/image00146.jpeg][normal-img-60][width=.6\textwidth]
\stopplacefigure

是年秋七月,武宗皇帝尚在南都。许泰、江彬欲自献俘,以为己功。张永曰:\quota{不可。昔未出京时,宸濠已擒。
献俘北上,过玉山,渡钱塘,在杭州交割于吾手,经人耳目,岂可袭也?}
于是,用威武大将军钧帖,下于南赣,令先生重上捷音。
先生乃节略前奏,尽嵌入许泰、江彬、张忠、魏彬、张永、刘翚、王宪等扈驾诸官,
疏中言:\quota{逆濠不日就擒,此皆总督提督诸臣密授方略所致。}
于是,群小稍稍回嗔作喜,止将冀元亨坐濠党系狱,先生遂得无恙。
后世宗皇帝登极,先生备咨刑部,为元亨辩冤。\buffernote[5-6]
科道亦交章论之。将释放,而元亨死。\buffernote[5-7]
同门陆澄、应典辈备棺盛殓。先生闻讣,为设位恸哭之。此是后话。
\stopsectionlevel

\startsectionlevel[default][title={此心光明}]%

是年九月,先生再至南昌,檄各道院,取宸濠废地,改易市廛,以济饥代税,百姓稍得苏息。
时有泰州王银者,服古冠,执木简,写二诗为赘,以宾礼见。先生下阶迎之,银踞然上坐。
先生问:\quota{何冠?}曰:\quota{有虞氏之冠。}
又问:\quota{何服?}曰:\quota{老莱子之服。}
先生曰:\quota{君学老莱乎?}对曰:\quota{然。}
先生曰:\quota{君学老莱,止学其服耶,抑学其上堂诈跌为小儿啼也?}银不能答,色动,渐将坐椅移侧。
及论致知格物,遂恍然悟曰:\quota{他人之学,饰情抗节,出于矫强;先生之学,精深极微,得之心者也。}
遂反常服,执弟子之礼。先生易其名为\quota{艮},字曰\quota{汝止}。\buffernote[6-0-1]
同时,陈九川、夏良胜、万潮、欧阳德、魏良弼、李遂、裘衍日侍讲席,有洙泗杏坛之风。
是年冬,武宗皇帝自南京起驾,行至临清,将宸濠一班逆贼并正刑诛,人心大快。
正德十六年春正月,武宗皇帝还京,三月晏驾。
四月,世宗皇帝登极,改元嘉靖,诛江彬、许泰、张忠、刘翚等诸奸,录先生功,降敕召之。
先生以六月二十日起程,方至钱塘,科道官迎阁臣意,建言国丧多费,不宜行宴赏之事。
先生复上疏,乞便道省亲。\buffernote[6-1]得旨,升南京兵部尚书,赐蟒玉,准其归省。
九月,至余姚,拜见龙山公。公当宸濠谋逆时,有言先生助逆者,公曰:\quota{吾儿素在天理上用工夫,必不为此。}
又或传先生与孙、许同被害者,公曰:\quota{吾儿得为忠臣,吾复何忧?}
及闻先生起兵讨濠,又传言:\quota{濠怒先生,欲遣人来刺公,公宜少避。}
公笑曰:\quota{吾儿方举大义,吾为国大臣,恨年老不能荷戈同事,奈何先去以为民望乎?}
怡然不变。\buffernote[6-2]至是相见,欢如再生。\buffernote[6-3]
值龙山公诞日,朝廷存问适至,先生服蟒腰玉,献觞称贺。
至明旦,谓门人曰:\quota{昨日蟒玉,人谓至荣,晩来解衣就寝,依旧一身穷骨头,何曾添得分毫?
乃知荣辱原不在人,人自迷耳。}乃吟诗一首云:

\startverse[leftalign=yes,sample={百战归来白发新,青山从此作闲人。}]%
百战归来白发新,青山从此作闲人。\\
峰攒尚忆冲蛮阵,云起犹疑见虏尘。\\
岛屿微茫沧海暮,桃花烂熳武陵春。\\
而今始信还丹诀,却笑当年识未真。\buffernote[6-4]
\stopverse

先生日与亲友及门人辈宴游山水,随地指点良知,一时新及门就学者七十四人。

\startplacefigure[location={middle,top},title={通天岩,濂溪公所游,至是夏良胜、邹守益、陈九川宿岩中,肄所问,刘寅亦至。先生乘霁入,尽历,忘归忘言。各岩和诗立就,题玉虚宫壁,命蔡世新绘为图。}]%
  \externalfigure[imgs/王阳明图传/image00147.jpeg][normal-img-60][width=.6\textwidth]
\stopplacefigure

是年十二月,朝廷论江西功,封先生为新建伯,食禄一千石,荫封三代。\buffernote[6-5]
少时梦威宁伯王越解剑相赠,至是始验。明年正月,先生疏辞封爵,不允。
时龙山公年七十有七,病笃在床,将属纩\buffernote[6-6],闻部咨已至,促先生及诸弟出迎,
曰:\quota{虽仓遽,乌可以废礼?}少顷,问:\quota{已成礼否?}家人曰:\quota{诏书已迎至矣。}乃瞑。
先生戒家人勿哭。加新冕服,拖绅。事毕,然后举哀,一哭顿绝,病不能胜。
门人子弟纪丧,因才任使,仙居金克厚典厨,内外井井。\buffernote[6-7]

\startplacefigure[location={middle,top},title={九月龙山公寿旦,适封爵使至,封公勋阶、爵邑如其子。四方缙绅、门弟子咸捧觞为寿,公蹙然曰:\quota{吾父子乃得复相见耶!仰仗宗社、神灵、朝廷威德,岂一书生所能辨?谗构横行,祸机四发,赖武庙英明保全,以膺新宠。宠荣极矣,宜以盈满为戒,庶不致覆成功而毁令名。}先生跪受教。}]%
  \externalfigure[imgs/王阳明图传/image00148.jpeg][normal-img-60][width=.6\textwidth]
\stopplacefigure

先生以先后平贼,皆赖兵部尚书王琼从中主持,又同事诸臣多有劳绩,己何敢独居其功?
再上疏辞爵,归功于琼。时宰方忌琼,并迁怒于先生。御史程启充、给事中毛玉相率论劾先生,指为邪学。
先生讲论如故。门人谦之\buffernote[6-8]临去,先生赠诗云:

\startverse[leftalign=yes,sample={珍重江船冒暑行,一宵心话更分明。}]%
珍重江船冒暑行,一宵心话更分明。\\
须从根本求生死,莫向支离辩浊清。\\
久奈世儒横臆说,竞搜物理外人情。\\
良知底用安排得,此物由来是浑成。
\stopverse

嘉靖三年,海宁董澐,号萝石,以能诗闻于江湖。年六十八,来游会稽。
闻先生讲学,戴笠携瓢,执杖来访。入门长揖上坐。先生敬异之,与语连日夜,澐言下有悟。
因门人何秦请拜先生门下,先生以其年高不许。归家,与其妻织一缣以为贽,复因何秦来强。\buffernote[6-9]
先生不得已,与之倘佯山水间。澐日有所闻,欣然乐而忘归。
其乡之亲友,皆来劝之还乡,曰:\quota{翁老矣,何自苦如此?}
澐曰:\quota{吾今方扬鬐于渤海,振羽于云霄,安能复投网罟而入樊笼乎?去矣,吾将从吾所好。}
遂自号\quota{从吾道人}。时郡守南大吉,先生所取士也,以座主故,拜于门下,
然性豪旷不羁,不甚相信,遣弟南逢吉觇之,归,述先生讲论。
如此数次,大吉乃服,始数来见,且曰:\quota{大吉临政多过失,先生何无一言?}
先生曰:\quota{过失何在?}大吉历数某事某事。
先生曰:\quota{吾固尝言之矣。}大吉曰:\quota{先生未尝见教也?}先生曰:\quota{吾不言,汝何以知之?}大吉曰:\quota{良知。}
先生笑曰:\quota{良知非我常言而何?}大吉笑谢而去。于是,辟稽山书院,聚八邑彦士讲学。
萧璆、杨汝荣、杨绍芳等,来自湖广;杨仕鸣、薛宗铠、黄梦星等,来自广东;王艮、周冲等,来自南直;
何秦、黄弘纲\buffernote[6-10]等,来自南赣;刘邦采、刘文敏等,来自安福;曾忭,来自泰和;
魏良政、魏良器等,来自新建。宫刹卑隘,至不能容。每一发讲,环而听者,三百余人。
一日,讲\quota{君子喻义,小人喻利}章,众人俱发汗泣下。\buffernote[6-11]
邑庠生王畿与魏良器相厚,每言妨废举业,劝勿听讲。
及是日闻讲,自悔失言,即日执贽为弟子。\buffernote[6-12]
嘉靖四年,门人辈立阳明书院于越城西郭门内,光相桥之西。
明年正月,邹守益以直谏谪判广德州,筑复初书院,集生徒讲学。先生为书赞之。\buffernote[6-13]
四月,南大吉入觐,被黜,略无愠色,惟以闻道为喜,其得力于先生之熏陶者多矣。
是夏,御史聂豹巡按福建,特渡钱塘,来谒先生,听讲而去。时席书为礼部尚书,特疏荐先生。
御史石金等亦交章论荐,不报。

\startplacefigure[location={middle,top},title={遂亲至南宁府,尽撤调集防守之兵数万,惟留湖兵数千,分南宁、宾州解甲修养。二酋闻之,以正月七日遣头目赴军门,愿扫境投生。下牌省谕,申以圣天子威德。二酋且惧其喜,撤守备,具衣粮,卢苏以众五万,王受以众三万,于二十六日至南宁。次日,囚首自缚,又与头目数百号哀投诉。复申牌谕:\quota{宥尔一死,各于军门杖一百。}众叩首请命。乃解其缚,谕以:\quota{宥一死者,朝廷、天地好生之仁;杖一百者,我等人臣执法之义。}皆叩首悦服。}]%
  \externalfigure[imgs/王阳明图传/image00149.jpeg][normal-img-60][width=.6\textwidth]
\stopplacefigure

嘉靖六年,广西田州岑猛作乱。提督都御史姚镆征之,擒猛父子。
未几,其头目卢苏、王受构众复乱,攻陷思恩。镆复调四省兵征之,弗克。
阁老张璁、桂萼共荐先生,起用,总督两广及江西湖广军务。
先生闻命,力辞,不允,乃于九月起马,由杭、衢,历常山、南昌、吉安诸处,一路门人迎接者,动数百人,不必细说。
十一月,至梧州。先生以土官之叛皆由流官掊克所致,乃下令尽撤调集防守之兵,使人招卢苏、王受,喻以祸福。
二人见守兵尽撤,遂自缚谢罪。先生杖而释之,抚定其众,凡七万余人。不动声色,一境悉平。\buffernote[6-14]
\topnote{一语洞见祸本。防守之兵,惟先生可撤,以其制贼有余也。}
时八寨、断藤峡等处,自韩都堂雍平定以后,至是,复据险作乱。
先生因湖广归师之便,密授方略,令袭之。卢苏、王受请出兵饷,当先效力。
三月之间,斩首三千余级,扫荡其巢而还。朝中当事大臣犹以先生擅兵讨贼为罪,
赖学士霍韬力诵其功,乃得免议,止以招抚思、田之功,颁赐奖赏。\buffernote[6-0-2]
\topnote{若养成大寇,罪将何任?当事者嫉贤忌功,不顾国家,从来久矣。可叹!}
先生一日谒伏波将军庙,庙在梧州。拜其像,
叹曰:\quota{吾十五岁梦谒马伏波,今日所见,宛如梦中。人生出处,岂偶然哉?}因赋诗云:

\startverse[leftalign=yes,sample={四十年前梦里诗,此行天定岂人为?}]%
四十年前梦里诗,此行天定岂人为?\\
徂征敢倚风云阵,所过须同时雨师。\\
尚喜远人知向望,却惭无术救疮痍。\\
从来胜算归廊庙,耻说兵戈定四夷。\buffernote[6-15]
\stopverse

先生大兴思、田学较,广西士民始知有理学。\buffernote[6-16]

\startplacefigure[location={middle,top},title={九月初八日,行人冯恩赍敕奖励,赏金帛羊酒,有开诚宣恩,处置得宜,叛夷畏,率归众降,罢兵息民之褒。因具奏谢恩,恳求养疾。自梧道广,待命于韶、雄之间。力疾答聂文蔚,论勿忘勿助宗旨。十一月二十九日,薨于南安之青龙铺。}]%
  \externalfigure[imgs/王阳明图传/image00150.jpeg][normal-img-60][width=.6\textwidth]
\stopplacefigure

十月,先生以积劳成疾,病剧,上疏乞休。不候旨,遂发。\buffernote[6-17]
布政使王大用亦先生门人,备美材以随。十一月廿五日,逾梅岭,至南安登舟。\buffernote[6-18]
南安府推官、门人周积来见。先生犹起坐,咳喘不已,犹以进学相勉。
廿八日晩,泊船,问:\quota{何地?}侍者对曰:\quota{青龙铺。}
明日,召周积至船中,积拱俟良久。先生开目视曰:\quota{吾去矣。}
积泣下,问:\quota{有何遗言?}先生笑曰:\quota{此心光明,复何言哉?}
少顷,瞑目而逝。时廿九日也,享年五十七岁。\buffernote[6-19]
南赣兵备、门人张思聪进迎于南野驿,用王布政所赠美材制棺,周积就驿中堂沐浴,衾殓如礼。
明日为十二月朔,安成门人刘邦采适至,同官属师生设奠入棺。
初四日,舆衬登舟,士民远近遮道,哭声震地,如丧考妣。舟过地方,门生故吏连路设祭哭拜。
将发南昌,东风大逆,舟不能行。
门人赵渊祝于柩前曰:\quota{先生岂为南昌士民留耶?越中子弟门人相候已久矣。}
祝毕,忽变西风,舟人莫不惊异。\buffernote[6-20]
门人王畿等数人以会试起身,闻先生讣音,还舟执丧。
二月,抵家。子弟门人辈奉柩于中堂,遂饰丧纪。
妇人哭于门内,孝子及亲族子弟哭于幕外,门人哭于门外。
每日,四方门人来者百余人。

\startplacefigure[location={middle,top},title={八年己丑正月,先生丧过江西,有司分道而迎。储御史良材、赵提学渊哭之哀,或问之,曰:\quota{吾岂徒为乃公哭耶?}士民哭声载道。钱德洪、王畿赴廷试,闻变,迎至广信。为疏赴同志。十一月二十九日,四方学者会葬于山阴兰亭之紫洪山。\buffernote[6-23]}]%
  \externalfigure[imgs/王阳明图传/image00151.jpeg][normal-img-60][width=.6\textwidth]
\stopplacefigure

十一月,葬横溪,先生所自择地也。先是,前溪水入怀,与左溪会,冲啮右麓,术者心嫌,欲弃之。
有山翁梦见一神人,绯袍玉带,\buffernote[6-21]立于溪上,曰:\quota{吾欲还水故道。}
明日,雷雨大作,溪水泛溢,忽从南岸而行,明堂周阔数百丈,遂定穴。
门人李珙等更番筑治,昼夜不息,月余墓成。
会葬者数千人,门人中有自初丧迄葬不归者,即孔门弟子之怀师,亦不是过矣。
御史聂豹原未拜门下,及闻讣之后,遣吊奠,亦称门人。盖素佩先生之训,中心悦而诚服也。
后十二年,浙江巡按御史周汝贞,亦先生门人,为建祠于阳明书院之楼前,扁曰\quota{阳明先生祠}。
各处书院俱立先生牌位,朝夕瞻礼,比于仲尼。今子孙世世袭爵为新建伯不绝。
先生幼时常言:\quota{一代状元不为希罕。}又言:\quota{须作圣贤,方是人间第一流!}斯言岂妄发哉!
先生殁后,忌其功者或斥为伪学,久而论定。
至今道学先生尊奉阳明良知之说,圣学赖以大明,公议从祀圣庙。\buffernote[6-22]
后学有诗云:

\startverse[leftalign=yes,sample={三言妙诀致良知,孔孟真传不用疑。}]%
三言妙诀致良知,孔孟真传不用疑。\\
今日讲坛如聚讼,惜无新建作明师。
\stopverse

又,髯翁有诗云:

\startverse[leftalign=yes,sample={平蛮定乱奏奇功,只在先生掌握中。}]%
平蛮定乱奏奇功,只在先生掌握中。\\
堪笑伪儒无用处,一张利口快如风。
\stopverse

\stopsectionlevel

\startsectionlevel[appendix][title={附录一\crlf 三教偶拈叙 / 冯梦龙}]%

\ldots{}\ldots{}
仙人,于是鼎湖瑶池神其说,蓬莱方壶侈其胜,安期羡门异其人,咒禁符水岐其术。
要之,方外别是一种,与道无与。故刘歆《七略》以道家为诸子,神仙为方技,良有以尔。
迨李少君、寇谦之之辈,务为迂怪附会,以干人主之泽,而神仙与道合为一家,遂与儒教绝不相似。
此道与儒分合之大略也。

若夫佛乃胡神,西荒所奉。相传秦时,沙门利室房入朝,始皇囚之,有金人穿牖而去。
至汉明帝时,金人入梦,遣使请经四十二章于西域,而佛之名始闻。
浸假而琳宫创于孙吴,法藏广于苻秦,忏科备于萧梁,释教乃大行,而俨然与儒、道鼎立为三,甚且掩而上之。
此三教始终之大略也。

是三教者,互相讥而莫能相废。吾谓得其意皆可以治世,而袭其迹皆不免于误世。
舜之被袗鼓琴,清净无为之旨也;禹之胼手胝足,慈悲徇物之仁也。
谓舜禹为儒可,即谓舜禹为仙、为佛,亦胡不可?
而儒者乃谓汉武惑于仙而衰,梁武惑于佛而亡,不知二武之惑正在不通仙佛之教耳。
汉武而真能学仙,则必清净无为,而安有算商车、征匈奴之事?
梁武而真能学佛,则必慈悲徇物,而安有筑长堰、贪河南之事?
宋之崇儒讲学,远过汉唐,而头巾习气刺于骨髓,国家元气日以耗削。
试问航海而犹讲《大学》,与戎服而讲《老子》《仁王经》者,其蔽何异?
则又安得以此而嗤彼哉!余于三教概未有得,然终不敢有所去取其间。
于释教,吾取其慈悲;于道教,吾取其清净;于儒教,吾取其平实。所谓得其意皆可以治世者,此也。

偶阅《王文成公年谱》,窃叹谓文事武备,儒家第一流人物,暇日演为小传,
使天下之学儒者,知学问必如文成,方为有用。因思向有济颠、旌阳小说,合之而三教备焉。
夫释如济颠,道如旌阳,儒者未或过之,又安得以此而废彼也?

\startalignment[flushright]\hw
\hfill 东吴畸人七乐生撰\\
\hfill (录自中华书局1990年影印《古本小说丛刊》第四辑,原缺一页。)\\
\stopalignment
\stopsectionlevel

\startsectionlevel[appendix][title={附录二\crlf 王阳明先生图谱序 / 王宗沐}]%

昔者孔子之没也,游、夏门人以有若貌似孔子,欲以所事孔子事之,而曾子独以为不可,
曰:\quota{江汉以濯之,秋阳以暴之,皜皜乎不可尚已。}盖深言之也。
本体之在人,流贯圆莹,昭明灵变,所谓建于天地而彻于古今者,一刻未尝息,一毫不可污,其斯以为皜皜也。
孔子之所以为孔子,全是而已。如徒以貌也,则途之人有肖者焉。
至语其精心,则不极于皜皜者,不可以语精,而况于形乎?心无似者也。
曾子之称孔子也,不道其绥来动和之所为用,而指其光辉洁白之所以妙。
盖自颜子而后,惟曾子得其深,此曾子、游、夏之辩也。
虽然,余尝思之矣,曾子盖亦有未尽者,三千笃信,沦浃肌髓,
一旦泰山颓坏,众志茕然,如孺子之丧慈母,无所依归,其学不皆曾子,
苟一有所存焉,亦足以收其将散之心,而植其未废之教。
故余尝谓项氏梁、籍之强,用兵如神,业已破秦,乃从民间求牧竖怀王立之,彼安所资哉?
楚人思故主,从其心而立之。怀王不足以兴楚,而足以系楚,系则由以兴。游夏之意,何以异此?

阳明王先生天挺间出,少志圣贤,出入二氏,晚悟正脉,的然以良知为入门,盖有见于皜皜者。
故自髫年以比白首,凡所作用,以其学取力焉。
忠挠权嬖,志坚拂抑,崎崎甲兵以及临民处变,染翰吐词,靡不精解融彻,而功业、理学盖宇宙百世师矣。
当时及门之士,相与依据尊信,不啻三千之徒。今没才三十年,学亦稍稍失指趣。
高弟安成东廓邹公辈相与绘图勒石,取先生平生经历之所及,与功用之大,谱而载焉。
嗟夫!皜皜之体,人人同具,先生悟而用之,则凡后之求先生者,于心足矣。
而公犹为是,非独思其师,亦以著教也,所谓系而待其兴焉者也。据其渐,则觉其进;
考其终,则见其成。而其中之备尝辛苦艰难,仅得悟于百死一生之际者,学之道良在于兹,而独载其事耶?

余少慕先生,十四岁游会稽,而先生已没。
两官先生旧游之地,凡事先生者皆问而得概焉,然不若披图而溯之为尤详也。
以余之尤有待于是,则后世可知,而邹公之意远矣。公遣金生应□来请余序,为道曾子之未尽者,以明公旨焉。

\startalignment[flushright]\hw
\hfill 时嘉靖丁巳冬十有一月长至\\
\hfill \hbox{赐进士出身中顺大夫江西按察司副使奉敕再提督学政}\\
\hfill \hbox{临海后学王宗沐书}\\
\hfill(录自1941年影印本《王阳明先生图谱》,
      漫漶不清处据明万历刻本《敬所王先生文集》卷一《阳明先生图谱序》校补,
      版刻错误依校本径改。)\crlf
\stopalignment

\stopsectionlevel
\stopbodymatter
\stoptext
