
% 注释
\startbuffer[1]
会元:会、元,皆为时间单位,来自宋代思想家邵雍提出的一种宇宙观,以元、会、运、世为单位推演宇宙变化。三十年为一世,十二世为一运,三十运为一会,十二会为一元。每一元即为一次宇宙生灭周期。下文『盖闻天地之数』一节,即在解释这一观念。
\stopbuffer


\startbuffer[2]
『子时』句:古用『夜半、鸡鸣、平旦、日出、食时、隅中、日中、日昳、晡时、日入、黄昏、人定』命名十二时辰,与十二地支命名一一相对。本句中『鸡鸣』『日出』等皆属于此。蹉,倾斜、下坠。晡(bū),即申时。
\stopbuffer


\startbuffer[3]
大数:自然法则;气数。
\stopbuffer


\startbuffer[4]
否(pǐ):闭塞。
\stopbuffer


\startbuffer[5]
贞下起元:《周易》有『乾,元亨利贞』句。后世以『元、亨、利、贞』象征阴阳演化的一个循环。『贞下起元』即『贞』终而『元』始。这里指『亥会』和『子会』的接续。
\stopbuffer


\startbuffer[6]
邵康节:即邵雍。北宋哲学家。字尧夫,谥康节。
\stopbuffer


\startbuffer[7]
『冬至』四句:大意为,冬至是子月(农历十一月)过了一半,天道规律并没有改变。此时阳气刚刚萌动,万物尚未生长。冬至后白昼渐长,传统哲学认为此时正是阴极阳生之时。
\stopbuffer


\startbuffer[8]
三才:古人以『天、地、人』为三才。
\stopbuffer


\startbuffer[9]
『四大部洲』句:四大部洲的概念源自古印度,在其神话中,人类居住世界的中心位于须弥山,须弥山被咸海包围,海中分布有四大部洲。
\stopbuffer


\startbuffer[10]
十洲之祖脉,三岛之来龙:十洲、三岛,道教称神仙在大海中的居所。十洲包括祖洲、瀛洲等处。三岛指蓬莱、方丈、瀛洲。祖脉、来龙,堪舆学术语,指山的源头。
\stopbuffer


\startbuffer[11]
『木火』句:意指花果山在大海的东南方。木火,指东南方。五行观念中,东方属木,南方属火。隅(yú),角落。
\stopbuffer


\startbuffer[12]
政历:即历法。
\stopbuffer


\startbuffer[13]
拜了四方:有些畜类初生时因习走路而摔倒,像是朝拜动作,故民间有『拜四方』之说。
\stopbuffer


\startbuffer[14]
斗府:指斗宿星宫。
\stopbuffer


\startbuffer[15]
服饵水食:大意指吃了人间的水和食物。饵,泛指食物,也指服食、吃。
\stopbuffer


\startbuffer[16]
甲子:中国传统干支纪年中,每六十年为一个循环,首年为甲子。这里引申为时间。
\stopbuffer


\startbuffer[17]
邷(wǎ)麽儿:一种儿童游戏。把瓦片磨成许多小块抓着玩。邷,抓。
\stopbuffer


\startbuffer[18]
跑:同『刨』。
\stopbuffer


\startbuffer[19]
造字蜡:即蝗虫,也作『八蜡』。
\stopbuffer


\startbuffer[20]
帓(wà):同『袜』,袜子。
\stopbuffer


\startbuffer[21]
上溜头:上游,河川的上水方向。
\stopbuffer


\startbuffer[22]
挈(qiè):带,领。
\stopbuffer


\startbuffer[23]
『千寻』句:形容瀑布极高而声势浩大。古以八尺为一寻,千寻形容极高或极长。雷,形容如雷的水声。
\stopbuffer


\startbuffer[24]
潺湲(chán yuán):形容河水慢慢流动的样子。
\stopbuffer


\startbuffer[25]
石窍:石洞。
\stopbuffer


\startbuffer[26]
乳窟:生长有钟乳石的洞穴。
\stopbuffer


\startbuffer[27]
罍(léi):一种盛酒器。
\stopbuffer


\startbuffer[28]
碣(jié):圆顶碑石。
\stopbuffer


\startbuffer[29]
造化:福分,运气。
\stopbuffer


\startbuffer[30]
人而无信,不知其可:语出《论语》,意为一个人若没有信用,不知道他还可以做什么。
\stopbuffer


\startbuffer[31]
序齿排班:按年龄长幼排定等第次序。齿,年龄。
\stopbuffer


\startbuffer[32]
三阳交泰:原为《周易》卦象,后成为一种象征好运的祝福语。《周易》以月份匹配卦象,因正月为泰卦,此时三阳生,故名。也作『三阳开泰』。古人认为此时冬去春来,阴消阳长,有吉亨之象。
\stopbuffer


\startbuffer[33]
合契同情:意气相投,同心共志。合契,融洽,意气相投。同情,犹同心,一心。
\stopbuffer


\startbuffer[34]
黄精:一种草药,又名老虎姜、鸡头参,中医以根茎入药。
\stopbuffer


\startbuffer[35]
通臂猿猴:长臂猿。通臂,长臂。一说为传说中的猿,两臂相通,一臂缩短时另一臂伸长。
\stopbuffer


\startbuffer[36]
道心:佛教语。菩提心,悟道之心。
\stopbuffer


\startbuffer[37]
五虫:古人把动物分为五类,即羽虫(禽类)、毛虫(兽类)、甲虫(昆虫类)、鳞虫(鱼类)、裸虫(人类),合称『五虫』。虫,动物的通称。
\stopbuffer


\startbuffer[38]
名色:名目,名称。
\stopbuffer


\startbuffer[39]
阎浮世界:指人世间。阎浮,『阎浮提』的省称,即『南赡部洲』。
\stopbuffer


\startbuffer[40]
劚(zhǔ):挖。
\stopbuffer


\startbuffer[41]
缃苞:浅浅的黄色一丛。缃,浅黄色。
\stopbuffer


\startbuffer[42]
酲(chéng):酒醉后神志不清。解酲即解酒。
\stopbuffer


\startbuffer[43]
榧柰(fěi nài):榧,坚果类,两端尖,仁可食。柰,果木名,俗称柰子,苹果的一种。
\stopbuffer


\startbuffer[44]
茯苓:一种寄生于松根的菌类。薏苡:草本植物,其仁含淀粉,可食用、酿酒、入药。
\stopbuffer


\startbuffer[45]
穵(wā):同『挖』。
\stopbuffer


\startbuffer[46]
造字(qiā)虎:吓唬人的人。
\stopbuffer


\startbuffer[47]
市廛(chán):指店铺集中的市区。
\stopbuffer


\startbuffer[48]
耽:沉湎。
\stopbuffer


\startbuffer[49]
取勾:即勾取,传讯、提审犯人,借指鬼卒勾摄人的魂灵。
\stopbuffer


\startbuffer[50]
欺:遮蔽。
\stopbuffer


\startbuffer[51]
巉(chán)崖:高耸险峻的山崖。巉,险峻陡峭。
\stopbuffer


\startbuffer[52]
柯烂:斧柄朽烂。典出晋人伐木遇仙的故事,说晋代王质在山中伐木时遇下棋的童子,送他一枚枣核大小的食物。王质口含而不觉饥,片刻后,手中斧柄已烂,回到家中才发现物是人非,已去数年。柯,斧子的柄。
\stopbuffer


\startbuffer[53]
丁丁(zhēng):指伐木声。
\stopbuffer


\startbuffer[54]
箨(tuò):竹笋壳。包在新竹外面的皮叶,竹长成逐渐脱落。
\stopbuffer


\startbuffer[55]
爽:通『造字』,草鞋上的绞绳。
\stopbuffer


\startbuffer[56]
衠(zhūn)钢:纯钢。衠,纯、真。
\stopbuffer


\startbuffer[57]
火麻:即大麻,又名线麻。纤维长而坚韧,可纺线制绳索、织渔网,或织麻布、造纸等。
\stopbuffer


\startbuffer[58]
起手:即稽首。古代跪拜礼的一种,也称道士举手向人行礼的动作。
\stopbuffer


\startbuffer[59]
不当人:也作『不当人子』。不当价。犹言罪过。
\stopbuffer


\startbuffer[60]
斫(zhuó):砍。
\stopbuffer


\startbuffer[61]
货:卖。
\stopbuffer


\startbuffer[62]
籴(dí):买进谷物。
\stopbuffer


\startbuffer[63]
此处原注『灵台方寸,心也』。
\stopbuffer


\startbuffer[64]
此处原注『斜月象一勾,三星象三点也。是心。言学仙不必在远,只在此心』。
\stopbuffer


\startbuffer[65]
九皋:曲折深远的沼泽。
\stopbuffer


\startbuffer[66]
髽髻(zhuā jì):梳在头顶两旁或脑后的发髻。
\stopbuffer


\startbuffer[67]
大觉金仙:因宋徽宗时期废除佛教而对释迦牟尼佛的改称。这里形容须菩提祖师。
\stopbuffer


\startbuffer[68]
历劫:佛教语。经历宇宙的成毁。后形容经历各种灾难。劫,宇宙在时间上的一成一毁。
\stopbuffer


\startbuffer[69]
登界游方:登上天界,游历四方。指周游世界。
\stopbuffer


\startbuffer[70]
化育:产生发育、变化生长。
\stopbuffer


\startbuffer[71]
子系:二字合起来即繁体的『孫』字。
\stopbuffer


\startbuffer[72]
顽空:佛道教语。指一种僵死、没有生机的修炼状态。
\stopbuffer


\startbuffer[73]
三乘:指佛教引导众生达到解脱的三种方法、途径或教义。乘,比喻载众生到达彼岸的教法。
\stopbuffer


\startbuffer[74]
麈(zhǔ)尾:用麈尾做的一类工具,形似扇子,也可用于驱虫、拂尘。麈,鹿类动物。
\stopbuffer


\startbuffer[75]
三家:指儒、释、道。
\stopbuffer


\startbuffer[76]
扶鸾:即扶乩(jī),一种占卜活动。术士制丁字形木架,其直端顶部悬锥下垂。架放在沙盘上,由两人各以食指分扶横木两端,依法请神,木架的下垂部分即在沙上画成文字,作为神的启示。传说神仙来时驾凤乘鸾,故名。
\stopbuffer


\startbuffer[77]
揲蓍(shé shī):数蓍草。古代问卜的一种方式。
\stopbuffer


\startbuffer[78]
朝真降圣:朝真,道教谓朝见真人。降圣,谓帝王诞生。
\stopbuffer


\startbuffer[79]
市语:行话。
\stopbuffer


\startbuffer[80]
『动』字门:此处主要指道家房中术。
\stopbuffer


\startbuffer[81]
红铅:旧时术士称妇女初次月经或其炼取物为红铅。
\stopbuffer


\startbuffer[82]
秋石:丹药名,从童男童女尿液中提炼的春药。
\stopbuffer


\startbuffer[83]
諕(xià):古同『吓』,使人害怕。
\stopbuffer


\startbuffer[84]
报怨:义同『抱怨』。
\stopbuffer


\startbuffer[85]
陪笑:义同『赔笑』。
\stopbuffer


\startbuffer[86]
进步:向前行步。
\stopbuffer


\startbuffer[87]
支更传箭:打更报时。传箭,古用铜壶滴漏计时,通过水平面箭上的刻度来判断时刻。
\stopbuffer


\startbuffer[88]
八极迥无尘:八极,八方极远之地。迥,遥远。无尘,表示超尘脱俗。
\stopbuffer


\startbuffer[89]
溜汾(liù fén):形容水大而湍急。溜,湍急。汾,大。
\stopbuffer


\startbuffer[90]
踡跼(quán jú):屈曲。
\stopbuffer


\startbuffer[91]
至人:道家指超凡脱俗,达到无我境界的人。
\stopbuffer


\startbuffer[92]
六耳:指第三者。
\stopbuffer


\startbuffer[93]
性命:性,指人的精神、意识等层面;命,对应肉体、身体机能等层面。
\stopbuffer


\startbuffer[94]
精炁(qì)神:道教认为先天的精、炁、神,是修炼的基础。炁,同『气』。
\stopbuffer


\startbuffer[95]
丹台:道教指神仙的居处。
\stopbuffer


\startbuffer[96]
工:通『功』。
\stopbuffer


\startbuffer[97]
打混:含混过日,做事不认真。
\stopbuffer


\startbuffer[98]
公案比语:公案,佛教禅宗指前辈祖师的言行范例。比语,比方的话或故事。
\stopbuffer


\startbuffer[99]
外像包皮:外像,佛教语,指显露在外表上的善恶美丑和言语行动。包皮,指表面现象。
\stopbuffer


\startbuffer[100]
泥垣(yuán)宫:即泥丸宫,指人脑。古人称脑神名精根,字泥丸,居于泥丸宫。
\stopbuffer


\startbuffer[101]
和薰金朔风:一年四季,风各有名。春为和风,夏为薰风,秋为金风,冬为朔风。
\stopbuffer


\startbuffer[102]
赑(bì)风:巨风。佛教所称大三灾之一的风灾名。
\stopbuffer


\startbuffer[103]
囟(xìn)门:婴儿头顶骨未合缝的地方,在头顶的前部中央,也叫脑门、顶门。
\stopbuffer


\startbuffer[104]
孤拐面:上部凸出、下部尖削的脸。
\stopbuffer


\startbuffer[105]
嗉袋:猿猴类、啮齿类动物的颊囊,用于暂时存放食物。
\stopbuffer


\startbuffer[106]
天罡(gāng):即北斗七星的柄,道教称北斗丛星中有三十六天罡星,七十二地煞星。
\stopbuffer


\startbuffer[107]
多里捞摸:指往多了捞取,学习。捞摸,向水中探物,泛指寻取。
\stopbuffer


\startbuffer[108]
连扯跟头:连续腾翻的跟头。
\stopbuffer


\startbuffer[109]
扠手:一种礼仪。两手手指交叉放在胸前,表示恭敬。多用在站立、回话时。
\stopbuffer


\startbuffer[110]
跌足:以脚蹬地。
\stopbuffer


\startbuffer[111]
铺兵:宋以后的政府邮政机构称急递铺,其中递送公文的兵卒称铺兵。
\stopbuffer


\startbuffer[112]
报单:旧时向得官、复官、升官和考试得中的人家送去的喜报,或向上呈请的文书。
\stopbuffer


\startbuffer[113]
家(jiè):今作『价』,助词,相当于『地』『的』。
\stopbuffer


\startbuffer[114]
每:古同『们』。
\stopbuffer


\startbuffer[115]
精神:神通。
\stopbuffer


\startbuffer[116]
九幽之处:地下极深之处,亦指阴间。
\stopbuffer


\startbuffer[117]
丢:施展,使出。
\stopbuffer


\startbuffer[118]
家火:家用器具。
\stopbuffer


\startbuffer[119]
睍睆(xiàn huǎn):鸟鸣声清脆圆润。
\stopbuffer


\startbuffer[120]
八节:古代以立春、立夏、立秋、立冬、春分、夏至、秋分、冬至为八节。
\stopbuffer


\startbuffer[121]
坎:八卦之一,代表水。
\stopbuffer


\startbuffer[122]
皂罗:一种色黑质薄的丝织品。
\stopbuffer


\startbuffer[123]
吃人笑:被人笑。吃,介词,被,让。
\stopbuffer


\startbuffer[124]
架子:拳术术语。指拳式动作和套路。
\stopbuffer


\startbuffer[125]
丫裆:指人体两股之间的地方。
\stopbuffer


\startbuffer[126]
筋节:强劲有力。
\stopbuffer


\startbuffer[127]
挦(xián)毛:拔毛。挦,扯,拔(毛发)。
\stopbuffer


\startbuffer[128]
抬鼓弄:儿童游戏。许多人把一个人抬起来翻倒在地。
\stopbuffer


\startbuffer[129]
攒盘:数个小盘围成的大盘。这里喻多人包围殴打一人。
\stopbuffer


\startbuffer[130]
躧(xǐ):踩,踏。
\stopbuffer


\startbuffer[131]
忻(xīn):同『欣』。
\stopbuffer


\startbuffer[132]
仙箓:仙人簿。箓,道教记载上天神名的书。
\stopbuffer


\startbuffer[133]
标:指标枪。
\stopbuffer


\startbuffer[134]
认此犯头:接受挑拨。犯头,挑拨。一说犯头指冒犯的由头,『认此犯头』用于被人无意触怒而发生误会时。
\stopbuffer


\startbuffer[135]
赤尻(kāo)马猴:红屁股猕猴。尻,屁股。马猴,猕猴。
\stopbuffer


\startbuffer[136]
巽(xùn)地:东南方位。巽,八卦之一。
\stopbuffer


\startbuffer[137]
炮云:犹炮火。
\stopbuffer


\startbuffer[138]
『斧钺(yuè)』等:皆兵器名。钺,圆刃,似斧而较大。毛镰,长柄,前端弯刃。挝(zhuā),形似爪子,带钩。简,似鞭,长条形,四棱,无刃,下端有柄。
\stopbuffer


\startbuffer[139]
丫丫叉叉:杂乱交叉的样子。
\stopbuffer


\startbuffer[140]
挝:这里作动词,抓。
\stopbuffer


\startbuffer[141]
『麖麂(jīng jǐ)』等:麖,马鹿;麂,小型鹿;造字(bā),同豝,母野猪;狸,也叫豹猫、狸猫等,形状似猫,圆头大尾;獾(huān),形如狗而足短,穴居,昼伏夜出;狢(hé),即貉,外形似狐,穴居,昼伏夜出;狻猊(suān ní),狮子,传说中的猛兽;青兕(sì),青兕牛,古犀牛类兽名;狡(jiǎo)儿,少壮的狗;獒(áo),高大凶猛的狗。
\stopbuffer


\startbuffer[142]
点卯(mǎo):官衙官员查点到班人数。因在卯时进行,故称。亦泛指点名。
\stopbuffer


\startbuffer[143]
日逐家:逐日地。日逐,每日、每天。
\stopbuffer


\startbuffer[144]
榔槺:指器物长大、笨重,使用不便。
\stopbuffer


\startbuffer[145]
撒两个解(xiè)数:指施展手段。撒,施展,表现。解数,本事,手段。
\stopbuffer


\startbuffer[146]
奉价:作价偿还。
\stopbuffer


\startbuffer[147]
海藏:海中宝库。
\stopbuffer


\startbuffer[148]
鼋鼍(yuán tuó):两种动物名。鼋,大鳖,俗称癞头鼋。鼍,扬子鳄,也称鼍龙、猪婆龙。
\stopbuffer


\startbuffer[149]
鳌:传说中海中能负山的大鳖或大龟。
\stopbuffer


\startbuffer[150]
相趁:相称,匹配。
\stopbuffer


\startbuffer[151]
披挂:盔甲。
\stopbuffer


\startbuffer[152]
奉承:奉送,馈赠。
\stopbuffer


\startbuffer[153]
赊三不敌见二:赊卖三文钱,不如现结二文钱。比喻空有的好处再多,也不如眼前的实惠。
\stopbuffer


\startbuffer[154]
随高就低:犹言可高可低,随便怎样。
\stopbuffer


\startbuffer[155]
聒噪:江湖上打招呼用的习惯语。犹言打扰了,对不起。多见于早期白话作品。
\stopbuffer


\startbuffer[156]
禁动:摇动,撼动。
\stopbuffer


\startbuffer[157]
可可的:恰好,正巧。
\stopbuffer


\startbuffer[158]
揌(sāi):同『塞』。
\stopbuffer


\startbuffer[159]
揝(zuàn):古同『攥』。抓,握。
\stopbuffer


\startbuffer[160]
马、流:取自词语『马流』,指猴子。源自北方民族的语言。
\stopbuffer


\startbuffer[161]
造字狨(yù róng):造字,即禺,传说中的一种猴,似猕猴而大,赤目长尾。狨,即金丝猴。
\stopbuffer


\startbuffer[162]
斝(jiǎ):青铜酒器,后借指酒杯、茶杯。
\stopbuffer


\startbuffer[163]
庭闱:内舍。
\stopbuffer


\startbuffer[164]
赏劳:犒赏,慰劳。
\stopbuffer


\startbuffer[165]
欹(qī):歪斜,倾斜。亦通『倚』,倚靠,斜靠。
\stopbuffer


\startbuffer[166]
朦胧:犹言糊涂。
\stopbuffer


\startbuffer[167]
勾死人:即勾死鬼,传说中勾摄人的灵魂的鬼。
\stopbuffer


\startbuffer[168]
牛头鬼:与『马面鬼』皆为传说中地狱的鬼卒。
\stopbuffer


\startbuffer[169]
森罗殿:传说阴间阎罗王所居之殿。
\stopbuffer


\startbuffer[170]
十殿冥王:佛教传说冥界主管地狱的十个阎王。
\stopbuffer


\startbuffer[171]
官差吏差,来人不差:不管官吏怎样不对,奉命执行的人没有错。
\stopbuffer


\startbuffer[172]
十类:五仙(天仙、地仙、神仙、人仙、鬼仙)和五虫合称十类。
\stopbuffer


\startbuffer[173]
掭(tiàn):用笔调蘸墨汁。
\stopbuffer


\startbuffer[174]
捽(zuó):投,摔。
\stopbuffer


\startbuffer[175]
纥繨(gē da):同『疙瘩』。球状或块状的东西,多指用绳线等物所结成的。
\stopbuffer


\startbuffer[176]
躘踵(lóng zhǒng):踉跄欲跌貌。
\stopbuffer


\startbuffer[177]
丘弘济真人:或为道教全真派祖师丘处机及其弟子李志常(封号『弘济』)捏合而来。真人,道家称修真得道的人。
\stopbuffer


\startbuffer[178]
葛仙翁天师:葛仙翁,三国方士葛玄的别号,后世奉为天师。
\stopbuffer


\startbuffer[179]
赍(jī)奉:携带。赍,携带。
\stopbuffer


\startbuffer[180]
冒渎(dú):冒犯,亵渎。多用作谦辞。
\stopbuffer


\startbuffer[181]
太白长庚星:即太白金星。长庚,指傍晚时出现在西方天空的金星,亦名太白星。
\stopbuffer


\startbuffer[182]
一角:一包,一份,一封。多用于文书等物。
\stopbuffer


\startbuffer[183]
空退:指让客人无所受用而归。犹言怠慢。
\stopbuffer


\startbuffer[184]
增长天王:佛教四天王之一。四天王为东方持国天王,身白色,持琵琶;南方增长天王,身青色,执宝剑;西方广目天王,身红色,执罥索;北方多闻天王,身绿色,执宝叉。
\stopbuffer


\startbuffer[185]
天丁:即天兵。
\stopbuffer


\startbuffer[186]
觌(dí)面:当面,迎面。
\stopbuffer


\startbuffer[187]
注:记载。
\stopbuffer


\startbuffer[188]
持铣(xiǎn)拥旄(máo):铣,两端用金装饰的弓,一说指铲状的仪仗。旄,古代用牦牛尾做竿饰的旗子,仪仗的一种。
\stopbuffer


\startbuffer[189]
斗口:指北斗星的斗柄。
\stopbuffer


\startbuffer[190]
吻兽:传统建筑屋脊上用来辟邪的兽形装饰。
\stopbuffer


\startbuffer[191]
卸:凋谢。
\stopbuffer


\startbuffer[192]
绛(jiàng)纱:红纱。
\stopbuffer


\startbuffer[193]
紫绶金章:指古代高级官员的服饰。紫绶,紫色丝带,古代高级官员用作印组,或作服饰。金章,金质官印,也称高级官服。
\stopbuffer


\startbuffer[194]
三曹:道教称天官大帝、地官大帝、水官大帝各自管辖的官署。曹,古代的官署、部门。
\stopbuffer


\startbuffer[195]
太乙丹:道教炼丹托名太乙神所传。
\stopbuffer


\startbuffer[196]
金乌:古代神话传说太阳中有三足乌,因用为太阳的代称。
\stopbuffer


\startbuffer[197]
唱个大喏(rě):唱喏,古代男子所行之礼,叉手行礼,同时出声致敬。
\stopbuffer


\startbuffer[198]
文选武选:主管文武官任命的部门。明代吏部设文选司和武选司。
\stopbuffer


\startbuffer[199]
除授:拜官授职。
\stopbuffer


\startbuffer[200]
武曲星君:星名。民间传说其主掌人间武事。
\stopbuffer


\startbuffer[201]
弼马温:『辟马瘟』的谐音,是作者的戏言。古人认为猴能辟马之瘟。
\stopbuffer


\startbuffer[202]
『骅骝骐骥』段:皆历史上或传说中的骏马名。如骅骝、騄駬等,为周穆王所养八骏;弥景、飞翮等为秦始皇之马;紫燕、逸飘等,为汉武帝所养九逸;龙媒、大宛,为汉时西域名马等。
\stopbuffer


\startbuffer[203]
抿耳攒蹄:合拢耳朵,并拢四蹄。
\stopbuffer


\startbuffer[204]
堂尊:明清时县里属吏对知县的尊称。
\stopbuffer


\startbuffer[205]
尪羸(wāng léi):瘦弱。
\stopbuffer


\startbuffer[206]
公案:官府处理公事时用的几案。
\stopbuffer


\startbuffer[207]
同寮:即同僚,同朝或同官署做官的人。寮,古同『僚』,官。
\stopbuffer


\startbuffer[208]
张天师:东汉人张道陵,道教的实际创始人,后被尊为『天师』,世称『张天师』。
\stopbuffer


\startbuffer[209]
丹墀(chí):指宫殿的赤色台阶或赤色地面。
\stopbuffer


\startbuffer[210]
查:宽。
\stopbuffer


\startbuffer[211]
身才:即身材。『才』通『材』。
\stopbuffer


\startbuffer[212]
尖嘴咨牙:义同『尖嘴龇牙』。
\stopbuffer


\startbuffer[213]
虀(jī)粉:粉末,碎屑。
\stopbuffer


\startbuffer[214]
时间:立即,马上。
\stopbuffer


\startbuffer[215]
会家不忙:行家对自己熟悉的事应付自如,不会慌乱。会家,行家。
\stopbuffer


\startbuffer[216]
嗳(ǎi):吐。
\stopbuffer


\startbuffer[217]
扢扠(gē chā):象声词。亦作『扢揸』『扢咋』。
\stopbuffer


\startbuffer[218]
总角:古时儿童束发为两结,向上分开,形状如角,故称总角。
\stopbuffer


\startbuffer[219]
苫(shàn):用席、布等遮盖。
\stopbuffer


\startbuffer[220]
斗牛:二十八宿中的斗宿和牛宿。此处泛指天宫。
\stopbuffer


\startbuffer[221]
演:欺骗,迷惑。
\stopbuffer


\startbuffer[222]
恁(nèn)的:如此,这样。
\stopbuffer


\startbuffer[223]
拈(niān):习弄,摆弄。
\stopbuffer


\startbuffer[224]
工干官:古代负责制作器物的官员。
\stopbuffer


\startbuffer[225]
注名:谓记名于名册。
\stopbuffer


\startbuffer[226]
三清:道教最高神,玉清境洞元始天尊、上清境洞灵宝天尊、太清境洞道德天尊的合称。
\stopbuffer


\startbuffer[227]
四帝:即道教『四御』,昊天金阙至尊玉皇大帝、勾陈上宫南极天皇大帝、中天紫微北极大帝、承天效法后土皇地祇的合称。
\stopbuffer


\startbuffer[228]
九曜星:指金、木、水、火、土、太阳、太阴、计都和罗睺九位星君。
\stopbuffer


\startbuffer[229]
五方五老:道教称东南西北中的五位神灵,分居五岳,掌管五行。《西游记》中指东方崇恩圣帝、南方南极观音、西方如来佛祖、北方北极玄灵、中央黄极黄角大仙。
\stopbuffer


\startbuffer[230]
河汉:指银河。
\stopbuffer


\startbuffer[231]
许旌阳:传说中的仙人。相传为晋道士,曾为旌阳令,后弃官周游江湖。
\stopbuffer


\startbuffer[232]
俯囟:俯首。
\stopbuffer


\startbuffer[233]
执事:差事,工作。
\stopbuffer


\startbuffer[234]
酡(tuó)颜:饮酒脸红貌。亦泛指脸红。
\stopbuffer


\startbuffer[235]
上八洞:八洞,道教谓神仙所居住的洞府,有上八洞、中八洞、下八洞诸称。
\stopbuffer


\startbuffer[236]
九垒:道教将大地分为九层,称为九垒,每一垒皆有神。
\stopbuffer


\startbuffer[237]
睖睖睁睁:眼睛直视发呆。
\stopbuffer


\startbuffer[238]
赚哄:骗取,哄骗。
\stopbuffer


\startbuffer[239]
翥(zhù):飞举,鸟向上飞。
\stopbuffer


\startbuffer[240]
扆(yǐ):指置于门窗之间的屏风。
\stopbuffer


\startbuffer[241]
八宝:也称八吉祥,佛教象征吉祥的八件器物,一般为法轮、法螺、宝伞、白盖、莲花、宝瓶、金鱼、盘长结。这里指八宝的纹样。
\stopbuffer


\startbuffer[242]
醪(láo):浊酒。
\stopbuffer


\startbuffer[243]
酕醄(máo táo):大醉的样子。
\stopbuffer


\startbuffer[244]
残步:谓途中顺路(前往他处)。
\stopbuffer


\startbuffer[245]
燃灯古佛:佛教过去世诸佛之一。佛经说他生时周身光明如灯,故名。
\stopbuffer


\startbuffer[246]
咨牙俫嘴:义同『龇牙咧嘴』。
\stopbuffer


\startbuffer[247]
一周天:指一定时间的循环。有多种解释。
\stopbuffer


\startbuffer[248]
四个大天师:《西游记》中指张道陵、葛玄、许逊、丘弘济。
\stopbuffer


\startbuffer[249]
五方揭谛:佛教五方守护大力神。
\stopbuffer


\startbuffer[250]
四值功曹:民间信仰和道教所奉的的天庭中值年、值月、值日、值时四神。
\stopbuffer


\startbuffer[251]
五瘟:又称五鬼,民间传说中的瘟神。
\stopbuffer


\startbuffer[252]
六丁六甲:道教神名。六丁(丁卯、丁巳、丁未、丁酉、丁亥、丁丑)为阴神,六甲(甲子、甲戌、甲申、甲午、甲辰、甲寅)为阳神,为天帝所役使,能行风雷,制鬼神。
\stopbuffer


\startbuffer[253]
公然:犹完全,全然。
\stopbuffer


\startbuffer[254]
老大:很,非常。
\stopbuffer


\startbuffer[255]
浪言:大言,大话。
\stopbuffer


\startbuffer[256]
楮(chǔ)白枪:也称出白枪,精钢制成的枪。
\stopbuffer


\startbuffer[257]
的(dí)实:真实,确实。
\stopbuffer


\startbuffer[258]
提铃喝号:指夜间警戒之事。提铃,古时从傍晚至拂晓定时摇铃,以示太平无事。
\stopbuffer


\startbuffer[259]
虚日鼠:古代术数家以二十八宿、七曜(金、木、水、火、土、日、月)和各种动物名排列组合,称禽星。如此处『虚』为二十八宿之一,『日』为七曜之一。
\stopbuffer


\startbuffer[260]
阴手棍:与下文『夹枪棒』皆为棍法招式。
\stopbuffer


\startbuffer[261]
叵(pǒ)耐:不可容忍,可恨。
\stopbuffer


\startbuffer[262]
赍调:奉旨调遣。
\stopbuffer


\startbuffer[263]
当:人称后缀。
\stopbuffer


\startbuffer[264]
行:副词。将,要。
\stopbuffer


\startbuffer[265]
蹭蹬:倒霉。
\stopbuffer


\startbuffer[266]
只情:只管,只顾。
\stopbuffer


\startbuffer[267]
造字鹰儿:一种体型较小、善捕雀的鹰。
\stopbuffer


\startbuffer[268]
鹚老:即鸬鹚,俗称『兹老』。一说为秃鹙。
\stopbuffer


\startbuffer[269]
海鹤:海鸟名。或说即江鸥。
\stopbuffer


\startbuffer[270]
嗛(xián):古同『衔』,用嘴含。
\stopbuffer


\startbuffer[271]
淬:指浸入或沉入水中。
\stopbuffer


\startbuffer[272]
青庄:即苍鹭,又名灰鹭,毛青灰色。
\stopbuffer


\startbuffer[273]
打个花:打水激起浪花。
\stopbuffer


\startbuffer[274]
花鸨(bǎo):一种像雁而背部有花色纹的鸟。常比喻淫贱之人,或借指妓女。
\stopbuffer


\startbuffer[275]
木木樗樗(chū):形容呆呆的样子。樗,臭椿,也比喻无用之材。
\stopbuffer


\startbuffer[276]
拢傍:接近。拢,靠近。
\stopbuffer


\startbuffer[277]
弄喧:弄玄虚,耍花招。
\stopbuffer


\startbuffer[278]
才自:方言。刚才。又作『才子』。
\stopbuffer


\startbuffer[279]
锟(kūn)钢抟(tuán)炼:用优质赤铁炼成的钢锻炼。锟,一种优质赤铁。抟,聚集。
\stopbuffer


\startbuffer[280]
化胡为佛:史载老子晚年西出函关,不知所终。传说他在印度化为释迦牟尼,教化当地百姓。
\stopbuffer


\startbuffer[281]
掼(guàn):扔,掷。
\stopbuffer


\startbuffer[282]
滴流流:飞旋、旋转的样子。
\stopbuffer


\startbuffer[283]
亡人:死人。骂人之语。
\stopbuffer


\startbuffer[284]
家长:一家之主。
\stopbuffer


\startbuffer[285]
琵琶骨:即锁骨。穿琵琶骨是一种古老的刑罚,受刑者双足可以走动,但双手因疼痛无法发力,用于限制人犯的反抗能力。
\stopbuffer


\startbuffer[286]
等时:义同『登时』。
\stopbuffer


\startbuffer[287]
罔极:无穷尽。罔,无,没有。
\stopbuffer


\startbuffer[288]
规箴:劝勉告诫。
\stopbuffer


\startbuffer[289]
刳(kū):剖开;挖空。
\stopbuffer


\startbuffer[290]
雷榍:传说中雷公打雷的法器。榍,同『楔』。
\stopbuffer


\startbuffer[291]
文武火:指文火与武火。文火,火力小而弱;武火,火力大而猛。
\stopbuffer


\startbuffer[292]
八卦:《周易》中八个基本图形,象征天、水、山、雷、风、火、地、泽八种自然现象,配八方。
\stopbuffer


\startbuffer[293]
煼(chǎo):熏。
\stopbuffer


\startbuffer[294]
侮:古同『捂』。后文又作『仵』。
\stopbuffer


\startbuffer[295]
如如:佛教语。指永恒存在的真如。引申为永存,常在。
\stopbuffer


\startbuffer[296]
太虚:谓空寂玄奥之境。也指天空,宇宙。
\stopbuffer


\startbuffer[297]
心即猿猴:佛教语有『心猿』一词,比喻攀缘外境、浮躁不安之心有如猿猴。
\stopbuffer


\startbuffer[298]
双林:也称『娑罗双树』,传说中释迦牟尼涅槃的地方。
\stopbuffer


\startbuffer[299]
佑圣真君:即真武大帝,北极四圣之一。四圣分别为天蓬大元帅真君、天猷副元帅真君(又名天佑副帅)、翊圣保德真君(又名黑煞将军)、灵应佑圣真君(又名真武将军)。明代民间对真武的崇拜高涨,升格为真武大帝。
\stopbuffer


\startbuffer[300]
佐使王灵官:传说佑圣真君有五百子弟,为五百灵官,王灵官为之首。佐使,官名。
\stopbuffer


\startbuffer[301]
鏖(áo)战:激烈地战斗,苦战。
\stopbuffer


\startbuffer[302]
礼佛三匝:向佛礼拜,绕行三圈。佛教特有的礼拜方式。
\stopbuffer


\startbuffer[303]
炼魔:降伏魔障。指道家克制欲念。
\stopbuffer


\startbuffer[304]
风车子:传说中驾风而行的车子。
\stopbuffer


\startbuffer[305]
双毫笔:也称兼毫笔。毛笔用毛分狼毫、羊毫,二者皆有称兼毫。
\stopbuffer


\startbuffer[306]
『十都』等:即传说中的各路神仙。六司即南斗六司、南斗六星君;七元即北斗七元解厄星君;八极指八方之神;十都指十大洞天之神。
\stopbuffer


\startbuffer[307]
七宝:佛教中的七种宝物,各佛经说法不一,一般有金﹑银﹑砗磲﹑珊瑚、琉璃等。
\stopbuffer


\startbuffer[308]
玄虚一应灵通:指天上的各路神仙。玄虚,指天空。
\stopbuffer


\startbuffer[309]
毛女:传说中得道于华山的仙女。
\stopbuffer


\startbuffer[310]
更差:更替。
\stopbuffer


\startbuffer[311]
辐辏:聚会。
\stopbuffer


\startbuffer[312]
恒沙:恒河沙数,形容数量之多。
\stopbuffer


\startbuffer[313]
九品花:九品莲台。佛教中,莲台依修行深浅有九等之别。
\stopbuffer


\startbuffer[314]
赊:多。
\stopbuffer


\startbuffer[315]
交梨、火枣:皆为传说中的仙果,吃了可以羽化成仙。
\stopbuffer


\startbuffer[316]
展挣:挣扎。
\stopbuffer


\startbuffer[317]
弄乖:耍手段;卖乖。
\stopbuffer


\startbuffer[318]
禅关:佛教语。比喻悟彻佛教教义必须越过的关口。
\stopbuffer


\startbuffer[319]
『毛吞』二句:大意为一个毛孔能吸四海之水,一粒芥子可纳须弥之山。比喻微小事物中蕴含宏大教义。
\stopbuffer


\startbuffer[320]
『金色』句:指佛祖拈花示众,众皆默然,唯迦叶尊者微笑,故佛祖传以禅宗的故事。头陀,梵文音译,意为去掉尘垢烦恼,用以代称僧人。传说迦叶身皆金色,故称金色头陀。
\stopbuffer


\startbuffer[321]
十地:佛教语。佛家谓菩萨修行所经历的十个境界。不同经论对此有不同描述。
\stopbuffer


\startbuffer[322]
曹溪:曹溪宝林寺,佛教禅宗六祖慧能曾在此演法。
\stopbuffer


\startbuffer[323]
鹫岭:鹫山,亦称灵鹫山、灵山。如来讲法之处。
\stopbuffer


\startbuffer[324]
龙王三宝:佛经中有海龙王化作金、银、琉璃三宝阶引导佛、菩萨等入龙宫的记载。
\stopbuffer


\startbuffer[325]
般若:佛教语。智慧。
\stopbuffer


\startbuffer[326]
殄(tiǎn)伏:消灭,降服。殄,灭绝,绝尽。
\stopbuffer


\startbuffer[327]
白虹四十二道:传说释迦牟尼将涅槃时,西方有白虹四十二道。
\stopbuffer


\startbuffer[328]
庆云:五色云。古人以为喜庆、吉祥之气。
\stopbuffer


\startbuffer[329]
世尊:与下文的『法王』皆为佛教对佛的尊称。
\stopbuffer


\startbuffer[330]
沙界:恒河沙数的世界。
\stopbuffer


\startbuffer[331]
盂(yú)兰盆会:原佛教节日,定于每年农历七月十五。典出目连僧求佛救母故事。盂兰,梵文音译,意为苦厄之状。盆,指盛供品的器皿。盂兰盆意为解救苦厄。
\stopbuffer


\startbuffer[332]
概众:众,众人。
\stopbuffer


\startbuffer[333]
五蕴《楞严》:五蕴,佛教指人的色、受、想、行、识五种刹那变化的成分,这五种成分的暂时结合形成了个我。《楞严》,即《楞严经》,其中讲述了五蕴等禅法要义。
\stopbuffer


\startbuffer[334]
落索:同『络索』。链子。
\stopbuffer


\startbuffer[335]
灵汉:云汉,高空。
\stopbuffer


\startbuffer[336]
栴(zhān)檀:即檀香。
\stopbuffer


\startbuffer[337]
沙碛(qì):沙漠。
\stopbuffer


\startbuffer[338]
纳头:低头。
\stopbuffer


\startbuffer[339]
豕(shǐ)、彘(zhì):皆指猪,彘本指大猪。
\stopbuffer


\startbuffer[340]
可:语义不详。一说为方言,同『嗑』,咬。有版本作『打』。
\stopbuffer


\startbuffer[341]
倒蹅门:即倒插门,入赘。
\stopbuffer


\startbuffer[342]
赡身:养活自己。
\stopbuffer


\startbuffer[343]
本等:本分。
\stopbuffer


\startbuffer[344]
五荤三厌:五荤,佛教忌食的五种刺激性气味的蔬菜。三厌,道教认为食之不能成仙的食物,分天厌、地厌、水厌。
\stopbuffer


\startbuffer[345]
迦持:佛教戒律。
\stopbuffer


\startbuffer[346]
京国:京城,国都。
\stopbuffer


\startbuffer[347]
城隍:守护城池的神。
\stopbuffer


\startbuffer[348]
社令:即土地神。
\stopbuffer


\startbuffer[349]
己巳:历史上这一年为己亥年。第九回、第十二回同。
\stopbuffer


\startbuffer[350]
门闾(lǘ):家门,家庭,门庭。
\stopbuffer


\startbuffer[351]
细乐:指管弦之乐。与锣鼓等音响大的音乐相对而言。
\stopbuffer


\startbuffer[352]
宾人赞礼:司仪宣唱仪节。宾人,宾相,举行典礼时导行仪节的人,也专指举行婚礼时赞礼者。赞礼,举行典礼时司仪宣唱仪节,叫人行礼。
\stopbuffer


\startbuffer[353]
素手:洁白的手。多形容女子之手。
\stopbuffer


\startbuffer[354]
忝(tiǎn):辱,有愧于。表自谦。
\stopbuffer


\startbuffer[355]
行程:上路,就道。
\stopbuffer


\startbuffer[356]
䁪(zhǎn)斬眼:眨眼。
\stopbuffer


\startbuffer[357]
梢水:船夫。也作『梢子』。
\stopbuffer


\startbuffer[358]
须索:必须、一定。
\stopbuffer


\startbuffer[359]
叨受:犹承受。自谦。
\stopbuffer


\startbuffer[360]
都领:总领,总管。
\stopbuffer


\startbuffer[361]
区处:处理,筹划安排。
\stopbuffer


\startbuffer[362]
吏书、门皂:官府的文书、看门的差役。
\stopbuffer


\startbuffer[363]
根脚:底细,来历。
\stopbuffer


\startbuffer[364]
开载:逐一记载。
\stopbuffer


\startbuffer[365]
无生:佛教语。谓没有生灭,不生不灭。
\stopbuffer


\startbuffer[366]
禀阴阳而资五行:大意为承受阴阳之气而生,依赖五行之材而长。
\stopbuffer


\startbuffer[367]
方丈:指僧尼长老、住持的居室。也代指寺院主持。
\stopbuffer


\startbuffer[368]
头顶香盆:头顶焚香之盆。表示对神明或人的尊敬。
\stopbuffer


\startbuffer[369]
抄化:募化,求乞。
\stopbuffer


\startbuffer[370]
中年:此处指中期、中途。
\stopbuffer


\startbuffer[371]
完纳:交纳。
\stopbuffer


\startbuffer[372]
斋衬:施舍的财物。
\stopbuffer


\startbuffer[373]
心香:佛教语。谓中心虔诚,如供佛之焚香。此处指虔诚地上香。
\stopbuffer


\startbuffer[374]
分俵(biào):分施,分给。俵,把东西分给人。
\stopbuffer


\startbuffer[375]
昏了眼:目不明,昏花。这里是眼瞎的意思。
\stopbuffer


\startbuffer[376]
教场:古代操练和检阅军队的场地。
\stopbuffer


\startbuffer[377]
靦(tiǎn)颜:厚颜。靦,同『腼』。
\stopbuffer


\startbuffer[378]
以盛衰改节:因为富贵贫贱的变化而改变节操。
\stopbuffer


\startbuffer[379]
木驴:刑具。为装有轮轴的木架,载犯人示众并处死。
\stopbuffer


\startbuffer[380]
市曹:市内商业集中之处。古代常于此处决人犯。
\stopbuffer


\startbuffer[381]
如意珠:相传用佛舍利制成的佛珠。一说出于龙王脑中,常用作龙王的首饰。
\stopbuffer


\startbuffer[382]
走盘珠:指可以在盘中滚动的正圆形珍珠。
\stopbuffer


\startbuffer[383]
鲛绡十端:鲛绡,传说中鲛人织的绡。绡,生丝、生丝织物。端,帛类的长度单位。
\stopbuffer


\startbuffer[384]
安禅:佛教语。指静坐入定,俗称打坐。
\stopbuffer


\startbuffer[385]
抱虎而眠、袖蛇而走:比喻名利场中处境危险,随时可能遭害。
\stopbuffer


\startbuffer[386]
鸡头:即鸡头米,亦称芡实,一种水生植物的种子。
\stopbuffer


\startbuffer[387]
黄练芽:即黄连芽。
\stopbuffer


\startbuffer[388]
茆(máo)舍数椽(chuán):数间茅舍。『茆』同『茅』。椽,指房屋的间数。
\stopbuffer


\startbuffer[389]
待漏:百官清晨入朝等待朝拜天子,谓之『待漏』。漏,古代计时器。
\stopbuffer


\startbuffer[390]
山妻:隐士之妻。后多用为自称其妻的谦词。
\stopbuffer


\startbuffer[391]
相狎(xiá):彼此亲昵、接近。
\stopbuffer


\startbuffer[392]
檀板:乐器名。檀木制的拍板。
\stopbuffer


\startbuffer[393]
唳鸿:鸣叫的鸿雁。
\stopbuffer


\startbuffer[394]
颠风:疯癫,颠狂。
\stopbuffer


\startbuffer[395]
爊(āo):把食物埋在灰火中煨熟,或用文火久煮。
\stopbuffer


\startbuffer[396]
冲:山间平地。
\stopbuffer


\startbuffer[397]
僭(jiàn)先:越礼占先。
\stopbuffer


\startbuffer[398]
余丁:充军役以外的丁口。
\stopbuffer


\startbuffer[399]
阃(kǔn)令:军令,将令。阃,城郭的门槛,借指将军。
\stopbuffer


\startbuffer[400]
惫懒:无赖。
\stopbuffer


\startbuffer[401]
捉摸:料想,揣测。
\stopbuffer


\startbuffer[402]
袖传一课:袖中课。一种占卜方式,通过计点手指节来占卜凶吉,即『掐指一算』。
\stopbuffer


\startbuffer[403]
罗襕服:古代丝制公服。按官品高下,有紫襕、绯襕、绿襕等。
\stopbuffer


\startbuffer[404]
合局:指生辰相合。
\stopbuffer


\startbuffer[405]
日犯岁君:岁君,即太岁神。古代占卜,年月日干支与太岁神所在干支相冲,则为凶,其中『日犯太岁』最为不吉。
\stopbuffer


\startbuffer[406]
宝鸭:指鸭形香炉。
\stopbuffer


\startbuffer[407]
《火珠林》:相传为唐宋年间麻衣道者所著的占卜书。
\stopbuffer


\startbuffer[408]
郭璞:晋代玄学家,善于占卜。
\stopbuffer


\startbuffer[409]
台政新经:国家司天台颁布的新的历法。
\stopbuffer


\startbuffer[410]
一盘子午:即子午盘,罗盘。
\stopbuffer


\startbuffer[411]
作戏:作耍,开玩笑。
\stopbuffer


\startbuffer[412]
掉嘴口讨春:耍嘴皮子算命。掉嘴口,掉弄唇舌,耍嘴皮子。讨春,算命,打卦。
\stopbuffer


\startbuffer[413]
点札:点名调遣。
\stopbuffer


\startbuffer[414]
人曹官:指主管人间事物的人官。人曹,人群。
\stopbuffer


\startbuffer[415]
业龙:孽龙,恶龙。
\stopbuffer


\startbuffer[416]
三象:日、月、星合称三象。
\stopbuffer


\startbuffer[417]
爇(ruò):烧。
\stopbuffer


\startbuffer[418]
慧剑:佛教语。谓能斩断一切烦恼的智慧。此处为这种观念的具象化运用。
\stopbuffer


\startbuffer[419]
便殿:正殿以外的别殿,古时帝王休息、消闲之处。
\stopbuffer


\startbuffer[420]
对着(zhāo):对弈。着,下棋时下一子或走一步。
\stopbuffer


\startbuffer[421]
一递一:一下一下轮流、交替的动作。
\stopbuffer


\startbuffer[422]
塌伏:倒伏,趴伏。
\stopbuffer


\startbuffer[423]
霜锋:明亮锐利的锋刃。
\stopbuffer


\startbuffer[424]
箝(qián)口:闭口。
\stopbuffer


\startbuffer[425]
具告:备文告发。
\stopbuffer


\startbuffer[426]
虚而又数(shuò):中医术语。虚,指脉搏虚弱。数,指脉搏急促。
\stopbuffer


\startbuffer[427]
十动一代:中医术语。十次脉搏中有一次紊乱,是病危的症状。
\stopbuffer


\startbuffer[428]
着然:着实,实在。
\stopbuffer


\startbuffer[429]
金瓜钺斧:金瓜,古代卫士所执的一种兵仗,棒端呈瓜形,铜制,金色。钺斧,圆刃大斧。
\stopbuffer


\startbuffer[430]
狮蛮:古代武官腰带钩上饰有狮子、蛮王的形象,因以指武官腰带。
\stopbuffer


\startbuffer[431]
氅(chǎng):大衣,外套。
\stopbuffer


\startbuffer[432]
亚赛垒荼:指容貌好似郁垒和神荼。亚赛,类似,好似。垒荼,郁垒、神荼,上古传说中的神人,后世以为门神。
\stopbuffer


\startbuffer[433]
酆(fēng)都:道教传说中罗酆山洞天六宫为鬼神治事之所,后人以四川省酆都附会,认为是阴曹地府所在之处。
\stopbuffer


\startbuffer[434]
梓宫:皇帝、皇后的棺材。
\stopbuffer


\startbuffer[435]
犀角:指犀角带,饰有犀角的腰带,非品官不能用。
\stopbuffer


\startbuffer[436]
牙笏(hù):象牙手板。笏,古代大臣上朝拿着的手板,上面可以记事。
\stopbuffer


\startbuffer[437]
三曹对案:诉讼中原告、被告、证人三方一同会审对质。
\stopbuffer


\startbuffer[438]
阳曹:指阳间,人世间。相对『阴曹』而言。
\stopbuffer


\startbuffer[439]
清教:高明的教诲。对人的意见的尊称。
\stopbuffer


\startbuffer[440]
耿耿:明亮的样子。
\stopbuffer


\startbuffer[441]
迎迓(yà):犹迎接。
\stopbuffer


\startbuffer[442]
轮藏:转轮藏,能旋转的藏置佛经的书架。设机轮,可旋转,故名。这里指轮回。
\stopbuffer


\startbuffer[443]
一首:一幅。
\stopbuffer


\startbuffer[444]
引魂幡:丧葬时用以招引鬼魂的旗子,呈垂直悬挂状。
\stopbuffer


\startbuffer[445]
哨:喷。
\stopbuffer


\startbuffer[446]
猖亡:恶鬼。一说同『苍茫』。
\stopbuffer


\startbuffer[447]
急脚子:急行传送书信或探送情报的人。
\stopbuffer


\startbuffer[448]
㪥(zhā):手指张开后拇指和中指(或小指)间的距离。
\stopbuffer


\startbuffer[449]
钧语:对长官话语的敬称。
\stopbuffer


\startbuffer[450]
浮沤(ōu):水面上的泡沫,常比喻变化无常的世事和短暂的生命。沤,水泡。
\stopbuffer


\startbuffer[451]
白蚁阵残:即『南柯一梦』的典故。广陵人淳于棼梦中成为槐安国驸马,任南柯郡太守,历尽荣辱。醒后发现自己躺在大槐树下,槐安国是树上的大蚁穴,南柯郡是另一个小蚁穴。
\stopbuffer


\startbuffer[452]
阴骘(zhì):阴德。
\stopbuffer


\startbuffer[453]
六道轮回:佛教语。轮回的六去处,天道、人道、阿修罗道、畜生道、饿鬼道和地狱道。
\stopbuffer


\startbuffer[454]
霞帔(pèi):以云霞为服。帔,帔肩。
\stopbuffer


\startbuffer[455]
金鱼:即金鱼符。金质的鱼符,唐代高官佩戴,用以表示品级身份。
\stopbuffer


\startbuffer[456]
海骝马:海骝,蒙语的音译。
\stopbuffer


\startbuffer[457]
鞍韂(chàn):鞍,马鞍。韂,垫在马鞍下,垂于马背两旁,以挡泥土的马具。
\stopbuffer


\startbuffer[458]
撮着脚:抓着脚。
\stopbuffer


\startbuffer[459]
渰(yān):同『淹』。
\stopbuffer


\startbuffer[460]
梁冠:有横脊的礼冠,此处为孝帽。
\stopbuffer


\startbuffer[461]
彩女:身份较低的宫女。
\stopbuffer


\startbuffer[462]
理烈:持理刚正。
\stopbuffer


\startbuffer[463]
欺心:起坏心思。
\stopbuffer


\startbuffer[464]
无忧履:古时帝王所穿的鞋子。
\stopbuffer


\startbuffer[465]
托产:谓依托以生。
\stopbuffer


\startbuffer[466]
元龙:指皇帝。
\stopbuffer


\startbuffer[467]
十七宗:指唐太宗之后,唐朝有十七位称『宗』的皇帝。
\stopbuffer


\startbuffer[468]
家缘:家业,家产。
\stopbuffer


\startbuffer[469]
黄钱:旧时用黄表纸折成,焚化给鬼神的纸钱。
\stopbuffer


\startbuffer[470]
害黄病:指皇宫多用黄色装饰,像得了黄疸病一般。
\stopbuffer


\startbuffer[471]
妆奁(lián):嫁妆。
\stopbuffer


\startbuffer[472]
乌盆:瓦盆。
\stopbuffer


\startbuffer[473]
烧纸记库:古时认为生前焚烧纸钱,可寄存在阴间钱库中,供死后使用。
\stopbuffer


\startbuffer[474]
生祠:为活人建立的祠庙。
\stopbuffer


\startbuffer[475]
上疏止浮图:进呈废除佛教的奏章。历史上傅奕曾写《请除释教疏》。上疏,进呈奏章。浮图,也作『浮屠』,佛教语,指佛,佛教,也指宝塔。
\stopbuffer


\startbuffer[476]
三涂:即三途。六道轮回中的火途(地狱道)、血途(畜生道)、刀途(饿鬼道)。
\stopbuffer


\startbuffer[477]
年祚:指立国的年数。
\stopbuffer


\startbuffer[478]
桑门:即沙门,指僧侣。
\stopbuffer


\startbuffer[479]
舞蹈:臣下朝见君上时的礼节。
\stopbuffer


\startbuffer[480]
僧纲:一种僧官名。
\stopbuffer


\startbuffer[481]
毗卢帽:一种绣有毗卢佛像的帽子。亦泛称僧帽。
\stopbuffer


\startbuffer[482]
阇(shé)黎:也作『阇梨』,梵语『阿阇梨』的省称。意谓高僧,亦泛指僧。
\stopbuffer


\startbuffer[483]
功德:这里指功德幡。
\stopbuffer


\startbuffer[484]
前后三:佛教禅宗机语『前三三后三三』的省称,出自《五灯会元》。
\stopbuffer


\startbuffer[485]
五卫:元代有右卫、左卫、中卫、前卫、后卫五军,掌宿卫扈从,兼屯田。
\stopbuffer


\startbuffer[486]
鹗荐鹰扬:指文官贤良,武官威风。鹗荐,表示举荐贤才。鹰扬,威武貌。
\stopbuffer


\startbuffer[487]
介福:大福。
\stopbuffer


\startbuffer[488]
降:『降真香』的省称。烧之烟直上,能入药。传说能降神。亦名鸡骨香、紫藤香。
\stopbuffer


\startbuffer[489]
拔孤魂:此处指超度孤魂野鬼。
\stopbuffer


\startbuffer[490]
六趣:即六道。
\stopbuffer


\startbuffer[491]
清都绛阙:亦作『清都紫微』,传说中天帝所居之宫阙。
\stopbuffer


\startbuffer[492]
丹衷:赤诚之心。
\stopbuffer


\startbuffer[493]
村钞:犹言臭钱。村,粗俗。
\stopbuffer


\startbuffer[494]
头踏:古代官员出行时,走在前面的仪仗。
\stopbuffer


\startbuffer[495]
三宝:佛教语。指佛、法、僧。
\stopbuffer


\startbuffer[496]
随喜:佛教语。此处谓见到他人行善而生欢喜之意。
\stopbuffer


\startbuffer[497]
黄门官:戏曲小说中负责皇宫宿卫、门户值守、随皇帝出行的宦官。
\stopbuffer


\startbuffer[498]
大鹏吞噬之灾:佛教认为大鹏金翅鸟以龙为食,佛经载佛将皂衣分给众龙王,只需一丝一缕,就可避免被吞噬。
\stopbuffer


\startbuffer[499]
七佛:佛教语。谓释迦牟尼及其先出世的六佛。
\stopbuffer


\startbuffer[500]
筘(kòu):织布机机件,状如梳子。
\stopbuffer


\startbuffer[501]
斗:拼合,凑。
\stopbuffer


\startbuffer[502]
本体:佛教称诸法的根本自体或与应身相对的法身。
\stopbuffer


\startbuffer[503]
销金锁:指用金线锁边。销金,嵌金色线。
\stopbuffer


\startbuffer[504]
四生:佛教分世界众生为四类:一曰胎生,如人畜;二曰卵生,如禽鸟鱼鳖;三曰湿生,如某些昆虫;四曰化生,无所依托,唯借业力而忽然出现者,如诸天与地狱及劫初众生。
\stopbuffer


\startbuffer[505]
人天:六道中的人道和天道,泛指众生。
\stopbuffer


\startbuffer[506]
庄严:佛教谓以福德等净化身心。有戒、三昧、智慧、陀罗尼四种庄严。
\stopbuffer


\startbuffer[507]
青骨:指仙骨。
\stopbuffer


\startbuffer[508]
罗卜:即目连。释迦牟尼的弟子,曾通过法力和功德拯救堕入饿鬼道的母亲。
\stopbuffer


\startbuffer[509]
福田:佛教语。佛教以为供养布施、行善修德能受福报,犹如播种田亩有秋收之利,故称。
\stopbuffer


\startbuffer[510]
端的:到底,究竟。
\stopbuffer


\startbuffer[511]
光禄寺:官署名,唐以后专司膳食。明代光禄寺主管宫中宴饮、日常膳食等。
\stopbuffer


\startbuffer[512]
讽:诵读,诵念。
\stopbuffer


\startbuffer[513]
兜罗:一种绵,梵语音译。出自云南等地。
\stopbuffer


\startbuffer[514]
阿罗:即阿罗汉。小乘佛教理想中的最高果位。亦称修得小乘果的人。
\stopbuffer


\startbuffer[515]
夸官:士子考中进士或官员升迁时,排列鼓乐仪仗游街,谓之『夸官』。
\stopbuffer


\startbuffer[516]
以一七继七七:水陆大会分为七段,共七七四十九天。
\stopbuffer


\startbuffer[517]
失瞻:客套语。谓失于瞻仰拜候。
\stopbuffer


\startbuffer[518]
浑俗和光:谓不露锋芒,与世无争。
\stopbuffer


\startbuffer[519]
无量寿身:无量寿原为阿弥陀佛的意译。此处取字面意,指长生不老。
\stopbuffer


\startbuffer[520]
无来无去:此处为无生无灭之意。
\stopbuffer


\startbuffer[521]
颂子:偈颂。佛经中的唱颂词。
\stopbuffer


\startbuffer[522]
牒:公文,证件。
\stopbuffer


\startbuffer[523]
山门:佛寺的外门。
\stopbuffer


\startbuffer[524]
人专吉星:道教『九星择日法』认为天有九星,其中之一的『人专』为吉星。
\stopbuffer


\startbuffer[525]
银騔的马:即银騔马。一种蒙古名白马。
\stopbuffer


\startbuffer[526]
三藏:梵文意译。佛教经典的总称,分经、律、论三部分。通晓三藏的僧人,称三藏法师。
\stopbuffer


\startbuffer[527]
素酒:指酒精度数低,饮后不至醉的酒。
\stopbuffer


\startbuffer[528]
大有:《周易》的一卦。象征大和多。也指丰收。
\stopbuffer


\startbuffer[529]
砧(zhēn)韵:捣衣声的美称。砧,捣衣石。
\stopbuffer


\startbuffer[530]
法轮:佛教语。谓佛说法,圆通无碍,运转不息,能摧破众生的烦恼。
\stopbuffer


\startbuffer[531]
促趱(zǎn)行程:催促赶快上路。趱,加快,赶快。
\stopbuffer


\startbuffer[532]
宵碎:细碎。
\stopbuffer


\startbuffer[533]
整治:办理。
\stopbuffer


\startbuffer[534]
心忙:心急。
\stopbuffer


\startbuffer[535]
东海黄公:传说东海有人名黄公,能用法术降服老虎。
\stopbuffer


\startbuffer[536]
夯(bèn):笨,呆。
\stopbuffer


\startbuffer[537]
符吉梦:喜梦的象征。古人认为梦到熊是生男之兆。符,征兆。吉梦,生男育女的喜梦。
\stopbuffer


\startbuffer[538]
谕时:告知时间。此处指熊冬眠预示冬天来临。
\stopbuffer


\startbuffer[539]
牯(gǔ):俗称阉割过的公牛,亦泛指牛。
\stopbuffer


\startbuffer[540]
牸(zì):母牛。
\stopbuffer


\startbuffer[541]
特:公牛。
\stopbuffer


\startbuffer[542]
守素:保持平素的志愿。此处指吃素。
\stopbuffer


\startbuffer[543]
随时:顺应时势;切合时宜。此处指饮食跟随时令。
\stopbuffer


\startbuffer[544]
啯啅(guō zhuo):象声词。吞咽之声。
\stopbuffer


\startbuffer[545]
失落:谓迷路流落。
\stopbuffer


\startbuffer[546]
元明:佛教语。谓众生固有的清净光明的本性。
\stopbuffer


\startbuffer[547]
自分:自料,自以为。
\stopbuffer

\startbuffer[548]
灾迍(zhūn):同『灾屯』。灾难,祸患。
\stopbuffer


\startbuffer[549]
叵(pǒ)罗:织物名。产自甘肃一带,用羊绒毛织成,常用作猎人的服饰。
\stopbuffer


\startbuffer[550]
吊客:凶神。流年十二神之一。主有疾病、哀泣等事。
\stopbuffer


\startbuffer[551]
河奎:即河魁。六壬占卜法中十二神的二月神,为凶神。
\stopbuffer


\startbuffer[552]
太保:对绿林好汉的尊称。
\stopbuffer


\startbuffer[553]
山虫:老虎的别称。
\stopbuffer


\startbuffer[554]
膂(lǚ)力:体力。膂,脊梁。
\stopbuffer


\startbuffer[555]
哏(hěn):同『狠』。
\stopbuffer


\startbuffer[556]
浼(měi):恳托,请求。
\stopbuffer


\startbuffer[557]
点剁:指剁成碎块的肉。
\stopbuffer


\startbuffer[558]
干巴:干硬的食物,指肉干。
\stopbuffer


\startbuffer[559]
短头经:篇幅很短的经文。
\stopbuffer


\startbuffer[560]
揭斋之咒:僧人用饭前所念的咒语。
\stopbuffer


\startbuffer[561]
腌脏:同『肮脏』。
\stopbuffer


\startbuffer[562]
呢呢痴痴:形容温驯自在的样子。
\stopbuffer


\startbuffer[563]
口业:佛教语。佛教以身、口、意为三业。下文『净身心』指身、意二业。佛教徒正式诵经前,须念《净三业真言》。
\stopbuffer


\startbuffer[564]
荐亡疏:超度的祷文。荐亡,指为死者念经或做佛事,使其亡灵早日脱难超升。疏,僧道拜忏时所焚化的祈祷文。
\stopbuffer


\startbuffer[565]
苾蒭(bì chú)洗业:指佛为病比丘洗身的故事。佛经中记载贤提国有一病比丘,前生作恶多端,死后堕入地狱受苦,历经五百余世。复生为人,依旧重病缠身,身上臭秽,人不敢近。但因当年曾饶过一善人,即佛的前生,因此蒙得佛用神水为之洗浴,使其病愈。苾蒭,即比丘,俗称和尚。
\stopbuffer


\startbuffer[566]
长者:指显贵的人。
\stopbuffer


\startbuffer[567]
合家儿:全家,全家人。
\stopbuffer


\startbuffer[568]
袂(mèi):衣袖。
\stopbuffer


\startbuffer[569]
打挣:犹挣扎,用力支撑。
\stopbuffer


\startbuffer[570]
法身佛:佛教语。佛有法身、报身、化身三身。『法身』谓证得清净自性,成就一切功德之身,不生不灭,无形而随处现形,包含万象。
\stopbuffer


\startbuffer[571]
朔腮:犹缩腮。腮部凹陷。
\stopbuffer


\startbuffer[572]
薜(bì)萝:薜荔和女萝。皆野生植物,常攀缘于山野林木或屋壁之上。
\stopbuffer


\startbuffer[573]
灵山:灵鹫山。
\stopbuffer


\startbuffer[574]
劈手:谓出手迅捷。
\stopbuffer


\startbuffer[575]
赤淋淋:赤身露体,无衣着貌。
\stopbuffer


\startbuffer[576]
薅(hāo):拔除。
\stopbuffer


\startbuffer[577]
扣鞴:谓装好鞍辔。
\stopbuffer


\startbuffer[578]
头陀:这里指行脚乞食的僧人。
\stopbuffer


\startbuffer[579]
踵谢:谓登门道谢。
\stopbuffer


\startbuffer[580]
剪尾:扫动尾巴。
\stopbuffer


\startbuffer[581]
囫囵:完整,整个儿。
\stopbuffer


\startbuffer[582]
自在:随意。
\stopbuffer


\startbuffer[583]
筇(qióng):竹名。可以作杖。
\stopbuffer


\startbuffer[584]
谵(zhān)语:胡言乱语。
\stopbuffer


\startbuffer[585]
檀府:僧人对施主住宅的敬称。
\stopbuffer


\startbuffer[586]
犯:临,到。
\stopbuffer


\startbuffer[587]
脱体:脱身。
\stopbuffer


\startbuffer[588]
华宗:对同族或同姓者的美称。
\stopbuffer


\startbuffer[589]
直裰(duō):僧袍。偏衫、裙子相缀合的长袍。
\stopbuffer


\startbuffer[590]
马面样的褶子:即马面裙。类似今百褶裙,裙面中间有一尺左右无褶的平面,称『马面』。
\stopbuffer


\startbuffer[591]
耳闭:耳背。
\stopbuffer


\startbuffer[592]
剪径:拦路抢劫。
\stopbuffer


\startbuffer[593]
『身本忧』等:此六人名化自佛教语『六贼』,即眼、耳、鼻、舌、意、身六根,可以产生喜、怒、爱、思、欲、忧的烦恼。佛教谓六根妄逐尘境,如贼劫财。
\stopbuffer


\startbuffer[594]
撞祸:闯祸,惹事生非。
\stopbuffer


\startbuffer[595]
白客:清白无罪的人。
\stopbuffer


\startbuffer[596]
绪咶(guō):犹絮叨。咶,同『聒』,嘈杂,吵闹。
\stopbuffer


\startbuffer[597]
撮土焚香:指旧时人在野外撮土代替香炉,烧香敬神。撮土,用手把土聚拢成堆。
\stopbuffer


\startbuffer[598]
恳恳:诚挚殷切貌。
\stopbuffer


\startbuffer[599]
惩创:惩戒;警戒。
\stopbuffer


\startbuffer[600]
圯(yí)桥:传说秦末张良在此遇一老父,授其《太公兵法》。
\stopbuffer


\startbuffer[601]
赤松子:传为上古神仙。
\stopbuffer


\startbuffer[602]
裁处:裁决处置。
\stopbuffer


\startbuffer[603]
海藏:此处指龙宫。
\stopbuffer


\startbuffer[604]
竖蜻蜓:指倒立。
\stopbuffer


\startbuffer[605]
鸥鹭相忘:即鸥鹭忘机。古时海上有好鸥者,每日从鸥鸟游,鸥鸟至者以百数。其父令取来玩之。次日至海上,鸥鸟舞而不下。谓无机心者则异类亦与之相亲。后喻指淡泊隐居。
\stopbuffer


\startbuffer[606]
高埠(bù):高土丘。
\stopbuffer


\startbuffer[607]
溜缰:指驴、马等脱掉缰绳。
\stopbuffer


\startbuffer[608]
不济:不顶用。
\stopbuffer


\startbuffer[609]
护驾伽蓝:佛教寺院中的护法神。佛典原谓有美音、梵音、雷音、师子等十八神护伽蓝。
\stopbuffer


\startbuffer[610]
招声:应声,答话。
\stopbuffer


\startbuffer[611]
犯对:作对。
\stopbuffer


\startbuffer[612]
抢白:奚落,指责。
\stopbuffer


\startbuffer[613]
搪:抵御,抵挡。
\stopbuffer


\startbuffer[614]
草科:草窠,草丛。
\stopbuffer


\startbuffer[615]
三尸神咋(zhà):三尸神,道教称在人体内作祟的三位神。咋,借作『炸』,爆发,激怒。
\stopbuffer


\startbuffer[616]
孤拐:脚踝,脚腕两旁突起的部分。
\stopbuffer


\startbuffer[617]
混元上真:指吸收天地元气而成的真仙。混元,天地元气。上真,真仙。
\stopbuffer


\startbuffer[618]
向年:往年。
\stopbuffer


\startbuffer[619]
侮手:交手,过手。指搏斗。
\stopbuffer


\startbuffer[620]
争持:争斗、争执而不相让。
\stopbuffer


\startbuffer[621]
七佛之师:佛教中七佛之师是文殊菩萨,曾助过去七佛成就佛果。此处为小说的挪用。
\stopbuffer


\startbuffer[622]
赤尻:红屁股。指猴子。常作骂人语。
\stopbuffer


\startbuffer[623]
魔头:对头,克星。
\stopbuffer


\startbuffer[624]
瑜伽:佛教的一个宗派。
\stopbuffer


\startbuffer[625]
作做:当作,算作。
\stopbuffer


\startbuffer[626]
项下明珠:传说生在龙颈下的明珠。
\stopbuffer


\startbuffer[627]
毛片:毛色。
\stopbuffer


\startbuffer[628]
横骨:指舌骨。口衔横骨,即不随意吐人言。
\stopbuffer


\startbuffer[629]
信心:诚心。
\stopbuffer


\startbuffer[630]
刬(chǎn)马:无鞍辔之马。
\stopbuffer


\startbuffer[631]
眼乖:眼尖,眼力好。
\stopbuffer


\startbuffer[632]
迍邅(zhān):处境不利,困顿。
\stopbuffer


\startbuffer[633]
下梢:将来,以后。
\stopbuffer


\startbuffer[634]
庙祝:庙宇中管香火的人。
\stopbuffer


\startbuffer[635]
衬屉缰笼:衬屉,即鞍鞯,马鞍下的垫子。缰,缰绳。笼,即辔头。
\stopbuffer


\startbuffer[636]
云扇:鞍翅,鞍桥下两侧的扇形板。
\stopbuffer


\startbuffer[637]
两垂:垂在马鞍两侧的缨穗装饰。
\stopbuffer


\startbuffer[638]
皮丁儿寸扎:用细皮条一寸寸扎制。
\stopbuffer


\startbuffer[639]
挽手儿:马鞭。
\stopbuffer


\startbuffer[640]
问讯:问候。也指僧尼等向人合掌致敬。
\stopbuffer


\startbuffer[641]
柳眼:早春初生的柳叶如人睡眼初展,因以为称。
\stopbuffer


\startbuffer[642]
三山门:佛教将寺院比作涅槃,须由『三解脱门』进入,故其门常有三,称三门殿或山门殿。
\stopbuffer


\startbuffer[643]
祇(qí)园:祇树给孤独园。印度佛教圣地之一。后用为佛寺的代称。
\stopbuffer


\startbuffer[644]
招提:四方之僧称招提僧,四方僧之住处称为招提僧坊。
\stopbuffer


\startbuffer[645]
娑罗:佛在印度阿利罗跋提河边的娑罗树下入灭。印度佛教圣地。
\stopbuffer


\startbuffer[646]
一众:即一个。众,这里作量词,专饰僧人。
\stopbuffer


\startbuffer[647]
道人:佛寺中打杂的人。
\stopbuffer


\startbuffer[648]
铺胸纳地:一种上半身贴近地面的跪拜礼,表示虔敬与隆重。
\stopbuffer


\startbuffer[649]
院主:寺监。寺院的高级管理人员。后以院主称方丈,改院主为寺监,俗称当家。
\stopbuffer


\startbuffer[650]
偏衫:一种僧尼服装。开脊接领,斜披在左肩上,类似袈裟。
\stopbuffer


\startbuffer[651]
骊山老母:古代神话中的女仙,为老妇人形象。文中也作『黎山老母』。
\stopbuffer


\startbuffer[652]
樗(chū)朽:腐朽的樗木,喻无用之人。自谦辞。樗,木名,即臭椿,喻无用之材。
\stopbuffer


\startbuffer[653]
幸童:贴身的童仆。
\stopbuffer


\startbuffer[654]
法蓝:即珐琅。一种涂于金属器上的釉面装饰,如搪瓷、景泰蓝等均为珐琅制品。
\stopbuffer


\startbuffer[655]
污眼:弄脏眼睛。言不值得一看。
\stopbuffer


\startbuffer[656]
穿花纳锦:以穿纱的方法做成的刺绣花样。
\stopbuffer


\startbuffer[657]
管整:包办打理。
\stopbuffer


\startbuffer[658]
发过了:指发达过了。
\stopbuffer


\startbuffer[659]
房头:僧人的师徒支系,一支系为一房头。
\stopbuffer


\startbuffer[660]
嚣:方言。犹薄。
\stopbuffer


\startbuffer[661]
马、赵、温、关:道教四大元帅,分别为马灵耀、赵公明、温琼、关羽。
\stopbuffer


\startbuffer[662]
控背:弯着背。控,弯曲。
\stopbuffer


\startbuffer[663]
相应:相宜。此处指合算,便宜。
\stopbuffer


\startbuffer[664]
调嘴:耍嘴皮子。
\stopbuffer


\startbuffer[665]
南方三炁:道教的火神。
\stopbuffer


\startbuffer[666]
回禄:上古传说中的火神。
\stopbuffer


\startbuffer[667]
燧人:燧人氏,传说中的古帝王。钻木取火的发明者。
\stopbuffer


\startbuffer[668]
槅(ɡé)扇:一种一对对相连的门,装饰雕花格子,门背糊纸或装玻璃。槅,窗上格子。
\stopbuffer


\startbuffer[669]
上下房:规模较大的寺院通常有上下院之分。上院居住资历较老、专意修行的僧人,下院受上院管辖,负责世俗佛事和日常管理。
\stopbuffer


\startbuffer[670]
选剥:犹跣剥。脱光。
\stopbuffer


\startbuffer[671]
焜(kūn)耀:照耀;辉煌。焜,明亮,光耀。
\stopbuffer


\startbuffer[672]
奉承:侍奉。
\stopbuffer


\startbuffer[673]
供奉:侍奉,伺候。
\stopbuffer


\startbuffer[674]
老剥皮:骂人话。犹言该死的,该杀的。
\stopbuffer


\startbuffer[675]
抟砂炼汞:指用朱砂、水银来提炼仙丹。抟,聚集。砂,朱砂。汞,水银。
\stopbuffer


\startbuffer[676]
白雪黄芽:道教语。白雪,指水银。黄芽,从铅里炼出的精华。
\stopbuffer


\startbuffer[677]
母难之日:即生日。
\stopbuffer


\startbuffer[678]
道官:对僧道的敬称。
\stopbuffer


\startbuffer[679]
槎(chá):树杈。
\stopbuffer


\startbuffer[680]
軃(duǒ):下垂。
\stopbuffer


\startbuffer[681]
筑煤:制墨的工序,用杵将煤捣细。筑,捣。煤,古代烧松枝聚松烟为煤。
\stopbuffer


\startbuffer[682]
刷炭:制墨的工序,在竹棚内燃烧松枝,刷取附着在棚顶的煤烟。
\stopbuffer


\startbuffer[683]
海口浪言:说大话。
\stopbuffer


\startbuffer[684]
大品:指佛经之全本或繁本,与节略本的『小品』相对。此处指传授完全的道法。
\stopbuffer


\startbuffer[685]
日月坎离交:日与离、月与坎,分别是元神、元气的代称。
\stopbuffer


\startbuffer[686]
无漏:佛教语。谓涅槃、菩提和断绝一切烦恼根源之法。漏,烦恼。
\stopbuffer


\startbuffer[687]
活该:活了有。该,有,拥有。
\stopbuffer


\startbuffer[688]
阴棍手:一种枪法招式。
\stopbuffer


\startbuffer[689]
白虎爬山、黄龙卧道:白虎探爪,棍法招式。黄龙卧道,枪法招式。
\stopbuffer


\startbuffer[690]
管情:包管。
\stopbuffer


\startbuffer[691]
侍生:明清官场中后辈对前辈的自称。一般用于名帖。
\stopbuffer


\startbuffer[692]
上人:对和尚的尊称。
\stopbuffer


\startbuffer[693]
回禄之难:指火灾。
\stopbuffer


\startbuffer[694]
花酌:赏花饮酒的宴席。
\stopbuffer


\startbuffer[695]
奉扳(pān):奉,敬辞。扳,高攀。
\stopbuffer


\startbuffer[696]
是荷:意谓对你的帮助或恩惠表示感谢。多用于书信的末尾。
\stopbuffer


\startbuffer[697]
服气:吐纳。道家养生延年之术。
\stopbuffer


\startbuffer[698]
纻(zhù)丝:即缎。袢(pàn)袄:一种有衬里的对襟夹衣。
\stopbuffer


\startbuffer[699]
鸦青:暗青色。
\stopbuffer


\startbuffer[700]
乌角软巾:乌角巾。葛制黑色有折角的头巾。常为隐士所戴。
\stopbuffer


\startbuffer[701]
皂靴:黑色高帮、白色厚底的鞋子,旧时官绅所穿。
\stopbuffer


\startbuffer[702]
华翰:对他人来信的美称。
\stopbuffer


\startbuffer[703]
小校:低级武官名。亦指小卒。
\stopbuffer


\startbuffer[704]
绰(chāo)着经儿:绰经,顺着线索,趁着话头。绰,顺,趁。
\stopbuffer


\startbuffer[705]
风信:信息,消息。
\stopbuffer


\startbuffer[706]
留云下院:指神仙可以暂时驻留的香火院。
\stopbuffer


\startbuffer[707]
放刁:耍无赖,用狡猾的手段使人为难。
\stopbuffer


\startbuffer[708]
便益:方便,便利。
\stopbuffer


\startbuffer[709]
了劣:了结。指死亡。
\stopbuffer


\startbuffer[710]
贽(zhì)见:见面礼。贽,初次见人时所执的礼物。
\stopbuffer


\startbuffer[711]
勾头:拘票,缉拿犯人的凭证。
\stopbuffer


\startbuffer[712]
计较:计策,打算,主张。
\stopbuffer


\startbuffer[713]
遭瘟:遭受祸害,倒霉。
\stopbuffer


\startbuffer[714]
熟嘴:能说会道。
\stopbuffer


\startbuffer[715]
步虚:指凌空步行。
\stopbuffer


\startbuffer[716]
『三三勾』二句:言晋葛洪、汉少翁炼丹之事。三三为九,指丹要精炼九次。勾漏,山名,葛洪炼丹处。六六为三十六,炼丹常用三十六这个数字,如三十六时辰等。商,计算。
\stopbuffer


\startbuffer[717]
铄(shuò):熔化金属。
\stopbuffer


\startbuffer[718]
牟尼:牟尼珠,也作『摩尼珠』,宝珠。
\stopbuffer


\startbuffer[719]
岑(cén):小而高的山。
\stopbuffer


\startbuffer[720]
鄙意:谦辞,称自己的意见。
\stopbuffer


\startbuffer[721]
觑(qù)定:看准。
\stopbuffer


\startbuffer[722]
理起四平:武术练功的一种动作。纵身跃起,四肢挺直,仰身跌下。又作『跌四平』。
\stopbuffer


\startbuffer[723]
耽阁:义同『耽搁』,文中也作『担阁』。
\stopbuffer


\startbuffer[724]
一灵不损:一个生灵都不杀害。
\stopbuffer


\startbuffer[725]
公道:的确,实在。
\stopbuffer


\startbuffer[726]
愿心儿:发下的心愿;亦指对神佛祈求时许下的酬谢。
\stopbuffer


\startbuffer[727]
散了福:旧时祭祀后,将祭祀食品分给众人,叫『散福』。
\stopbuffer


\startbuffer[728]
骢(cōng):青白色的马。
\stopbuffer


\startbuffer[729]
金线:比喻初生柳条。
\stopbuffer


\startbuffer[730]
爨(cuàn):灶。
\stopbuffer


\startbuffer[731]
裩(kūn):有裆的裤子。
\stopbuffer


\startbuffer[732]
清气:犹闲气。
\stopbuffer


\startbuffer[733]
支吾:抗拒。
\stopbuffer


\startbuffer[734]
老女儿:最小的女儿。
\stopbuffer


\startbuffer[735]
纥剌(gē là)星:喻指找麻烦的人。纥剌,绊人的土块。
\stopbuffer


\startbuffer[736]
凑四合六:四和六加在一起是十。形容一拍即合,十分凑巧。
\stopbuffer


\startbuffer[737]
带累:连累。
\stopbuffer


\startbuffer[738]
马台:旧时高门大户前供上下马的石台。
\stopbuffer


\startbuffer[739]
慢皮:顽劣,不听话。吴地方言。
\stopbuffer


\startbuffer[740]
干净:完全。
\stopbuffer


\startbuffer[741]
谆谆:啰嗦,絮絮不休貌。
\stopbuffer


\startbuffer[742]
家怀:不见外,不客套。
\stopbuffer


\startbuffer[743]
小价(jiè):亦作『小介』,仆人,对己仆的谦称。价,被派遣传递信息或供役使的人。
\stopbuffer


\startbuffer[744]
替:被。
\stopbuffer


\startbuffer[745]
作耗:指妖物作怪。耗,祸乱,祸祟。
\stopbuffer


\startbuffer[746]
掠:梳理。
\stopbuffer


\startbuffer[747]
尘淄(zī):灰尘,尘垢。淄,染黑,污染意。
\stopbuffer


\startbuffer[748]
蹙蹙(cù):皱眉,忧惧不安貌。
\stopbuffer


\startbuffer[749]
喙(huì):嘴。
\stopbuffer


\startbuffer[750]
梭布:家庭木机所织之布。
\stopbuffer


\startbuffer[751]
漫头一料:顺着头一摔。漫,顺。料,通『撂』,扔,甩。
\stopbuffer


\startbuffer[752]
出个恭:俗称入厕。古代科举考场中设有『出恭』『入敬』牌,考生入厕须领牌,故称。
\stopbuffer


\startbuffer[753]
趁心:即称心,符合心愿。趁,通『称』。
\stopbuffer


\startbuffer[754]
姨夫:妻子姐妹的丈夫。
\stopbuffer


\startbuffer[755]
刚鬣(liè):古代祭祀所用猪的专称。一说为豪猪。
\stopbuffer


\startbuffer[756]
荡魔祖师:即真武大帝。宋代以来的民间信仰,真武将军和天蓬元帅同为北极四圣之一。
\stopbuffer


\startbuffer[757]
磕:凸出。
\stopbuffer


\startbuffer[758]
辍(chuò):中止,停止。
\stopbuffer


\startbuffer[759]
姹(chà)女:少女、美女。道教炼丹术常以婴儿、姹女代称铅和汞。
\stopbuffer


\startbuffer[760]
三花聚顶、五气朝元:皆为道教的修炼之法。
\stopbuffer


\startbuffer[761]
宪节:风宪官(监察执行法纪的官吏)所持的符节。
\stopbuffer


\startbuffer[762]
竭绝:尽,穷尽。
\stopbuffer


\startbuffer[763]
馕(nǎng)糠:吃糠。骂人话。馕,拼命地往嘴里塞食物。
\stopbuffer


\startbuffer[764]
茶红酒礼:旧俗婚姻娶嫁时,以茶礼、花红和酒为聘礼。
\stopbuffer


\startbuffer[765]
真犯:指情真罪实的犯人。
\stopbuffer


\startbuffer[766]
荧惑:火星;火神名。这里指火德星君。
\stopbuffer


\startbuffer[767]
软:使之不坚定,动摇。
\stopbuffer


\startbuffer[768]
捣碓(duì):在碓臼中舂米。形容频频磕头。碓,木石做成的捣米器具。
\stopbuffer


\startbuffer[769]
拙荆:对自己妻子的谦称。
\stopbuffer


\startbuffer[770]
火居道士:有妻室的道士。
\stopbuffer


\startbuffer[771]
上盖:外衣,罩衫。
\stopbuffer


\startbuffer[772]
挂脚粮:旧时入赘女婿的长工钱。
\stopbuffer


\startbuffer[773]
浑家:妻子。旧时家中妻子主内,故称。
\stopbuffer


\startbuffer[774]
伐毛:即『伐毛洗髓』的传说。谓仙人涤除尘垢,脱胎换骨。
\stopbuffer


\startbuffer[775]
琴堂:州、府、县署的代称。
\stopbuffer


\startbuffer[776]
主簿:官名。主管文书,办理官府日常事务。
\stopbuffer


\startbuffer[777]
作御:统治,称王。御,统治,治理;指帝王之在位。
\stopbuffer


\startbuffer[778]
门户:指修行的门径。
\stopbuffer


\startbuffer[779]
喝风屙(ē)烟:吃进去的是风,排出来的是烟。犹言喝西北风。屙,排泄。
\stopbuffer


\startbuffer[780]
赃埋:犹栽赃,诬陷。
\stopbuffer


\startbuffer[781]
恨苦:犹痛苦。
\stopbuffer


\startbuffer[782]
斗篷:方言。斗笠。
\stopbuffer


\startbuffer[783]
厌钝:扫兴,不顺遂。
\stopbuffer


\startbuffer[784]
虎諕:吓唬。
\stopbuffer


\startbuffer[785]
好道:此处义犹好歹。将就,勉强。
\stopbuffer


\startbuffer[786]
别颏(kē)腮:瘪嘴。别,借用作『瘪』。颏,下巴。
\stopbuffer


\startbuffer[787]
俊刮:俊俏,漂亮。
\stopbuffer


\startbuffer[788]
灾愆(qiān):罪孽招致的灾祸,灾殃。愆,罪过,过失。
\stopbuffer


\startbuffer[789]
踢天弄井:上天入地的事都能做。形容本事大,能力强。
\stopbuffer


\startbuffer[790]
打瓦:方言。倒楣,家业破落。
\stopbuffer


\startbuffer[791]
撞头:乱闯乱撞。
\stopbuffer


\startbuffer[792]
山恶人善:方言。指相貌丑陋,心地善良。
\stopbuffer


\startbuffer[793]
掬(jū):掀,翘起。
\stopbuffer


\startbuffer[794]
令嗣:称对方儿子的敬辞。
\stopbuffer


\startbuffer[795]
汤:碰到,撞到。
\stopbuffer


\startbuffer[796]
滴答:指言语啰苏。
\stopbuffer


\startbuffer[797]
发课:起课。旧时卜卦、占算法之一。
\stopbuffer


\startbuffer[798]
五爻(yáo)六爻:爻,组成八卦的基本符号。此处『五爻』谐音『无肴』,是猪八戒打趣之语。
\stopbuffer


\startbuffer[799]
不虞:意料之外的事。
\stopbuffer


\startbuffer[800]
哈话:傻话,丢人话。方言『哈』,傻子。
\stopbuffer


\startbuffer[801]
骨都都:腾涌貌。屹嶝嶝(dèng):峻峭耸立貌。
\stopbuffer


\startbuffer[802]
泥泥蚩蚩(chī):同『呢呢痴痴』。蚩蚩,敦厚貌。
\stopbuffer


\startbuffer[803]
巴:爬,攀登。
\stopbuffer


\startbuffer[804]
扑轳轳、掬造字造字(lǜ):皆为象声词。
\stopbuffer


\startbuffer[805]
磨担:将扁担换肩的动作。
\stopbuffer


\startbuffer[806]
空头:无根据,没来由。
\stopbuffer


\startbuffer[807]
渍:滑,溜。
\stopbuffer


\startbuffer[808]
眼膛:眼眶。
\stopbuffer


\startbuffer[809]
媸(chī):丑陋,丑恶。
\stopbuffer


\startbuffer[810]
搠(shuò):刺,戳。
\stopbuffer


\startbuffer[811]
案酒:也作『按酒』。下酒,下酒菜。
\stopbuffer


\startbuffer[812]
撒:塞、系。
\stopbuffer


\startbuffer[813]
劣蹶(jué):乖戾,顽劣,不驯顺。
\stopbuffer


\startbuffer[814]
稳便:恰当,方便,稳妥。
\stopbuffer


\startbuffer[815]
起倒:高低,好歹,轻重。
\stopbuffer


\startbuffer[816]
对命:抵命。
\stopbuffer


\startbuffer[817]
鄙猥:丑陋矮小。
\stopbuffer


\startbuffer[818]
鐏(zūn):戈柄下端的圆锥形金属套。
\stopbuffer


\startbuffer[819]
坫(diàn):土台。
\stopbuffer


\startbuffer[820]
真武龟蛇、梓潼骡子:真武大帝的护法神将龟蛇;梓潼真君的坐骑白螺。
\stopbuffer


\startbuffer[821]
先生:对医生的敬称。
\stopbuffer


\startbuffer[822]
苍头:指奴仆。
\stopbuffer


\startbuffer[823]
『夹脑风』等:皆是猪八戒的玩笑语。夹脑风,傻而疯癫,精神不正常。羊耳风,又叫羊儿风、羊角风、羊癫风,即癫痫。大麻风,即麻风。头风,头痛。
\stopbuffer


\startbuffer[824]
明杖儿:盲人探路的手杖。旧时盲人从事相面、算卦等业,人称先生。故八戒以此打趣。
\stopbuffer


\startbuffer[825]
躲门户:为躲避徭役和杂税而离家逃亡,称为『躲门户』。
\stopbuffer


\startbuffer[826]
里长:一里之长。里,古代地方行政组织。
\stopbuffer


\startbuffer[827]
扯架子:骄傲自大,虚张声势。
\stopbuffer


\startbuffer[828]
点化:指用法术使物变化。
\stopbuffer


\startbuffer[829]
钻疾:灵活,机灵。
\stopbuffer


\startbuffer[830]
疙疸(gē da):即疙瘩。
\stopbuffer


\startbuffer[831]
磨:摇,挥动。
\stopbuffer


\startbuffer[832]
庞眉:眉毛黑白杂色。形容年老的样子。庞,用同『庬』(máng),乱,杂。
\stopbuffer


\startbuffer[833]
去来人:过来人。
\stopbuffer


\startbuffer[834]
李长庚:长庚即太白金星。民间传说李白出生时,其母梦见金星入怀,说太白金星姓李。
\stopbuffer


\startbuffer[835]
班首:班列之首。指寺院的高层僧人。
\stopbuffer


\startbuffer[836]
大火向西流:大火星向西方天空偏移,说明秋天到了。大火,星名。流,落下。
\stopbuffer


\startbuffer[837]
真字:即楷书。
\stopbuffer


\startbuffer[838]
造字(lóng):兽名。
\stopbuffer


\startbuffer[839]
峥嵘:面貌凶恶的样子。
\stopbuffer


\startbuffer[840]
戗(qiāng):支撑。
\stopbuffer


\startbuffer[841]
榔杭:即榔槺,形容笨拙。
\stopbuffer


\startbuffer[842]
木母金公:道教炼丹术中分别代称汞、铅。
\stopbuffer


\startbuffer[843]
三千功满:道教认为行三千功善,可以成仙。
\stopbuffer


\startbuffer[844]
明华:仙宫名。此处指天宫。
\stopbuffer


\startbuffer[845]
虎头牌:犹虎符。刻有虎头形的牌子。元代俗称虎头牌。
\stopbuffer


\startbuffer[846]
典刑:正法,受死刑。
\stopbuffer


\startbuffer[847]
消停:从容,舒徐。
\stopbuffer


\startbuffer[848]
鲊(zhǎ)酱:鱼酱。鲊,用腌、糟等方法加工的鱼类食品,泛指腌制食品。
\stopbuffer


\startbuffer[849]
掐出水沫儿:指皮肤娇嫩,可以掐出水来。
\stopbuffer


\startbuffer[850]
走硝:指皮肉中的硝性流失,会变得干硬。
\stopbuffer


\startbuffer[851]
丧门:丧门星。星命家以为一岁十二辰都随着善神和凶煞,叫丛辰,丧门是凶煞之一。
\stopbuffer


\startbuffer[852]
崖坎:山坡。这里指岸边的平地。
\stopbuffer


\startbuffer[853]
哭丧杖:即哭丧棒。旧时出殡时孝子所持的哀杖。亦用来贬称其他棍棒,以诟骂持棒者。
\stopbuffer


\startbuffer[854]
娑罗派:娑罗树的一根枝。宋代以后民间传说月中桂树为娑罗树。派,分枝。
\stopbuffer


\startbuffer[855]
玠(jiè):大圭,一种礼器。此处指玉饰。
\stopbuffer


\startbuffer[856]
木母克刀圭:木母,此处指猪八戒。刀圭,原指量中药的匙,此处指沙和尚。
\stopbuffer


\startbuffer[857]
沃(wò):淹。
\stopbuffer


\startbuffer[858]
头直上:头顶上。
\stopbuffer


\startbuffer[859]
赊:长,远。
\stopbuffer


\startbuffer[860]
兜:指勒马。
\stopbuffer


\startbuffer[861]
『木母金公』二句:小说中常以『木母』指代猪八戒,以『金公』指代孙悟空,『黄婆』指代沙和尚,学界对此有多种解释,大体与五行和道教观念有关。
\stopbuffer


\startbuffer[862]
『咬开铁弹』句:铁弹,佛教喻无法明心见性。咬开铁弹,喻顿悟。消息,奥妙,真谛。
\stopbuffer


\startbuffer[863]
性海:佛教语。指真如之理性深广如海。
\stopbuffer


\startbuffer[864]
累坠:同『累赘』。
\stopbuffer


\startbuffer[865]
䉭(liè):巡铺床的竹垫,竹席。
\stopbuffer


\startbuffer[866]
攀:牵扯。
\stopbuffer


\startbuffer[867]
大达赸(shàn)步:指大踏步走。达,踏。赸,走。
\stopbuffer


\startbuffer[868]
围圜(huán):围墙。圜,围绕。
\stopbuffer


\startbuffer[869]
早是:幸亏,幸好。
\stopbuffer


\startbuffer[870]
过当:生活过得去,有家产。
\stopbuffer


\startbuffer[871]
石鼓:旧时人家大门两旁的鼓形大石。
\stopbuffer


\startbuffer[872]
帘栊高控:指窗帘高高垂下。帘栊,窗帘和窗牖,泛指门窗的帘子。控,下垂。
\stopbuffer


\startbuffer[873]
饧(xíng)眼:目光凝滞、蒙眬,半睁半闭的样子。
\stopbuffer


\startbuffer[874]
官绿:正绿色,纯绿色。
\stopbuffer


\startbuffer[875]
时样造字(dí)髻:时样,流行的式样。造字髻,用丝编织,蒙以纱布,罩在发上的假髻。
\stopbuffer


\startbuffer[876]
盘龙发:即盘龙髻。指妇女盘绕卷曲的发髻,造型如盘曲之龙。
\stopbuffer


\startbuffer[877]
针指:针线活。
\stopbuffer


\startbuffer[878]
家长:一家之主,丈夫。
\stopbuffer


\startbuffer[879]
缁衣:黑色衣服,僧尼的一种服装。
\stopbuffer


\startbuffer[880]
打仰:身子朝后仰。形容害怕。也指因害怕而退却、反悔。
\stopbuffer


\startbuffer[881]
佯佯不睬:装作没听到、没看到的样子,不予理会。
\stopbuffer


\startbuffer[882]
在家人:泛指僧、尼、道士以外的世俗之人,对『出家人』而言。
\stopbuffer


\startbuffer[883]
篘(chōu):滤酒用的竹具。这里指滤酒。
\stopbuffer


\startbuffer[884]
血食:谓吃鱼肉之类荤腥食物。
\stopbuffer


\startbuffer[885]
臭皮囊:喻指人之躯壳。释、道以人体内多污秽不洁之物,故有此称。
\stopbuffer


\startbuffer[886]
者者谦谦:和和气气、唯唯诺诺的样子。
\stopbuffer


\startbuffer[887]
栽人:捉弄人,陷害人。
\stopbuffer


\startbuffer[888]
腰门:正门以内的第二重门,亦指两厅中间的隔门。
\stopbuffer


\startbuffer[889]
说杀了:即说死了。杀,死板,无可变动。
\stopbuffer


\startbuffer[890]
活着些脚儿:指留一些退路。
\stopbuffer


\startbuffer[891]
停妻再娶妻:抛弃未离异的妻子,与别人结婚。犹言重婚。
\stopbuffer


\startbuffer[892]
勾当:事情。
\stopbuffer


\startbuffer[893]
拿班儿:装腔作势,摆架子。
\stopbuffer


\startbuffer[894]
裂:破碎,败坏。
\stopbuffer


\startbuffer[895]
弄精细:卖弄精明。
\stopbuffer


\startbuffer[896]
挂搭僧:游方和尚。
\stopbuffer


\startbuffer[897]
蹡(qiāng)路:赶路。
\stopbuffer


\startbuffer[898]
奈上祝下:指畏首畏尾,很为难的样子。
\stopbuffer


\startbuffer[899]
口号儿:指打油诗、顺口溜或俗谚之类。
\stopbuffer


\startbuffer[900]
牵马:做媒又称牵马。此处为双关语。
\stopbuffer


\startbuffer[901]
说合:从中介绍,促使事情成功。
\stopbuffer


\startbuffer[902]
停停当当:停当,妥当。
\stopbuffer


\startbuffer[903]
通书:历书。指挑选适合嫁娶的日期。
\stopbuffer


\startbuffer[904]
天恩上吉日:天恩,阴阳家所谓的吉神之一,天恩值日为吉日。
\stopbuffer


\startbuffer[905]
者嚣:掩饰,隐瞒。
\stopbuffer


\startbuffer[906]
趄趄(jū):欲进又退的样子。
\stopbuffer


\startbuffer[907]
会亲:旧时结婚后男女两家互邀亲属相见之礼。
\stopbuffer


\startbuffer[908]
碾房:碾谷磨面的屋子或作坊。
\stopbuffer


\startbuffer[909]
阴阳:阴阳先生。
\stopbuffer


\startbuffer[910]
撒帐:一种婚俗。新婚夫妇交拜毕,并坐床沿,妇女散掷金钱彩果。
\stopbuffer


\startbuffer[911]
谢亲:一种婚俗。迎娶后,婿往女家致感谢意;招女婿的,新娘往夫家致感谢意。
\stopbuffer


\startbuffer[912]
撞个天婚:撞天婚,意谓任凭天意促成的婚姻。如旧小说、戏剧里抛彩球之类。
\stopbuffer


\startbuffer[913]
柱科:柱子。吴地方言,也作『柱棵』『柱窠』。
\stopbuffer


\startbuffer[914]
蒙直:忠厚爽直。
\stopbuffer


\startbuffer[915]
卖解儿:指表演武艺、杂技,借以谋生。
\stopbuffer


\startbuffer[916]
绷巴吊拷:剥去衣裳,用绳捆绑,吊打拷问。绷,束缚,捆绑。巴,同『扒』,剥去。
\stopbuffer


\startbuffer[917]
秾(nóng)华:繁盛艳丽的花朵。秾,繁盛,艳丽。
\stopbuffer


\startbuffer[918]
阆(làng)苑:阆风之苑,传说中仙人的住处。
\stopbuffer


\startbuffer[919]
短头:犹总共,整整。
\stopbuffer


\startbuffer[920]
佛子:受佛戒者,佛门弟子。
\stopbuffer


\startbuffer[921]
罗唣:叫闹,纠缠。
\stopbuffer


\startbuffer[922]
紫极:道教称天上仙人居所。
\stopbuffer


\startbuffer[923]
青鸟:神话传说中为西王母取食传信的神鸟。
\stopbuffer


\startbuffer[924]
紫鸾:传说中的神鸟。
\stopbuffer


\startbuffer[925]
鬅(péng):头发散乱貌。
\stopbuffer


\startbuffer[926]
格子:指上部有格眼供采光的门或窗。
\stopbuffer


\startbuffer[927]
谄佞(chǎn nìng):花言巧语,阿谀逢迎。
\stopbuffer


\startbuffer[928]
捆风:扯谎。
\stopbuffer


\startbuffer[929]
空心架子:空话,谎言。
\stopbuffer


\startbuffer[930]
泼牛蹄子:牛蹄子,也称『牛鼻子』,对道士的蔑称。泼,表示厌恶、鄙视。
\stopbuffer


\startbuffer[931]
方情:犹交情,情谊。
\stopbuffer


\startbuffer[932]
真元不昧:真元,本性。不昧,不晦暗,明亮。
\stopbuffer


\startbuffer[933]
土宜:专作礼品用的土产,即土仪。
\stopbuffer


\startbuffer[934]
年丰时稔(rěn):年成好,庄稼大丰收。稔,庄稼成熟。
\stopbuffer


\startbuffer[935]
作荒:旧谓荒年将至,百姓饭量增大的现象。
\stopbuffer


\startbuffer[936]
狼犺(kàng):笨拙,笨重。
\stopbuffer


\startbuffer[937]
泱:流,淌。
\stopbuffer


\startbuffer[938]
溜撒:行动迅速、敏捷。
\stopbuffer


\startbuffer[939]
『四时蔬菜』一段:莙荙(jūn dá),莙荙菜,又称甜菜;瓠(hù),即瓠瓜;芫荽(yán suī),俗称香菜;薤(xiè),类似蒜的一种蔬菜,地下有圆锥形鳞茎;窝蕖(qú),即莴苣;童蒿(hāo),即茼蒿;藚(xù),又名泽泻,形似蒲公英;蔓菁,又名芜菁,俗称大头菜;红苋(xiàn),红苋菜;菘,即白菜;芥,芥菜。
\stopbuffer


\startbuffer[940]
扢(gē)蒂:瓜果和枝茎接连的部分。
\stopbuffer


\startbuffer[941]
地仙:道教将仙分为五等,即鬼仙、人仙、地仙、神仙、天仙。
\stopbuffer


\startbuffer[942]
襟儿:衣服胸前的部分。
\stopbuffer


\startbuffer[943]
蛮话:指不易听懂的南方话。
\stopbuffer


\startbuffer[944]
省得:晓得,明白。
\stopbuffer


\startbuffer[945]
馋痞:馋病。
\stopbuffer


\startbuffer[946]
决撒:败露,戳穿。
\stopbuffer


\startbuffer[947]
偏手:外快。指正当收入之外的收入。这里指孙悟空不与大家共分而独得的人参果。
\stopbuffer


\startbuffer[948]
打骸垢:也作『打颏歌』。战栗,哆嗦。
\stopbuffer


\startbuffer[949]
趁脚儿跷:顺势,趁势。
\stopbuffer


\startbuffer[950]
转背摇车:指死后投胎,从婴儿做起。背,死亡。摇车,摇篮。
\stopbuffer


\startbuffer[951]
冰轮:指明月。
\stopbuffer


\startbuffer[952]
炉儿匠:以锔锅、做焊活等为职业的人。
\stopbuffer


\startbuffer[953]
掭(tiàn)子:补锅人的工具,一种细扁的铁棍,也可用来开锁。
\stopbuffer


\startbuffer[954]
拢:牵,拉。
\stopbuffer


\startbuffer[955]
猜枚:一种游戏,多用为酒令。
\stopbuffer


\startbuffer[956]
心肠:心情。
\stopbuffer


\startbuffer[957]
噀(xùn):含在口中而喷出。
\stopbuffer


\startbuffer[958]
吕公绦:衣带名。两头有五色丝绦,传说八仙中的吕洞宾常用之,故名。
\stopbuffer


\startbuffer[959]
渔鼓:旧时道士唱道情用的敲击乐器。用竹筒制作,常和简子一起使用。
\stopbuffer


\startbuffer[960]
九阳巾:道士的冠名,为黑色软帽,脑后有两条飘带。
\stopbuffer


\startbuffer[961]
鸦翎:比喻黑发。
\stopbuffer


\startbuffer[962]
褡裢:一种长方形的布袋,中间开口,两端盛钱物,系在衣外作腰巾,亦可肩负或手提。
\stopbuffer


\startbuffer[963]
膂烈:强硬,刚烈。
\stopbuffer


\startbuffer[964]
搠开门:用力推开门。
\stopbuffer


\startbuffer[965]
摇桩:形容摇晃的样子。
\stopbuffer


\startbuffer[966]
蝇帚儿:即蝇拂子,又称拂尘。
\stopbuffer


\startbuffer[967]
演架:防御,抵抗。
\stopbuffer


\startbuffer[968]
调问:单独审问。
\stopbuffer


\startbuffer[969]
中袖:一种僧道服装,袖大宽广。
\stopbuffer


\startbuffer[970]
一口钟:一种无袖不开衩的长外衣,即斗篷。其形如古乐器的钟,故称。
\stopbuffer


\startbuffer[971]
大殓(liàn):将已装裹的尸体放入棺材。丧礼之一。
\stopbuffer


\startbuffer[972]
拗:方言。舀。一说指沿着锅边浇油。
\stopbuffer


\startbuffer[973]
扎:同『炸』。
\stopbuffer


\startbuffer[974]
抟砂弄汞:比喻枉费力气,无法管束。砂,朱砂,化学成分为硫化汞。道教炼丹,从朱砂中提炼水银,水银易挥发,难以收集,故称。
\stopbuffer


\startbuffer[975]
倒灶:时运不济,倒霉。此处谐音『捣灶』,是孙悟空戏语。
\stopbuffer


\startbuffer[976]
开风:方言。指解手。
\stopbuffer


\startbuffer[977]
腾那:拳术中窜跳躲闪的动作。这里指变化的本领。
\stopbuffer


\startbuffer[978]
光皮散儿:表面,外表。
\stopbuffer


\startbuffer[979]
者:借口。
\stopbuffer


\startbuffer[980]
三茶六饭:谓茶饭周全。
\stopbuffer


\startbuffer[981]
造字(ráng):肮脏。
\stopbuffer


\startbuffer[982]
浆洗:洗净并浆挺衣物。
\stopbuffer


\startbuffer[983]
玉籁:仙乐的美称。
\stopbuffer


\startbuffer[984]
金鳌:传说中的巨龟,驮着蓬莱仙山浮在海上。
\stopbuffer


\startbuffer[985]
棋枰(píng):棋盘,棋局。
\stopbuffer


\startbuffer[986]
醋:惧怯,骇怕。
\stopbuffer


\startbuffer[987]
太乙散数,未入真流:太乙散数,同『太乙散仙』,指成仙不久,还未有『真人』称号的初级仙人。
\stopbuffer


\startbuffer[988]
䎘(sù):飞。
\stopbuffer


\startbuffer[989]
执星筹,添海屋:星筹、海屋,皆为高寿的代称。出自苏轼《东坡志林·三老语》。
\stopbuffer


\startbuffer[990]
添寿、添福、添禄:旧时大户人家常用『添寿』『添福』『添禄』为仆人命名。
\stopbuffer


\startbuffer[991]
褪(tùn):使穿着的衣服或套着的东西脱离。
\stopbuffer


\startbuffer[992]
番番是福:每事都吉利。番番,次次,谐音『翻』。
\stopbuffer


\startbuffer[993]
䠚(wǎ):走路挪蹭的样子。
\stopbuffer


\startbuffer[994]
东华大帝:又称东王公,传说中主管东方的仙人。与西王母对称。
\stopbuffer


\startbuffer[995]
东方朔:汉武帝的臣子,被后世神化。传说他曾三度偷吃蟠桃。
\stopbuffer


\startbuffer[996]
阁气:斗气,置气。
\stopbuffer


\startbuffer[997]
珠树:传说中的仙树,似柏树,叶为珍珠。
\stopbuffer


\startbuffer[998]
玉液锟鋘:传说瀛洲地生玉酒,饮之长生。又产昆吾之铁,制刀切玉如泥。玉液,美酒。锟鋘,古利剑名,用昆吾石冶炼成铁制作的刀剑。
\stopbuffer


\startbuffer[999]
壶中景:指仙境。传说汉代有一卖药老翁,有一只仙壶,可以进出,壶中别有一方天地。
\stopbuffer


\startbuffer[1000]
绸缪:情意殷切。
\stopbuffer


\startbuffer[1001]
四圣:佛教语。指声闻、缘觉、菩萨、佛。
\stopbuffer


\startbuffer[1002]
六凡:即天、人、阿修罗、畜生、饿鬼、地狱。『四圣』『六凡』合称『十界』。
\stopbuffer


\startbuffer[1003]
少林:佛教禅宗祖庭少林寺,泛指佛教寺院,亦引申指禅宗。
\stopbuffer


\startbuffer[1004]
困滞:犹囚困。
\stopbuffer


\startbuffer[1005]
经验:效验,验证。
\stopbuffer


\startbuffer[1006]
清话:高雅不俗的言谈;闲谈。
\stopbuffer


\startbuffer[1007]
小可:犹小小。小可的勾当,指小事。
\stopbuffer


\startbuffer[1008]
旧契:旧交。
\stopbuffer


\startbuffer[1009]
淹留:羁留,逗留。
\stopbuffer


\startbuffer[1010]
无已:不得已。
\stopbuffer


\startbuffer[1011]
巴:靠近,贴近。
\stopbuffer


\startbuffer[1012]
上分:上等,好的。
\stopbuffer


\startbuffer[1013]
气象:气概,气派。
\stopbuffer


\startbuffer[1014]
嘈人:胃不舒服,有恶心的感觉。
\stopbuffer


\startbuffer[1015]
下坠:医家谓腹部感觉沉重,似欲如厕。
\stopbuffer


\startbuffer[1016]
虚情:虚假的情况。
\stopbuffer


\startbuffer[1017]
扳门第:即攀门第。攀附大户人家。
\stopbuffer


\startbuffer[1018]
客子:雇工。
\stopbuffer


\startbuffer[1019]
五黄六月:亦作『五荒六月』。指农历六月,五谷成熟,农事正忙的时节。
\stopbuffer


\startbuffer[1020]
芹献:称礼品菲薄的谦辞。语出《列子·杨朱》,有一人因觉得水芹等野菜好吃,就向乡豪推荐,乡豪尝之,其味苦涩,众人于是哂笑责怪那人,那人觉得很惭愧。
\stopbuffer


\startbuffer[1021]
罢(pí)软:没有主见。
\stopbuffer


\startbuffer[1022]
漏八分儿:又叫『露八分』。曲折地说某件事。
\stopbuffer


\startbuffer[1023]
唆(suō)嘴:搬弄口舌。
\stopbuffer


\startbuffer[1024]
演幌:蒙骗,迷惑。
\stopbuffer


\startbuffer[1025]
撺唆:怂恿挑唆。
\stopbuffer


\startbuffer[1026]
大限:寿数,死期。
\stopbuffer


\startbuffer[1027]
睡:躺,躺下。
\stopbuffer


\startbuffer[1028]
亚腰儿:形容中间细、两头粗的样子。
\stopbuffer


\startbuffer[1029]
逼法:用尽方法,勉强支撑。
\stopbuffer


\startbuffer[1030]
摆站:古时处徒刑的人被发配到驿站中去充驿卒。
\stopbuffer


\startbuffer[1031]
定盘星:原指戥子或秤杆上的第一星儿(重量为零)。多比喻正确的基准或一定的主意。
\stopbuffer


\startbuffer[1032]
弄个风儿:装扮,搞把戏骗人。
\stopbuffer


\startbuffer[1033]
粉骷髅:对美貌妇女的轻蔑之词。意谓姣好容颜不过傅粉骷髅而已。此处指妇女的骸骨。
\stopbuffer


\startbuffer[1034]
潜灵:幽魂。
\stopbuffer


\startbuffer[1035]
败本:谓败乱人的本性。
\stopbuffer


\startbuffer[1036]
凑集:密集,集中。
\stopbuffer


\startbuffer[1037]
昧着惺惺使糊涂:内心明白,表面装糊涂。惺惺,清醒的样子。
\stopbuffer


\startbuffer[1038]
滚鞍下马:迅速地离开鞍子跳到马下。
\stopbuffer


\startbuffer[1039]
软款:温柔,殷勤。
\stopbuffer


\startbuffer[1040]
詀(diān)言詀语:花言巧语,胡说八道。詀,巧言。
\stopbuffer


\startbuffer[1041]
三春:农历正月称孟春,二月称仲春,三月称季春,合称『三春』。这里指春天。
\stopbuffer


\startbuffer[1042]
柘(zhè):木名。桑科。叶可喂蚕。
\stopbuffer


\startbuffer[1043]
捱苦:受苦。
\stopbuffer


\startbuffer[1044]
跳天搠地:上蹿下跳,跳跃不停的样子。
\stopbuffer


\startbuffer[1045]
粘竿:顶端涂有黏质,用来捕鸟的竹竿。
\stopbuffer


\startbuffer[1046]
舚(tiān):本意为以舌取物。此处或用作计数量词。
\stopbuffer


\startbuffer[1047]
噫气:气壅塞而得通,吐气。这里指风。
\stopbuffer


\startbuffer[1048]
『石打乌头粉碎』一诗:诗中出现的『石打(石打穿)、乌头、沙飞(沙飞草)、海马、人参、官桂、朱砂、附子、槟榔、轻粉、红娘子』等词皆为中药名,作者借字面意义或谐音比喻战场景象,是一种文字游戏。如乌头指人头,海马指马匹,人参指人身,附子指父子,槟榔指郎君,红娘子指妻子等。
\stopbuffer


\startbuffer[1049]
斗:拼合,凑。
\stopbuffer


\startbuffer[1050]
狡性:猜疑的心性。
\stopbuffer


\startbuffer[1051]
齁齁(hōu):熟睡时的鼻息声。
\stopbuffer


\startbuffer[1052]
云窦(dòu):云气出没的山洞。
\stopbuffer


\startbuffer[1053]
曙星:拂晓之星。多指启明星。
\stopbuffer


\startbuffer[1054]
丫槎(chá):树枝纵横交叉的样子。
\stopbuffer


\startbuffer[1055]
蜂衙:群蜂早晚聚集,簇拥蜂王,如旧时官吏到上司衙门排班参见。
\stopbuffer


\startbuffer[1056]
仙子种田生白玉:传说有人在路边为路人供水行善,遇仙人赠送一斗石子,让他种下,数年后田中生出白玉。出自《搜神记》。
\stopbuffer


\startbuffer[1057]
伏火:道家炼丹,调低炉火的温度谓『伏火』。
\stopbuffer


\startbuffer[1058]
观风:观赏风光。
\stopbuffer


\startbuffer[1059]
饶:丰富。
\stopbuffer


\startbuffer[1060]
磲石:砗磲(chē qú)的壳。佛教七宝之一。磲,砗磲,一种海洋贝类。
\stopbuffer


\startbuffer[1061]
蘸钢:经过淬火工艺的钢。
\stopbuffer


\startbuffer[1062]
咋:通『炸』。爆裂。
\stopbuffer


\startbuffer[1063]
自性:佛教语。指诸法各自具有的不变不灭之性。
\stopbuffer


\startbuffer[1064]
善友:指佛教教友。
\stopbuffer


\startbuffer[1065]
袖:藏于袖中。
\stopbuffer


\startbuffer[1066]
筛锣:敲锣。
\stopbuffer


\startbuffer[1067]
挟生儿:犹言不待烧煮,活吃,生吃。挟,同『夹』。
\stopbuffer


\startbuffer[1068]
嚌嚌嘈嘈(jì jì cáo cáo):形容声音嘈杂。
\stopbuffer


\startbuffer[1069]
嵂嵂崒崒(lǜ lǜ zú zú):即崒嵂。高耸的样子。
\stopbuffer


\startbuffer[1070]
金汤:金城汤池。金属造的城,沸水流淌的护城河。形容城池险固。
\stopbuffer


\startbuffer[1071]
弁(biàn):古代贵族的一种帽子,搭配礼服。文冠用赤黑色的布做成,叫爵弁;武冠用白鹿皮做成,叫皮弁。武官服皮弁,因称武官为弁。
\stopbuffer


\startbuffer[1072]
籥(yuè):古管乐器。像编管之形,似为排箫之前身。
\stopbuffer


\startbuffer[1073]
御沟:流经宫苑的河道。
\stopbuffer


\startbuffer[1074]
息肩:栖止休息。
\stopbuffer


\startbuffer[1075]
凉德:薄德,缺少仁义。多用为王侯的自谦之词。
\stopbuffer


\startbuffer[1076]
丕基:巨大的基业。丕,大。
\stopbuffer


\startbuffer[1077]
谴:谴责,问罪。
\stopbuffer


\startbuffer[1078]
皇皇后帝:天;天帝。
\stopbuffer


\startbuffer[1079]
须至牒者:『须至……者』,旧时公文执照结束语的习惯用语。表示肯定、劝勉的语气。
\stopbuffer


\startbuffer[1080]
宝:帝王的印信。
\stopbuffer


\startbuffer[1081]
花押:旧时文书契约末尾的草书签名或代替签名的特种符号。
\stopbuffer


\startbuffer[1082]
平安:古人写信,常在信封上写『平安』二字,表示没有重大的变故。
\stopbuffer

\startbuffer[1083]
台次:对人的尊称,类似『台下』。台,敬辞。用于称呼对方或跟对方有关的行为。次,旁边。
\stopbuffer


\startbuffer[1084]
劬(qú)劳:劳累,劳苦。
\stopbuffer


\startbuffer[1085]
片楮(chǔ):片纸。指简短的文字。楮,制造桑皮纸和宣纸的原料,故为纸的代称。
\stopbuffer


\startbuffer[1086]
侵陵:也作『侵凌』,侵犯欺凌。
\stopbuffer


\startbuffer[1087]
随分:谓依例送的一份。
\stopbuffer


\startbuffer[1088]
三停:指身体的总长。臂阔三停是一种夸张的说法。
\stopbuffer


\startbuffer[1089]
下书:投递书信。
\stopbuffer


\startbuffer[1090]
开路神:驱鬼的神祇。旧俗丧家出殡时,做纸糊偶像,行进在送葬队列的最前面,躯体高大,状貌狰狞。
\stopbuffer


\startbuffer[1091]
矬:身材短小。此处指向下降,变得短小。
\stopbuffer


\startbuffer[1092]
蜃(shèn):传说中的蛟属。能吐气成海市蜃楼。亦指海市蜃楼。
\stopbuffer


\startbuffer[1093]
叆叇(ài dài):云气浓盛的样子。
\stopbuffer


\startbuffer[1094]
罗:兜揽,张罗。
\stopbuffer


\startbuffer[1095]
手停:手平。停,古代口语。
\stopbuffer


\startbuffer[1096]
四马攒蹄:比喻两手两脚捆在一起。
\stopbuffer


\startbuffer[1097]
风汛:风声,消息。
\stopbuffer


\startbuffer[1098]
粗卤:即粗鲁。
\stopbuffer


\startbuffer[1099]
撮哄:哄骗;怂恿。
\stopbuffer


\startbuffer[1100]
水性:水随势而流,比喻性情柔弱,无主见。
\stopbuffer


\startbuffer[1101]
鹊尾冠:汉高祖所制的竹皮冠,形似鹊尾。
\stopbuffer


\startbuffer[1102]
褶(xí):一种短身广袖的上衣。
\stopbuffer


\startbuffer[1103]
灭:方言。欺负,亏待。
\stopbuffer


\startbuffer[1104]
口面:口角,争吵。
\stopbuffer


\startbuffer[1105]
配合:谓成为夫妇。
\stopbuffer


\startbuffer[1106]
赤绳曾系足:传说月下老人用红色绳子系住男女双方的脚,二人就注定成为夫妻。
\stopbuffer


\startbuffer[1107]
文引:准予通行的文书。
\stopbuffer


\startbuffer[1108]
骍骍(xīng):赤色。
\stopbuffer


\startbuffer[1109]
白马垂缰:古诗文中常用『马有垂缰之恩』作报恩的典故。前秦世祖皇帝苻坚在一次战役中被慕容冲所袭,落下山涧,他的坐骑跪地将缰绳垂下,苻坚于是抓缰上马,脱离险境。
\stopbuffer


\startbuffer[1110]
满堂红:一种铁质朱漆的蜡烛架。
\stopbuffer


\startbuffer[1111]
躧扁秤砣:比喻急于做某事而不管不顾。
\stopbuffer


\startbuffer[1112]
素手:徒手,空手。
\stopbuffer


\startbuffer[1113]
咤(chà):诧异,惊奇。
\stopbuffer


\startbuffer[1114]
猪浑塘:猪拱的烂泥坑。
\stopbuffer


\startbuffer[1115]
大不祥之事:古人以牲畜说人语为不祥之兆。
\stopbuffer


\startbuffer[1116]
禁:受,耐。
\stopbuffer


\startbuffer[1117]
挣:奋力摆脱;竭力支撑。
\stopbuffer


\startbuffer[1118]
风篷:船帆。
\stopbuffer


\startbuffer[1119]
各样:与众不同。
\stopbuffer


\startbuffer[1120]
雷堆:方言。粗笨,累赘。
\stopbuffer


\startbuffer[1121]
脚色手本:脚色,履历。手本,明清时见上司、座师或贵官所用的名帖。
\stopbuffer


\startbuffer[1122]
专专:专门,特地。
\stopbuffer


\startbuffer[1123]
孤:同『辜』,辜负。
\stopbuffer


\startbuffer[1124]
孔怀:原谓甚相思念。后用来代称兄弟。
\stopbuffer


\startbuffer[1125]
背花:一种杖刑。将背部打破,故称。
\stopbuffer


\startbuffer[1126]
张:张望。
\stopbuffer


\startbuffer[1127]
带挈:带领。
\stopbuffer


\startbuffer[1128]
废坠:因懈怠而中止。
\stopbuffer


\startbuffer[1129]
抢窝:一种儿童游戏。用弯头棍把毛发缠成的球打进洞里。
\stopbuffer


\startbuffer[1130]
顶搭子:孩童剃发时,留在头顶上的一撮头发。
\stopbuffer


\startbuffer[1131]
捽手:甩开手,放手。
\stopbuffer


\startbuffer[1132]
弄嘴:搬嘴,搬弄是非。有时也指耍贫嘴,多嘴。
\stopbuffer


\startbuffer[1133]
心忍:心中难忍。表同情。
\stopbuffer


\startbuffer[1134]
左:骗。
\stopbuffer


\startbuffer[1135]
扛(gāng):冲撞,相斗。
\stopbuffer


\startbuffer[1136]
柳柳惊:压压惊。
\stopbuffer


\startbuffer[1137]
鞠:养育,抚养。此句出自《诗经·小雅·蓼莪》。
\stopbuffer


\startbuffer[1138]
谨:严禁;严防。
\stopbuffer


\startbuffer[1139]
种火心空,顶门腰软:比喻无用。在炉灶中点火,需用致密的拨火木棍在炉膛里搅动。插门上闩,需用坚实的木棍。如果是『心空』『腰软』的木料则没有用。
\stopbuffer


\startbuffer[1140]
羇:同『羁』。束缚。
\stopbuffer


\startbuffer[1141]
泛头:同『犯头』,挑拨,见第三回。
\stopbuffer


\startbuffer[1142]
害酒:过量饮酒引起的不适。
\stopbuffer


\startbuffer[1143]
关窍:诀窍,机关。
\stopbuffer


\startbuffer[1144]
鸡子:鸡蛋。
\stopbuffer


\startbuffer[1145]
估倒:搞,弄,收拾。
\stopbuffer


\startbuffer[1146]
老婆舌头:谓人长于花言巧语,搬弄是非。
\stopbuffer


\startbuffer[1147]
生金棒:指金箍棒。在元杂剧中,孙行者的武器是生金棍,当为如意金箍棒的前身。
\stopbuffer


\startbuffer[1148]
高探马:武术的一种招数。脚尖点地,攻击对方上身。
\stopbuffer


\startbuffer[1149]
大中平:持棍居身体中部,攻击对方胸腹。
\stopbuffer


\startbuffer[1150]
带俸差操:明代对军官的一种处分,免除职事,带俸禄编入军队服役。差操,犹差使。
\stopbuffer


\startbuffer[1151]
起动:敬辞。烦劳,劳驾。
\stopbuffer


\startbuffer[1152]
魇住:谓以法术、符咒镇服、压制。
\stopbuffer


\startbuffer[1153]
一旦:一天之间。
\stopbuffer


\startbuffer[1154]
揭挑:揭露别人的短处。
\stopbuffer


\startbuffer[1155]
飞星:流星。形容速度快。
\stopbuffer


\startbuffer[1156]
东阁:古代称宰相招致、款待宾客的地方。
\stopbuffer


\startbuffer[1157]
紫陌:指京师郊野的道路。
\stopbuffer


\startbuffer[1158]
斗草:一种古代游戏。竞采花草,以对仗形式互报花草名,比赛多寡优劣。或以叶柄相勾拉拽,断者为输。
\stopbuffer


\startbuffer[1159]
传卮:传杯。宴饮中传递酒杯劝酒。
\stopbuffer


\startbuffer[1160]
玉毫斑:玉毫,佛眉间的白毫,佛教谓其有巨大神力。也代指佛像。斑,花白。
\stopbuffer


\startbuffer[1161]
䮷(dú):马行进的样子。此处指马。
\stopbuffer


\startbuffer[1162]
毛皂衲衣:浅黑色的旧衣服。毛皂,浅黑色。衲衣,补缀过的衣服。泛指破旧衣服。
\stopbuffer


\startbuffer[1163]
随分:依据本性;按照本分。
\stopbuffer


\startbuffer[1164]
把势:老手,行家。
\stopbuffer


\startbuffer[1165]
雏儿:年轻而无阅历的新手。
\stopbuffer


\startbuffer[1166]
递解:旧时指把犯人解往远地,由沿途官衙派人轮流押送。
\stopbuffer


\startbuffer[1167]
抄:查点,搜查。
\stopbuffer


\startbuffer[1168]
捽脱:摔脱,挣脱。
\stopbuffer


\startbuffer[1169]
日里鬼:比喻蹊跷的事。
\stopbuffer


\startbuffer[1170]
演漾:迷惑,欺骗。后文亦作『掩样』『魇样』。
\stopbuffer


\startbuffer[1171]
切切在心:牢牢地记在心里。
\stopbuffer


\startbuffer[1172]
芦圩(wéi):芦苇塘。圩,南方低洼地区防水护田的堤,及其所围的田。
\stopbuffer


\startbuffer[1173]
羁勒:管束;驾驭。
\stopbuffer


\startbuffer[1174]
愁帽:亦作『愁帽子』『愁帽儿』。喻指招致忧烦的事情。
\stopbuffer


\startbuffer[1175]
扭捏:生编硬造。
\stopbuffer


\startbuffer[1176]
逗出:引出。
\stopbuffer


\startbuffer[1177]
揾(wèn):拭,擦。
\stopbuffer


\startbuffer[1178]
健猪:方言。公猪。
\stopbuffer


\startbuffer[1179]
獐智:模样,神态;装模作样。
\stopbuffer


\startbuffer[1180]
喑(yīn):缄默,不说话。指声音轻。
\stopbuffer


\startbuffer[1181]
机深:心计深密。
\stopbuffer


\startbuffer[1182]
促掐:刁钻;爱捉弄人。
\stopbuffer


\startbuffer[1183]
面弱:犹面软。顾及情面,板不起面孔来。
\stopbuffer


\startbuffer[1184]
搴(qiān):举;扛。
\stopbuffer


\startbuffer[1185]
啄木虫儿:即啄木鸟。
\stopbuffer


\startbuffer[1186]
馁(něi):饥饿。
\stopbuffer


\startbuffer[1187]
丢灵:活溜;灵巧。
\stopbuffer


\startbuffer[1188]
虫鹥(yì):对禽鸟等小动物的通称。
\stopbuffer


\startbuffer[1189]
窠(kē)巢:动物栖身的地方。窠,昆虫、鸟兽的巢穴。
\stopbuffer


\startbuffer[1190]
瞎帐:比喻白费心力、毫无功效的蠢事。
\stopbuffer


\startbuffer[1191]
对问:问答,对质。
\stopbuffer


\startbuffer[1192]
地头:地方。指终点,目的地。
\stopbuffer


\startbuffer[1193]
影神:指画像。古人认为人的影和像有灵魂寄托,被画像或照镜、摄影,灵魂会脱离本体。
\stopbuffer


\startbuffer[1194]
猪头三牲:用于祭祀的牛、羊、猪。指祭品。
\stopbuffer


\startbuffer[1195]
清醮(jiào)二十四分:醮,道士设坛祈祷,祭祀鬼神的仪式,有清醮和幽醮之分,分别为活人和死人所设。分,指神位。根据神位的多寡,有六分、十二分、二十四分,乃至三千六百分等。
\stopbuffer


\startbuffer[1196]
『城隍』句:人们常到城隍处许愿,故猪八戒有此举。
\stopbuffer


\startbuffer[1197]
蓏(luǒ):瓜类植物的果实。
\stopbuffer


\startbuffer[1198]
星冠:道士的帽子。原指通晓星象的人所戴之冠。
\stopbuffer


\startbuffer[1199]
素券先生:素券,道士的一种高级法位称号。先生,道教称有一定法位的道士。
\stopbuffer


\startbuffer[1200]
淋津:流,滴。
\stopbuffer


\startbuffer[1201]
禳(ráng)星:禳除凶星。禳,古代除邪消灾的祭祀名,泛指除邪消灾。
\stopbuffer


\startbuffer[1202]
一命之人:命运相同的人。
\stopbuffer


\startbuffer[1203]
颠倒:反而。
\stopbuffer


\startbuffer[1204]
《北斗经》:全名《太上玄灵北斗本命延生真经》,道教认为北斗诸神掌管凡人的罪福善恶,念此经可以消灾避难。
\stopbuffer


\startbuffer[1205]
外好里丫槎:外表好看,内里丫杈。比喻人表面很好,实际却难相处。
\stopbuffer


\startbuffer[1206]
这们:这么。们,相当于『么』。
\stopbuffer


\startbuffer[1207]
黑杀神:小说中谓凶星,恶神。一说即『黑煞神』『黑煞将军』,北方神,宋代被尊为翊圣真君,后为『北极四圣』之一。
\stopbuffer


\startbuffer[1208]
不应:古代法律名词。谓非有意犯罪。
\stopbuffer


\startbuffer[1209]
坐:判罪,治罪。
\stopbuffer


\startbuffer[1210]
发放:释放,遣散。
\stopbuffer


\startbuffer[1211]
放刁:耍无赖,用狡猾的手段使人为难。
\stopbuffer


\startbuffer[1212]
尾子一抉(jué):尾子,即尾巴。抉,此处义同『撅』,翘。
\stopbuffer


\startbuffer[1213]
善庆:谓善行多福。语出《周易·坤》:『积善之家,必有馀庆。』
\stopbuffer


\startbuffer[1214]
皂纛(dào)旗:真武大帝所执的黑色旗帜。
\stopbuffer


\startbuffer[1215]
屙绵花屎:拖延,磨时间。
\stopbuffer


\startbuffer[1216]
伴当:随从的差役或仆人。
\stopbuffer


\startbuffer[1217]
顶上:顶礼拜上。极表尊敬。
\stopbuffer


\startbuffer[1218]
海底眼:底细,隐秘。
\stopbuffer


\startbuffer[1219]
仵(wŭ):同『捂』。
\stopbuffer


\startbuffer[1220]
渫:通『煠』(zhá),炸。
\stopbuffer


\startbuffer[1221]
泪出痛肠:指因内心难过而流泪。痛肠,悲痛忧伤的心肠。
\stopbuffer


\startbuffer[1222]
使碎六叶连肝肺:中医论肺有六叶两耳,在肝之旁,故有此说。
\stopbuffer


\startbuffer[1223]
用尽三毛七孔心:指心脏。比喻心思,心机。古说心重十二两,有七孔、三毛,主藏神。
\stopbuffer


\startbuffer[1224]
减妆:古代妇女的梳妆匣子。
\stopbuffer


\startbuffer[1225]
塌嘴:多嘴。
\stopbuffer


\startbuffer[1226]
皂隶:皂、隶皆为古代贱役。后专以称旧衙门里的差役。
\stopbuffer


\startbuffer[1227]
抵强:抵抗,逞强。
\stopbuffer


\startbuffer[1228]
大四声:古时官员出行,衙役在前喝道开路,常连呼四声『行人回避』。
\stopbuffer


\startbuffer[1229]
为人:做人,犹体面。
\stopbuffer


\startbuffer[1230]
四拜之礼:古代表示庄重的拜礼。多用于幼者对长者行礼。
\stopbuffer


\startbuffer[1231]
箫韶:舜所奏的乐名。后指美妙的仙乐。
\stopbuffer


\startbuffer[1232]
博山炉:古香炉名。因炉盖上的造型似传闻中的海中名山博山而得名。后指名贵的香炉。
\stopbuffer


\startbuffer[1233]
献生:指打猎后先献给神灵或尊长。
\stopbuffer


\startbuffer[1234]
镔铁:泛指精铁。
\stopbuffer


\startbuffer[1235]
靴靿(yào):靴筒。靿,靴或袜子的筒。
\stopbuffer


\startbuffer[1236]
自家搓根绳儿:指自己搓根绳上吊。
\stopbuffer


\startbuffer[1237]
爬蹅:来回爬动磨搓。
\stopbuffer


\startbuffer[1238]
抹:扫视。
\stopbuffer


\startbuffer[1239]
酾(shī)酒:斟酒。
\stopbuffer


\startbuffer[1240]
关防:防范。
\stopbuffer


\startbuffer[1241]
风息:消息,情况。
\stopbuffer


\startbuffer[1242]
演:缓步行进。
\stopbuffer


\startbuffer[1243]
扯个腿子:指单腿下跪。
\stopbuffer


\startbuffer[1244]
溺(niào):同『尿』。
\stopbuffer


\startbuffer[1245]
动耽:义同『动弹』。
\stopbuffer


\startbuffer[1246]
火上弄冰:比喻事情容易成功。
\stopbuffer


\startbuffer[1247]
焦:着急。
\stopbuffer


\startbuffer[1248]
解化:分解,变化。道教认为太上老君是道的化身,天、地、人、物皆由其所化而来,包括伏羲、女娲。
\stopbuffer


\startbuffer[1249]
乾宫夬地:指北偏西的方位。八卦配八方,将一个平面的圆分为八个圆弧,每个圆弧为一宫,乾卦所在的方位即乾宫,包括正北至正西北区域。在此基础上,八卦两两组合,生出六十四卦,将圆继续分成六十四个方位。每宫包含八个卦象,其中乾宫包括乾(天)、夬(泽天)、大有(火天)、大壮(雷天)、小畜(风天)、需(水天)、大畜(山天)、泰(地天)。乾位于正北,夬位于正北偏西。
\stopbuffer


\startbuffer[1250]
圈盘腿:即罗圈腿。
\stopbuffer


\startbuffer[1251]
漷漷(huò)索索:形容水声。漷,水势相激貌。
\stopbuffer


\startbuffer[1252]
大端:大约,大抵。
\stopbuffer


\startbuffer[1253]
发课的筒子:卜卦的竹筒,内装写有各种运势的竹签,占卜时摇动竹筒抽取竹签。发课,亦作『起课』,卜卦、占算法之一。
\stopbuffer


\startbuffer[1254]
扛丧:举哀,哭泣。
\stopbuffer


\startbuffer[1255]
《受生经》:又名《寿生经》《受生尊经》,常用于超度亡魂。
\stopbuffer


\startbuffer[1256]
打觑:犹打趣。
\stopbuffer


\startbuffer[1257]
造字(juàn)户:圈养的牲畜。比野生的皮肉柔嫩,更易煮烂。造字,养家畜的围栏。
\stopbuffer


\startbuffer[1258]
讨死:犹找死。
\stopbuffer


\startbuffer[1259]
事不谐矣、难矣乎哉:二句是古书中用来打趣的话。『事不谐矣』出自《后汉书·宋弘传》,汉光武帝命宋弘抛弃原妻,迎娶公主,宋弘以『糟糠之妻不下堂』拒绝了,光武帝于是对公主说『事不谐矣』。不谐,不成。『难矣哉』是《论语》中孔子常说的话。
\stopbuffer


\startbuffer[1260]
东南丙丁火:古代以十天干配五行四方,丙、丁属火,配南方。
\stopbuffer


\startbuffer[1261]
离宫:八卦中离宫在正南方。离,八卦之一,代表火。
\stopbuffer


\startbuffer[1262]
熯(hàn):烘烤,燃烧。
\stopbuffer


\startbuffer[1263]
红绡:红色薄绸,比喻闪电。
\stopbuffer


\startbuffer[1264]
消耗:音信,声息。
\stopbuffer


\startbuffer[1265]
急递铺:金、元、明代传递文书的驿站。十里或十五里、二十五里设一铺。凡遇官府公文至,即行递送,不分昼夜,风雨无阻。
\stopbuffer


\startbuffer[1266]
拢:接近,靠近。
\stopbuffer


\startbuffer[1267]
锁子甲:一种铠甲。其甲五环相衔,一环受镞,诸环拱护,箭不能入。亦泛指制作精细的铠甲。
\stopbuffer


\startbuffer[1268]
毛团:泛指禽兽类动物。用于对人的咒骂。
\stopbuffer


\startbuffer[1269]
瞽(gǔ)者:盲人。
\stopbuffer


\startbuffer[1270]
玉局宝座:四脚弯曲的玉座。局,弯曲。相传局脚玉床为太上老君的宝座。
\stopbuffer


\startbuffer[1271]
钤束:管束,约束。
\stopbuffer


\startbuffer[1272]
掯(kèn)害:刁难加害。掯,压制,刁难。
\stopbuffer


\startbuffer[1273]
不的:不可靠,不确实。
\stopbuffer


\startbuffer[1274]
大罗天:道教所称三十六天中最高一重天。
\stopbuffer


\startbuffer[1275]
办:具有,抱有。
\stopbuffer


\startbuffer[1276]
堂屋:正屋。
\stopbuffer


\startbuffer[1277]
山魅:山中精怪。
\stopbuffer


\startbuffer[1278]
成器:成精。
\stopbuffer


\startbuffer[1279]
『自从益智登山盟』诗:益智、王不留行、三棱子、马兜铃、荆芥、茯苓、防己、竹沥、茴香,皆为中药名,取其谐音或字义成诗,如王不留行指皇帝送行之事,三棱子指三徒弟,马兜铃指马的铃铛,荆芥指经,茯苓指佛,茴香指回乡。
\stopbuffer


\startbuffer[1280]
钓纶:钓鱼线。
\stopbuffer


\startbuffer[1281]
松关:柴门。
\stopbuffer


\startbuffer[1282]
妙高:梵语『须弥山』的意译。
\stopbuffer


\startbuffer[1283]
昙花:佛教比喻人生苦短,佛法宝贵。此处指代佛法。
\stopbuffer


\startbuffer[1284]
贝叶:古代印度人用以写经的树叶。亦借指佛经。
\stopbuffer


\startbuffer[1285]
崚嶒(líng céng):高耸突兀。这里指骨节显露的样子。
\stopbuffer


\startbuffer[1286]
倒座观音普度南海之相:佛寺常在释迦牟尼正殿背后塑海岛和海岛观音像。
\stopbuffer


\startbuffer[1287]
打勤劳:做杂务劳动。
\stopbuffer


\startbuffer[1288]
羁迟:淹留、耽搁。
\stopbuffer


\startbuffer[1289]
二十五条达摩衣:一种僧衣。僧衣由布条缝制而成,有五、七、九至二十五条不等。其中九条以上为正式服装,以二十五条为最,由二十五条横布条,每条辅以四长一短的竖布条缀成。
\stopbuffer


\startbuffer[1290]
达公鞋:达摩和尚的鞋子。泛指僧鞋。
\stopbuffer


\startbuffer[1291]
方上人:即方外人。不涉尘世或不拘世俗礼法的人,多指僧、道、隐者。
\stopbuffer


\startbuffer[1292]
牙香:即香角。沉香之别名。
\stopbuffer


\startbuffer[1293]
琉璃:佛前的琉璃油灯。
\stopbuffer


\startbuffer[1294]
捣叉子:找岔子,故意挑毛病。捣,胡搞。
\stopbuffer


\startbuffer[1295]
恶躁:凶猛,凶恶。
\stopbuffer


\startbuffer[1296]
没脊骨:不正经,没规矩。
\stopbuffer


\startbuffer[1297]
折作:因做坏事或过分的事而受到报应。折,损失。
\stopbuffer


\startbuffer[1298]
杠子:粗长的棍棒。
\stopbuffer


\startbuffer[1299]
锅门:灶门。火灶进燃料及出灰的洞口。
\stopbuffer


\startbuffer[1300]
度牒:僧道出家,由官府发给凭证,称之为『度牒』。
\stopbuffer


\startbuffer[1301]
玉宇:指太空。
\stopbuffer


\startbuffer[1302]
高剔银缸:剔,挑起。缸,油灯。
\stopbuffer


\startbuffer[1303]
瞋(chēn):瞪着眼。指生气,恼火。
\stopbuffer


\startbuffer[1304]
舍眼:睁眼,抬眼。
\stopbuffer


\startbuffer[1305]
珪:同『圭』。古帝王或诸侯所持的一种长条形玉礼器,上端呈圆形或剑形,下端方形。
\stopbuffer


\startbuffer[1306]
荒歉:荒年歉收。
\stopbuffer


\startbuffer[1307]
尚义:重义气;崇尚道义。
\stopbuffer


\startbuffer[1308]
伸诉:义同『申诉』。
\stopbuffer


\startbuffer[1309]
夜游神:在夜间巡行的神。
\stopbuffer


\startbuffer[1310]
画虎刻鹄:比喻好事做不成,反变了坏事。东汉马援告诫其侄,希望他们学习敦厚谦谨的人,不要学习豪侠仗义的人。前者学不像,至少还做老实人,就如刻鹄不像,至少还像鸭子。而后者学不像,就会成为轻薄的人,就如画虎不成,反而画成了狗。
\stopbuffer


\startbuffer[1311]
表记:信物。
\stopbuffer


\startbuffer[1312]
唿哨:同『呼哨』。口哨。
\stopbuffer


\startbuffer[1313]
貔(pí):传说中的一种野兽,似熊,一说似虎。比喻勇猛的军队。
\stopbuffer


\startbuffer[1314]
裹肚:宋元时男子长衣外包裹腰肚的绣袍肚。
\stopbuffer


\startbuffer[1315]
札:铠甲的叶片。
\stopbuffer


\startbuffer[1316]
半朝銮驾:古时皇帝出巡的马车仪仗称为銮驾,太子的仪仗为皇帝的一半左右,因此小说、戏曲中常称『半朝銮驾』。
\stopbuffer


\startbuffer[1317]
魆魆(xū):暗暗,悄悄。
\stopbuffer


\startbuffer[1318]
造字(bá):走路一跛一拐的样子。
\stopbuffer


\startbuffer[1319]
爱小:谓贪小利。
\stopbuffer


\startbuffer[1320]
空头祸:没来由引出的祸事。
\stopbuffer


\startbuffer[1321]
争奈:怎奈,无奈。
\stopbuffer


\startbuffer[1322]
笼:方言。把手或东西放在袖筒里。
\stopbuffer


\startbuffer[1323]
正阳门、后宰门:二名借用自明代京城皇宫的正门和后门。
\stopbuffer


\startbuffer[1324]
受生:投生,投胎。
\stopbuffer


\startbuffer[1325]
羑里:殷代监狱名。传说商纣王在此囚禁过周文王姬昌。代指监狱。
\stopbuffer


\startbuffer[1326]
唵㘕净法界:佛教密宗有『净法界真言』的咒语,为『唵㘕』二字,后被道教袭用。
\stopbuffer


\startbuffer[1327]
失惊打怪:犹大惊小怪。
\stopbuffer


\startbuffer[1328]
萧何的律法:萧何制定了汉代的法律。后以『萧何律』指代法律。
\stopbuffer


\startbuffer[1329]
执照:证明,凭据。
\stopbuffer


\startbuffer[1330]
干碍:妨碍;关涉。
\stopbuffer


\startbuffer[1331]
骨榇(chèn):死人埋葬腐烂后剩下的骨头。借指尸体。
\stopbuffer


\startbuffer[1332]
对头:冤家,仇人。指诉讼的对方。
\stopbuffer


\startbuffer[1333]
头觉:第一觉。指刚睡着的一段时间。
\stopbuffer


\startbuffer[1334]
梆铃:梆子和铃。打更的响器。
\stopbuffer


\startbuffer[1335]
紧急:严密,严紧。
\stopbuffer


\startbuffer[1336]
里罗城墙:指内城墙。罗城,城墙外的大城。
\stopbuffer


\startbuffer[1337]
䟕(chà):踏。
\stopbuffer


\startbuffer[1338]
垓垓(gāi):多而杂乱貌。
\stopbuffer


\startbuffer[1339]
缄(jiān)书:书信。
\stopbuffer


\startbuffer[1340]
没头蹲:突然沉入水中,水没过头。
\stopbuffer


\startbuffer[1341]
负水:义同『洑水』,游泳。
\stopbuffer


\startbuffer[1342]
打个猛子:也叫『扎猛子』。游泳时头朝下迅速钻到水里。
\stopbuffer


\startbuffer[1343]
牌楼:一种装饰性建筑。有二根或四根并列直柱,上有檐额。
\stopbuffer


\startbuffer[1344]
其实:实在,确实。
\stopbuffer


\startbuffer[1345]
軃:此处指躺着。
\stopbuffer


\startbuffer[1346]
腌臜(ā zā):肮脏,不干净。
\stopbuffer


\startbuffer[1347]
一头水:谓没有主见,像水一样,哪头重就向那头倒。
\stopbuffer


\startbuffer[1348]
邪风:没有根据的消息,胡言。
\stopbuffer


\startbuffer[1349]
招承:招供承认。
\stopbuffer


\startbuffer[1350]
扭搜:硬挤。
\stopbuffer


\startbuffer[1351]
数黄道黑:说长道短,挑唆是非。
\stopbuffer


\startbuffer[1352]
没搭撒:同『没搭煞』。没有出息,无用。
\stopbuffer


\startbuffer[1353]
稽迟:迟延,滞留。
\stopbuffer


\startbuffer[1354]
捘(zùn):搓捏。
\stopbuffer


\startbuffer[1355]
缠帐:纠缠,搅绕。
\stopbuffer


\startbuffer[1356]
手脚不稳:指有偷窃行为。
\stopbuffer


\startbuffer[1357]
皮笊篱(zhào li):笊篱,用竹篾或铁丝、柳条编成网状用于捞物沥水的器具。皮笊篱没有孔,可以连汤一起盛出。
\stopbuffer


\startbuffer[1358]
重楼:道教语。称气管或喉咙。
\stopbuffer


\startbuffer[1359]
明堂:指喉下或肺。
\stopbuffer


\startbuffer[1360]
靸:只把脚尖伸进鞋内,拖着走。
\stopbuffer


\startbuffer[1361]
坠镫:向下拉正马镫,侍候尊长上马。亦表示对人敬仰,甘执贱役之意。
\stopbuffer


\startbuffer[1362]
杌樗(wù chū):光秃的臭椿树。喻不成材料,没有出息。
\stopbuffer


\startbuffer[1363]
撒村:说粗鲁话。
\stopbuffer


\startbuffer[1364]
端门:宫殿的正南门。
\stopbuffer


\startbuffer[1365]
铜斗:形容富足而牢固。
\stopbuffer


\startbuffer[1366]
愚浊:愚昧昏浊。
\stopbuffer


\startbuffer[1367]
敷演:陈述而加以发挥。
\stopbuffer


\startbuffer[1368]
行童:供寺院役使的和尚。
\stopbuffer


\startbuffer[1369]
心苗:内心。
\stopbuffer


\startbuffer[1370]
切手:绝招,狠手。
\stopbuffer


\startbuffer[1371]
三伏靛:制靛必须在盛夏,蓝草易烂,故称。
\stopbuffer


\startbuffer[1372]
骟(shàn):对牲畜的阉割。
\stopbuffer


\startbuffer[1373]
南面称孤:面朝南坐,自称孤家。指统治一方,称帝称王。
\stopbuffer


\startbuffer[1374]
喜容:生时的画像。
\stopbuffer


\startbuffer[1375]
捧毂推轮:扶着车毂推车前进。古代帝王任命将帅时的隆重礼遇。
\stopbuffer


\startbuffer[1376]
阁泪:含着眼泪。
\stopbuffer


\startbuffer[1377]
搊(zŏu):抓,揪。
\stopbuffer


\startbuffer[1378]
明轿:敞开的轿子。
\stopbuffer


\startbuffer[1379]
倚草附木:谓精灵倚托草木等物而成妖作怪。
\stopbuffer


\startbuffer[1380]
卯酉星法:卯时日出,酉时日落,意指『两不相见』。星法,推测祸福的一种方术。
\stopbuffer


\startbuffer[1381]
无籍之人:不纳税的人。指无赖。
\stopbuffer


\startbuffer[1382]
黄沙盖面:喻指死亡。
\stopbuffer


\startbuffer[1383]
架空:虚浮不实,没有基础。
\stopbuffer


\startbuffer[1384]
姑娘:姑母。
\stopbuffer


\startbuffer[1385]
扛(gāng):阻拦,横架。
\stopbuffer


\startbuffer[1386]
浮财:指金钱、粮食、衣服、什物等动产。
\stopbuffer


\startbuffer[1387]
解:脱落。
\stopbuffer


\startbuffer[1388]
停留长智:谓耽搁得久了,会想出主意来。
\stopbuffer


\startbuffer[1389]
灯草:剥去外皮的灯心草的茎。质轻。
\stopbuffer


\startbuffer[1390]
失惊:吃惊。
\stopbuffer


\startbuffer[1391]
常例钱:按惯例送的钱。古时官员、吏役向人勒索的名目之一。
\stopbuffer


\startbuffer[1392]
跳风:翻跟斗。
\stopbuffer


\startbuffer[1393]
富胎:即富态。
\stopbuffer


\startbuffer[1394]
白了面皮:生气翻脸。人发怒时脸色变白。
\stopbuffer


\startbuffer[1395]
胡柴:胡说,胡扯。
\stopbuffer


\startbuffer[1396]
哥哥:对子侄辈的称呼。
\stopbuffer


\startbuffer[1397]
人气:指人的意气、感情等。
\stopbuffer


\startbuffer[1398]
南北:计谋;本领。
\stopbuffer


\startbuffer[1399]
騃(ái):愚,呆。
\stopbuffer


\startbuffer[1400]
着了忙:着慌,着急。
\stopbuffer


\startbuffer[1401]
红眼马郎:赤眼鳟的别名。
\stopbuffer


\startbuffer[1402]
昆玉:称人兄弟的敬辞。
\stopbuffer


\startbuffer[1403]
当:抵挡。
\stopbuffer


\startbuffer[1404]
炮燥:灼热,燥热。
\stopbuffer


\startbuffer[1405]
见像作佛:佛教认为见到佛像而作真佛想,可以得到保佑。这里戏称八戒见到假佛而当作真佛。
\stopbuffer


\startbuffer[1406]
驮梁:方言。指人字梁。
\stopbuffer


\startbuffer[1407]
肿头天瘟:一种病。又叫大头瘟,发病者头面部红肿。
\stopbuffer


\startbuffer[1408]
腥风:喻凶残的气氛。
\stopbuffer


\startbuffer[1409]
补衬:补衣服的碎布块。
\stopbuffer


\startbuffer[1410]
门枢:门扇的转轴。
\stopbuffer


\startbuffer[1411]
旗枪:战争工具。比喻威风,士气。
\stopbuffer


\startbuffer[1412]
高作:高明的招数。
\stopbuffer


\startbuffer[1413]
月斋:佛教徒在每年正月、五月、九月斋戒。
\stopbuffer


\startbuffer[1414]
雷斋:奉祀雷神的人所持的斋戒。在每年雷神辛元帅神诞(六月十五日)及每月的三个辛日和初六茹素。
\stopbuffer


\startbuffer[1415]
子平:传说宋有徐子平,精于星命之学,后世术士宗之,因以『子平』指星命之学。是一种根据星象或生辰八字推算命运的术数。
\stopbuffer


\startbuffer[1416]
捏脓:编造假话。
\stopbuffer


\startbuffer[1417]
鬼子母:佛教神。原为专吃小儿的恶鬼,后经佛度化,转为保护小儿之神。早期的《西游记》故事中,有鬼子母之子爱奴儿诱捉唐僧,被佛祖降服的情节,或为红孩儿故事的前身。
\stopbuffer


\startbuffer[1418]
好是:大概是。
\stopbuffer


\startbuffer[1419]
人事:指人情,礼品。
\stopbuffer


\startbuffer[1420]
帮泥:龟的代称。相传大禹治水的时候,有玄龟负青泥相助。
\stopbuffer


\startbuffer[1421]
二十四拜:明代的一种正式礼仪。朱元璋时规定祭祀天地、宗庙,各环节共设二十四拜。
\stopbuffer


\startbuffer[1422]
五十三参:佛教传说中善财童子受文殊菩萨指点,南行五十三处,参访名师,听受佛法,才终成正果。因第二十八参参拜的是观音,民间遂以善财童子为观音的座下童子。这里是小说的化用,仅指参拜观音。
\stopbuffer


\startbuffer[1423]
拙口钝腮:犹拙嘴笨舌。
\stopbuffer


\startbuffer[1424]
热擦:发急,恼火。
\stopbuffer


\startbuffer[1425]
潦(lăo):大水。
\stopbuffer


\startbuffer[1426]
渺弥:水流旷远貌。
\stopbuffer


\startbuffer[1427]
心术:心计。
\stopbuffer


\startbuffer[1428]
暖寿:旧俗于寿诞之前一日置酒食祝贺曰『暖寿』。
\stopbuffer


\startbuffer[1429]
不然:不以为是。
\stopbuffer


\startbuffer[1430]
迩(ěr):近。
\stopbuffer


\startbuffer[1431]
菲筵:菲薄的宴席。
\stopbuffer


\startbuffer[1432]
元戎:指主将,统帅。
\stopbuffer


\startbuffer[1433]
拿捏:要挟,刁难。
\stopbuffer


\startbuffer[1434]
太昊乘震:指春天。太昊,伏羲氏,为东方之神、太阳神,主管春天。震,八卦之一,位在东方,五行属木,主春。
\stopbuffer


\startbuffer[1435]
勾芒御辰:勾芒,木神名。御辰,掌管时辰。
\stopbuffer


\startbuffer[1436]
云水:指云游四方的僧道,其行迹如行云流水,故称。
\stopbuffer


\startbuffer[1437]
道情词:用渔鼓和简板伴奏,道士演唱道教故事的曲子。
\stopbuffer


\startbuffer[1438]
亢旱:大旱。
\stopbuffer


\startbuffer[1439]
倒悬:把人或物倒挂起来,后以人之倒挂比喻处境极其困苦,以物之倒挂比喻极其贫困。
\stopbuffer


\startbuffer[1440]
星星:一点点。
\stopbuffer


\startbuffer[1441]
靠胸贴肉:形容十分亲密。
\stopbuffer


\startbuffer[1442]
告斗:也称拜斗。礼拜北斗星。道教祈祷的一种。
\stopbuffer


\startbuffer[1443]
长揖:拱手高举,自上而下行礼。
\stopbuffer


\startbuffer[1444]
长俊:长进,上进。
\stopbuffer


\startbuffer[1445]
长川:经常,不断。
\stopbuffer


\startbuffer[1446]
快手:旧时衙署中专管缉捕的差役。
\stopbuffer


\startbuffer[1447]
病状:生病的证明文件。
\stopbuffer


\startbuffer[1448]
死状:死亡的证明文件。
\stopbuffer


\startbuffer[1449]
『将车儿拽过两关』句:这里借用了道教内丹术术语。内丹术认为静坐运气时,真气要经过人体内的夹脊小路,上达头中泥丸宫,下至脚底涌泉穴双关,再回到丹田,行满一周天。
\stopbuffer


\startbuffer[1450]
咒水发檄:咒水,对水行咒,作法后的水能涤荡污秽。发檄,诵读召请或驱使神灵的咒文。
\stopbuffer


\startbuffer[1451]
踏罡布斗:道教法师祈天或作法的步伐,表示脚踏在天宫罡星斗宿之上。罡,北斗斗柄最末端的星。
\stopbuffer


\startbuffer[1452]
表白:佛、道教中专主宣唱的人。
\stopbuffer


\startbuffer[1453]
通脚:两人同卧而伸脚的方向相反。
\stopbuffer


\startbuffer[1454]
烧果:烧饼。
\stopbuffer


\startbuffer[1455]
衬饭:斋供的饭食。
\stopbuffer


\startbuffer[1456]
畜:方言。熏,呛。
\stopbuffer


\startbuffer[1457]
灒(zàn):方言。溅。
\stopbuffer


\startbuffer[1458]
簇盘:犹拼盘。
\stopbuffer


\startbuffer[1459]
饼锭:厚而大的烧饼。
\stopbuffer


\startbuffer[1460]
油煠:油炸食品。
\stopbuffer


\startbuffer[1461]
倥(kōng)着脸:低垂着脸。倥,俯、低、垂。
\stopbuffer


\startbuffer[1462]
申文:呈文。
\stopbuffer


\startbuffer[1463]
镇日:整天,从早到晚。
\stopbuffer


\startbuffer[1464]
寂寂:犹悄悄。
\stopbuffer


\startbuffer[1465]
吕梁洪:在今江苏省徐州市铜山区东南。有上下二洪,泗水自此流过,巨石齿列,波流汹涌。
\stopbuffer


\startbuffer[1466]
坂:山坡,斜坡。
\stopbuffer


\startbuffer[1467]
酣造字(dān):形容酒变质后的腐败之味。造字,劣酒。
\stopbuffer


\startbuffer[1468]
道号:修道者的别号。此处指道士。
\stopbuffer


\startbuffer[1469]
坐名:指名。
\stopbuffer


\startbuffer[1470]
筋节:指言语上的分寸或关键。
\stopbuffer


\startbuffer[1471]
混赖:耍赖,抵赖。
\stopbuffer


\startbuffer[1472]
煌煌:明亮的样子。
\stopbuffer


\startbuffer[1473]
风色:风;风势。
\stopbuffer


\startbuffer[1474]
志诚:诚实。
\stopbuffer


\startbuffer[1475]
九天应元雷声普化天尊:道教神仙体系中雷部的最高天神。主宰万物生杀枯荣、善恶赏罚、行云布雨、斩妖伏魔等事。
\stopbuffer


\startbuffer[1476]
无款:没有派头、样儿。
\stopbuffer


\startbuffer[1477]
『拿上柴蓬』句:古时有让求雨者坐在柴堆上,将其焚烧,以感动上天,使之降雨的习俗。
\stopbuffer


\startbuffer[1478]
冉冉:迷离的样子。
\stopbuffer


\startbuffer[1479]
蛰:动物冬眠、藏伏。据说春天的雷声可以将它们唤醒。
\stopbuffer


\startbuffer[1480]
造字检:疑为『瓦笕』或『瓦枧』之误。瓦,竹瓦。笕,一作枧,表以竹通水。
\stopbuffer


\startbuffer[1481]
抓了丢去:指抢了功,占了先。
\stopbuffer


\startbuffer[1482]
冒了天风:指吹了风。天风,风。风行天空,故称。
\stopbuffer


\startbuffer[1483]
昆仲:兄弟。长曰兄,次曰仲。
\stopbuffer


\startbuffer[1484]
破烂流丢:破烂不堪的样子。流丢,方言,常用于形容词后。
\stopbuffer


\startbuffer[1485]
梓童:皇后的自称,或皇帝对皇后的称呼。
\stopbuffer


\startbuffer[1486]
积年:指有多年实践、经验丰富的人。
\stopbuffer


\startbuffer[1487]
抵:抵换,代替。
\stopbuffer


\startbuffer[1488]
纥络(gē lào):方言。角落,旮旯。
\stopbuffer


\startbuffer[1489]
待诏:原为官名,后也称理发师为『待诏』。
\stopbuffer


\startbuffer[1490]
扁食:方言。水饺、馄饨之类的面食。
\stopbuffer


\startbuffer[1491]
温汤:温泉。汤,热水。
\stopbuffer


\startbuffer[1492]
禅和子:参禅人的通称。和,谓和尚。
\stopbuffer


\startbuffer[1493]
荡荡:洗涤,清除。
\stopbuffer


\startbuffer[1494]
劖(chán)言讪(shàn)语:刻薄嘲讽玩笑之言。劖,讽刺。讪,讥笑。
\stopbuffer


\startbuffer[1495]
氼(nì)子:潜水。氼,同『溺』,沉溺。
\stopbuffer


\startbuffer[1496]
背心:背脊。
\stopbuffer


\startbuffer[1497]
凉浆水饭:凉浆,冷菜汤,或冷酒。水饭,稀饭,也指祭奠时用的酒、饭。
\stopbuffer


\startbuffer[1498]
纸马:旧俗祭祀时所用的神像纸,祭毕随即焚化。
\stopbuffer


\startbuffer[1499]
陌纸:指纸钱。陌,通『佰』,为计算钱数的单位。
\stopbuffer


\startbuffer[1500]
黄钱:用黄表纸折成,焚化给鬼神的纸钱。
\stopbuffer


\startbuffer[1501]
干净:犹言敢情。
\stopbuffer


\startbuffer[1502]
支吾:支撑,支持住不倒下。
\stopbuffer


\startbuffer[1503]
圆明:佛教语。谓彻底领悟。
\stopbuffer


\startbuffer[1504]
鱼津:鱼在水中蹿跃所溅起的水泡。
\stopbuffer


\startbuffer[1505]
扳罾(zēng):拉起罾网捕鱼。罾,一种用木棍或竹竿做支架的方形鱼网。
\stopbuffer


\startbuffer[1506]
钹(bó):打击乐器。铜制,圆形,中部隆起如半球状。以两片为一副,相击发声。
\stopbuffer


\startbuffer[1507]
举事:行事,办事。
\stopbuffer


\startbuffer[1508]
此处原注『此时入夜矣』。
\stopbuffer


\startbuffer[1509]
寒砧(zhēn):指寒秋的捣衣声。砧,捣衣石。
\stopbuffer


\startbuffer[1510]
赶斋:指僧人化斋。
\stopbuffer


\startbuffer[1511]
门限:门槛。
\stopbuffer


\startbuffer[1512]
满散:做佛事或道场期满谢神的一种仪式。
\stopbuffer


\startbuffer[1513]
盛介:对他人仆役的尊称。
\stopbuffer


\startbuffer[1514]
噇(chuánɡ):没有节制地大吃大喝。
\stopbuffer


\startbuffer[1515]
青苗斋:祈求庄稼生长的斋事。
\stopbuffer


\startbuffer[1516]
了场斋:或为秋季祈祷丰年的斋事。
\stopbuffer


\startbuffer[1517]
寄库:指生前烧纸钱,托管冥官,以备死后在地府使用。
\stopbuffer


\startbuffer[1518]
填还:道教认为人在得人身时向地府冥司借贷了受生钱,因而在活着的时候需要还债。
\stopbuffer


\startbuffer[1519]
祭赛:祭祀酬神。赛,酬报。
\stopbuffer


\startbuffer[1520]
醴(lǐ):甜酒。
\stopbuffer


\startbuffer[1521]
三十斤为一秤:按古代计量单位,十五斤为一秤,三十斤为一钧。此处『三十斤为一秤』出处不详,或为误写。
\stopbuffer


\startbuffer[1522]
偏出:义同『庶出』,古称妾所生的子女。
\stopbuffer


\startbuffer[1523]
缴缠:费用、开销。
\stopbuffer


\startbuffer[1524]
灵感:指神灵的感应。也形容感觉敏锐。
\stopbuffer


\startbuffer[1525]
好道:此处表示反诘,相当于莫非,难道。
\stopbuffer


\startbuffer[1526]
大老:称呼排行居长的人,犹言老大。
\stopbuffer


\startbuffer[1527]
弄精神:伤神,费心思。
\stopbuffer


\startbuffer[1528]
宝眷:称人眷属的敬辞。
\stopbuffer


\startbuffer[1529]
上台盘:谓有脸面,有身份。
\stopbuffer


\startbuffer[1530]
刬着:等到,待到。
\stopbuffer


\startbuffer[1531]
年甲:年龄。
\stopbuffer


\startbuffer[1532]
冰盘:大瓷盘。
\stopbuffer


\startbuffer[1533]
跬(kuĭ)跬拜拜:形容极恭敬的样子。跬,半步,古称行走时举足一次为跬,两次为步。
\stopbuffer


\startbuffer[1534]
衾(qīn):被子。
\stopbuffer


\startbuffer[1535]
裀:通『茵』,褥子、垫子。
\stopbuffer


\startbuffer[1536]
彤云:指下雪前密布的浓云。
\stopbuffer


\startbuffer[1537]
六出花:雪花的别称。
\stopbuffer


\startbuffer[1538]
此处原有夹批『好雪』二字。此夹批曾误入正文。
\stopbuffer


\startbuffer[1539]
骨柮(duò):亦作『榾柮』,木柴块。
\stopbuffer


\startbuffer[1540]
苍头:此处言头发斑白,指老人。
\stopbuffer


\startbuffer[1541]
潭府:对他人居宅的尊称。
\stopbuffer


\startbuffer[1542]
七贤过关:古画题材。
\stopbuffer


\startbuffer[1543]
汤寒:抵挡寒冷。汤,抵挡。
\stopbuffer


\startbuffer[1544]
沍(hù):冻结,凝结。
\stopbuffer


\startbuffer[1545]
衬钱:『衬施钱』的省称。施舍给僧道的钱物。衬通『嚫』,佛教谓施舍财物给僧尼。
\stopbuffer


\startbuffer[1546]
凌眼:严冬河水冰封后冰层特别薄的地方。
\stopbuffer


\startbuffer[1547]
地凌:冰,冰凌。
\stopbuffer


\startbuffer[1548]
宁耐:忍耐。
\stopbuffer


\startbuffer[1549]
紾(zhěn)掠:拧干,晒干。紾,扭,拧。掠,方言,晒干。
\stopbuffer


\startbuffer[1550]
耽忧:义同『担忧』。
\stopbuffer


\startbuffer[1551]
捻弄:捻诀变化。捻,指捻着避水诀,因捻诀需要做手势,所以无法抡棒。弄,指弄变化,变换鱼蟹之形,因而亦无法使棒。
\stopbuffer


\startbuffer[1552]
得故子:借故,故意。
\stopbuffer


\startbuffer[1553]
影:方言。对未知事物的惧怕。
\stopbuffer


\startbuffer[1554]
姆姆:弟妻对兄妻的称呼。
\stopbuffer


\startbuffer[1555]
鼻准:鼻尖;鼻梁。
\stopbuffer


\startbuffer[1556]
峤(qiáo):高而锐的山。
\stopbuffer


\startbuffer[1557]
哨:古代军事术语。泛称战阵的两翼或军队的一支、一队。
\stopbuffer


\startbuffer[1558]
扯炉:拉风箱烧火。
\stopbuffer


\startbuffer[1559]
菂(dì):莲子。
\stopbuffer


\startbuffer[1560]
磨博士:以磨粉为业的人。
\stopbuffer


\startbuffer[1561]
识俊:知趣,识相。
\stopbuffer


\startbuffer[1562]
性摊了:指泄劲了。摊,同『瘫』,弛缓,软弱无力。
\stopbuffer


\startbuffer[1563]
造字(duò):闯入。
\stopbuffer


\startbuffer[1564]
不喜欢:不高兴。
\stopbuffer


\startbuffer[1565]
擅干:随意冒犯。
\stopbuffer


\startbuffer[1566]
掐指巡纹:即掐指算卦。
\stopbuffer


\startbuffer[1567]
信心:诚心。
\stopbuffer


\startbuffer[1568]
九肋:一种珍惜龟种,甲纹呈多根肋条分布状。
\stopbuffer


\startbuffer[1569]
团圞(luán):团聚。圞,圆,团圆。
\stopbuffer


\startbuffer[1570]
挨土帮泥:比喻生活无着。挨、帮,皆为靠近义。
\stopbuffer


\startbuffer[1571]
毗卢:『毗卢舍那』的省称,释迦牟尼佛的法身佛,又称大日如来。也是法身佛的通称。
\stopbuffer


\startbuffer[1572]
曹溪:禅宗南宗别号。以六祖慧能在曹溪宝林寺演法而得名。
\stopbuffer


\startbuffer[1573]
至祝:至嘱,谓极恳切的嘱咐。
\stopbuffer


\startbuffer[1574]
会事:懂事,晓事。
\stopbuffer


\startbuffer[1575]
挜(yà):压。这里指塞、舀的动作。
\stopbuffer


\startbuffer[1576]
门枕石鼓:门枕,设于门槛两端的墩台,承托门扇转轴,使大门得以开关,以石制居多。石鼓,门枕石上的一种常见装饰。
\stopbuffer


\startbuffer[1577]
畏风:这里指避风。畏,避开,躲避。
\stopbuffer


\startbuffer[1578]
穿堂:房屋之间的过道。
\stopbuffer


\startbuffer[1579]
巴斗:一种容器,底为半球形,一般用竹、藤或柳条等编制而成。
\stopbuffer


\startbuffer[1580]
串楼:两楼之间有连廊相通。
\stopbuffer


\startbuffer[1581]
一程儿:一段日子。
\stopbuffer


\startbuffer[1582]
玄帝:指真武大帝,又称玄天大帝、玄武大帝。
\stopbuffer


\startbuffer[1583]
暗室亏心,神目如电:指人在暗处偷做亏心事,神灵的眼睛会像闪电一样看得一清二楚。
\stopbuffer


\startbuffer[1584]
背剪:两手交叉于背,只将手放在背部捆绑起来。
\stopbuffer


\startbuffer[1585]
毡(zhān)衣:用羊毛等动物毛压制而成的布料所制的衣服。
\stopbuffer


\startbuffer[1586]
油靴:用桐油涂制的可以防水的长筒靴。
\stopbuffer


\startbuffer[1587]
比犀难照水:古时有传说燃烧犀角能照亮水底。这里指这只犀牛的角不能照水。
\stopbuffer


\startbuffer[1588]
喘月犁云:喘月,相传吴地之牛畏热,见月亦疑为日,喘息不已,故称。犁云,即耕地。
\stopbuffer


\startbuffer[1589]
欺天振地:形容欺骗上天,作乱大地。
\stopbuffer


\startbuffer[1590]
卓是:着实,确实。
\stopbuffer


\startbuffer[1591]
不胜战栗屏营之至:用于奏章后的套语,类似于不胜惶恐。屏营,惶恐,彷徨。
\stopbuffer


\startbuffer[1592]
以闻:以此闻奏。奏章后的套语。
\stopbuffer


\startbuffer[1593]
可韩丈人真君:道教神霄派九宸之一。
\stopbuffer


\startbuffer[1594]
三微垣:当指『三垣』。中国古代天文学对恒星的划分。即上垣之太微垣、中垣之紫微垣及下垣之天市垣。
\stopbuffer


\startbuffer[1595]
七政、四余:初为古代天文学和占星术的术语,后成为道教星神。其中炁(紫炁)、孛(月孛)、罗睺、计都四星是为测算天文虚拟的星,故称四余。
\stopbuffer


\startbuffer[1596]
儿曹:犹儿辈。
\stopbuffer


\startbuffer[1597]
头势:情势,形势。
\stopbuffer


\startbuffer[1598]
不成人:形容人行为恶劣。
\stopbuffer


\startbuffer[1599]
筋䯞(kuā)子:形容皮包骨的样子。
\stopbuffer


\startbuffer[1600]
橐(tuó):骆驼。
\stopbuffer


\startbuffer[1601]
秀溜:轻巧灵活。
\stopbuffer


\startbuffer[1602]
奘(zhuăng):粗而大。
\stopbuffer


\startbuffer[1603]
夙(sù)话:以前的事,旧事。夙,早,早年。
\stopbuffer


\startbuffer[1604]
惯家熟套:指熟悉套路的老手。惯家,行家,老手。熟套,旧套。
\stopbuffer


\startbuffer[1605]
揲揲锤锤:象声词。形容促织的叫声。
\stopbuffer


\startbuffer[1606]
涤涤托托:象声词。敲打梆子的响声。
\stopbuffer


\startbuffer[1607]
脱脚:脱去鞋袜。
\stopbuffer


\startbuffer[1608]
虼蚤:方言。跳蚤。
\stopbuffer


\startbuffer[1609]
刺闹:皮肤发痒难受。
\stopbuffer


\startbuffer[1610]
梦梦查查:犹迷迷糊糊。
\stopbuffer


\startbuffer[1611]
本家:自己家。
\stopbuffer


\startbuffer[1612]
谢土:古代房屋盖成后酬谢土神的一种祭祀形式。
\stopbuffer


\startbuffer[1613]
面不厮睹:互不相看。
\stopbuffer


\startbuffer[1614]
脚色:本色,真相。这里指本相。
\stopbuffer


\startbuffer[1615]
优婆:佛教语。优婆塞(男性)和优婆夷(女性)的略称。意为善男信女。
\stopbuffer


\startbuffer[1616]
卖放人:受贿私放。引申指欺骗。
\stopbuffer


\startbuffer[1617]
卖法:谓贪赃枉法。引申指耍花样,取巧。
\stopbuffer


\startbuffer[1618]
腯(tú):肥壮。多用以形容牲畜。
\stopbuffer


\startbuffer[1619]
拗步:武术的一种招式。
\stopbuffer


\startbuffer[1620]
挂面:照面,见面。这里指当面一拳。
\stopbuffer


\startbuffer[1621]
闪赚:欺诳,哄骗。
\stopbuffer


\startbuffer[1622]
吊:掉落,跌落。
\stopbuffer


\startbuffer[1623]
地里鬼:指熟悉地方情况、善于查访内情的人。
\stopbuffer


\startbuffer[1624]
拘儿:此处指鼻环,钩环。拘,拘束,牵引。
\stopbuffer


\startbuffer[1625]
『德行』二句:道教认为人的善行积满八百,可成地仙;积满三千,可成天仙。
\stopbuffer


\startbuffer[1626]
棹(zhào):形似桨的划船工具。短棹,短把的船桨。
\stopbuffer


\startbuffer[1627]
桡(ráo):桨,楫。
\stopbuffer


\startbuffer[1628]
造字(gǎn)堂:谓船的两旁。
\stopbuffer


\startbuffer[1629]
艎板:船板。
\stopbuffer


\startbuffer[1630]
牙樯(qiáng):象牙装饰的桅杆。后用作对桅杆的美称。
\stopbuffer


\startbuffer[1631]
骨冗(rǒng):方言。蠕动的意思。也写作『咕容』。
\stopbuffer


\startbuffer[1632]
草把:旧时酒店以杉叶束成球状,悬放门前,作为标识。
\stopbuffer


\startbuffer[1633]
绩麻:把麻析成细缕,搓捻成线或绳。
\stopbuffer


\startbuffer[1634]
胁下:从腋下到肋骨尽处的部分。
\stopbuffer


\startbuffer[1635]
养儿肠:指孕妇的肚腹。
\stopbuffer


\startbuffer[1636]
摧阵疼:阵痛,孕妇分娩前断续的疼痛。
\stopbuffer


\startbuffer[1637]
浆包:即胎膜。
\stopbuffer


\startbuffer[1638]
花红表礼:指礼品。花红,有关婚姻等喜庆事的礼物。表礼,用作礼品或赏赐的衣料。
\stopbuffer


\startbuffer[1639]
侵哄:侵犯、欺哄。
\stopbuffer


\startbuffer[1640]
天台:天台山,在浙江天台县北。佛教天台宗和道教内丹南宗的发源地。
\stopbuffer


\startbuffer[1641]
三峰西华山:即西岳华山。三峰,指华山的莲花、毛女、松桧三峰。
\stopbuffer


\startbuffer[1642]
如意钩子:如意钩,一种兵器,前端为钩,后端为柄。
\stopbuffer


\startbuffer[1643]
不识起倒:不知好歹,不识时务。
\stopbuffer


\startbuffer[1644]
僝僽(chán zhòu):此处指责骂,埋怨。
\stopbuffer


\startbuffer[1645]
挛:卷,卷曲。
\stopbuffer


\startbuffer[1646]
硍(kěn):同『啃』。
\stopbuffer


\startbuffer[1647]
机见:见识,智谋。
\stopbuffer


\startbuffer[1648]
绸缪:这里指纠缠,连绵不断。
\stopbuffer


\startbuffer[1649]
蹼辣:跌倒的声音。这里指跌跟头。
\stopbuffer


\startbuffer[1650]
门枋(fāng):门框的竖木。枋,长方形木材。
\stopbuffer


\startbuffer[1651]
占房:方言。分娩。孕妇生产时,外人不得进入产房,故称。
\stopbuffer


\startbuffer[1652]
实落:结实。
\stopbuffer


\startbuffer[1653]
沙包肚:产妇产后进食太多,以致肚腹膨脝,无法消退,俗称『沙包肚』。
\stopbuffer


\startbuffer[1654]
关厢:城门外大街和附近的地区。
\stopbuffer


\startbuffer[1655]
人种:传宗接代的人。
\stopbuffer


\startbuffer[1656]
销猪:即劁(qiāo)猪。阉割过的猪。
\stopbuffer


\startbuffer[1657]
粉郎:三国魏何晏美仪容,面如傅粉,人称粉侯、粉郎。后用作对俊美心爱郎君的美称。
\stopbuffer


\startbuffer[1658]
烟花:这里指风月,情爱。
\stopbuffer


\startbuffer[1659]
货殖:财物;商品。
\stopbuffer


\startbuffer[1660]
旗亭:酒楼。悬旗为酒招,故称。
\stopbuffer


\startbuffer[1661]
候馆:接待过往官员或外国使者的驿馆。
\stopbuffer


\startbuffer[1662]
领给:指付予他人的钱财和生活用品。
\stopbuffer


\startbuffer[1663]
本官:指本部门的主管官员。
\stopbuffer


\startbuffer[1664]
碓梃(duì tǐng)嘴:形容长嘴。碓梃,捣碓的棒子。梃,棍棒,棍状物。
\stopbuffer


\startbuffer[1665]
关付:行文交付。关,发放。
\stopbuffer


\startbuffer[1666]
玉成:对他人成全的敬称。
\stopbuffer


\startbuffer[1667]
口里摆菜碟儿:比喻空许诺,说空话。
\stopbuffer


\startbuffer[1668]
肯酒:允婚酒。
\stopbuffer


\startbuffer[1669]
佥(qiān)押:在文书上签名画押表示负责。
\stopbuffer


\startbuffer[1670]
会喜:庆贺恭喜。
\stopbuffer


\startbuffer[1671]
合卺(jǐn):古代婚仪之一。剖一瓠为两瓢,新婚夫妇各执一瓢,斟酒以饮。后多以『合卺』代指成婚。卺,古代结婚时用作酒器的瓢。
\stopbuffer


\startbuffer[1672]
汲汲:心情急切貌。
\stopbuffer


\startbuffer[1673]
恣恣:放纵,放肆。
\stopbuffer


\startbuffer[1674]
见成:现成。
\stopbuffer


\startbuffer[1675]
安席:宴会入坐时敬酒的一种礼节。
\stopbuffer


\startbuffer[1676]
觥(gōng):古代酒器。
\stopbuffer


\startbuffer[1677]
鸬鹚杓(sháo):刻为鸬鹚形的酒杓。鹦鹉杯、鸬鹚杓及以下提及的几种酒器皆为古史或文学记载中的著名酒器。
\stopbuffer


\startbuffer[1678]
叵罗:西域语音译,当地的一种饮酒器,口敞底浅。
\stopbuffer


\startbuffer[1679]
凿落:以镌镂金银为饰的酒盏。
\stopbuffer


\startbuffer[1680]
鸾交凤友:比喻情侣、夫妻。
\stopbuffer


\startbuffer[1681]
邓沙:指澄沙。纯净的豆沙。
\stopbuffer


\startbuffer[1682]
荣:与下句的『卫』皆为中医学名词。指血、气的循环周流,对人体起滋养和保卫作用。
\stopbuffer


\startbuffer[1683]
倒马毒:指称蝎尾部的钩刺。
\stopbuffer


\startbuffer[1684]
瘇(zhǒng):肌肉肿胀。
\stopbuffer


\startbuffer[1685]
痈(yōng):肿疡。一种皮肤病。
\stopbuffer


\startbuffer[1686]
委:尾随,跟随。
\stopbuffer


\startbuffer[1687]
『月阇梨』二句:南宋年间有歌妓柳翠,好佛法,喜施与,后被高僧度化。后人据此演绎成文学作品,讲述观音菩萨净瓶内的杨柳枝因偶污微尘,被罚在人间作风尘妓女,名为柳翠。三十年后罗汉月明尊者化成风魔和尚,三次说法,使柳翠醒悟,坐化升天。
\stopbuffer


\startbuffer[1688]
猱(náo)狮:长毛狮子。
\stopbuffer


\startbuffer[1689]
干鱼可好与猫儿作枕头:大意为晒干的鱼如给猫做枕头,就会被猫吃掉,比喻不当的投其所好会导致不良结果。
\stopbuffer


\startbuffer[1690]
摩弄:调弄,引诱。
\stopbuffer


\startbuffer[1691]
生成:自然形成,生就。
\stopbuffer


\startbuffer[1692]
巡札:巡察点札。
\stopbuffer


\startbuffer[1693]
发天阴:天阴发病。
\stopbuffer


\startbuffer[1694]
距:雄鸡腿后突出像爪的部分。
\stopbuffer


\startbuffer[1695]
五德:古谓鸡有文、武、勇、仁、信五德。头上有冠,是文;足后有距,是武;敌在前敢斗,是勇;得食相告,是仁;守夜不失时,是信。
\stopbuffer


\startbuffer[1696]
朱明:指夏季。
\stopbuffer


\startbuffer[1697]
角黍:即粽子。状如三角,古用黏黍,故称。
\stopbuffer


\startbuffer[1698]
中天之节:中天节,端午节的别称。端午前后太阳位于一年中最高的位置,故称。
\stopbuffer


\startbuffer[1699]
嗒笞笞(dā chī chī):赶马的吆喝声。
\stopbuffer


\startbuffer[1700]
磕脑:古代男子裹头的巾。
\stopbuffer


\startbuffer[1701]
悭(qiān)吝:吝啬。悭,小气。
\stopbuffer


\startbuffer[1702]
打罢春:指打春。古时于立春日鞭土牛以祈丰年的习俗。
\stopbuffer


\startbuffer[1703]
揾土:贴地。揾,吻,啃。
\stopbuffer


\startbuffer[1704]
《倒头经》:旧俗人初死时,僧道做法事祷诵的经、咒。倒头,谓人死。旧俗忌讳『死』字,讳称『倒头』。
\stopbuffer


\startbuffer[1705]
左使:错使唤。左,错,不对。
\stopbuffer


\startbuffer[1706]
掆(gāng):这里指顶。
\stopbuffer


\startbuffer[1707]
冢(zhǒng):坟墓。
\stopbuffer


\startbuffer[1708]
五路猖神:五猖,又名『五神通』。旧时江南民间供奉的邪神。
\stopbuffer


\startbuffer[1709]
面是背非:当面赞成,背后反对。
\stopbuffer


\startbuffer[1710]
住场:住处。
\stopbuffer


\startbuffer[1711]
畦(qí):田园中长条的田块,泛指田园。古代也作土地面积单位,通常五十亩为一畦。
\stopbuffer


\startbuffer[1712]
爨烟:炊烟。爨,烧火煮饭。
\stopbuffer


\startbuffer[1713]
迢递:遥远貌。
\stopbuffer


\startbuffer[1714]
抓乖弄俏:耍聪明,卖弄乖巧。
\stopbuffer


\startbuffer[1715]
普同:犹普通,一般。意谓不分彼此。
\stopbuffer


\startbuffer[1716]
生理:生计,活计,职业。
\stopbuffer


\startbuffer[1717]
送:指送交官府。
\stopbuffer


\startbuffer[1718]
以次人丁:指更小的儿子。以次,表示位次在后的。
\stopbuffer


\startbuffer[1719]
团瓢:圆形草屋。
\stopbuffer


\startbuffer[1720]
和气:古人认为天地间阴气与阳气交合而成之气。万物由『和气』而生。故杀戮称为『伤和气』。
\stopbuffer


\startbuffer[1721]
伏气:同『服气』。
\stopbuffer


\startbuffer[1722]
刁嘴:犹言油嘴滑舌。
\stopbuffer


\startbuffer[1723]
三花:道教指人的精、气、神。
\stopbuffer


\startbuffer[1724]
四大:佛教以地、水、火、风为四大,人身即由此构成。因亦用作人身的代称。
\stopbuffer


\startbuffer[1725]
任命:谓听任命运的支配。
\stopbuffer


\startbuffer[1726]
砑(yà):碾,压。
\stopbuffer


\startbuffer[1727]
搻(nuò):古同『搦』,这里或为按压义。
\stopbuffer


\startbuffer[1728]
满眼抛珠:指流泪。
\stopbuffer


\startbuffer[1729]
尸灵:尸体,遗体。
\stopbuffer


\startbuffer[1730]
尖担担柴两头脱:比喻两头落空。
\stopbuffer


\startbuffer[1731]
臻(zhēn):增加。
\stopbuffer


\startbuffer[1732]
桑榆晚景:日暮之景。比喻晚年。
\stopbuffer


\startbuffer[1733]
大走:出远门。
\stopbuffer


\startbuffer[1734]
影瞒:遮掩,欺骗。
\stopbuffer


\startbuffer[1735]
别土星:指鼻梁塌陷。相术称鼻子为『土星』。别,借用作『瘪』。
\stopbuffer


\startbuffer[1736]
缩地:传说中化远为近的仙术。
\stopbuffer


\startbuffer[1737]
帮:靠拢,挨近。
\stopbuffer


\startbuffer[1738]
说从:说动。
\stopbuffer


\startbuffer[1739]
前件:前已述及的人或事物。
\stopbuffer


\startbuffer[1740]
『不有中有,不无中无』一段:此段说法文字实际出自道教经典《太上升玄消灾护命妙经》。
\stopbuffer


\startbuffer[1741]
轻造:轻率冒昧地造访。谦辞。
\stopbuffer


\startbuffer[1742]
广会:指广泛、全面地了解,知晓。
\stopbuffer


\startbuffer[1743]
出入:这里指出入生死界限。
\stopbuffer


\startbuffer[1744]
休咎:吉凶;善恶。
\stopbuffer


\startbuffer[1745]
耍刁:犹放刁。
\stopbuffer


\startbuffer[1746]
六识:即六贼。
\stopbuffer


\startbuffer[1747]
䪊(lóng):马笼头。
\stopbuffer


\startbuffer[1748]
深衣:古代上衣、下裳相连缀的一种服装。为古代诸侯、大夫、士家居常穿的衣服,也是庶人的常礼服。
\stopbuffer


\startbuffer[1749]
㧳(bāi)靸䩺(wèng)鞋:拖着穿的棉鞋。㧳,用手分开。䩺,棉鞋。
\stopbuffer


\startbuffer[1750]
咍(hāi)口:笑口。
\stopbuffer


\startbuffer[1751]
隔年焦:比喻为遥远的事情担忧。
\stopbuffer


\startbuffer[1752]
一度:犹一次。
\stopbuffer


\startbuffer[1753]
溲话:指过时不顶用的老话。
\stopbuffer


\startbuffer[1754]
毛儿女:毛女,本是传说中得道于华山的仙女。此处指女仙童。
\stopbuffer


\startbuffer[1755]
月婆:原指月亮。这里指七政四余中的月孛,属凶星。民间传其为女凶神。
\stopbuffer


\startbuffer[1756]
令正:称对方嫡妻的敬辞。旧时以嫡妻为正室,故称。
\stopbuffer


\startbuffer[1757]
倾:害,坑害。
\stopbuffer


\startbuffer[1758]
沉酣:沉浸于某种境界。
\stopbuffer


\startbuffer[1759]
见其肺肝:比喻将人看得清清楚楚。语出《礼记·大学》:『人之视己,如见其肺肝然,则何益矣?』此处为孙悟空的戏谑之语。
\stopbuffer


\startbuffer[1760]
偃月冠:道士戴的一种冠,形如偃月,故名。偃月,横卧形的半弦月。
\stopbuffer


\startbuffer[1761]
显圣:即二郎神。又称二郎显圣真君。
\stopbuffer


\startbuffer[1762]
咳咳:喜笑貌。
\stopbuffer


\startbuffer[1763]
吸吸:呼吸急促貌。
\stopbuffer


\startbuffer[1764]
奉渎:指冒昧打扰。奉,敬辞。
\stopbuffer


\startbuffer[1765]
槃(pán):同『盘』。
\stopbuffer


\startbuffer[1766]
骊颔之珠:传说深渊之底骊龙颔下所生的明珠。
\stopbuffer


\startbuffer[1767]
樽俎:古代盛酒食的器皿。樽以盛酒,俎以盛肉。
\stopbuffer


\startbuffer[1768]
八音:古称金、石、丝、竹、匏、土、革、木八种不同质材的乐器为八音。
\stopbuffer


\startbuffer[1769]
横行介士:螃蟹的戏称。介士,武士。
\stopbuffer


\startbuffer[1770]
落索:冷落,萧索。
\stopbuffer


\startbuffer[1771]
玉山颓:古人记载嵇康身形高大,醒时如孤松挺立,醉时如玉山将崩。后以玉山崩、玉山颓形容人酒醉。
\stopbuffer


\startbuffer[1772]
弄丑:犹出丑。
\stopbuffer


\startbuffer[1773]
家乐儿:家中蓄养的歌妓。这里泛指侍候的人。
\stopbuffer


\startbuffer[1774]
掮(qián):用肩扛东西。
\stopbuffer


\startbuffer[1775]
叨餂(tiǎn):骗取。餂,诱取。
\stopbuffer


\startbuffer[1776]
强能:精明强干。
\stopbuffer


\startbuffer[1777]
生受:辛苦,劳烦。
\stopbuffer


\startbuffer[1778]
鹐(qiān):鸟啄物。
\stopbuffer


\startbuffer[1779]
血皮胀:和下文的『结心癀』皆为牛病。这里用作骂牛魔王的话。
\stopbuffer


\startbuffer[1780]
『大海里』句:豆腐是豆浆经卤水点化而成,故称。比喻用同样的方式得到又失去某物。
\stopbuffer


\startbuffer[1781]
行不由径:走路不抄小道。径,小路。
\stopbuffer


\startbuffer[1782]
牛王本是心猿变:佛、道教有时用牛比喻躁动的心性,与『心猿』寓意相似。
\stopbuffer


\startbuffer[1783]
『木生在亥配为猪』下数句:借五行之名来讲述火焰山一节的故事。木克土,喻猪八戒来战牛魔王;土生金,喻牛魔王是孙悟空成功的条件;金生水,喻孙悟空借到芭蕉扇;水克火,喻芭蕉扇扇灭火焰山。
\stopbuffer


\startbuffer[1784]
䬲(gōu)草:吃草的畜生。䬲,牛饱。
\stopbuffer


\startbuffer[1785]
嗛(xián):指以喙啄物。
\stopbuffer


\startbuffer[1786]
乜乜(miē)些些:装痴作呆。乜,眼睛微睁,斜视。
\stopbuffer


\startbuffer[1787]
色服:也作『色衣』。彩色的衣服。对『素服』(孝服)而言。
\stopbuffer


\startbuffer[1788]
叩齿:上下牙齿碰击,是佛、道教的一种祈祷仪式。
\stopbuffer


\startbuffer[1789]
五漏:道教内丹术称魂、魄、神、精、意的散漏为五漏。一说为第五更天,借指一整夜。
\stopbuffer


\startbuffer[1790]
三关:道教内丹术称尾闾、夹脊、玉枕为真气运行的三关。
\stopbuffer


\startbuffer[1791]
不倜傥:不爽快。
\stopbuffer


\startbuffer[1792]
血食:指用于祭祀的食品。古代杀牲取血以祭,故有此称。
\stopbuffer


\startbuffer[1793]
百刻:指一昼夜。古代用刻漏计时,一昼夜分百刻。
\stopbuffer


\startbuffer[1794]
散诞:放诞不羁;逍遥自在。
\stopbuffer


\startbuffer[1795]
时序:节候,时节。
\stopbuffer


\startbuffer[1796]
应钟:古乐律名,十二律之一。这里指十月。古以十二律与十二月相配,应钟对应十月。
\stopbuffer


\startbuffer[1797]
闭蛰:指虫类藏伏冬眠。
\stopbuffer


\startbuffer[1798]
纯阴阳月:阳月,农历十月的别称。汉董仲舒《雨雹对》:『十月,阴虽用事,而阴不孤立。此月纯阴,疑于无阳,故谓之阳月。』故称『纯阴阳月』。
\stopbuffer


\startbuffer[1799]
帝元溟:元溟,即玄冥。水神,亦为冬神。古人认为一年中每三个月有一位天帝、一位天神主管。主管十月、十一月、十二月的天帝为颛顼,天神为玄冥。
\stopbuffer


\startbuffer[1800]
盛水德:指冬季具备五行中水的特征。
\stopbuffer


\startbuffer[1801]
舜日怜晴:指冬季的晴天可贵。舜日,形容太平之日。
\stopbuffer


\startbuffer[1802]
华盖:古星名,属紫微垣,共十六星,在五帝座上,今属仙后座。
\stopbuffer


\startbuffer[1803]
紫垣:星座名。即紫微垣、紫微宫。天帝居住之所。也借指皇宫。
\stopbuffer


\startbuffer[1804]
邦畿(jī):王城及其所属周围千里的地域。借指国家。
\stopbuffer


\startbuffer[1805]
隅(yú)头:墙角。隅,角落。
\stopbuffer


\startbuffer[1806]
谯(qiáo)楼:城门上的瞭望楼。
\stopbuffer


\startbuffer[1807]
软公鞋:一说为长筒皮靴。
\stopbuffer


\startbuffer[1808]
宝瓶:佛教尊称盛佛具法具的瓶器。佛塔顶部常用作装饰。
\stopbuffer


\startbuffer[1809]
铎(duó):古代乐器。大铃的一种。形如铙、钲而有舌。
\stopbuffer


\startbuffer[1810]
口词:口供。
\stopbuffer


\startbuffer[1811]
凿眼:犹眼线。暗中帮助侦察、窥探,或担任向导的人。
\stopbuffer


\startbuffer[1812]
驾帖:明代秉承皇帝意旨,由刑科签发的逮捕人的公文。
\stopbuffer


\startbuffer[1813]
专心:一心。
\stopbuffer


\startbuffer[1814]
黄伞:黄罗伞盖。皇帝仪仗之一。
\stopbuffer


\startbuffer[1815]
八抬八绰:用八个人抬的大轿抬。绰,扛、抬。
\stopbuffer


\startbuffer[1816]
本身:原身。这里指原来的身份。
\stopbuffer


\startbuffer[1817]
寂寂密密:悄悄地,秘密地。
\stopbuffer


\startbuffer[1818]
翻席:一席未终,在别处另设一席。
\stopbuffer


\startbuffer[1819]
特余事:只是小事。特,仅仅,只是。余事,正事以外或不相干的小事。
\stopbuffer


\startbuffer[1820]
太岳:即岳父。
\stopbuffer


\startbuffer[1821]
兜鍪(móu):古代战士的头盔。
\stopbuffer


\startbuffer[1822]
没要紧:指随便、轻率。
\stopbuffer


\startbuffer[1823]
罗织:犹兜揽。
\stopbuffer


\startbuffer[1824]
刻日:即日。比喻即将。
\stopbuffer


\startbuffer[1825]
未审:不知。
\stopbuffer


\startbuffer[1826]
郎丈:女婿和丈人。
\stopbuffer


\startbuffer[1827]
更(gēng):更鼓。指时间。
\stopbuffer


\startbuffer[1828]
九头虫滴血:传说中有鬼车鸟,又称九头鸟,原有十头,被犬咬去一头,若滴血入人家,必为灾祸。
\stopbuffer


\startbuffer[1829]
家无全犯:没有全家人都犯罪的。
\stopbuffer


\startbuffer[1830]
度口:维生,糊口。
\stopbuffer


\startbuffer[1831]
匝地:遍地。
\stopbuffer


\startbuffer[1832]
蕃(fán)盛:繁茂,兴盛。蕃,茂盛。
\stopbuffer


\startbuffer[1833]
搂:锄地。
\stopbuffer


\startbuffer[1834]
十八公:松树的隐语。『松』字可拆作『十八公』,故称。
\stopbuffer


\startbuffer[1835]
孤直公:暗指柏树。出自唐李白《古风五十九首》:『松柏本孤直,难为桃李颜。』
\stopbuffer


\startbuffer[1836]
凌空子:暗指桧树。出自宋苏轼《王复秀才所居双桧二首》:『凛然相对敢相欺,直干凌空未要奇。』
\stopbuffer


\startbuffer[1837]
拂云叟:暗指竹。出自唐杜甫《严郑公宅同咏竹》:『但令无剪伐,会见拂云长。』
\stopbuffer


\startbuffer[1838]
劲节:暗指松树。出自南朝范云《咏寒松》:『凌风识劲节,负霜知贞心。』
\stopbuffer


\startbuffer[1839]
乌栖凤宿:史载汉御史大夫朱博的府中有柏树,常有数千乌鸦栖息。凤宿,唐杜甫《病柏》有『丹凤领九雏,哀鸣翔其外』。
\stopbuffer


\startbuffer[1840]
七贤:晋阮籍、嵇康等七贤士,常于竹林中游乐,人称『竹林七贤』。
\stopbuffer


\startbuffer[1841]
六逸:唐李白和五位友人在徂徕山隐居,人称『竹溪六逸』。
\stopbuffer


\startbuffer[1842]
戛(jiá)玉敲金:指风吹竹林的声音。戛,敲击。
\stopbuffer


\startbuffer[1843]
四皓:秦末隐居商山的东园公、甪里先生、绮里季、夏黄公四人,须眉皆白,故称。
\stopbuffer


\startbuffer[1844]
俄捱:拖延一会儿。俄,短暂的时间。捱,慢慢行进;拖延。
\stopbuffer


\startbuffer[1845]
台颜:犹尊颜。称呼对方的敬辞。
\stopbuffer


\startbuffer[1846]
希夷:指虚寂玄妙。《老子》中有言『视之不见名曰夷,听之不闻名曰希』。
\stopbuffer


\startbuffer[1847]
象罔:有无心、无形迹之意。象罔是《庄子》寓言中的人物。
\stopbuffer


\startbuffer[1848]
体用:指事物的本体和作用。
\stopbuffer


\startbuffer[1849]
冲虚:恬淡虚静。也指升天成仙。
\stopbuffer


\startbuffer[1850]
葛藤谜语,萝蓏浑言:佛、道教比喻使人心混乱的文字。
\stopbuffer


\startbuffer[1851]
没底竹篮汲水,无根铁树生花:道教比喻修炼丹成得道。
\stopbuffer


\startbuffer[1852]
吟哦(é):作诗吟咏。
\stopbuffer


\startbuffer[1853]
四始:旧说《诗经》有四始,即『风』『小雅』『大雅』『颂』。此句言孔子删诗的典故。
\stopbuffer


\startbuffer[1854]
宪乌:御史台的别称。因御史台又称乌台、宪台,故以『宪乌』称之。
\stopbuffer


\startbuffer[1855]
三秀:灵芝草的别名。灵芝一年开花三次,故称。
\stopbuffer


\startbuffer[1856]
木王:指梓树。
\stopbuffer


\startbuffer[1857]
琥珀:为松脂的化石。
\stopbuffer


\startbuffer[1858]
四绝堂:唐代岳麓山道林寺建有『四绝堂』,堂前有柏树,相传为晋代名将陶侃所植。
\stopbuffer


\startbuffer[1859]
元日迎春:古代有元旦用柏叶酒祝寿和辟邪的习俗。
\stopbuffer


\startbuffer[1860]
月胁天心:比喻诗句语出惊人,意境高绝。
\stopbuffer


\startbuffer[1861]
太清宫:亳州太清宫有八株桧树,相传为老子手植。
\stopbuffer


\startbuffer[1862]
淇澳:即淇奥,出自《诗经·卫风·淇奥》:『瞻彼淇奥,绿竹猗猗。』意为淇水弯曲处有竹。
\stopbuffer


\startbuffer[1863]
『翠筠』句:传说舜的妃子娥皇、女英泪滴于竹,留下斑点,称湘妃竹。筠,竹之青皮。
\stopbuffer


\startbuffer[1864]
子猷:晋王羲之之子王徽之,字子猷,性爱竹。
\stopbuffer


\startbuffer[1865]
游夏:子游与子夏的并称。二者均为孔子学生,长于文学。
\stopbuffer


\startbuffer[1866]
赓(gēng)酬:谓以诗歌与人相赠答。赓,连续,接续。
\stopbuffer


\startbuffer[1867]
『上盖留名汉武王』四句:汉武王,即汉武帝,传说汉武帝宫中多杏树。孔子立坛场,指孔子曾在杏坛授徒。董仙,三国时期吴国的名医董奉,给人看病收取杏树为酬。孙楚,晋代人士,曾在寒食节用杏酪祭祀春秋时晋国名臣介子推。
\stopbuffer


\startbuffer[1868]
挨挨轧轧:同『挨挨擦擦』。谓以肌体相挤擦。一说为挨挨拶拶。挨近,挤逼。
\stopbuffer


\startbuffer[1869]
蜜合:蜜合色。即浅黄白色。
\stopbuffer


\startbuffer[1870]
无心:佛教语。指摆脱种种妄想的心。
\stopbuffer


\startbuffer[1871]
黑海:指肾,五行属水,配黑色,故称。
\stopbuffer


\startbuffer[1872]
愍(mĭn):怜悯。
\stopbuffer


\startbuffer[1873]
涴(wò):浸染,染上。
\stopbuffer


\startbuffer[1874]
虚徐:从容不迫,舒缓。
\stopbuffer


\startbuffer[1875]
未免:不免,免不了。
\stopbuffer


\startbuffer[1876]
铙(náo):一种似钹而大的打击乐器。
\stopbuffer


\startbuffer[1877]
乾元亨利贞:出自《周易》,为乾卦的卦辞,后成为道教的常用咒语。
\stopbuffer


\startbuffer[1878]
悬胆:悬挂着的胆囊,常用于形容人的鼻形,鼻头圆大,山根窄。
\stopbuffer


\startbuffer[1879]
窊(wā)皱:凹陷起皱。
\stopbuffer


\startbuffer[1880]
海峤:海边山岭。
\stopbuffer


\startbuffer[1881]
搭包子:即褡包。长而宽的腰带,内可装钱物。
\stopbuffer


\startbuffer[1882]
荆扬:荆州和扬州。
\stopbuffer


\startbuffer[1883]
翼轸:二十八宿中的翼宿和轸宿。古人以二十八星宿对应地上的区域,互称分星和分野。翼轸为楚地的分星,武当山在楚地,楚是翼和轸的分野。
\stopbuffer


\startbuffer[1884]
甲辰:历史上开皇元年为辛丑年。
\stopbuffer


\startbuffer[1885]
六合:四方加上下为六合。泛指宇宙。
\stopbuffer


\startbuffer[1886]
上紧:赶快,加紧。
\stopbuffer


\startbuffer[1887]
盱眙(xū yí)山蠙(pín)城:今江苏省盱眙县。
\stopbuffer

\startbuffer[1888]
水母娘娘:传说为淮河流域的水怪,后民间传说演化为女性,称为水母。
\stopbuffer


\startbuffer[1889]
水猿大圣:淮河流域民间传说的妖怪。
\stopbuffer


\startbuffer[1890]
颐:下颌,下巴。
\stopbuffer


\startbuffer[1891]
东来佛祖:即弥勒佛,也称未来佛。世俗所见弥勒佛的形象源自五代明州的僧人布袋和尚,人们认为他是弥勒佛的化身。
\stopbuffer


\startbuffer[1892]
蒯(kuǎi):方言。挠,抓。
\stopbuffer


\startbuffer[1893]
金查:即金渣。查,同『渣』。
\stopbuffer


\startbuffer[1894]
虾着腰:像虾一样弯着腰。
\stopbuffer


\startbuffer[1895]
东圊(qīng):即厕所。旧时厕所多在屋子东角,故称。圊,厕所。
\stopbuffer


\startbuffer[1896]
窄偪(bī):窄小逼仄。偪,同『逼』。
\stopbuffer


\startbuffer[1897]
惊嘬(zuō)嘬:惊慌紧张貌。嘬,聚缩嘴唇的样子。
\stopbuffer


\startbuffer[1898]
骨挝脸:形容瘦削的脸型。
\stopbuffer


\startbuffer[1899]
法官:对道士的尊称。
\stopbuffer


\startbuffer[1900]
僧伽:梵语译音。意为大众。原指出家佛教徒四人以上组成的团体,后也称单个和尚。
\stopbuffer


\startbuffer[1901]
『吃了磨刀水的』句:指人内心聪慧。『秀』谐音『锈』,因磨刀水含铁锈,故称。
\stopbuffer


\startbuffer[1902]
拄:支撑,顶着。
\stopbuffer


\startbuffer[1903]
盐酱口:指说的话灵验。
\stopbuffer


\startbuffer[1904]
经过帐的:旧时卖田产需要先开账单,成为经帐。引申为经历过、有经验的。
\stopbuffer


\startbuffer[1905]
行止:品行。
\stopbuffer


\startbuffer[1906]
旭人:旭,疑为『熏』之误。
\stopbuffer


\startbuffer[1907]
信㮇(tiàn):蛇的舌头,又称信子。
\stopbuffer


\startbuffer[1908]
倒扯蛇:方言。比喻费劲而无效果。这里是双关语。
\stopbuffer


\startbuffer[1909]
没蛇弄:民间有谚语『叫化子没蛇弄』,比喻失掉依靠,无法维持。此处用作双关语。
\stopbuffer


\startbuffer[1910]
使风:借用风力,张帆行船。
\stopbuffer


\startbuffer[1911]
科猪:即窠猪。母猪。科,同『窠』。
\stopbuffer


\startbuffer[1912]
馉饳(ɡǔ duo):古代的一种面食,有馅。
\stopbuffer


\startbuffer[1913]
齆齆(wèng):因鼻孔堵塞而发音不清。这里形容猪的声音。
\stopbuffer


\startbuffer[1914]
喁(yú)哮:兽类的喘息声。
\stopbuffer


\startbuffer[1915]
魈(xiāo):传说中山里的鬼怪。
\stopbuffer


\startbuffer[1916]
垛堞(duǒ dié):城墙上凹凸状的矮墙。
\stopbuffer


\startbuffer[1917]
梯航:梯与船。登山渡水的工具。
\stopbuffer


\startbuffer[1918]
会同馆:元、明、清三朝将接待藩属贡使的机构称作『会同馆』。
\stopbuffer


\startbuffer[1919]
人夫:受雇用的民夫。
\stopbuffer


\startbuffer[1920]
支应:接待,供应;也指供应之物。
\stopbuffer


\startbuffer[1921]
趱钱:即攒钱。趱,积蓄。
\stopbuffer


\startbuffer[1922]
回席:犹回请。
\stopbuffer


\startbuffer[1923]
调和:调味的佐料。
\stopbuffer


\startbuffer[1924]
经纪:经营买卖。这里指会买东西。
\stopbuffer


\startbuffer[1925]
淹延:长久;拖延。
\stopbuffer


\startbuffer[1926]
造字:即噘。
\stopbuffer


\startbuffer[1927]
现钟不打打铸钟:现有的钟不敲,而要敲还没铸成的钟。比喻舍近求远。
\stopbuffer


\startbuffer[1928]
村强:粗野蛮横。
\stopbuffer


\startbuffer[1929]
招架:承认,应承。
\stopbuffer


\startbuffer[1930]
装胖:充数,作幌子。
\stopbuffer


\startbuffer[1931]
经纶手:比喻治国的良才。经纶,整理丝缕、理出丝绪和编丝成绳,统称经纶。引申为治理国家大事。
\stopbuffer


\startbuffer[1932]
三折肱(gōng):古有『三折肱,知为良医』之语,意为多次断臂,就有了治疗断臂的经验。因以『三折肱』指代良医。肱,上臂,也泛指手臂。
\stopbuffer


\startbuffer[1933]
哑请:疑为『延请』之误。有版本作『跟请』,跟随邀请。
\stopbuffer


\startbuffer[1934]
折杀:古人认为过分享受该有的待遇会减损福寿。
\stopbuffer


\startbuffer[1935]
与:给,替。
\stopbuffer


\startbuffer[1936]
慢人:轻慢,怠慢人。
\stopbuffer


\startbuffer[1937]
大要:要旨。
\stopbuffer


\startbuffer[1938]
神圣功巧:四字分别为中医对望、闻、问、切的别称。
\stopbuffer


\startbuffer[1939]
壮观:增添雄伟宏壮的气象,这里指增光添彩。
\stopbuffer


\startbuffer[1940]
庸医杀人罪名:按明代律例,庸医过失杀人,将会被禁止行医,不会判死罪。
\stopbuffer


\startbuffer[1941]
圣躬:犹圣体。臣下称皇帝的身体。亦代指皇帝。
\stopbuffer


\startbuffer[1942]
寸、关、尺三部:中医切脉有三部,桡骨茎突处为关,关前为寸,关后为尺。
\stopbuffer


\startbuffer[1943]
『四气』以下数词:皆为中医术语。四气,指原气、营气、卫气、宗气;一说为中药寒、热、温、凉的四种药性。五郁,五种郁证,即木郁、火郁、土郁、金郁、水郁。七表、八里,脉象的分类。七表即浮、芤(kōu)、滑、数、弦、紧、洪七种。八里即微、沉、缓、涩、迟、伏、濡、弱八种。九候,古代中医诊脉方法,把人体分为上、中、下三部,每部取上、中、下三处动脉诊断;或在手的寸、关、尺三部,以浮、中、沉三种手法切脉,此处指第二种。浮中沉,三种切脉手法,分别为轻按、中按、重按。
\stopbuffer


\startbuffer[1944]
生熟药铺:卖生药和卖熟药的药铺。生药,指简单加工而未精制的药物,或天然药材。熟药,指经加工炮制的药材。
\stopbuffer


\startbuffer[1945]
罗:罗筛。
\stopbuffer


\startbuffer[1946]
乳:研磨。
\stopbuffer


\startbuffer[1947]
乳钵:研细药物的器具,形如臼而小。
\stopbuffer


\startbuffer[1948]
当头:抵押品,典押品。
\stopbuffer


\startbuffer[1949]
精虔制度:精心诚敬地制作。制度,制作。
\stopbuffer


\startbuffer[1950]
欲作生涯无本:想做生意,没有本钱。生涯,生计,生意。
\stopbuffer


\startbuffer[1951]
油蜡:即蜡烛。
\stopbuffer


\startbuffer[1952]
破结宣肠:破除郁结,疏导肠胃。宣,疏通,疏导。
\stopbuffer


\startbuffer[1953]
起夺人:耍弄,戏弄人。
\stopbuffer


\startbuffer[1954]
丸药:揉制药丸。
\stopbuffer


\startbuffer[1955]
衬:衬垫;凑近。
\stopbuffer


\startbuffer[1956]
众毛攒裘:聚集众多皮毛,能够缝成皮衣。比喻积少成多。
\stopbuffer


\startbuffer[1957]
龁(hé):咬,嚼。此处『龁支支』为象声词。
\stopbuffer


\startbuffer[1958]
耽病:带着病。耽,义同『担』。
\stopbuffer


\startbuffer[1959]
辅星:与下文『弼宿』同为北斗星斗柄附近的两颗星。
\stopbuffer


\startbuffer[1960]
烟娇:即胭娇,美女。
\stopbuffer


\startbuffer[1961]
吃一看十:指筵席上菜肴丰富,但实际吃的少,摆着看的多。
\stopbuffer


\startbuffer[1962]
斗糖龙缠:指糖缠,又称象生缠糖。用糖和果仁为主料,制成各种形状的糖果。
\stopbuffer


\startbuffer[1963]
透糖:一种油炸的糖饼。
\stopbuffer


\startbuffer[1964]
马兜铃:一种镇咳袪痰的中药。
\stopbuffer


\startbuffer[1965]
血蛊:中医指血瘀积聚在脾肠等处,导致腹大膨胀,甚至鼓满的病症。
\stopbuffer


\startbuffer[1966]
獬豸(xiè zhì):传说中的异兽。一角,能辨曲直。
\stopbuffer


\startbuffer[1967]
摇席破坐:谓在饮宴中中途离席。
\stopbuffer


\startbuffer[1968]
魆(xū)净:干干净净。魆,表示程度深。
\stopbuffer


\startbuffer[1969]
糟准:酒糟鼻。准,鼻子。
\stopbuffer


\startbuffer[1970]
东岳天齐:东岳天齐仁圣大帝。道教所奉泰山神。
\stopbuffer


\startbuffer[1971]
夜不收:古代军队中的哨探。彻夜在外活动,故名。
\stopbuffer


\startbuffer[1972]
乌兔:指日月。神话谓日中有乌,月中有兔,故称。
\stopbuffer


\startbuffer[1973]
玄关:佛教称入道的法门。
\stopbuffer


\startbuffer[1974]
凌云渡:传说中凡俗世界和佛国世界之间的深涧。
\stopbuffer


\startbuffer[1975]
纷纷垓垓:众多杂乱的样子。
\stopbuffer


\startbuffer[1976]
邓邓浑浑:混沌不清的样子。
\stopbuffer


\startbuffer[1977]
妆新:称新婚时所用的衣服、被褥、枕头等物。
\stopbuffer


\startbuffer[1978]
知疼着(zháo)热:形容关怀、体贴,多用于亲人间。着,感受。
\stopbuffer


\startbuffer[1979]
顺腿:指古时杖刑打在腰腿部,血流顺腿。
\stopbuffer


\startbuffer[1980]
胖袄:棉上衣。也专指将士的冬装。
\stopbuffer


\startbuffer[1981]
断头香:断折的线香或棒香。俗谓以断头香供佛,来生会得与亲人离散的果报。
\stopbuffer


\startbuffer[1982]
解识:知晓,熟悉。
\stopbuffer


\startbuffer[1983]
魇寐:即魇昧。用法术使人神智迷糊。
\stopbuffer


\startbuffer[1984]
脑麝:香料名。龙脑与麝香的并称。
\stopbuffer


\startbuffer[1985]
含忖:犹寒伧。讥笑,揭短。
\stopbuffer


\startbuffer[1986]
猥衰:狼狈不堪的样子。
\stopbuffer


\startbuffer[1987]
火牌:古代军中符信之一。兵丁至各地传令,给火牌一面,沿途凭牌向各驿站支领口粮。
\stopbuffer


\startbuffer[1988]
会垓:会战。刘邦曾围项羽于垓下,戏剧、小说因称会战为『会垓』。
\stopbuffer


\startbuffer[1989]
微躯:微贱的身躯。常用作谦辞。
\stopbuffer


\startbuffer[1990]
山坞(wù):山坳,山间平地。
\stopbuffer


\startbuffer[1991]
怛(dá)突:犹忐忑。怛,惊惧。
\stopbuffer


\startbuffer[1992]
瘢(bān)儿:疮痕,斑点。与下文『蒂儿』皆指器物上的瑕疵。
\stopbuffer


\startbuffer[1993]
犼(hǒu):一种形似犬的野兽。传说中观音的坐骑。
\stopbuffer


\startbuffer[1994]
啾疾:喻丧偶之痛。啾,形容鸟类失去伴侣之后的痛鸣声。
\stopbuffer


\startbuffer[1995]
解魇:禳解魇魅。魇,妖邪。
\stopbuffer


\startbuffer[1996]
张紫阳:即张伯端,北宋道士。后世奉为金丹派南宗始祖,称紫阳真人。
\stopbuffer


\startbuffer[1997]
气球:古代游戏用具,用以蹴踢的球。
\stopbuffer


\startbuffer[1998]
『飘扬翠袖』一段:包含了各种蹴鞠的身段玩法。如『拿头过论』为用头顶传球;『张』指观察;『泛』指踢球到位;『楷』指合乎规范。『出墙花』或为『转花枝』,一种三人踢法;『大过海』或为『八仙过海』,一种八人踢法。又如『明珠上佛头』『卧鱼』『平腰折膝蹲』等等,皆为踢球的姿势。绞裆,即裆下。臁(lián),小腿两侧。
\stopbuffer


\startbuffer[1999]
缘簿:化缘的簿本。
\stopbuffer


\startbuffer[2000]
打我的情:打情,指打主意。
\stopbuffer


\startbuffer[2001]
檀口:红艳的嘴唇。檀,浅绛色。
\stopbuffer


\startbuffer[2002]
老拳:指鹰爪。
\stopbuffer


\startbuffer[2003]
七年男女不同席:语出《礼记·内则》,以为七岁的男童女童需要分席吃饭。
\stopbuffer


\startbuffer[2004]
滑扢虀(jī):方言。指滑溜溜的。
\stopbuffer


\startbuffer[2005]
搀胸:江淮官话。齐胸。
\stopbuffer


\startbuffer[2006]
蠦(lú)蜂:一说为方言,即胡蜂。
\stopbuffer


\startbuffer[2007]
斑毛:即斑蝥。黑色硬壳,鞘翅上有黄黑色斑纹。
\stopbuffer


\startbuffer[2008]
牛蜢:即牛虻。
\stopbuffer


\startbuffer[2009]
抹蜡:即造字蜡。蝗虫。
\stopbuffer


\startbuffer[2010]
翛翛(xiāo):象声词。这里形容蚁虫密集、振翅疾飞的声音。
\stopbuffer


\startbuffer[2011]
树杪(miǎo):树梢。杪,树枝的细梢,末梢。
\stopbuffer


\startbuffer[2012]
撒货头口:指喂马。撒货,也作『撒和』,蒙古语,撒花。多引申指以饮食款客或喂饲、牵溜驴马。头口,指骡马驴牛之类大牲畜。
\stopbuffer


\startbuffer[2013]
云头履:云履,鞋头呈云头如意状。
\stopbuffer


\startbuffer[2014]
回回:指回族,也指伊斯兰教教徒。
\stopbuffer


\startbuffer[2015]
达达:鞑靼。始为中国北方的一个部落,后用作北方民族的统称。
\stopbuffer


\startbuffer[2016]
忙冗:忙碌。
\stopbuffer


\startbuffer[2017]
装幌子:比喻张扬,招摇。也谓出丑。
\stopbuffer


\startbuffer[2018]
唾着脸:形容不悦的神情。唾,表示鄙弃。
\stopbuffer


\startbuffer[2019]
一打三分低:意为一旦动手打人,便输了三分理。
\stopbuffer


\startbuffer[2020]
杓:同『勺』,勺子。
\stopbuffer


\startbuffer[2021]
等子:即戥(děng)子。一种小型的秤,用来称金、银、药品等分量小的东西。
\stopbuffer


\startbuffer[2022]
『在家不是贫』句:大意为在家时可以互相周济,不会有真的穷困;但在游途上遇到困难,就一点办法都没有。
\stopbuffer


\startbuffer[2023]
气淤淤:犹气吁吁。
\stopbuffer


\startbuffer[2024]
劖肉:指碎肉。劖,割,剜。
\stopbuffer


\startbuffer[2025]
金吾:一种手执的仪仗棒。铜制,两端涂金。
\stopbuffer


\startbuffer[2026]
的:相当于『着』。
\stopbuffer


\startbuffer[2027]
贡脓:溃烂生脓。
\stopbuffer


\startbuffer[2028]
阳沟:露天的排水沟。
\stopbuffer


\startbuffer[2029]
毗蓝婆:毗蓝,梵语音译,意为狂风、暴风。佛经中有十位罗刹女,第二位即毗蓝婆。
\stopbuffer


\startbuffer[2030]
龙华会:庙会名。在四月初八佛诞日,诸寺设斋,用五色香水浴佛,以为弥勒下生的象征,称为『龙华会』。
\stopbuffer


\startbuffer[2031]
嵎(yú):山势曲折险峻处。
\stopbuffer


\startbuffer[2032]
社前:春社日之前。社日,祭祀设神的日子,多在春分前后,此时燕子已归北方。
\stopbuffer


\startbuffer[2033]
四谛:佛教语。佛教基本教义之一。释迦牟尼最初说教的内容,即苦、集、灭、道四谛。
\stopbuffer


\startbuffer[2034]
了了:指了脱一切的智慧。
\stopbuffer


\startbuffer[2035]
哕(yuě):呕吐。
\stopbuffer


\startbuffer[2036]
蛩(qióng):蟋蟀。
\stopbuffer


\startbuffer[2037]
短路:拦路抢劫。
\stopbuffer


\startbuffer[2038]
贬解:押解。
\stopbuffer


\startbuffer[2039]
言不帮寸:指说话没有分寸,不着边际。帮,靠近。
\stopbuffer


\startbuffer[2040]
十一大曜:『七政四余』(见第五十一回注)的统称,共十一星曜。
\stopbuffer


\startbuffer[2041]
过头话:超过分寸的话;夸口的话。
\stopbuffer


\startbuffer[2042]
不尴不尬:犹言不明不白。
\stopbuffer


\startbuffer[2043]
走花弄水:指说大话,不着边际。
\stopbuffer


\startbuffer[2044]
失智:失神落魄的样子。
\stopbuffer


\startbuffer[2045]
赶面:即擀面。用棍棒把面团来回碾平压薄。
\stopbuffer


\startbuffer[2046]
以告者,过也:大意为把这件事告诉你的人说错了。语出《论语·子路》。
\stopbuffer


\startbuffer[2047]
执结:对官署开立,表示负责的字据。
\stopbuffer


\startbuffer[2048]
机括:机关。比喻计谋,心思。
\stopbuffer


\startbuffer[2049]
弄獐弄智:装模作样。
\stopbuffer


\startbuffer[2050]
抟风运海,振北图南:抟风,扶摇,乘风。运海,言乘着海水涌动。振北图南,指从北海飞向南海。化用自《庄子·逍遥游》对大鹏的描写。
\stopbuffer


\startbuffer[2051]
左右是左右:犹言反正如此。
\stopbuffer


\startbuffer[2052]
『楚歌』句:言楚汉垓下之战,项羽被刘邦围困,四面汉军唱起楚地歌谣,楚军军心动摇,军士纷纷跑散。
\stopbuffer


\startbuffer[2053]
变生翱翔,鷃笑龙惨:变生翱翔,指《庄子·逍遥游》所载北冥之鲲化为鹏的传说。鷃笑,《逍遥游》中写斥鷃曾嘲笑大鹏。龙惨,佛教传说大鹏以龙为食,故称『龙惨』。
\stopbuffer


\startbuffer[2054]
翮(hé):鸟的翅膀。
\stopbuffer


\startbuffer[2055]
递年:一年年,每年。
\stopbuffer


\startbuffer[2056]
赥赥(xī):同『嘻嘻』。笑声。
\stopbuffer


\startbuffer[2057]
才子:方言。刚才。又作『才自』。
\stopbuffer


\startbuffer[2058]
展:揩,抹。
\stopbuffer


\startbuffer[2059]
券:方言。用身子撞,撑。
\stopbuffer


\startbuffer[2060]
硬抢:方言。手感硬。也作『硬枪』。
\stopbuffer


\startbuffer[2061]
控:将容器出口朝下,让里边的液体慢慢流出。
\stopbuffer


\startbuffer[2062]
扛风:指因为身躯大而顶风。
\stopbuffer


\startbuffer[2063]
见阵:对阵,交战。
\stopbuffer


\startbuffer[2064]
坐家虎:比喻在家门口依仗自己的势力摆威风的人。
\stopbuffer


\startbuffer[2065]
嘑(hù)头:指只有一时冲动,没有后劲。
\stopbuffer


\startbuffer[2066]
寿器:指生前预制的棺木。
\stopbuffer


\startbuffer[2067]
坚饥:疑为『济饥』之误。济饥,即救饥。
\stopbuffer


\startbuffer[2068]
杂碎:指煮熟切碎的动物内脏。
\stopbuffer


\startbuffer[2069]
盘缠:供养。此处指消耗。
\stopbuffer


\startbuffer[2070]
三叉骨:脊椎骨最末端的一节。
\stopbuffer


\startbuffer[2071]
磨害:折磨伤害。
\stopbuffer


\startbuffer[2072]
三才阵势:一种将士兵分为左、中、右三路的阵法。
\stopbuffer


\startbuffer[2073]
邓:借作『扽』(dèn)。用力猛地一拉。
\stopbuffer


\startbuffer[2074]
掭掭:这里指喉咙里有东西的感觉。掭,轻轻拨动。
\stopbuffer


\startbuffer[2075]
棺材座子:垫棺材之物。比喻倒霉的东西。
\stopbuffer


\startbuffer[2076]
蓝旗手:明代军中的前哨,执蓝旗和令旗,负责清道、通报等事。
\stopbuffer


\startbuffer[2077]
说谎不瞒当乡人:指面对知情人不能说谎。
\stopbuffer


\startbuffer[2078]
撮弄:调唆,教唆。
\stopbuffer


\startbuffer[2079]
舍命之材:卖命的人。
\stopbuffer


\startbuffer[2080]
丢花棒儿:指使棍棒旋转的动作。
\stopbuffer


\startbuffer[2081]
象奴:古代专职驯象的仆役。
\stopbuffer


\startbuffer[2082]
儾(nànɡ):同『齉』,鼻子不通气。
\stopbuffer


\startbuffer[2083]
善胜:非常善良。
\stopbuffer


\startbuffer[2084]
尚气:赌气。
\stopbuffer


\startbuffer[2085]
红沙:旧时阴阳家称凶星当值为红沙。沙,亦作『煞』。红沙日不宜出行、动土等事。
\stopbuffer


\startbuffer[2086]
三十六宫:本意指帝王宫殿之多。宋邵雍有『三十六宫春自在』句,指《周易》的三十六个卦象,可以衍生万物。此处只为配合六六之数而作。
\stopbuffer


\startbuffer[2087]
秋风过耳:像秋风从耳边吹过。比喻漠不关心、毫不在意。
\stopbuffer


\startbuffer[2088]
圆了气:指蒸气上升。
\stopbuffer


\startbuffer[2089]
打紧:真的。
\stopbuffer


\startbuffer[2090]
空心:空腹。
\stopbuffer


\startbuffer[2091]
腾腾:即熥熥(tēng)。把食物重新蒸热。
\stopbuffer


\startbuffer[2092]
双掭灯:同时在两边拨动灯火。这里作名词。
\stopbuffer


\startbuffer[2093]
延地:到处。延,通『沿』。
\stopbuffer


\startbuffer[2094]
直身:一种日常长衫。
\stopbuffer


\startbuffer[2095]
师范:师父,老师。
\stopbuffer


\startbuffer[2096]
那玉:即挪玉,挪动玉趾。玉趾,对人脚步的敬称。
\stopbuffer


\startbuffer[2097]
鹊巢贯顶之头:传说如来佛在雪山修道时,有野鹊在其头顶筑巢。故称『鹊巢贯顶』。
\stopbuffer


\startbuffer[2098]
就了筋:就筋,犹抽筋。
\stopbuffer


\startbuffer[2099]
先祭汝口:佛教午斋前念诵的出食偈咒有句子:『大鹏金翅鸟,旷野鬼神众,罗刹鬼子母,甘露悉充满。』意为吃斋前要先施食给金翅鸟等众鬼。传说大鹏以龙为食,这样做可以使其不再吃龙。『先祭汝口』句即从此处化用而来。
\stopbuffer


\startbuffer[2100]
意攘心劳:形容心慌意乱。
\stopbuffer


\startbuffer[2101]
破玉:比喻开出洁白的花。
\stopbuffer


\startbuffer[2102]
西邸:王侯的官舍。此处指代王侯。
\stopbuffer


\startbuffer[2103]
月城:即瓮城。城外所筑用于防御的小城,呈半圆形,故称。
\stopbuffer


\startbuffer[2104]
高衙:高大的衙门,对对方官署的尊称。
\stopbuffer


\startbuffer[2105]
迸去:屏去,屏除。迸通『屏』。
\stopbuffer


\startbuffer[2106]
戕(qiāng):杀害。
\stopbuffer


\startbuffer[2107]
专把别人棺材抬在自家家里哭:比喻为他人伤心,无故自寻烦恼。
\stopbuffer


\startbuffer[2108]
先天之要旨:道教内丹术认为修炼先天精气神方可成道,而服食、炼药等旁门不能得道。
\stopbuffer


\startbuffer[2109]
真官:有官职的仙人。
\stopbuffer


\startbuffer[2110]
揖让差池:指所行的礼节有差错。
\stopbuffer


\startbuffer[2111]
昂昂烈烈:神气十足,志得意满的样子。
\stopbuffer


\startbuffer[2112]
五城兵马:『五城兵马司』的省称。明代在北京城设中、东、西、南、北五城兵马指挥司,负责治安、火禁、疏理沟渠、街道等事。
\stopbuffer


\startbuffer[2113]
我慢:佛教语。谓执我见而倨傲。
\stopbuffer


\startbuffer[2114]
遗体:旧谓子女的身体为父母所生,因称子女的身体为父母的『遗体』。
\stopbuffer


\startbuffer[2115]
总:聚集,汇集。
\stopbuffer


\startbuffer[2116]
南极老人星:即寿星。古人认为南极星主寿。
\stopbuffer


\startbuffer[2117]
龙骨:比喻瘦劲的枝干。
\stopbuffer


\startbuffer[2118]
煼(chǎo):古同『炒』。
\stopbuffer


\startbuffer[2119]
绰摸:捞取,指取食。
\stopbuffer


\startbuffer[2120]
『我自天牌传旨意』以下八句:『天牌』『锦屏风』『观灯十五』『天地分』『风云会』『拗马军』『峰十二』『对子』等词,皆来自骨牌术语。整首诗化用各种牌面名称,取其谐音而成诗。
\stopbuffer


\startbuffer[2121]
百舌:百舌鸟。又名乌鸫。善于模仿百鸟之声。
\stopbuffer


\startbuffer[2122]
先茔(yíng):先人的坟墓。茔,坟墓。
\stopbuffer


\startbuffer[2123]
散盘:散伙。
\stopbuffer


\startbuffer[2124]
唝(gǒng):撅起,翘起。
\stopbuffer


\startbuffer[2125]
替:给,被。此句指的是『四圣试禅心』中的旧事。
\stopbuffer


\startbuffer[2126]
贻(yí)累:招致祸害,牵连。
\stopbuffer


\startbuffer[2127]
格:方言。语助词。
\stopbuffer


\startbuffer[2128]
香厨:即香积厨。僧家的厨房。
\stopbuffer


\startbuffer[2129]
轮藏堂:收藏佛经的建筑。轮藏,藏置佛经的书架,设机轮,可旋转,故名。
\stopbuffer


\startbuffer[2130]
左笄(jī)绒锦帽:藏传佛教僧人所戴的一种帽子。
\stopbuffer


\startbuffer[2131]
播郎鼓:拨浪鼓。此处指藏传佛教的一种手鼓,名嘎巴拉鼓,形似拨浪鼓。
\stopbuffer


\startbuffer[2132]
捻手捻脚:捏手捏脚。
\stopbuffer


\startbuffer[2133]
脱空:没有着落;弄虚作假。
\stopbuffer


\startbuffer[2134]
倒沁着头:即倒着头。沁,方言,头向下垂。
\stopbuffer


\startbuffer[2135]
出票:出官府拿人的传票。
\stopbuffer


\startbuffer[2136]
泼泼撒撒:水从容器中散洒出来的样子。这里指抛洒浪费。
\stopbuffer


\startbuffer[2137]
秇:通『艺』,种植。这里指学习。
\stopbuffer


\startbuffer[2138]
慈航:佛教语。谓佛、菩萨以慈悲之心度人,如航船之济众,使脱离生死苦海。
\stopbuffer


\startbuffer[2139]
法云:佛教语。谓佛法如云,能覆盖一切。
\stopbuffer


\startbuffer[2140]
生生万法:指万事万物。生生,犹众生。万法,佛教语,指一切事物。
\stopbuffer


\startbuffer[2141]
檀越:梵语音译。施主。
\stopbuffer


\startbuffer[2142]
挨挨拶拶(zā):挨近,挤逼。犹言挤来挤去。
\stopbuffer


\startbuffer[2143]
略节:简要。多指书面报告。
\stopbuffer


\startbuffer[2144]
不至紧:不要紧,不打紧。
\stopbuffer


\startbuffer[2145]
两下里角斗:双方争斗。两下里,双方。角斗,搏斗,争斗。
\stopbuffer


\startbuffer[2146]
不当:不太,不大。
\stopbuffer


\startbuffer[2147]
偷生㧚(wǎ)熟:指偷情不分生人熟人。㧚,方言,舀。
\stopbuffer


\startbuffer[2148]
鸾俦(chóu):比喻夫妻。俦,伴侣。
\stopbuffer


\startbuffer[2149]
小坐跌法:一种擒拿摔跤的手法。
\stopbuffer


\startbuffer[2150]
戴山鳌:传说中驮起海上仙山的巨鳌。戴,用头顶着。
\stopbuffer


\startbuffer[2151]
雷焕剑:雷焕,东晋人,相传他在丰城发现了龙泉、太阿两把宝剑。
\stopbuffer


\startbuffer[2152]
吕虔刀:吕虔,三国时魏国将领,有一口宝刀,因人言配此刀者将登三公之位,遂将刀赠予王祥。后来王祥果然位至三公,其族人也开启了琅琊王氏的兴盛。
\stopbuffer


\startbuffer[2153]
阒(qù):寂静。
\stopbuffer


\startbuffer[2154]
猛可:突然。
\stopbuffer


\startbuffer[2155]
管鲍分金:管,管仲。鲍,鲍叔牙。春秋时人。二人曾一起经商,管仲总是多分得一些利润,用于赡养老母,鲍叔牙也非常理解。后人以『管鲍分金』比喻情谊深厚,相知相悉。
\stopbuffer


\startbuffer[2156]
孙庞斗智:孙,孙膑。庞,庞涓。战国时人。二人曾是同学,后庞涓仕魏国,因嫉妒陷害孙膑,孙膑逃至齐国为将,在战役中射死了庞涓。『孙庞斗智』形容昔日友人反目成仇。
\stopbuffer


\startbuffer[2157]
气心风:指因发怒而精神失常。心风,指癫症。
\stopbuffer


\startbuffer[2158]
结掳:结伙掳掠。
\stopbuffer


\startbuffer[2159]
打伙儿:合伙,结伴。
\stopbuffer


\startbuffer[2160]
篾丝鬏髻:用篾丝编的假髻。
\stopbuffer


\startbuffer[2161]
吊得嘴惯:惯开玩笑。吊嘴,搬弄唇舌,说玩笑话。
\stopbuffer


\startbuffer[2162]
柞撒:旧式榨油用的大木楔。
\stopbuffer


\startbuffer[2163]
顺口话儿:随声附和的话。
\stopbuffer


\startbuffer[2164]
交媾(gòu):阴阳结合。
\stopbuffer


\startbuffer[2165]
标:盯,注视。
\stopbuffer


\startbuffer[2166]
二滴水:滴水,一种中式的瓦。一端有下垂的边,呈圆尖形,盖房时置于檐口,用于屋顶排水。有两层滴水瓦构成的屋檐,称二滴水。
\stopbuffer


\startbuffer[2167]
喜花儿:倒酒时酒面上起的泡沫。
\stopbuffer


\startbuffer[2168]
肉落千斤:比喻心情极为痛苦。
\stopbuffer


\startbuffer[2169]
金猊:一种香炉,炉盖作狻猊形,焚香时,烟从口出。
\stopbuffer


\startbuffer[2170]
垒钿(diàn):即嵌螺钿。一种工艺,将各色贝壳镶嵌在器物上。
\stopbuffer


\startbuffer[2171]
石花菜:一种海藻,紫红色。
\stopbuffer


\startbuffer[2172]
王瓜:亦名土瓜,葫芦科,果椭圆,熟时呈红色,也是黄瓜的别称。
\stopbuffer


\startbuffer[2173]
方旦:一说『旦』为『菹』的脱坏。古人云冬瓜宜切方小菹(腌制)。
\stopbuffer


\startbuffer[2174]
鹾(cuó):盐;咸味。
\stopbuffer


\startbuffer[2175]
耳报:比喻暗中报告消息者。
\stopbuffer


\startbuffer[2176]
皮袋:指肚皮。
\stopbuffer


\startbuffer[2177]
捽手:甩开手,放手。
\stopbuffer


\startbuffer[2178]
绮(qǐ)罗:指穿着绮罗的人。贵妇、美女的代称。
\stopbuffer


\startbuffer[2179]
箔:帘子。多以苇子或秫秸织成。
\stopbuffer


\startbuffer[2180]
绿蚁:酒面上浮起的绿色泡沫。借指酒。
\stopbuffer


\startbuffer[2181]
姚黄魏紫:牡丹的两个名品,因分别由姚家和魏家栽培而得名。
\stopbuffer


\startbuffer[2182]
梆子精:梆子,巡更或旧时衙门用以集散人众所敲的响器,用竹子或挖空的木头制成。这里比喻空的躯壳。
\stopbuffer


\startbuffer[2183]
夙(sù)世:前世。
\stopbuffer


\startbuffer[2184]
蓝桥水涨:民间故事,版本各异。大致为一书生与女子相爱,约定在蓝桥相会,书生等女子不至,水涨而不愿离去,终于溺死。
\stopbuffer


\startbuffer[2185]
祆(xiān)庙烟沉:民间有『火烧祆庙』的故事,相传蜀帝公主与乳母陈氏之子相爱,陈氏子出宫后相思成疾,二人遂约定在祆庙相会。及公主入庙,陈氏子正熟睡,于是解下小时候的玉环,放在陈氏子怀中而去。陈氏子醒后,悔恨不已,怨气化火身,烧毁了祆庙。后用此典形容姻缘不遂。祆庙,祆教祭祀火神的寺院。
\stopbuffer


\startbuffer[2186]
贴换:双方兑换时,一方给另一方贴补价值差额。
\stopbuffer


\startbuffer[2187]
吓:象声词,形容笑声。
\stopbuffer


\startbuffer[2188]
倒坐儿:即倒座。古代建筑大厅多为坐北朝南,坐南朝北者称为『倒座』。
\stopbuffer


\startbuffer[2189]
龙吞口:指龙口吞衔住器物的某一部分的纹饰。
\stopbuffer


\startbuffer[2190]
鎏金:一种器物表面饰金的工艺。
\stopbuffer


\startbuffer[2191]
告人死罪得死罪:诬告别人死罪,自己也会被判死罪。明代法律中有同类条例。比喻诬陷者会承担相应罪责。
\stopbuffer


\startbuffer[2192]
年甲在牒:年龄记录在度牒上。古代僧人度牒上登记有姓名、相貌、年龄等项。
\stopbuffer


\startbuffer[2193]
假:借。
\stopbuffer


\startbuffer[2194]
焚香:古代接圣旨时需要有焚香叩拜的仪式。
\stopbuffer


\startbuffer[2195]
『死了莫与老头儿同墓』句:指老年人知道的事情多,会揭人老底。
\stopbuffer


\startbuffer[2196]
喜花:原指贴在婚嫁物品上的花样剪纸。这里指结婚生子。花,也指小孩。
\stopbuffer


\startbuffer[2197]
顶:顶撞,以语言相逆。
\stopbuffer


\startbuffer[2198]
分摆:分配安排。
\stopbuffer


\startbuffer[2199]
弢(tāo):同『韬』。隐藏,收敛。
\stopbuffer


\startbuffer[2200]
嶷嶷(nì):高耸,高峻。
\stopbuffer


\startbuffer[2201]
青田:山名。道教称三十六洞天之一。
\stopbuffer


\startbuffer[2202]
和尚拖木头,做出了寺:歇后语。『寺』谐音『事』。比喻发生不测。
\stopbuffer


\startbuffer[2203]
啈(hèng)声:厉声,发狠的声音。
\stopbuffer


\startbuffer[2204]
恁(nín):同『您』。
\stopbuffer


\startbuffer[2205]
卖嘴:耍嘴皮子。
\stopbuffer


\startbuffer[2206]
黑话:这里指吓唬人的话。
\stopbuffer


\startbuffer[2207]
冲融:充盈弥漫。
\stopbuffer


\startbuffer[2208]
硗(qiāo):薄。
\stopbuffer


\startbuffer[2209]
君子小人不同:指有君子,也有小人。古时店家提醒客人注意的套话。
\stopbuffer


\startbuffer[2210]
吊搭:指帘子。
\stopbuffer


\startbuffer[2211]
勒掯:为难,刁难。
\stopbuffer


\startbuffer[2212]
收顶绳:明人戴头巾,用布包头后,顶上用绳系住发髻。
\stopbuffer


\startbuffer[2213]
打花:闲谈,说笑。
\stopbuffer


\startbuffer[2214]
瓦查儿:瓦渣,碎瓦片。
\stopbuffer


\startbuffer[2215]
打火:行人在旅途中生火做饭或吃饭。
\stopbuffer


\startbuffer[2216]
客纲客纪:经常旅行者的经验之谈。
\stopbuffer


\startbuffer[2217]
撰:用同『赚』。
\stopbuffer


\startbuffer[2218]
客子:雇工。
\stopbuffer


\startbuffer[2219]
院:指妓院。
\stopbuffer


\startbuffer[2220]
表子:娼妓。『表』是『外』的意思,意为外室。后多作『婊子』。
\stopbuffer


\startbuffer[2221]
漏肩风:一种病。因风寒湿邪引起的肩部疼痛。
\stopbuffer


\startbuffer[2222]
羞明:畏光。
\stopbuffer


\startbuffer[2223]
十日滩头坐,一日行九滩:比喻淡季时生意冷清,一到旺季则利润数倍。
\stopbuffer


\startbuffer[2224]
单浪瓦儿:建筑的屋顶只铺一层浪板瓦,光线可透进屋内。
\stopbuffer


\startbuffer[2225]
销钉:也称『销子』,形似钉子,插在器物中用于连接或固定。
\stopbuffer


\startbuffer[2226]
本身:本钱。
\stopbuffer


\startbuffer[2227]
手眼:手段,经验。
\stopbuffer


\startbuffer[2228]
火甲人夫:火甲,明代城市中负责巡夜、维护治安的应役。人夫,受雇用或服役的人。
\stopbuffer


\startbuffer[2229]
款段:马行迟缓貌。
\stopbuffer


\startbuffer[2230]
千金市骨:战国时郭隗以古人用五百金买已死千里马的骨头为喻,劝谏燕昭王要礼遇天下贤士,则能为其效劳。
\stopbuffer


\startbuffer[2231]
登答:对答。
\stopbuffer


\startbuffer[2232]
守困:指守夜。
\stopbuffer


\startbuffer[2233]
军余:指未取得正式军籍的军人。
\stopbuffer


\startbuffer[2234]
合同:和合齐同。
\stopbuffer


\startbuffer[2235]
暴云:翻涌的云气。
\stopbuffer


\startbuffer[2236]
间关:拟声词,鸟鸣的声音。
\stopbuffer


\startbuffer[2237]
躲懒:偷懒。
\stopbuffer


\startbuffer[2238]
口紧:犹言嘴馋。
\stopbuffer


\startbuffer[2239]
吃嘴:贪吃,嘴馋。
\stopbuffer


\startbuffer[2240]
上大人:旧时识字读物中开篇的三个字。
\stopbuffer


\startbuffer[2241]
觜:此处义同『龇』。
\stopbuffer


\startbuffer[2242]
染博士:专职染色的工匠。
\stopbuffer


\startbuffer[2243]
断幺绝六:骨牌中两种最不利的牌面。手中没有带『幺』的牌为『断幺』,没有带『六』的牌为『绝六』。这里指断根绝户。
\stopbuffer


\startbuffer[2244]
谶(chèn)语:谶言,预言。
\stopbuffer


\startbuffer[2245]
拿云手:比喻高强的本领。
\stopbuffer


\startbuffer[2246]
鱼水盆里捻苍蝇:杀鱼用的盆容易招来苍蝇。比喻完成某事轻而易举。
\stopbuffer


\startbuffer[2247]
一言既出,如白染皂:比喻说出的话不能反悔。如白染皂,白色染成了黑色。皂,黑色。
\stopbuffer


\startbuffer[2248]
主子:角色。传统戏曲中演员的类别。比喻生活中某种类型的人物。
\stopbuffer


\startbuffer[2249]
刮毒:犹狠毒。
\stopbuffer


\startbuffer[2250]
鳏(guān)居:独身无妻室。鳏,无妻或丧妻的男子。
\stopbuffer


\startbuffer[2251]
劈心里:从当中,从中。
\stopbuffer


\startbuffer[2252]
溜雨:从沟中流下雨水。
\stopbuffer


\startbuffer[2253]
生气:活人的气息。
\stopbuffer


\startbuffer[2254]
庐墓:古人于父母或师长死后,服丧期间在墓旁搭盖小屋居住,守护坟墓,谓之庐墓。
\stopbuffer


\startbuffer[2255]
手插鱼篮,避不得鯹(xīnɡ):意指已插手某事,就无法避免会有的后果。鯹,鱼腥味。
\stopbuffer


\startbuffer[2256]
出名:犹具名,署名。指对某事负责。
\stopbuffer


\startbuffer[2257]
俦:相比。
\stopbuffer


\startbuffer[2258]
貔貅(pí xiū):传说中的猛兽。
\stopbuffer


\startbuffer[2259]
猪牙子:小猪。
\stopbuffer


\startbuffer[2260]
缠长:即缠帐。难缠,纠缠。
\stopbuffer


\startbuffer[2261]
脚多:指多手多脚,好乱插手别人的事。
\stopbuffer


\startbuffer[2262]
人肉巴子:即人肉干。
\stopbuffer


\startbuffer[2263]
玄驹:亦作『玄蚼』。蚂蚁的别名。
\stopbuffer


\startbuffer[2264]
复了三:逝者埋葬三天后,家人招魂祭奠的仪式。
\stopbuffer


\startbuffer[2265]
大料:即八角茴香,一种香料。
\stopbuffer


\startbuffer[2266]
毒情:冤仇。
\stopbuffer


\startbuffer[2267]
母儿:即模子。
\stopbuffer


\startbuffer[2268]
殢(tì):滞留,停留。
\stopbuffer


\startbuffer[2269]
没是处:犹言不得了,没办法,不知怎么办好。
\stopbuffer


\startbuffer[2270]
思想:想念。
\stopbuffer


\startbuffer[2271]
握头:挑一头重的担子。
\stopbuffer


\startbuffer[2272]
趁头:与另一端重量相等的支挂物体。
\stopbuffer


\startbuffer[2273]
『嫩焯黄花菜』一段:本段韵语中数词——黄花菜、白鼓丁、浮蔷、马齿苋、江荠、雁肠英、燕子不来香、芽儿拳、马蓝头、狗脚迹、猫耳朵、野落荜、灰条、剪刀股、牛塘利、倒灌、窝螺、操帚荠、碎米荠、莴菜荠、乌英花、菱科、蒲根菜、茭儿菜、看麦娘、破破纳、苦麻台、藩篱架、雀儿绵单、猢狲脚迹、油灼灼、斜蒿、青蒿、抱娘蒿、灯蛾儿、板荞荞、羊耳秃、枸杞头、乌蓝——皆为野菜名。酸虀,酸渍。熝(āo),熬,煮。
\stopbuffer


\startbuffer[2274]
青衣:古代求雨者常穿青衣。
\stopbuffer


\startbuffer[2275]
逊避:退让,退避。逊,退让。
\stopbuffer


\startbuffer[2276]
沟浍(kuài):泛指田间水道。浍,田间水渠。
\stopbuffer


\startbuffer[2277]
民瘼(mò):民众的疾苦。瘼,病痛,泛指疾苦。
\stopbuffer


\startbuffer[2278]
老爹:衙役对长官、仆人对家主的尊称。
\stopbuffer


\startbuffer[2279]
拔济:佛教语。犹济度。亦泛指拯救。
\stopbuffer


\startbuffer[2280]
苏:缓解,解除。
\stopbuffer


\startbuffer[2281]
苍蝇包网儿,好大面皮:大意为苍蝇戴网巾,冒充头大脸大。喻人自不量力,好充脸大。
\stopbuffer


\startbuffer[2282]
十二月二十五日:民间十二月二十五日有接玉皇的习俗,认为玉皇大帝在这一天会下界勘察民间善恶,定下来年祸福。
\stopbuffer


\startbuffer[2283]
餂:同『舔』。
\stopbuffer


\startbuffer[2284]
锁梃:锁销,锁杆。
\stopbuffer


\startbuffer[2285]
解释:消除,消解。
\stopbuffer


\startbuffer[2286]
列缺:高空中闪电所现的空隙,指代闪电。
\stopbuffer


\startbuffer[2287]
调弄:原指演奏乐器,这里指施弄,摆布。
\stopbuffer


\startbuffer[2288]
雷车:传说中雷神出行乘坐的车子。也用来指雷声。
\stopbuffer


\startbuffer[2289]
元运:天运,天命。
\stopbuffer


\startbuffer[2290]
高真:仙人。
\stopbuffer


\startbuffer[2291]
扳留:挽留。
\stopbuffer


\startbuffer[2292]
赆仪:送行的礼物。
\stopbuffer


\startbuffer[2293]
掳嘴:白吃人家东西,吃白食。
\stopbuffer


\startbuffer[2294]
菽(shū):豆类的总称。
\stopbuffer


\startbuffer[2295]
般般:众多貌。
\stopbuffer


\startbuffer[2296]
『长史府』等:明藩王府下设长史司、审理所、典膳所。玉华王府或参照此写成。
\stopbuffer


\startbuffer[2297]
灶君:即灶神,脸黑,故用来比喻沙和尚。
\stopbuffer


\startbuffer[2298]
不等齐:不等齐声。
\stopbuffer


\startbuffer[2299]
活淘气:指活受气。淘气,怄气。
\stopbuffer


\startbuffer[2300]
歪缠:无理取闹,胡搅蛮缠。
\stopbuffer


\startbuffer[2301]
承应:指妓女、艺人应宫廷或官府之召表演、侍奉。
\stopbuffer


\startbuffer[2302]
撮弄:这里指变戏法。
\stopbuffer


\startbuffer[2303]
杳窎(diào):遥远。窎,深远,遥远。
\stopbuffer


\startbuffer[2304]
一藏之数:一藏佛经的数量,小说中设定为五千零四十八卷。
\stopbuffer


\startbuffer[2305]
子午周天:道教的一种功法,又称小周天。
\stopbuffer


\startbuffer[2306]
着力:用力,尽力。
\stopbuffer


\startbuffer[2307]
埂头:山埂的尽头。
\stopbuffer


\startbuffer[2308]
盘桓:玩弄。
\stopbuffer


\startbuffer[2309]
花帐儿:亦作『花账』,虚报的帐目。
\stopbuffer


\startbuffer[2310]
落:得到。形容经手钱财时,私下扣取小部分,以充私囊。
\stopbuffer


\startbuffer[2311]
不管:不关,不涉及。
\stopbuffer


\startbuffer[2312]
腔:量词。多用于猪和羊。
\stopbuffer


\startbuffer[2313]
邀:方言。驱赶,吆喝。
\stopbuffer


\startbuffer[2314]
扳:同『攀』。
\stopbuffer


\startbuffer[2315]
生口:牲口,畜生。
\stopbuffer


\startbuffer[2316]
物见主,必定取:物主在别处看见自己的东西,必定会取回。
\stopbuffer


\startbuffer[2317]
夤(yín)夜:深夜。
\stopbuffer


\startbuffer[2318]
当情:一定。
\stopbuffer


\startbuffer[2319]
都头:指头领。
\stopbuffer


\startbuffer[2320]
白泽:传说中的神兽,形似狮子,能言。
\stopbuffer


\startbuffer[2321]
抟象:传说中一种能攻击大象的狮子。
\stopbuffer


\startbuffer[2322]
强:此处同『僵』,僵硬。
\stopbuffer


\startbuffer[2323]
坎宫:正北方。
\stopbuffer


\startbuffer[2324]
镵(chán):尖锐,锋利。
\stopbuffer


\startbuffer[2325]
看看:渐渐。
\stopbuffer


\startbuffer[2326]
八万四千:佛教表示事物众多的数字,后用来形容极多。
\stopbuffer


\startbuffer[2327]
绵蛮:指小鸟或鸟鸣声。
\stopbuffer


\startbuffer[2328]
桯(tīng):横木。
\stopbuffer


\startbuffer[2329]
被:遭遇,遭受。
\stopbuffer


\startbuffer[2330]
天渊:高天和深渊。指天壤之别。
\stopbuffer


\startbuffer[2331]
顿脱群思:顿时摆脱各种杂念。思谐音『狮』,此处以狮精喻杂念。
\stopbuffer


\startbuffer[2332]
玉性:玉坚贞的性质。
\stopbuffer


\startbuffer[2333]
浮屠:这里指佛塔。
\stopbuffer


\startbuffer[2334]
应佛僧:也叫『应付僧』『应赴僧』,专门支应佛事的和尚。
\stopbuffer


\startbuffer[2335]
上刹:对佛寺的敬称。
\stopbuffer


\startbuffer[2336]
试灯:民间习俗元宵节晚上张灯,在节前张灯预赏谓之试灯。
\stopbuffer


\startbuffer[2337]
金谷园:晋代富豪石崇所筑的园馆。后指代豪华的园林。
\stopbuffer


\startbuffer[2338]
《辋川图》:唐王维曾在辋川隐居,将其别业绘成《辋川图》。后借指风景幽胜之处。
\stopbuffer


\startbuffer[2339]
冯夷:传说中的黄河之神,即河伯。泛指水神。
\stopbuffer


\startbuffer[2340]
檠(qíng):同『擎』。举,托。
\stopbuffer


\startbuffer[2341]
李白高乘:传说李白去世前,在船头赏月,有鲸出水面载其上天而去。
\stopbuffer


\startbuffer[2342]
金吾不禁:古代由金吾禁人夜行,只在正月十四、十五、十六这三天开放夜禁。金吾,官名,掌管京城警卫。
\stopbuffer


\startbuffer[2343]
躧跷:即踩高跷。
\stopbuffer


\startbuffer[2344]
旻(mín):天,天空。
\stopbuffer


\startbuffer[2345]
吃累:受累,负担重。
\stopbuffer


\startbuffer[2346]
酥合香油:即苏合香油。苏合,木名,原产小亚细亚,其树脂称苏合香。
\stopbuffer


\startbuffer[2347]
灯马:成束的灯心。
\stopbuffer


\startbuffer[2348]
否塞:闭塞,困厄。
\stopbuffer


\startbuffer[2349]
邓邓呆呆:呆笨迟钝貌。邓,借用作『钝』。
\stopbuffer


\startbuffer[2350]
号头:即号角。
\stopbuffer


\startbuffer[2351]
灵窍:指眼睛。
\stopbuffer


\startbuffer[2352]
蜚英:飞舞的花瓣。蜚,同『飞』。
\stopbuffer


\startbuffer[2353]
权:威势。
\stopbuffer


\startbuffer[2354]
天文之象:相传天上星曾入犀牛角,致使犀牛角中有白星,称为通天。
\stopbuffer


\startbuffer[2355]
『有兕犀』句:兕犀,角生于额上的犀,一说为雌犀的一种。牯犀,一种毛犀,角上有粟纹,可制腰带。斑犀,角上斑白分明的雌犀。胡冒犀,即胡帽犀,角生于鼻上的犀。堕罗犀,产于古南海堕和罗国的犀牛。通天花文犀,指犀角上有白色花纹从根部直至顶尖者。
\stopbuffer


\startbuffer[2356]
能开水道:传说通天犀的犀角可以分开水道。
\stopbuffer


\startbuffer[2357]
四木禽星:指禽星中的角木蛟、斗木獬、奎木狼、井木犴,星命学中四木禽星与牛相克。
\stopbuffer


\startbuffer[2358]
望月之犀:相传犀牛望月,月形入角,故有了月牙状的犀牛角。这里指普通的犀牛。
\stopbuffer


\startbuffer[2359]
降手儿:犹克星。
\stopbuffer


\startbuffer[2360]
艮方:即东北方。
\stopbuffer


\startbuffer[2361]
造字琚:即砗磲。
\stopbuffer


\startbuffer[2362]
漓:通『离』。背离,丧失。
\stopbuffer


\startbuffer[2363]
连喊是喊:意为连声喊叫。『连……是……』,方言,表示连续做某事。
\stopbuffer


\startbuffer[2364]
的决:旧律,受杖刑,按判定数施行,谓之的决。亦泛指定罪。
\stopbuffer


\startbuffer[2365]
佐贰:辅佐主司的官员。明清时,凡知府、知州、知县的辅佐官,如通判、州同、县丞等,统称佐贰。其品级略低于主管官。
\stopbuffer


\startbuffer[2366]
情真罪当:指案件情况清楚,犯罪情节属实。古代判决的常用语。
\stopbuffer


\startbuffer[2367]
满斗焚香:多股香攒扎堆成塔形,名为斗香。点着时由上而下层层燃烧,烟火旺盛。
\stopbuffer


\startbuffer[2368]
蠲(juān):免除。
\stopbuffer


\startbuffer[2369]
饮爱:享受爱戴。饮,受。
\stopbuffer


\startbuffer[2370]
梦梦乍:犹迷迷糊糊。
\stopbuffer


\startbuffer[2371]
三台:三台星,古人认为天上有三台六星,两两而居。又天子有三台,用以观察天地。
\stopbuffer


\startbuffer[2372]
个里:这里,其中。
\stopbuffer


\startbuffer[2373]
清清的:犹言白白地。
\stopbuffer


\startbuffer[2374]
替:介词。和,同。
\stopbuffer


\startbuffer[2375]
长话:冗长拉杂的话。
\stopbuffer


\startbuffer[2376]
十方常住:佛教语。谓接待往来僧人的寺院。常住,僧、道称寺舍、田地、什物等。
\stopbuffer


\startbuffer[2377]
随喜:这里指欢喜之意随瞻拜佛像而生。用以称游谒寺院。
\stopbuffer


\startbuffer[2378]
长住:长期住在寺院的僧人。
\stopbuffer


\startbuffer[2379]
挂榻:寺院中暂住的过往僧人。
\stopbuffer


\startbuffer[2380]
头底:底细。
\stopbuffer


\startbuffer[2381]
一望:指目力所及的距离。
\stopbuffer


\startbuffer[2382]
檀那:梵语音译。施主;布施。
\stopbuffer


\startbuffer[2383]
须达多:给孤独长者的本名,意译为『善施』。
\stopbuffer


\startbuffer[2384]
听说有马,就知是官差的:古代驿站配备有铺马,需要有官方凭证方可使用,故有此说。
\stopbuffer


\startbuffer[2385]
款迓:款待,接待。
\stopbuffer


\startbuffer[2386]
彩楼:用彩色绸帛结扎的棚架。一般用于祝贺节日盛典喜庆之事。也指七夕乞巧楼。
\stopbuffer


\startbuffer[2387]
引袋:招文袋,古代一种挂在腰带上装文件或财物的小袋子。
\stopbuffer


\startbuffer[2388]
绣女:少女。也指古代备选为妃嫔宫女的少女。
\stopbuffer


\startbuffer[2389]
撮拥:簇拥。
\stopbuffer


\startbuffer[2390]
放赦:释放,赦免。
\stopbuffer


\startbuffer[2391]
领纳:接受。
\stopbuffer


\startbuffer[2392]
嘴巴姑子:即嘴巴骨子。颔骨,下巴骨。借指嘴脸,长相。
\stopbuffer


\startbuffer[2393]
三钱银子买个老驴,自夸骑得:夸耀自己用极低的价钱买来的老驴好骑。比喻东西不好却还自吹自擂,没有自知之明。
\stopbuffer


\startbuffer[2394]
退送纸:用来送走鬼神的冥纸。退送,指把作祟的鬼神驱退送走。
\stopbuffer


\startbuffer[2395]
一得:一点可取之处,一点长处。
\stopbuffer


\startbuffer[2396]
贵人:新郎的雅称。
\stopbuffer


\startbuffer[2397]
啸聚:呼啸聚集。多指盗匪或贼寇以呼啸声聚众集合。
\stopbuffer


\startbuffer[2398]
周堂通利:指宜办婚事的吉日。阴阳家有『嫁娶周堂图』,为判断嫁娶日期吉凶的依据。
\stopbuffer


\startbuffer[2399]
猿猴献果:古代占卜术中的一种吉卦。
\stopbuffer


\startbuffer[2400]
把热舌头铎我:比喻嘲笑、讥讽人。铎,此处作动词,戳,啄。
\stopbuffer


\startbuffer[2401]
五更三点:古时一夜分五更,每更分五点。第五更为早三点至五点。五更三点,为早四点十分左右,指天即将亮的时候。
\stopbuffer


\startbuffer[2402]
春罍:赏春的酒宴。
\stopbuffer


\startbuffer[2403]
上头:指女子束发插笄,是成亲的标识。这里表示预备成亲。
\stopbuffer


\startbuffer[2404]
添妆:指向新娘赠送财物礼品。
\stopbuffer


\startbuffer[2405]
餪(nuǎn)子:古代的一种礼仪,女嫁后三日,娘家或亲戚馈送食品或办酒祝贺。
\stopbuffer


\startbuffer[2406]
随喜:这里指因别人的欣喜而欣喜。
\stopbuffer


\startbuffer[2407]
洪钧:指天。洪,大。钧,制陶器所用的转轮。
\stopbuffer


\startbuffer[2408]
芰(jì)荷:菱叶与荷叶。
\stopbuffer


\startbuffer[2409]
饱饫(yù):吃饱。饫,饱食。
\stopbuffer


\startbuffer[2410]
板脂:即板油。猪的体腔内壁上呈片状的脂肪。
\stopbuffer


\startbuffer[2411]
此句后,世本有『只谨言只谨言』六字,疑为衍文。
\stopbuffer


\startbuffer[2412]
乘鸾:传说春秋时秦有萧史善吹箫,赢得了穆公女弄玉的爱慕,穆公遂将女儿嫁给萧史。后夫妻二人吹箫引来了凤凰,双双乘凤凰升天而去。
\stopbuffer


\startbuffer[2413]
会合:指姻缘、婚姻匹配,配合。引申为交合。
\stopbuffer


\startbuffer[2414]
鳷(zhī)鹊:传说中的异鸟。
\stopbuffer


\startbuffer[2415]
碓嘴:舂米的杵。末梢略尖如鸟嘴,故名。
\stopbuffer


\startbuffer[2416]
虚的:虚实,真假。
\stopbuffer


\startbuffer[2417]
精:光。
\stopbuffer


\startbuffer[2418]
明暗:比喻真假,是非。
\stopbuffer


\startbuffer[2419]
毛颖:毛笔的别称。毛笔多用兔毛制作,暗合妖精的本相。
\stopbuffer


\startbuffer[2420]
太阴星君:指月神。
\stopbuffer


\startbuffer[2421]
姮娥:即嫦娥。嫦娥原作姮娥,汉代因避文帝刘恒讳,改称嫦娥。
\stopbuffer


\startbuffer[2422]
素娥:白衣美女,用指月宫仙女。也是嫦娥的别称。
\stopbuffer


\startbuffer[2423]
逭(huàn):宽恕,免除。
\stopbuffer


\startbuffer[2424]
拉闲:闲谈,闲扯。
\stopbuffer


\startbuffer[2425]
虹流:与『电绕』均为帝王降生的瑞应。传说上古帝少昊氏之母有梦,见星如虹,流入华渚,遂意感而生少昊。
\stopbuffer


\startbuffer[2426]
电绕:传说黄帝之母在郊野见大电绕北斗枢星,然后怀孕,诞下黄帝。
\stopbuffer


\startbuffer[2427]
斗南:夏季黄昏时,北斗的斗柄指向南方,故称。
\stopbuffer


\startbuffer[2428]
日永:指夏至。此日白昼在一年中最长,故称。也泛指夏天白昼长。
\stopbuffer


\startbuffer[2429]
虎坐门楼:中间略高,两侧略低,四柱落地,好似猛虎立坐的门楼。门楼,牌楼形式的门脸,多用于祠堂、寺庙或豪宅等建筑。
\stopbuffer


\startbuffer[2430]
影壁:大门内或屏门内做屏蔽的墙壁。又称照壁、照墙。
\stopbuffer


\startbuffer[2431]
妈妈:称年长的已婚妇女。也用于丈夫称老妻。
\stopbuffer


\startbuffer[2432]
上紧:着急。
\stopbuffer


\startbuffer[2433]
结梢:收尾。梢,末尾。
\stopbuffer


\startbuffer[2434]
尺工:即工尺。中国传统音乐音阶上各个音的总称,也是乐谱上各个记音符号的总称。
\stopbuffer


\startbuffer[2435]
《水忏》:佛教经文之一,又叫《慈悲水忏》。据传是唐代悟达禅师遇圣僧用水替他洗好面疮后所作。
\stopbuffer


\startbuffer[2436]
回向:佛教语。指将自己所修的功德回转趋向众生和佛果。
\stopbuffer


\startbuffer[2437]
束修:义同『束脩』。本意为十条干肉。常用作馈赠的礼物。也指学生送给老师的酬金。
\stopbuffer


\startbuffer[2438]
拿过了班儿;摆架子过了头。
\stopbuffer


\startbuffer[2439]
书办:管办文书的属吏。亦泛指掌管文书翰墨的人。
\stopbuffer


\startbuffer[2440]
努嘴胖唇:嘟着嘴。形容不高兴。
\stopbuffer


\startbuffer[2441]
骈填:聚会,连属。形容多。
\stopbuffer


\startbuffer[2442]
抢丧踵魂:抢着奔丧,追赶亡灵。骂人话。斥人行动慌张,急不可待。
\stopbuffer


\startbuffer[2443]
火焰五光佛:即光焰王佛,阿弥陀佛的别称。因其佛光照耀最第一,故称。
\stopbuffer


\startbuffer[2444]
花扑扑:形容隆重铺张,繁华热闹。
\stopbuffer


\startbuffer[2445]
失状:被偷盗、抢劫的财物清单或状纸文书。
\stopbuffer


\startbuffer[2446]
雪案:映雪读书的几案,泛指书桌。晋代孙康家贫,无钱买灯油,因此在冬天映雪读书。
\stopbuffer


\startbuffer[2447]
龚黄:汉代有名的循吏龚遂与黄霸的并称。
\stopbuffer


\startbuffer[2448]
黄堂:古代太守衙中的正堂。
\stopbuffer


\startbuffer[2449]
卓鲁:汉循吏卓茂、鲁恭的并称。
\stopbuffer


\startbuffer[2450]
放告牌:旧时官府每月定期坐衙受理案件时挂出的通告牌。
\stopbuffer


\startbuffer[2451]
马步快手:骑马和步行的衙兵。快手,衙署中专管缉捕的差役。
\stopbuffer


\startbuffer[2452]
只有错拿,没有错放:意思是对作案人错捉了还可以放掉,错放了就难再捉回。
\stopbuffer


\startbuffer[2453]
无干:没有差事、职业。
\stopbuffer


\startbuffer[2454]
民快:义同『快手』,官府专管缉捕的差役。
\stopbuffer


\startbuffer[2455]
躧看:实地探访察看。
\stopbuffer


\startbuffer[2456]
脑箍:古时箍头的刑具。用刑时将绳子箍住头部,加钉木楔箍紧。
\stopbuffer


\startbuffer[2457]
都下:指京都。
\stopbuffer


\startbuffer[2458]
辖床:古代一种刑具,用于固定人体。
\stopbuffer


\startbuffer[2459]
滚肚、敌脑、攀胸:皆为辖床的部件。分别用于束缚腹部、头部和胸部。
\stopbuffer


\startbuffer[2460]
禁子:称监狱中看守罪犯的人,狱卒。
\stopbuffer


\startbuffer[2461]
提控:这里指问案。
\stopbuffer


\startbuffer[2462]
劈着:决断,裁判。
\stopbuffer


\startbuffer[2463]
嚛(hù):这里形容哭泣的声音。
\stopbuffer


\startbuffer[2464]
枉口诳舌:肆意胡言,造谣生事。
\stopbuffer


\startbuffer[2465]
合门:全家。
\stopbuffer


\startbuffer[2466]
解状:指撤销诉讼的状子。
\stopbuffer


\startbuffer[2467]
存殁(mò):生死。
\stopbuffer


\startbuffer[2468]
点子马:有斑点的马。
\stopbuffer


\startbuffer[2469]
交床:胡床的别称,一种有靠背、能折叠的坐具。
\stopbuffer


\startbuffer[2470]
伯考:对已故伯父的称呼。
\stopbuffer


\startbuffer[2471]
神子:谓祖先遗像。
\stopbuffer


\startbuffer[2472]
投文牌:官府用以晓示百姓使之前来投递诉状的牌子。
\stopbuffer


\startbuffer[2473]
堂口:公堂的台阶前。
\stopbuffer


\startbuffer[2474]
拷较:犹拷问。
\stopbuffer


\startbuffer[2475]
卦数:六十四卦的数目。指六十四岁。
\stopbuffer


\startbuffer[2476]
不染床席:指并非卧病而亡。
\stopbuffer


\startbuffer[2477]
一纪:十二年。古称岁星(木星)绕地球一周为一纪。
\stopbuffer


\startbuffer[2478]
真堂:影堂。奉祀祖先名人遗像之所。
\stopbuffer


\startbuffer[2479]
旃(zhān)檀:檀香。这里指唐僧,因其取经功成后被封为『旃檀功德佛』。
\stopbuffer


\startbuffer[2480]
法门:佛教语。指修行者入道的门径。亦泛指佛门。
\stopbuffer


\startbuffer[2481]
此句李卓吾本批:『禅玄原是一家。』从道教观宇连接佛教法门,或是佛道合一的体现。
\stopbuffer


\startbuffer[2482]
望山走倒马:望着山走会累倒马。比喻看着很近,实际很远。
\stopbuffer


\startbuffer[2483]
无底的舟儿:佛教禅宗常以『无底船』比喻使人解脱而不可表述的佛法。
\stopbuffer


\startbuffer[2484]
宝幢光王佛:佛教密宗的佛之一。其身在东,赤白如日初升,故称『光』。大发菩提心,作为统帅万行、降妖伏魔的标识,如率军出征的将帅有幢旗,故称『宝幢』。
\stopbuffer


\startbuffer[2485]
渡群生:佛教常以船渡人比喻普度众生。
\stopbuffer


\startbuffer[2486]
造字造字(hàn táng):称佛门渡人之船。
\stopbuffer


\startbuffer[2487]
鸳鸯:鸳鸯瓦。成对的瓦。
\stopbuffer


\startbuffer[2488]
优钵花:梵语音译。即青莲,产于天竺,其花清净无染,常用以指称和佛教有关的事物。
\stopbuffer


\startbuffer[2489]
打供:供养。
\stopbuffer


\startbuffer[2490]
飞锡:佛教语。谓僧人等执锡杖飞空。借指僧人游方。
\stopbuffer


\startbuffer[2491]
三光:指日、月、星。
\stopbuffer


\startbuffer[2492]
人事:指赠送的礼品。
\stopbuffer


\startbuffer[2493]
叫噪:喧闹,喧叫。
\stopbuffer


\startbuffer[2494]
掯财:敲诈、勒索钱财。
\stopbuffer


\startbuffer[2495]
须索:索取,勒索,也作『需索』。
\stopbuffer


\startbuffer[2496]
龟鉴:比喻可供人对照学习的榜样或引以为戒的教训。鉴,镜子。龟可卜吉凶,镜能别美丑,故称。
\stopbuffer


\startbuffer[2497]
不二门户:即不二法门。佛教语。谓平等而无差异的至道。
\stopbuffer


\startbuffer[2498]
归着:着落,归宿。
\stopbuffer


\startbuffer[2499]
盘弄:徘徊,逗留。
\stopbuffer


\startbuffer[2500]
略见他意:略表心意。见意,表示心意。
\stopbuffer


\startbuffer[2501]
久淹:长久滞留。
\stopbuffer


\startbuffer[2502]
驮垛:捆扎成垛供驮运的货物或行李。
\stopbuffer


\startbuffer[2503]
礼貌:这里指礼仪。
\stopbuffer


\startbuffer[2504]
唐虞:唐尧与虞舜的并称。亦指尧与舜的时代,古人以为太平盛世。
\stopbuffer


\startbuffer[2505]
二仪:指天地。
\stopbuffer


\startbuffer[2506]
含生:一切有生命者,众生。多指人类。
\stopbuffer


\startbuffer[2507]
化物:化育万物。
\stopbuffer


\startbuffer[2508]
罕穷:很少能查究识破。
\stopbuffer


\startbuffer[2509]
挹挹(yì):谦逊貌。
\stopbuffer


\startbuffer[2510]
腾汉庭而皎梦:传说汉明帝曾在梦中见佛,故遣使向西域求之,后有白马驮经而来。是佛教正式传入中国之始。
\stopbuffer


\startbuffer[2511]
不镜三千之光:不用镜子而到处都是佛的光辉。佛教称佛的金光可照三千世界。
\stopbuffer


\startbuffer[2512]
四八之相:佛教语。谓释迦牟尼佛三十二个显著的容貌特征。
\stopbuffer


\startbuffer[2513]
三空:佛教语。指我空、法空、我法俱空。也称空、无相、无愿三解脱为三空。
\stopbuffer


\startbuffer[2514]
四忍:佛教语。指得无生忍、得无灭法忍、得因缘忍、得无住忍。
\stopbuffer


\startbuffer[2515]
讵:岂,怎。
\stopbuffer


\startbuffer[2516]
无累:无所挂碍。
\stopbuffer


\startbuffer[2517]
六尘:佛教语。即色、声、香、味、触、法。与『六根』相接,能染污净心,导致烦恼。
\stopbuffer


\startbuffer[2518]
翘心:仰慕,悬想。
\stopbuffer


\startbuffer[2519]
八水:印度的八条河流。
\stopbuffer


\startbuffer[2520]
鹿苑:即鹿野苑。在中天竺波罗奈国。释迦成道后,始来此说四谛之法,度五比丘,故名仙人论处。
\stopbuffer


\startbuffer[2521]
赜(zé):幽深奥妙。
\stopbuffer


\startbuffer[2522]
火宅:佛教语。佛教认为生、老、病、死如火之燃烧不息。故用以比喻充满众苦的尘世。
\stopbuffer


\startbuffer[2523]
金水:佛教语。金刚界把智慧比喻成水,故名。
\stopbuffer


\startbuffer[2524]
矧(shěn):况且。
\stopbuffer


\startbuffer[2525]
遐敷:远布,长布。敷,传播,散布。
\stopbuffer


\startbuffer[2526]
恧(nǜ):惭愧。
\stopbuffer


\startbuffer[2527]
此处原注『此太宗御制之文,缀于《心经》之首』。
\stopbuffer


\startbuffer[2528]
八部:佛教分诸天鬼神及龙(人眼看不到的众生)为八部。一天,二龙,三夜叉,四乾闼婆,五阿修罗,六迦楼罗,七紧那罗,八摩睺罗伽。因以天、龙居首,又称『天龙八部』。
\stopbuffer
