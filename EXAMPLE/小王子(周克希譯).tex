\usemodule[memos][paperdesign=kindle,hdrstyle=fctext]
\definelayer[lay:cover][x=0mm, y=0mm,width=\paperwidth, height=\paperheight]
\setlayer[lay:cover][hoffset=0cm, voffset=0cm]{\externalfigure[imgs/小王子(周克希譯)/cover.jpeg][height=\paperheight]}
 \startsetups setups:cover
     \setupbackgrounds[page][background=lay:cover]
 \stopsetups
\setupmakeup[page][before=\setups{setups:cover},]
\startmakeup[page]
\stopmakeup

\definelayer[lay:00001][x=0mm, y=0mm,width=\paperwidth, height=\paperheight]
\setlayer[lay:00001][hoffset=0cm, voffset=0cm]{\externalfigure[imgs/小王子(周克希譯)/00001.jpeg][height=\paperheight]}
 \startsetups setups:00001
     \setupbackgrounds[page][background=lay:00001]
 \stopsetups
\setupmakeup[page][before=\setups{setups:00001},]
\startmakeup[page]
\stopmakeup

\starttitle[title={献给莱翁·维尔特},]

请孩子们原谅我把这本书献给了一个大人。我有一个很认真的理由:这个大人是我在世界上最好的朋友。我还有另外一个理由:这个大人什么都能懂,即使是给孩子看的书他也懂。我的第三个理由是:这个大人生活在法国,正在挨饿受冻。他很需要得到安慰。倘若所有这些理由加在一起还不够,那我愿意把这本书献给还是孩子时的这个大人。所有的大人起先都是孩子(可是他们中间不大有人记得这一点)。因此我把题献改为:

献给还是小男孩的 莱翁·维尔特

\stoptitle

\starttitle[title={1},]

{\startalignment[center]
 \placefigeasy[][imgs/小王子(周克希譯)/00002.jpeg][maxwidth=\textwidth,maxheight=\textheight,location={middle,none}]{}
 \stopalignment}

我六岁那年,在一本描写原始森林的名叫《真实的故事》的书上,看见过一幅精彩的插图,画的是一条蟒蛇在吞吃一头猛兽。我现在把它照样画在上面。

书中写道:“蟒蛇把猎物囫囵吞下,嚼都不嚼。然后它就无法动弹,躺上六个月来消化它们。”

当时,我对丛林里的奇妙景象想得很多,于是我也用彩色铅笔画

了我的第一幅画:我的作品1号。它就像这样:

{\startalignment[center]
 \placefigeasy[][imgs/小王子(周克希譯)/00003.jpeg][maxwidth=\textwidth,maxheight=\textheight,location={middle,none}]{}
 \stopalignment}

我把这幅杰作给大人看,问他们我的图画吓不吓人。

他们回答说:“一顶帽子怎么会吓人呢?”

我画的不是一顶帽子。我画的是一条蟒蛇在消化大象。于是我把蟒蛇肚子的内部画出来,好让这些大人看得明白。他们老是要人给他们解释。我的作品2号是这样的:

{\startalignment[center]
 \placefigeasy[][imgs/小王子(周克希譯)/00004.jpeg][maxwidth=\textwidth,maxheight=\textheight,location={middle,none}]{}
 \stopalignment}

那些大人劝我别再画蟒蛇,甭管它是剖开的,还是没剖开的,全都丢开。他们说,我还是把心思放在地理、历史、算术和语法上好。就这样,我才六岁,就放弃了辉煌的画家生涯。作品1号和作品2号都没成功,我泄了气。那些大人自个儿什么也弄不懂,老要孩子们一遍一遍给他们解释,真烦人。

我只好另外选择一个职业,学会了开飞机。世界各地我差不多都飞过。的确,地理学对我非常有用。我一眼就能认出哪是中国,哪是亚利桑那。要是夜里迷了路,这很有用。

就这样,我这一生中,跟好多严肃的人打过好多交道。我在那些大人中间生活过很长时间。我仔细地观察过他们。观察下来印象并没好多少。

要是碰上一个人,看上去头脑稍许清楚些,我就拿出一直保存着的作品1号,让他试试看。我想知道,他是不是真的能看懂。可是人家总是回答我:“这是一顶帽子。”这时候,我就不跟他说什么蟒蛇啊,原始森林啊,星星啊,都不说了。我就说些他能懂的事情。我跟他说桥牌,高尔夫,政治,还有领带。于是大人觉得很高兴,认识了这么个通情达理的人。

\reference[part0004.html]{}

\stoptitle

\starttitle[title={2},reference={part0004.html_a005}]

我孤独地生活着,没有一个真正谈得来的人,直到六年前,有一次飞机出了故障,降落在撒哈拉大沙漠。发动机里有样什么东西碎掉了。因为我身边既没有机械师,也没有乘客,我就打算单枪匹马来完成一项困难的修复工作。这在我是个生死攸关的问题。我带的水只够喝一星期了。

第一天晚上,我睡在这片远离人烟的大沙漠上,比靠一块船板在大海中漂流的遇难者还孤独。所以,当天蒙蒙亮,有个奇怪的声音轻轻把我喊醒的时候,你们可以想象我有多么惊讶。这个声音说:

“对不起\ldots{}\ldots{}请给我画只绵羊!”

“嗯!”

{\startalignment[center]
 \placefigeasy[][imgs/小王子(周克希譯)/00005.jpeg][maxwidth=\textwidth,maxheight=\textheight,location={middle,none}]{}
 \stopalignment}

后来我给他画了一幅非常出色的肖像。

“请给我画只绵羊\ldots{}\ldots{}”

我像遭了雷击似的,猛地一下子跳了起来。我使劲地揉了揉眼睛,仔细地看了看。只见一个从没见过的小人儿,正一本正经地看着我呢。后来我给他画了一幅非常出色的肖像,就是旁边的这幅。不过我的画,当然远远不及本人可爱。这不是我的错。我的画家生涯在六岁那年就让大人给断送了,除了画剖开和不剖开的蟒蛇,后来再没画过什么。

我吃惊地瞪大眼睛瞧着他。你们别忘记,这儿离有人住的地方好远好远呢。可是这个小人儿,看上去并不像迷了路,也不像累得要命、饿得要命、渴得要命或怕得要命。他一点不像在远离人类居住地的沙漠里迷路的孩子。等我总算说得出话时,我对他说:

“可是\ldots{}\ldots{}你在这儿干吗?”

他轻声轻气地又说了一遍,好像那是件很要紧的事情:

“对不起\ldots{}\ldots{}请给我画一只绵羊\ldots{}\ldots{}”

受到神秘事物强烈冲击时,一个人是不敢不听从的。尽管在我看来,离一切有人居住的地方远而又远,又处于死亡的威胁之下,在这儿想到画画真是匪夷所思,可我还是从口袋里掏出一张纸、一支钢笔。但我想起我只学了地理、历史、算术和语法,所以我就(有点没好气地)对那小人儿说,我不会画画。他回答说:

“没关系。请给我画一只绵羊。”

我因为从没画过绵羊,就在我只会画的两张图画里挑一张给他画了:没剖开的蟒蛇图。可我听到小人儿下面说的话,简直惊呆了:

“不对!不对!我不要在蟒蛇肚子里的大象。蟒蛇很危险,大象呢,太占地方。在我那儿,什么都是小小的。我要的是一只绵羊。请给我画一只绵羊。”

{\startalignment[center]
 \placefigeasy[][imgs/小王子(周克希譯)/00006.jpeg][maxwidth=\textwidth,maxheight=\textheight,location={middle,none}]{}
 \stopalignment}

我只得画了起来。

他专心地看了一会儿,然后说:

“不对!这只羊已经病得不轻了。另外画一只吧。”

我画了下面的这只。

我的朋友温和地笑了,口气宽容地说:“你看看\ldots{}\ldots{}这只不是绵羊,是山羊。头上长着角\ldots{}\ldots{}”

{\startalignment[center]
 \placefigeasy[][imgs/小王子(周克希譯)/00007.jpeg][maxwidth=\textwidth,maxheight=\textheight,location={middle,none}]{}
 \stopalignment}

{\startalignment[center]
 \placefigeasy[][imgs/小王子(周克希譯)/00008.jpeg][maxwidth=\textwidth,maxheight=\textheight,location={middle,none}]{}
 \stopalignment}

于是我又画了一张。

但这一张也跟前几张一样,没能通过:

“这只太老了。我要一只可以活得很久的绵羊。”

我已经没有耐心了,因为我急于要去把发动机拆下来,所以我就胡乱画了一张。

我随口说道:

“这个呢,是个箱子。你要的绵羊就在里面。”

但是令我吃惊的是,这个小评判的脸上顿时变得容光焕发了:

“我要的就是这个!你说,这只绵羊会要很多草吗?”

“问这干吗?”

“因为我那儿样样都很小\ldots{}\ldots{}”

“肯定够了。我给你的是只很小的绵羊。”

{\startalignment[center]
 \placefigeasy[][imgs/小王子(周克希譯)/00009.jpeg][maxwidth=\textwidth,maxheight=\textheight,location={middle,none}]{}
 \stopalignment}

他低下头去看那幅画:

“不算太小\ldots{}\ldots{}瞧!它睡着了\ldots{}\ldots{}”

就这样,我认识了小王子。

\reference[part0005.html]{}

\stoptitle

\starttitle[title={3},reference={part0005.html_a006}]

很久以后,我才弄明白他是从哪儿来的。

这个小王子,对我提了好多问题,而对我的问题总像没听见似的。我是从他偶尔漏出来的那些话里,一点一点知道这一切的。比如,他第一次瞧见我的飞机时(我没画我的飞机,对我来说,这样的画实在太复杂了),就问我:

“这是什么东西?”

“这不是什么东西,它会飞。这是一架飞机,是我的飞机。”

我自豪地讲给他听,我在天上飞。他听了就大声说:

“怎么!你是天上掉下来的?”

“是的。”我谦虚地说。

“喔!真有趣\ldots{}\ldots{}”

小王子发出一阵清脆的笑声,这下可把我惹恼了。我不喜欢别人拿我的不幸逗趣儿。接着他又说:

“这么说,你也是从天上来的!你从哪个星球来?”

我脑子里闪过一个念头,他的降临之谜好像有了线索,我突如其来地发问:

“那你是从别的星球来的啰?”

可是他没有回答。他看着我的飞机,轻轻地点了点头:

“是啊,就靠它,你来的地方不会太远\ldots{}\ldots{}”

说着,他出神地遐想了很久。而后,从袋里拿出我画的绵羊,全神贯注地凝望着这宝贝。

你想想看,这个跟“别的星球”有关,说了一半打住的话头,会让我多么惊讶啊。我竭力想多知道一些:

{\startalignment[center]
 \placefigeasy[][imgs/小王子(周克希譯)/00010.jpeg][maxwidth=\textwidth,maxheight=\textheight,location={middle,none}]{}
 \stopalignment}

“你从哪儿来,我的小家伙?‘我那儿'是哪儿?你要把我画的绵羊带到哪儿去?”

他若有所思地沉默了一会儿,然后开口对我说:

“你给了我这个箱子,这就好了,晚上可以给它当屋子。”

“当然。要是你乖,我还会给你一根绳子,白天可以把它拴住。木桩也有。”

这个提议好像使小王子很不以为然:“拴住?真是怪念头!”

“可要是你不把它拴住,它就会到处跑,还会跑丢了\ldots{}\ldots{}”

我的朋友又格格地笑了起来:

“你叫它往哪儿跑呀?”

“到处跑。笔直往前\ldots{}\ldots{}”

这时,小王子一本正经地说:

“那也没关系,我那儿就一丁点儿大!”

然后,他又说了一句,语气中仿佛有点儿忧郁:

“就是笔直往前跑,也跑不了多远\ldots{}\ldots{}”

{\startalignment[center]
 \placefigeasy[][imgs/小王子(周克希譯)/00011.jpeg][maxwidth=\textwidth,maxheight=\textheight,location={middle,none}]{}
 \stopalignment}

小王子在B612小行星上

\reference[part0006.html]{}

\stoptitle

\starttitle[title={4},reference={part0006.html_a007}]

我由此知道了另一件很重要的事情:他居住的星球比一座房子大不了多少!

这并没让我感到很吃惊。我知道,除了像地球、木星、火星、金星这些取了名字的大星球,还有成千上万的星球,它们有时候非常非常小,用望远镜都不大看得见。天文学家找到其中的一个星球,给它编一个号码就算名字了。比如说,他把它叫作“3251号小行星”。

我有很可靠的理由,足以相信小王子原先住的那个星球,就是B612号小行星。这颗小行星只在1909年被人用望远镜望见过一次,那人是一个土耳其天文学家。

{\startalignment[center]
 \placefigeasy[][imgs/小王子(周克希譯)/00012.jpeg][maxwidth=\textwidth,maxheight=\textheight,location={middle,none}]{}
 \stopalignment}

{\startalignment[center]
 \placefigeasy[][imgs/小王子(周克希譯)/00013.jpeg][maxwidth=\textwidth,maxheight=\textheight,location={middle,none}]{}
 \stopalignment}

当时,他在一次国际天文学大会上作了长篇论证。可是就为了他的服装的缘故,谁也不信他的话。大人哪,就是这样。

幸好,有一个土耳其独裁者下令,全国百姓都要穿欧洲的服装,违令者处死,这一下B612号小行星的名声总算保全了。那个天文学家在1920年重新作报告,穿着一套非常体面的西装。这一回所有的人都同意了他的观点。

{\startalignment[center]
 \placefigeasy[][imgs/小王子(周克希譯)/00014.jpeg][maxwidth=\textwidth,maxheight=\textheight,location={middle,none}]{}
 \stopalignment}

我之所以要跟你们一五一十地介绍B612号小行星,还把它的编号也讲得明明白白,完全是为了大人。那些大人就喜欢数字。你跟他们讲起一个新朋友,他们总爱问些无关紧要的问题。他们不会问你;“他说话的声音是怎样的?他喜欢玩哪些游戏?他是不是收集蝴蝶标本?”他们问的是:“他几岁?有几个兄弟?他有多重?他父亲挣多少钱?”这样问过以后,他们就以为了解他了。你要是对大人说:“我看见一幢漂亮的房子,红砖墙,窗前种着天竺葵,屋顶上停着鸽子\ldots{}\ldots{}”他们想象不出这幢房子是怎样的。你得这么跟他们说:“我看见一幢十万法郎的房子。”他们马上会大声嚷嚷:“多漂亮的房子!”

所以,如果你对他们说:“小王子是存在的,证据就是他那么可爱,他格格地笑,他还想要一只绵羊。一个人想要有只绵羊,这就是他存在的证据嘛”,他们会耸耸肩膀,只当你还是个孩子!可要是你对他们说:“他来自B612号小行星”,他们就会深信不疑,不再问这问那地烦你了。他们就是这样。不必怪他们。孩子应该对大人多多原谅才是。

不过,当然,我们懂得生活,我们才不把数字放在眼里呢!我真愿意像讲童话那样来开始讲这个故事。我真想这样说:

“从前呀,有一个小王子,住在一个跟他身体差不多大的星球上,他想有个朋友\ldots{}\ldots{}”对那些懂得生活的人来说,这样听上去会真实得多。

我不想人家轻率地来读我这本书。我讲述这段往事时,心情是很难过的。我的朋友带着他的绵羊已经离去六年了。我之所以在这儿细细地描述他,就是为了不要忘记他。忘记朋友是件令人伤心的事情。并不是人人都有过一个朋友的。再说,我早晚也会变得像那些只关心数字的大人一样的。也正是为了这个缘故,我买了一盒颜料和一些铅笔。到了我这年纪再重握画笔,是挺费劲的,况且当初我只画过剖开和没剖开的蟒蛇,还是六岁那年!当然,我一定要尽力把它们画得像一些。但做不做得到,我可说不准。有时这一张还行,那一张就不大像了。比如说,身材我就有点记不准确了。这一张里小王子画得太高了。那一张呢太矮了。衣服的颜色也挺让我犯难。我只好信手拿起色笔这儿试一下,那儿试一下。到头来,有些最要紧的细部,说不定都弄错了。不过这一切,大家都得原谅我才是。我的朋友从来不跟我解释什么。他大概以为我是跟他一样的。可是,很遗憾,我已经瞧不见箱子里面的绵羊了。我也许已经有点像那些大人了。我一定是老了。

\reference[part0007.html]{}

\stoptitle

\starttitle[title={5},reference={part0007.html_a008}]

每天我都会知道一些情况,或者是关于他的星球,或者是关于他怎么离开那儿、怎么来到这儿。这些情况,都是一点一点,碰巧知道的。比如说,在第三天,我知道了猴面包树的悲剧。

这一回,起因又是那只绵羊,因为小王子突然向我发问,好像忧心忡忡似的:

“绵羊当真吃灌木吗?”

“对。当真。”

“啊!我真高兴。”

我不明白,绵羊吃灌木,为什么会这么重要。小王子接着又说:

“这么说,它们也吃猴面包树喽?”

我告诉小王子,猴面包树不是灌木,而是像教堂那么高的大树,他就是领一群大象来,也吃不完一棵猴面包树呢。

领一群大象来的想法,惹得小王子笑了起来:

“那得让它们叠罗汉了\ldots{}\ldots{}”

{\startalignment[center]
 \placefigeasy[][imgs/小王子(周克希譯)/00015.jpeg][maxwidth=\textwidth,maxheight=\textheight,location={middle,none}]{}
 \stopalignment}

{\startalignment[center]
 \placefigeasy[][imgs/小王子(周克希譯)/00016.jpeg][maxwidth=\textwidth,maxheight=\textheight,location={middle,none}]{}
 \stopalignment}

不过他很聪明,接着又说:

“猴面包树在长高以前,起初也是小小的。”

“一点不错。可你为什么想让绵羊去吃小猴面包树呢?”

他回答说:“咦!这还不明白吗!”就像这是件不言而喻的事情。可是我自己要弄懂这个问题,还着实得动一番脑筋哩。

原来,在小王子的星球上,就像在别的星球上一样,有好的植物,也有不好的植物。结果呢,好植物有好种子,坏植物有坏种子。而种子是看不见的。它们悄悄地睡在地底下,直到有一天,其中有一颗忽然想起要醒了\ldots{}\ldots{}于是它舒展身子,最先羞答答地朝太阳伸出一枝天真可爱的嫩苗。假如那是萝卜或玫瑰的幼苗,可以让它爱怎么长就怎么长。不过,假如那是一株不好的植物,一认出就得拔掉它。在小王子的星球上有一种可怕的种子\ldots{}\ldots{}就是猴面包树的种子。星球的土壤里有好多猴面包树种子。而猴面包树长得很快,动手稍稍一慢,就甭想再除掉它了。它会占满整个星球,根枝钻来钻去,四处蔓延。要是这颗星球太小,而猴面包树又太多,它们就会把星球撑裂。

“这就得有个严格的约束,”小王子后来告诉我说,“你早晨梳洗好以后,就该仔仔细细地给星球梳洗了。猴面包树小的时候,跟玫瑰幼苗是很像的,那你就得给自己立个规矩,只要分清了哪是玫瑰,哪是猴面包树,就马上把猴面包树拔掉。这个工作很单调,但并不难。”

有一天,他劝我好好画一幅画,好让我那儿的孩子们都知道这回事。“要是他们有一天出门旅行,”他对我说,“说不定会用得着。有时候,你把一件该做的事耽搁一下,也没什么关系。可是,碰到猴面包树,这就要捅大娄子了。我知道有一个星球,上面住着一个懒人。有三株幼苗他没在意\ldots{}\ldots{}”

在小王子的指点下,我画好了那颗星球。我一向不愿意摆出说教的架势。可是对猴面包树的危害,一般人都不了解,要是有人碰巧迷了路停在一颗小行星上,情况就会变得极其严峻。所以这一次,我破例抛开了矜持。我说:“孩子们!当心猴面包树啊!”这幅画我画得格外卖力,就是为了提醒朋友们有这么一种危险存在,他们也像我一样,对在身边潜伏了很久的危险一直毫无觉察。要让大家明白这道理,我多费点劲也是值得的。你们也许会想:“在这本书里,别的画为什么都没有这幅来得奔放有力呢?”回答很简单:我同样努力了,但没能成功。画猴面包树时,我内心非常焦急,情绪就受到了感染。

{\startalignment[center]
 \placefigeasy[][imgs/小王子(周克希譯)/00017.jpeg][maxwidth=\textwidth,maxheight=\textheight,location={middle,none}]{}
 \stopalignment}

猴面包树

\reference[part0008.html]{}

\stoptitle

\starttitle[title={6},reference={part0008.html_a009}]

{\startalignment[center]
 \placefigeasy[][imgs/小王子(周克希譯)/00018.jpeg][maxwidth=\textwidth,maxheight=\textheight,location={middle,none}]{}
 \stopalignment}

哦,小王子!就这样,我一点一点知道了你那段忧郁的生活。过去很长的时间里,你唯一的乐趣就是观赏夕阳沉落的温柔晚景。这个新的细节,我是在第四天早晨知道的。当时你对我说:“我喜欢看日落。我们去看一回日落吧\ldots{}\ldots{}”

“可是得等\ldots{}\ldots{}”

“等什么?”

“等太阳下山呀。”

开始,你显得很惊奇,随后你自己笑了起来。你对我说:

“我还以为在家乡呢!”

可不。大家都知道,美国的中午,在法国正是黄昏。要是能在一分钟内赶到法国,就可以看到日落。可惜法国实在太远了。而在你那小小的星球上,你只要把椅子挪动几步就行了。那样,你就随时可以看到你想看的夕阳余晖\ldots{}\ldots{}

“有一天,我看了四十三次日落!”

过了一会儿,你又说:

“你知道\ldots{}\ldots{}一个人感到非常忧伤的时候,他就喜欢看日落\ldots{}\ldots{}”

“这么说,看四十三次的那天,你感到非常忧伤啰?”

但是小王子没有回答。

\reference[part0009.html]{}

\stoptitle

\starttitle[title={7},reference={part0009.html_a010}]

第五天,还是羊的事情,把小王子生活的秘密向我揭开了。他好像有个问题默默地思索了很久,终于得出了结论,突然没头没脑地问我:“绵羊既然吃灌木,那它也吃花儿啰?”

“它碰到什么吃什么。”

“连有刺的花儿也吃?”

“对。有刺的也吃。”

“那么,刺有什么用呢?”

我不知道该怎么回答。当时我正忙着要从发动机上卸下一颗拧得太紧的螺钉。我发现故障似乎很严重,饮用水也快完了。我担心会发生最坏的情况,心里很着急。

“那么,刺有什么用呢?”

小王子只要提了一个问题,就不依不饶地要得到答案。而那个螺钉正弄得我很恼火,我就随口回答了一句:

“刺呀,什么用都没有,纯粹是花儿想使坏呗。”

“喔!”

但他沉默了一会儿以后,忿忿然地冲着我说:

“我不信你的话!花儿是纤弱的,天真的。它们想尽量保护自己。它们以为有了刺就会显得很厉害\ldots{}\ldots{}”

我没作声。我当时想:“要是这颗螺钉再不松开,我就一锤子敲掉它。”小王子又打断了我的思路:

“可你,你却认为花儿\ldots{}\ldots{}”

“行了!行了!我什么也不认为!我只是随口说说。我正忙着干正事呢!”

他惊愕地望着我。

“正事!”

他看我握着锤子,手指沾满油污,俯身对着一个他觉得非常丑陋的物件。

“你说话就像那些大人!”

这话使我有些难堪。而他毫不留情地接着说:

“你什么都分不清\ldots{}\ldots{}你把什么都搅在一起!”

他真的气极了,一头金发在风中摇曳:

“我到过一个星球,上面住着一个红脸先生。他从没闻过花香。他从没望过星星。他从没爱过一个人。除了算账,他什么事也没做过。他成天像你一样说个没完:‘我有正事要干!我有正事要干!'变得骄气十足。可是这算不得一个人,他是个蘑菇。”

“是个什么?”

“是个蘑菇!”

小王子这会儿气得脸色发白了。

“几百万年以前,花儿就长刺了。可几百万年以前,羊也早就在吃花儿了。刺什么用也没有,那花儿为什么要费那份劲去长刺呢?把这弄明白难道不是正事吗?绵羊和花儿的战争难道不重要吗?这难道不比那个胖子红脸先生的算账更重要,更是正事吗?还有,如果我认识一朵世上独一无二的花儿,除了我的星球,哪儿都找不到这样的花儿,而有天早上,一只小羊甚至都不明白自己在做什么,就一口把花儿吃掉了,这难道不重要吗!”

他的脸红了起来,接着又往下说:

“如果有个人爱上一朵花儿,好几百万好几百万颗星星中间,只有一颗上面长着这朵花儿,那他只要望着许许多多星星,就会感到很幸福。他对自己说:‘我的花儿就在其中的一颗星星上\ldots{}\ldots{}'可要是绵羊吃掉了这朵花儿,这对他来说,就好像满天的星星突然一下子都熄灭了!这难道不重要吗!”

{\startalignment[center]
 \placefigeasy[][imgs/小王子(周克希譯)/00019.jpeg][maxwidth=\textwidth,maxheight=\textheight,location={middle,none}]{}
 \stopalignment}

他说不下去了,突然抽抽噎噎地哭了起来。夜色降临。我放下手中的工具。锤子呀,螺钉呀,口渴呀,死亡呀,我全都丢在了脑后。在一颗星星,在一颗我所在的行星,在这个地球上,有个小王子需要安慰!我把他抱在怀里。我摇着他,对他说:“你爱的那朵花儿不会有危险的\ldots{}\ldots{}我会给你的绵羊画一只嘴罩\ldots{}\ldots{}我会给你的花儿画一个护栏\ldots{}\ldots{}我\ldots{}\ldots{}”我不知道再说什么好了。我觉得自己笨嘴笨舌的。我不知道怎样去接近他,打动他\ldots{}\ldots{}泪水的世界,是多么神秘啊!

\reference[part0010.html]{}

\stoptitle

\starttitle[title={8},reference={part0010.html_a011}]

我很快就对这朵花儿有了更多的了解。在小王子的星球上,过去一直长着些很简单的花儿,这些花儿只有一层花瓣,不占地方,也不妨碍任何人。某个早晨她们会在草丛中绽放,一到晚上又都悄悄凋谢了。有一天,一颗不知从哪儿来的种子发了芽,长出的嫩苗跟别的幼苗都不一样。小王子小心翼翼地观察着这株嫩苗,它说不定是猴面包树的一株幼芽呢。但是这株嫩苗很快就不再长大,做起了开花的准备。小王子眼看着它长出一个很大很大的花蕾,心想花蕾绽放开来一定很奇妙。可是这朵花儿待在绿色的花萼里面,磨磨蹭蹭地打扮个没完。她精心挑选着自己的颜色,慢吞吞地穿上衣裙,一片一片地理顺花瓣。她不愿像虞美人\reference[part0010.html_w1]{}\goto{\high{{[}1{]}}}[part0010.htmlux5cux23m1]那样一亮相就是满脸皱纹。她要让自己美艳照人地来到世间。噢!对。她很爱俏!她那神秘的装扮,就这样日复一日地延续着。然后,有一天早晨,就在太阳升起的那一刻,她绽放了。

她精心打扮了那么久,这会儿却打着哈欠说:

“啊!我刚睡醒\ldots{}\ldots{}真对不起\ldots{}\ldots{}头发还是乱蓬蓬的\ldots{}\ldots{}”

这时,小王子的爱慕之情油然而生:

“您真美!”

{\startalignment[center]
 \placefigeasy[][imgs/小王子(周克希譯)/00020.jpeg][maxwidth=\textwidth,maxheight=\textheight,location={middle,none}]{}
 \stopalignment}

“可不是吗,”花儿柔声答道,“我是跟太阳同时出生的嘛\ldots{}\ldots{}”

小王子感觉到了她不太谦虚,不过她实在太楚楚动人了!

{\startalignment[center]
 \placefigeasy[][imgs/小王子(周克希譯)/00021.jpeg][maxwidth=\textwidth,maxheight=\textheight,location={middle,none}]{}
 \stopalignment}

“我想,现在该是用早餐的时间了,”她随即又说,“麻烦您也给我\ldots{}\ldots{}”

小王子很不好意思,于是就打来一壶清水,给这朵花儿浇水。

就这样,她带着点多疑的虚荣心,很快就把他折磨得够呛。比如说,有一天说起她的四根刺,她对小王子说:

“那些老虎,让它们张着爪子来好了!”

“我的星球上没有老虎,”小王子顶了她一句,“再说,老虎也不吃草呀。”

“我不是草。”花儿柔声答道。

{\startalignment[center]
 \placefigeasy[][imgs/小王子(周克希譯)/00022.jpeg][maxwidth=\textwidth,maxheight=\textheight,location={middle,none}]{}
 \stopalignment}

“对不起\ldots{}\ldots{}”

“我不怕老虎,可我怕风。您没有风障吗?”

“怕风\ldots{}\ldots{}一棵植物到了这份上,那可惨了,”小王子轻声说,“花儿可真难伺候\ldots{}\ldots{}”

“晚上您要把我罩起来。您这儿很冷。又没安顿好。我来的那地方\ldots{}\ldots{}”

可是她没说下去。她来的时候是颗种子。她不可能知道别的世界是怎么样的。让人发现她说的谎这么不高明,她又羞又恼,就咳了两三声嗽,想让小王子觉得理亏:

“风障呢?”

“我正要去拿,可您跟我搭话了!”

{\startalignment[center]
 \placefigeasy[][imgs/小王子(周克希譯)/00023.jpeg][maxwidth=\textwidth,maxheight=\textheight,location={middle,none}]{}
 \stopalignment}

于是她咳得更重了些,不管怎么说,她非让他感到内疚不可。

就这样,小王子尽管真心真意喜欢这朵花儿,可还是很快就对她起了疑心。他对那些无关紧要的话太当真了,结果自己很苦恼。

“我本来不该去听她说什么的,”有一天他对我说了心里话,“花儿说的话,是听不得的。花儿是让人看,让人闻的。这朵花儿让我的星球芳香四溢,我却不会享受这快乐。老虎爪子那些话,惹得我那么生气,其实我该同情她才是\ldots{}\ldots{}”

{\startalignment[center]
 \placefigeasy[][imgs/小王子(周克希譯)/00024.jpeg][maxwidth=\textwidth,maxheight=\textheight,location={middle,none}]{}
 \stopalignment}

他还对我说:

“我当时什么也不懂!看她这个人,应该看她做什么,而不是听她说什么。她给了我芳香,给了我光彩。我真不该逃走!我本该猜到她那小小花招背后的一片柔情。花儿总是这么表里不一!可惜当时我太年轻,还不懂得怎么去爱她。”

\thinrule

\reference[part0010.html_m1]{}\goto{{[}1{]}}[part0010.htmlux5cux23w1]
一种夏季开花的植物,花未开前即下垂。

\reference[part0011.html]{}

\stoptitle

\starttitle[title={9},reference={part0011.html_a012}]

我想他是趁一群野鸟迁徙的机会出走的。动身的那天早晨,他把星球收拾得井井有条。他仔细地疏通了活火山。星球上有两座活火山,热早餐很方便。还有一座死火山。不过,正像他所说的:“谁说得准呢!”所以这座死火山也照样要疏通。火山疏通过了,就会缓缓地、均匀地燃烧,不会喷发。火山喷发跟烟囱冒火是一样的。当然,在地球上,我们实在太小了,没法去疏通火山。它们造成那么多麻烦,就是由于这个缘故。

小王子还拔掉了刚长出来的几株猴面包树幼苗。他心情有点忧郁,心想这一走就再也回不来了。所有这些习惯的活儿,这天早上都显得格外亲切。而当他最后一次给花儿浇水,准备给她盖上罩子的时候,他只觉得想哭。

“再见啦。”他对花儿说。

可是她没有回答。

“再见啦。”他又说了一遍。

花儿咳嗽起来。但不是由于感冒。

“我以前太傻了,”她终于开口了。“请你原谅我。但愿你能幸福。”

他感到吃惊的是,居然没有一声责备。他举着罩子,茫然不知所措地站在那儿。他不懂这般恬淡的柔情。

“是的,我爱你,”花儿对他说,“但由于我的过错,你一点儿也没领会。这没什么要紧。不过你也和我一样傻。但愿你能幸福\ldots{}\ldots{}把这罩子放在一边吧,我用不着它了。”

“可是风\ldots{}\ldots{}”

“我并不是那么容易感冒的\ldots{}\ldots{}夜晚的新鲜空气对我有好处。我是一朵花儿。”

{\startalignment[center]
 \placefigeasy[][imgs/小王子(周克希譯)/00025.jpeg][maxwidth=\textwidth,maxheight=\textheight,location={middle,none}]{}
 \stopalignment}

他仔细地疏通了活火山

“可是那些虫子和野兽\ldots{}\ldots{}”

“我既然想认识蝴蝶,就应该受得了两三条毛虫。我觉得这样挺好。要不然有谁来看我呢?你,你到时候已经走得远远的了。至于野兽,我根本不怕。我也有爪子。”

说着,她天真地让他看那四根刺。随后她又说:

“别磨磨蹭蹭的,让人心烦。你已经决定要走了。那就走吧。”

因为她不愿意让他看见自己流泪。她是一朵如此骄傲的花儿\ldots{}\ldots{}

\reference[part0012.html]{}

\stoptitle

\starttitle[title={10},reference={part0012.html_a013}]

这颗星球附近,还有325号、326号、327号、328号、329号和330号小行星。于是他开始拜访这些星球,好给自己找点事干,也好增长些见识。

第一颗小行星上住着一个国王。这个国王身穿紫红镶边白鼬皮长袍,端坐在一张简朴而又气派庄严的王座上。

“哈!来了一个臣民。”国王看见小王子,大声叫了起来。

可小王子觉得纳闷:

“他以前从没见过我,怎么会认识我呢?”

他不知道,对国王来说,世界是非常简单的。所有的人都是臣民。

“你走近点,让我好好看看你,”国王说,他觉得非常骄傲,因为他终于成了某个人的国王。

小王子朝四下里看看,想找个地方坐下来,可是整个星球都被那袭华丽的白鼬皮长袍占满了。所以他只好站着,不过,由于他累了,就打了个哈欠。

“在国王面前打哈欠,有违宫廷礼仪,”国王对他说,“我禁止你打哈欠。”

“我没忍住,”小王子歉疚地说,“我走了好长的路,一直没睡觉\ldots{}\ldots{}”

“那么,”国王对他说,“我命令你打哈欠。我有好几年没见人打哈欠了。我觉得打哈欠挺好玩。来!再打个哈欠。这是命令。”

“我给吓着了\ldots{}\ldots{}打不出\ldots{}\ldots{}”小王子涨红着脸说。

“呣!呣!”国王回答说,“那么我\ldots{}\ldots{}我命令你一会儿打哈欠,一会儿\ldots{}\ldots{}”

他嘟嘟哝哝的,看上去不大高兴。

国王其实是要别人尊重他的权威。他不能容忍别人不服从命令。他是个专制的君主。不过,因为他很善良,他下的命令都是通情达理的。

“要是我命令,”这番话他说得流畅极了,“要是我命令一个将军变成一只海鸟,那个将军不服从,这就不是那个将军的错,这是我的错。”

“我可以坐下吗?”小王子怯生生地问。

“我命令你坐下。”国王回答他说,庄重地挪了挪白鼬皮长袍的下摆。

可是小王子感到很奇怪。这么小的星球,国王能统治什么呢?

“陛下\ldots{}\ldots{}”他说,“请允许我向您提个\ldots{}\ldots{}”

“我命令你向我提问题。”国王赶紧抢着说。

“陛下\ldots{}\ldots{}您统治什么呢?”

“一切。”国王的回答简单明了。

“一切?”

国王小心翼翼地做了个手势,指了指他的行星、其他的行星和所有的星星。

“全归您统治?”小王子问。

“全归我统治\ldots{}\ldots{}”国王回答说。

因为他不仅是一国的专制君主,还是宇宙的君主。

“那些星星都服从您?”

“当然,”国王回答说,“我一下命令,它们马上就服从。我不能容忍纪律涣散。”

这样的权力使小王子惊叹不已。他如果拥有这样的权力,那么一天就不是看四十三次,而是七十二次,一百次,甚至两百次日落,连椅子都不用挪一挪!想起被他遗弃的小星球,他有点难过,所以就壮着胆子向国王提出一个请求:“我想看一次日落\ldots{}\ldots{}请您为我\ldots{}\ldots{}命令太阳下山\ldots{}\ldots{}”

“要是我命令一个将军像蝴蝶一样从一朵花儿飞到另一朵花儿,或者让他写一部悲剧,或者让他变成一只海鸟,而这个将军拒不执行命令,那是谁,是他还是我的错呢?”

“那是您的错。”小王子肯定地说。

“正是如此。得让每个人去做他能做到的事情。”国王接着说,“权威首先得建立在合理的基础上。如果你命令你的老百姓都去投海,他们就会造反。我之所以有权让人服从,就是因为我的命令都是合情合理的。”

{\startalignment[center]
 \placefigeasy[][imgs/小王子(周克希譯)/00026.jpeg][maxwidth=\textwidth,maxheight=\textheight,location={middle,none}]{}
 \stopalignment}

“那么我想看的日落呢?”小王子想起了这件事,他对自己提过的问题是不会忘记的。

“你会看到日落的。我会要它下山的。不过按照我的统治原则,要等到条件成熟的时候。”

“要等到什么时候呢?”小王子问。

“呣!呣!”国王先翻看一本厚厚的历书,然后回答说,“呣!呣!要等到,大概\ldots{}\ldots{}大概\ldots{}\ldots{}要等到今晚七点四十分左右!你会看到它乖乖地服从我的命令的。”

小王子打了个哈欠。看不到日落,让他感到挺遗憾。再说他也已经有点腻烦了:

“我在这儿没什么事好做了,”他对国王说,“我要走了!”

“别走,”国王回答说,他有了一个臣民,正骄傲着呢,“别走,我任命你当大臣!”

“什么大臣?”

“这个\ldots{}\ldots{}司法大臣!”

“可是这儿没有人要审判呀!”

“那可说不定,”国王对他说,“我还没巡视过我的王国。我太老了,我没地方放马车,走路又累得慌。”

“噢!可是我已经看过了,”小王子说着,又朝这颗小行星的另一边瞥了一眼,“那边也没有一个人\ldots{}\ldots{}”

“那你就审判你自己,”国王回答他说,“这是最难的。审判自己要比审判别人难得多。要是你能审判好自己,你就是个真正的智者。”

“可我,”小王子说,“我在哪儿都可以自己审判自己。我不必留在这儿呀。”

“呣!呣!”国王说,“我想哪,在我的星球上有只老耗子,夜里我听见它的声音。你可以审判这只老耗子。你可以不时判它死刑。这样啊,它的生命就取决于你的判决了。不过,这只耗子你得悠着点儿用,每次判决后都得赦免它。因为只有这么一只耗子。”

“可我,”小王子回答说,“我不喜欢判死刑,我想我还得走。”

“不行。”国王说。

去意已决的小王子不想让老国王难过:

“陛下如果想让命令立刻得到服从,那就不妨下一道合情合理的命令。比如说,陛下可以命令我在一分钟内离开此地。我觉得条件已经成熟\ldots{}\ldots{}”

国王一声不吭,小王子起先有点犹豫,而后叹了口气,就起程了。

“我任命你当我的大使。”这时国王赶紧喊道。

他的神态威严极了。

“这些大人真奇怪。”小王子在旅途中自言自语地说。

\reference[part0013.html]{}

\stoptitle

\starttitle[title={11},reference={part0013.html_a014}]

{\startalignment[center]
 \placefigeasy[][imgs/小王子(周克希譯)/00027.jpeg][maxwidth=\textwidth,maxheight=\textheight,location={middle,none}]{}
 \stopalignment}

第二颗行星上住着一个爱虚荣的人。

“哈哈!有个崇拜者来看我了!”这个爱虚荣的人刚看见小王子,大老远就喊了起来。

因为,在爱虚荣的人眼里,别人都是他们的崇拜者。

“您好,”小王子说,“您这顶帽子挺有趣的。”

“这是用来致意的,”爱虚荣的人回答说,“人家向我欢呼时,我就用帽子向他们致意。可惜啊,一直没人经过这儿。”

“是吗?”小王子说,他没明白那人的意思。

“你用一只手去拍另一只手。”于是爱虚荣的人这样教他。

小王子就拍起巴掌来了。爱虚荣的人抬起帽子,谦逊地致意。

“这比访问那个国王好玩多了。”小王子心想。他又拍起巴掌来了。爱虚荣的人就又抬起帽子致意。

这样玩了五分钟,小王子觉得太单调,他都玩累了:

“要想叫这顶帽子掉下来,该怎么做呢?”

可是爱虚荣的人没听见他的话。爱虚荣的人只听得见颂扬的话。

“你真的很崇拜我吗?”他问小王子。

“崇拜是什么意思?”

“崇拜的意思就是,承认我是这个星球上最英俊、最摩登、最富有、最有学问的人。”

“可是这个星球上只有你一个人呀!”

“你得帮我这个忙。你只管崇拜我就是了!”

“我崇拜你,”小王子说着,微微耸了耸肩膀,“可是你要这个干什么呢?”

说着,小王子就走开了。

“这些大人真的很怪哟。”一路上,他这么对自己说了一句。

\reference[part0014.html]{}

\stoptitle

\starttitle[title={12},reference={part0014.html_a015}]

下一颗行星上住着一个酒鬼。这次访问时间很短,却使小王子陷入了深深的怅惘之中。

他看见那个酒鬼静静地坐在桌前,面前有一堆空酒瓶和一堆装得满满的酒瓶,他就问:“你在那儿干什么呢?”

{\startalignment[center]
 \placefigeasy[][imgs/小王子(周克希譯)/00028.jpeg][maxwidth=\textwidth,maxheight=\textheight,location={middle,none}]{}
 \stopalignment}

“我喝酒。”酒鬼神情悲伤地回答。

“你为什么要喝酒呢?”小王子问。

“为了忘记。”酒鬼回答。

“忘记什么?”小王子已经有些同情他了。

“忘记我的羞愧。”酒鬼垂下脑袋坦白说。

“为什么感到羞愧?”小王子又问,他想帮助这个人。

“为喝酒感到羞愧!”酒鬼说完这句话,就再也不开口了。

小王子茫然不解地走了。

“这些大人真的很怪很怪。”一路上,他自言自语地说。

{\startalignment[center]
 \placefigeasy[][imgs/小王子(周克希譯)/00029.jpeg][maxwidth=\textwidth,maxheight=\textheight,location={middle,none}]{}
 \stopalignment}

\reference[part0015.html]{}

\stoptitle

\starttitle[title={13},reference={part0015.html_a016}]

第四颗行星是个商人的星球。这个人实在太忙碌了,看见小王子来,连头也没抬一下。

“您好,”小王子对他说,“您的烟卷灭了。”

“三加二等于五。五加七等于十二。十二加三等于十五。你好。十五加七等于二十二。二十二加六是二十八。没时间再去点着它。二十六加五,三十一。嚯!一共是五亿一百六十二万二千七百三十一。”

“五亿什么呀?”

“呣?你还在这儿?五亿一百六十二万\ldots{}\ldots{}我也不知道是什么了\ldots{}\ldots{}我的工作太多了!我做的都是正事,我没有工夫闲聊!二加五等于七\ldots{}\ldots{}”

“五亿一百万什么?”小王子又问一遍,他向来是不提问题则罢,提了就决不放过。

商人抬起头来:

“我在这个星球上住了五十四个年头,只被打搅过三次。第一次是二十二年以前,有只不知从哪儿跑来的金龟子,弄出一片可怕的声音,害得我在一笔账目里出了四个差错。第二次是十一年前,我风湿病发作。我平时缺乏锻炼。我没工夫去闲逛。我是干正事的人。第三次\ldots{}\ldots{}就是这次!所以我刚才说了,五亿一百六十二万\ldots{}\ldots{}”

“五亿一百六十二万什么?”

商人明白他是甭想太平了:

“五亿一百六十二万个小东西,有时候在天空里看得见它们。”

“苍蝇?”

“不对,是闪闪发亮的小东西。”

“蜜蜂?”

“不对。是些金色的小东西,无所事事的人望着它们会胡思乱想。可我是干正事的人!我没工夫去胡思乱想。”

“噢!是星星?”

“对啦。星星。”

“你拿这五亿颗星星做什么呢?”

“五亿一百六十二万二千七百三十一颗。我是个认真的人,我讲究精确。”

“那你拿这些星星来做什么呢?”

“我拿它们做什么?”

“是啊。”

“不做什么。我占有它们。”

“你占有这些星星?”

“对。”

“可我已经见到有个国王,他\ldots{}\ldots{}”

“国王并不占有。他们只是‘统治'。这完全是两码事。”

“占有这些星星对你有什么用呢?”

“可以使我富有。”

“富有对你有什么用呢?”

“可以去买其他的星星------只要有人发现了这样的星星。”

“这个人,”小王子暗自思忖,“想问题有点像那个酒鬼。”

话虽这么说,他还是接着提问题:

“一个人怎么能够占有这些星星呢?”

“它们属于谁了?”商人没好气地顶了他一句。

“我不知道。谁也不属于。”

“那么它们就属于我,因为是我第一个想到这件事的。”

“这就够了?”

“当然。当你发现一颗不属于任何人的钻石,它就属于你。当你发现一个不属于任何人的岛屿,它就属于你。当你最先想出一个主意,你去申请发明专利,它就属于你。现在我占有了这些星星,因为在我以前没有人想到过占有它们。”

“这倒也是,”小王子说,“可你拿它们来做什么呢?”

“我经营它们。我一遍又一遍地计算它们的数目,”商人说,“这并不容易。可我是个干正事的人!”

小王子还是不满意。

“我呀,如果我有一块方围巾,我可以把它围在脖子上带走它。如果我有一朵花儿,我可以摘下这朵花儿带走它。可是你没法摘下这些星星呀!”

“没错,但是我可以把它们存入银行。”

“这是什么意思?”

“这就是说,我把我的星星的总数写在一张小纸片上。然后我把这张小纸片放进一个抽屉锁好。”

“就这些?”

“这就够了!”

“真有趣,”小王子心想,“倒挺有诗意的。可这算不上什么正事呀。”

小王子对正事的看法,跟大人对正事的看法很不相同。

“我有一朵花儿,”他又说道,“我每天都给她浇水。我有三座火山,我每星期都把它们疏通一遍。那座死火山我也疏通。因为谁也说不准它还会不会喷发。我占有它们,对火山有好处,对花儿也有好处。可是你占有星星,对它们没有好处。”

商人张口结舌,无言以对。小王子就走了。

“这些大人真的好古怪。”一路上,他只是自言自语说了这么一句。

\reference[part0016.html]{}

\stoptitle

\starttitle[title={14},reference={part0016.html_a017}]

第五颗行星非常奇怪。这是最小的一颗。上面刚好只能容得下一盏路灯和一个点灯人。小王子好生纳闷,在天空的一个角落,在一个既没有房子也没有居民的行星上,要一盏路灯和一个点灯人,又能有什么用呢?不过他还是对自己说:

“很可能这个人是有点不正常。但是跟那个国王,那个爱虚荣的人,那个商人和那个酒鬼比起来,他还是要比他们正常些。至少他的工作还有意义。他点亮路灯,就好比唤醒了另一个太阳或者一朵花儿。他熄灭路灯,就好比让这朵花儿或这个太阳睡觉了。这是件很美的事情。既然很美,自然就有用啰。”

他一到这个星球,就很尊敬地向点灯人打招呼:

“早上好。你刚才为什么把路灯熄掉呢?”

“这是规定,”点灯人回答说,“早上好。”

“什么规定?”

“熄灭路灯呗。晚上好。”

说着他又点亮了路灯。

“那你刚才为什么又点亮路灯呢?”

“这是规定。”点灯人回答说。

“我弄不懂。”小王子说。

“没什么要弄懂的,”点灯人说,“规定就是规定。早上好。”

说着他熄灭了路灯。

然后他用一块有红方格的手帕擦了擦额头。

“我干的是桩非常累人的差事。以前还说得过去。我早晨熄灯,晚上点灯。白天我有时间休息,夜里也有时间睡觉\ldots{}\ldots{}”

{\startalignment[center]
 \placefigeasy[][imgs/小王子(周克希譯)/00030.jpeg][maxwidth=\textwidth,maxheight=\textheight,location={middle,none}]{}
 \stopalignment}

我干的是桩非常累人的差事

“那么,后来规定改变了?”

“规定没有改变,”点灯人说。“惨就惨在这儿!这颗行星一年比一年转得快,可规定却没变!”

“结果呢?”小王子说。

“结果现在每分钟转一圈,我连一秒钟的休息时间都没有。我每分钟就要点一次灯,熄一次灯!”

“这可真有趣!你这儿一天只有一分钟!”

“一点也不有趣!”点灯人说,“我们说着话,就已经一个月过去了。”

“一个月?”

“对。三十分钟。三十天!晚上好。”

说着他点亮了路灯。

小王子瞧着他,心里喜欢上了这个忠于职守的点灯人。他想起了自己以前的挪椅子看日落。他挺想帮助这个朋友:

“你知道\ldots{}\ldots{}我有一个办法,好让你想休息就能休息\ldots{}\ldots{}”

“我一直想休息。”点灯人说。

因为,一个人可以同时是忠于职守的,又是生性疏懒的。

小王子接着说:“你的星球小得很,你走三步就绕了一圈。所以你只要走得慢一些,就可以一直待在阳光下。你要想休息了,就往前走\ldots{}\ldots{}你要白天有多长,它就有多长。”

“这办法帮不了我多少忙,”点灯人说,“我这人,平生就喜欢睡觉。”

“真不走运。”小王子说。

“真不走运,”点灯人说,“早上好。”

说着他熄灭了路灯。

“这个人呀,”小王子一边继续他的旅途,一边在想,“国王也好,爱虚荣的人也好,酒鬼也好,商人也好,他们都会瞧不起这个人。可是,就只有他没让我感到可笑。也许,这是因为他关心的是别的事情,而不是自己。”

他惋惜地叹了口气,又自言自语说:

“只有这个人我可以跟他交朋友。可是他的星球实在太小了。两个人挤不下\ldots{}\ldots{}”

小王子不敢承认的是,他留恋这颗受上苍眷顾的星球,是因为每二十四小时就有一千四百四十次日落!

\reference[part0017.html]{}

\stoptitle

\starttitle[title={15},reference={part0017.html_a018}]

第六颗行星,是一颗比第五颗大十倍的行星。上面住着一个老先生,他在写一本大部头的著作。

“瞧!来了一位探险家!”他一看见小王子,就喊道。

小王子坐在桌边,喘了喘气。他刚走了那么多路!

“你从哪儿来啊?”老先生问他。

“这一大本是什么书?”小王子说。“您在这儿干什么呢?”

“我是地理学家。”老先生说。

“什么叫地理学家?”

“地理学家是个学者,他知道哪儿有海洋,有河流,有城市,有山脉和沙漠。”

“这挺有趣,”小王子说,“啊,这才是真正的职业!”说着他朝地理学家的星球四周望了一眼。他还从没见过这么雄伟壮丽的星球呢。

{\startalignment[center]
 \placefigeasy[][imgs/小王子(周克希譯)/00031.jpeg][maxwidth=\textwidth,maxheight=\textheight,location={middle,none}]{}
 \stopalignment}

“您的星球真美。它有海洋吗?”

“这我没法知道。”地理学家说。

“哦!”小王子有点失望。“那么山脉呢?”

“这我没法知道。”地理学家说。

“城市、河流和沙漠呢?”

“这我也没法知道。”地理学家说。

“可您是地理学家呀!”

“一点不错,”地理学家说,“但我不是探险家。我这里一个探险家也没有。地理学家是不出去探测城市、河流、山脉、海洋和沙漠的。地理学家非常重要,他不能到处闲逛。他从不离开自己的书房。不过他会在那里接见探险家。他向他们提问,把他们的旅行回忆记下来。要是他觉得他们中间哪个人的回忆有意思,他就会让人对这个探险家的品行作一番调查。”

“这是为什么?”

“因为一个说谎的探险家会给地理书带来灾难性的后果。一个贪杯的探险家也是如此。”

“这是为什么?”小王子问。

“因为酒鬼会把一样东西看成两样东西。这样一来,地理学家就会把明明只有一座山的地方写成有两座山了。”

“我认识一个人,”小王子说,“他当探险家就不行。”

“这有可能。所以,要等到了解探险家品行良好以后,才能对他的发现进行调查。”

“去看一下?”

“不。这太复杂了。地理学家只要求探险家提供物证。比如说,他发现了一座大山,地理学家就要求他带一块大石头来。”

地理学家忽然激动起来。

“嗨,你是大老远来的!你是探险家!你给我说说你的星球!”

说着,地理学家打开笔记本,削了支铅笔。地理学家一开始只用铅笔记下探险家讲的话。要等到这个探险家提供物证以后,才换用钢笔来记录。

“怎么样?”地理学家问。

“哦!我那儿,”小王子说,“并不很有趣,那是颗很小的星球。我有三座火山。两座活火山,一座死火山。不过这也说不定。”

“这可说不定。”地理学家说。

“我还有一朵花儿。”

“花儿我们是不记下来的。”地理学家说。

“这是为什么?花儿是最美的呀!”

“因为花是转瞬即逝的。”

“什么叫‘转瞬即逝'呢?”

“地理书,”地理学家说,“是所有的书中间最宝贵的。地理书永远不会过时。山脉移位的情形是极其罕见的。海洋干涸的情形也是极其罕见的。我们写的都是永恒的事物。”

“可是死火山说不定也会醒来。”小王子插话说,“什么叫‘转瞬即逝'呢?”

“火山睡也好,醒也好,对我们地理学家来说是一码事,”地理学家说,“我们关心的是山。山是一成不变的。”

{\startalignment[center]
 \placefigeasy[][imgs/小王子(周克希譯)/00032.jpeg][maxwidth=\textwidth,maxheight=\textheight,location={middle,none}]{}
 \stopalignment}

“可是,什么叫‘转瞬即逝'呢?”小王子追问道,他向来提了问题就不肯放过。

“意思就是‘随时有消逝的危险'。”

“我的花儿随时有消逝的危险吗?”

“当然。”

“我的花儿是转瞬即逝的,”小王子想道,“她只有四根刺可以自卫,可以用来抵御这个世界!而我却丢下她孤零零地在那儿!”

想到这儿,他不由得感到了后悔。不过他马上又振作起来:

“依您看,我再去哪儿访问好呢?”他问。

“地球吧,”地理学家回答说。“它的名气挺响\ldots{}\ldots{}”

于是小王子走了,一边走一边想着他的花儿。

\reference[part0018.html]{}

\stoptitle

\starttitle[title={16},reference={part0018.html_a019}]

所以,第七颗行星就是地球了。

地球可不是普普通通的行星!它上面有一百十一个国王(当然,黑人国王也包括在内),七千个地理学家,九十万个商人,七百五十万个酒鬼,三亿一千一百个爱虚荣的人,总共大约有二十亿个大人。

为了让你们对地球的大小有个概念,我就这么对你们说吧,在发明电以前,地球的六大洲上,需要维持一支四十六万二千五百十一人的浩浩荡荡的点灯人大军。

从稍远些的地方看去,这是一幅壮丽的景观。这支大军行动起来,就像在歌剧院里跳芭蕾舞那样有条不紊。最先上场的是新西兰和澳大利亚的点灯人。点着了灯,他们就退下去睡觉。接着是中国和西伯利亚的点灯人上场,随后他们也退到幕后。下面轮到了俄罗斯和印度的点灯人。接下去是非洲和欧洲的,而后是南美的。再后来是北美的。所有这些点灯人从来不会搞乱上场的次序。这场面真是蔚为壮观。

只有北极(那儿只有一盏路灯)的点灯人和南极(那儿也只有一盏路灯)的那个同行,过着悠闲懒散的生活:他俩一年干两回活儿。

\reference[part0019.html]{}

\stoptitle

\starttitle[title={17},reference={part0019.html_a020}]

一个人如果想把话说得有趣些,免不了会稍稍撒点谎。我给你们讲点灯人大军的那会儿,就不是很诚实。那些不了解我们行星的人,听了我讲的故事,可能会造成一种错觉。其实人在地球上只占一点点地方。倘若让地球上的二十亿居民全都挨个儿站着,就像集会时那样,那么二十海里长、二十海里宽的一个广场就容得下他们。全人类可以挤在太平洋上最小的一个岛屿上。

当然,大人是不会相信你们的。他们自以为占了好多好多地方。他们把自己看得跟猴面包树一样重要。你们也许会想劝他们好好算一算。他们喜欢数字,说到计算就来劲。不过你们可别浪费时间,去给自己添麻烦。根本不用去做。你们相信我就行了。

所以小王子一踏上地球,就觉得奇怪,怎么一个人也看不见呢?他正在担心是不是来错了星球,忽然看见沙地上一个月白色的圆环在挪动。

“晚上好。”小王子没把握地招呼说。

“晚上好。”蛇说。

“我落在哪个行星上了?”小王子问。

“在地球上,这是非洲。”蛇回答。

“噢!难道地球上一个人也没有吗?”

“这儿是沙漠。在沙漠上是一个人也没有的。地球大着呢。”蛇说。

小王子在一块石头上坐下,抬头望着天空:

“我在想,”他说,“这些星星闪闪发亮,大概是要让每个人总有一天能找到自己的那颗星星吧。瞧我的那颗星星。它正好在我们头顶上\ldots{}\ldots{}可是它离得那么远!”

{\startalignment[center]
 \placefigeasy[][imgs/小王子(周克希譯)/00033.jpeg][maxwidth=\textwidth,maxheight=\textheight,location={middle,none}]{}
 \stopalignment}

“你真是种奇怪的动物,”最后他说,“细得像根手指\ldots{}\ldots{}”

“它很美。”蛇说,“你到这儿来干吗?”

“我和一朵花儿闹了别扭。”小王子说。

“噢!”蛇说。

他俩都沉默了。

“哪儿见得到人呢?”小王子终于又开口了,“在沙漠里真有点孤独\ldots{}\ldots{}”

“在人群中间,你也会感到孤独。”蛇说。

小王子久久地注视着蛇:

“你真是种奇怪的动物,”最后他说,“细得像根手指\ldots{}\ldots{}”

“可我比一个国王的手指还厉害呢。”蛇说。

小王子笑了:

“你厉害不到哪儿去\ldots{}\ldots{}你连脚都没有\ldots{}\ldots{}要出远门你就不行吧?”

“我可以把你带到很远很远的地方去,比一艘船去的地方还远。”蛇说。

它盘在小王子的脚踝上,像一只金镯子:

“凡是我碰过的人,我都把他们送回老家去,”它又说,“可你这么纯洁,又是从一颗星星那儿来的\ldots{}\ldots{}”

小王子没有作声。

“在这个花岗石的地球上,你是这么弱小,我很可怜你。哪天你要是想念你的星星了,我可以帮助你。我可以\ldots{}\ldots{}”

“噢!我明白你的意思,”小王子说,“可为什么你说的话都像谜似的?”

“这些谜我都能解开。”蛇说。

然后他们又都沉默了。

\reference[part0020.html]{}

\stoptitle

\starttitle[title={18},reference={part0020.html_a021}]

小王子穿过沙漠,只见到了一朵花儿。一朵长着三片花瓣的花儿,一朵不起眼的花儿\ldots{}\ldots{}

“你好。”小王子说。

“你好。”花儿说。

“人们在哪儿呢?”小王子有礼貌地问。

花儿看见过一支沙漠驼队经过:

“人们?我想是有的,不是六个就是七个。好几年以前,我见过他们。不过谁也不知道,要上哪儿才能找到他们。风把他们一会儿吹到这儿,一会儿吹到那儿。他们没有根,活得很辛苦。”

“再见了。”小王子说。

“再见。”花儿说。

{\startalignment[center]
 \placefigeasy[][imgs/小王子(周克希譯)/00034.jpeg][maxwidth=\textwidth,maxheight=\textheight,location={middle,none}]{}
 \stopalignment}

\reference[part0021.html]{}

\stoptitle

\starttitle[title={19},reference={part0021.html_a022}]

小王子攀上一座高山。他过去只见过三座齐膝高的火山。他还把那座死火山当凳子坐哩。“从一座这么高的山上望下去,”他心想,“我一眼就能看到整个星球和所有的人们\ldots{}\ldots{}”可是,他看到的只是些陡峭的山峰。

“你们好。”他怯生生地招呼说。

“你们好\ldots{}\ldots{}你们好\ldots{}\ldots{}你们好\ldots{}\ldots{}”回声应道。

“你们是谁呀?”小王子问。

“你们是谁呀\ldots{}\ldots{}你们是谁呀\ldots{}\ldots{}你们是谁呀\ldots{}\ldots{}”回声应道。

“请做我的朋友吧,我很孤独,”他说。

“我很孤独\ldots{}\ldots{}我很孤独\ldots{}\ldots{}我很孤独\ldots{}\ldots{}”回声应道。

“这颗行星可真怪!”他心想。“又干,又尖,又锋利。人们一点想象力都没有。他们老是重复别人对他们说的话\ldots{}\ldots{}在我那儿有一朵花儿,她总是先开口说话的\ldots{}\ldots{}”

{\startalignment[center]
 \placefigeasy[][imgs/小王子(周克希譯)/00035.jpeg][maxwidth=\textwidth,maxheight=\textheight,location={middle,none}]{}
 \stopalignment}

这颗行星又干,又尖,又锋利

\reference[part0022.html]{}

\stoptitle

\starttitle[title={20},reference={part0022.html_a023}]

小王子在沙漠、山岩和雪地上走了很长时间以后,终于发现了一条路。所有的路都通往有人住的地方。

“你们好。”他说。

眼前是一座玫瑰盛开的花园。

“你好。”玫瑰们说。

小王子瞧着她们。她们都长得和他的花儿一模一样。

“你们是什么花呀?”他惊奇地问。

“我们是玫瑰花。”玫瑰们说。

“噢!\ldots{}\ldots{}”小王子说。

他感到非常伤心。他的花儿跟他说过,她是整个宇宙中独一无二的花儿。可这儿,在一座花园里就有五千朵,全都一模一样!

{\startalignment[center]
 \placefigeasy[][imgs/小王子(周克希譯)/00036.jpeg][maxwidth=\textwidth,maxheight=\textheight,location={middle,none}]{}
 \stopalignment}

“要是让她看到了,”他想,“她一定会非常生气\ldots{}\ldots{}她会拼命咳嗽,她还会假装死去,免得让人耻笑。我呢,还得假装去照料她,否则她为了让我感到羞愧,说不定真的会让自己死去\ldots{}\ldots{}”

随后他又想:“我还以为自己拥有的是独一无二的一朵花儿呢,可我有的只是普普通通的一朵玫瑰花罢了。这朵花儿,加上那三座只到我膝盖的火山,其中有一座还说不定永远不会再喷发,就凭这些,我怎么也成不了一个伟大的王子\ldots{}\ldots{}”想着想着,他趴在草地上哭了起来。

{\startalignment[center]
 \placefigeasy[][imgs/小王子(周克希譯)/00037.jpeg][maxwidth=\textwidth,maxheight=\textheight,location={middle,none}]{}
 \stopalignment}

他趴在草地上哭了起来

\reference[part0023.html]{}

\stoptitle

\starttitle[title={21},reference={part0023.html_a024}]

就在这时狐狸出现了。

“早哇,”狐狸说。

“早。”小王子有礼貌地回答,他转过身来,却什么也没看到。

“我在这儿呢,”那声音说,“在苹果树下面\ldots{}\ldots{}”

“你是谁?”小王子说,“你很漂亮。”

“我是一只狐狸。”狐狸说。

“来和我一起玩吧,”小王子提议。“我很不快活\ldots{}\ldots{}”

“我不能和你一起玩,”狐狸说,“还没人驯养过我呢。”

“噢!对不起。”小王子说。

不过,他想了想又说:

“‘驯养'是什么意思?”

“你一定不是这儿的人,”狐狸说,“你来寻找什么呢?”

“我来找人,”小王子说,“‘驯养'是什么意思?”

“人哪,”狐狸说,“他们有枪,还打猎。讨厌极了!他们还养母鸡,这总算有点意思。你也找母鸡吗?”

“不找,”小王子说,“我找朋友。‘驯养'是什么意思?”

“这是一件经常被忽略的事情,”狐狸说,“意思是‘建立感情联系'\ldots{}\ldots{}”

“建立感情联系?”

“可不是,”狐狸说。“现在你对我来说,只不过是个小男孩,跟成千上万别的小男孩毫无两样。我不需要你。你也不需要我。我对你来说,也只不过是个狐狸,跟成千上万别的狐狸毫无两样。但是,你要是驯养了我,我俩就彼此都需要对方了。你对我来说是世界上独一无二的。我对你来说,也是世界上独一无二的\ldots{}\ldots{}”

“我有点明白了,”小王子说,“有一朵花儿\ldots{}\ldots{}我想她是驯养了我\ldots{}\ldots{}”

“有可能,”狐狸说,“这个地球上各色各样的事都有\ldots{}\ldots{}”

“哦!不是在地球上。”小王子说。

狐狸看上去很惊讶:

“在另一个星球上?”

“对。”

“在那个星球上有没有猎人呢?”

“没有。”

“哈,这很有意思!那么母鸡呢?”

“没有。”

“没有十全十美的事呵。”狐狸叹气说。

不过,狐狸很快又回到刚才的想法上来:

“我的生活很单调。我去捉鸡,人来捉我。母鸡全都长得一个模样,人也全都长得一个模样。所以我有点腻了。不过,要是你驯养我,我的生活就会变得充满阳光。我会辨认出一种和其他所有人都不同的脚步声。听见别的脚步声,我会往地底下钻,而你的脚步声,会像音乐一样,把我召唤到洞外。还有,你看!你看到那边的麦田了吗?我是不吃面包的。麦子对我来说毫无用处。我对麦田无动于衷。可悲就可悲在这儿!而你的头发是金黄色的。所以,一旦你驯养了我,事情就变得很美妙了!金黄色的麦子,会让我想起你。我会喜爱风儿吹拂麦浪的声音\ldots{}\ldots{}”

狐狸停下来,久久地注视着小王子:

{\startalignment[center]
 \placefigeasy[][imgs/小王子(周克希譯)/00038.jpeg][maxwidth=\textwidth,maxheight=\textheight,location={middle,none}]{}
 \stopalignment}

“请你\ldots{}\ldots{}驯养我吧!”他说。

“我很愿意,”小王子回答说,“可是我时间不多了。我得去找朋友,还得去了解许多东西。”

“只有驯养过的东西,你才会了解它,”狐狸说,“人们根本没有时间去了解任何东西。他们总到商店去购买现成的东西。可是朋友是商店里买不到的,所以他们就不会有朋友。你如果想要有朋友,就驯养我吧!”

“那么应当做些什么呢?”小王子说。

“应当很有耐心。”狐狸回答说,“你先坐在草地上,离我稍远一些,就像这样。我从眼角里瞅你,而你什么也别说。语言是误解的根源。不过,每天你都可以坐得离我稍稍近一些\ldots{}\ldots{}”

{\startalignment[center]
 \placefigeasy[][imgs/小王子(周克希譯)/00039.jpeg][maxwidth=\textwidth,maxheight=\textheight,location={middle,none}]{}
 \stopalignment}

{\startalignment[center]
 \placefigeasy[][imgs/小王子(周克希譯)/00040.jpeg][maxwidth=\textwidth,maxheight=\textheight,location={middle,none}]{}
 \stopalignment}

如果你能在下午四点钟来,那么我在三点钟就会有一种幸福的感觉

第二天,小王子又来了。

“最好你能在同一时间来,”狐狸说,“比如说,下午四点钟吧,那么我在三点钟就会开始感到幸福了。时间越来越近,我就越来越幸福。到了四点钟,我会兴奋得坐立不安;我会觉得,幸福原来也折磨人唷!可要是你随便什么时候来,我就没法知道什么时候该准备好我的心情\ldots{}\ldots{}还是得有个仪式。”

“什么叫仪式?”小王子问。

“这也是一件经常被忽略的事情。”狐狸说,“就是定下一个日子,使它不同于其他的日子,定下一个时间,使它不同于其他的时间。比如说,猎人有一种仪式。每星期四他们都和村里的姑娘跳舞。所以呢,星期四就是个美妙的日子!这一天我总要到葡萄地里去转悠转悠。要是猎人们随便什么时候都跳舞,每天不就都一模一样,我不也就没有假期了吗?”

就这样,小王子驯养了狐狸。而后,眼看分手的时刻临近了:

“哎!”狐狸说,“\ldots{}\ldots{}我要哭了。”

“这可是你的不是哟,”小王子说,“我本来没想让你受伤害,可你却要我驯养你\ldots{}\ldots{}”

“可不是。”狐狸说。

“现在你要哭了!”小王子说。

“可不是。”狐狸说。

“你什么好处也没得到!”

“我得到了,”狐狸说,“是麦田的颜色给我的。”

他随即又说:

“你再去看看那些玫瑰花吧。你会明白你那朵玫瑰是世界上独一无二的。然后你再回来跟我告别,我要告诉你一个秘密作为临别礼物。”

小王子就去看那些玫瑰。

“你们根本不像我那朵玫瑰,你们还什么都不是呢,”他对她们说,“谁都没驯养过你们,你们也谁都没驯养过。你们就像狐狸以前一样。那时候的它,和成千上万别的狐狸毫无两样。可是我现在和它做了朋友,它在世界上就是独一无二的了。”

玫瑰们都很难为情。

“你们很美,但你们是空虚的,”小王子接着说,“没有人能为你们去死。当然,我那朵玫瑰在一个过路人眼里跟你们也一样。然而对于我来说,单单她这一朵,就比你们全体都重要得多。因为我给浇过水的是她,我给盖过罩子的是她,我给遮过风障的是她,我给除过毛虫的(只把两三条要变成蝴蝶的留下)也是她。我听她抱怨和自夸,有时也和她默默相对。她,是我的玫瑰。”

说完,他又回到狐狸跟前:

“再见了\ldots{}\ldots{}”他说。

“再见,”狐狸说,“我告诉你那个秘密,它很简单:只有用心才能看见。本质的东西用眼是看不见的。”

“本质的东西用眼是看不见的。”小王子重复了一遍,他要记住这句话。

“正是你为你的玫瑰花费的时光,才使你的玫瑰变得如此重要。”

“正是我为我的玫瑰花费的时光,才使我的玫瑰变得如此重要。”小王子说,他要记住这句话。

“人们已经忘记了这个道理,”狐狸说,“但你不该忘记它。对你驯养过的东西,你永远负有责任。你必须对你的玫瑰负责\ldots{}\ldots{}”

“我必须对我的玫瑰负责\ldots{}\ldots{}”小王子重复一遍,他要记住这句话。

\reference[part0024.html]{}

\stoptitle

\starttitle[title={22},reference={part0024.html_a025}]

“你好。”小王子说。

“你好。”扳道工说。

“你在这儿做什么?”小王子问。

“我在分送旅客,一千人一拨。”扳道工说,“我发送运载旅客的列车,一会儿往右,一会儿往左。”

说着,一列灯火通明的快车,像打雷似的轰鸣着驶过,震得扳道房直打颤。

“他们好匆忙,”小王子说,“他们去找什么呢?”

“开火车的人自己也不知道。”扳道工说。

说话间,又一列灯火通明的快车,朝相反的方向轰鸣而去。

“他们已经回来了?”小王子问。

“不是刚才的那列,”扳道工说,“这是对开列车。”

“他们对原来的地方不满意吗?”

“人们对自己的地方从来不会满意。”扳道工说。

第三列灯火通明的快车轰鸣着驶过。

“他们是去追赶第一批旅客吗?”小王子问。

“他们没追赶谁,”扳道工说,“他们在里面睡觉,或者打哈欠。只有孩子把鼻子贴在窗上看外面。”

“只有孩子知道自己在找什么,”小王子说,“他们在一个布娃娃身上花了好些时间,她对他们来说就成了很重要的东西。要是有人夺走他们的布娃娃,他们会哭的\ldots{}\ldots{}”

“他们真幸运。”扳道工说。

\reference[part0025.html]{}

\stoptitle

\starttitle[title={23},reference={part0025.html_a026}]

“你好。”小王子说。

“你好。”商人说。

他是个卖复方止渴丸的商人。每星期只要吞服一粒,就不会感到口渴了。

“你为什么要卖这东西?”小王子问。

“它可以大大节约时间,”商人说,“专家做过计算。每星期可以省下五十三分钟。”

“省下的五十三分钟做什么用呢?”

“随便怎么用都行\ldots{}\ldots{}”

“我呀,”小王子心想,“要是我省下这五十三分钟,我就不慌不忙地朝泉水走去\ldots{}\ldots{}”

{\startalignment[center]
 \placefigeasy[][imgs/小王子(周克希譯)/00041.jpeg][maxwidth=\textwidth,maxheight=\textheight,location={middle,none}]{}
 \stopalignment}

\reference[part0026.html]{}

\stoptitle

\starttitle[title={24},reference={part0026.html_a027}]

这是我降落在沙漠后的第八天,我听着这个商人的故事,喝完了最后一滴备用水。

“喔!”我对小王子说,“你的回忆很动人,可是我飞机还没修好,水也喝完了,要是我能朝泉水走去,那真是有福了!”

“我那狐狸朋友\ldots{}\ldots{}”他说。

“小家伙,这可不干狐狸的事!”

“为什么?”

“因为我快要渴死了\ldots{}\ldots{}”

他没明白我的思路,回答我说:

“有朋友真好,即使就要死了,我也还是这么想。我真高兴,有过一个狐狸朋友\ldots{}\ldots{}”

“他没明白情势有多凶险,”我心想,“他从来不知道饥渴。只要有点阳光,他就足够了\ldots{}\ldots{}”

然而他注视着我,好像知道我心里在想什么:

“我也渴\ldots{}\ldots{}我们去找一口井吧\ldots{}\ldots{}”

我做了个表示厌烦的手势:在一望无垠的沙漠中,漫无目标地去找井,简直是荒唐。然而,我们到底还是上路了。

默默地走了几个钟头以后,夜幕降临了,星星在天空中闪烁起来。由于渴得厉害,我有点发烧,望着天上的星星,仿佛在梦中。小王子的话在脑海里盘旋舞蹈。

“你也渴?”我问。

他没有回答我的问题,只对我说:

“水对心灵也有好处\ldots{}\ldots{}”

我没听懂他的话,但我没做声\ldots{}\ldots{}我知道,这会儿不该去问他。

他累了。他坐了下来。我坐在他身旁。沉默了一会儿,他又说:

“星星很美,因为有一朵看不见的花儿\ldots{}\ldots{}”

我说了声“可不是”,就静静地注视着月光下沙漠的褶皱。

“沙漠很美。”他又说。

没错。我一向喜欢沙漠。我们坐在一个沙丘上。什么也看不见。什么也听不见。然而有什么东西在寂静中发出光芒\ldots{}\ldots{}

“沙漠这么美,”小王子说,“是因为有个地方藏着一口井\ldots{}\ldots{}”

我非常吃惊,突然间明白了沙漠发光的奥秘。我小时候住在一座老宅里,传说宅子里埋着宝藏。当然,从来没人发现过这宝藏,或许根本没人寻找过它。但是它使整座宅子变得令人着迷。我的宅子,把一个秘密藏在我心灵深处了\ldots{}\ldots{}

“对,”我对小王子说,“不管是宅子,还是星星或沙漠,使它们变美的东西,都是看不见的!”

“我很高兴,”他说,“你和狐狸的看法一样了。”

看小王子睡着了,我把他抱起来,重新上路。我很激动。我觉得就像捧着一件易碎的宝贝。我甚至觉得在地球上,再没有更娇弱的东西了。我在月光下看着他苍白的前额,紧闭的眼睛,还有那随风飘动的发绺,在心里对自己说:“我所看到的只是外貌。最重要的东西是看不见的\ldots{}\ldots{}”

当他微微张开的嘴唇绽出一丝笑意时,我又对自己说:“在这个熟睡的小王子身上,最让我感动的,是他对一朵花儿的忠贞,这朵玫瑰的影像,即使在他睡着时,仍然在他身上发出光芒,就像一盏灯的火焰一样\ldots{}\ldots{}”这时我把他想得更加娇弱了。应该好好保护灯火呵,一阵风就会吹灭它\ldots{}\ldots{}

就这样走啊走啊,我在拂晓时发现了水井。

\reference[part0027.html]{}

\stoptitle

\starttitle[title={25},reference={part0027.html_a028}]

“人们挤进快车,”小王子说,“可是又不知道还要去寻找什么。所以他们忙忙碌碌,转来转去\ldots{}\ldots{}”

他接着又说:

“其实何必呢\ldots{}\ldots{}”

我们找到的这口井,跟撒哈拉沙漠的那些井不一样。那些井,只是沙漠上挖的洞而已。这口井很像村庄里的那种井。可这儿根本就没有村庄呀,我觉得自己在做梦。

“真奇怪,”我对小王子说,“样样都是现成的:辘轳,水桶,吊绳\ldots{}\ldots{}”

他笑了,拉住吊绳,让辘轳转起来。辘轳咕咕作响,就像一只吹不到风、沉睡已久的旧风标发出的声音。

“你听见吗?”小王子说,“我们唤醒了这口井,它在唱歌呢\ldots{}\ldots{}”

我不想让他多用力气:

“让我来吧,”我说,“这活儿对你来说太重了。”

我把水桶缓缓地吊到井栏上,稳稳地搁住。辘轳的歌声还在耳边响着,而在依然晃动着的水面上,我瞧见太阳在晃动。

“我想喝水,”小王子说,“给我喝吧\ldots{}\ldots{}”

我这时明白了他在寻找的是什么!

{\startalignment[center]
 \placefigeasy[][imgs/小王子(周克希譯)/00042.jpeg][maxwidth=\textwidth,maxheight=\textheight,location={middle,none}]{}
 \stopalignment}

他笑了,拉住吊绳,让辘轳转起来

我把水桶举到他的嘴边。他喝着水,眼睛没张开。水像节日一般美好。它已经不只是一种维持生命的物质。它来自星光下的跋涉,来自辘轳的歌唱,来自臂膀的用力。它像礼物一样愉悦着心灵。当我是个小男孩时,圣诞树的灯光,午夜弥撒的音乐,人们甜蜜的微笑,都曾像这样辉映着我收到的圣诞礼物,让它熠熠发光。

“你这儿的人,”小王子说,“在一座花园里种出五千朵玫瑰,却没能从中找到自己要找的东西\ldots{}\ldots{}”

“他们是没能找到\ldots{}\ldots{}”我应声说。

“然而他们要找的东西,在一朵玫瑰或者一点儿水里就能找到\ldots{}\ldots{}”

“可不是。”我应声说。

小王子接着说:

“但是用眼是看不见的。得用心去找。”

我喝了水。我痛快地呼吸着空气。沙漠在晨曦中泛出蜂蜜的色泽。这种蜂蜜的色泽,也使我心头洋溢着幸福的感觉。我为什么要难过呢\ldots{}\ldots{}

“你该实践自己的诺言了。”小王子柔声对我说,他这会儿又坐在了我的身边。

“什么诺言?”

“你知道的\ldots{}\ldots{}给我的羊画个嘴罩\ldots{}\ldots{}我要对我的花儿负责!”

我从衣袋里掏出几幅画稿。小王子瞥了一眼,笑着说:

“你的猴面包树呀,有点像白菜\ldots{}\ldots{}”

“哦!”

可我还为这几棵猴面包树感到挺得意呢!

“你的狐狸\ldots{}\ldots{}它的耳朵\ldots{}\ldots{}有点像两只角\ldots{}\ldots{}再说也太长了!”

说着他又笑了起来。

“你不公平,小家伙,我可就画过剖开和不剖开的蟒蛇,别的都没学过。”

“噢!这就行了,”他说,“孩子们会看懂的。”

我用铅笔画了一只嘴罩。把画递给他时,我的心揪紧了:

“你有些什么打算,我都不知道\ldots{}\ldots{}”

他没回答,却对我说:

“你知道,我降落到地球上\ldots{}\ldots{}到明天就满一年了\ldots{}\ldots{}”

然后,一阵静默过后,他又说道:

“我就落在这儿附近\ldots{}\ldots{}”

说着他的脸红了起来。

我也不知是什么原因,只觉得又感到一阵异样的忧伤。可是我想到了一个问题:

“这么说,一星期前我遇见你的那个早晨,你独自在这片荒无人烟的沙漠里走来,并不是偶然的了?你是要回到当初降落的地方来吧?”

小王子的脸又红了。

我有些犹豫地接着说:

“也许,是为了周年纪念?\ldots{}\ldots{}”

小王子脸又红了。他往往不回答人家的问题,但他脸一红,就等于在说“对的”,可不是吗?

“哎!”我对他说,“我怕\ldots{}\ldots{}”

他却回答我说:

“现在你该去工作了。你得回到你的飞机那儿去。我在这儿等你。明天晚上再来吧\ldots{}\ldots{}”

可是我放心不下。我想起了狐狸的话。一个人要是被驯养过,恐怕难免要哭的\ldots{}\ldots{}

\reference[part0028.html]{}

\stoptitle

\starttitle[title={26},reference={part0028.html_a029}]

在水井边上,有一堵残败的旧石墙。第二天傍晚,我干完活儿回来,远远地看见小王子两腿悬空地坐在断墙上。我还听见他在说话:

“难道你不记得了?”他说,“根本不是这儿!”

想必有一个声音在回答他,只见他在反驳:

“对!对!是今天,可不是这个地方\ldots{}\ldots{}”

我往石墙走去。我既没看见人影,也没听见人声。但是小王子又在说:

“\ldots{}\ldots{}那当然。在沙地上,你会看到我的足迹从哪儿开始的。你只要等着我就行了。今天夜里我就去那儿。”

我离石墙只有二十米了,可还是什么也没看见。

停了一会儿,小王子又说:

“你的毒液管用吗?你有把握不会让我难受很久吗?”

我心头猛地揪紧,停下了脚步,可我还是什么也不明白。

“现在,来吧,”小王子说,“\ldots{}\ldots{}我要下来了!”

{\startalignment[center]
 \placefigeasy[][imgs/小王子(周克希譯)/00043.jpeg][maxwidth=\textwidth,maxheight=\textheight,location={middle,none}]{}
 \stopalignment}

“现在,来吧,”小王子说,“\ldots{}\ldots{}我要下来了!”

这时,我低头朝墙脚看去,不由得吓了一跳!只见一条半分钟就能叫人致命的黄蛇,昂然竖起身子对着小王子。我一边伸手去掏手枪,一边撒腿往前奔去。可是,那条蛇听见我的声音,就像一条水柱骤然跌落下来,缓缓渗入沙地,不慌不忙地钻进石缝中去,发出轻微的金属声。

我赶到墙边,正好接住从墙上跳下的小王子,把这个脸色白得像雪的小家伙抱在怀里。

“这是怎么回事!你居然跟蛇在谈话!”

我解开他一直戴着的金黄色围巾。我用水沾湿他的太阳穴,给他喝了点水。可此刻我不敢再问他什么。他神色凝重地望着我,用双臂搂住我的脖子。我感觉到他的心跳,就像被枪弹击中濒临死亡的小鸟的心跳。他对我说:“我很高兴,你找到了飞机上缺少的东西。你可以回家了\ldots{}\ldots{}”

“你怎么知道的?”

我正想告诉他,就在刚才,在眼看没有希望的情况下,我修好了飞机!

他没回答我的问题,但接着说:

“我也一样,今天,我要回家了\ldots{}\ldots{}”

然后,忧郁地说:

“那要远得多\ldots{}\ldots{}难得多\ldots{}\ldots{}”

我意识到发生了一件非同寻常的事情。我把他像小孩那样抱在怀里,只觉得他在笔直地滑入一个深渊,而我全然无法拉住他\ldots{}\ldots{}

他的目光很严肃,视线消失在很远很远的地方。

“我有你的绵羊。我有绵羊的箱子。还有嘴罩\ldots{}\ldots{}”

说着,他忧郁地微微一笑。

{\startalignment[center]
 \placefigeasy[][imgs/小王子(周克希譯)/00044.jpeg][maxwidth=\textwidth,maxheight=\textheight,location={middle,none}]{}
 \stopalignment}

我等了很久。我感到他的身子渐渐暖了起来:

“小家伙,你受惊了\ldots{}\ldots{}”

他刚才受惊了,可不是!但他轻轻地笑了起来:

“今天晚上我要受更大的惊\ldots{}\ldots{}”

一种无法补救的感觉,再一次使我凉到了心里。想到从此就再也听不到他的笑声,我感到受不了。他的笑声对我来说,就像沙漠中的清泉。

“小家伙,我还想听到你咯咯地笑\ldots{}\ldots{}”

可是他对我说:

“到今天夜里,就是一年了。我的星星就在我去年降落的地方顶上\ldots{}\ldots{}”

“小家伙,蛇啊,相约啊,星星啊,敢情只是场恶梦吧\ldots{}\ldots{}”

可是他不回答我的问题。他对我说:

“重要的东西是看不见的\ldots{}\ldots{}”

“可不是\ldots{}\ldots{}”

“这就好比花儿一样。要是你喜欢一朵花儿,而她在一颗星星上,那你夜里看着天空,就会觉得很美。所有的星星都像开满了花儿。”

“可不是\ldots{}\ldots{}”

“这就好比水一样。昨天你给我喝的水,有了那辘轳和吊绳,就像一首乐曲\ldots{}\ldots{}你还记得吧\ldots{}\ldots{}那水真好喝。”

“可不是\ldots{}\ldots{}”

“夜里,你要抬头望着满天的星星。我那颗实在太小了,我都没法指给你看它在哪儿。这样倒也好。我的星星,对你来说就是满天星星中的一颗。所以,你会爱这满天的星星\ldots{}\ldots{}所有的星星都会是你的朋友。我还要给你一件礼物\ldots{}\ldots{}”

他又笑了起来。

“呵!小家伙,小家伙,我喜欢听到这笑声!”

“这正是我的礼物\ldots{}\ldots{}就像那水\ldots{}\ldots{}”

“你想说什么?”

“人们眼里的星星,并不是一样的。对旅行的人来说,星星是向导。对有些人来说,它们只不过是天空微弱的亮光。对另一些学者来说,它们就是要探讨的问题。对我那个商人来说,它们就是金子。但是所有这些星星都是静默的。而你,你的那些星星是谁也不曾见过的\ldots{}\ldots{}”

“你想说什么呢?”

“当你在夜里望着天空时,既然我就在其中的一颗星星上面,既然我在其中一颗星星上笑着,那么对你来说,就好像满天的星星都在笑。只有你一个人,看见的是会笑的星星!”

说着他又笑了。

“当你感到心情平静以后(每个人总会让自己的心情平静下来),你会因为认识了我而感到高兴。你会永远是我的朋友。你会想要跟我一起笑。有时候,你会心念一动,就打开窗子\ldots{}\ldots{}你的朋友会惊奇地看到,你望着天空在笑。于是你会对他们说:‘是的,我看见这些星星就会笑!'他们会以为你疯了。我给你闹了个恶作剧\ldots{}\ldots{}”

说着他又笑了。

“这样一来,我给你的仿佛不是星星,而是些会笑的小铃铛\ldots{}\ldots{}”

说着他又笑了。随后他变得很严肃:

“今天夜里\ldots{}\ldots{}你知道\ldots{}\ldots{}你不要来。”

“我决不离开你。”

“我看上去会很痛苦\ldots{}\ldots{}会有点像死去的样子。就是这么回事。你还是别看见的好,没这必要。”

“我决不离开你。”

可是他担心起来。

“我这么说\ldots{}\ldots{}也是因为蛇的缘故。你可别让它咬着了\ldots{}\ldots{}蛇,都是很坏的。它们无缘无故也会咬人\ldots{}\ldots{}”

“我决不离开你。”

不过,他想到了什么,又觉得放心了:

“可也是,它们咬第二口时,已经没有毒液了\ldots{}\ldots{}”

{\startalignment[center]
 \placefigeasy[][imgs/小王子(周克希譯)/00045.jpeg][maxwidth=\textwidth,maxheight=\textheight,location={middle,none}]{}
 \stopalignment}

当天夜里,我没看见他起程。他悄没声儿地走了。我好不容易赶上他时,他仍然执著地快步往前走。他只是对我说:

“啊!你来了\ldots{}\ldots{}”

说完他就拉住我的手。可是他又感到不安起来:

“你不该来的。你会难过的。我看上去会像死去一样,可那不是真的\ldots{}\ldots{}”

我不做声。

“你是明白的。路太远了。我没法带走这副躯壳。它太沉了。”

我不做声。

“可这就像一棵老树脱下的树皮。脱下一层树皮,是用不着伤心的\ldots{}\ldots{}”

我不做声。

他有点气馁。但他重又打起精神:

“你知道,这样挺好。我也会望着满天星星的。每颗星星都会有一个辘轳嘎嘎作响的水井。所有的星星都会倒水给我喝\ldots{}\ldots{}”

我不做声。

“这真是太有趣了!你有五亿个铃铛,我有五亿个水井\ldots{}\ldots{}”

他也不做声了,因为他哭了\ldots{}\ldots{}

“到了。让我独自跨出一步吧。”

说着他坐了下来,因为他害怕。

他又说:

“你知道\ldots{}\ldots{}我的花儿\ldots{}\ldots{}我对她负有责任!她是那么柔弱!她是那么天真。她只有四根微不足道的刺,用来抵御整个世界\ldots{}\ldots{}”

我也坐下,因为我没法再站住了。他说:

“好了\ldots{}\ldots{}没别的要说了\ldots{}\ldots{}“

他稍微犹豫了一下,随即站了起来。他往前跨出了一步,而我却动弹不得。

只见他的脚踝边上闪过一道黄光。片刻间他一动不动。他没有叫喊。他像一棵树那样,缓缓地倒下。由于是沙地,甚至都没有一点声响。

{\startalignment[center]
 \placefigeasy[][imgs/小王子(周克希譯)/00046.jpeg][maxwidth=\textwidth,maxheight=\textheight,location={middle,none}]{}
 \stopalignment}

他像一棵树那样,缓缓地倒下

\reference[part0029.html]{}

\stoptitle

\starttitle[title={27},reference={part0029.html_a030}]

现在,当然,已经过去六年了\ldots{}\ldots{}我还从来没跟人讲过这个故事。同伴们看见我活着回来,都很高兴。我很忧伤,但我对他们说:“我累了\ldots{}\ldots{}”

现在我的心情有点平静了。也就是说\ldots{}\ldots{}还没有完全平静。而我知道,他已经回到了他的星球,因为那天天亮以后,我没发现他的躯体。他的躯体并不太沉\ldots{}\ldots{}我喜欢在夜里倾听星星的声音。它们就像五亿个铃铛。

可是,我想到有件事出了意外。我给小王子画的嘴罩,忘了加上皮带!他没法把它系在绵羊嘴上了。于是我一直在想:“在他的星球上到底会发生什么事呢?说不定绵羊真的把花儿给吃了\ldots{}\ldots{}”

有时我对自己说:“肯定不会!小王子每天夜里给花儿盖上玻璃罩,再说他也会仔细看好绵羊的\ldots{}\ldots{}”于是我感到很幸福。满天的星星轻轻地笑着。

有时我对自己说:“万一有个疏忽,那就全完了!没准哪天晚上,他忘了盖玻璃罩,或者绵羊在夜里悄悄钻了出来\ldots{}\ldots{}”于是满天的铃铛全都变成了泪珠!\ldots{}\ldots{}

这可是一个很大很大的秘密哟。对于也爱着小王子的你们,就像对于我一样,要是在我们不知道的哪个地方,有一只我们从没见过的绵羊,吃掉了或者没有吃掉一朵玫瑰,整个宇宙就会完全不一样\ldots{}\ldots{}

你们望着天空,想一想:绵羊到底有没有吃掉花儿?你们就会看到一切都变了样\ldots{}\ldots{}

而没有一个大人懂得这有多重要呵!

{\startalignment[center]
 \placefigeasy[][imgs/小王子(周克希譯)/00047.jpeg][maxwidth=\textwidth,maxheight=\textheight,location={middle,none}]{}
 \stopalignment}

对我来说,这是世界上最美丽、最伤感的景色。它跟前一页上画的是同一个景色,而我之所以再画一遍,是为了让你们看清这景色。就是在这儿,小王子在地球上出现,而后又消失。请仔细看看这景色,如果有一天你们到非洲沙漠去旅行,就肯定能认出它来。而要是你们有机会路过那儿,请千万别匆匆走过,请在那颗星星下面等上一会儿!如果这时有个孩子向你们走来,如果他在笑,如果他的头发是金黄色的,如果问他而他不回答,你们一定能猜到他是谁了。那么就请你们做件好事吧!请别让我再这么忧伤:赶快写信告诉我,他又回来了\ldots{}\ldots{}

\reference[part0030.html]{}

\stoptitle

\starttitle[title={Le Petit Prince
附录},reference={part0030.html_a031}]

\startsubject[title={初版译序},reference={part0030.html_b001}]

《小王子》是法国作家圣埃克絮佩里(1900---1944)的代表作。这位作家写了好多部著名的小说,同时也写了这样一部充满智慧闪光的童话。

一个多世纪以前,安托万·德·圣埃克絮佩里于1900年6月29日出生在法国里昂。他在姨妈家度过了童年时代,又去瑞士读中学。回国后,一边在巴黎美术学院就学,一边准备报考海军学院;结果却没有通过口试,未能如愿进入海军学院。他没能当成海军,却成了一名空军。二十一岁的圣埃克絮佩里应征服义务兵役,被派往斯特拉斯堡附近的空军基地,先后担任空军地勤人员和飞行员。

他1923年退役后,先后从事多种不同的职业。1925年开始写作,第一部作品就是以飞行为题材的。

1926年,圣埃克絮佩里进入拉泰科埃尔航空公司,担任法国图卢兹至塞内加尔达喀尔航空邮班的飞行员,继而被派往摩洛哥担任航线中途站站长。在此期间,出版小说《南方邮件》(1929)。后来他随同梅尔莫兹、吉约姆等老资格的飞行员前往南美洲开辟新航线。1931年出版小说《夜航》,从此他在文学上的名声就大起来了。

1935年,拉泰科埃尔公司倒闭。圣埃克絮佩里随公司人员并入新成立的法国航空公司后,曾尝试打破巴黎至西贡的飞行时间记录,但没有成功。1938年在重建纽约至火地岛航线途中身受重伤,于纽约治疗多月后才逐渐康复。出版《人类的大地》(1939)。

第二次世界大战期间他加入法国空军。得悉贝当政府签订屈辱的停战协定后,辗转去纽约开始流亡生活。在这期间,他写出了《空军飞行员》、《给一个人质的信》、《小王子》(1943)等作品。1944年重返同盟国地中海空军部队,因明显超龄,没有被列入飞行员编制。但他坚决要求驾机上天,经司令部特许终于如愿。1944年的7月31日,他从科西嘉岛的博尔戈出发,只身前往里昂地区执行侦察任务。飞机驶上湛蓝的天空,就此再也没有回来。

《小王子》是一部儿童文学作品,也是一部写给成年人看的童话,用圣埃克絮佩里自己的话来说,是写给“还是孩子时”的那个大人看的文学作品。整部小说充满诗意的忧郁、淡淡的哀愁,用明白如话的语言写出了引人深思的哲理和令人感动的韵味。这种韵味,具体说来,就是简单的形式和深刻的内涵的相契合。整部童话,文字很干净,甚至纯净,形式很简洁,甚至简单。因此,这部童话的译文,也应该是明白如话的。

不过要做到这一点,并不容易。举个例子来说,第二十一章里狐狸提出了一个很重要的(后来反复出现的)概念,法文中用的是apprivoiser,这个词我先是译作“驯养”,但放在上下文中间,似乎总觉得有些突兀,所以心里一直在犹豫。后来有位朋友看了初稿,对这个词提出意见,我受他的启发,决定改用“跟\ldots{}\ldots{}处熟”的译法。这个译法未必理想,但我们最终还是没能找到更满意的译法。暂且,就是它吧。

所有的大人起先都是孩子------但愿我们都能记得这一点。

\rightaligned{\rm\it 译者 2000年11月}

\reference[part0031.html]{}

\stopsubject

\startsubject[title={第二版译序},reference={part0031.html_b002}]

《小王子》在西方国家是本家喻户晓的书。很多年前,我在法国进修数学的时候,买了这本漂亮的小书,书里的彩色插图是作者自己画的。后来我还买了钱拉·菲利普(他是我最喜欢的法国演员)和一个声音银铃般清脆的孩子朗读的录音带。

这是一本非常好的书。虽然我们把它叫作童话,其实它是给大人看的。童话中能像《小王子》这么打动人心的,想起来还真不多呢。我印象很深的,还有一本《夏洛的网》,其中的主人公是蜘蛛和猪。看了书,我很感动,从此以后觉得这两种动物挺可爱了。

翻译《小王子》,比想象的要难。这次趁译本出第二版的机会,我对译文做了修改、打磨。谢谢张文江和其他朋友,他们给了我很多帮助。张文江在电话里把他的想法告诉我,帮我一起磨。我俩煲的电话粥,时间加起来不止十小时。

书中有个词,原文是apprivoiser,相当于英文的tame。我一开始译成“跟\ldots{}\ldots{}处熟”,重新印刷时改成“跟\ldots{}\ldots{}要好”。但这次再版,我又改成了“驯养”。这样改,我有一个很认真的理由:这个词“确实不是孩子的常用词”------我的一个法国朋友这样告诉我,法语是他的母语。我还有另外一个理由:“跟\ldots{}\ldots{}要好”(它比“跟\ldots{}\ldots{}处熟”自然)虽然明白易懂,但缺乏哲理性,没有力度。而apprevoiser在原书中是表现出哲理性和力度的(狐狸在书中以智者的形象出现)。我的第三个理由是:译作“跟\ldots{}\ldots{}要好”,当时就并不满意。后来跟许多朋友讨论过。其中有个大人,叫王安忆,她劝我“两害相权取其轻”。还有个小男孩叫徐振,年纪大概跟小王子差不多,他告诉我“驯养”的意思他懂。我听了他们的话,又想了半天,最后用了“驯养”。倘若所有这些理由加在一起还不够,那我愿意把这个词的译法当作一个open
question(有待解决的问题),请大家有以教我。

\rightaligned{\rm\it 译者 2002年4月}

\stopsubject

\stoptitle%
