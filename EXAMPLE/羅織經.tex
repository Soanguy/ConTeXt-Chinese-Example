%\part{罗织经}
\usemodule[memos]

\let\transpar\footnote

\chapter{}

历史上最著名的酷吏,请君入瓮的发明者来俊臣所著;乃中国几千年文明史中孽生出来的集邪恶智慧之大成的诡计全书;是一部酷吏赤裸裸施酷行恶的告白。

整人专家酷吏周兴临死之际,看过《罗织经》,自叹弗如,竟甘愿受死;一代人杰宰相狄仁杰阅罢《罗织经》,冷汗迭出,却不敢喊冤;雄才女皇武则天面对《罗织经》,叹道:“如此机心,朕未必过也。”杀机遂生。

\chapter{阅人卷}

人之情多矫,世之俗多伪,岂可信乎?子曰:『巧言、令色、足恭,左丘明耻之,丘亦耻之。』耻其匿怨而友人也。\transpar{人们的情感许多是做出来的,世间的习俗许多是虚假的怎么可以相信呢?孔子说:『甜言蜜语、和颜悦色、毕恭毕敬,左丘明认为可耻,我也认为可耻。』可耻的是他们心中藏着怨恨,表面却与人要好的虚伪行径。}

人者多欲,其性尚私。成事享其功,败事委其过,且圣人弗能逾者,概人之本然也。\transpar{人的欲望是多种多样的,人的本性是自私的。事情成功了便享受功劳,事情失败了便推托过错,圣人尚且不能超越这一点,这大概是人的本性所决定的吧。}

多欲则贪,尚私则枉,其罪遂生。民之畏惩,吏之惧祸,或以敛行;但有机变,孰难料也。\transpar{欲望多了就会起贪心,极端自私就会有偏差,罪恶从此便产生了。老百姓害怕惩罚,官吏恐遭祸患,不得不收敛自己的行为;一旦有了机缘变故,谁都无法预料了。}

为害常因不察,致祸归于不忍。桓公溺臣,身死实哀;夫差存越,终丧其吴。亲无过父子,然广逆恒有;恩莫逾君臣,则莽奸弗绝。是以人心多诈,不可视其表;世事寡情,善者终无功。信人莫若信己,防人毋存幸念。此道不修,夫庸为智者乎?\transpar{人们受害常常是因为对人没有仔细的察验,人们遭受祸患往往是由于对人心慈手软。齐桓公过份相信他的臣子,以致死亡实在让人哀痛。吴王夫差没有吞并越国,最后却导致吴国的灭亡。关系亲密没有超过父子的,可是像杨广那样的逆子却总是存在;施以恩德没有超过君对臣的,但是像王莽那样的奸臣起却从未断绝。因此说人的内心隐含着太多的欺骗,不能光看他的外表。世上的事缺少情爱,做好事的人最后却得不到功劳。相信别人不如相信自己,防范别人不要心存侥幸。这种技艺学习,难道还能成为一个有智能的人吗?}

\chapter{事上卷}

为上者疑,为下者惧。上下背德,祸必兴焉。\transpar{上司的疑心重,下属的恐惧就多。上司和下属的心意不一致,祸事便由此产生了。}

上者骄,安其心以顺。上者懮,去其患以忠。顺不避媚,忠不忌曲,虽为人诟亦不可少为也。上所予,自可取,生死于人,安能逆乎?是以智者善窥上意,愚者固持己见,福祸相异,咸于此耳。\transpar{高高在上的人骄傲,顺从他可使其心安。高高在上的人懮虑,忠于他可使其免除懮患。顺从不要回避献媚,忠心不要忌讳无理,虽然遭人诋毁也不能少做。上司能给你什么,自然能拿回什么,生死都控于人手,怎么能违背他们呢?因此有智能的人擅长暗中猜度上司的心意,愚蠢的人只坚持自己的见解,他们福祸不同,都是源于这个原因。}

人主莫喜强臣,臣下戒怀妄念。臣强则死,念妄则亡。周公尚畏焉,况他人乎?\transpar{当主子的没有喜欢手下的人势力过于强大,当臣子的要戒除心中存有的非分之想。臣子权势过大会招致死祸,想法荒谬会导致灭亡。周公姬旦尚且惧怕这些,何况是其它人呢?}

上无不智,臣无至贤。功归上,罪归己。戒惕弗弃,智勇弗显。虽至亲亦忍绝,纵为恶亦不让。诚如是也,非徒上宠,而又宠无衰矣。\transpar{上司没有不聪明的,下属绝无最有德行的。功劳让给上司,罪过留给自己。戒备警惕之心不要丢失,智能勇力不要显露。虽然是最亲近的人也要忍心断绝,纵然是干邪恶的事也不躲避。如果真的做到这样,不但上司会宠爱有加,而且宠信不会衰减。}

\chapter{治下卷}

甘居人下者鲜。御之失谋,非犯,则篡耳。\transpar{自愿处于下属的人很少。上级对下级的管理如果没有计谋,不是下级抵触上级,就是下级夺取上级的权力。}

上无威,下生乱。威成于礼,恃以刑,失之纵。私勿与人,谋必辟。幸非一人,专固害。机心信隐,交接靡密,庶下者知威而畏也。\transpar{上司没有威严,下属就会闹出祸事。威严从礼仪中树立,依赖于刑罚,放任它就会丧失。秘密的事不要让人参与,参与谋划的人一定要清除。宠信不要固定在一个人的身上,让一个人专权一定会带来祸害。心思一定要隐藏起来,与人交往不能过分亲密,希望下属由此感知上司的威严而生敬畏。}

下附上以成志,上恃下以成名。下有所求,其心必进,迁之宜缓,速则满矣。上有所欲,其神若亲,礼下勿辞,拒者无助矣。\transpar{下属依附上司纔能成就志向,上司依靠纔属纔能取得功名。下属有贪求的东西,他的心自然会要求上进,提升他应该慢慢的来,太快他就满足了。上司有想使用的人,他的神态要亲切,以礼相待下属不要推辞,不这样做就没有人协助他了。}

人有所好,以好诱之无不取,人有所惧,以惧迫之无不纳。纔可用者,非大害而隐忍。其不可制,果大材而亦诛。赏勿吝,以坠其志。罚适时,以警其心。恩威同施,纔德相较,苟无功,得无天耶?\transpar{人有喜好的东西,用喜好的东西引诱他没有收伏不了的。人有惧怕的东西,用惧怕的东西逼迫他没有不接受的。有纔能可以使用的人,没有大的害处要暗中容忍。其人不能驯服,确实是纔能出众的也要诛杀。赏赐不要吝惜,用此消磨他们的意志。惩罚要适合时宜,以此让他的心得到告诫。恩惠和威力一起施行,纔能和品德互相比较,如果这样做还没有成效,莫非这就是天意吧?}

\chapter{控权卷}

权者,人莫离也。取之非易,守之尤艰;智不足弗得,谋有失竟患,死生事也。\transpar{权力,是人们不可以缺少的。获取权力不容易,保住权力更加艰难;智能欠缺的人不能得到,谋略不当的人最终却能带来祸患,这是关系生死的大事。}

假天用事,名之顺也。自绝于天,敌之罪也。民有其愚,权有其智。德之不昭,人所难附焉。\transpar{借用天意行事,名义上纔适合正道。逆天而行,自作自受,这是敌人的罪名。让老百姓愚昧无知,这是掌权者的聪明之处。恩惠不显示出来,人们就很难依从了。}

乱世用能,平则去患。盛事惟忠,庸则自从。名可易,实必争;名实悖之,权之丧矣。嗜权逾命者,莫敢不为;权之弗让也,其求乃极。机为要,无机自毁;事可绝,人伦亦灭。利禄为羁,去其实害;赏以虚名,收其本心。若此为之,权无不得,亦无失也。\transpar{混乱动荡的时代要使用有能力的人,天下平定了要铲除他们以绝后患。大治时期只使用忠于自己的人,平庸无纔的人最易掌握和归顺。名称可以改变,实权必须力争;名称和实权完全相反,权力就丧失了。酷好权力超过他性命的人,是没有什么不敢做;权力没有主动让给别人的,所以争夺它的方法无所不用。时机十分重要,时机不当就会自取灭亡;事情可以做绝,尊卑长幼也能狠心灭杀。用钱财爵禄来拘束他们,以消除他们可能造成的实际危害;用虚假的名位来赏赐他们,以收买他们的人心。如果这样行事,什么权力都可以获得,也不会失去。}

\chapter{制敌卷}

人皆有敌也。敌者,利害相冲,死生弗容;未察之无以辨友,非制之无以成业。此大害也,必绝之。\transpar{人都有敌人的。敌人,是与他有利害冲突,生死不能兼容的人。不能认清敌人就无法分辨朋友,不能制伏敌人就不能成就事业,这是最大的祸害,一定要根除它。}

君子敌小人,亦小人也。小人友君子,亦君子也。名为虚,智者不计毁誉;利为上,愚者惟求良善。\transpar{君子和小人为敌,也就变成小人了。小人和君子友善,也就变成君子了。名声是虚的,有智能的人不会计较别人的毁谤和称赞;利益是至高无上的愚蠢的人纔只是求取好的善名。}

众之敌,未可谓吾敌;上之敌,虽吾友亦敌也。亲之故,不可道吾亲;刑之故,向吾亲亦弃也。惑敌于不觉,待时也。制敌于未动,先机也。构敌于为乱,不赦也。害敌于淫邪,不耻也。敌之大,无过不知;祸之烈,友敌为甚。使视人若寇,待亲如疏,接友逾仇,纵人之恶余,而避其害,何损焉?\transpar{人们共同的敌人,不能说一定是我的敌人;上司的敌人,虽然是我的朋友也要与他为敌。亲戚的缘故,不能说就是我该亲近的人;刑罚的缘故,如果是我的亲人也要舍弃。在不知不觉中迷惑敌人,以等待时机。在敌人没有行动的时候制伏他,这就是抢先占有有利时机。在犯上作乱上构陷敌人,这是不能赦免的罪名。在淫秽邪恶之事上加害敌人,这最能让人鄙视他。最大的敌人,是没有比不知道谁是敌人更大的了;最深的祸害,以和敌人友善最为严重。假如把天下人看得像强盗一样,对待亲人像陌生人一样,交接朋友超过了对仇人的态度,纵然人们厌恶我,却能躲避祸害,以有什么损失呢?}

\chapter{固荣卷}

荣宠有初,鲜有终者;吉凶无常,智者少祸。荣宠非命,谋之而后善;吉凶择人,慎之方消愆。\transpar{显达和宠幸有开始的时候,能保持到最后的就很少了;吉利和凶险没有不变的,有智能的人纔能减少祸事。显达和宠幸不是命里就有的,先有谋划后纔有成;吉利和凶险是选择人的,谨慎小心纔能消灾免祸。}

君命无违,荣之本也,智者舍身亦存续。后不乏人,荣之方久,贤者自苦亦惠嗣。官无定主,百变以悦其君。君有幸臣,无由亦须结纳。人孰无亲,罪人慎察其宗。人有贤愚,任人勿求过己。\transpar{君主的命令不要违抗,这是显达的根本,有智能的人宁肯牺牲自己也要让显达延续下去。后代不缺乏人材,显达纔可持久,贤明的人情愿自己吃苦也要惠及后人。官位没有固定的主人,用机智多变取悦他的君主。君主都有宠幸的臣子,没有什么原由也必须和他们结交往来。人都有三亲六故,惩罚人的时候一定要仔细审察他的家族。人有贤明和愚蠢之别,任用人不要要求他们的纔能高过自己。}

荣所众羡,亦引众怨。示上以足,示下以惠,怨自削减。大仇必去,小人勿轻,祸不可伏。喜怒无踪,慎思及远,人所难图焉。\transpar{显达为众人所羡慕,也能引发众人的怨恨。对上司要表示心满意足,对手下要施以恩惠,怨恨自然就会减少了。大的仇人一定要铲除,无耻小人不要轻视,祸患就不能隐藏。高兴和愤怒的心情不露踪迹,谨慎思考放眼远处,人们就很难图谋他了。}

\chapter{保身卷}

世之道,人不自害而人害也;人之道,人不恕己而自恕也。\transpar{世间的道理,人们不伤害自己却遭到别人的伤害;做人的道理,别人不原谅自己而自己却能原谅。}

君子惜名,小人爱身。好名羁行,重利无亏。名德不昭,毁谤无损其身;义仁莫名,奸邪不以为患。阳以赞人,置其难堪而不觉;阴以行私,攻其讳处而自存。\transpar{君子爱惜名誉,小人爱护自己。喜好名誉就会束缚人的行为,重视利益就不会吃亏。名望和德行不显示,诽谤就不能损害他本身的清誉;义气和仁德不显露,奸诈邪恶的人就不会把他视为祸患。表面上赞美别人,让他难以忍受却不知真意;背地里为达私利,攻击他最忌讳的地方而保存自己。}

庶人莫与官争,贵人不结人怨。弱则保命,不可作强;强则敛翼,休求尽善。罪己宜苛,人怜不致大害。责人勿厉,小惠或有大得。\transpar{老百姓不要与官府争斗,富贵的人不要轻易和人结下怨仇。身为弱者要保全性命,不能逞强显能;身为强者要收敛羽翼,不可求取完美无缺。责备自己应该苛刻,使人怜悯就不会招致大的祸害。责罚他人不要过于严厉,小的恩惠有时能带来大的收获。}

恶无定议,莫以恶为恶者显;善无定评,勿以善为善者安。自怜人怜,自弃人弃。心无滞碍,害不侵矣。\transpar{恶没有固定的说法,不把恶当作恶的显达;善没有固定的评判,不把善视为善的人平安。自己怜惜自己别人纔会怜惜,自己厌弃自己别人自会厌弃。思想没有停滞阻碍,祸害就无法侵犯了。}

\chapter{察奸卷}

奸不自招,忠不自辩。奸者祸国,忠者祸身。\transpar{奸臣不会自己招认,忠臣不能自己辩解。奸臣损害国家,忠臣损害自身。}

无智无以成奸,其智阴也。有善无以为奸,其知存也。\transpar{没有智谋不能成为奸臣,他们的智谋都是阴险的。心存良善不会成为奸臣,他们的良知没有丧失。}

智不逾奸,伐之莫胜;知不至大,奸者难拒。忠奸堪易也。上所用者,奸亦为忠;上所弃者,忠亦为奸。\transpar{智谋不超过奸臣,讨伐他就不能获胜;良知不深远广大,对奸臣就难以抗拒。忠臣和奸臣是可以变换的。君主任用信任的人,虽然是奸臣也被看做忠臣;君主拋弃不用的人,即使是忠臣也被视为奸臣。}

势变而人非,时迁而奸异,其名难恃,惟上堪恃耳。好恶生奸也。人之敌,非奸亦奸;人之友,其奸亦忠。\transpar{时势变了人就不同,时间变了奸臣就有分别,忠奸的名称难以依赖,只有君主纔可作为依仗。喜欢和厌恶产生奸臣。人们的敌人,不是奸臣的也被视为奸臣;人们的朋友,是奸臣的也被视为忠臣。}

道同方获其利,道异惟受其害。奸有益,人皆可为奸;忠致祸,人难为忠。奸众而忠寡,世之实也;言忠而恶奸,世之表也。\transpar{道义相同纔能获得利益,道义不同只有得到灾害。当奸臣有好处,人们都可以成为奸臣;当忠臣招致祸患,人们就很难做忠臣了。奸臣多而忠臣少,这是世间真实的状况;说自己是忠臣而厌恶奸臣,这是世间表面的现象。}

惟上惟己,去表求实,奸者自见矣。\transpar{只献媚君主就是为了自己,去除表面探求实质,奸臣自然就会显现出来了。}

\chapter{谋划卷}

上不谋臣,下或不治;下不谋上,其身难晋;臣不谋僚,敌者勿去。官无恒友,祸存斯虚,势之所然,智者弗怠焉。料敌以远,须谋于今;去贼以尽,其谋无忌。欺君为大,加诸罪无可免;枉法不容,纵其为祸方惩。\transpar{君主不用计谋统御臣子,有的下属就无法治理;下属不用计谋对君主,他们自身的官职就难以晋升;官员不用计谋对付同僚,他的敌人不能铲除。官场上没有永远的朋友,祸患常在片刻之间,这是形势的必然,有智能的人对此不能松懈。预料敌人能达到远处,必须要在今天谋划;铲除贼人要达到全歼,他的谋划就不能有所忌讳。欺骗君主是大罪,把这个罪名强加在别人身上他就不能幸免;破坏法律不能宽容,放纵他以致出生祸乱纔加以惩罚。}

上谋臣以势,势不济者以术。下谋上以术,术有穷者以力。臣谋以智,智无及者以害。事贵密焉,不密祸己;行贵速焉,缓则人先。其功反罪,弥消其根;其言设缪,益增人厌。行之不辍,不亦无敌乎?\transpar{君主凭借权势谋划臣子,势力衰弱的时候要依靠权术。下属依靠权术谋划君主,权术穷尽的时候就凭借实力。臣子用智计谋划同僚,智计达不到的时候就用伤害。事情贵在保守秘密,不能保守秘密,就祸及自身;行动贵在迅速快捷,缓慢拖拉就让别人占了先机。设定一个荒谬的说法诬指是他说的,这最能增加人们的厌恶。谋划行为不停止,不是没有敌手了吗?}

\chapter{问罪卷}

法之善恶,莫以文也,乃其行焉;刑之本哉,非罚罪也,乃明罪焉。\transpar{法律的好坏,不在条文本身,而是它的执行;刑罚的根本,不在如何处罚犯罪,而是如何确定犯罪。}

人皆可罪,罪人须定其人。罪不自招,密而举之则显。上不容罪,无谕则待,有谕则逮。人辩乃常,审之勿悯,刑之非轻,无不招也。或以拒死,畏罪释耳。人无不党,罪一人可举其众;供必无缺,善修之毋违其真。事至此也,罪可成矣。\transpar{人都是可以定罪的,加罪于人必须先确定对象。罪行不会自动暴露,密告并检举他就会让罪行显现。君主不会容忍犯罪,没有谕旨就耐心等待,有谕旨就马上逮捕。人们自辨无罪是正常的,审讯他们不要心存怜悯,刑罚他们不能出手轻微,这样做他们就没有不招认的。有的人因为拒不认罪被责打致死,这种情况可用畏罪自杀来解释说明。人没有不结党营私的,给一人定罪便可揭发出他的同伙;供状必须没有破绽,把被告供状编撰修补使之不违反真实。事情做到这样,罪案就可以成立了。}

人异而心异,择其弱者以攻之,其神必溃。\transpar{人不同他们的思想就有差异,选择他们的薄弱之处加以攻击,他们的精神就会崩溃。}

身同而惧同,以其至畏而刑之,其人固屈。怜不可存,怜人者无证其忠。友宜重惩,援友者惟其害。\transpar{人的身体相同害怕责罚也相同,用他最畏惧的东西给他动刑,他就一定会屈服。不可以存有怜惜,怜惜别人的人并不能以此证明他的忠正。朋友应该从重惩处,帮助朋友的人只能给他自己招来祸害。}

罪人或免人罪,难为亦为也。\transpar{加罪于人或许能避免被人加罪,此事虽不容易也要勉为其难了。}

\chapter{刑罚卷}

致人于死,莫逾构其反也;诱人以服,非刑之无得焉。刑有术,罚尚变,无所不施,人皆授首矣。\transpar{让人达到死亡的境地,没有比构陷他谋反更能奏效的事了;诱导人们做到服从,不刑罚他们就达不到目的。刑讯是讲究方法的,责罚贵在有所变化,施行的手段没有限制,人们就都会伏法认罪了。}

智者畏祸,愚者惧刑;言以诛人,刑之极也。明者识时,顽者辩理;势以待人,罚之肇也。\transpar{有智能的人畏惧祸事,愚笨的人害怕刑罚;用言语来杀人,这是刑罚中最高明的。聪明的人能认清当前的客观形势,愚顽的人却一味辩说有理与无理;按照形势的要求对待他人,这是责罚人的出发点。}

死之能受,痛之难忍,刑人取其不堪。士不耐辱,人患株亲,罚人伐其不甘。人不言罪,加其罪逾彼;证不可得,伪其证率真。刑有不及,陷无不至;不患罪无名,患上不疑也。\transpar{死亡可以接受,痛苦难以忍耐,给人动刑选取他们不能忍受的。读书人忍耐不了屈辱,人们都担心株连自己的亲人,惩罚人要攻取他们不情愿处。人们不承认有罪,就此加害他的罪名比原来的罪名还大;证据不能得到,伪造证据大概像真的一样。刑罚有做不到的地方,诬陷却什么都可以做到;不要担心给人加罪没有名义,只担心君主没有猜疑之心。}

人刑者非人也,罚人者非罚也。非人乃贱,非罚乃贵。贱则鱼肉,贵则生死。人之取舍,无乃得此乎?\transpar{被人用刑的人会受到非人的待遇,惩罚别人的人自己也会避免惩罚。遭受非人的待遇就低贱,不受惩罚就高贵。低贱的人就任人宰割,高贵的人就主宰别人的生死命运。人们的选择态度和行为,恐怕是源于此吧?}

\chapter{瓜蔓卷}

事不至大,无以惊人。案不及众,功之匪显。上以求安,下以邀宠,其冤固有,未可免也。\transpar{事情不是很大,就不能让人震惊。案件不是牵扯人多,功劳就不能显现。君主用它来求取安定,臣子用它来邀功取宠,这里的冤情一定会有,却是不可能避免的。}

荣以荣人者荣,祸以祸人者祸。荣非己莫恃,祸惟他勿纵。罪无实者,他罪可代;恶无彰者,人恶以附。心之患者,置敌一党;情之怨者,陷其奸邪。\transpar{真正显达是能让他人也显达的显达,真正的祸患是能使他人也致祸的祸患。不是自己挣来的显达不要倚仗,只要是他人的祸患就不要放过。罪名没有实证,用其它的罪名来替代;恶行没有显露,用他人的恶行来依附。心腹的祸害,把他诬指为是敌人的同伙;情感上怨恨的人,陷害他是奸诈邪恶的小人。}

官之友,民之敌;亲之友,仇之敌,敌者无常也。荣之友,败之敌;贱之友,贵之敌,友者有时也。是以权不可废,废则失本,情不可滥,滥则人忌;人不可密,密则疑生;心不可托,托则祸伏。智者不招己害,能者寻隙求功。饵之以逮,事无悖矣。\transpar{官吏的朋友,在以官吏为敌的百姓眼里便是帮凶;亲人的朋友,在和亲人有仇的仇人眼中也成了敌人,所以说敌人是变化不定的。显达时的朋友,败落时就是敌人;贫贱时的朋友,富贵时就是敌人,所以说朋友是暂时的。因此说权力是不可废弃的,废弃了就失掉了根本;同情心是不能随便施予的,太随便了就会招人忌恨;与人交往不能过于亲密,太亲密就会产生疑虑;心里话不能说出来,毫无保留就潜藏着祸患。有智能的人不会为自己招来祸害,有能力的人总是寻找别人的漏洞以求取功劳。引诱他们上钩再据此把他们逮捕,事情就没有悖理之说了。}
