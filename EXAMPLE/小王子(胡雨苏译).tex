\usemodule[memos][paperdesign=kindle,hdrstyle=fctext]
\definelayer[lay:title][x=0mm, y=0mm,width=\paperwidth, height=\paperheight]
\setlayer[lay:title][hoffset=0cm, voffset=0cm]{\externalfigure[imgs/小王子(胡雨蘇譯)/calibre_cover.jpg][height=\paperheight]}
 \startsetups setups:title
     \setupbackgrounds[page][background=lay:title]
 \stopsetups
\setupmakeup[page][before=\setups{setups:title},pagestate=start]
\startmakeup[page]
\stopmakeup

\preface{序言 法兰西玫瑰}

\midaligned{\rm\it周国平}

像《小王子》这样的书,本来是不需要有一篇序言的,不但不需要,而且不可能有。莫洛亚曾经表示,他不会试图去解释《小王子》中的理,就像人们不对一座大教堂或布满星斗的天穹进行解释一样。

我也不会无知和狂妄到要给天穹写序,所能做仅是借这个彩色插图全译本出版之机,再一次表达我对圣爱克苏佩里的这部天才之作的崇拜和热爱。我说《小王子》是一部天之作,说的完全是我自己的真心感觉,与文学专家们的评论无关。我甚至要说,它是一个奇迹。世上只有极少数作品,如此精美又如此质朴,如此深刻又如此平易近人,从内容到形式都几近于完美,却不落丝毫斧凿痕迹,宛若一块浑然天成的美玉。

令我感到不可思议的一件事是,一个人怎么能够写出这样美妙的作品。令我感到不可思议的另一件事是,一个人翻开这样一本书,怎么会不被它吸引和感动。我自己许多次翻开它时都觉得新如初,就好像第一次翻开它时觉得一见如故一样。每次读,免不了的是常常含着泪花微笑,在惊喜的同时又感到辛酸。我知道许多读者有过和我相似的感受,我还相信这样的感受将会在更多的读者身上得到印证。

按照通常的归类,《小王子》被称作哲理童话。你们千万不要望文生义,设想它是一本给孩子们讲哲学道理的书。一般来说,童话是大人讲给孩子听的故事。这本书诚然也非常适合于孩子们阅读,但同时也是写给某一些成人看的。用作者的话来,它是献给那些曾经是孩子并且记得这一点的大人的。我觉得比较准确的定位是,它是一个始终葆有童心的大人对孩子们、也对与他性情相通的大人们说的知心话,他向他们讲述了对于成人世界的观感和自己身处其中的孤独。\crlf
的确,作者的讲述饱含哲理,但他的哲理决非抽象的观念和教条,所以我们无法将其归纳为一些简明的句子而又不使之受到损害。譬如说,我们或许可以把全书的中心思想归为一种人生信念,便是要像孩子们那样凭真性情直接生活在本质之中,而不要像许多成人那样为权力、虚荣、占有、职守、学问之类表面的东西无事空忙。可是,倘若你不是跟随小王子到各个星球上去访问一下那个命令太阳在日落时下降的国王,那个请求小王子为他不断鼓掌然后不断脱帽致礼的虚荣迷,那个热衷于统计星星的数目并将之锁进抽屉里的商人,那个从不出门旅行的地理家,你怎么能够领会孩子和作者眼中功名利禄的可笑呢?倘若你不是亲耳听见作者谈论大人们时的语气,例如,他谈到大人们热爱数目字,如果你对人们说起一座砖房的颜色、窗台上的花、屋顶上的鸽子,他们就无动于衷,如果你说这座房子值十万法郎,他们就会叫起来:“多么漂亮的房子啊!”他还告诉孩子们,大人们就是这样的,孩子们对他们应该宽宏大量。你不亲自读这些,怎么能够体会那讽刺中的无奈,无奈中的悲凉呢?

我还可以从书中摘录一些精辟的句子,例如:“正因为你在你的玫瑰身上花费了时间,这才使她变得如此名贵。”“使沙漠变得这样美丽的,是它在什么地方隐藏着一眼井。”可是,这样的句子摘不胜摘,而要使它们真正属于你,你就必须自己去摘取。且这本小书当作一朵玫瑰,在她身上花费你的时间,且把它做一片沙漠,在它里面寻找你的井吧。我相信,只要你把它翻开来,读下去,它一定会对你也变得名贵而美丽。

圣爱克苏佩里一生有两大爱好:飞行和写作。他在写作中品味人间的孤独,在飞行中享受四千米高空的孤独。《小王子》是他生前出版的最后一本书,出版一年后,他在一次驾机执行任务时一去不复返了。没有人知道他去了哪里,地球上再也没有发现他的那架飞机的残骸。我常常觉得,他一定是到小王子所住的那个小小的星球上去了,他其实就是小王子。

有一年夏天,我在巴黎参观先贤祠。先贤祠的宽敞正厅里只有两座坟墓,分别埋葬着法兰西精神之父伏尔泰和卢梭,唯一的例外是有一面巨柱上铭刻着圣爱克苏佩里的名字。站在那柱子面前,我为法国人对这个大孩子的异乎寻常的尊敬而感到意外和欣慰。当我心想,圣爱克苏佩里诞生在法国并非偶然,一个懂得《小王子》作者之伟大民族有多么可爱。我还想,应该把《小王子》译成各种文字,印行几亿册,让世界上每个孩子和每个尚可挽救的大人都读到,这样世界一定也会变得可爱起来。

\midaligned{\rm\it 献给莱昂·韦尔特}

请小朋友们原谅,我把这本书献给一个大人。我有一个重要的理由:这个大人是我在人世间最要好的朋友。我还有另外一个理由:这个大人什么都能看懂,甚至能看懂给孩子们写的书。我的第三个理由是:这个大人住在法国。他在挨饿受冻,很需要得到安慰。如果这些理由还不充足的话,我愿把这本书献给曾做过孩子的这个大人。所有的大人都曾经是个孩子。(可惜,他们当中记得这一点的并不多。)为此,我把献词改为:

\midaligned{\rm\it 献给童年时代的莱昂·韦尔特}

\title{第一章}

我六岁的时候,看到过一本写原始森林的书,名叫《真实的故事》,书中有一幅精彩的插图。上面画着一条蟒蛇正在吞吃一只猛兽。照原样画下来就是这个样子。

{\startalignment[center]
 \placefigeasy[][imgs/小王子(胡雨蘇譯)/00018.jpg][maxwidth=\textwidth,location={middle,none}]{calibre_10}
 \stopalignment}

书中是这样写的:“这些蟒蛇不加咀嚼,就把它们捕获的食物整个吞到肚里,之后就不再能动弹了。它们睡上六个月,来消化吞下的食物。”

那时,我对丛林中的惊险事情想得很多,于是,我用彩色铅笔画出了我的第一幅图画。我的第一张图画是这样的。

{\startalignment[center]
 \placefigeasy[][imgs/小王子(胡雨蘇譯)/00021.jpg][maxwidth=\textwidth,location={middle,none}]{calibre_11}
 \stopalignment}

我把我的杰作拿给大人们看,并问他们看了害怕不害怕。

他们回答我说:“一顶帽子有什么好怕的?”

我画的不是一顶帽子,是一条蟒蛇正在消化一头大象。于是我又把蟒蛇肚子里的情形画了出来,好让大人们能够看懂。这些大人啊,总得要别人给作解释。下面就是我的第二张图画。

{\startalignment[center]
 \placefigeasy[][imgs/小王子(胡雨蘇譯)/00025.jpg][maxwidth=\textwidth,location={middle,none}]{calibre_12}
 \stopalignment}

大人们劝我,还是把这些剖开的,或者完整的蟒蛇画丢到一边去吧,多关心点地理、历史、算术和语法为好。就这样,在我六岁那年,我只好放弃了美好的画家生涯。我的第一张和第二张图画都不成功,使得我灰心丧气了。大人们自己总是什么也弄不明白,还得要孩子们给他们翻来覆去地作解释,真叫烦死人。

后来,我不得不选择另一种职业,学会了驾驶飞机。我几乎飞遍了整个世界。的确,地理知识可真帮了我的大忙。我一眼就能把中国和美国的亚利桑那州分辨出来。要是夜里迷航的话,地理知识是非常有用的。

就这样,在我的生活经历中,我跟许多正经的人有过频繁的接触。我在大人们的圈子里生活了很长时间,仔细地观察过他们,但这并没有怎么改变我对他们的看法。

每当我遇到一个头脑稍为清醒的大人,我就拿出我一直保存着的第一张画试他一试。我想知道,他是否真能看懂。可他总是回答说:“这是一顶帽子。”于是我就再也不跟他谈什么蟒蛇啊,原始森林啊,星星啊,而是说些他能听得懂的事情。我跟他谈桥牌啊,谈打高尔夫球呀,聊聊政治和领带。这么一来,这个大人倒挺高兴,因为他结识了一个通情达理的人。

\title{第二章}

我就这样孤独地生活着,不曾和任何人真正地谈过心,一直到六年前在撒哈拉沙漠发生了那次事故,这种生活才告结束。我的飞机引擎里有个东西坏了。当时我身边既没有机械师,也没有乘客,我就试图独自完成这个困难的修理工作。对我来说,这可是个生死攸关的问题,我带的饮水只够用八天了。

第一夜,我就睡在那远离人间烟火十万八千里的荒漠里。我比汪洋大海中的一叶扁舟上的遇难者更感到孤单无援。而次日黎明时分,当一个奇怪的小声音把我唤醒的时候,你们可以想象,我当时是多么吃惊啊。那小声音说道:

“劳驾\ldots{}\ldots{}请给我画一只绵羊吧!”

“嗯!”

“给我画一只绵羊吧\ldots{}\ldots{}”

好像受到了惊雷的轰击,我腾地一下站了起来。我使劲揉了揉眼睛,仔细瞧看,我看见一个极不寻常的小家伙严肃地凝眸望着我。这是后来我给他画的一幅最好的肖像。当然啦,这幅肖像远远没有他本人那样光彩夺目。这可不是我的过错。我六岁那年,那些大人们断送了我的画家前程。我那时除了画过完整的和剖开的蟒蛇而外,以后再没有学过画。

{\startalignment[center]
 \placefigeasy[][imgs/小王子(胡雨蘇譯)/00030.jpg][maxwidth=\textwidth,location={middle,none}]{calibre_13}
 \stopalignment}

我惊愕地望着这位不速之客。请不要忘记,我当时是在远离人间十万八千里的荒漠之中。在我看来,这个小家伙不像是迷了路,他既无累得要死、饥渴得要命的神情,也毫无惧色。总之,他一点也不像一个人在荒无人烟的茫茫沙漠中迷路的孩子。好半天我才说出话来,我问他道:

“你\ldots{}\ldots{}.你在这儿干什么?”

他又郑重其事地轻声对我重复道:

“劳驾\ldots{}\ldots{}给我画一只绵羊吧\ldots{}\ldots{}”

当某一种神秘的东西把你镇住的时候,你怎敢不俯首听命呢?尽管是在千里之外的沙漠上和面临死亡的危险中,我的举动显得十分荒诞,不可思议,我还是从口袋里掏出了一张纸和一支钢笔。这时我才想起,我过去学习的主要是地理、历史、算术和语法,就有点不耐烦地对小家伙说,我不会画画。他却回答我说:

“没关系,给我画一只绵羊吧。”

因为我从来没有画过绵羊,就在我会画的两幅画中选了一张,就是那张囫囵吞象的蟒蛇图,给他重画了出来。可是小家伙的答话却使我十分惊异:

“不!不!我不要蟒蛇,它肚子里还有一头大象。蟒蛇这东西太危险,大象也太占地方。我那个地方很小,我需要一只绵羊。给我画一只吧。”

于是我就给他画了一只。

{\startalignment[center]
 \placefigeasy[][imgs/小王子(胡雨蘇譯)/00034.jpg][maxwidth=\textwidth,location={middle,none}]{calibre_14}
 \stopalignment}

他聚精会神地看着,随后又说:

“我不要,这只羊已经病得很厉害了。给我重新画一只吧。”

我又画了一张。

{\startalignment[center]
 \placefigeasy[][imgs/小王子(胡雨蘇譯)/00038.jpg][maxwidth=\textwidth,location={middle,none}]{calibre_15}
 \stopalignment}

我的这位朋友天真可爱地微笑起来,并且客气地拒绝道:

“你看,你画的分明不是一头小绵羊,它是一头公羊,还长着角呢\ldots{}\ldots{}”

于是我又重新画了一张。

{\startalignment[center]
 \placefigeasy[][imgs/小王子(胡雨蘇譯)/00042.jpg][maxwidth=\textwidth,location={middle,none}]{calibre_16}
 \stopalignment}

这幅画同前几幅一样,又被他拒绝了:

“这一只太老了。我要一只能活很久很久的小绵羊。”

我不耐烦了,因为我急着要拆卸我的飞机引擎。于是我就草草地画了这张画。

{\startalignment[center]
 \placefigeasy[][imgs/小王子(胡雨蘇譯)/00047.jpg][maxwidth=\textwidth,location={middle,none}]{calibre_17}
 \stopalignment}

我冲着他说道:

“这是一个箱子,你要的小绵羊就在里面。”

这时我却十分惊愕地看到,我的小审判官脸上闪现出喜悦的光彩。他说:

“这才是我想要的呢\ldots{}\ldots{}你说,这只羊要吃很多草吗?”

“你问这个干什么?”

“因为我那个地方很小\ldots{}\ldots{}”

“地方小,也足够喂养它的了。我给你画的是一只很小的小绵羊。”

他俯首看着这幅画儿说:

“并不那么小\ldots{}\ldots{}你瞧!它睡着了\ldots{}\ldots{}.”

就这样,我认识了小王子。

\title{第三章}

{\startalignment[center]
 \placefigeasy[][imgs/小王子(胡雨蘇譯)/00002.jpg][maxwidth=\textwidth,location={middle,none}]{calibre_18}
 \stopalignment}

我费了好长时间才弄清小王子是从哪里来的。他向我提出了许多问题,而对于我提出的问题,他却好像从来不想予以理睬。不过,他无意中吐露的一些话,使我逐渐地搞清了他的全部秘密。比如,当他第一次瞅见我的飞机时(我就不画出我的飞机了,因为这种画对我来说是太复杂了),他问我道:

“这是什么东西呀?”

“这不是东西,它会飞。这是一架飞机,是我的飞机。”

我当时很自豪地告诉他,我能在天空中飞行。于是他叫了起来:

“怎么!你是从天上掉下来的?”

“是的。”我谦逊地答道。

“啊!可真有意思\ldots{}\ldots{}”

此时小王子笑声朗朗,使我分外地恼火。我希望别人严肃地对待我的不幸。接着他又说道:

“那么,你也是从天上来的了!你是从哪个星球上来的?”

顷刻之间,我对有关小王子来历的秘密隐约发现了一点线索,于是,我就突然问道:

“这么说,你是从别的星球上来的了?”

可是他没有回答我的问题。他一面看着我的飞机,一面微微地点点头:

“可不是么,你乘坐这玩艺儿,不可能是从很远的地方来的\ldots{}\ldots{}”

说到这里,他便长时间地陷入沉思之中。然后,他从口袋里掏出了我画的小绵羊,看着他的这个宝贝出神。

你们设身处地想一想吧,这个关于“别的星球”的露了端倪的秘密使我心里多么好奇啊!所以,我要想方设法弄清他的来历。

“你是从哪里来的呀,我的小家伙?‘你的家'在什么地方呀?你想把我的小绵羊带到哪里去呢?”

他深思了一会儿,然后回答说:

“好在有你给我的那只箱子,夜里可以给小绵羊当房子。”

“那当然。要是你乖的话,我再给你画一条绳子,画一根小木桩,白天可以把羊拴起来。”

我的建议好像惹起了小王子的反感:

“把羊拴起来?亏你想得出来!”

“要是不拴起来,它就到处乱跑,那么就会跑丢的。”

我的朋友又是一阵爽朗的笑声:

“你要它跑到哪里去呀?”

“哪里都可以,一直往前跑\ldots{}\ldots{}”

这时,小王子郑重其事地指出:

“那倒没什么关系,反正我那里小得很。”

他仿佛有点忧伤,又补充了一句:

“一直往前跑,也不会跑出多远的\ldots{}\ldots{}”

{\startalignment[center]
 \placefigeasy[][imgs/小王子(胡雨蘇譯)/00011.jpg][maxwidth=\textwidth,location={middle,none}]{calibre_19}
 \stopalignment}

\title{第四章}

就这样,我了解到第二件非常重要的事情:他的老家所在的那个星球,仅比一座房子大一点点!

这倒并没有使我感到太奇怪。我知道,除了地球、木星、火星、金星这些早已命名的大行星以外,还有数百颗小行星,它们之中有的非常小,就是用望远镜观察也很难看到。当一个天文学家发现了其中的一颗,他就给它编上一个号码,例如把它称作“3251号小行星”。

{\startalignment[center]
 \placefigeasy[][imgs/小王子(胡雨蘇譯)/00015.jpg][maxwidth=\textwidth,location={middle,none}]{calibre_20}
 \stopalignment}

我有充分的理由相信,小王子来自B612号小行星。这颗小行星,只是在1909年被一位土耳其天文学家用望远镜看见过一次。

{\startalignment[center]
 \placefigeasy[][imgs/小王子(胡雨蘇譯)/00022.jpg][maxwidth=\textwidth,location={middle,none}]{calibre_21}
 \stopalignment}

当时他曾在一次国际天文学大会上对他的发现提出了重要的论证。但由于他身着土耳其民族服装,那里没有人相信他。大人们就是这样。

{\startalignment[center]
 \placefigeasy[][imgs/小王子(胡雨蘇譯)/00032.jpg][maxwidth=\textwidth,location={middle,none}]{calibre_22}
 \stopalignment}

幸好,为了挽回B612号小行星的声誉,土耳其的一个独裁者强令他的人民都穿上欧式服装,违者格杀勿论。1920年,那位天文学家身着非常漂亮的西服,再次论证了他的发现。这一次,所有的人都同意了他的看法。

我之所以如此详尽地向你们介绍B612号小行星,甚至把它的编号都告诉了你们,这都是由于大人们的缘故。那些大人们就爱数目字。当你对他们谈起你的一位新朋友时,他们从来不向你打听主要的情况。他们从来不问:“他说话的声音怎么样呀?他喜欢哪些游戏?他采集蝴蝶标本吗?”他们却问你:“他几岁了?有几个弟兄?他的体重是多少?他的父亲一个月挣多少钱呀?”他们以为只有这样才算了解你的新朋友了。如果你对他们说:“我看到一幢漂亮的粉红色的砖房,窗户上摆着绣球花,屋顶上落着成群的鸽子\ldots{}\ldots{}”他们怎么也不可能想象出这幢房子有多么漂亮。必须对他们说:“我看见一幢价值十万法郎的房子。”此时他们就会叫起来:“多么漂亮的房子啊!”

同样,如果你对他们说:“有过一个小王子,总是笑着,招人喜爱,他还想要一只绵羊呢!因为他想要一只绵羊,这就足以证明有这么一个小王子存在。”大人们听了定会耸耸肩膀,把你当小孩子看待!但是,如果你对他们说:“小王子是从B612号小行星来的。”他们就会十分信服,不再用他们的问题来纠缠你了。他们就是这样,不应怪罪他们。小孩子对大人们应该宽宏大量。

当然啦,我们懂得什么是生活,我们对那些数目字嗤之以鼻!我真愿意像讲童话故事那样开始讲这个故事,我真想这样说:

“从前呀,有一个小王子,住在一个只比他稍大一点的小行星上,他希望有一个朋友\ldots{}\ldots{}”对于那些懂得生活的人们来说,这样讲就显得真实多了。

我可不喜欢人们用轻率的态度来读我的书。提起往事,我满腹辛酸。六年前,我的朋友带着他的绵羊一起离开了我。我之所以要在这里尽力地把他描绘出来,就是为了不要忘怀他。忘记一个朋友是可悲的。并不是所有的人都有过一个朋友的。再说,我也可能变得和那些大人们一样,只对数目字感兴趣。也正是为了这个缘故,我买了一盒水彩和一些铅笔。我只是在六岁时尝试着画过完整的和剖开的蟒蛇,以后再没有画过别的东西,所以到了如今这般年纪,重新提笔作画,可真够费劲啊!当然,我将尽力而为,要把小王子的肖像尽量地画得逼真,但是没有真正的把握。一张画画得还可以,另一张画就不太像。我在人体比例上还出了点儿差错。这一张小王子的肖像画得太大了,那一张又画得太小了。对于他的衣服该着什么色,我也拿不准。于是,我就摸索着,这么试试那么改改,凑合着往下画。总之,我很可能在某些比较重要的细节上画错了。这就得请大家多多包涵了。我的这位朋友是从来不加解释的。他大概觉得我同他一样。可是很遗憾,我却不会透过箱子看见小绵羊;我也许有点像大人们,我一定是变老了。

\title{第五章}

每天,我都了解到一些有关他那个星球、他的出走和旅途中的情况。这些都是通过逐渐了解和偶然发问而琢磨出来的。就这样,到第三天,我就了解到关于猴面包树的悲剧。

这一次又是多亏了那只绵羊。小王子像是心事重重,突然问我道:

“绵羊吃小灌木,这是真的吗?”

“是的,是真的。”

“啊!这我就放心了。”

我真不明白,绵羊吃小灌木这件事为什么如此重要。可小王子又问道:

“这么说,绵羊也吃猴面包树啰?”

我提醒小王子说,猴面包树可不是小灌木,而是像教堂那么高的大树;即便是进来一群大像,也没有一棵猴面包树那么高。

{\startalignment[center]
 \placefigeasy[][imgs/小王子(胡雨蘇譯)/00040.jpg][maxwidth=\textwidth,location={middle,none}]{calibre_23}
 \stopalignment}

一群大象的说法,把小王子逗乐了。

“那就只好把这些大象摞起来啦\ldots{}\ldots{}”

但他又很聪明地指出:

“猴面包树在长大以前,也是小树。”

“这很对呀!可是,你为什么想叫你的绵羊去吃小猴面包树呢?”

他回答我道:“怎么!这还用说么!”似乎这是不言而喻的。可是,我却费了很多心思才终于弄懂了这个问题。

其实,小王子的星球和其他行星一样。上面长的草也有好有坏。益草结良种,杂草结坏种。要是单从种子看,是难以分辨的。它们在大地的怀抱里酣睡,一直睡到其中的一粒一时高兴,从梦中醒来。它伸伸懒腰,羞答答地向着太阳生出一片娇嫩喜人的幼芽。如果是一棵红皮白萝卜或玫瑰花的嫩芽,可以让它自由自在地生长。如果它是一棵有害植物的恶苗,一经辨别,就应立即拨掉。在小王子的星球上,就有一些可怕的种子\ldots{}\ldots{}这就是猴面包树的种子。这个星球的土地深受其害。万一有一棵猴面包树我幼苗没能及时拨掉,长起来后就再也拨不掉了。它会遮天蔽地覆盖整个星球,盘根错节地把星体穿透。如果这个行星特别小,而猴面包树又出奇地多,它们能把这个星球撑得四分五裂。

“这是个规矩。”小王子后来向我解释道。“当你早上梳洗之后,必须仔细地给星球梳妆打扮一番。必须遵守规定按时地去拨猴面包树幼苗。猴面包树和玫瑰在幼苗时期十分相似,一旦可以将它们区别开来的时候,就要把猴面包树幼苗拨掉。这是一件非常乏味的工作,但是做起来并不难。”

{\startalignment[center]
 \placefigeasy[][imgs/小王子(胡雨蘇譯)/00048.jpg][maxwidth=\textwidth,location={middle,none}]{calibre_24}
 \stopalignment}

有一天,小王子劝我用心画一幅美丽的图画,好让我们地球上的孩子们对猴面包树的危害有所了解。他还对我说:“如果将来有一天他们外出旅行,这对他们会有用的。有时候把自己的工作推迟一下,等以后再做,不会带来什么麻烦。而拔除猴面包树苗这种事要是耽搁了,那就非造成一场大灾难不可。我就知道有那么一个星球,上面住着一个懒汉,他忽略了三棵小树苗,就\ldots{}\ldots{}.”

{\startalignment[center]
 \placefigeasy[][imgs/小王子(胡雨蘇譯)/00007.jpg][maxwidth=\textwidth,location={middle,none}]{calibre_25}
 \stopalignment}

于是,我就根据小王子的说明,把这个星球画了出来。我从来不大愿意以一位道学家的口吻训人。可是,人们对猴面包树的危害了解得是如此之少,小行星上迷路之人所冒的风险又是如此之大,因此这一回我贸然打破了自己不喜欢教训人的惯例。我说:“孩子们!要当心猴面包树啊!”为了叫我的朋友警惕这种危险------他们跟我一样,长期以来就面临这种危险,却还蒙在鼓里------我才花了很大的功夫画出这幅画。我这里提出的忠告有重大的意义,多在这幅画上花些功夫是很值得的。你们也许要问:为什么这本书别的图画都不及这幅画如此有气派呢?回答很简单:别的图画,我也曾试图把它们画好的,却未能成功。而当我画猴面包树的时候,有一种迫切感在激励着我。

\title{第六章}

啊!小王子,我就这样逐渐知道了你那小天地里忧郁的生活。你在过去很长时间仅有的消遣,就是观赏那夕阳西下的温柔晚景。那是在第四天早晨,我才得知了这个新的情况。你当时对我说道:

“我喜欢看日落。咱们去看日落吧!”

“可是得等\ldots{}\ldots{}”

“还等什么呀?”

“等太阳落山。”

你先是十分惊奇,随后你就自个笑起来了。你对我说:

“我一直以为是在自己的星球上呢!”

其实大家都知道,在美国是正午时分,在法国正是太阳落山的时候,只要在一分钟之内赶到法国,就能看到日落的美景。遗憾的是,离法国太遥远了。而在你的小星球上,你只要把椅子挪动几步就行了。这样,你便随时看得到你想看的晚霞\ldots{}\ldots{}

{\startalignment[center]
 \placefigeasy[][imgs/小王子(胡雨蘇譯)/00009.jpg][maxwidth=\textwidth,location={middle,none}]{calibre_26}
 \stopalignment}

“有一天,我一连看了四十三次日落。”

过了一会儿,你又说:

“你知道\ldots{}\ldots{}.当一个人非常悲伤的时候,总是喜欢看看日落的。”

“你一连看了四十三次日落那天,你就是这么悲伤吗?”

小王子没有回答。

\title{第七章}

第五天,还是那只绵羊的事,向我提示了小王子的生活秘密。他似乎是经过了长时间的苦思冥想,问题有了着落,才开门见山地突然问我:

“要是绵羊吃小灌木,它也要吃花儿啰?”

“绵羊碰到什么就吃什么。”

“连有刺的花也吃吗?”

“是的。连有刺的花也吃。”

“那么说,那些刺还有什么用呢。”

这我可不知道。我当时正忙得不可开交,要把引擎上拧得太紧的一颗螺钉卸下来。我忧心忡忡,因为我发现机器故障看来很严重,饮水眼看就要用光了。我担心最坏的情况发生。

“那么刺还有什么用呢?”

小王子一旦提出问题,就要问个水落石出。我正在为卸不下螺钉而生气,于是就随便答道:

“刺么,没有什么用处,纯粹是花儿们不怀好意。”

“哦!”

但是他沉默片刻之后,满怀怨恨地连声指责我说:

“我不信你的话!花儿们弱不禁风,天真无邪。它们总是设法给自己壮壮胆子,以为有了刺就可以吓退\ldots{}\ldots{}.”

我默不作声。当时我心里在想,要是这个螺丝钉再和我作对,我就一锤子把它敲下来。小王子再次打断了我的思路:

“那么,你认为花\ldots{}\ldots{}.”

“不!不!我什么也不认为!我只是顺口说的。我正在忙正经事儿呢!”

他目瞪口呆地望着我。

“正经事儿?”

他看见我手里拿着锤子,手指上沾满乌黑的油泥,伏在一个他觉得非常难看的东西上。

“你说起话来像那些大人们似的!”

这话使我有点羞愧。可他仍罢休,又无情地说道:

“你都搞错了\ldots{}\ldots{}.你把什么都混在一起了!”

他着实非常恼火。他摇晃着脑袋,金黄色的头发随风摆动。

“我到过一个星球,上面住着一个红脸先生。他从来没闻过一朵花,他从来没看过一颗星星,也从来没喜欢过任何人。除了加法运算以外,他什么事也没没做过。他同你一样,整天的唠叨:‘我是个正经人!我是个正经人'他以此为荣,傲气十足。这哪里是个人啊!这是一个蘑菇。”

“是个什么?”

“是个蘑菇!”

小王子当时怒气冲冲,气得脸色发白。

{\startalignment[center]
 \placefigeasy[][imgs/小王子(胡雨蘇譯)/00041.jpg][maxwidth=\textwidth,location={middle,none}]{calibre_27}
 \stopalignment}

“千万年来花儿都长着刺,千万年来绵羊照样把花儿吃掉。要搞清楚为什么花儿费那么大劲却长出了总是毫无用处的刺来,这难道不是正经事儿?难道羊和花儿之间的斗争不重要?这难道不比那个大胖子红脸先生的加法运算更重要?比如说,我知道宇宙间有一株举世无双的花,她就在我的星球上,别的什么地方都没有,而有一天,一只小绵羊竟然糊里糊涂地把她吃掉了,这难道还不严重吗?”

他气得脸都红了,然后又说:

“如果一个人爱上了一株花,这株花只长在亿万颗星星之中的一个上面,那么观看群星就足以使这个人感到幸福。他会自言自语地说:‘我的花就在其中的一颗星星上\ldots{}\ldots{}.',但是一旦绵羊吃掉了那花,对他来说就是群星突然熄灭!这难道还不严重吗?”

他再也说不下去了,突然抽抽噎噎地哭起来了。夜幕已经降临。我放下手中的工具,觉得锤子螺钉都不屑一顾,干渴和死亡都不值一提。在一颗星星上,也就是在我居住的行星------地球上,有个小王子多么需要安慰啊!我把他抱在怀里,左右摇动着,对他说道:“你喜爱的那朵花没有危险\ldots{}\ldots{}我给你的羊画一个嘴套子\ldots{}\ldots{}.我给你的花画个玻璃罩\ldots{}\ldots{}我\ldots{}\ldots{}.”我也不知道该说些什么是好了,我觉得自己太笨嘴拙舌了。我不知道怎样才能安慰他,才能和他心心相通\ldots{}\ldots{}.唉,泪水之国是多么奥秘啊!

\title{第八章}

很快我就弄清了这株花儿的来历。在小王子的星球上,过去一直生长一些只有一层花瓣的极其简单的花。这些小花既不占地方,也不妨碍任何人。她们清晨在草丛中开放,晚上又自行凋谢。不知从什么地方来的一棵种子,有一天它忽然萌发了。小王子精心地看守着这个与众不同的幼芽:这也许是猴面包树的一个新品种吧。可是小苗很快就不再生长,开始孕育着一个花朵。小王子看到小苗上长出了一个大花蕾,预感到这一定会开出一朵奇异的鲜花。然而这朵花藏在她的绿房中,不断地修饰她那秀丽的容貌。她精心地选择着自己将来的色泽,慢慢地梳妆打扮,一片一片整理自己的花瓣。她不愿像丽春花那样,穿着又皱又破的衣裳来到人间。她只想一旦开放,就要让她那美丽动人的姿容大放异彩。啊!是的,她是一朵非常爱美的花儿。她那神秘的梳妆打扮进行了多少个日日夜夜啊!终于在一天早晨,恰好是在太阳刚升起的时刻,她露出了自己的容颜。

{\startalignment[center]
 \placefigeasy[][imgs/小王子(胡雨蘇譯)/00006.jpg][maxwidth=\textwidth,location={middle,none}]{calibre_28}
 \stopalignment}

尽管已进行过非常精心的打扮了,她却打着哈欠说道:

“哟!我刚刚睡醒\ldots{}\ldots{}.对不起\ldots{}\ldots{}.我的头发还乱着呢\ldots{}\ldots{}.”

小王子此时情不自禁地称赞:

“您是多么美丽啊!”

“可不是么,”花儿温柔地答道:“我跟太阳是同时出生的\ldots{}\ldots{}.”

小王子看出了这花儿不太谦虚,可是这花儿的姿色是何等动人啊!

花儿随后又说道:

“现在该吃早点了吧,请您也想着我点儿\ldots{}\ldots{}”

{\startalignment[center]
 \placefigeasy[][imgs/小王子(胡雨蘇譯)/00035.jpg][maxwidth=\textwidth,location={middle,none}]{calibre_29}
 \stopalignment}

小王子很不好意思地拿来了一个装满清水的喷壶,给那花儿浇水。

就这样,这朵花就以她那有些多疑的虚荣心折磨着小王子。例如,有一天,她对小王子谈起了她身上的那四根刺时说道:

“很可能会有张牙舞爪的老虎跑来的!”

“我的这个星球上没有老虎,何况老虎是不吃草的。”小王子反驳道。

“我不是草。”花儿轻声道。

“真对不起。”

{\startalignment[center]
 \placefigeasy[][imgs/小王子(胡雨蘇譯)/00043.jpg][maxwidth=\textwidth,location={middle,none}]{calibre_30}
 \stopalignment}

“我不怕什么老虎,可我讨厌过堂风。您没有屏风吗?”

“讨厌过堂风\ldots{}\ldots{}这对花草来说可真叫不走运。”小王子思量着,“这朵花可真叫难伺候\ldots{}\ldots{}”

{\startalignment[center]
 \placefigeasy[][imgs/小王子(胡雨蘇譯)/00013.jpg][maxwidth=\textwidth,location={middle,none}]{calibre_31}
 \stopalignment}

“晚上您把我罩起来吧,您这里太冷了。住在这里够受的,我原来住的那个地方\ldots{}\ldots{}.”

可她却欲言又止了。她来的时候是粒种子,对别的世界一无所知。她为自己编造了谎话而感到不好意思,便咳嗽了两三声,好使小王子觉得也有过错。

“屏风呢?”

“我刚才正要去拿,可您在跟我说话呢!”

这时花儿又拼命咳嗽了几声,还要使小王子后悔自己的过错。

这样一来,尽管小王子诚挚地爱上了这朵花,却使他很快对花儿产生了怀疑。小王子对一些无关紧要的话看得过于认真,结果至今深感不幸。

有一天,他向我吐露了真情:“我本不应该听信她,绝不该听信花儿们的话。应当观赏她们的艳容,闻闻她们的芳香。我的那朵花使我的星球清香四溢,可惜我没有福气享受。老虎张牙舞爪的故事本应该打动我的心,却反而使我大为恼火\ldots{}\ldots{}”

小王子继续向我倾吐肺腑之言:

“我当时什么也不懂!我本应该根据她的行为来判断她,而不该只听信她的话。她花香四溢,沁我心脾,给我光明。我真不该离开她跑了出来!我本应体会到,隐藏在她那不高明的花招后面的是一片脉脉温情。花儿是多么自相矛盾啊!可惜我那时太年轻,还不懂得爱她。”

{\startalignment[center]
 \placefigeasy[][imgs/小王子(胡雨蘇譯)/00023.jpg][maxwidth=\textwidth,location={middle,none}]{calibre_32}
 \stopalignment}

\title{第九章}

{\startalignment[center]
 \placefigeasy[][imgs/小王子(胡雨蘇譯)/00026.jpg][maxwidth=\textwidth,location={middle,none}]{calibre_33}
 \stopalignment}

我想,小王子是趁着一次候鸟迁徙的机会出走的。临走的那天早上,他把自己的星球整理得井井有条,细心地把活火山口通一通------他有两座活火山,热早饭很方便。他还有一座死火山。他常说:“不怕一万,就怕万一呀!”因此他把死火山也通好。火山口都通好了,火势就会缓和而均匀,火山就不致突然爆发了。即使爆发,火势也不过像是壁炉中的火焰一样。当然,在我们的地球上,我们的个子太小了,我们不能够去通火山,所以火山给我们带来许多麻烦。

{\startalignment[center]
 \placefigeasy[][imgs/小王子(胡雨蘇譯)/00014.jpg][maxwidth=\textwidth,location={middle,none}]{calibre_34}
 \stopalignment}

小王子还把剩余的最后几株猴面包树幼苗统统拨掉了。他感到有点儿忧伤,以为自己大概永远不会回来了。可是那天早上,他觉得所有的活计都分外亲切。当他最后一次给花浇水,并准备把她罩起来的时候,他发现自己真想哭。

“永别了。”他对花儿说道。

可是花儿没有回答。

“永别了。”他又说了一遍。

花儿咳嗽起来,但她的咳嗽并不是由于感冒所致。

她终于对他说道:

“过去我真傻。请你原谅我吧。祝你幸福。”

花儿对他未加责怪,使他感到惊讶。他手举保护花儿的玻璃罩子,不知所措地伫立在那里。他没体会到花儿这种沉静的柔情。

“我确实爱你。”花儿对他说道,“是我的过错,没叫你了解我的一片心意。这都没有什么。不过,你过去也和我一样地傻。祝你幸福\ldots{}\ldots{}请你把玻璃罩子放到一边去吧!我再也用不着它了!”

“可是风会\ldots{}\ldots{}”

“我的感冒并不那么厉害\ldots{}\ldots{}夜里的凉风倒会对我有好处。我是花儿呀。”

“可是虫子野兽会\ldots{}\ldots{}”

“我要是想认识蝴蝶,就得经得住两三只毛毛虫的打扰。听说蝴蝶美丽极了。不然的话,还有谁来看望我呢?你就要远方去了。那些大野兽么,我一点儿也不怕它们,我也有爪子呀。”

于是,她天真地给小王子看她那四根刺,然后又说道:

“别这么磨磨蹭蹭的,真叫人心烦!你既然决定要走了,那就快走吧。”

她这么说,是因为她不愿意叫小王子看到她掉眼泪。她是一朵多么骄傲的花儿呀\ldots{}\ldots{}

\title{第十章}

小王子来到325号、326号、327号、328号、329号和330号小行星一带。为了找些事做,学点知识,他开始访问这些星球。

第一个星球上住着一个国王。他身穿紫红色的貂皮长袍,坐在一个非常简朴而又极其威严的宝座上。

{\startalignment[center]
 \placefigeasy[][imgs/小王子(胡雨蘇譯)/00046.jpg][maxwidth=\textwidth,location={middle,none}]{calibre_35}
 \stopalignment}

“啊!来了一个臣民。”国王看见小王子,就喊了起来。

小王子心想:

“他从来没有见过我,怎么能认得我呢?”

他哪里知道,在所有的国王眼里,世界非常简单:天下所有的人都是他的臣民。

“走近一些,让我好好看看你。”国王对小王子说道,他为多了一个臣民而神气十足。

小王子举目四顾,想找个地方坐下来。可是整个星球都被国王那豪华的貂皮长袍遮住了。小王子只好站在那里。由于路途劳顿,他打起哈欠来。

“在一个国王面前打哈欠,违反宫廷礼节。我不准你打哈欠。”君主对他说道。

“我实在忍不住了。”小王子挺难为情地说,“我长途跋涉来到这里,还没有睡觉呢\ldots{}\ldots{}.”

“那好吧,我命令你打哈欠。”国王说,“多年以来,我还没有见过有人打哈欠呢!对我来说,打哈欠倒是件新鲜事。来吧!再打一个,这是命令。”

“这可真吓死人了\ldots{}\ldots{}我再也不能\ldots{}\ldots{}”小王子涨红着脸说。

“嗯!嗯!”国王说道,“那么,我\ldots{}\ldots{}我就命令你一会儿打哈欠,一会儿不打\ldots{}\ldots{}”

他嘴里嘟嚷着,看样子很恼火。

因为国王认为,最重要的是他的权威,不允许有人违抗他的命令。他是一个专制君主。但是由于他心地善良,他下的命令倒也合乎情理。

“假如我命令,”他说起话来滔滔不绝,“假如我命令一位将军变成一只海鸟,而这位将军不服从命令,那么,这就不是将军的过错,而是我错了。”

“我可以坐下吗?”小王子怯生生地问道。

“我命令你坐下。”国王回答说,同时威严地拉了拉貂皮长袍的下摆。

可是小王子感到很奇怪。这个星球很小,国王究竟统治什么呢?

“陛下\ldots{}\ldots{}”他对国王说道,“请原谅,让我向您提个问题。”

“我命令你向我提问题。”国王急忙说道。

“陛下\ldots{}\ldots{}您统治什么呢?”

“统治一切。”国王非常简单明了地答道。

“统治一切?”

国王以不太引人注目的手势指了指他的星球、其他的星球以及满天的星星。

“您统治这一切?”小王子又问。

“统治这一切\ldots{}\ldots{}”国王答道。

原来他不仅是一个专制的君主,而且是一个宇宙之王。

“那么,所有的星星都向您称臣了?”

“那当然!”国王对他说道:“我一声令下,诸星从命,倘有违抗,决不容情。”

如此这般的权力真叫小王子敬佩。要是他也有这么大的权力,他就可以连椅子也不用挪动,在一天之内不止看四十次,而是看七十二次,甚至一百次或二百次的夕阳西下了!由于他想起了他那被遗弃的小星球,他心中有点惆怅。他鼓起勇气请求国王恩典:

“我想看一次日落\ldots{}\ldots{}请求您让我高兴高兴吧\ldots{}\ldots{}请您命令太阳落下去\ldots{}\ldots{}”

国王说道:“如果我命令一位将军像蝴蝶那样从这朵花飞到那朵花,如果我命令他写一出悲剧或者变成一只海鸟,而这位将军拒绝从命的话,那么你说是他不对还是我不对呢?”

“是您不对。”小王子肯定地答道。

“完全正确。”国王接着说,“应该要求每个人做他力所能及的事情。权威首先应该建立在理性的基础上。如果你命令你的人民去投海自尽,他们就非起来革命不可。我的命令是合理的,所以我有权要求别人服从。”

“那么,我请求看日落的事怎么办?”小王子提醒国王道。他一旦提出问题,得不到答复是不会罢休的。

“日落么,你会看到的。我一定要命令太阳落山。不过,根据我的治国方针,我要等到时机成熟。”

“要等到什么时候呢?”小王子问道。

“嗯!嗯!”国王先翻阅了一下一本厚厚的日历,慢腾腾地答道,“日落大约是在\ldots{}\ldots{}是在今晚七点四十分!你将看到,星球们对我都是唯命是从的。”

小王子又打了个哈欠。因为看不到日落,他感到很扫兴,也觉得有点儿无聊。

“我在这儿没事可做了,”他对国王说,“我要走了。”

“别走,”国王说道,有人来做他的臣民,他是那么得意,“别走,我封你做我的大臣!”

“什么大臣呀?”

“司\ldots{}\ldots{}司法大臣!”

“可是没有人需要受审判啊!”

“不见得吧,”国王对他说,“我还没巡视过我的国土。我太老了,走不动了,而这里,连停一辆马车的地方都没有。”

“哦,我已经看到了。”小王子探着身子向星球的另一边看了一眼说,“那边也没有人呀\ldots{}\ldots{}”

“那么你就自己来评判自己吧,”国王回答他说,“这是最难的了。自己评判自己要比评判别人困难多了。如果你能做到有自知之明,那你就是个名副其实的圣人啦。”

“我呀,”小王子说,“我在哪儿都能自己评价自己,何必在您这儿呢!”

“唔!唔!”国王说,“我想起来了,在我的星球的一个什么地方,有一只很老的大老鼠,夜里我经常听到它出来活动。你可以去审判这只老鼠。你可以不时地宣判它死刑,它的性命将由你主宰。但为了留它一条活命,每次判刑之后,你可得赦免它。因为这儿只有这么一只老鼠了。”

“我,”小王子回答说,“我才不喜欢宣判什么死刑呢,我认为我该走了。”

“不行。”国王说。

小王子还是做好了走的准备。可是他不愿意让这位年迈的国王难过,就说:

“如果陛下希望人们对您唯命是从的话,那么您可以给我下一道合情合理的命令。比方说,可以在我走的前一分钟下令叫我离开。我觉得条件成熟了。”

国王默不作声。小王子犹豫片刻,接着叹了一口气就出发了。

“我封你做我的使臣。”国王连忙喊道。

他摆出一副威风凛凛的神态。

大人们可真怪,一路上小王子心里一直这么想。

\title{第十一章}

第二个行星上住着一个爱好虚荣的人。

“啊!啊!一个崇拜我的人来拜访了!”这个虚荣迷一看见小王子,老远就喊叫起来。

因为在一切虚荣迷的眼里,所有的人都是崇拜者。

“您好!”小王子说道。“您的帽子可真滑稽。”

{\startalignment[center]
 \placefigeasy[][imgs/小王子(胡雨蘇譯)/00033.jpg][maxwidth=\textwidth,location={middle,none}]{calibre_36}
 \stopalignment}

“这是用来向人们致意的。”虚荣迷回答道,“当人们向我喝彩的时候,我就用帽子向他们致意。可惜,从来没有人来过这里。”

“真的吗?”小王子莫明其妙地问。

“拍手啊!用这一只手去拍另一只手。”虚荣迷要小王子鼓掌。

小王子的两只手拍打起来,虚荣迷谦逊地脱帽致意。

“这比拜访那位国王还有意思。”小王子心里想。他于是再次拍手,虚荣迷再次举帽致意。

练习了五分钟以后,小王子就对这枯燥单调的游戏不感兴趣了。

“告诉我,要你把帽子放下,”小王子问,“该怎么办呢?”

可是虚荣迷根本听不见他的话,除了赞美和颂扬,虚荣迷从来听不见别的话。

“您是不是真的非常崇拜我呀?”他问小王子。

“崇拜是什么意思呀?”

“崇拜么,就是承认我是这个星上最英俊、最华丽、最富有、最聪明的人。”

“可是你的星球上只有你一个哪!”

“请你成人之美,还是崇拜我吧!”

“我崇拜你,”小王子无可奈何地耸肩膀说,“可是,这对你又有什么好处呢?”

于是小王子离去了。

大人们的的确确够古怪的,一路上小王子心里总是这么想。

\title{第十二章}

{\startalignment[center]
 \placefigeasy[][imgs/小王子(胡雨蘇譯)/00031.jpg][maxwidth=\textwidth,location={middle,none}]{calibre_37}
 \stopalignment}

第三个星球上住着一个酒鬼。访问的时间颇为短促,却使小王子非常忧伤。

“您在这儿干什么呢?”小王子问酒鬼。这个酒鬼默默地坐在那里,面前有一堆空瓶子和一堆装满酒的瓶子。

“我在喝酒。”酒鬼神情忧郁地答道。

“你为什么要喝酒哇?”小王子又问。

“为了忘却。”酒鬼回答。

“忘却什么呢?”小王子又问。

“忘却耻辱。”酒鬼低头承认。

“什么耻辱哇?”小王子追问道。他很想助他一臂之力。

“喝酒的耻辱。”酒鬼说完就再也不作声了。

小王子迷惑不解地离去了。

这些大人们的的确确太古怪了,一路上小王子只有这个想法。

\title{第十三章}

第四颗是个商人的星球。这个人忙得不可开交,所以小王子到来的时候,他连头都没有抬一下。

{\startalignment[center]
 \placefigeasy[][imgs/小王子(胡雨蘇譯)/00003.jpg][maxwidth=\textwidth,location={middle,none}]{calibre_38}
 \stopalignment}

小王子对他说道:“您好。您的香烟来灭了。”

“三加二等于五。五加七等于十二。十二加三等于十五。你好。十五加七,二十二。二十二加六,二十八。没有时间去重新点燃香烟了。二十六加五,三十一。喔唷!一共是五亿零一百六十二万二千七百三十一。”

“五亿什么呀?”

“嗯?你还待在这儿哪?五亿一百万\ldots{}\ldots{}我也不知道是些什么了。我的工作多极了!我办事严肃认真,我可没有工夫去闲聊!二加五等于七\ldots{}\ldots{}”

“五亿一百万什么呀?”小王子重复地问道。他生来就是这样,非打破砂锅问到底不可。

商人这才抬起头来:

“我住在这个星球上五十四年以来,只被打扰过三次。第一次是二十二年前,不知从什么地方掉下来一只金龟子,发出一种可怕的声响,使我在一次加法运算中出了四个错。第二次发生在十一年前,那是因为我的关节炎发作。我的身体缺乏锻炼。我没有工夫去闲逛。我是个严肃认真的人。现在\ldots{}\ldots{}这是第三次!我算的结果是五亿零一百万\ldots{}\ldots{}”

“零一百万个什么?”

商人明白了,不回答小王子他就休想得到安静:

“零一百万个小东西。人们有时看见这些小东西出现在天空中。”

“是苍蝇吗?”

“不是,是些闪闪发光的小东西。”

“是蜜蜂?”

“也不是,是金黄色的小东西,这些小东西叫懒汉们想入非非。我是个严肃认真的人!我可没有时间去胡思乱想。”

“啊,是星星吧?”

“正是星星。”

“你要这五亿颗星星做什么呀?”

“五亿零一百六十二万二千七百三十一颗。我是个严肃认真的人,我讲究准确。”

“你要这些星星做什么呀?”

“我要做什么?”

“对呀!”

“什么也不做,我占有它们。”

“你占有星星?”

“是的。”

“可是,我已经见到过一个国王,他\ldots{}\ldots{}”

“国王不占有,他们只‘统治'。这完全是两码事。”

“那么,你占有星星对你有什么用呢?”

“使我发财致富。”

“发财致富又有什么用?”

“富了就可以去买别的星星,如果有人发现了别的星球的话。”

小王子心中暗想,这个人哪,说起话来真有点像那个酒鬼。

然而他继续提出问题:

“你怎么能够占有星星呢?”

“你说星星是谁的?”商人不耐烦地反问道。

“我不知道。星星不属于任何人。”

“那那啦,它们是属于我的,因为我第一个有这种想法。”

“这就够了吗?”

“那当然啦。如果你发现了一颗不属于任何人的金钢石,那么它就是你的。当你发现一个海岛没有主人时,这个海岛就是你的。当你第一个提出某种设计的时候,你就去申请一个专利证,这种设计就是你的了。既然在我之前不曾有人想到过占有这些星星,是我第一个想到了这件事情,那么我就占有这些星星了。”

“确实如此。可是,你占有星星做什么?”小王子说。

“我管理它们。我统计它们的数目,反反复复的计算,”商人说,“难哪。不过,我是个严肃认真的人。”

小王子对这样的回答并不满意。

“要是我有一条围巾,我就把它围在脖子上。要是我有一朵花,我就把它摘下来戴上。可是你不能去摘星星呀!”

“是不能,但是我可以把它们存在银行里。”

“这是什么意思?”

“这就是说,我把星星的数目写在一张小纸片上,然后就把它锁在抽屉里。”

“这就行了吗?”

“这就行了。”

真有意思,小王子想。挺有诗意,可就是不太严肃。

在重大的事情上,小王子与大人们的想法截然不同。

“我呀,”他又说,“我有一朵花,我每天都给她浇水。我有三座火山,我每星期都给它们通一次火山口,连死火山也不放过。谁知道死火山会不会再变活呢。我拥有花和火山,我这样做对我的花有好处,对我的火山也有好处。可是你对星星并没有好处\ldots{}\ldots{}”

那商人被问得张口结舌,无言以对。于是小王子就离去了。

这些大人们简直古怪得出奇,一路上小王子总这么想。

\title{第十四章}

第五个行星非常奇特,是所有这些小行星中最小的一颗。这颗行星上面正好容得下一盏路灯和一个点灯人。小王子怎么也弄不明白,在这个位于天空某一角落,既无房屋又无居民的行星上,要一盏路灯和一个点灯人有什么用呢?他又不禁暗自思量:

“这人很可能是个糊涂虫。但和国王、虚荣迷、商人以及那个酒鬼比起来,他还不那么愚蠢。至少他的工作具有某种意义。当他点着了他的路灯的时候,就好像他为天空增添了一颗星星或一朵花。当他熄灭了自己的路灯时,就好比是让星星或花儿入睡了。这个工作挺有意思。既然有意思,它就是真正有益的了。”

一来到这个行星上,小王子就恭恭敬敬地向点灯人致意:

“早上好。你刚才为什么把路灯熄灭呢?”

“这是照章办事。早上好。”点灯人回答道。

“什么叫照章办事呀?”

“就是熄掉我的路灯。晚上好。”

这时他重新点着了他的路灯。

“可是你刚才为什么又把它点着了呢?”

“这是照章办事呀。”点灯人答道。

“我不明白。”小王子说道。

“没有什么要明白的。照章办事就是照章办事。”点灯人答道。“早上好。

这时他又熄灭了路灯。

然后他用一块红方格手绢擦掉额头上的汗珠。

“干我这一行可真够受的。从前还说得过去。早上我把路灯熄灭,晚上我再把它点着。白天有休息的空儿,夜里有睡觉的时间\ldots{}\ldots{}”

“你是说从那以后规章变了吗?”

“规章倒没变,”点灯人说,“倒霉就倒霉在这里!这个星球一年比一年转得快,而规章却始终没有变。”

“那么现在呢?”小王子问。

“现在它每分钟转一圈,我连一秒钟的休息时间都没有了。我每分钟都要点一次,熄一次呢!”

“真稀奇!你这里每天只有一分钟长啊!”

“一点儿也不稀奇。”点灯人说。“我们俩说话的这会儿,已经是一个月了。”

“一个月了?”

“是的。三十分钟就是三十天,一个月。晚上好。”

于是他又点了他的路灯。

小王子望着他,打心眼里喜欢这个如此忠于职守的人。这时,他想起了自己从前常常挪动椅子,追着看日落的事。他很想帮助他的这位朋友:

{\startalignment[center]
 \placefigeasy[][imgs/小王子(胡雨蘇譯)/00036.jpg][maxwidth=\textwidth,location={middle,none}]{calibre_39}
 \stopalignment}

“你可知道\ldots{}\ldots{}我有个办法能叫你想什么时候休息就什么时候休息\ldots{}\ldots{}”

“我一直想找个休息的办法呢。”点灯人说。

看来这人可能既是忠于职守的人,同时又是个懒汉。

小王子接着说:

“你的星球这么小,三步就能绕一圈。你只要慢慢地走,就总能面向太阳。你什么时候想休息,你就什么时候走动\ldots{}\ldots{}这样你想让白天有多长,它就有多长。”

“这也帮不了我多大忙,”点灯人说,“我最喜欢的就是睡觉。”

“这可不走运。”小王子说。

“是不走运,”点灯人说,“早上好。”

这时他又把路灯熄灭了。

当小王子继续往前赶路时,他自言自语道:

“这个点灯人哪,可能别人看不起他,国王、虚荣迷、酒鬼和商人都不把他放在眼里。但在我看来,他是唯一不显得荒唐可笑的人。这也许是因为他只关心其他的事情,而毫不利己的缘故吧。”

小王子惋惜地叹了口气,心中还在想:

“这个人我唯一可以交朋友的人。可是他的星球实在太小,容不下两个人\ldots{}\ldots{}”

小王子不敢承认的是,他十分留恋这颗令人赞美的星球,特别是因为,那里每二十四小时就日落一千四百四十次!

\title{第十五章}

第六颗星星比第五颗大十倍,上面住着一位正在写巨著的老先生。

{\startalignment[center]
 \placefigeasy[][imgs/小王子(胡雨蘇譯)/00027.jpg][maxwidth=\textwidth,location={middle,none}]{calibre_40}
 \stopalignment}

“啊!来了一位探险家!”老先生一见小王子就叫了起来。

小王子在桌旁坐下来,有点儿气喘吁吁。他这段旅程可不近呵!

“你是从哪里来的?”老先生问小王子。

“这大厚本子是什么书呀?”小王子问道,“您在这里干什么呢?”

“我是地理学家。”老先生回答。

“什么是地理学家?”

“地理学家是一种学者,他知道哪里有海洋,哪里有江河、城市、山脉和沙漠。”

“这可真有意思。”小王子说道。“这才是一种真正的职业呢!”他环顾了一下这位地理学家的星球。他还从来没有见过一个如此壮观的行星呢。

“您的星球可真漂亮。这里有海洋吗?”

“这我可不知道。”地理学家说。

“啊!”小王子有点失望。“那么有山脉吗?”

“这我怎么能知道呢。”地理学家又说。

“那么有城市、江河和沙漠吗?”

“这些我更无法知道。”地理学家说。

“可您是地理学家呀!”

“我是地理学家,一点儿不错。”地理学家说道。“可我不是探险家。我连一个探险家都没有呢。去统计一下有多少城市、江河、山脉和沙漠,这不是地理学家的事。地理学家很重要,不能出去到处跑。他不能离开他的办公室。但是,他可以在自己的办公室里接待探险家。他询问探险家,并把他们的回忆记录下来。如果他对其中的一位探险家的回忆感兴趣,那么他就对这位探险家的品德进行一番调查。”

“这是为什么呢?”

“因为一个说谎的探险家会给地理学家的著作带来灾难,同样,一个喝得酩酊大醉的探险家也会如此。”

“这又是为什么呢?”小王子问。

“因为醉汉会把一个看成两个。这么一来,地理学家就会只有一座山的地方标上两座山。”

“我认识一个人,”小王子说,“他大概是个很蹩脚的探险家。”

“这很可能。所以嘛,即使探险家的品德是好的,还是得对他的发现进行一番调查。”

“去实地看一看吗?”

“不,那太费事了。但是我要求探险家提出证据。比如说他发现了一座大山,我就要他把山上的大石头带回一些来。”

地理学家突然激动起来。

“你,你是从远方来的吧!你是个探险家!你来给我描述一下你那个星球吧!”

说话间地理学家打开了记录本,削好了铅笔。他先用铅笔记下探险家的叙述,等探险家提出了证据,他再用墨水笔记录下来。

“怎么样?”地理学家问。

“啊!我的星球呀,”小王子说,“没有多大意思,它非常小。我有三座火山。两座活火山,还有一座死火山。谁也说不准它今后会不会爆发。”

“谁也说不准。”地理学家说道。

“我还有一棵花呢。”

“我们是不记录花卉的。”地理学家说道。

“这是为什么?花儿最美丽!”

“因为花卉是转瞬即逝的东西。”

“‘转瞬即逝'是什么意思?”

“地理著作是各种书中最珍贵的书。”地理学家说,“这种书从来不会过时。大山搬家古今罕见,大海干涸世上未闻。我们只记载永恒的东西。”

“但是死火山还能复活呀,”小王子打断了地理学家的话,“‘转瞬即逝'是什么意思?”

“不管火山是死了还是复活了,这对我们来说都是一回事。”地理学家说,“重要的一点,它是山,山是不会变的。”

“但是,‘转瞬即逝'是什么意思?”小王子追问道。他从来都是这样,一旦提出问题,得不到回答是不肯罢休的。

“意思就是:有很快就要消亡的危险。”

“我的花有很快就要消亡的危险吗?”

“那当然啦。”

“我的花的生命也是转瞬即逝的,”小王子自言自语地说,“面对世界,她只有四根刺来进行自卫呀!而她却被我抛下了,孤零零地留在家里!”

这是他有生以来的第一件憾事。然而,他又重新鼓起勇气问道:

“您能告诉我应去看些什么吗?”

“去看看地球吧,”地理学家回答他,“它的名声很好\ldots{}\ldots{}”

小王子惦记着他的花,离去了。

\title{第十六章}

第七个行星就是地球了。

地球可不是一颗普普通通的行星!它上面有一百一十个国王(当然啦,没有遗漏黑人国王),七千个地理学家,九十万个商人,七百五十万个酒鬼,三亿一千一百万个虚荣迷,也就是说,大约有二十亿大人。

为了使你们对地球的大小有个概念,我想告诉你们,在发明电灯之前,在地球的六大洲上,养活着一支拥有四十六万二千五百一十二个点灯人的真正大军。

从稍远的地方望去,那景象好不壮丽辉煌。这支大军的动作宛如芭蕾舞剧中的动作一般,谐调而优美。首先是新西兰和澳大利亚的点灯人登场,他们把路灯点着,随后回去睡觉。这时中国和西伯利亚的点灯人翩翩起舞,接着就隐入幕后去了。之后,俄国和印度的点灯人出场,随后是非洲和欧洲的点灯人,然后就是南美洲的,再就是北美洲的点灯人出场了。他们从来不会弄错登场的顺序。这种场面可谓壮观极了。

唯独北极和南极总共只有两个点灯人,他们过着悠闲自得的生活,一年之内他们只工作两次。

\title{第十七章}

卖弄小聪明的往往要说点假话。当我跟你们谈到点灯人的时候,我就不那么诚实,险些使那些不了解我们地球的人产生错觉。人类在地球上只占很小一块地方。如果居住在地球上的二十亿人都站着,像开群众大会那样稍微挤紧一点,就能宽宽绰绰地在一个二十英里见方的广场上住下。这就是说,可以把整个人类堆在太平洋的一个最小的岛屿上。

大人们当然不会相信你们的。他们自以为要占很大的地方,他们把自己看得像猴面包树那样大得了不起。那么你们就建议他们去做计算题吧。他们对数目字简直着了迷,数目字能使他们笑逐颜开。但是你们千万不要在这种无聊的演算上浪费时间,这是徒劳无益的。在这点,你们尽管相信我好啦。

小王子来到地球上,看不见一个人影,感到很吃惊。要不是看到一个月白色的圆环在沙地上蠕动的话,他真担心是搞错了星球。

“晚安。”小王子想碰碰运气,冒然说了一声。

“晚安。”蛇说道。

“我落在了什么星球上啦?”小王子问。

“落在了地球上,在非洲。”蛇回答说。

{\startalignment[center]
 \placefigeasy[][imgs/小王子(胡雨蘇譯)/00004.jpg][maxwidth=\textwidth,location={middle,none}]{calibre_41}
 \stopalignment}

“啊!\ldots{}\ldots{}难道说地球上没有一个人?”

“这里是沙漠,沙漠里没有人。地球大着呢。”蛇说。

小王子坐在一块石头上,仰望着天空,说:

“我心里在想,这些星星闪闪发亮,会不会是为了让每个人有朝一日都能重新找到自己的星球。请看看我的那颗星球吧,它正好处在我们的上方\ldots{}\ldots{}可是它离我多么遥远哪!”

“它很美。”蛇说。“你来这里干什么呀?”

“我和一朵花闹了别扭。”小王子说道。

“啊!”蛇说。

于是他们都沉默不语了。

“人们都在什么地方呢?”小王子终于又开了腔,“在沙漠里,我真觉得有点孤独\ldots{}\ldots{}”

“就是到了有人的地方,也是同样的孤独。”蛇说。

小王子久久看着蛇。

“你是个奇怪的动物,细得像个手指头\ldots{}\ldots{}”小王子终于对蛇说道。

{\startalignment[center]
 \placefigeasy[][imgs/小王子(胡雨蘇譯)/00005.jpg][maxwidth=\textwidth,location={middle,none}]{calibre_42}
 \stopalignment}

“但我比国王的手指还要厉害呢。”蛇说道。

小王子的脸上露出了笑容:

“我看你没有那么厉害\ldots{}\ldots{}你连脚都没有\ldots{}\ldots{}恐怕你连旅行都不能够\ldots{}\ldots{}”

“我能够把你带到很远的地方去,比一条海船能去的地方还远呢。”蛇说。

蛇围着小王子的脚腕盘了起来,好像一只金镯子。

“凡是我接触到的人,我都把他送回老家去。”蛇又说。“可你很纯洁,而且是从另一个星球上来的\ldots{}\ldots{}”

小王子什么也没有回答。

“在这花岗岩一般的地球上,你是这么弱小,我很可怜你。如果有一天你心中悲伤,非常怀念你的星球,那时我可以帮助你。我可以\ldots{}\ldots{}”

“啊!我完全明白了你的意思。”小王子说道。“但是,为什么你说的话都像谜语那样隐晦呢?”

“可我把一切谜底都说破了。”蛇说。

于是他们又沉默不语了。

\title{第十八章}

小王子穿行在沙漠中,但他只遇到一朵花。这是一朵有着三个花瓣的花,一朵很不起眼的小花儿\ldots{}\ldots{}

{\startalignment[center]
 \placefigeasy[][imgs/小王子(胡雨蘇譯)/00017.jpg][maxwidth=\textwidth,location={middle,none}]{calibre_43}
 \stopalignment}

“你好。”小王子说道。

“你好。”花儿说道。

“人都在什么地方呢?”小王子有礼貌地问。

有一天,这朵花曾看见一支沙漠商队走了过去:

“人吗?是有的,好像有那么六七个人。好几年以前,我看见过他们。可是,我从来不知道到什么地方能找到他们。风吹着他们到处乱跑。他们没有根儿,这使他们很不方便。”

“再见。”小王子说。

“再见。”花儿说。

\title{第十九章}

小王子登上一座高山。以往他所见过的山,就是那三座高不过他膝盖的火山,而且他把那座死火山当凳子坐。因此小王子自言自语地说:“从这么高的一座山望去,我一眼可以看到整个星球,看到所有的人\ldots{}\ldots{}”可是他所看到的,只是一些嶙峋的怪石、突兀的山峰。

{\startalignment[center]
 \placefigeasy[][imgs/小王子(胡雨蘇譯)/00019.jpg][maxwidth=\textwidth,location={middle,none}]{calibre_44}
 \stopalignment}

“你们好。”小王子试探着问道。

“你们好\ldots{}\ldots{}你们好\ldots{}\ldots{}你们好\ldots{}\ldots{}”回声答道。

“你们是谁?”小王子问。

“你们是谁\ldots{}\ldots{}你们是谁\ldots{}\ldots{}你们是谁\ldots{}\ldots{}”回声答道。

“请你们做我的朋友吧,我很孤单。”他又说道。

“我很孤单\ldots{}\ldots{}我很孤单\ldots{}\ldots{}我很孤单\ldots{}\ldots{}”回声再次答道。

小王子心中思量:“这是个多么奇怪的星球啊!它一片干旱,到处是突兀的怪石,还弥漫着咸味。这里的人居然连一点想象力都没有,只是重复别人对他们说的话\ldots{}\ldots{}在我的星上,我有一朵花:她总是第一个开口说话\ldots{}\ldots{}”

\title{第二十章}

小王子走啊走啊,穿沙漠、翻山岩、过雪地,经过长途跋涉终于发现了条大路。这里条条大路都通向人们居住的地方。

“你们好。”小王子说。

这是一个玫瑰盛开的花园。

“你好。”玫瑰花们说道。

{\startalignment[center]
 \placefigeasy[][imgs/小王子(胡雨蘇譯)/00024.jpg][maxwidth=\textwidth,location={middle,none}]{calibre_45}
 \stopalignment}

小王子瞧着这些花,发现她们全都跟他自己那朵花一模一样。

“你们都是什么花?”小王子惊得发呆,问她们道。

“我们是玫瑰花。”玫瑰花们说道。

“啊!\ldots{}\ldots{}”小王子说。

他感到自己十分不幸。他的那朵花曾对他说过,她在宇宙间是独一无二的一朵玫瑰花。可是现在,仅此一处花园里,就有五千朵和她一模一样的花!

小王子自言自语道:“要是她看到这些,她定会很生气的\ldots{}\ldots{}她会咳嗽得非常厉害,甚至装死,以免别人耻笑。而我呢,就不得不装出照料她的样子,因为不这样的话,她为叫我丢脸,也许真的会死去\ldots{}\ldots{}”

接着他又自语道:“我一直以为自己拥有一朵独一无二的花,其实我有的仅是一朵普普通通的玫瑰花。这朵花,连同那三座高不过膝的火山,而且其中一座也许是永远熄灭了,这一切都不会使我成为一个极其伟大的王子\ldots{}\ldots{}”想到此,他一头扑在草地上哭了。

{\startalignment[center]
 \placefigeasy[][imgs/小王子(胡雨蘇譯)/00049.jpg][maxwidth=\textwidth,location={middle,none}]{calibre_46}
 \stopalignment}

\title{第二十一章}

这时来了一只狐狸。

{\startalignment[center]
 \placefigeasy[][imgs/小王子(胡雨蘇譯)/00037.jpg][maxwidth=\textwidth,location={middle,none}]{calibre_47}
 \stopalignment}

“你好。”狐狸说。

“你好。”小王子彬彬有礼地回答。他转过身子,但什么也没有看到。

“我在这儿呢,在苹果树底下\ldots{}\ldots{}”那声音说。

“你是谁?”小王子问,“你真漂亮\ldots{}\ldots{}”

“我是一只狐狸。”狐狸说。

“来跟我一起玩吧,”小王子向狐狸建议说,“我苦恼极了\ldots{}\ldots{}”

“我不能跟你一起玩,”狐狸说,“我还没有被你驯养呢。”

“啊!对不起。”小王子说。

但是他思索了一阵子,又说道,

“‘驯养'是什么意思?”

“看来你不是本地人,”狐狸说,“你来寻找什么呢?”

“我来找人。”小王子说,“‘驯养'是什么意思?”

“人,”狐狸说道,“人有枪,他们打猎,这可真讨厌!他们也养鸡,这是他们唯一关心的事。你也找鸡吗?”

“不,”小王子说,“我是来找朋友的。‘驯养'是什么意思?”

“这是早就被人忘了的事情了,”狐狸说,“它的意思是,‘建立联系'。”

“建立联系?”

“当然啦。”狐狸说,“对我来说,你跟成千上万个小男孩一模一样。我不需要你,你也不需要我。对你来说,我跟成千上万只狐狸毫无差别。但是,如果你驯养了我,我们就谁也离不开谁了。那时候,我在世界上只有你,你在世界上只有我\ldots{}\ldots{}”

“我有点明白了。”小王子说道,“有一朵花,我想,她已经把我驯服了\ldots{}\ldots{}”

“这是可能的。”狐狸说,“在这个地球上,可以说是无奇不有\ldots{}\ldots{}”

“哎呀!这不是地球上的事。”小王子说道。

狐狸显露出非常惊奇的神色。

“在另一个星球上?”

“是的。”

“那个星球上有猎人吗?”

{\startalignment[center]
 \placefigeasy[][imgs/小王子(胡雨蘇譯)/00044.jpg][maxwidth=\textwidth,location={middle,none}]{calibre_48}
 \stopalignment}

“没有。”

“这可真有意思!那么,有老母鸡吗?”

“没有。”

“没有十全十美的事物。”狐狸叹息道。

可是,狐狸又把话题转了回来:

“我的生活单调乏味:我捉鸡,人捉我。所有的鸡都一模一样,所有的人都一模一样。因此,我感到有些厌烦了。但是,如果你驯服了我,我的生活就会充满快乐。我会分辨出一种与众不同的脚步声。别的脚步声会叫我躲进洞里去,而唯独你的脚步声会像音乐一样,唤我出洞来。再说,你瞧瞧!你看见那边的麦田了吗?我从来不吃面包,小麦对我毫无用处。麦田也不会使我产生任何联想。这是多么可悲啊!但是,你有一头金黄色的头发。一旦你驯服了我,那将是多么美好啊!那金黄色的小麦会使我想起你来。于是就连那滚动在麦浪里的风声,也会叫我喜欢听的\ldots{}\ldots{}”

狐狸说到这里就不作声了,它久久地看着小王子。

“请你\ldots{}\ldots{}请你驯养我吧。”它说。

“我很想驯养你,”小王子回答,“但是我没有那么多时间。我得去寻找朋友,我还有许多事物要认识呢。”

“只有被人们驯服了的事物,才能为人们所认识。”狐狸说,“人们再也没有时间去认识别的什么事物了。他们总是到商人那里去买现成的东西。但是,由于世界上还没有出售朋友的商店,所以人也就没有朋友。要是你想交一个朋友的话,你就驯养我吧!”

“那么应该怎么办呢?”小王子问。

“应该很耐心。”狐狸答道,“开头时你就这样坐在草地上,要离我稍远些。我偷眼看你,你什么也别说。言语是误会的根源。但是,你每天都可以坐得离我更近些\ldots{}\ldots{}”

第二天,小王子又来了。

“最好在同一个时间来。”狐狸说,“比如说你下午四点钟来,我从三点钟起就会开始感到幸福了。愈是临近四点钟,我就愈是感到幸福。四点钟一到,我就会坐立不住,惴惴不安起来:我将发现幸福是有代价的!但是,如果你随便什么时候来,我就不知道该在什么时候做好心理准备了\ldots{}\ldots{}这需要养成习惯。”

{\startalignment[center]
 \placefigeasy[][imgs/小王子(胡雨蘇譯)/00008.jpg][maxwidth=\textwidth,location={middle,none}]{calibre_49}
 \stopalignment}

“什么叫习惯呢?”小王子问。

“这也是一件早被人忘掉了的事情。”狐狸说,“所谓习惯,就是使某一天不同于其他的日子,使某一时刻不同于其他的时刻。比如说,捉我的那些猎人们就有个习惯。他们每星期四都和村里的姑娘们跳舞。于是,星期四就是一个美妙的日子!我外出散步,一直走到葡萄园。如果猎人们随便什么时候都跳舞,每天又都是一个样,那么我也就没休息的日子了。”

就这样,小王子驯服了狐狸。分手的时候快要到了,狐狸说道:

“哎!我肯定会哭的。”

“这是你的过错。”小王子说道,“我本来一点儿不希望你难过的,可你偏偏要我驯服你\ldots{}\ldots{}”

“是这样的。”狐狸说。

“你都要哭出来了。”小王子说。

“那当然。”狐狸说。

“可你什么好处还没有得到呢。”

“由于麦子颜色的缘故,我还是得到了好处。”狐狸说。

然后狐狸又说:

“你再去看看那些玫瑰花吧。你一定会明白,你的那朵花是天下独一无二的玫瑰。当你回来向我告别的时候,我将赠你一个秘密作礼物。”

于是小王子就跑去看那些玫瑰花。

“你们一点也不像我的那朵玫瑰花,你们还什么都不是呢。”小王子对她们说。“没有人驯养过你们,你们也没有驯服过任何人。你们就像我的狐狸过去那样,它那时只是一只与成千上万只狐狸一样的狐狸。可是,我现在已经和它交上朋友,它现在就是世界上一只独一无二的狐狸了。”

这时,那些玫瑰花们感到很难为情。

“你们美丽,但是你们空虚。”小王子又对她们说道,“没有人能为你们去死。当然,一个普通的过路人会以为我的那朵玫瑰花和你们一样。但是,她单独一朵花就比你们全体都名贵。因为她是我浇灌的花。因为她是我放到玻璃罩下的。因为她是我用屏风保护起来的。因为她身上的毛毛虫(除了两三只变蝴蝶的幼虫外)都是我除掉的。因为我听过她倾诉愁苦或自夸自赞,有时甚至还倾听过她沉默无言。因为她是我的玫瑰花。”

小王子又回到了狐狸身边。

“再见。”他说。

“再见。”狐狸说,“这就是我的秘密,它很简单:只有心灵才能洞察一切,肉眼是看不见事物本质的。”

“肉眼看不见事物的本质。”小王子重复着这句话,要把它记在心间。

“正因为你在你的玫瑰花身上花费了时间,这才使她变得如此名贵。”

“正因为我在我的玫瑰花身上花费了时间\ldots{}\ldots{}”小王子重复着这句话,要把它记在心间。

“人们已经忘记了这个真理,”狐狸说,“但是你不应忘记它。你要对你驯养过的一切永远负责,你要对你的那朵玫瑰花负责\ldots{}\ldots{}”

“我要对我的那朵玫瑰花负责\ldots{}\ldots{}”小王子重复着,要把它记在心间。

\title{第二十二章}

“你好。”小王子说。

“你好。”扳道工说。

“你在这里做什么呢?”小王子问。

“我在成千成千地运送旅客,”扳道工说,“我把运载旅客的火车发往各地,时而向东,时而向西。”

说话间,一列灯火辉煌的特别快车雷鸣般吼叫着开过去了,震得扳道房摇摇晃晃。

“他们好匆忙啊,”小王子说,“他们去寻找什么呢?”

“连火车司机自己也不知道。”扳道工说。

这时,第二列灯火通明的特别快车轰轰隆隆地向相反方向急驰而去。

“他们已经回来了啦?\ldots{}\ldots{}”小王子问。

“这不是刚才那批旅客,”扳道工说,“这是对开的火车。”

“他们不满意他们那个方向吗?”

“人们从来都不满意自己所在的地方。”扳道工说。

这时,第三列灯火明亮的特别快车又风驰电掣般地呼啸而去。

“他们是在追第一批旅客吗?”小王子问。

“他们什么也不追。”扳道工说,“他们在车厢里睡大觉,或者打哈欠。只有孩子们把鼻子贴在玻璃窗上往外看。”

“只有孩子们知道他们自己所寻找的东西。”小王子说,“他们为一个布娃娃花费了好多时间,这个布娃娃就成了一件非常重要的东西。如果有人夺走了他们的布娃娃,他们就哭起来\ldots{}\ldots{}”

“他们真有福气。”扳道工说。

\title{第二十三章}

“你好。”小王子说。

“你好。”商人说。

这是一个贩卖止渴药丸的商人。药丸精良,每周吞服一丸就不会感到口渴了。

“你为什么卖这个?”小王子说。

“这可以节约出许多时间。”商人说,“专家们计算过,服用这种药丸,每周可以节约时间五十三分钟。”

“那么,用这五十三分钟干什么呢?”

“想干什么就干什么,随便\ldots{}\ldots{}”

小王子自言自语地说道:

“我呀,要是我有五十三分钟可以支配,我就慢悠悠地朝一池清泉走去\ldots{}\ldots{}”

{\startalignment[center]
 \placefigeasy[][imgs/小王子(胡雨蘇譯)/00012.jpg][maxwidth=\textwidth,location={middle,none}]{calibre_50}
 \stopalignment}

\title{第二十四章}

这是我的飞机在沙漠上出故障的第八天。我听完了有关药丸商人的故事,也喝完了我备用的最后一滴水。

“啊!”我对小王子说道,“你回忆的这些故事可真有趣。可是,我的飞机还没有修好呢。我的水喝完了。要是我能够慢悠悠地朝一池清泉走去,我也一定会很高兴!”

小王子对我说:“我的朋友狐狸\ldots{}\ldots{}”

“我的小家伙,不要再提狐狸了。”

“为什么?”

“因为我就要渴死了\ldots{}\ldots{}”

他没弄懂我的意思,便回答我说:

“即使快要死了,有过一个朋友也很好嘛!我就为自己有过一个狐狸朋友而感到很高兴\ldots{}\ldots{}”

“他没有估计到这种危险。”我心中想道,“他从来不饥也不渴,只要有点阳光就够了\ldots{}\ldots{}”

他瞧了我一眼,并对我的想法做出了答复:

“我也渴了\ldots{}\ldots{}咱们去找一眼水井吧\ldots{}\ldots{}”

我显出厌倦的样子,在漫无边际的沙漠上盲目地去找一眼水井,岂非荒唐!然而,我们却不约而同地走了起来。

我们默默地走了好几个小时之后,夜幕降临大地,满天的星斗开始闪烁。由于渴,我有点发烧,我仰望着星空,仿佛在做梦一般。小王子的话在我的脑海中翻腾着。

“这么说,你也渴了?”我问他。

可他没有回答我的问题,只是对我说道:

“水对心田可能也是有益的\ldots{}\ldots{}”

我没有弄懂他的话,可我也默不作声了\ldots{}\ldots{}我清楚地知道不该去追问他。

他累了,坐了下来。我也挨着他坐下。片刻沉寂之后,他又说道:

“星星之所以美丽是因为有一朵人们看不见的花儿\ldots{}\ldots{}”

“那当然。”我应道。而后,我便默默地看着那月光下的层层沙浪。

“沙漠很美。”他又说道。

沙漠确实很美。我一直很喜欢沙漠。我们坐在一个沙丘上。举目四望,一无所见;侧耳细听,又寂静无声。但是,在这一片幽静之中,却有个什么东西在闪光\ldots{}\ldots{}

“使沙漠变得这样美丽的,”小王子说,“是它在什么地方了隐藏着一眼井。”

我为突然明白了沙漠上的神秘之光而惊讶不已。当我还是个小孩子的时候,我住在一幢古老的房子里,据传说那里面埋藏着一件宝贝。当然啦,从来没有人能发现它。甚至也没有人去寻找过它。可是这件宝贝却使整个房子令人神往。我家的房子在它的心灵深处隐藏着一个秘密\ldots{}\ldots{}

“是的”,我对小王子说,“不论是房子、星星或是沙漠,使得它们美丽的东西都是肉眼看不见的!”

“我感到很高兴”,小王子说道,“你也同意我那狐狸的看法。”

这时小王子睡着了,我就把他抱在怀里重新上路。我很激动,好像是抱着一个娇嫩的宝贝。我甚至觉得,地球上没有什么比这更娇贵的东西了。我借着月光,看着这苍白的前额,看着他这紧闭的双眼和这随风飘动的绺绺头发。这时我自语道:“我所看到的,只不过是外表而已。那最重要的东西,用肉眼是看不见的\ldots{}\ldots{}”

只见他双唇微开,嘴角挂着一丝微笑。我自言自语道:“这个正在酣睡的小王子,感人至深处是他对一朵花的忠贞。这朵玫瑰花的形象有如一盏明灯的火焰在他心中发光,甚至映照他进入梦乡\ldots{}\ldots{}”这时,我猜想他是更加脆弱了。必须好好保护那灯火:一阵风就可能把它吹灭的\ldots{}\ldots{}

于是就这样走呀走呀,在红日跃出地平线时,我终于找到了一眼井。

\title{第二十五章}

“那些人呐,”小王子说道,“他们拥挤着上了特别快车,可是他们却不知道自己要寻找的是什么。于是,他们就焦躁不安起来,急得团团转\ldots{}\ldots{}”

他接着又说:

“这没有必要\ldots{}\ldots{}”

我们遇到的这眼井,和撒哈拉沙漠中的那些水井不一样。撒哈拉大沙漠里的水井只是些在沙地上挖的坑。而这眼井却很像村庄里面的水井。可是,这里并无任何村庄,我以为自己在做梦。

“这可真奇怪,”我对小王子说,“井上样样俱全:辘轳、水桶,还有井绳\ldots{}\ldots{}”

他笑着,抓住绳就摇起辘轳来。于是辘轳吱吱作响,就像一个久久没有被风吹动的旧风标,风一吹就吱呀吱呀地响起来。

“你听见了吧,”小王子说道,“我们唤醒了这眼井,它现在唱起歌来了\ldots{}\ldots{}”

我不想让他劳累,于是对他说:

“让我来吧。这活儿太重,你受不了。”

我慢慢地将水桶提到石头井台上,稳稳当当地把它放好。那辘轳的歌声仍在我的耳边回响。我依然看到那太阳的影子在水上荡漾。

{\startalignment[center]
 \placefigeasy[][imgs/小王子(胡雨蘇譯)/00045.jpg][maxwidth=\textwidth,location={middle,none}]{calibre_51}
 \stopalignment}

“我渴望这水呀,”小王子说,“快给我喝点儿吧\ldots{}\ldots{}”

这时,我才恍然明白了他要寻找的是什么!

我把水桶提到他的嘴边,他闭着眼睛喝了起来。这水,就像节日一般美好、甜蜜。这永远不止是一种饮料。这水,是在披星戴月的旅途中发现,是在辘轳的歌声中,经过我双臂的劳动得来的。它像一件礼物似的慰藉着心田。我童年时候,是那圣诞树的灯光,午夜弥撒的乐曲,甜蜜的微笑,使我所收到的圣诞节礼物光彩夺目。

“你这里的人们在同一座花园里就种了五千棵玫瑰花。”小王子说道,“可是,他们却不能从中找到自己要寻找的东西\ldots{}\ldots{}”

“他们找不到的\ldots{}\ldots{}”我答道。

“可是他们所寻找的东西,却可以从一朵玫瑰花或一滴水里找得到\ldots{}\ldots{}”

“那当然。”我答道。

小王子又补充说:

“但是眼睛看不见,必须用心灵去寻找。”

我喝了水,感到很宽慰。沙漠在晨曦中呈现出蜂蜜的色泽。我也为这蜂蜜般的色泽而感到幸福。为什么我非要感到痛苦不可呢\ldots{}\ldots{}

小王子又重新坐在我身边,温柔地对我说:“你应当信守诺言。”

“什么诺言?”

“你知道\ldots{}\ldots{}你得给我的小绵羊画一个嘴套子\ldots{}\ldots{}我要对我的那朵花负责呀!”

我从口袋里拿出我的画稿。小王子一见就笑着说:

“你画的猴面包树真有点像卷心菜\ldots{}\ldots{}”

“啊!”

我还为我画的猴面包树感到非常自豪呢!

“你画的狐狸\ldots{}\ldots{}它那双耳朵\ldots{}\ldots{}有点像犄角\ldots{}\ldots{}而且太长了!”

这时,他又笑了。

“你太不公平了,小家伙。我过去不会画别的,只会画完整的和剖开肚皮的蟒蛇呀。”

“啊!这就蛮好的。”他说,“孩子们看得懂。”

于是,我就用铅笔勾画出一个嘴套子。当我把它递给小王子的时候,我的心里却感到很难过:

“我还不了解你有什么打算呢\ldots{}\ldots{}”

但是,他没有回答。他对我说:

“你知道,我落到地球上\ldots{}\ldots{}明天就是一周年了\ldots{}\ldots{}”

他沉默了片刻,又说道:

“我就落在离这儿挺近的地方\ldots{}\ldots{}”

这时,他的脸红了。

我不知为什么,又感到一阵莫名其妙的心酸。然而,我却想起了一个问题:

“一个星期以前,我认识你的那天早上,你独自一个人在这远离人烟的大沙漠里游来逛去,看来,这就不是偶然的了?你要回到你降落的地点去,是吗?”

小王子的脸又红了。

我犹豫不决,又追问了一句。

“大概是因为周年纪念吧?\ldots{}\ldots{}”

小王子的脸又红了起来。他从来不回答我的问题,但是他脸红,这就意味着“是的”,不是吗?

“啊!”我对他说,“我怕\ldots{}\ldots{}”

可是他却回答我说:

“你现在应该工作了,应该回到你的飞机那里去。我在这里等你,你明天晚上再来吧\ldots{}\ldots{}”

可是,我仍然放心不下。我想起了狐狸的话。要是叫人驯服,就很可能要掉些眼泪的\ldots{}\ldots{}

\title{第二十六章}

井旁边有一堵旧石墙的残垣断壁。第二天晚上我干完活回来,远远望见小王子耷拉着双腿坐在墙头上。我听见他在说话:

“你怎么想不起来了?”他说,“肯定不是在这里!”

大概还有另一个声音在回答他,因为他搭了腔:

“不!不!就是那一天,但地点不是这儿\ldots{}\ldots{}”

我继续朝石墙走过去。我还是没有看见、也没有听到有谁在和小王子讲话。可是小王子又回答道:

“\ldots{}\ldots{}当然啰。你一定会在沙上看到我的脚印是从哪里开始的。你只要在那里等我就行了。我今天夜里就到那里去。”

我走到离墙二十米的地方了,却仍然没有看到是什么在和小王子讲话。

小王子沉默了一会儿又说:

“你的毒液很厉害吗?你保证不会使我长时间感到痛苦吗?”

我焦虑地止住了脚步,但我依然不明白是怎么一回事。

“你现在就走吧\ldots{}\ldots{}我要下去了!”小王子说道。

我这时朝墙脚下看去,不由得吓了一跳!就在那里,有一条黄色的毒蛇,它冲着小王子把身子竖了起来。这种黄蛇的毒液,不消半分钟就能致人于死命。我一面摸口袋,掏出了手枪,一面跑过去。可是一听到我的脚步声,那条蛇就像一股干涸的水柱似的,慢慢钻进了沙子里。它不慌不忙地在石头缝隙中钻来钻去,发出一阵轻微的铿锵声。

{\startalignment[center]
 \placefigeasy[][imgs/小王子(胡雨蘇譯)/00010.jpg][maxwidth=\textwidth,location={middle,none}]{calibre_52}
 \stopalignment}

我跑到墙下,正好把我的这位小王子接在怀里。他的脸色像雪一样惨白。

“怎么搞的?你怎么跟蛇说话呀!”

我解开了他从不离身的金黄色围巾,用水湿了湿他的太阳穴,并让他喝了一点水。可是,我现在却什么也不敢再问他了。他严肃地看着我,用双臂搂住我的脖子。我感到,他的心脏犹如受到枪击而濒临死亡的小鸟的心脏一样,在微弱地跳动着,他对我说道:

“你找到了你的机器所缺少的东西,我很高兴。你不久就可以回家去了\ldots{}\ldots{}”

“你怎么知道的?”

我这次来正是要告诉他,在没有任何希望的情况下,我成功地完成了修理工作!

他没有回答我的问题,却又说道:

“我也一样,我今天要回家去了\ldots{}\ldots{}”

他接着忧伤地说道:

“我回家要远得多\ldots{}\ldots{}难得多\ldots{}\ldots{}”

我真的感到发生了某种不寻常的事情。我把他当作小孩子一样紧抱在怀里,可是我感到他在笔直地坠下万丈深渊,我想拉住他,却无能为力\ldots{}\ldots{}

他目光严峻,望着遥远的地方。

“我有你画的小羊,有羊的箱子和羊的嘴套子\ldots{}\ldots{}”

他带着忧伤的神情微笑了。

我等了良久,方才觉得他的身上渐渐暖和起来。

“小家伙,你刚才害怕了吧\ldots{}\ldots{}”

他害怕,这是无疑的!可他却温柔地笑了:

“今天晚上,我会更害怕的\ldots{}\ldots{}”

我再度感到要发生一件无可挽回的事情。我觉得我的心一下子就凉了。我心里明白了,我一想到再也听不到他的笑声,我就难以忍受。这笑声对我来说,就好比是沙漠中的一池清泉。

“小家伙。我还想听你笑\ldots{}\ldots{}”

而他却对我说:

“到今天夜里就正好是一年了。我的星球将正好处在我去年落下来的那个地方的上空\ldots{}\ldots{}”

“小家伙,难道说蛇、约会、星星的故事都是一场噩梦吗?”

但他没有回答我的问题,只是说:

“重要的东西是看不见的\ldots{}\ldots{}”

“当然啦\ldots{}\ldots{}”

“这就好比是花。要是你爱上了某颗星星上的一朵花,那么,当你在夜间仰望星空的时候,你就会感到甜蜜愉快,满天的星星都开遍了鲜花。”

“当然啦\ldots{}\ldots{}”

“这就好比是水。由于那辘轳和井绳的缘故,你给我喝的井水就像是一种美妙的音乐\ldots{}\ldots{}你还记得吧\ldots{}\ldots{}那水多么甜哪!”

“当然啦\ldots{}\ldots{}”

“夜晚,你看看满天的星斗吧。我的那颗星太小了,我无法给你指出它在哪里。这样倒更好了。你可以认为我的那颗星星就在群星之中。那么,你就会喜欢看满天的所有星斗\ldots{}\ldots{}这些星星都将成为你的朋友。此外,我还要送你一件礼物\ldots{}\ldots{}”

他又笑了起来。

“啊!小家伙,小家伙,我喜欢听你这笑声!”

“这正是我送给你的礼物\ldots{}\ldots{}这就好比是水\ldots{}\ldots{}”

“你这是什么意思?”

“人们眼里的星星并不是一样的。对旅行者来说星星是向导。对别的人来说星星只是些小亮点。而对于学者来说,星星就是他们研究的对象。对我遇到的那个商人来说,星星就是金钱。但是这些星星却从不开口分辩,唯独你的星星将是任何人都不曾有过的\ldots{}\ldots{}”

“你说什么?”

“夜晚,当你仰望星空的时候,因为我住在其中的一颗星星上,因为我在那里笑,那么对你来说,就好像所有的星星都在笑。你看到的那些星星就是些会笑的星星了!”

这时,他又笑了。

“那么,当你得到安慰以后(人们总是自我安慰的),你一定会因结识了我而感到高兴。你将永远是我的朋友。你将会禁不住和我一起欢笑。有时,你会不知不觉地打开窗户\ldots{}\ldots{}那时,你的朋友们看见你望着星空笑,他们定会非常惊讶。那时,你就可以对他们说:‘是的,星星永远使我欢笑!'而他们会以为你发疯了。我也许将使你感到难为情\ldots{}\ldots{}”

这时,他又笑了起来。

“这就好像我送给你的不是星星,而是许许多多会笑的小铃铛\ldots{}\ldots{}”

他又笑了。过后,他一下子变得严肃起来:

“今天夜里\ldots{}\ldots{}你知道吧\ldots{}\ldots{}你就别来了。”

“我决不离开你。”

“我的样子会很痛苦\ldots{}\ldots{}会有点像是要死去了。就是这么回事,你就别来看这些了,不必\ldots{}\ldots{}”

“我决不离开你。”

可是他忧虑起来。

“我对你说这些\ldots{}\ldots{}这也是因为那条蛇。别让毒蛇咬了你\ldots{}\ldots{}毒蛇很坏。它咬人是为了取乐\ldots{}\ldots{}”

“我决不离开你。”

而这时,似乎有什么事情又使他放心了:

“对了,毒蛇咬第二口的时候就没毒液了\ldots{}\ldots{}”

这天夜里,我没有看见他上路。他不声不响地走了。当我终于追上他的时候,他正坚定果断地快步走着。他只是对我说道:

“哎呀!你怎么来了\ldots{}\ldots{}”

于是他拉住我的手。但他仍然很忧虑:

“你不该来,你会难过的。我的样子会像是死去了似的,但这不会是真的\ldots{}\ldots{}”

我默不作声。

“路途太遥远,你是知道的。我不能带着这身子走,它太重了。”

{\startalignment[center]
 \placefigeasy[][imgs/小王子(胡雨蘇譯)/00001.jpg][maxwidth=\textwidth,location={middle,none}]{calibre_53}
 \stopalignment}

我默不作声。

“我的驱壳就像一块扔掉的老树皮,用不着为它伤心的\ldots{}\ldots{}”

我还是默不作声。

他有点灰心,却仍强打起精神说:

“你想,这将是多么美好呀!我也将观赏那满天的星斗。每颗星星都将变成一口水井,水井上都安装着生了锈的辘轳。所有的星星都将倒水给我喝\ldots{}\ldots{}”

我还是默不作声。

“这将是多么有趣啊!你将有五亿个小铃铛,我将有五亿池清泉\ldots{}\ldots{}”

这时,他也默不作声了,因为他在哭。

“就在这里。让我单独地走一步吧。”

他这时坐了下来,因为他害怕了。他却又说道:

{\startalignment[center]
 \placefigeasy[][imgs/小王子(胡雨蘇譯)/00039.jpg][maxwidth=\textwidth,location={middle,none}]{calibre_54}
 \stopalignment}

“你知道\ldots{}\ldots{}我的花\ldots{}\ldots{}我要对她负责呀!可她是多么弱不禁风啊!她又是那么天真烂漫!她只有四根微不足道的刺,以保护自己,抵御外侮\ldots{}\ldots{}”

我实在支持不住了,也坐了下来。他说:

“喏\ldots{}\ldots{}就是这些了。”

他迟疑了一下,然后站起身,往前迈了一步。我却动弹不得。

只见他的脚腕附近有一道黄光闪过。他一动不动地站立了片刻,没有喊叫,便像一棵树似的慢慢倒在了地上。因为是在沙地上,没有发出一点儿声响。

{\startalignment[center]
 \placefigeasy[][imgs/小王子(胡雨蘇譯)/00020.jpg][maxwidth=\textwidth,location={middle,none}]{calibre_55}
 \stopalignment}

\title{第二十七章}

不错,事到如今已经六年了\ldots{}\ldots{}我还从来不曾讲过这个故事呢。我的同伴们都为看到我活着归来而高兴,我自己却闷闷不乐。我对他们解释说:“这是由于累的缘故\ldots{}\ldots{}”

现在,我略微感到一点宽慰,就是说\ldots{}\ldots{}我心中的郁闷还没有完全排除。但我清楚地知道,小王子已经回到他的星球上。因为那天拂晓,我不曾找到他的身躯。他的躯体并非那么沉重\ldots{}\ldots{}从此,我就喜欢在夜间倾听星星的欢笑,就仿佛在倾听五亿个小铃铛\ldots{}\ldots{}

但是,现在又有一件非同小可的事情发生了。我忽然想起,我给小王子画的羊嘴套子上忘了画皮带了!看来,他再也不可能给的羊载上嘴套子了。于是我寻思道:“他的星球上发生了什么事情呢?也许绵羊把花儿吃掉了\ldots{}\ldots{}”

有时我心里又想:“肯定不会的!小王子每天夜里都用玻璃罩子罩住他的玫瑰花,而且他会很好地看住他的羊\ldots{}\ldots{}”想到这里,我就感到非常高兴,所有的星星也就甜蜜地笑了。

我有时转而又想:“人总难免有疏忽的时候,疏忽一次可就完了!要是有一天晚上他忘记盖上玻璃罩子,或者他的绵羊悄悄地溜了出来\ldots{}\ldots{}”想到这里,那些小铃铛就都变得泪汪汪了!

这是一个深奥的谜。对于你们这些和我一样热爱小王子的人来说,如果在一个谁都不知道的什么地方,有一只我们谁也没见过的绵羊,吃掉了或者没有吃掉一朵玫瑰花,这都会使宇宙万物产生天渊之别的变化\ldots{}\ldots{}

请你们仰望星空,想一想吧:绵羊吃掉了还是没有吃掉那朵花?你们将会看到,宇宙的一切都随之发生变化\ldots{}\ldots{}

然而任何一个大人将永远不会明白这有多么重要!

对我来说,这是世界上最美丽也是最凄凉的景色。这景色和前面的一幅画的景色完全相同。为了让你们看个一清二楚,我再次把它画了出来。就是在这里,小王子出现在地球上,后来,他也是在这里消失的。请你们仔细地看看吧,以便有一天你们在非洲的沙漠上旅行的时候,能够准确地辨认出这个地方。如果你们有机会经过这个地方,我求你们不要匆匆而过,请你们在小王子的那颗星球底下稍等片刻!如果这时有个小孩子向你们走来,如果他笑着,如果他有一头金黄色的头发,如果你们询问他时他不回答,你们一定会猜着他是谁。那么就请你们帮个忙吧,不要让我如此悲伤:请赶快写封信,告诉我他又回来了\ldots{}\ldots{}

{\startalignment[center]
 \placefigeasy[][imgs/小王子(胡雨蘇譯)/00016.jpg][maxwidth=\textwidth,location={middle,none}]{calibre_56}
 \stopalignment}